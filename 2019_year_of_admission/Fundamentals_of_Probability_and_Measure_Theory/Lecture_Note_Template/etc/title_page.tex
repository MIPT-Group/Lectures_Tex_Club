%%% Всю шаблонную информацию можно менять тут
\newcommand{\FullCourseNameFirstPart}{\so{НАЗВАНИЕ~ПРЕДМЕТА}}
\newcommand{\FullCourseNameSecondPart}{\so{ПРОДОЛЖЕНИЕ~НАЗВАНИЯ}}
\newcommand{\SemesterNumber}{III}
\newcommand{\LecturerInitials}{Фамилия Имя Отчество}
\newcommand{\AutherInitials}{Фамилия Имя}
\newcommand{\VKLink}{https://vk.com/}
\newcommand{\OverleafLink}{https://www.overleaf.com/}
\newcommand{\GithubLink}{https://github.com}

\begin{titlepage}
	\clearpage\thispagestyle{empty}
	\centering
	
	\textit{Федеральное государственное автономное учреждение \\
		высшего профессионального образования}
	\vspace{0.5ex}
	
	\textbf{Московский Физико-Технический Институт \\ КЛУБ ТЕХА ЛЕКЦИЙ}
	\vspace{20ex}
	\vspace{13ex}
	
	{\textbf{\FullCourseNameFirstPart}}
	\\
	{\textbf{\FullCourseNameSecondPart}}
	
	\SemesterNumber\ СЕМЕСТР  
	\vspace{1ex}
	
	Лектор: \textit{\LecturerInitials}
	
	\begin{figure}[!ht]
		\centering
		\includegraphics[width=0.4\textwidth]{logo_LTC.png}
		\label{fig:my_label}
	\end{figure}
\begin{flushright}
	\noindent
	Автор: \href{\VKLink}{\textit{\AutherInitials}}
	\\
	\href{\OverleafLink}{\textit{Проект на overleaf}}
	\\
	\href{\GithubLink}{\textit{Проект на github}} % Опционально, если хотите учавствовать в рейтинге
\end{flushright}
	
	\vfill
	\CourseDate\ года
	\pagebreak
	
\end{titlepage}
