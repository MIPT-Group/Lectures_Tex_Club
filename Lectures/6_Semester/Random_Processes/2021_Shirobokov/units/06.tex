\newpage
\lecture{6}{Введение в Фурье-анализ стационарных процессов}
\section{Введение в Фурье-анализ стационарных процессов}
\subsection{О нестационарных процессах}
У всех стационарных процессов одномерные распределения не зависят от времени. Такой процесс не может постоянно возрастать/убывать, он должен <<колебаться>> вокруг математического ожидания.
\begin{example}
    Процесс
    \begin{align*}
      & X(t) = A \cos \nu t
    \end{align*}
    где $A$~--- случайная, $\nu$~--- нет, не будет стационарным, хоть и колеблется; его математическое ожидание в неслучайные моменты времени~--- ноль.
\end{example}
\begin{example}
    Процесс
    \begin{align*}
      & X(t) = A \cos(\nu t +\varphi)
    \end{align*}
    где $A$~--- случайная, $\nu$~--- нет, $\varphi$~--- случайная, не будет
    стационарным, если $\varphi$ распределена неравномерно; иначе будет
    стационарным.
\end{example}
Обобщенный гармонический анализ~--- исследование таких колебаний, это стало
возможным при рассмотрении их как случайных процессов.
\subsection{Комплексные случайные процессы}
\begin{Def}
    \textbf{Комплексный случайный процесс}~--- процесс
    \begin{align*}
      & \{Z(t), \ t \in T\}, \ (\Omega, \cF, \PP)
    \end{align*}
    такой, что
    \begin{align*}
      & Z(\omega, t) = X(\omega, t) + iY(\omega,t) = \Real Z + \Img Z
    \end{align*}
    где $X$, $Y$~--- вещественные случайные процессы.
\end{Def}
\begin{Def}
    \textbf{Математическое ожидание}
    \begin{align*}
      & \EE Z(\omega, t) = \EE X(\omega, t) + i \EE Y(\omega,t)
    \end{align*}
\end{Def}
\begin{Def}
    \textbf{Дисперсия}
    \begin{align*}
      & \DD Z(t) = \EE\cent{Z}(t) \oL{\cent{Z}(t)} = \EE \left| \cent{Z}(t) \right|^2 \geq 0
    \end{align*}
\end{Def}
\begin{Def}
    \textbf{Корреляционная функция}
    \begin{align*}
      & R_Z(t,s) = \EE\cent{Z}(t) \oL{\cent{Z}(s)} = \oL(R_Z(s,t))
    \end{align*}
\end{Def}
\begin{Def}
    \textbf{Комплексный процесс второго порядка}~--- процесс, у которого
    \begin{align*}
      & \EE \left| Z(t) \right|^2 < \infty
    \end{align*}
\end{Def}
Считаем, что $T = [0; +\infty)$.
\subsection{Свойства корреляционной функции стационарных в широком смысле
  процессов}
\begin{Des}
    Для стационарных процессов
    \begin{align*}
      & R_Z(t) = R_Z(t,0) = \EE\cent{Z}(t) \oL{\cent{Z}(0)}
    \end{align*}
\end{Des}
\textbf{Свойства корреляционной функции}
\begin{enumerate}
    \item Эрмитовость.
    \begin{align*}
      & R_Z(t,s) = R_Z(t-s,0) = R_Z(t-s) = R_Z(0,s-t) = \oL{R_z(s-t,0)} = \oL{R_Z(s-t)} \Rightarrow \forall t \  R_Z(t) = \oL{R_x(-t)}
    \end{align*}
    \item Неотрицательность в нуле.
    \begin{align*}
      & \DD Z(t) = \DD Z(0) = R_Z(0) \geq 0
    \end{align*}
    \item Ограниченность
    \\
    Из неравенства Коши-Буняковского
    \begin{align*}
      & \forall t \in \RR \ \left| R_Z(t) \right| \leq R_Z(0)
    \end{align*}
    \item Непрерывность корреляционной функции и СК-непрерывность процесса
    \begin{align*}
      & \EE \left| Z(t+h)-Z(t) \right|^2 = \EE \left| \cent{Z}(t+h) - \cent{Z}(t) \right|^2 = R_Z(0) + R_Z(0) -2R_Z(h) = 2(R_Z(0)-R_Z(h))
    \end{align*}
    Стационарный процесс СК-непрерывен тогда и только тогда, когда его
    корреляционная функция непрерывна в нуле; если же она в нуле разрывна, то
    процесс не непрерывен ни в одной точке.
    \item Непрерывность корреляционной функции в нуле и всюду
    \begin{align*}
      & \left| \EE \left( \cent{Z}(t+h)-\cent{Z}(t) \right) \oL{\cent{Z}(t-s)} \right|^2 \leq \EE \left| \cent{Z}(t+h)-\cent{Z}(t) \right|^2 \EE \left|\oL{\cent{Z}(t-s)} \right|^2
    \end{align*}
    \begin{align*}
      & \left| R_Z(h+s) - R_Z(s) \right|^2 \leq 2(R_Z(0) - R_Z(h))R_Z(0)
    \end{align*}
    Корреляционная функция непрерывна всюду тогда и только тогда, когда она
    непрерывна в нуле.
    \item Неотрицательная определенность
    \begin{align*}
      & \forall  t_1, \dots, t_n \in \RR, \ \forall z_1, \dots, z_n \in \CC \ \EE\left| \sum_{i=1}^nz_i\cent{Z}(t_i) \right|^2 = \sum_{i=1}^n\sum_{j=1}^nz_i\oL{z_j}R_Z(t_i,t_j) \geq 0
    \end{align*}
    Для стационарного $R_Z(t_i, t_j) \sim R_Z(t_i-t_j)$
    \\
    Это то же самое, что неотрицательная определенность матрицы
    \begin{align*}
      & R = \left| \left| \begin{matrix} R_Z(t_it_j) \end{matrix} \right| \right|_{i,j=1}^n
    \end{align*}
    \begin{Def}
        Функция $f(t,s)$ \textbf{неотрицательно определена}, если
        \begin{align*}
          & \forall n \geq 1, \ \forall  t_1, \dots, t_n \in \RR, \ \forall z_1, \dots, z_n \in \CC \ \sum_{i=1}^n\sum_{j=1}^nz_i\oL{z_j}f(t_i,t_j) \geq 0
        \end{align*}
    \end{Def}
    \begin{Def}
        Функция $f(t)$ \textbf{неотрицательно определена}, если
        \begin{align*}
          & \forall n \geq 1, \ \forall  t_1, \dots, t_n \in \RR, \ \forall z_1, \dots, z_n \in \CC \ \sum_{i=1}^n\sum_{j=1}^nz_i\oL{z_j}f(t_i-t_j) \geq 0
        \end{align*}
    \end{Def}
    \begin{Note}
        Числа $z$ берутся всегда комплексными, суммы вещественны и неотрицательны.
    \end{Note}
    \begin{Note}
        Неотрицательно определенные функции являются эрмитовыми.
    \end{Note}
    \begin{Note}
        Эрмитова матрица неотрицательно определена тогда и только тогда, когда все
        её главные миноры неотрицательны.
    \end{Note}
    \begin{Note}
        Из неотрицательной определенности $f(t)$ следует:
        \begin{align*}
          & f(0) \geq 0
        \end{align*}
        \begin{align*}
          & \forall t \in \RR \ \left| f(t) \right| \leq f(0)
        \end{align*}
    \end{Note}
\end{enumerate}
\begin{theorem} (без доказательства)
    \\
    Функция $R(t)$ есть корреляционная функция некоторого стационарного в
    широком смысле случайного процесса тогда и только тогда, когда она
    неотрицательно определена.
    \\
    Более того, если $R(t)$ неотрицательно определена, то существует нормальный
    случайный процесс с корреляционной функцией $R(t)$.
\end{theorem}
\subsection{Фурье-анализ стационарных процессов}
Рассмотрим процесс \textbf{комплексная гармоника}:
\begin{align*}
  & X(t) = \xi e^{i\lambda t}, \ \EE \xi = 0, \ \EE \left| \xi \right|^2 < \infty
\end{align*}
$\lambda$ неслучайна. Для него
\begin{align*}
  & \EE X(t) = 0
\end{align*}
\begin{align*}
  & R_X(t,s) = \EE X(t) \oL{X(s)} - \EE X(t) \oL{\EE X(s)} = \EE\left| \xi \right|^2e^{i\lambda(t-s)}
\end{align*}
Он стационарен в широком смысле.
\\
Рассмотрим теперь процесс
\begin{align*}
  & X(t) = \xi_1 e^{i\lambda_1 t} + \xi_2 e^{i\lambda_2 t}, \ \EE \xi_1 = \EE \xi_2 = 0, \ \EE \left| \xi_1 \right|^2, \EE \left| \xi_2 \right|^2 < \infty, \ \lambda_1 \neq \lambda_2
\end{align*}
Для него
\begin{align*}
  & \EE X(t) = 0
\end{align*}
\begin{align*}
  & R_X(t,s) = \EE X(t) \oL{X(s)} = \EE \left| \xi_1 \right|^2 e^{i \lambda_1 (t-s)} + \EE \left( \xi_1\oL{\xi_2} \right) e^{i(\lambda_1-\lambda_2)t+i\lambda_1(t-s)} + \EE \left( \oL{\xi_1}\xi_2 \right) e^{-i(\lambda_1-\lambda_2)t+i\lambda_2(t-s)} + \\
  & + \EE \left| \xi_2 \right|^2 e^{i \lambda_2 (t-s)}
\end{align*}
Это функция $t-s$ тогда и только тогда, когда
\begin{align*}
  & \EE \left( \xi_1\oL{\xi_2} \right) e^{i(\lambda_1-\lambda_2)t+i\lambda_1(t-s)} + \EE \left( \oL{\xi_1}\xi_2 \right) e^{-i(\lambda_1-\lambda_2)t+i\lambda_2(t-s)} = 0
\end{align*}
\begin{align*}
  & R_X(t,s) = \EE \left| \xi_1 \right|^2 e^{i \lambda_1 (t-s)} + \EE \left| \xi_2 \right|^2 e^{i \lambda_2 (t-s)}
\end{align*}
Соответственно, процесс стационарен в широком смысле тогда и только тогда, когда
$\xi_1$ и $\xi_2$ некоррелированы.
\begin{theorem} (без доказательства)
    \\
    Процесс
    \begin{align*}
      & X(t) = \sum_{k=1}^n\xi_k e^{i\lambda_k t}, \ \EE \xi_k = 0, \ \EE \left| \xi_k \right|^2 < \infty, \ \lambda_k \neq \lambda_m
    \end{align*}
    стационарен в широком смысле тогда и только тогда, когда все $\xi_k$ попарно
    некоррелированы.
    \\
    Его корреляционная функция тогда
    \begin{align*}
      & R_X(t,s) = R_X(\tau) = \sum_{k=1}^n \EE \left| \xi_k \right|^2 e^{i \lambda_k \tau}
    \end{align*}
\end{theorem}
\begin{Def}
    Центрированный комплексный процесс $\{V(\lambda) \mid \lambda \in \RR\}$
    второго порядка называется \textbf{процессом с ортогональными
      приращениями}, если
    \begin{align*}
      & \forall \lambda_1 < \lambda_2 < \lambda_3 < \lambda_4 \ \EE \left( V(\lambda_2) - V(\lambda_1) \right)\oL{\left( V(\lambda_4) + V(\lambda_3) \right)} = 0
    \end{align*}
    Фактически скалярное произведение приращений равно нулю.
\end{Def}
\begin{theorem} Крамера (доказательство на следующей лекции)
    \\
    Любому стационарному СК-непрерывному процессу $X(t)$ м матожиданием $m_X(t)$
    соответствует случайный процесс с ортогональными приращениями $V_X(t)$
    такой, что п.~н.
    \begin{align*}
      & X(t) = m_X(t) + \int_{-\infty}^{+\infty}e^{i\lambda t} dV(\lambda)
    \end{align*}
    где интеграл есть СК-интеграл Римана-Стилтьеса.
\end{theorem}
\begin{theorem} Хинчина (без доказательства)
    \\
    Для того, чтобы непрерывная функция $R(t)$ была корреляционной функцией
    некоторого стационарного СК-непрерывного процесса, неорходимо и достаточно,
    чтобы она была представима в виде
    \begin{align*}
      & R(t) = \int_{-\infty}^{+\infty}e^{i\lambda t} dS(\lambda)
    \end{align*}
    где интеграл есть СК-интеграл Римана-Стилтьеса, а $S(\lambda)$~---
    неотрицательная монотонно неубывающая непрерывная слева функция, то есть
    функция, с точностью до неотрицательного множителя совпадающая с функцией
    распределения некоторой случайной величины.
\end{theorem}
\begin{Note}
    Функции $V$ и $S$ связаны.
\end{Note}
\begin{align*}
  & X(t) = int_{-\infty}^{+\infty}e^{i\lambda t} dV(\lambda) = \LIM{^{a \to -\infty}_{b \to +\infty}} \LIM{\sup \Delta \lambda_j \to 0}\sum_{\lambda_j=a}^be^{i\lambda_jt}\left( V(\lambda_j) - V(\lambda_{j-1}) \right)
\end{align*}
\begin{align*}
  & R_X(t,s) = \cent{X}(t)\oL{\cent{X}(s)} = \EE \left( \LIM{^{a \to -\infty}_{b \to +\infty}}\LIM{\sup \Delta \lambda_j \to 0} \sum_{\lambda_j=a}^be^{i\lambda_jt}\left( V(\lambda_j) - V(\lambda_{j-1}) \right)\right)\cdot \\
  & \cdot \left( \LIM{^{a \to -\infty}_{b \to +\infty}}\LIM{\sup \Delta \lambda_k \to 0} \sum_{\lambda_k=a}^be^{-i\lambda_ks}\oL{\left( V(\lambda_k) - V(\lambda_{k-1}) \right)}\right) = \lim_{^{a \to -\infty}_{b \to +\infty}}\lim_{\sup \Delta \lambda_j \to 0} \sum_{\lambda_j=a}^be^{i\lambda_j(t-s)}\EE \left| V(\lambda_j) - \right. \\
  & \left. - V(\lambda_{j-1}) \right|^2 = \int_{-\infty}^{+\infty} e^{i\lambda(t-s)}\EE \left| dV(\lambda) \right|^2
\end{align*}
\subsection{Спектральная функция и плотность}
\begin{Def}
    Если
    \begin{align*}
      & R(t) = \int_{-\infty}^{+\infty}e^{i\lambda t} dS(\lambda)
    \end{align*}
    то функция $S(\lambda)$ называется \textbf{спектральной функцией}.
\end{Def}
\begin{Def}
    Если
    \begin{align*}
      & \exists \rho(\lambda)\geq 0: \ S(\lambda) = \int_{-\infty}^{\lambda}\rho(\tau) d\tau
    \end{align*}
    то функция $\rho(\lambda)$ называется \textbf{спектральной плотностью}.
\end{Def}
Если есть плотность, то
\begin{align*}
  & R(t) = \int_{-\infty}^{+\infty}e^{i\lambda t} \rho(\lambda)d\lambda
\end{align*}
\begin{Note}
    Непрерывная функция $R(t)$ будет неотрицательно определенной тогда и только
    тогда, когда она представима в виде интеграла
    \begin{align*}
      & R(t) = \int_{-\infty}^{+\infty}e^{i\lambda t} \rho(\lambda) d\lambda
    \end{align*}
    Часто теоремой Хинчина называют это утверждение.
\end{Note}
\begin{Prop} ~
    \\
    Если $R(t)$ непрерывна всюду, абсолютно интегрируема, а ее Фурье-образ
    \begin{align*}
      & \rho(\lambda)= \int_{-\infty}^{+\infty}e^{i\lambda t} R(t) dt
    \end{align*}
    тоже интегрируем на $\RR$, то из неотрицательности $\rho(\lambda)$ следует
    неотрицательная определенность $R(t)$.
\end{Prop}
