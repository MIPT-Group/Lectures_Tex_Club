\newpage
\lecture{11}{Непрерывные цепи Маркова}
\section{Непрерывные цепи Маркова}
\subsection{Непрерывные цепи Маркова}
\begin{Def}
    Случайная функция
    \begin{align*}
      & X(t), \ t \geq 0, \ S \subseteq \ZZ, \ \left| S \right| \leq \infty
    \end{align*}
    со свойством
    \begin{align*}
      & \forall n \geq 1, \ \forall m_0 < m_1 < \dots < m_n \ \PP\left( X(t_n) = x_n \mid X(t_{n-1}) = x_{n-1}, \dots, X(t_0) = x_0 \right) = \\
      & = \PP\left( X(t_{n}) = x_n \mid X(t_{n-1}) = x_{n-1}\right)
    \end{align*}
    где эти вероятности определены, называется \textbf{непрерывной марковской
      цепью (НМЦ)}.
\end{Def}
\begin{Def}
    Если состояний конечное число, то НМЦ называется \textbf{конечной}, иначе \textbf{счетной}.
\end{Def}
\begin{Def}
    Вероятность
    \begin{align*}
      & p_{ij}(t_1,t_2) = \PP \left( X(t_1)=j \mid X(t_2) = i\right)
    \end{align*}
    назовем \textbf{вероятностью перехода из из состояния $i$ в момент $t_1$ в
      состояние $j$ в момент $t_2$} ($t_1 \leq t_2$).
\end{Def}
\begin{Def}
    Матрица
    \begin{align*}
      & P(t_1,t_2) = \left| \left| \begin{matrix} p_{ij}(t_1,t_2) \end{matrix} \right| \right|
    \end{align*}
    называется \textbf{матрицей перехода цепи от момента $t_1$ к моменту $t_2$}
    ($t_1 \leq t_2$).
\end{Def}
\textbf{Свойства}
\begin{enumerate}
    \item $\forall i \in S, \ 0 \leq s \leq t \ \dst \sum_{j \in S} p_{ij}(s,t) = 1$
    \item $\forall i \in S, \ 0 \leq t \ p_{ii}(t,t) = 1$
    \item $\forall i \neq j \in S, \ 0 \leq t \ p_{ij}(t,t) = 0$
\end{enumerate}
\begin{theorem} Уравнение Колмогорова-Чепмена
    \\
    Пусть $t_1 < t < t_2$. Тогда выполняется
    \begin{align*}
      & P(t_1,t_2) = P(t_1,t)P(t,t_2)
    \end{align*}
\end{theorem}
\begin{Proof}
    \begin{align*}
      & p_{ij}(t_1,t_2) = \PP\left( X(t_2) = j \mid X(t_1) = i \right) = \sum_{k \in S}\PP\left( X(t_2) = j \mid X(t)=k \right)\PP\left( X(t) = k \mid X(t_1) = i \right) = \\
      & = \sum_{k \in S}p_{kj}(t,t_2)p_{ik}(t,t_1)
    \end{align*}
\end{Proof}
\begin{Def}
    Вероятность $\pi_k(t) = \PP\{X_n = t\}$ называется \textbf{вероятностью
      состояния $k$ в момент $t$}.
\end{Def}
\begin{Def}
    Вектор $\pi(t) = \left[ \pi_0(t), \pi_1(t), \dots \right]^T$ называется
    \textbf{распределением вероятностей состояний в момент $t$}.
\end{Def}
\begin{theorem}
    \begin{align*}
      & \pi(t) = P(s,t)\pi(s)
    \end{align*}
\end{theorem}
\begin{Proof}
    Аналогично дискретному случаю.
\end{Proof}
\subsection{Однородные цепи Маркова}
\begin{Def}
    Если
    \begin{align*}
      & \forall t, s \geq 0 \ P(0,t) = P(s, t+s)
    \end{align*}
    то цепь Маркова называется \textbf{однородной}, иначе \textbf{неоднородной}.
\end{Def}
\begin{Des}
    Введем обозначение
    \begin{align*}
      & p_{ij}(t) = \PP\{X(t) = j\mid X(0) = i\}
    \end{align*}
\end{Des}
\begin{Def}
    Случайный процесс $\{X(t), \ t \geq 0\}$, определенный на вероятностном
    пространстве $\{\Omega, \cF, \PP\}$, называется \textbf{непрерывным справа},
    если его $\PP$-почти все траектории непрерывны справа.
\end{Def}
\begin{theorem}~
    \\
    Пусть $\{X(t), \ t \geq 0\}$~--- непрерывная справа цепь Маркова, тогда
    \begin{align*}
      & \forall s \geq 0 \ P(s,t) \To{t \to s+0} I
    \end{align*}
    В частности, для однородного процесса
    \begin{align*}
      & P(t) \To{t \to 0+0} I
    \end{align*}
\end{theorem}
\begin{Proof}
    Пусть $\{t_n\} \to s$ справа. $\forall \omega \in \Omega$ $f: t \mapsto
    X(\omega, t)$ непрерывно в $s$ справа тогда и только тогда, когда $\exists
    n_0: \forall n \geq n_0 \ X(t_n) = i$.
    \\
    Пусть $E = \{\omega\}$, для которых $f: t \mapsto X(\omega, t)$
    непрерывно в $s$ справа, тогда
    \begin{align*}
      & \left\{ \omega \in E: X(\omega,s)=i \right\} = \bigcup_{n=1}^\infty \bigcap_{k \leq n}\left\{ \omega \in \Omega: X(\omega, t_k) = i \right\}
    \end{align*}
    В силу непрерывности
    \begin{align*}
      & \mu \left( \left\{ \omega \in \Omega: X(\omega,s)=i \right\} \triangle \bigcup_{n=1}^\infty \bigcap_{k \leq n}\left\{ \omega \in \Omega: X(\omega, t_k) = i \right\} \right) = 0
    \end{align*}
    Пусть теперь
    \begin{align*}
      & \PP\{X(s) = i\} > 0
    \end{align*}
    Тогда
    \begin{align*}
      & 1 = p_{ii}(s,s) = \PP\left( X(s) = i \mid X(s) = i \right) = \PP\left( \bigcup_{n=1}^\infty \bigcap_{k \leq n}\left\{ \omega \in \Omega: X(\omega, t_k) = i \mid X(\omega, s)\right\} \right) = \lim_{n\to \infty}\PP\left( \bigcap_{k \leq n}\left\{ \omega \in \Omega: X(\omega, t_k) = i \mid X(\omega, s)\right\} \right) \leq \PP\left( X(t_n) = i \mid X(s) = i \right) \leq 1
    \end{align*}
    \begin{align*}
      & \lim_{n \to \infty} \PP\left( X(t_n) = i \mid X(s) = i \right) \leq 1
    \end{align*}
    \begin{align*}
      & p_{ii}(s,t_n) \To{n \to \infty} p_{ii}(s,s) \Rightarrow p_{ii}(s,t) \To{t \to s+0} p_{ii}(s,s)
    \end{align*}
    Далее,
    \begin{align*}
      & p_{ij}(s,t) = \PP\left( X(t) = j \mid X(s) = i \right)\leq \PP\left( X(t)\neq i \mid X(s) = i \right) = 1 - p_{ii}(s,t)
    \end{align*}
    \begin{align*}
      & p_{ij}(s,t) \To{t \to s+0} 0
    \end{align*}
    Для однородного случая
    \begin{align*}
      & P(0) = I, \ P(t+s) = P(t)P(s), \ P(t) \To{t \to 0} I
    \end{align*}
\end{Proof}
\begin{theorem}~
    \\
    Пусть $\{X(t), \ t\geq 0\}$~--- однородная НМЦ, непрерывная справа и с
    конечным числом состояний. Тогда существует матрица
    \begin{align*}
      & Q = \lim_{h \to 0+0}\frac{P(t+h)-P(t)}{h}
    \end{align*}
    а $P(t)$ удовлетворяет системе дифференциальных уравнений
    \begin{align*}
      & \dot{P(t)} = P(t)Q, \ P(0) = I \\
      & \dot{P(t)} = QP(t), \ P(0) = I
    \end{align*}
    которая имеет единственное решение
    \begin{align*}
      & P(t) = \exp\left( Qt \right)
    \end{align*}
\end{theorem}
\begin{Proof}
    В силу непрерывности $P(t)$ в нуле имеем:
    \begin{align*}
      & P(h) \To{h\to 0} I
    \end{align*}
    а значит, есть окрестность нуля такая, что $P(h)$ обратима для всех $h$ из
    неё, и
    \begin{align*}
      & P^{-1}(h) \To{h\to 0} I
    \end{align*}
    Тогда
    \begin{align*}
      & P(t+h)  = P(t)P(h) \To{t\to 0} P(t)
    \end{align*}
    \begin{align*}
      & P(t-h)  = P(t)P^{-1}(h) \To{t\to 0} P(t)
    \end{align*}
    А значит, $P(t)$ непрерывна в каждой $t$.
    \\
    Из уравнения Колмогорова-Чепмена следует
    \begin{align*}
      & P(t) = P(t-h)P(h) \Rightarrow P^{-1}(h) = P(-h)
    \end{align*}
    Более того, $P(t)$ дифференцируема для всех $t \geq 0$.
    \\
    Пусть
    \begin{align*}
      & Q = \lim_{h \to +0}\frac{P(h)-I}{h}
    \end{align*}
    Заметим:
    \begin{align*}
      & P(t+h) = P(t)P(h) = P(h)P(t) \Rightarrow \frac{P(t+h)-P(t)}{h} = P(t)\frac{P(h)+I}{h} = \frac{P(h)+I}{h}P(t)
    \end{align*}
    \begin{align*}
      & \dot{P}(t) = P(t)Q=QP(t), \ P(0) = I
    \end{align*}
    \begin{align*}
      & P(t) = \exp(Qt)
    \end{align*}
\end{Proof}
\begin{Prop}
    \begin{align*}
      & \dot{\pi}(t) = Q^T\pi(t)
    \end{align*}
\end{Prop}
\begin{Proof}
    Продифференцируем
    \begin{align*}
      & \pi(t) = P^T\pi(t)
    \end{align*}  
\end{Proof}
\begin{Def}
    Матрица $Q$ называется \textbf{матрицей интенсивностей (инфинитезимальной матрицей)}.
\end{Def}
\begin{Def}
    $q_{ij}$ называется \textbf{интенсивностью перехода из $i$ в $j$}.
\end{Def}
\begin{Prop}
    \begin{align*}
      & q_{ii} = \lim_{h \to 0+0}\frac{p_{ii}-1}{h}
    \end{align*}
    \begin{align*}
      & q_{ij} = \lim_{h \to 0+0}\frac{p_{ij}}{h}
    \end{align*}
    \begin{align*}
      & p_{ii}(h) = 1 + q_{ii}h + o(h)
    \end{align*}
    \begin{align*}
      & p_{ij}(h) = q_{ij}h + o(h)
    \end{align*}
\end{Prop}
\begin{Proof}
    \begin{align*}
      & \sum_{j}p_{ij}(h) = 1
    \end{align*}
    \begin{align*}
      & 1 = 1+q_{ii}h+o(h)+\sum_{i\neq j}q_{ij}h+o(h)
    \end{align*}
    \begin{align*}
      & \sum_{i\neq j}q_{ij} = -q_{ii}
    \end{align*}    
\end{Proof}
\begin{Def}
Пусть $\{X(t), t \geq 0\}$~--- однородная НМЦ с конечным или счетным числом
состояний, непрерывными справа траекториями. Определим \textbf{время пребывания
  в состоянии $i \in S$} по формуле
\begin{align*}
  & \tau_i(\omega) = \inf\left( t: X(\omega, t) \neq i \right)
\end{align*}
\end{Def}
\begin{Prop}
    Для таких процессов
    \begin{align*}
      & \tau_i(\omega) = \inf\left( t: X(\omega, t) \neq i \right) = \inf\left( t \in D: X(\omega, t) \neq i \right) = \tau_i^D, \ \left| D \right| = \aleph_0, \ [D] = \RR
    \end{align*}
    почти наверное.
\end{Prop}
\begin{Proof}
    Очевидно,
    \begin{align*}
      & \tau_i \leq \tau_i^D
    \end{align*}
    Если $\tau_i = +\infty$, то $\tau_i^D = +\infty$. Иначе в силу всюду
    плотности есть последовательность $\{t_n\} \to \tau_i$. Но тогда в силу
    непрерывности справа $X(\omega, \tau_i(\omega)) \neq i$, а значит,
    \begin{align*}
      & \tau_i \geq \tau_i^D
    \end{align*}  
\end{Proof}
Будем рассматривать $\tau_i^D(\omega)$, являющийся случайной величиной.
\\
В качестве $D$ удобно брать
\begin{align*}
  & D_t = \bigcup_{n=0}^\infty D_n = \bigcup_{n=0}^\infty \left\{ \frac{jt}{2^n} \right\}^{2^n}_{j=0}; \ D_n \subseteq D_{n+1}
\end{align*}
\begin{align*}
  & \PP\left( \tau_i^D>t \right)  = \PP\left( \bigcap_{n=0}^\infty \bigcap_{j=0}^{2^n} \left\{ X\left( \frac{jt}{2^n} \right) = i \right\} \right) = \lim_{n \to \infty}\left( \bigcap_{j=0}^{2^n} \left\{ X\left( \frac{jt}{2^n} \right) = i \right\} \right) = \lim_{n \to \infty} p_{ii}^{2n}\left( \frac{t}{2^n} \right) \cdot \\
  & \cdot \pi_i(0) = \lim_{n \to \infty} \left( 1+p_{ii}\left( \frac{t}{2^n} \right) +o\left( \frac{t}{2^n} \right)\right)^{2^n}\pi_i(0) = e^{q_{ii}t}\pi_i(0)
\end{align*}
\begin{Note}
    Если цепь стартует из $i$, то время пребывания в состоянии $i$ имеет
    экспоненциальное распределение интенсивности $-q_{ii}$
\end{Note}
\begin{Des}
    Пуусть
    \begin{align*}
      & q_i = -q_{ii} = \sum_{j \neq i}q_{ij}
    \end{align*}
\end{Des}
\begin{Note}
    \begin{align*}
      & \pi_i(0) = 1 \Rightarrow \tau_i^D \in Exp(q_i)
    \end{align*}
\end{Note}
\begin{Des}
    Пусть
    \begin{align*}
      & F_{ij}(h) = \PP\left( X(t+h) = j\mid X(t) = i, X(t+h) \neq i \right)
    \end{align*}
\end{Des}
Тогда по определению условной вероятности
\begin{align*}
  & f_{ij}(t+h) = \frac{\PP\left( X(t+h) = j\mid X(t) = i \right)}{\PP\left( X(t+h) \neq i \mid X(t) = i \right)}\To{h \to 0} \frac{q_{ij}}{q_i}
\end{align*}
\begin{Def}
    $\dst \frac{q_{ij}}{q_i}$~--- вероятность перейти из $i$ в $j$ в момент прыжка.
\end{Def}
Пусть цепь стартует из $i$ и находится там $Exp(q_i)$ времени, и цепь попадает в
$j$ с вероятностью $\dst \frac{q_{ij}}{q_i}$, потом спустя $Exp(q_j)$ происходит
ещё один прыжок и так далее.