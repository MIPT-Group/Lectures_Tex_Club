\newpage
\lecture{12}{Непрерывные цепи Маркова (продолжение)}
\begin{Def}
    \textbf{Прямое уравнение Колмогорова}
    \begin{align*}
      & \dot{P}(t) = P(t)Q
    \end{align*}
\end{Def}
\begin{Def}
    \textbf{Обратное уравнение Колмогорова}
    \begin{align*}
      & \dot{P}(t) = QP(t)
    \end{align*}
\end{Def}
\subsection{Эргодические непрерывные цепи Маркова}
\begin{Def}
    Распределение $\pi^0$ называется \textbf{стационарным}, если
    \begin{align*}
      & \forall t \geq 0 \ P^T(t)\pi^0 = \pi^0
    \end{align*}
\end{Def}
\begin{Def}
    Для стационарного
    \begin{align*}
      & \dot{\pi}(t) = Q^T\pi(t) \Rightarrow 0 = Q^T\pi^0(t)
    \end{align*}
    \textbf{стационарные уравнения Колмогорова.}
\end{Def}
\begin{Def}
    Если
    \begin{align*}
      & \exists t > 0: \ p_{ij}(t) > 0
    \end{align*}
    то говорят, что \textbf{за $i$ следует $j$}.
\end{Def}
\begin{Def}
    Если $i \rightarrow j$ и $j \rightarrow i$, то говорят, что \textbf{$i$ и
      $j$~--- сообщающиеся}.
\end{Def}
Отношение делит цепь на \textbf{классы сообщающихся состояний}.
\begin{Def}
    Если
    \begin{align*}
      \forall j \in S \ (i \rightarrow j) \rightarrow (j \rightarrow i)
    \end{align*}
    то говорят, что \textbf{$i$~--- существенное состояние}, иначе
    \textbf{несущественное}.
\end{Def}
\begin{Def}
    Если $\dst \lim_{t \to \infty} p_{ii}(t) = 0$, то состояние $i$ называется
    \textbf{нулевым}, иначе \textbf{ненулевым}.
\end{Def}
\begin{Def}
    Если все состояния сообщаются, то цепь называется \textbf{неразложимой
      (неприводимой)}.
\end{Def}
\begin{Des}
    Обозначим вероятность возврата в $i$ за конечное время
    \begin{align*}
      & F_i = \int_0^{\infty}f_i (t)dt
    \end{align*}
\end{Des}
\begin{Def}
    Если $F_i = 1$, то состояние $i$ наывается \textbf{возвратным}, иначе
    \textbf{невозвратным}.
\end{Def}
\begin{theorem} (о солидарности)
    \\
    Для неразложимых марковских цепей выполняется:
    \begin{enumerate}
        \item Если есть хотя бы одно нулевое состояние, то нулевыми будут все;
        \item Если есть хотя бы одно возвратное состояние, то возвратными будут
        все.
    \end{enumerate}
\end{theorem}
\begin{Def}
    ДМЦ, у которой
    \begin{align*}
      & p_{ij} = \begin{cases}
          \dst \frac{q_{ij}}{q_i}, \ i \neq j \\
          0, \ i = j
          \end{cases}
    \end{align*}
    называется \textbf{цепью скачков}.
\end{Def}
\begin{Def}
    Марковская цепь называется \textbf{эргодической}, если $\forall i,j \in S \
    \exists \dst \lim_{t\to \infty}p_{ij}(t) = p_j > 0$, не зависящий от $i$.
\end{Def}
\begin{theorem} Эргодическая для конечных цепей
    \\
    Для эргодичности конечной непрерывной цепи Маркова необходимо и достаточно
    наличие неразложимости.
\end{theorem}
\begin{Proof}
    Докажем <<идейно>>.
    \begin{itemize}
        \item Непрерывность.
        \\
        Аналогично дискретной.
        \item Достаточность.
        \\
        Заметим, что в силу неразложимости $\forall t \geq 0 \ p_{ij}(t)>0$.
        \\
        В силу экспоненциального распределения и сообщаемости вероятность
        попасть в любое состояние из любого за сколь угодно малое время не ноль;
        далее доказываем точь-в-точь как для дискретной цепи.
    \end{itemize}
\end{Proof}
\begin{theorem} (о предельном распределении)
    \\
    В конечных эргодических цепях
    \begin{enumerate}
        \item $\exists C > 0, \ \rho \in (0;1): \ \forall t \geq 0 \ \left| p_{ij}(t) \right| \leq
        C\rho^t$;
        \item числа $p_{j}$ задают распределение состояний;
        \item это распределение стационарно;
        \item это стационарное распределение~--- единственное стационарное
        распределение для данной цепи Маркова;
        \item $\forall j \ \pi_{j}(t) \To{n \to \infty} p_j$
    \end{enumerate}
\end{theorem}
\begin{Proof}
    Доказательство проводим точь-в-точь как для непрерывного случая.
\end{Proof}
\begin{theorem} (без доказательства)
    \\
    В конечной непрерывной эргодической цепи Маркова
    \begin{align*}
      & \frac{1}{T}\int_{0}^T\chi_{X(t)=i}dt \os{\text{п.~н.}}{\To{t\to \infty}}p_j
    \end{align*}
\end{theorem}
\subsection{Процессы гибели и рождения}
\begin{Def}
    Однородная непрерывная марковская цепь называется \textbf{процессом гибели и
      рождения}, если ее стохастический граф имеет вид
    (число состояний может быть конечным или бесконечным).
\end{Def}
\begin{Def}
    $\lambda_j$~--- \textbf{интенсивности рождения}.
\end{Def}
\begin{Def}
    $\mu_j$~--- \textbf{интенсивности гибели}.
\end{Def}
\begin{theorem} (без доказательства)
    \\
    Пусть $X(t)$~--- конечный процесс гибели и рождения, $\lambda_j, \mu_j > 0$.
    Тогда стационарное распределение существует, единственно и определяется
    соотношениями
    \begin{align*}
      & \pi_0^0 = \left( \sum_{i=0}^N \prod_{j=0}^i \frac{\lambda_j}{\mu_j} \right)^{-1} \\
      & \pi_j^0 = \frac{\lambda_j}{\mu_j}\pi_{j-1}^0
    \end{align*}
\end{theorem}
\begin{Def}
    Если все $\lambda_j = 0$, то это \textbf{процесс (чистой) гибели}.
\end{Def}
\begin{Def}
    Если все $\mu_j = 0$, то это \textbf{процесс (чистого) рождения}.
\end{Def}
\subsection{Взрывные марковские цепи}
\begin{Des}
    Пусть $T_n, \ n \geq 1$~--- моменты времени, когда осуществляются переходы
    между состояниями, $T_{\infty} = \dst \lim_{n \to \infty} T_n$.
\end{Des}
\begin{Des}
    Если
    \begin{align*}
      & \forall i \in S \ \PP\left( T_\infty = \infty \mid X(0) = i \right) = 1
    \end{align*}
    то это \textbf{невзрывная}, иначе \textbf{взрывная} марковская цепь.
\end{Des}
Бесконечное число переходов за конечное время.
\begin{theorem}
    Пусть $\{\xi_n\} \in Exp(\lambda_n)$ независимы. Тогда
    \begin{align*}
      & \sum_{j=1}^N\frac{1}{\lambda_j} = \infty \Rightarrow \PP\left( \sum_{j=1}^N \xi_j =\infty \right) =1
    \end{align*}
    \begin{align*}
      & \sum_{j=1}^N\frac{1}{\lambda_j} < \infty \Rightarrow \PP\left( \sum_{j=1}^N \xi_j <\infty \right) =1
    \end{align*}
\end{theorem}
\begin{Proof}
    Пусть
    \begin{align*}
      & \sum_{j=1}^N\frac{1}{\lambda_j} = \EE \sum_{j=1}^N \xi_j < \infty \Rightarrow \PP\left( \sum_{j=1}^N \xi_j <\infty \right) =1
    \end{align*}
    Пусть
    \begin{align*}
      & \sum_{j=1}^\infty\frac{1}{\lambda_j} = \infty
    \end{align*}
    Тогда
    \begin{align*}
      & \EE \exp\left( -\sum_{j=1}^\infty\xi_j\right) = \prod_{j=1}^\infty \EE \exp \left( -\xi_j \right) = \prod_{j=1}^\infty \left( 1 + \frac{1}{\lambda_n} \right)^{-1} = 0
    \end{align*}
    \begin{align*}
      & \PP\left( \sum_{j=1}^N \xi_j =\infty \right) = 1
    \end{align*}
\end{Proof}
\begin{corollary}
    Процесс рождения с интенсивностями $\lambda_j$ является взрывным тогда и
    только тогда, когда
    \begin{align*}
      & \sum_{j=1}^N\frac{1}{\lambda_j} < \infty
    \end{align*}
\end{corollary}
\begin{Note}
    Пуассоновский процесс есть непрерывный невзрывной процесс рождения.
\end{Note}
\subsection{Потоки событий}
\begin{Def}
    \textbf{Поток событий}~--- последовательность одинаковых событий,
    происходящих одно за другим через промежутки времени случайной длины.
\end{Def}
\begin{Des}
    Пусть $n(t_1,t_2)$~--- число событий из потока, произошедщших на промежутке
    $[t_1, t_2)$.
\end{Des}
\begin{Def}
    Предположим, что для всех $t$ существует и конечен предел
    \begin{align*}
      & \lambda(t) = \lim_{n \to 0}\frac{P(n(t,t+h) > 0)}{h}
    \end{align*}
    Он называется \textbf{интенсивностью потока событий.}
\end{Def}
\begin{Def}
    Потко называется \textbf{однородным (стационарным)}, если законы
    распределения $n(t_1,t_2)$ и $n(t_1+s, t_2+s)$ совпадают для всех $s \geq
    0$.
\end{Def}
\begin{Des}
    Для однородных потоков
    \begin{align*}
      & \PP\left( n(s,s+t) = m \right) = \PP\left( n(0,t) = m \right)
    \end{align*}
    и обозначим
    \begin{align*}
      & \eta(t) = n(0, t)
    \end{align*}
    а
    \begin{align*}
      & \lambda(t) = \lim_{n \to 0}\frac{P(n(t,t+h) > 0)}{h} = \lim_{n \to 0}\frac{P(n(0,h) > 0)}{h} = \lambda(0) = \lambda
    \end{align*}  
\end{Des}
\begin{Def}
    Поток называется \textbf{ординарным}, если
    \begin{align*}
      & \forall t \geq 0, \ h > 0 \ \PP\left( n(t,t+h) = 1 \right) = \lambda(t)h+o(h), \ h \to 0
    \end{align*}
    \begin{align*}
      & \forall t \geq 0, \ h > 0 \ \PP\left( n(t,t+h) > 1 \right) = o(h), \ h \to 0
    \end{align*}
\end{Def}
\begin{corollary}
    \begin{align*}
      & \forall t \geq 0, \ h > 0 \ \PP\left( n(t,t+h) = 0 \right) = 1-\lambda(t) h + o(h), \ h \to 0
    \end{align*}
\end{corollary}
\begin{Prop}
    Для ожнородного потока
    \begin{align*}
      & \forall t \geq 0, \ h > 0 \ \PP\left( n(t,t+h) = 1 \right) = \lambda h+o(h), \ h \to 0
    \end{align*}
    \begin{align*}
      & \forall t \geq 0, \ h > 0 \ \PP\left( n(t,t+h) = 0 \right) = 1 - \lambda h+o(h), \ h \to 0
    \end{align*}
\end{Prop}
\begin{Def}
    Рассмотрим $0 \leq t_1 \leq \dots \leq t_{m+1}$, $\xi_k = n(t_k,t_{k+1})$.
    Если эти величины независимы в совокупности, то поток называется
    \textbf{потоком без последействия.}
\end{Def}
\begin{Prop}
    Рассмотрим ординарный поток без последействия $X(t) = n(0,t)$. У него
    независимы приращения, а значит, он марковский. Его множество состояний есть
    $\NN_0$, а значит, это марковская цепь. Пусть между событиями обязательно
    происходит конечное время. Если процесс однороден, то и цепь однородна с
    интенсивностями перехода между состояниями $\lambda$.
    \\
    Это будет являться пуассоновским процессом.
\end{Prop}
\begin{Def}
    Ординарный поток без последействия называют \textbf{пуассоновским потоком
      событий}.
\end{Def}
\begin{Def}
    Однородный пуассоновский поток называют \textbf{простейшим потоком событий}.
\end{Def}
\subsection{Эквивалентные определения пуассоновского процесса}
\begin{enumerate}
    \item Первый подход.
    \begin{Def}
        Случайная функция $\left\{ K(t) \mid t \geq 0 \right\}$ называется
        \textbf{пуассоновским процессом интенсивности $\lambda$}, если
        \begin{enumerate}
            \item $K(0) = 0$ почти наверное;
            \item $K(t)$~--- процесс с независимыми приращениями;
            \item $\forall t > s \geq 0$ $K(t) - K(s) \in Po(\lambda(t-s))$
        \end{enumerate}
    \end{Def}
    \item Второй подход.
    \begin{Def} Пусть $\{\xi_n\}$~--- последовательность независимых случайных
        величин с распределением $Exp(\lambda)$. Обозначим
        \begin{align*}
          & S_0 = 0, \ S_n = \sum_{i=1}^n \xi_i
        \end{align*}
        Введём процесс
        \begin{align*}
          & X(t) = \sup \{n \mid S_n \leq t\}
        \end{align*}
        Это будет пуассоновский с параметром $\lambda$ процесс.
    \end{Def}
    \item Третий подход.
    \begin{Def}
        Пусть $K(t), \ t \geq 0$~--- непрерывная марковская цепь с $S = \NN_0$,
        непрерывным временем, непрерывная справа, процесс чистого рождения
        интенсивности $\lambda$. Тогда это пуассоновский процесс интенсивности
        $\lambda$.
    \end{Def}
    \item Четвертый подход.
    \begin{Def}
        Пусть дан простейший пуассоновский поток событий. Тогда $K(t) = n(0,t)$
        будет пуассоновским процессом.
    \end{Def}
\end{enumerate}