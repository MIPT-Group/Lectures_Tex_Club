\newpage
\lecture{8}{Дискретные цепи Маркова}
\section{Марковские процессы}
\subsection{Марковские процессы}
Динамические системы~--- системы, будущие состояния которых определяются лишь
текущим.
\begin{Def}
    Случайный процесс $\{X(t), \ t \in T\}$ называется \textbf{марковским},
    если
    \begin{align*}
      & \forall n \geq 2, \ \forall t_1 < t_2 < \dots < t_n \in T, \ \forall X_1, \dots, X_n, \ \forall \cB \ \PP\left( X(t_{n+1}) \in\cB \mid X(t_n) = X_n, \right. \\
      & \left. X(T_{n-1}) = X_{n-1}, \dots, X(t_1) = X_1 \right) = \PP\left( X(t_{n+1}) \in\cB \mid X(t_n) = X_n \right)
    \end{align*}
    (для которых вероятности существуют).
\end{Def}
Говорят, что верхняя условная вероятность равна нижней; будущее при
фиксированном настоящем не зависит от прошлого.
\begin{theorem} (без доказательства)
    \\
    Всякий процесс с независимыми приращениями марковский.
\end{theorem}
\begin{corollary}
    Винеровский, пуассоновский процессы, случайные блуждания есть марковские процессы.
\end{corollary}
\begin{Note}
    Из марковости не следует независимость приращений.
\end{Note}
\subsection{Классификация марковских процессов}
\begin{Def}
    $S$~--- \textbf{множество состояний}.
\end{Def}
\begin{Def}
    $T$~--- \textbf{множество времён}.
\end{Def}
\textbf{Классы марковских процессов}
\begin{enumerate}
    \item Дискретное $S$, дискретное $T$~--- дискретные цепи Маркова;
    \item Дискретное $S$, непрерывное $T$~--- непрерывные цепи Маркова;
    \item Непрерывное $S$, дискретное $T$
    \item Непрерывное $S$, непрерывное $T$
\end{enumerate}
\begin{example}
    Винеровский процесс есть дискретная марковская цепь, пуассоновский~---
    непрерывная, случайное блуждание~--- третий тип.
\end{example}
\section{Дискретные цепи Маркова}
\subsection{Дискретные цепи Маркова}
\begin{Def}
    Марковский процесс с дискретным $S \subseteq \ZZ$ и счетным $T = \NN_0$
    называется \textbf{дискретной марковской цепью}.
\end{Def}
\begin{Def}
    Случайная последовательность
    \begin{align*}
      \{x_k\}_{k=1}^\infty, \ \forall k \ x_k \in S \subseteq \ZZ, \ \left| S \right|\leq \infty
    \end{align*}
    со свойством
    \begin{align*}
      & \forall n \geq 1, \ \forall m_0 < m_1 < \dots < m_n \ \PP\left( X_{m_n} = x_n \mid X_{m_{n-1}} = x_{n-1}, \dots, X_{m_0} = x_0 \right) = \\
      & = \PP\left( X_{m_n} = x_n \mid X_{m_{n-1}} = x_{n-1}\right)
    \end{align*}
    где эти вероятности определены, называется \textbf{дискретной марковской
      цепью (ДМЦ)}.
\end{Def}
\begin{Def}
    Если состояний конечное число, то ДМЦ называется \textbf{конечной}, иначе \textbf{счетной}.
\end{Def}
\begin{Note}
    Свойство из определения $10.2$ равносильно свойству
    \begin{align*}
      & \forall n \geq 1 \ \PP\left( X_{n} = x_n \mid X_{n-1} = x_{n-1}, \dots, X_{0} = x_0 \right) = \PP\left( X_{n} = x_n \mid X_{n-1} = x_{n-1} \right)
    \end{align*}  
\end{Note}
\begin{Def}
    \textbf{Траектория} марковской цепи~--- последовательность её состояний.
    $X_n$~--- состояние цепи в момент $n$, запись $\{X_n = j\}$ читается как
    <<на шаге $n$ цепь находится в состоянии $j$>>.
\end{Def}
\begin{Def}
    \textbf{Стохастический граф}~--- граф, вершины которого есть состояния, а
    ребра~--- переходы (с некоторыми вероятностями).
\end{Def}
\subsection{Конечномерные распределения}
Пусть
\begin{align*}
  & m_0 < m_1 < \dots < m_n \\
  & x_0, x_1, \dots, x_n
\end{align*}
и для вектора
\begin{align*}
  & \left( X_{m_0}, X_{m_1}, \dots, X_{m_n} \right)
\end{align*}
запишем функцию вероятности
\begin{align*}
  & \PP \left( X_{m_0}=x_0, X_{m_1}=x_1, \dots, X_{m_n}=x_n \right) = \PP\left( X_{m_n} = x_n \mid X_{m_{n-1}} = x_{n-1}, \dots, X_{m_0} = x_0 \right)\cdot \\
  & \cdot \PP\left( X_{m_{n-1}} = x_{n-1}, \dots, X_{m_0} = x_0 \right) = \PP\left( X_{m_n} = x_n \mid X_{m_{n-1}} = x_{n-1}\right)\PP\left( X_{m_{n-1}} = x_{n-1}, \dots, \right. \\
  & \left. X_{m_0} = x_0 \right) = \dots = \PP\left( X_{m_0} = x_0 \right) \prod_{k=1}^n \PP\left( X_{m_k} = x_{k} \mid X_{m_{k-1}} = x_{k-1} \right)
\end{align*}
\begin{Def}
    Вероятность
    \begin{align*}
      & p_{ij}(m,n) = \PP \left( X_n=j \mid X_{m} = i\right)
    \end{align*}
    назовем \textbf{вероятностью перехода из из состояния $i$ в момент $m$ в
      состояние $j$ в момент $n$} ($m \leq n$).
\end{Def}
\begin{Def}
    Матрица
    \begin{align*}
      & P(m,n) = \left| \left| \begin{matrix} p_{ij}(m,n) \end{matrix} \right| \right|
    \end{align*}
    называется \textbf{матрицей перехода цепи от момента $m$ к моменту $n$} ($m \leq n$).
\end{Def}
\begin{example}
    Пусть цепь с состояниями
    \begin{align*}
      & S = \{1, 2, \dots, N\}, \ N < \infty
    \end{align*}
    и матрицей переходов
    \begin{align*}
      & P(m,n) = \left[ \begin{matrix}
              p_{11}(m,n) & p_{12}(m,n) & \dots & p_{1N}(m,n) \\
              p_{21}(m,n) & p_{22}(m,n) & \dots & p_{2N}(m,n) \\
              \dots & \dots & \dots & \dots \\
              p_{N1}(m,n) & p_{N2}(m,n) & \dots & p_{NN}(m,n) \\
          \end{matrix} \right]
    \end{align*}
\end{example}
\begin{Note}
    Для всякой цепи Маркова
    \begin{align*}
      & \sum_{j \in S} \PP \left( X_n = j \mid X_{m}=i \right) =1
    \end{align*}
    что равносильно
    \begin{align*}
      & \sum_{j \in S} p_{ij} = 1
    \end{align*}   
\end{Note}
\begin{Def}
    Матрицы вида
    \begin{align*}
      & P(m,n) = \left[ \begin{matrix}
              p_{11}(m,n) & p_{12}(m,n) & \dots & p_{1N}(m,n) \\
              p_{21}(m,n) & p_{22}(m,n) & \dots & p_{2N}(m,n) \\
              \dots & \dots & \dots & \dots \\
              p_{N1}(m,n) & p_{N2}(m,n) & \dots & p_{NN}(m,n) \\
          \end{matrix} \right]
    \end{align*}
    у которых
    \begin{align*}
      & \sum_{j \in S} p_{ij} = 1
    \end{align*}   
    \begin{align*}
      & \forall i, j \ p_{ij} \in [0;1]
    \end{align*}
    называются \textbf{стохастическими матрицами}.
\end{Def}
\begin{theorem} Уравнение Колмогорова-Чепмена
    \\
    Для любых $n \geq k \geq m \geq 0$ выполняется
    \begin{align*}
      & P(m,n) = P(m,k)P(k,n) \Leftrightarrow p_{ij}(m,n) = \sum_{l \in S}p_{il}(m,k)p_{lj}(k,n)
    \end{align*}
\end{theorem}
\begin{Proof}
    По формуле полной вероятности
    \begin{align*}
      & p_{ij}(m,n) = \PP\left( X_n = j \mid X_m = i \right) = \sum_{l \in S}\PP\left( X_n = j \mid X_m = i, X_k = l \right)\PP\left( X_k = l \mid X_m = i \right) = \\
      & = \sum_{l \in S}\PP\left( X_n = j \mid X_k = l \right)\PP\left( X_k = l \mid X_m = i \right) = \sum_{l \in S} p_{il}(m,k)p_{lj}(k,n)
    \end{align*}
\end{Proof}
\begin{corollary}
     \begin{align*}
      & P(0,n) = P(0,1)P(1,n) = \dots = P(0,1)P(1,2)\dots P(n-1,n)
    \end{align*}  
\end{corollary}
\begin{Def}
    Вероятность $\pi_k(n) = \PP\{X_n = k\}$ называется \textbf{вероятностью
      состояния $k$ в момент $n$}.
\end{Def}
\begin{Def}
    Вектор $\pi(n) = \left[ \pi_0(n), \pi_1(n), \dots \right]^T$ называется
    \textbf{распределением вероятностей состояний в момент $n$}.
\end{Def}
\begin{theorem}
    \begin{align*}
      & \pi(n) = P(n-1,n)\pi(n-1)
    \end{align*}
\end{theorem}
\begin{Proof}
    \begin{align*}
      & \PP\{X_n=k\} = \sum_{j \in S}\{X_n-k \mid X_{n-1}=j\}\PP\{X_{n-1}=j\} = \sum_{j \in S}p_{jk}(n-1,n)\pi_{j}(n-1) = \pi_k(n)
    \end{align*}
\end{Proof}
\subsection{Однородные цепи Маркова}
\begin{Def}
    Если
    \begin{align*}
      & \forall n\geq m, \ \forall k \ P(m,n) = P(m+k, n+k)
    \end{align*}
    то цепь Маркова называется \textbf{однородной}, иначе \textbf{неоднородной}.
\end{Def}
\begin{Des}
    Введем обозначение
    \begin{align*}
      & P(m,n) = P(0, n-m) \Rightarrow P(0,n) = P(n)
    \end{align*}
\end{Des}
\begin{theorem} Уравнение Колмогорова-Чепмена
    \\
    Для любых $n \geq m \geq 0$ выполняется
    \begin{align*}
      & P(n) = P(n-m)P(m) \Leftrightarrow p_{ij}(n) = \sum_{l \in S}p_{il}(n-m)p_{lj}(m)
    \end{align*}
\end{theorem}
\begin{corollary}
    \begin{align*}
      & P\left( \sum_{i=1}^m n_i \right) = \prod_{i=1}^m P(n_i) \Leftrightarrow p_{ij}\left( \sum_{i=1}^m n_i \right) = \sum_{k_1 \in S}\dots \sum_{k_m \in S} p_{ik_1}(n_1)\dots p_{k_m j}(n_m)
    \end{align*}
\end{corollary}
\begin{Note}
    \begin{align*}
      & \forall \forall p_{ij}\left( \sum_{i=1}^m n_i \right) \geq p_{ik_1}(n_1)\dots p_{k_m j}(n_m)
    \end{align*}
\end{Note}
\begin{Des}
    Введем обозначение
    \begin{align*}
      & P(n-1,n) = P(1) = P
    \end{align*}
\end{Des}
Тогда
\begin{theorem}
    \begin{align*}
      & P(n) = P(1)^n = P^n \Rightarrow \pi(n) = \pi(0)\left( P^T \right)^n
    \end{align*}
\end{theorem}
Всё, что происходит в однородной цепи, определяется $\pi(0)$ и $P$.
\begin{Note}
    $P$~--- стохастическая матрица, и $n$-я степень стохастической матрицы также
    стохастическая матрица.
\end{Note}
\begin{theorem} существования (без доказательства)
    \\
    Пусть дано счетное $S$, последовательность $\pi_k$ и стохастическая матрица
    $P$. Тогда существует вероятностное пространство и определенная на нем цепь
    Маркова с множеством состояний $S$, начальным распределением $\pi_k =
    \pi_k(0)$ и матрицей переходов за один шаг $P$.
\end{theorem}
\begin{Note}
    Конечное множество легко получить, добив с некоторого номера
    последовательность нулями.
\end{Note}
\begin{example}
    Стохастическому графу
    \\
    соответствует матрица переходов
    \begin{align*}
      & P = \left[ \begin{matrix}
              1/2 & 1/2 \\
              1 & 0
          \end{matrix} \right]
    \end{align*}
    \begin{align*}
      & P(2) = P^2 = \left[ \begin{matrix}
              1/2 & 1/2 \\
              1 & 0
          \end{matrix} \right]^2 = \left[ \begin{matrix}
              3/4 & 1/4 \\
              1/2 & 1/2
          \end{matrix} \right] \Rightarrow p_{00}(2) = \frac{3}{4}, \ p_{01}(2) = \frac{1}{4}, p_{10}(2) = \frac{1}{2}, p_{11}(2) = \frac{1}{2}
    \end{align*}
    Всё зависит от начального распределения. Допустим,
    \begin{align*}
      & \pi(0) = \left[ 1,0 \right]^T
    \end{align*}
    Тогда
    \begin{align*}
      & \pi(1) = P^T\pi(0) = \left[ \frac{1}{2},\frac{1}{2} \right]^T
    \end{align*}  
\end{example}