% \newpage
\lecture{2}{Моментные функции и пуассоновский процесс}
\section{Моментные функции случайных процессов}
\begin{Def}
    \textbf{Математическим ожиданием} случайного процесса $\left\{\xi(t), \ t \in T\right\}$ называется функция $m_\xi: T \mapsto \RR$, такая, что
    \begin{align*}
      & \forall t \in T \ m_\xi(t) = \EE \xi(t)
    \end{align*}
    \begin{align*}
      & m_\xi(t) = \int xdF_\xi(x,t)
    \end{align*}
    В дискретном случае этот интеграл приобретает вид
    \begin{align*}
      & m_\xi(t) = \sum_i x_i \PP\left(\xi \mid t\right)
    \end{align*}
    а в непрерывном
    \begin{align*}
      & m_\xi(t) = \int_\RR xf_\xi(x,t) dx
    \end{align*}
\end{Def}
\begin{Def}
    \textbf{Корреляционной функцией} случайного процесса $\left\{\xi(t), \ t \in T\right\}$ называется функция $R_\xi: T\times T \mapsto \RR$, такая, что
    \begin{align*}
      & \forall t_1, t_2 \in T \ R_\xi(t_1, t_2) = \cov(\xi(t_1), \xi(t_2) = \EE \cent{\xi}(t_1)\cent{\xi}(t_2) = \EE \left(\xi(t_1) - \EE \xi(t_1)\right)\left(\xi(t_2) - \EE \xi(t_2)\right) = \\
      & = \EE \xi(t_1)\xi(t_2) - \EE \xi(t_1) \EE \xi(t_2)
    \end{align*}
\end{Def}
\begin{Def}
    \textbf{Ковариационной функцией} случайного процесса $\left\{\xi(t), \ t \in T\right\}$ называется функция $K_\xi: T\times T \mapsto \RR$, такая, что
    \begin{align*}
      & \forall t_1, t_2 \in T \ K_\xi(t_1, t_2) = \EE \xi(t_1)\xi(t_2)
    \end{align*}
\end{Def}
Эти две функции связаны между собой:
\begin{align*}
  & R_\xi(t_1, t_2) = K_\xi(t_1, t_2) - m_\xi(t_1)m_\xi(t_2)
\end{align*}
Также они могут быть выражены в терминах многомерной функции распределения:
\begin{align*}
  & R_\xi(t_1, t_2) = \int_{\RR} (x-m_\xi(t_1))(y-m_\xi(t_2))dF_\xi(x,y;t_1,t_2)
\end{align*}
\begin{align*}
  & K_\xi(t_1, t_2) = \int_{\RR} xy dF_\xi(x,y;t_1,t_2)
\end{align*}
\begin{Def}
    \textbf{Дисперсией} случайного процесса $\left\{\xi(t), \ t \in T\right\}$ называется функция $D_\xi: T \mapsto \RR$, такая, что
    \begin{align*}
      & D_\xi(t) = \EE \cent{\xi}^2(t) = \EE \left(\xi(t) - \EE \xi(t)\right)^2 = \EE \xi^2(t) - \EE^2 \xi(t)
    \end{align*}
\end{Def}
Корреляционная функция имеет следующие свойства:
\begin{itemize}
    \item $R_\xi(t,t) = D_\xi(t) \geq 0$
    \item $R_\xi(t_1,t_2) = R_\xi(t_2,t_2)$
    \item $\left|R_\xi(t_1,t_2)\right|\leq \sqrt{D\xi(t_1)D_\xi(t_2)}$
\end{itemize}
\begin{Def}
    \textbf{Взаимной корреляционной функцией} случайных процессов $\left\{\xi(t), \ t \in T\right\}$ и $\left\{\eta(t), \ t \in T\right\}$ называется функция $R_{\xi, \eta}: T\times T \mapsto \RR$, такая, что
    \begin{align*}
      & \forall t_1, t_2 \in T \ R_{\xi, \eta}(t_1, t_2) = \EE \cent{\xi}(t_1)\cent{\eta}(t_2)
    \end{align*}
\end{Def}
\begin{Def}
    \textbf{Характеристической функцией} случайного процесса $\left\{\xi(t), \ t \in T\right\}$ называется функция
    \begin{align*}
      & \varphi_{\xi(t)}(s) = \EE \exp (i s \xi(t)), \ s \in \RR
    \end{align*}
\end{Def}
\section{Пуассоновский процесс}
\subsection{Пуассоновский процесс}
Пуассоновский процесс определен на $T = [0; +\infty]$, обозначается как $K(t)$ и
представляет собой \textit{счетчик событий}.
\begin{theorem} Пуассона (из теории вероятностей)
    \\
    Пусть
    \begin{align*}
      & \{\xi_i\}_{i=1}^n, \ \xi_i \in Be(p), \ pn \to \lambda > 0, \ n \to \infty
    \end{align*}
    Тогда
    \begin{align*}
      & \sum_{i=1}^n\xi_i \to \xi \in Po(\lambda), \ n \to \infty
    \end{align*}    
\end{theorem}
\begin{Def}
    Случайная функция $\left\{ K(t) \mid t \geq 0 \right\}$ называется
    \textbf{пуассоновским процессом интенсивности $\lambda$}, если
    \begin{enumerate}
        \item $K(0) = 0$ почти наверное;
        \item $K(t)$~--- процесс с независимыми приращениями;
        \item $\forall t > s \geq 0$ $K(t) - K(s) \in Po(\lambda(t-s))$
    \end{enumerate}
\end{Def}
\begin{Def}
    $K(t)$ называется \textbf{процессом с независимыми приращениями}, если
    \begin{align*}
      & \forall n \geq 1, \ 0 \leq t_1 \leq t_2 \leq t_n \ K(t_1), K(t_2) - K(t_1), \dots, K(t_n) - K(t_{n-1})
    \end{align*}
    независимы в совокупности.
\end{Def}
\textbf{Свойства пуассоновского процесса}
\begin{enumerate}
    \item Все значения~--- натуральные числа.
    \item Каждая реализация не убывает.
    \item $\EE K(t) = \DD K(t) = \lambda t$
    \item Корреляционная функция
    \begin{align*}
      & t > s: \ R_K(t,s) = \EE K(t)K(s) - \EE K(t) \EE K(s) = \EE K(t)K(s) - \lambda^2 ts = \EE(K(t) - K(s))K(s) + \\
      & + \EE^2K(s) - \lambda^2 ts = \EE(K(t)-K(s))\EE K(s) + \DD K(s) + \EE^2K(s) - \lambda^2 ts = \lambda(t-s)\lambda s+ \lambda s + \\
      & + (\lambda s)^2 - \lambda^2 ts = \lambda s = K_K(t,s)
    \end{align*}
    \begin{align*}
      & t = s: \ R_K(t,t) = \EE K(t)K(t) - \EE K(t) \EE K(t) = \DD K(t) + \EE^2 K(t) - \EE^2 K(t) = \lambda t = K_K(t,t)
    \end{align*}
    \begin{align*}
      & R_K(t,s) = K_K(t,s) = \lambda \min\{t,s\}
    \end{align*}  
\end{enumerate}
\begin{theorem} Явная конструкция пуассоновского процесса
    \\
    Пусть $\{\xi_n\}$~--- последовательность независимых случайных величин с
    распределением $Exp(\lambda)$. Обозначим
    \begin{align*}
      & S_0 = 0, \ S_n = \sum_{i=1}^n \xi_i
    \end{align*}
    Введём процесс
    \begin{align*}
      & X(t) = \sup \{n \mid S_n \leq t\}
    \end{align*}
    Это будет пуассоновский с параметром $\lambda$ процесс.
\end{theorem}
\begin{Proof}
    Рассмотрим случайный вектор $(S_1, \dots, S_n)$.
    \begin{align*}
      & \left[ \begin{matrix}
              S_1 \\
              \dots \\
              S_n
          \end{matrix} \right] =\left[ \begin{matrix}
              1 & 0 & \dots & 0 \\
              1 & 1 & \dots & 0 \\
              \dots & \dots & \dots & \dots \\
              1 & 1 & \dots & 1
          \end{matrix} \right] = \left[ \begin{matrix}
              \xi_1 \\
              \dots \\
              \xi_n
          \end{matrix} \right]
    \end{align*}
    \begin{align*}
      & p_{S_1, \dots, S_n}(x_1, \dots, x_n) = p_{\xi_1, \dots, \xi_n}(x_1, x_2-x_1, \dots, x_n-x_{n-1}) = \prod_{j=1}^n p_{\xi_j} = \prod_{j=1}^n \lambda e^{-\lambda(x_j-x_{j-1})}\chi_{x_{j}\geq x_{j-1}} = \\
      & = \lambda^n e^{\lambda x_{n}} \chi_{x_n \geq x_{n-1} \geq \dots \geq x_1 \geq 0}
    \end{align*}
    Проверим пункты определения.
    \begin{itemize}
        \item Равенство нулю почти наверное очевидно следует из построения.
        \item Докажем теперь независимость в совокупности: покажем выполнимость
        определения.
        \begin{align*}
          & \forall n \geq 1, \ \forall t_1, \dots, t_n, \ \forall k_1, \dots, k_n \in \NN_0 \\
          & \PP(X(t_n)-X(t_{n-1}) = k_n-k_{n-1}, \dots, X(t_2) - X(t_1) = k_2-k_1, X(t_1) = k_1) = \PP(X(t_n)- \\
          & - X(t_{n-1}) = k_n-k_{n-1})\dots \PP(X(t_2) - X(t_1) = k_2-k_1) \PP(X(t_1) = k_1)
        \end{align*}
        Это значит, что
        \begin{align*}
          & S_{k_1} \in [0; t_{1}), S_{k_2} \in [t_1, t_2), \dots, S_{k_n} \in [t_{n-1}, t_n)
        \end{align*}
        \begin{align*}
          & \PP(X(t_n)-X(t_{n-1}) = k_n-k_{n-1})\dots \PP(X(t_2) - X(t_1) = k_2-k_1) \PP(X(t_1) = k_1) = \\
          & = \PP(S_{k_1} \in [0; t_1)) \dots \PP(S_{k_2} \in [t_1, t_2)) \PP(S_{k_1} \in [0; t_{1})) = \underset{\{x_{k_i+1}, \dots, x_{k_{i+1}}\} \subseteq (t_i, t_{i+1})}{\iint \dots \iint}\lambda^{k_n+1}e^{-\lambda x_{k_n+1}}\cdot \\
          & \cdot \chi_{0 < x_1 < \dots < x_{k_n+1}}dx_1dx_2\dots dx_{k_n+1} = \int_{t_n}^{\infty} \lambda e^{-\lambda x_{k_n+1}}dx_{k_n+1}\lambda^{k_n} \cdot \\
          & \cdot\underset{\{x_{k_i+1}, \dots, x_{k_{i+1}}\} \subseteq (t_i, t_{i+1})}{\iint \dots \iint}\chi_{0 < x_1 < \dots < x_{k_n}}dx_1dx_2\dots dx_{k_n} = e^{-\lambda t_{k_n+1}}\lambda^{k_n}\cdot \\
          & \cdot \prod_{i=1}^n \underset{\{x_{k_i+1}, \dots, x_{k_{i+1}}\} \subseteq (t_i, t_{i+1})}{\iint \dots \iint}\chi_{0 < x_{k_{i-1}} < \dots < x_{k_{i}-1}}dx_{k_{i-1}}\dots dx_{k_i-1} =  e^{-\lambda t_{k_n+1}}\lambda^{k_n} \cdot \\
          & \cdot \prod_{i=1}^n \frac{(t_i-t_{i-1})^{k_i-k_{i-1}}}{(k_i-k_{i-1})!} = \prod_{i=1}^{n} \frac{(\lambda(t_i-t_{i-1}))^{k_i-k_{i-1}}}{(k_i-k_{i-1})!}e^{-\lambda(t_i-t_{i-1})}
        \end{align*}
        Это распределения Пуассона с параметром $\lambda(t_i-t_{i-1})$ (для
        приращений).
        \item Вычислив это, можем получить, что итоговый процесс пуассоновский.
    \end{itemize}
\end{Proof}
\textbf{Свойства пуассоновского процесса}
\begin{enumerate}
    \item Скачки:
    \begin{align*}
      & \tau_1 = S_1, \dots, \tau_n = S_n; \ \tau_n \in Erl(n, \lambda)
    \end{align*}
    Распределение Эрланда.
    \item Длительность:
    \begin{align*}
      & \tau_n - \tau_{n-1} = S_n-S_{n-1}\in Exp(\lambda)
    \end{align*}
    \item Величины скачков:
     \begin{align*}
      & \PP\{\exists \text{ скачок величины }\geq 2\} = \PP\{\exists n: \ S_n = S_{n+1}\} = \PP\{\exists n: \ \xi_{n+1} = 0\} = 0
    \end{align*}  
\end{enumerate}
\begin{Def}
    Пусть $\{\xi_n\}$~--- последовательность независимых случайных величин.
    Обозначим
    \begin{align*}
      & S_0 = 0, \ S_n = \sum_{i=1}^n \xi_i
    \end{align*}
    Введём процесс
    \begin{align*}
      & X(t) = \sup \{n \mid S_n \leq t\}
    \end{align*}
    Это называется \textbf{процессом восстановления.}
\end{Def}