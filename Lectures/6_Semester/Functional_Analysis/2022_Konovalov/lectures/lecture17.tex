\begin{theorem}
Пусть $\displaystyle E$ -- линейное нормированное пространство, $\displaystyle \{x_{n}\} \subset E$. Тогда
\begin{equation*}
x_{n}\xrightarrow[\text{сл}]{E} x\Leftrightarrow \{\Vert x_{n}\Vert \}\text{ -- ограничена} ,\ \forall f\in S\subset E^{*} :\ \overline{[ S]} =E^{*} \hookrightarrow f( x_{n})\xrightarrow[n\rightarrow \infty ]{} f( x) .
\end{equation*}
\end{theorem}
\begin{proof}
Уже известно, что $\displaystyle x_{n}\xrightarrow[\text{сл}]{E} x\Leftrightarrow F_{x_{n}}\xrightarrow[n\rightarrow \infty ]{} F_{x}$ поточечно, где $\displaystyle F_{x} :\ E^{*}\rightarrow \mathbb{K}$, $\displaystyle F_{x}( f) =f( x)$. Следствие из теоремы Банаха-Штейнгауза утверждает, что, если $\displaystyle E_{1}$ -- банахово пространство, $\displaystyle E_{2}$ -- линейное нормированное пространство, $\displaystyle \{A_{n}\} \subset \mathcal{L}( E_{1} ,\ E_{2}) ,\ A\in \mathcal{L}( E_{1} ,\ E_{2})$, то $\displaystyle A_{n}\xrightarrow[n\rightarrow \infty ]{} A$ поточечно тогда и только тогда, когда $\displaystyle \{\Vert A_{n}\Vert \}$ -- ограниченная последовательность, $\displaystyle A_{n} s\rightarrow As\ \forall s\in S:\ \overline{[ S]} =E_{1}$. Теперь возьмем в качестве $\displaystyle E_{1}$ пространство $\displaystyle E^{*}$, а в качестве $\displaystyle E_{2}$ -- $\displaystyle \mathbb{K}$. Тогда $\displaystyle \{F_{x_{n}}\} \subset \mathcal{L}\left( E^{*} ,\ \mathbb{K}\right) ,\ F_{x} \in \mathcal{L}\left( E^{*} ,\ \mathbb{K}\right)$, и $\displaystyle F_{x_{n}}\xrightarrow[n\rightarrow \infty ]{} F_{x}$ поточечно тогда и только тогда, когда $\displaystyle \{\Vert F_{x_{n}}\Vert \}$ -- ограниченная последовательность, и $\displaystyle F_{x_{n}}( f)\xrightarrow[n\rightarrow \infty ]{} F_{x}( f) \ \forall f\in S:\ \overline{[ S]} =E^{*}$. Тогда $\displaystyle \Vert F_{x_{n}}\Vert =\sup _{\Vert f\Vert =1}| F_{x_{n}}( f)| =\sup _{\Vert f\Vert =1}| f( x_{n})| =\Vert x_{n}\Vert $ по следствию из теоремы Хана-Банаха, $\displaystyle F_{x_{n}}\xrightarrow[n\rightarrow \infty ]{} F_{x} \ \forall f\in S\Leftrightarrow f( x_{n})\xrightarrow[n\rightarrow \infty ]{} f( x) \ \forall f\in S$.
\end{proof}
\begin{example}
Пусть $\displaystyle E=l_{p}(\mathbb{R})$, где $\displaystyle p >1$. Тогда $\displaystyle ( l_{p})^{*} \cong l_{q}$, где $\displaystyle \dfrac{1}{p} +\dfrac{1}{q} =1$. То есть $\displaystyle \forall f\in l_{p}^{*} \ f( x) =\sum x_{n} y_{n}$ для $\displaystyle y\in l_{q}$. Пусть $\displaystyle \{e_{n}\}$ -- стандартный базис в $\displaystyle l_{q}$. Тогда по теореме получаем, что
\begin{equation*}
x_{n}\xrightarrow[\text{сл}]{l_{p}} x\Leftrightarrow \{\Vert x_{n}\Vert \}\text{ -- ограничена} ,\ \forall f\in S:\ \overline{[ S]} =l_{p}^{*} \hookrightarrow f( x_{n})\xrightarrow[n\rightarrow \infty ]{} f( x) .
\end{equation*}
Взяв в качестве $\displaystyle S$ -- множество функционалов, порожденных $\displaystyle \{e_{n}\}$ получаем, что $\displaystyle \forall f\in S\ \exists e_{k} :\ f( x_{n}) =\langle x_{n} ,e_{k}\rangle =x_{n}^{k}\xrightarrow[n\rightarrow \infty ]{} f( x) =\langle x,e_{k}\rangle =x^{k}$. То есть, слабая сходимость в $\displaystyle l_{p}$ эквивалентна ограниченности норм элементов и их покоординатной сходимости.
\end{example}

\begin{table}[!h]
        \centering
        
\begin{tabular}{|p{0.03\textwidth}|p{0.10\textwidth}|p{0.10\textwidth}|p{0.18\textwidth}|p{0.45\textwidth}|}
\hline 
    \begin{center}
\end{center}
    & \begin{center}
$\displaystyle E$
\end{center}
    & \begin{center}
$\displaystyle E\cong \dotsc $
\end{center}
    & \begin{center}
$\displaystyle f( x)$
\end{center}
    & \begin{center}
критерий слабой сходимости (в дополнение к ограниченности норм)
\end{center}
    \\
\hline 
    \begin{center}
1
\end{center}
    & \begin{center}
$\displaystyle l_{p} ,\ p >1$
\end{center}
    & \begin{center}
$\displaystyle l_{q}$
\end{center}
    & \begin{center}
$\displaystyle \sum _{n=1}^{\infty } x_{n} y_{n}$
\end{center}
    & \begin{center}
координатная сходимость
\end{center}
    \\
\hline 
    \begin{center}
2
\end{center}
    & \begin{center}
$\displaystyle l_{1}$
\end{center}
    & \begin{center}
$\displaystyle l_{\infty }$
\end{center}
    & \begin{center}
$\displaystyle \sum _{n=1}^{\infty } x_{n} y_{n}$
\end{center}
    & \begin{center}
координатная сходимость
\end{center}
    \\
\hline 
    \begin{center}
3
\end{center}
    & \begin{center}
$\displaystyle L_{p}[ a,\ b]$
\end{center}
    & \begin{center}
$\displaystyle L_{q}[ a,\ b]$
\end{center}
    & \begin{center}
$\displaystyle \int _{a}^{b} f( x) g( x) dx$
\end{center}
    & \begin{center}
$\displaystyle \forall t\in [ a,\ b] \ \int _{a}^{t} f_{n}( x) dx\xrightarrow[n\rightarrow \infty ]{}\int _{a}^{t} f( x) dx$
\end{center}
    \\
\hline 
    \begin{center}
4
\end{center}
    & \begin{center}
$\displaystyle C[ a,\ b]$
\end{center}
    & \begin{center}
$\displaystyle \widetilde{BV}[ a,\ b]$
\end{center}
    & \begin{center}
$\displaystyle \int _{a}^{b} f( x) dg( x)$
\end{center}
    & \begin{center}
$\displaystyle f_{n}\rightarrow f$ поточечно
\end{center}
    \\
    \hline
\end{tabular}
\end{table}


$\displaystyle \widetilde{BV}$ -- множество функций ограниченной вариации, определенных в точках разрыва полусуммой односторонних пределов в этих точках.
\begin{exercise}
Как связана сепарабельность $\displaystyle E$ с сепарабельностью $\displaystyle E^{*}$ и наоборот?
\end{exercise}
\begin{definition}
Множество $\displaystyle M\subset E$ называется \textit{секвенциально слабо замнкутым}, если из $\displaystyle \{x_{n}\} \subset M,\ x_{n}\xrightarrow[\text{сл}]{E} x$ следует, что $\displaystyle x\in M$. 
\end{definition}
\begin{definition}
Последовательность $\displaystyle \{x_{n}\} \subset E$ называется слабо фундаментальной, если $\displaystyle \forall f\in E^{*} \hookrightarrow \{f( x_{n})\}$ -- фундаментальная последовательность.
\end{definition}
\begin{definition}
Пространство $\displaystyle E$ называется секвенциально слабо полным, если любая слабо фундаментальная последовательность является слабо сходящейся.
\end{definition}
\begin{definition}
$\displaystyle E\supset S$ -- \textit{секвенциально слабо компактно}, если из любой последовательности $\displaystyle \{s_{n}\} \subset S$ можно выделить слабо сходящуюся подпоследовательность.
\end{definition}
\begin{exercise}
Является ли $\displaystyle \overline{B}( 0,\ 1)$ в $\displaystyle l_{2}$ секвенциально слабо компактным? Использовать теорему Банаха-Алаоглу.
\end{exercise}
\section{Сопряженные операторы}

Пусть $\displaystyle E_{1} ,\ E_{2}$ -- линейные нормированные пространства, $\displaystyle A\in \mathcal{L}( E_{1} ,\ E_{2}) ,\ g\in E_{2}^{*}$.
\begin{definition}
Оператор $\displaystyle A^{*} :\ E_{2}^{*}\rightarrow E_{1}^{*}$, $\displaystyle \left( A^{*} g\right)( x) =g( Ax)$ называется \textit{сопряженным оператором} к $\displaystyle A$.
\end{definition}