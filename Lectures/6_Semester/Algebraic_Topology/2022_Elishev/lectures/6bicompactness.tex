\section{Бикомпактные топологические пространства}

\subsection{Бикомпактные и компактные пространства}

\begin{definition}
    Топологическое пространство $X$ называется \textit{бикомпактным}, если любое его открытое покрытие содержит конечное подпокрытие.
\end{definition}

\begin{note}
    По теореме Бореля--Лебега, для метрических пространств понятия компактности и бикомпактности совпадают.
\end{note}

\begin{definition}
    Топологическое пространство $X$ называется \textit{компактным}, если любое его бесконечное подмножество имеет предельную точку.
\end{definition}

\begin{note}
    В силу критерия компактности, определение выше является обобщением понятия компактности метрического пространства. Однако компактность топологического пространства $X$, вообще говоря, не гарантирует возможности выделения сходящейся подпоследовательности из произвольной последовательности точек из $X$. Пространства с этим свойством носят название \textit{секвенциально компактных} пространств. Более того,существуют примеры компактных и бикомпактных хаусдорфовых топологических пространств, в которых нет ни одной нестационарной сходящейся последовательности.
\end{note}

\begin{definition}
    Топологическое пространство $X$ называется \textit{счетно-компактным}, если любое его счетное открытое покрытие содержит конечное подпокрытие.
\end{definition}

\begin{theorem}\label{comptheo}
    Пусть $X$ "--- топологическое пространство. Тогда следующие условия эквивалентны:
    \begin{enumerate}
        \item $X$ компактно
        
        \item Любая убывающая последовательность $\Phi_1\supset \Phi_2\supset \dotsb$ непустых замкнутых множеств в $X$ имеет непустое пересечение

        \item $X$ счетно-компактно
    \end{enumerate}
\end{theorem}

\begin{proof}~
    \begin{itemize}
        \item\imp{1}{2}Пусть $\Phi_1 \supset \Phi_2 \supset \dotsb$ "--- убывающая последовательность непустых замкнутых множеств в $X$. Если последовательность стабилизируется, то уже получено требуемое. Иначе --- без ограничения общности можно считать, что все элементы последовательности различны. Для каждого $n \in \N$ выберем точку $x_n\in\Phi_n\backslash \Phi_{n+1}$ и рассмотрим бесконечное множество $M := \lbrace x_n\rbrace$. По предположению, множество $M$ имеет предельную точку $\xi \in X$. Эта точка принадлежит каждому множеству $\Phi_n$. Действительно, если для некоторого $n \in \N$ выполнено $\xi \not\in \Phi_n$, то открытая окрестность $X \backslash \Phi_n$ точки $\xi$ содержит лишь конечное число точек из $M$. Таким образом, пересечение всех $\Phi_n$ непусто.
        
        \item\imp{2}{1}Пусть пространство $X$ некомпактно, тогда существует бесконечное множество $M\subset X$, не имеющее ни одной предельной точки. Выберем счетное подмножество $\{x_n\}\subset M$ и для каждого $n \in \N$ положим $M_n := \lbrace x_n, x_{n+1}, \dotsc\}$, $\Phi_n := \overline{M_n}$. Последовательность $\{\Phi_n\}$ является убывающей последовательностью непустых замкнутых множеств, но при этом имеет пустое пересечение. Действительно, если для некоторой точки $x \in X$ выполнено $x \in \bigcap_{n = 1}^{\infty} \Phi_n$, то любая окрестность точки $x$ пересекается с каждым множеством $M_n$ и, следовательно, содержит бесконечное число точек множества $M$, поэтому $x$ является предельной точкой множества $M$, не имеющего предельных точек --- противоречие.

        \item\imp{2}{3}Рассмотрим счетное открытое покрытие $\Sigma := \{G_n\}$ пространства $X$. Для каждого $n \in \N$ положим $\Gamma_n := G_1\cup \ldots \cup G_n$, $\Phi_n := X\backslash \Gamma_n$. Последовательность $\{\Phi_n\}$ является убывающей последовательностью замкнутых множеств с пустым пересечением, поэтому существует $m \in \N$ такое, что $\Phi_m = \Phi_{m+1} = \dotsb = \emptyset$. Тогда $\lbrace G_1,\ldots, G_m\rbrace$ "--- это искомое конечное подпокрытие.

        \item\imp{3}{2}Пусть $\Phi_1 \supset \Phi_2 \supset \dotsb$ "--- убывающая последовательность непустых замкнутых множеств с пустым пересечением. Для каждого $n \in \N$ положим $G_n := X \backslash \Phi_n$, тогда счетное открытое покрытие $\{G_n\}$ не имеет конечного подпокрытия, поскольку для любых $n_1, \dotsc, n_s \in \N$ таких, что $n_1 < \dotsb < n_s$, выполнено $ G_{n_1}\subset\ldots\subset G_{n_s}$, поэтому верно следующее:
        \[G_{n_1}\cup\ldots\cup G_{n_s} =  G_{n_s} = X \bs \Phi_{n_s} \ne X\]

        Значит, последовательность $\{\Phi_n\}$ имеет непустое пересечение.\qedhere
    \end{itemize}
\end{proof}

\begin{corollary}
    Любое бикомпактное топологическое пространство $X$ является компактным.
\end{corollary}

\begin{definition}
    Пусть $X$ "--- топологическое пространство, $M \subset X$. Точка $\xi \in X$ называется \textit{точкой полного накопления} множества $M$, если пересечение множества $M$ с любой окрестностью точки $\xi$ имеет ту же мощность, что и все множество $M$.
\end{definition}

\begin{theorem}[без доказательства]
    Пусть $X$ "--- топологическое пространство. Тогда следующие условия эквивалентны:
    \begin{enumerate}
        \item Любое бесконечное множество $M\subseteq X$ имеет точку полного накопления
        
        \item Любая убывающая вполне упорядоченная по индексам система $\{\Phi_\alpha\}_{\alpha \in \mc A}$ непустых замкнутых множеств в $X$ имеет непустое пересечение

        \item $X$ бикомпактно
    \end{enumerate}
\end{theorem}

\begin{definition}
    Пусть $X$ "--- топологическое пространство. Система $\{M_\alpha\}_{\alpha \in \mf A}$ подмножеств пространства $X$ называется \textit{центрированной}, если любой конечный набор множеств этой системы имеет непустое пересечение.
\end{definition}

\begin{note}
    Легко видеть, что топологическое пространство $X$ бикомпактно $\lra$ любая центрированная система замкнутых множеств $X$ имеет непустое пересечение. Кроме того, пространство $X$ компактно $\lra$ любое его счетное подмножество имеет точку полного накопления.
\end{note}

\begin{definition}
    Топологическое пространство $X$ называется:
    \begin{itemize}
        \item \textit{Инициально компактным} вплоть до мощности $\mathfrak{a}$, если любое его открытое покрытие мощности $\mathfrak{m} \leq \mathfrak{a}$ содержит конечное подпокрытие

        \item \textit{Финально компактным} начиная с мощности $\mathfrak{a}$, если любое его открытое покрытие мощности $\mathfrak{m} > \mathfrak{a}$ содержит подпокрытие мощности не больше $\mathfrak{a}$
    \end{itemize}    
\end{definition}

\begin{note}
    Бикомпактные пространства инициально компактны вплоть до любой мощности и одновременно финально компактны начиная с любой мощности.
\end{note} 

\subsection{Бикомпактные хаусдорфовы пространства}

\begin{definition}
    Бикомпактное хаусдорфово пространство $X$ называется \textit{бикомпактом}.
\end{definition}

\begin{note}
    Любое компактное метрическое пространство $X$ является бикомпактом.
\end{note}

\begin{proposition}
    Пусть $X$ "--- бикомпактное пространство, $\Phi \subset X$ "--- замкнутое множество. Тогда $\Phi$ тоже является бикомпактным пространством.
\end{proposition}

\begin{proof}
    Пусть $\Sigma = \lbrace G_{\alpha}\rbrace_{\alpha \in \mc A}$ "--- открытое покрытие множества $\Phi$, тогда для любого $\alpha \in \mc A$ множество $\Gamma_{\alpha} = \Phi\cap G_{\alpha}$ является открытым в $\Phi$. Система $\lbrace G_{\alpha}\rbrace_{\alpha \in \mc A} \cup \lbrace X\backslash \Phi\rbrace$ образует открытое покрытие пространства $X$, поэтому она имеет конечное подпокрытие. Пересечение множеств из этого подпокрытия с $\Phi$ дает искомое конечное подпокрытие множества $\Phi$.
\end{proof}

\begin{proposition}
    Пусть $X$ "--- топологическое пространство, и $M \subset X$ образует бикомпактное пространство. Тогда любая система $\Sigma = \lbrace \Gamma_{\alpha}\rbrace_{\alpha \in \mf A}$ открытых множеств, покрывающая $M$, содержит конечную подсистему, покрывающую $M$.
\end{proposition}

\begin{proof}
    Система $\lbrace M\cap \Gamma_{\alpha}\rbrace_{\alpha \in \mf A}$ образует открытое покрытие пространства $M$, так что она содержит конечную подсистему, также являющуюся покрытием. Соответствующие элементам этой системы множества $\Gamma_{\alpha}$ образуют искомую систему.
\end{proof}

\begin{theorem}
    Любое хаусдорфово бикомпактное пространство $X$ является нормальным.
\end{theorem}

\begin{proof}
    Пусть $X$ "--- хаусдорфово бикомпактное пространство.
    \begin{enumerate}
        \item Покажем, что $X$ регулярно. Пусть $x \in X$, и $B \subset X$ "--- замкнутое множество, не содержащее точку $x$. В силу хаусдорфовости, для любой точки $y \in B$ существуют непересекающиеся окрестности $U_y(x), U(y)$ точек $x$ и $y$. Система $\lbrace U(y): y\in B\rbrace$ покрывает множество $B$, которое, в силу замкнутости, тоже образует бикомпактное пространство, поэтому существует конечный набор точек $\{y_1,\ldots, y_s\}$, окрестности которых образуют конечное подпокрытие. Рассмотрим окрестности $U_{y_1}(x), \dotsc, U_{y_s}(x)$ точки $x$. Их пересечение является окрестностью точки $x$, не имеющей общих точек с окрестностью $\bigcup_{i = 1}^sU(y_i)$ множества $B$. В силу произвольности выбора точки $x$ и множества $B$, из этого следует регулярность пространства $X$.

        \item Аналогичным образом покажем, что $X$ нормально. Пусть $A, B \subset X$ "--- непустые дизъюнктные замкнутые подмножества в $X$. Для каждой точки $x \in A$ возьмем пару дизъюнктных окрестностей $U(x)$ и $U_x(B)$. Система $\lbrace U(x): x \in A\rbrace$ покрывает $A$ и, в силу утверждения выше, содержит конечную подсистему $U(x_1),\ldots, U(x_r)$, также покрывающую $A$. Окрестности $U(A) := \bigcup_{j = 1}^rU(x_j)$ и $U(B) := \bigcup_{j = 1}^rU_{x_j}(B)$ являются искомыми. В силу произвольности выбора множеств $A, B$, получено требуемое.\qedhere
    \end{enumerate}
\end{proof}

\subsection{Вторая метризационная теорема Урысона}

\begin{theorem}[вторая теорема Линделефа]
    Пусть $X$ "--- топологическое пространство со счетной базой. Тогда всякая несчетная система $\mathfrak{A}$ открытых множеств в $X$ обладает не более чем счетной подсистемой $\mathfrak{A}_0$, объединение элементов которой совпадает с объединением элементов системы $\mathfrak{A}$.
\end{theorem}

\begin{proof}
    Пусть $\{\Gamma_n\}$ "--- счетная база пространства $X$. Назовем множество $\Gamma_n$ из базы \textit{отмеченным}, если оно содержится по крайней мере в одном множестве из $\mathfrak{A}$. Выберем для каждого отмеченного множества $\Gamma_n$ какое-либо содержащее его множество $G \in \mathfrak{A}$, и получим не более чем счетную подсистему $\mf A_0 \subset \mf A$.
    
    Покажем, что объединение элементов полученной подсистемы равно объединению всех элементов системы $\mathfrak{A}$. Пусть $x \in X$ "--- произвольная точка, содержащаяся в некотором множестве $G \in \mf A$. По утверждению $\ref{propbase}$, существует множество $\Gamma_n$ из базы, содержаще $x$ и содержащееся в $G$. Значит, $\Gamma_n$ "--- отмеченный элемент базы, поэтому существует множество $G' \in \mf A_0$, содержащее его. Оно также содержит точку $x$, что и дает требуемое в силу произвольности выбора точки $x$.
\end{proof}

\begin{corollary}
    Любое компактное пространство $X$ со счетной базой бикомпактно.
\end{corollary}

\begin{proof}
    Пусть $X$ "--- компактное пространство, $\Sigma = \lbrace G_{\alpha}\rbrace_{\alpha \in \mf A}$ "--- его открытое покрытие. В силу второй теоремы Линделефа, $\Sigma$ содержит счетное подпокрытие $\Sigma_0$. Но пространство $X$ компактно и, по теореме \ref{comptheo}, счетно-компактно, поэтому система $\Sigma_0$ содержит конечное подпокрытие.
\end{proof}

\begin{theorem}[вторая метризационная теорема Урысона]
    Пусть $X$ "--- компактное хаусдорфово пространство. Тогда $X$ метризуемо $\lra$ $X$ имеет счетную базу.
\end{theorem}

\begin{proof}~
    \begin{itemize}
        \item[$\la$] По доказанному выше, компактное пространство $X$ со счетной базой бикомпактно, и в силу хаусдорфовости оно также нормально. Тогда, по первой метризационной теореме Урысона, оно метризуемо.

        \item[$\ra$] По теореме \ref{countabledensetheo}, компактное метрическое пространство $X$ содержит всюду плотное счетное множество, то есть является сепарабельным, тогда по теореме \ref{thmmetricseparablecond} оно имеет счетную базу.\qedhere
    \end{itemize}
\end{proof}

\begin{note}
    Бикомпакт, содержащий счетное всюду плотное подмножество, может не иметь счетной базы. Примером такого пространства является \textit{пространство двух стрелок}, получаемое из ординала $2\theta$, где $\theta$ "--- порядковый тип отрезка числовой прямой, отбрасыванием первой и последней точек.
\end{note}

\subsection{Непрерывные отображения бикомпактных пространств}

\begin{theorem}
	Пусть $X, Y$ "--- топологические пространства, $f : X \to Y$ "--- непрерывное отображение, и пространство $X$ бикомпактно. Тогда пространство $f(X)$ тоже бикомпактно.
\end{theorem}

\begin{proof}
    Аналогично теореме \ref{compactimage}.
\end{proof}

\begin{theorem}[Вейерштрасса]
    Пусть $X$ "--- бикомпактное топологическое пространство, $f : X \to \R$ "--- непрерывная функция. Тогда функция $f$ ограниченна и принимает наименьшее и наибольшее значения в некоторых точках множества $X$.
\end{theorem}

\begin{proof}
    Множество $f(X)$ образует бикомпактное подпространство в $\mathbb R$, поэтому оно ограниченно и замкнуто, из чего и следует требуемое.
\end{proof}