\section{Метрические и топологические пространства}

\subsection{Метрические пространства}

\begin{definition}
	\textit{Метрикой} на множестве $X$ называется функция $\rho : X^2 \to \R$, обладающая следующими свойствами:
	\begin{enumerate}
		\item $\forall x, y \in X: \rho(x, y) \ge 0$, причем $\rho(x, y) = 0 \lra x = y$
		\item $\forall x, y \in X: \rho(x, y) = \rho(y, x)$
		\item $\forall x, y, z \in X: \rho(x, z) \le \rho(x, y) + \rho(y, z)$
	\end{enumerate}
	
	Пара $(X, \rho)$ называется \textit{метрическим пространством}.
\end{definition}

\begin{example}
	Рассмотрим несколько примеров метрических пространств:
	\begin{enumerate}
		\item Множество $\R$ является метрическим пространством с метрикой $\rho$, заданной для произвольных $x, y \in \R$ как $\rho(x, y) := |x - y|$
		
		\item Множество $\R^n$ при любом $n \in \N$ является метрическим пространством с метрикой $\rho$, заданной для произвольных $x, y \in \R^n$ как $\rho(x, y) := \sqrt{(x_1 - y_1)^2 + \dotsb + (x_n - y_n)^2}$, а также множество $\R^\infty$ числовых последовательностей с конечной суммой квадратов элементов с аналогичной метрикой
		
		\item Произвольное множество $X$ является метрическим пространством с \textit{дискретной метрикой} $d$, заданной для произвольных $x, y \in \R$ как $d(x, y) := I(x \ne y)$
		
		\item Множество $C[0, 1]$ является метрическим пространством с метрикой $\rho$, заданной для произвольных $f, g \in C[0, 1]$ как $\rho(f, g) := \sup\{|f(x) - g(x)| : x \in [0, 1]\}$
	\end{enumerate}
\end{example}

\begin{note}
	Пусть $(X, \rho)$ "--- метрическое пространство, $Y \subset X$. Тогда пара $(Y, \rho|_{Y^2})$ тоже является метрическим пространством.
\end{note}

\begin{note}
	Пусть $(X_1, \rho_1), \dotsc, (X_n, \rho_n)$ "--- метрическое пространство. Тогда метрическим пространством является пара $(X_1 \times \dotsb \times X_n, \rho)$, где метрика $\rho$ задана для произвольных $x = (x_1, \dotsc, x_n), y = (y_1, \dotsc, y_n) \in X_1 \times \dotsb \times X_n$ следующим образом:
	\[\rho(x, y) := \sqrt{\rho_1(x_1, y_1)^2 + \dotsb + \rho_n(x_n, y_n)^2}\]
\end{note}

\begin{definition}
	Пусть $(X, \rho)$ "--- метрическое пространство, $M, N \subset X$. \textit{Расстоянием между множествами} $M, N$ называется следующая величина:
	\[\rho(M, N) := \inf\{\rho(x, y): x \in M, y \in N\}\]
\end{definition}

\begin{definition}
	Пусть $(X, \rho)$ "--- метрическое пространство, $\epsilon > 0$.
	\begin{itemize}
		\item \textit{$\epsilon$-окрестностью точки} $x \in X$ называется $U(x, \epsilon) := \{y \in X: \rho(y, x) < \epsilon\}$
		
		\item \textit{$\epsilon$-окрестностью множества} $M \subset X$ называется $U(x, \epsilon) := \{y \in X: \rho(y, M) < \epsilon\}$
	\end{itemize}
\end{definition}

\begin{definition}
	Пусть $(X, \rho)$ "--- метрическое пространство, $M \subset X$. Точка $x \in X$ называется \textit{точкой прикосновения} множества $M$, если для любого $\epsilon > 0$ выполнено $U(x, \epsilon) \cap M \ne \emptyset$. \textit{Замыканием} множества $M$ называется множество $\overline M \equiv [M]$ всех его точек прикосновения. Множество $M$ называется \textit{замкнутым}, если $\overline M = M$.
\end{definition}

\begin{note}
	Точка $x$ является точкой прикосновения множества $M \lra \rho(x, M) = 0$.
\end{note}

\begin{definition}
	Пусть $(X, \rho)$ "--- метрическое пространство, $M \subset X$. Точка $x \in X$ называется \textit{внутренней точкой} множества $M$, если существует $\epsilon > 0$ такое, что $U(x, \epsilon) \subset M$. \textit{Внутренностью} множества $M$ называется множество $M^0$ всех его внутренних точек. Множество $M$ называется \textit{открытым}, если $M^0 = M$.
\end{definition}

\begin{definition}
	Пусть $(X, \rho)$ "--- метрическое пространство. Множество всех открытых подмножеств множества $X$ называется \textit{открытой топологией} на $X$ и обозначается через $\tau$, множество всех замкнутых подмножеств называется \textit{замкнутой топологией} и обозначается через $\kappa$.
\end{definition}

\begin{note}
	Пусть $(X, \rho)$ "--- метрическое пространство. Тогда для топологий $\tau$ и $\kappa$ на $X$ выполнены следующие свойства:
	\begin{enumerate}
		\item $\forall G_1, G_2 \in \tau: G_1 \cap G_2 \in \tau$
		\item $\forall \{G_\alpha\}_{\alpha \in \mf A} \subset \tau: \bigcup_{\alpha \in \mf A}G_\alpha \in \tau$
		\item $\forall G \subset X \in \tau: X \bs G \in \kappa$
	\end{enumerate}
	
	В силу свойства $(3)$, свойства, двойственные свойствам $(1)$ и $(2)$, верны для замкнутой топологии $\kappa$.
\end{note}

\subsection{Топологические пространства}

\begin{definition}
	\textit{Топологией} на множестве $X$ называется семейство $\tau$ подмножеств множества $X$, обладающее следующими свойствами:
	\begin{enumerate}
		\item $\emptyset, X \in \tau$
		\item $\forall G_1, G_2 \in \tau: G_1 \cap G_2 \in \tau$
		\item $\forall \{G_\alpha\}_{\alpha \in \mf A} \subset \tau: \bigcup_{\alpha \in \mf A}G_\alpha\in \tau$
	\end{enumerate}
	
	Пара $(X, \tau)$ называется \textit{топологическим пространством}.
\end{definition}

\begin{note}
	Любое метрическое пространство является топологическим пространством с открытой топологией.
\end{note}

\begin{definition}
	Пусть $(X, \tau)$ "--- топологическое пространство, и $Y \subset X$. Множество ${N \subset Y}$ называется \textit{открытым в $Y$}, если существует $M \in \tau$ такое, что $N = M \cap Y$. Семейство подмножеств множества $Y$, открытых в $Y$, называется \textit{индуцированной топологией} на $Y$ и обозначается через $\tau_Y$.
\end{definition}

\begin{note}
	Пара $(Y, \tau_Y)$ из определения выше тоже является топологическим пространством, то есть $Y$ является \textit{подпространством} в $X$.
\end{note}

\begin{definition}
	Пусть $(X, \tau)$ "--- топологическое пространство, $x \in X$. Любое множество $U(x) \in \tau$, содержащее точку $x$, называется \textit{окрестностью точки $x$}.
\end{definition}

\begin{definition}
	Пусть $(X, \tau)$ "--- топологическое пространство, $M \subset X$. Точка $x \in X$ называется \textit{точкой прикосновения} множества $M$, если для любой окрестности $U(x)$ точки $x$ выполнено $U(x) \cap M \ne \emptyset$. \textit{Замыканием} множества $M$ называется множество $\overline M \equiv [M]$ всех его точек прикосновения.
\end{definition}

\begin{definition}
	Пусть $(X, \tau)$ "--- топологическое пространство. Множество $M \subset X$ назы\-вается \textit{замкнутым}, если $X \bs M \in \tau$.
\end{definition}

\begin{theorem}
	Пусть $(X, \tau)$ "--- топологическое пространство, $M \subset X$. Тогда множество $M$ является замкнутым $\lra [M] = M$.
\end{theorem}

\begin{proof}~
	\begin{itemize}
		\item[$\ra$] По условию, множество $X \bs M$ "--- открытое, поэтому никакая его точка не может являться точкой прикосновения множества $M$. Значит, $[M] = M$.
		
		\item[$\la$] По условию, для любой точки $x \in X \bs M$ существует ее окрестность $U(x)$ такая, что $U(x) \subset X \bs M$. Значит, выполнено следующее:
		\[X \bs M = \bigcup_{x \in X \bs M} U(x)\]
		
		Следовательно, $X \bs M \in \tau$, то есть множество $M$ "--- замкнутое. \qedhere
	\end{itemize}
\end{proof}

\begin{proposition}
	Пусть $(X, \tau)$ "--- топологическое пространство. Тогда операция замыкания на $(X, \tau)$ идемпотентна, то есть для любого $M \subset X$ выполнено $[M] = [[M]]$.
\end{proposition}

\begin{proof}
	Зафиксируем произвольное множество $M \subset X$. С одной стороны, выполнено включение $[M] \subset [[M]]$. С другой стороны, если $x \in [[M]]$, то любая окрестность $U(x)$ точки $x$ содержит точку из $[M]$ и потому также является окрестностью этой точки, а значит, содержит и точку из $M$. Следовательно, выполнено и обратное включение.
\end{proof}

\begin{theorem}
	Пусть $(X, \tau)$ "--- топологическое пространство, $M \subset X$. Тогда замыкание $[M]$ является наименьшим по включению замкнутым множеством, содержащим $M$.
\end{theorem}
	
\begin{proof}
	С одной стороны, множество $[M]$ замкнуто и содержит $M$. С другой стороны, если $F \subset X$ "--- замкнутое множество, для которого $M \subset F$, то $[M] \subset [F] = F$. Получено требуемое.
\end{proof}

\begin{proposition}
	Пусть $(X, \tau)$ "--- топологическое пространство. Тогда операция замыкания на $(X, \tau)$ дистрибутивна относительно объединения, то есть для любых множеств $M, N \subset X$ выполнено $[M \cup N] = [M] \cup [N]$.
\end{proposition}

\begin{proof}
	С одной стороны, поскольку $M, N \subset M \cup N$, то выполнены включения $[M], [N] \subset [M \cup N]$ и, следовательно, включение $[M] \cup [N] \subset [M \cup N]$. C другой стороны, множество $[M] \cup [N]$ замкнуто и содержит множество $M \cup N$, поэтому выполнено и обратное включение.
\end{proof}

\begin{note}
	Таким образом, оператор замыкания на топологическом пространстве $(X, \tau)$ удовлетворяет \textit{аксиомам Куратовского}:
	\begin{enumerate}
		\item $\forall M, N \subset X: [M \cup N] = [M] \cup [N]$
		\item $\forall M \subset X: M \subset [M]$
		\item $\forall M \subset X: [[M]] = [M]$
		\item $[\emptyset] = \emptyset$
	\end{enumerate}

	Кроме того, любой оператор, удовлетворяющий аксиомам выше, задает \textit{замкнутую топологию}, на которой он будет замыканием. Для открытой топологии и оператора внутренности можно построить двойственный набор аксиом.
\end{note}

\begin{note}
	Множество всех топологий на множестве $X$ образует частично упорядоченное множеством относительно отношения включения.
\end{note}

\begin{note}
	Не в любом топологическом пространстве каждое конечное множество является замкнутым. Например, на множестве $X = \{ 0, 1\}$ можно задать топологию $\{\emptyset, \{1\}, X\}$. В полученном топологическом пространстве, называемом \emph{связным двоеточием}, множество $\{0\}$ является замкнутым, а $\{1\}$ --- нет.
\end{note}

\begin{definition}
	Пусть $(X, \tau)$ "--- топологическое пространство, $M \subset X$. Точка ${x \in X}$ называется \textit{предельной точкой} множества $M$, если любая его окрестность содержит бесконечно много точек из $M$.
\end{definition}

\begin{note}
	Определение выше в общем случае не эквивалентно привычному определению предельной точки из математического анализа, согласно которому любая окрестность точки $x$ должна содержать хотя бы точку из $M$, отличную от $x$.
\end{note}

\begin{definition}
	Пусть $(X, \tau)$ "--- топологическое пространство. Точка $x \in X$ называется \textit{изолированной}, если множество $\{x\}$ "--- открытое.
\end{definition}

\begin{definition}
	Пусть $(X, \tau)$ "--- топологическое пространство. Множество $M \subset X$ называется:
	\begin{itemize}
		\item \textit{Плотным в $\Gamma \in \tau$}, если $[M] \supset \Gamma$
		\item \textit{Всюду плотным}, если $[M] = X$
		\item \textit{Нигде не плотным}, если оно не плотно ни в одном непустом множестве $\Gamma \in \tau$
	\end{itemize}
\end{definition}

\begin{note}
	Справедливы следующие свойства:
	\begin{enumerate}
		\item Множество ${M \subset X}$ нигде не плотно $\lra$ каждое непустое множество $\Gamma \in \tau$ содержит некоторое непустое множество $\Gamma_0 \in \tau$ такое, что $M \cap \Gamma_0 = \emptyset$.
		
		\item Замыкание нигде не плотного множества нигде не плотно. При этом два взаимно дополнительных множества могут быть всюду плотными, как, например, $\Q$ и $\R\bs \Q$.
		
		\item Если замкнутое множество всюду плотно, то оно совпадает со всем пространством.
		
		\item Если открытое множество всюду плотно, до дополнение к нему нигде не плотно.
		
		\item Любое множество $M$ всюду плотно в подпространстве $\overline{M}$.
	\end{enumerate}
\end{note}

\begin{proposition}
	Пусть $(X, \tau)$ "--- топологическое пространство, множество $F$ "--- замкнутое, множество $G$ "--- открытое. Тогда множество $F \setminus G$ "--- замкнутое, а множество $G \setminus F$ "--- открытое.
\end{proposition}

\begin{proof}
	Достаточно заметить, что выполнены равенства $F\setminus G = F\cap (X\setminus G)$ и $G\setminus F = G\cap(X\setminus F)$.
\end{proof}

\begin{note}
	Далее мы часто не будем явно указывать метрику или топологию на пространстве $X$, предполагая ее фиксированной.
\end{note}

\subsection{Непрерывные отображения}

\begin{definition}
	Пусть $X, Y$ "--- топологические пространства. Отображение $f : X \to Y$ называется:
	\begin{itemize}
		\item \textit{Непрерывным в точке $x_0 \in X$}, если для любой окрестности $U(y_0)$ точки $y_0 = f(x_0)$ существует окрестность $U(x_0)$ точки $x_0$ такая, что $f(U(x_0)) \subset U(y_0)$
		
		\item \textit{Непрерывным}, если оно непрерывно в каждой точке множества $X$
	\end{itemize} 
\end{definition}

\begin{proposition}
	Пусть $X, Y$ "--- топологические пространства, $f: X \to Y$. Тогда ото\-бра\-жение $f$ непрерывно $\lra$ для любого открытого множества $V \subset Y$ множество $f^{-1}(V)$ "--- тоже открытое.
\end{proposition}

\begin{proof}~
	\begin{itemize}
		\item[$\ra$] Зафиксируем произвольное открытое множество $V \subset Y$ и точку $x \in f^{-1}(V)$. По условию, существует окрестность $U$ точки $x$ такая, что $f(U) \subset V$, откуда ${U \subset f^{-1}(V)}$.
		
		\item[$\la$] Зафиксируем произвольную точку $x \in X$. Тогда для любой окрестности $U(y)$ точки $y= f(x)$ выполнено, что $f^{-1}(U(y))$ открыто, то есть $f^{-1}(U(y))$ и является искомой окрестностью точки $x$.\qedhere
	\end{itemize}
\end{proof}

\begin{corollary}
	Пусть $X, Y$ "--- топологические пространства, $f: X \to Y$. Тогда отображение $f$ непрерывно $\lra$ для любого замкнутого множества $F \subset Y$ множество $f^{-1}(F)$ "--- тоже замкнутое.
\end{corollary}

\begin{proposition}
	Пусть $X, Y$ "--- топологические пространства, $f: X \to Y$. Тогда отображение $f$ непрерывно $\lra$ для любого множества $M \subset X$ выполнено включение $f([M]) \subset [f(M)]$.
\end{proposition}

\begin{proof}~
	\begin{itemize}
		\item[$\ra$] Зафиксируем произвольное множество $M \subset X$. Множество $f^{-1}([f(M)])$ замкнуто в силу непрерывности отображения $f$ и содержит $M$, поэтому $f^{-1}([f(M)]) \subset [M]$, откуда $[f(M)] \subset f([M])$.
		
		\item[$\la$] Зафиксируем произвольное замкнутое множество $F \subset Y$, тогда:
		\[f([f^{-1}(F)]) \subset f(f^{-1}([F])) = [F] = F\]
		
		Значит, $[f^{-1}(F)] \subset f^{-1}(F)$, то есть множество $f^{-1}(F)$ "--- замкнутое, поэтому $f$ непрерывно.\qedhere
	\end{itemize}
\end{proof}

\begin{theorem}[о приклеивании непрерывных функций]\label{glueingthm}
	Пусть $X, Y$ "--- топологические пространства, $\{U_\alpha \}_{\alpha \in \mathfrak{A}}$ "--- открытое покрытие множества $X$, $\{f_\alpha: U_\alpha \rightarrow Y\}_{\alpha \in \mathfrak{A}}$ "--- сис\-те\-ма непрерывных отображений такая, что для любых $\alpha, \beta \in \mf A$ выполнено следующее равенство:
	\[f_\alpha|_{U_\alpha\cap U_\beta} \hm= f_\beta|_{U_\alpha\cap U_\beta}\]
	
	Тогда существует единственное непрерывное отображение $f: X\rightarrow Y$ такое, что $f|_{U_\alpha} = f_\alpha$ для любого $\alpha \in \mf A$.
\end{theorem}

\begin{proof}
	Отображение $f$, заданное на $U_\alpha$ как $f_{U_\alpha}$ для каждого $\alpha \in \mf A$, определено корректно и единственно по условию, остается проверить его непрерывность. Действительно, для произвольного открытого множества $V \subset Y$ множество $f^{-1}(V) = \bigcup_{\alpha \in \mf A}f_\alpha^{-1}(V)$ тоже является открытым. Получено требуемое.
\end{proof}

\begin{definition}
	Пусть $X, Y$ "--- топологические пространства. Непрерывное отображение $f : X \to Y$ называется \textit{открытым} (\textit{замкнутым}), если образ любого открытого (замкнутого) множества под действием $f$ открыт (замкнут).
\end{definition}

\begin{note}
	Можно показать, что любое отрытое отображение отрезка $[0, 1]$ в себя имеет конечное число точек минимума или максимума, причем в этих точках оно принимает значения $0$ или $1$. При этом требование замкнутости, как правило, оказывается намного менее жестким.
\end{note}

\begin{proposition}
	Пусть $X, Y$ "--- топологические пространства. Непрерывное отображение $f : X \to Y$ является замкнутым $\lra$ для любого множества $M \subset Y$ и любой окрестности $U$ множества $f^{-1}(M)$ существует окрестность $V$ множества $M$ такая, что $f^{-1}(V) \subset U$.
\end{proposition}

\begin{proof}~
	\begin{itemize}
		\item[$\ra$] Множество $F := X \bs U$ замкнуто, поэтому и множество $f(F)$ замкнуто, причем поскольку $F \cap f^{-1}(M) = \emptyset$, то $f(F) \cap M = \emptyset$. Значит, для окрестности $V := Y \bs f(F)$ выполнено $f^{-1}(V) \subset U$.
		
		\item[$\la$] Пусть для некоторого замкнутого множества $F$ условие не выполнено, тогда можно выбрать точку $y \in [f(F)] \bs f(F)$. Множество $X \bs F$ является окрестностью множества $f^{-1}(y)$. Поэтому существует окрестность $V$ множества $\{y\}$ такая, что $f^{-1}(V) \subset X \bs F$. Но тогда $V \cap f(F) = \emptyset$, что противоречит включению $\{y\} \subset V \cap f(F)$.\qedhere
	\end{itemize}
\end{proof}

\begin{proposition}
	Пусть $X, Y$ "--- топологические пространства, $f: X \to Y$ "--- непрерывное отображение. Если отображение $f$ замкнуто (открыто), то для любого множества $B \subset Y$ отображение $f_{f^{-1}}: f^{-1}(B) \rightarrow B$ тоже замкнуто (открыто).
\end{proposition}

\begin{proof}
	Рассмотрим замкнутое (открытое) в $f^{-1}(B)$ множество $T$ и выберем в $X$ такое замкнутое (открытое) множество $F$, что $F \cap f^{-1}(B) = T$. По условию, множество $f(F)$ замкнуто (открыто) в $Y$, поэтому множество $f(T) = f(F) \cap B$ замкнуто (открыто) в $B$, что и требовалось.
\end{proof}

\begin{note}
	Пусть отображение $f: X \rightarrow Y$ топологических пространств взаимно однозначно, тогда определено обратное отображение $f^{-1}: Y \to X$. При этом из непрерывности отображения $f$ не следует непрерывность отображения $f^{-1}$. Однако если отображение $f$ еще и замкнуто, то отображение $f^{-1}$ тоже замкнуто, причем оба они также являются открытыми в силу взаимной однозначности.
\end{note}

\begin{definition}
	Пусть $X, Y$ "--- топологические пространства. Отображение $f: X \rightarrow Y$ называется \textit{гомеоморфизмом}, или \textit{топологическим отображением}, если $f$ взаимно однозначно, непрерывно, и обратное отображение $f^{-1}$ также непрерывно. Топологические пространства $X, Y$, между которыми существует гомеоморфизм, называются \textit{гомеоморфными}. Обозначение "--- $X \cong Y$.
\end{definition}

\begin{definition}
	Пусть $(X, \tau)$ "--- топологическое пространство, $Y$ "--- некоторое множество, $f : X \to Y$ "--- функция. \textit{Фактортопологией} относительно топологии $\tau$ и отображения $f$ на $Y$ называется семейство $\mf U := \{U \subset Y: f^{-1}(U) \in \tau\}$.
\end{definition}

\begin{note}
	Аксиомы топологии для $\mf U$ выполнены в силу того, что выполены равенства $f^{-1}(\emptyset) = \emptyset$ и $f^{-1}(Y) = X$, а операция взятия прообраза сохраняет теоретико-множественные операции. Кроме того, топология $\mathfrak{U}$ "--- это наибольшая среди всех топологий на $Y$, относительно которых отображение $f$ непрерывно.
\end{note}

\begin{definition}
	Пусть $(X, \tau)$ "--- топологическое пространство, $R \subset X\times X$ "--- отношение эквивалентности на $X$. \textit{Проекцией} множества $X$ на множество классов эквивалентности $X / R$ называется отображение $\pi: X \to X / R$, сопоставляющее каждой точке из $X$ ее класс эквивалентности. \textit{Фактортопологией} на $X / R$ называется фактортопология относительно отображения $\pi$ и топологии $\tau$.
\end{definition}

\begin{note}
	Можно показать, что следующие условия равносильны:
	\begin{enumerate}
		\item $\pi$ является открытым отображением
		
		\item Для любого открытого множества $U \subset X$ множество $\{y \in X: \exists x \in U: (x,y) \in R\}$ открытым также является
		
		\item Для любого замкнутого множества $U \subset X$ объединение всех классов из $X/R$, являющихся подмножествами множества $U$, замкнутым также является
	\end{enumerate}
\end{note}