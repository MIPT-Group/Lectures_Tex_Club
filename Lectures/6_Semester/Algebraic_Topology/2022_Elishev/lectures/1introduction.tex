\section{Метрические и топологические пространства}

\begin{definition}
	\textit{Метрикой} на множестве $X$ называется функция $\rho : X^2 \to \R$, обладающая следующими свойствами:
	\begin{enumerate}
		\item $\forall x, y \in X: \rho(x, y) \ge 0$, причем $\rho(x, y) = 0 \lra x = y$
		\item $\forall x, y \in X: \rho(x, y) = \rho(y, x)$
		\item $\forall x, y, z \in X: \rho(x, z) = \rho(x, y) + \rho(y, z)$
	\end{enumerate}
	
	Пара $(X, \rho)$ называется \textit{метрическим пространством}.
\end{definition}

\begin{example}
	Рассмотрим несколько примеров метрических пространств:
	\begin{enumerate}
		\item Множество $\R$ является метрическим пространством с метрикой $\rho$, заданной для произвольных $x, y \in \R$ как $\rho(x, y) := |x - y|$
		\item Произвольное множество $X$ является метрическим пространством с \textit{дискретной метрикой} $d$, заданной для произвольных $x, y \in \R$ как $d(x, y) := I(x != y)$
		\item Множество $C[0, 1]$ является метрическим пространством с метрикой $\rho$, заданной для произвольных $f, g \in C[0, 1]$ как $\rho(f, g) := \sup\{|f(x) - g(x)| : x \in [0, 1]\}$
	\end{enumerate}
\end{example}

\begin{note}
	Пусть $(X, \rho)$ "--- метрическое пространство, $Y \subset X$. Тогда пара $(Y, \rho|_{Y^2})$ тоже является метрическим пространством.
\end{note}

\begin{note}
	Пусть $(X_1, \rho_1), \dotsc, (X_n, \rho_n)$ "--- метрическое пространство. Тогда метрическим пространством является пара $(X_1 \times \dotsb \times X_n, \rho)$, где метрика $\rho$ задана для произвольных $x = (x_1, \dotsc, x_n), y = (y_1, \dotsc, y_n) \in X_1 \times \dotsb \times X_n$ следующим образом:
	\[\rho(x, y) := \sqrt{\rho_1(x_1, y_1)^2 + \dotsb + \rho_n(x_n, y_n)^2}\]
\end{note}

\begin{definition}
	Пусть $(X, \rho)$ "--- метрическое пространство, $M, N \subset X$. \textit{Расстоянием между множествами} $M, N$ называется следующая величина:
	\[\rho(M, N) := \inf\{\rho(x, y): x \in M, y \in N\}\]
\end{definition}

\begin{definition}
	Пусть $(X, \rho)$ "--- метрическое пространство, $\epsilon > 0$.
	\begin{itemize}
		\item $\epsilon$-окрестностью точки $x \in X$ называется $O(x, \epsilon) := \{y \in X: \rho(y, x) < \epsilon\}$
		
		\item $\epsilon$-окрестностью множества $M \subset X$ называется $O(x, \epsilon) := \{y \in X: \rho(y, M) < \epsilon\}$
	\end{itemize}
\end{definition}

\begin{definition}
	Пусть $(X, \rho)$ "--- метрическое пространство, $M \subset X$. Точка $x \in X$ называется \textit{точкой прикосновения} множества $M$, если для любого $\epsilon > 0$ выполнено $O(x, \epsilon) \cap M \ne \emptyset$. \textit{Замыканием} множества $M$ называется множество $\overline M \equiv [M]$ всех его точек прикосновения. Множество $M$ называется \textit{замкнутым}, если $\overline M = M$.
\end{definition}

\begin{note}
	Точка $x$ является точкой прикосновения множества $M \lra \rho(x, M) = 0$.
\end{note}

\begin{definition}
	Пусть $(X, \rho)$ "--- метрическое пространство, $M \subset X$. Точка $x \in X$ называется \textit{внутренней точкой} множества $M$, если существует $\epsilon > 0$ такое, что $O(x, \epsilon) \subset M$. \textit{Внутренностью} множества $M$ называется множество $M^0$ всех его внутренних точек. Множество $M$ называется \textit{открытым}, если $M^0 = M$.
\end{definition}

\begin{definition}
	Пусть $(X, \rho)$ "--- метрическое пространство. Множество всех открытых подмножеств множества $X$ называется \textit{открытой топологией} на $X$ и обозначается через $\tau$, множество всех замкнутых подмножеств называется \textit{замкнутой топологией} и обозначается через $\kappa$.
\end{definition}

\begin{note}
	Пусть $(X, \rho)$ "--- метрическое пространство. Тогда выполнены следующие свойства:
	\begin{enumerate}
		\item $\forall G_1, G_2 \in \tau: G_1 \cap G_2 \in \tau$
		\item $\forall \{G_\alpha\}_{\alpha \in \mf A} \subset \tau: \bigcup_{\alpha \in \mf A}G_\alpha \in \tau$
		\item $\forall G \subset X \in \tau: X \bs G \in \kappa$
	\end{enumerate}
	
	В силу свойства $(3)$, свойства, двойственные свойствам $(1)$ и $(2)$, верны для замкнутой топологии $\kappa$.
\end{note}

\begin{definition}
	\textit{Топологией} на множестве $X$ называется семейство $\tau$ подмножеств множества $X$, обладающее следующими свойствами:
	\begin{enumerate}
		\item $\emptyset, X \in \tau$
		\item $\forall G_1, G_2 \in \tau: G_1 \cap G_2 \in \tau$
		\item $\forall \{G_\alpha\}_{\alpha \in \mf A} \subset \tau: \bigcup_{\alpha \in \mf A}G_\alpha\in \tau$
	\end{enumerate}
	
	Пара $(X, \tau)$ называется \textit{топологическим пространством}.
\end{definition}

\begin{note}
	Любое метрическое пространство является топологическим пространством с открытой топологией.
\end{note}

\begin{definition}
	Пусть $(X, \tau)$ "--- топологическое пространство, $x \in X$. Любое множество $U \in \tau$, содержащее точку $x$, называется \textit{окрестностью точки $x$}.
\end{definition}

\begin{definition}
	Пусть $(X, \tau)$ "--- топологическое пространство, $M \subset X$. Точка $x \in X$ называется \textit{точкой прикосновения} множества $M$, если для любой окрестности $U$ точки $x$ выполнено $U \cap M \ne \emptyset$. \textit{Замыканием} множества $M$ называется множество $\overline M \equiv [M]$ всех его точек прикосновения.
\end{definition}

\begin{definition}
	Пусть $(X, \tau)$ "--- топологическое пространство, $Y \subset X$. Множество $N \subset Y$ называется \textit{открытым в $Y$}, если существует $M \in \tau$ такое, что $N = M \cap Y$. Семейство подмножеств множества $Y$, открытых в $Y$, называется \textit{индуцированной топологией} на $Y$ и обозначается через $\tau_Y$.
\end{definition}

\begin{note}
	Пара $(Y, \tau_Y)$ из определения выше тоже является топологическим пространством, то есть $Y$ является \textit{подпространством} в $X$.
\end{note}

\begin{theorem}
	Пусть $(X, \tau)$ "--- топологическое пространство. Тогда множество $M \in X$ является замкнутым, то есть таким, что $X \bs M \in \tau \lra [M] = M$.
\end{theorem}

\begin{proof}
	To be done.
\end{proof}

\begin{theorem}
	Пусть $(X, \tau)$ "--- топологическое пространство, $M \subset X$. Тогда замыкание $[M]$ является наименьшим по включению замкнутым множеством, содержащим $M$.
\end{theorem}
	
\begin{proof}
	%С одной стороны, множество $[M]$ замкнуто.
	To be done.
\end{proof}

\begin{note}
	Теорема выше позволяет доказать, что оператор замыкания удовлетворяет \textit{аксиомам Куратовского}:
	\begin{enumerate}
		\item $\forall M, N \subset X: [M \cup N] = [M] \cup [N]$
		\item $\forall M \subset X: M \subset [M]$
		\item $\forall M \subset X: [[M]] = [M]$
		\item $[\emptyset] = \emptyset$
	\end{enumerate}

	Более того, любой оператор, удовлетворяющий аксиомам выше, задает замкнутую топологию, на которой он будет замыканием. Для открытой топологии и оператора внутренности можно построить двойственный набор аксиом.
\end{note}

\begin{note}
	Множество всех топологий на множестве $X$ образует частично упорядоченное множеством относительно отношения $\subset$.
\end{note}

\begin{definition}
	Пусть $(X, \tau)$ "--- топологическое пространство, $M \subset X$. Точка ${x \in X}$ называется предельной точкой множества $M$, если любая его окрестность содержит бесконечно много точек из $M$.
\end{definition}

\begin{note}
	Определение выше в общем случае не эквивалентно привычному определению предельной точки из математического анализа, согласно которому любая окрестность точки $x$ должна содержать хотя бы точку из $M$, отличную от $x$.
\end{note}