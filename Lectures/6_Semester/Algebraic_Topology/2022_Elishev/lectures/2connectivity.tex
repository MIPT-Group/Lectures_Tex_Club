\section{Связные топологические пространства}

\subsection{Связность и несвязность}

\begin{definition}
	Топологическое пространство $X$ называется:
	\begin{itemize}
		\item \textit{Несвязным}, если его можно представить в виде $X = \Phi_1\cup \Phi_2$, где $\Phi_1, \Phi_2 \subset X$ "--- не\-пус\-тые непересекающиеся замкнутые множества
		
		\item \textit{Связным}, если оно не является несвязным
	\end{itemize}
\end{definition}

\begin{note}
	В терминах определения выше, множества $\Phi_1, \Phi_2$ также являются открытыми, поскольку $\Phi_1 = X \bs \Phi_2$ и $\Phi_2 = X \bs \Phi_1$. Такие множества называются \textit{открыто-замкнутыми}. Ясно, что топологическое пространство $X$ несвязно $\lra$ в нем существует нетривиальное, то есть отличное от $\emptyset$ и $X$, открыто-замкнутое множество. Как следствие, пустое и одноточечное пространство всегда являются связными.
\end{note}

\begin{definition}
	Пусть $X$ "--- топологическое пространство. Множество $M \subset X$ называется \textit{связным} (\textit{несвязным}), если оно связно (несвязно) как топологическое пространство с индуцированной топологией.
\end{definition}

\begin{note}
	При рассмотрении связности или несвязности множества $M \subset X$ следует помнить, что речь идет об открытости и замкнутости в $M$, а не в $X$. Множество, замкнутое или открытое в $M$, может не быть таковым в объемлющем пространстве. Например, если $X = \R^2$ и $M = (0, 1) \cup (2, 3)$, то оба этих интервала является открыто-замкнутыми множествами в $M$, но не являются ни открытыми, ни замкнутыми в $X$. Более того, эти открыто-замкнутые множества в $M$ не могут быть получены как пересечения некоторых открыто-замкнутых множеств в $X$ с множеством $M$, поскольку, как мы покажем далее, пространство $\R^n$ связно.
\end{note}

\begin{theorem}
	Отрезок числовой прямой является связным множеством.
\end{theorem}

\begin{proof}
	Предположим, что отрезок $X = [a, b]$ может быть представлен в виде объединения непустых непересекающихся открыто-замкнутых немножеств $\Phi_1, \Phi_2 \subset X$. Пусть без ограничения общности $a \in \Phi_1$. Поскольку множество $\Phi_1$ открыто, то существует $\epsilon > 0$ такое, что $[a, a+\epsilon)\subset \Phi_1$.
	
	Назовем точку $x \in X$ \textit{отмеченной}, если $[a, x) \subset \Phi_1$, тогда все точки из $[a, a + \epsilon)$ являются отмеченными. Обозначим через $c$ точную верхнюю грань множества отмеченных точек, тогда $c > a$. Покажем, что $c$ "--- тоже отмеченная точка. Пусть $x \in [a, c)$, тогда существует точка $x' > x$, являющаяся отмеченной, откуда $[a, x') \subset \Phi_1$ и $x \in \Phi_1$. Значит, $[a, c) \subset \Phi_1$, то есть точка $c$ "--- действительно отмеченная.
	
	В силу замкнутости множества $\Phi_1$, имеем $c \in \Phi_1$. Если $c \ne b$, то в силу открытости $\Phi_1$ существует $\epsilon'>0$ такое, что множество $[c, c+\epsilon') \subset \Phi_1$, что противоречит выбору точки $c$. Значит, $c = b$, откуда $\Phi_1 = X$ и $\Phi_2 = \emptyset$ --- противоречие.
\end{proof}

\begin{proposition}
	Пусть $X$ "--- топологическое пространство, $\Phi_1, \Phi_2 \subset X$ "--- непересекающиеся замкнутые (открытые) множества, $M \subset X$ "--- непустое связное множество $M$ такое, что $M \subset \Phi_1 \cup \Phi_2$. Тогда $M$ содержится в одном из множеств $\Phi_1, \Phi_2$
\end{proposition}

\begin{proof}
	По условию, $M = (M \cap \Phi_1) \cup (M\cap \Phi_2)$, и оба множества в правой части замкнуты (открыты) в $M$, тогда, в силу связности множества $M$, одно из них пусто, а второе --- совпадает с $M$.
\end{proof}

\begin{theorem} \label{theoremconnectedsum}
	Пусть $X$ "--- топологическое пространство, $\{M_\alpha\}_{\alpha \in \mf A}$ "--- система связных подмножеств в $X$ такая, что $\bigcap_{\alpha \in \mf A}M_\alpha \ne \emptyset$. Тогда множество $M := \bigcup_{\alpha \in \mf A}M_{\alpha}$ связно.
\end{theorem}

\begin{proof}
	Пусть $M$ можно представить в виде объединения непустых непересекающихся открыто-замкнутых множеств $\Phi_1, \Phi_2 \subset M$. Тогда, в силу утверждения выше, каждое множество $M_{\alpha}$, $\alpha \in \mf A$, содержится либо в $\Phi_1$, либо в $\Phi_2$. Но тогда выполнено включение $\Phi_1 \cap \Phi_2 \supset \bigcap_{\alpha \in \mf A}M_\alpha \ne \emptyset$ --- противоречие.
\end{proof}

\begin{theorem} \label{theorempathconnected}
	Пусть $X$ "--- топологическое пространство такое, что для любых двух точек $x, y \in X$ можно найти содержащее эти две точки связное множество $C_{xy} \subset X$. Тогда пространство $X$ связно.
\end{theorem}

\begin{proof}
	Пусть $X$ можно представить в виде объединения непустых непересекающихся открыто-замкнутых множеств $\Phi_1, \Phi_2 \subset X$. Выберем $x \in \Phi_1$ и $y \in \Phi_2$, тогда связное множество $C_{xy}$, содержащее $x, y$, должно целиком содержаться либо в $\Phi_1$, либо в $\Phi_2$ --- противоречие.
\end{proof}

\begin{corollary}
	Всякое выпуклое множество связно. В частности, при любом $n \in \N$ евклидово пространство $\mathbb R^n$ связно.
\end{corollary}

\begin{proposition}
	Пусть $a, b$ "--- точки связного множества $C \subset \mathbb{R}$. Тогда $(a, b) \subset C$.
\end{proposition}

\begin{proof}
	Предположим, что точка $c \in (a, b)$ не принадлежит $C$. Обозначим через $\Phi_1$ множество всех точек из $C$, лежащих слева от $c$, через $\Phi_2$ --- множество всех точек из $C$, лежащих справа от $c$. Множества $\Phi_1, \Phi_2$ непусты, открыты в $C$ и не пересекаются, причем $\Phi_1 \cup \Phi_2 = C$ --- противоречие со связностью множества $C$.
\end{proof}

\begin{theorem}
	Связными множествами в $\R$ являются все промежутки, и только они.
\end{theorem}

\begin{proof}
	Все промежутки, вырожденные, конечные и бесконечные, выпуклы и потому связны, поэтому остается доказать, что любое связное множество $C \subset \R$ является промежутком. Положим $a := \inf C$, $b := \sup C$, тогда для любой точки $x \in (a, b)$ можно выбрать точки $a' \in [a, x) \cap C$, $b' \in (x, b] \cap C$, тогда $(a', b') \subset C$, откуда $x \in C$. Значит, $(a, b) \subset C$, из чего следует требуемое.
\end{proof}

\begin{theorem}\label{theoremmain}
	Пусть $X, Y$ "--- топологические пространства, $f : X \to Y$ "--- непрерывное отображение, и пространство $X$ связно. Тогда пространство $f(X)$ тоже связно.
\end{theorem}

\begin{proof}
	Будем без ограничения общности считать, что $f(X) = Y$. Пусть $Y$ можно представить в виде объединения непустых непересекающихся открыто-замкнутых множеств $\Phi_1, \Phi_2 \subset Y$. Тогда множества $f^{-1}(\Phi_1), f^{-1}(\Phi_2)$ непусты, открыто-замкнуты в силу непрерывности отображения $f$ и не пересекаются --- противоречие со связностью пространства $X$.
\end{proof}

\begin{corollary}
	Пусть $X$ "--- связное топологическое пространство, $f : X \to \R$ "--- непрерывная функция. Тогда если $f$ принимает значения $a, b \in \R$, то $f$ также принимает любое значение из отрезка $[a, b]$.
\end{corollary}

\begin{proof}
	Образ $f(X)$ является связным множеством в $\R$, то есть промежутком, что и означает требуемое.
\end{proof}

\begin{definition}
	Пусть $X$ "--- непустое множество. \textit{Коконечной замкнутой топологией} на множестве $X$ называется семейство $\kappa_{cofin} := \{\emptyset, X\} \cup \{M \subset X : M \text{ конечно}\}$.
\end{definition}

\begin{definition}
	Непустое топологическое пространство $X$ называется \textit{неприводимым}, если его нельзя представить в виде $X = \Phi_1 \cup \Phi_2$, где $\Phi_1, \Phi_2 \subset X$ "--- собственные замкнутые подмножества.
\end{definition}

\begin{theorem}
	Пусть множество $X$ бесконечно. Тогда коконечная топология $\kappa_{cofin}$ на $X$ неприводима.
\end{theorem}

\begin{proof}
	Пусть $X$ можно представить в виде $X = \Phi_1 \cup \Phi_2$, где $\Phi_1, \Phi_2 \in \kappa_{cofin}$. Тогда множество $X$ конечно --- противоречие.
\end{proof}

\begin{definition}
	Пусть $K$ "--- алгебраически замкнутое поле. \textit{Топологией Зарисского} на пространстве $K^n$ называется семейство множеств решений систем полиномиальных уравнений с $n$ неизвестными над $K$.
\end{definition}

\begin{theorem} \label{theoremclosureconnected}
	Пусть $X$ "--- топологическое пространство, $C \subset X$ -- связное множество. Тогда всякое множество $C_0$ такое, что $C \subset C_0 \subset \overline{C}$, тоже связно.
\end{theorem}

\begin{proof}
	Пусть $C_0$ можно представить в виде объединения непустых непересекающихся открыто-замкнутых множеств $\Phi_1, \Phi_2 \subset C_0$. Тогда связное множество $C$, содержащееся в $\Phi_1 \cup \Phi_2$, должно по предыдущей теореме содержаться в одном из этих двух множеств. Пусть без ограничения общности $C \subset \Phi_1$, тогда, поскольку $\Phi_1$ замкнуто в $C_0$, то всякая точка множества $C_0 \subset \overline C$ содержится в $\Phi_1$, откуда $\Phi_2 = \emptyset$ --- противоречие.
\end{proof}

\begin{definition}
	Пусть $X$ "--- топологическое пространство. \textit{Путем}, соединяющим точки $x_0, x_1 \in X$, называется непрерывное отображение $f: [0, 1] \rightarrow X$ такое, что выполнены равенства $f(0) = x_0$ и $f(1) = x_1$.
\end{definition}

\begin{definition}
	Топологическое пространство $X$ называется \textit{линейно связным}, если любые две его точки можно соединить путем.
\end{definition}

\begin{proposition}
	Всякое линейно связное топологическое пространство связно.
\end{proposition}

\begin{proof}
	Следует непосредственно из теорем \ref{theoremmain} и \ref{theorempathconnected}.
\end{proof}

\subsection{Компоненты связности}

\begin{definition}
	Пусть $X$ "--- топологическое пространство. \textit{Компонентой} точки $a \in X$ называется объединение $C_a$ всех связных подмножеств в $X$, содержащих точку $a$.
\end{definition}

\begin{note}
	Множество $C_a$ содержит точку $a$, поскольку множество $\{a\}$ связно, тогда по теореме \ref{theoremconnectedsum} множество $C_a$ связно. Значит, $C_a$ "--- это наибольшее по включению связное множество в $X$, содержащее точку $a$. Кроме того, по теореме \ref{theoremclosureconnected}, множество $C_a$ замкнуто.
\end{note}

\begin{note}
	Если компоненты $C_a, C_b \subset X$ точек $a, b \in X$ пересекаются, то по теореме \ref{theoremconnectedsum} они совпадают.
\end{note}

\begin{definition}
	Пусть $X$ "--- топологическое пространство. Максимальные по включению связные подмножества в $X$ называются \textit{компонентами связности} пространства $X$.
\end{definition}

\begin{note}
	Если $M\subset X$, то под компонентами связности множества $M$ понимают его компоненты связности как подпространства в $X$. Они замкнуты в $M$, но вообще говоря, не обязаны быть замкнутыми в $X$.
\end{note}

\begin{definition}
	Топологическое пространство $X$ называется \textit{вполне несвязным}, если все его компоненты связности являются одноточечными множествами.
\end{definition}

\begin{example}
	Пространство $\Q$ является вполне несвязным, поскольку в нем не являются связными множества, состоящие хотя бы из двух точек.
\end{example}

\subsection{Области}

\begin{note}
	Пусть $X$ "--- топологическое пространство. \textit{Цепью множеств} в $X$ называется последовательность $\{M_i\}_{i = 1}^s$ подмножеств в $X$ такая, что для любого $i \in \{1, \dotsc, s - 1\}$ выполнено $M_i \cap M_{i+1} \ne \emptyset$.
\end{note}

\begin{proposition}
	Пусть $X$ "--- топологическое пространство, цепь $\{M_1,\ldots, M_s\}$ состоит из связных множеств. Тогда множество $\bigcup_{i=1}^s M_i$ связно.
\end{proposition}

\begin{proof}
	Проведем индукцию по числу элементов цепи. База, $s = 1$, тривиальна, докажем переход. По предположению, множества $\bigcup_{i=1}^{s-1} M_i$ и $M_s$ связны, и они пересекаются, тогда по теореме \ref{theoremconnectedsum} множество $\bigcup_{i=1}^s M_i$ тоже связно.
\end{proof}

\begin{note}
	Из утверждения выше следует, что всякая ломаная линия в $\mathbb{R}^n$ является связным множеством. Тогда, по теореме \ref{theorempathconnected}, всякое множество $M\subset \mathbb{R}^n$, любые две точки которого могут быть соединены ломаной, связно.
\end{note}

\begin{theorem}
	Открытое множество $\Gamma \subset \mathbb{R}^n$ связно $\lra$ любые две точки из $\Gamma$ можно соединить ломаной, лежащей в $\Gamma$.
\end{theorem}

\begin{proof}
	Нетривиально только доказательство $(\ra)$. Для каждой точки $a \in \Gamma$ обозначим через $\Gamma_a$ множество точек из $\Gamma$, которые можно соединить с $a$ лежащей в $\Gamma$ ломаной, и покажем, что $\Gamma_a$ открыто-замкнуто в $\Gamma$.
	\begin{itemize}
		\item Докажем, что множество $\Gamma_a$ открыто в $\Gamma$. Пусть $x \in \Gamma_a$, тогда, в силу открытости множества $\Gamma$, существует $\epsilon>0$ такое, что $U(x,\epsilon) \subset \Gamma$. Зафиксируем произвольную точку $x' \in U(x, \epsilon)$, тогда отрезок $\overline{xx'}$ лежит в $U(x, \epsilon)$. Рассмотрим ломаную $\overline{ax}$, соединяющую точку $a$ с $x$, тогда ломаная $\overline{axx'} \subset \Gamma$ соединяет $a$ с $x'$. Значит, $U(x, \epsilon)\subset \Gamma_a$.
		
		\item Докажем, что множество $\Gamma_a$ замкнуто в $\Gamma$. Пусть $x \in \Gamma$ "--- точка прикосновения множества $\Gamma_a$, тогда, в силу открытости множества $\Gamma$, существует $\epsilon>0$ такое, что $U(x,\epsilon) \subset \Gamma$. Выберем точку $x' \in U(x,\epsilon) \cap \Gamma_a$, тогда отрезок $\overline{xx'}$ лежит в $U(x, \epsilon)$. Рассмотрим ломаную $\overline{ax'}$, соединяющую точку $a$ с $x'$, тогда ломаная $\overline{ax'x} \subset \Gamma$ соединяет $a$ с $x$. Значит, $x \in \Gamma_a$.
	\end{itemize}
	
	Таким образом, если существуют такая точки $a, b \in \Gamma$, которые нельзя соединить ломаной, лежащей в $\Gamma$, то $\Gamma$ имеет нетривиальное открыто-замкнутое подмножество --- противоречие.
\end{proof}

\begin{definition}
	Пусть $X$ "--- топологическое пространство. Множество $D \subset X$ называется \textit{областью}, если оно открыто и связно.
\end{definition}

\begin{theorem}\label{theoremcomponents}
	Пусть $\Gamma \subset \R^n$ "--- открытое множество. Тогда компоненты связности множества $\Gamma$ являются областями.
\end{theorem}

\begin{proof}
	Рассмотрим точку $x\in C$ некоторой компоненты связности множества $\Gamma \subset \mathbb{R}^n$. Выберем $\epsilon > 0$ такое, что $U(x,\epsilon) \subset \Gamma$. Окрестность, будучи выпуклым множеством, связна, тогда по теореме $\ref{theoremconnectedsum}$ множество $C \cup U(x,\epsilon)$ связно. Но $C$ "--- компонента точки $x$, поэтому $U(x,\epsilon) \subset C$, то есть множество $C$ открыто.
\end{proof}

\begin{corollary}
	Пусть $\Gamma \subset \R^n$ "--- открытое множество. Тогда $\Gamma$ является объединением не более чем счетного числа попарно непересекающихся областей.
\end{corollary}

\begin{proof}
	Множество $\Gamma$ представимо в виде объединения попарно непересекающихся компонент связности, являющихся областями, поэтому остается доказать, что их не более чем счетное число. Поскольку множество $\Q^n$ всюду плотно в $\R^n$, то каждой компоненте $C \subset \Gamma$ можно взаимно однозначно сопоставить некоторую точку $q \in C$ с рациональными координатами. Но множество $\Q^n$ счетно, из чего следует требуемое.
\end{proof}

\begin{note}
	Доказательство теоремы о компонентах открытых подмножеств в $\R^n$ опирается лишь на связность сферических окрестностей в $\R^n$ и потому применимо к более широкому классу пространств.
\end{note}

\begin{definition}
	Топологическое пространство $X$ называется \textit{локально связным}, если для любой точки $x \in X$ и любой ее окрестности $U(x) \subset X$ найдется связная окрестность $U$ такая, что $U \subset U(x)$.
\end{definition}

\begin{theorem}
	Пусть $X$ "--- локально связное пространство, $\Gamma \subset X$ "--- открытое множество. Тогда компоненты связности множества $\Gamma$ являются областями.
\end{theorem}

\begin{proof}
	Аналогично теореме \ref{theoremcomponents}.
\end{proof}