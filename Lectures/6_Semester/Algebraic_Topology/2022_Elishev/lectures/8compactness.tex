\section{Другие сведения о бикомпактных пространствах}

\subsection{Замкнутость бикомпактов}

\begin{definition}
    Хаусдорфово пространство $X$ называется \textit{$H$-замкнутым}, если оно замкнуто во всяком объемлющем хаусдорфовом пространстве $\overline{X}$.
\end{definition}

\begin{definition}
    Пусть $X$ "--- хаусдорфово пространство. Его \textit{одноточечным расширением} называется пара $(\overline{X}, f)$, состоящая из хаусдорфова пространства $\overline{X}$ и отображения $f: X \rightarrow \overline{X}$, являющегося гомеоморфизмом между $X$ и $f(X)$, для которого $\overline{X} \backslash f(X)$ состоит из одной точки $\xi \in \overline X$. Одноточечное расширение называется \textit{нетривиальным}, если точка $\xi$ является предельной точкой множества $f(X)$.
\end{definition}

\begin{note}
    В определении выше можно отождествить $X$ и $f(X)$.
\end{note}

\begin{note}
    Пусть $f : X \to Y$ "--- отображение топологических пространств. Тогда $f$ является гомеоморфизмом между $X$ и $f(X)$ $\lra$ $f$ является открытым вложением.
\end{note}

\begin{proposition}
    Пусть $X$ "--- хаусдорфово пространство. Тогда пространство $X$ является $H$-замкну\-тым $\lra$ $X$ не имеет нетривиальных одноточечных расширений.
\end{proposition}

\begin{proof}
    Нетривиально только доказательство $(\la)$. Пусть пространство $X$ не является замкнутым в некотором объемлющем пространстве $\overline X$. Тогда оно имеет предельную точку $\xi \in \overline X \bs X$, и, следовательно, не является замкнутым в пространстве $X \cup \{\xi\}$. Но тогда пара $(\id, X \cup \{\xi\})$ является нетривиальным одноточечным расширением пространства $X$.
\end{proof}

\begin{theorem}\label{hcriterion}
    Пусть $X$ "--- хаусдорфово пространство. Тогда пространство $X$ является $H$-замкнутым $\lra$ в любом открытом покрытии $\Sigma = \lbrace G_\alpha\rbrace_{\alpha \in \mf A}$ пространства $X$ имеется конечное число элементов $G_{\alpha_1}, \ldots, G_{\alpha_n}$, объединение которых всюду плотно в $X$.
\end{theorem}

\begin{proof}~
    \begin{itemize}
        \item[$\la$] Пусть $\overline{X} = X \cup \{\xi\}$ "--- одноточечное расширение пространства $X$. Покажем, что оно тривиально, то есть пространство $X$ замкнуто в $\overline{X}$. В силу хаусдорфовости, для каждой точки $x \in X$ существует окрестность $U(x)$, замыкание которой в $\overline{X}$ не содержит $\xi$ и, следовательно, совпадает с замыканием в $X$. Совокупность таких окрестностей образует открытое покрытие $X$, поэтому существуют окрестности $U(x_1), \ldots, U(x_s)$, объединение которых всюду плотно в $X$. Значит, выполнено следующее равенство:
        \[X = \overline{U(x_1)}\cup\ldots\cup\overline{U(x_s)}\]
        
        Таким образом, $X$ совпадает с замкнутым в $\overline{X}$ множеством, и получено требуемое.

        \item[$\ra$] Предположим, что существует бесконечное открытое покрытие $\Sigma = \lbrace G_{\alpha \in \mf A }\rbrace$ такое, что объединение элементов любого его конечного подмножества не является всюду плотным в $X$. Добавим к $X$ некоторую точку $\xi$ и объявим базой топологии на $X \cup \{\xi\}$ совокупность всех открытых в $X$ множеств и всех множества вида $(X \cup \lbrace \xi\rbrace) \backslash \bigcup_{i=1}^s\overline{G_{\alpha_i}}$ для произвольных $\alpha_1, \dotsc, \alpha_s \in \mf A$. Полученное пространство $\overline{X} := X \cup \{\xi\}$ хаусдорфово, поскольку для любой точки $x \in X$ и точки $\xi$ существует элемент $G_\alpha \in \Sigma$ такой, что $x \in G_\alpha$ и $\xi \in \overline{X}\backslash \overline{G_\alpha}$. Наконец, точка $\xi$ является предельной точкой множества $X$ по построению.\qedhere
    \end{itemize}
\end{proof}

\begin{theorem}\label{theorem1}
    Пусть $X$ "--- бикомпакт. Тогда пространство $X$ является $H$-замкнутым.
\end{theorem}

\begin{proof}
    Поскольку $X$ "--- бикомпакт, то $X$ удовлетворяет условию теоремы \ref{hcriterion}, что и дает требуемое.
\end{proof}

\begin{proposition}
    Пусть $X$ "--- регулярное $H$-замкнутое пространство. Тогда $X$ бикомпактно.
\end{proposition}

\begin{proof}
    Хаусдорфовость пространства $X$ следует из регулярности, докажем его бикомпактность. Пусть $\Sigma = \lbrace G_\alpha\rbrace_{\alpha \in \mf A}$ -- открытое покрытие пространства $X$. Для любой точки $x$ выберем $G_\alpha$ такое, что $x \in G$, а также открытое множество $\Gamma(x)$ такое, что $x \in \Gamma(x) \subset \overline{\Gamma(x)} \subset G$. Система $\{\Gamma(x): x \in X\}$ образует открытое покрытие, тогда, в силу $H$-замкнутости и теоремы \ref{hcriterion}, существует конечный набор $\Gamma(x_1), \ldots, \Gamma(x_s)$, для которого выполнено следующее равенство:
    \[X = \overline{\Gamma(x_1)} \cup \dotsb \cup \overline{\Gamma(x_s)}\]
    
    Значит, система множеств из покрытия $\Sigma$, соответствующих точкам $x_1, \dotsc, x_s$, образует конечное покрытие пространства $X$.
\end{proof}

\subsection{Компактификация Стоуна--Чеха}

\begin{definition}
    Пусть $X$ "--- вполне регулярное пространство. Его \textit{бикомпактным расширением} называется бикомпакт $bX$ такой, что $X \subset bX$ и $X$ всюду плотно в $bX$.
\end{definition}

\begin{note}
    Пусть $\tau$ "--- мощность некоторого расчленяющего семейства пространства $X$. Тогда, в силу второй теоремы Тихонова, отображение $\phi: X \to I^\tau$ осуществляет гомеоморфизм между $X$ и $Y := f(X)$. Пространство $\overline{Y}$ является бикомпактом как замкнутое подмножество бикомпакта $I^\tau$. Отождествляя $Y$ с $X$, получаем, что бикомпакт $\overline{Y}$ является бикомпактным расширением $X$.
\end{note}

\begin{note}
    Среди расчленяющих множеств функций, очевидно, существует максимальное, и даже наибольшее, множество --- множество $\Xi_{\max}$ всех непрерывных функций вида $f: X \to [0, 1]$.
\end{note}

\begin{definition}
    Ппусть $X$ "--- вполне регулярное пространство. Бикомпактное расширение $\beta X$ пространства $X$, соответствующее множеству $\Xi_{\max}$, называется его \textit{максимальным бикомпактным расширением}, или его \textit{компактификацией Стоуна--Чеха}.
\end{definition}

\begin{definition}
    Пусть $bX, b'X$ "--- бикомпактные расширения вполне регулярного пространства $X$. Непрерывное отображение $f: bX\rightarrow b'X$ называется \textit{естественным}, если $f|_{X} = \id_X$.
\end{definition}

\begin{note}
    Всякое естественное отображение сюръективно. Действительно, поскольку $X \subset f(bX)$, множество $X$ всюду плотно в $b'X$, а непрерывный образ бикомпакта $bX$ бикомпактен, и, как следствие, замкнут в $b'X$, то $f(bX) = b'X$.
\end{note}

\begin{proposition}\label{densesetprop}
    Пусть $X$ "--- топологическое пространство, $Y$ "--- хаусдорфово прост\-ранство, и пусть $f_1, f_2 : X \to Y$ "--- непрерывные отображения, совпадающие на всюду плотном множестве $X' \subset X$. Тогда $f_1 = f_2$.
\end{proposition}

\begin{proof}
    Предположим, что существует точка $x_0 \in X$ такая, что $f_1(x_0) \ne f_2(x_0)$. Положим $y_1 := f_1(x_0)$, $y_2 := f_2(x_0)$, и выберем непересекающиеся окрестности $U(y_1), U(y_2)$. В силу непрерывности функций $f_1, f_2$, существует окрестность $V$ точки $x_0$, такая, что $f_1(V)\subset U(y_1)$ и $f_2(V)\subset U(y_2)$, тогда выполнены следующие равенства:
    \[f_1(V) \cap f_2(V) = U(y_2) \cap U(y_1) = \emptyset\]
    
    Множество $X'$ всюду плотно в $X$, поэтому множество $V' := V\cap X'$ непусто, и по условию $f_1|_{V'} = f_2|_{V'}$, но $f_1(V') \cap f_2(V') = \emptyset$ --- противоречие.
\end{proof}

\begin{theorem}[Стоуна--Чеха]\label{stonecech}
    Пусть $X$ "--- вполне регулярное пространство, $bX$ "--- его бикомпактное расширение. Тогда следующие условия эквивалентны:
    \begin{enumerate}
        \item Существует естественный гомеоморфизм между $bX$ и $\beta X$
        
        \item Любая непрерывная функция $f: X \rightarrow [0, 1]$ может быть продолжена до непрерывной функции $\overline{f}: bX \rightarrow [0, 1]$
        
        \item Для любого бикомпакта $B$ непрерывное отображение $\phi: X \to B$ может быть про\-должено до непрерывного отображения $\overline \phi: bX \to B$
        
        \item Расширение $bX$ обладает естественным отображением на любое бикомпактное расширение $b'X$ пространства $X$
    \end{enumerate}
\end{theorem}

\begin{proof}~
    \begin{itemize}
        \item\imp{1}{2}Пусть $f: X\rightarrow [0, 1]$ "--- непрерывная функция, тогда $f \in \Xi_{\max} = \{f_{\alpha}\}_{\alpha \in \mf A}$. Пусть мощность множества $\mf A$ равна $\tau$, и пусть $\varphi: X \rightarrow I^{\tau} = \prod_{\alpha \in \mathfrak{A}}I_\alpha$ "--- отображение из второй теоремы Тихонова, имеющее вид $x \mapsto (f_\alpha(x))_{\alpha \in \mf A}$. Для каждого $\alpha \in \mf A$ рассмотрим проекцию $\pi_\alpha : I^\tau \to I_\alpha$, тогда $f_\alpha = \pi_\alpha \circ \varphi$.
        
        Пусть $f = f_{\alpha_0}$ для некоторого $\alpha_0 \in \mf A$. Тогда отображение $\pi_{\alpha_0}: \overline(\phi(X)) \to I_{\alpha_0}$ является искомым продолжением функции $f$, поскольку $\overline{\phi(X)} \cong bX$ и $I_{\alpha_0} = [0, 1]$.

        \item\imp{2}{3}Рассмотрим сначала случай, когда вес бикомпакта конечен. Тогда он имеет конечную топологию, и, поскольку одноточечные множества в $T_1$-пространстве являются замкнутыми, бикомпакт $B$ конечен. Значит, его можно рассматривать как подмножество отрезка $I = [0, 1]$. По условию, отображение $\varphi: X \rightarrow B \subset I$ продолжается до непрерывного отображения $\overline{\varphi}: bX \rightarrow I$. Наконец, выполнено следующее:
        \[\overline{\varphi}(bX) = \overline{\varphi}([X]) \subset [\overline{\varphi} (X)] = [\varphi(X)] \subset [B] = B\]

        Таким образом, получено требуемое. Пусть теперь вес $\tau$ бикомпакта $B$ не менее чем счетен. По второй теореме Тихонова, $B$ можно отождествить с подмножеством тихоновского кирпича $I^{\tau} = \prod_{\alpha \in \mathfrak{A}}I_\alpha$. Для любого $\alpha \in \mf A$ функция $f_{\alpha} = \pi_\alpha \circ \varphi: X\rightarrow I_\alpha$ непрерывна и потому имеет непрерывное продолжение $\overline{f_\alpha}: bX \rightarrow I_\alpha$. Диагональное произведение $\overline{\varphi}$ таких продолжений тоже непрерывно, и для любого $x \in X$ выполнены следующие равенства:
        \[\overline{\varphi}(x) = (\overline{f_\alpha}(x))_{\alpha \in \mf A} = (f_\alpha(x))_{\alpha \in \mf A} = (\pi_a\circ\varphi(x))_{\alpha \in \mf A} = \varphi(x)\]

        Значит, $\overline{\varphi}$ действительно является непрерывным продолжением отображения $\varphi$. Наконец, поскольку $\overline{\varphi}$ непрерывно, выполнено следующее:
        \[\overline{\varphi}(bX) = \overline{\varphi}([X]) \subset [\overline{\varphi}(X)] = [\varphi (X)] \subset B\]

        \item\imp{3}{4}Достаточно применить условие к отображению $\phi = \id_X: X \to b'X$.
        
        \item\imp{4}{1}Пусть $\beta X$ "--- расширение Чеха--Стоуна, $h: bX\rightarrow \beta X$ "--- естественное отображение. По уже доказанному, для расширения Чеха--Стоуна существует естественное отображение $h': \beta X\rightarrow bX$. Отображение $h'\circ h$ совпадает с $\id_X$ на $X$, тогда, по утверждению \ref{densesetprop}, $h'\circ h = \id_{\beta X}$. Аналогично, $h\circ h' = \id_{bX}$. Следовательно, непрерывное отображение $h$ биективно, и отображение $h^{-1} = h'$ непрерывно. Значит, $h$ "--- искомый естественный гомеоморфизм.\qedhere
    \end{itemize}
\end{proof}

\begin{theorem}
    Пусть $X, Y$ "--- вполне регулярные пространства, $f : X \to Y$ "--- непрерывное отображение. Тогда $f$ быть продолжено до непрерывного отображения $\overline{f}: \beta X \rightarrow \beta Y$.
\end{theorem}

\begin{proof}
    Отображение $f$ можно считать отображением из $X$ в $\beta Y$, тогда, в силу предыдущей теоремы, $f$ может быть продолжено до отображения $\overline{f}: \beta X \rightarrow \beta Y$.
\end{proof}