\section{Свойства типа компактности}

\subsection{Системы подмножеств и покрытия}

\begin{definition}
    Пусть $\sigma$ "--- система подмножеств множества $X$. \textit{Телом} системы $\sigma$ называется множество $\widetilde{\sigma} := \bigcup_{M \in \sigma} M$.
\end{definition}

\begin{definition}
    Пусть $\sigma$ "--- система подмножеств множества $X$, $E \subset X$. Обозначим через $\sigma_E$ подсистему системы $\sigma$, состоящую из множеств, пересекающихся с $E$. \textit{Звездой} $\St_{\sigma}E$ множества $E$ называется тело $\widetilde{\sigma_E}$ подсистемы $\sigma_E$.
\end{definition}

\begin{note}
    Если множество $E$ состоит из одной точки $x \in X$, то естественно обозначать $\St_{\sigma}x := \St_{\sigma}E$. Введем также обозначение $\sigma^* := \{\St_{\sigma}x\}_{x \in X}$.
\end{note}

\pagebreak

\begin{definition}
    Пусть $\sigma$ "--- система подмножеств множества $X$. \textit{Кратностью} системы в точке $x\in X$ называется мощность множества элементов системы $\sigma$, содержащих точку $x$. Обозначение "--- $\mul_x \sigma$.
\end{definition}

\begin{definition}
    Система $\sigma$ подмножеств множества $X$ называется:
    \begin{itemize}
        \item \textit{Конечнократной}, или \textit{точечно конечной}, если для любого $x\in X$ кратность $\mul_x\sigma$ конечна
        
        \item \textit{Звездно конечной}, если каждый элемент системы $\sigma$ пересекается лишь с конечным числом элементов системы $\sigma$
        
        \item \textit{Звездно счетной}, если каждый элемент системы $\sigma$ пересекается с не более чем счетным числом элементов системы $\sigma$
    \end{itemize}
\end{definition}

\begin{definition}
    Пусть $\sigma$ "--- система подмножеств множества $X$. Система $\sigma$ называется \textit{сцепленной}, если для любых двух элементов $M, N \in \sigma$ существует конечная последовательность $M_1, \ldots, M_s \in \sigma$ такая, что $M_1 = M$, $M_s = N$, и для каждого $i \in \{1, \dotsc, n - 1\}$ выполнено $M_i \cap M_{i + 1} = \emptyset$.
\end{definition}

\begin{note}
    Пусть $\Sigma$ "--- система подмножеств множества $X$. Объединение двух сцепленных систем $\sigma_1, \sigma_2 \subset \Sigma$ таких, что $\widetilde{\sigma_1}\cap\widetilde{\sigma_2}\neq \emptyset$, также является сцепленной системой. Это же верно для объединения любого количества сцепленных систем, тела которых имеют непустое пересечение. Значит, по лемме Цорна, любая сцепленная система $\sigma_0 \subset \Sigma$ содержится в некоторой максимальной сцепленной системе.
\end{note}

\begin{definition}
    Пусть $\Sigma$ "--- система подмножеств множества $X$. Максимальная сцепленная система $\sigma \subset \Sigma$ называется \textit{компонентой сцепленности} системы $\Sigma$.
\end{definition}

\begin{note}
    Пусть $X$ "--- топологическое пространство, $\Sigma$ "--- система, состоящая из открытых подмножеств множества $X$. Тогда, поскольку тела компонент сцепленности системы $\Sigma$ дизъюнктны, то тело всякой компоненты сцепленности является открыто-замкнутым множеством.
\end{note}

\begin{proposition}
    Пусть $\Sigma$ "--- звездно счетная система подмножеств множества $X$. Тогда каждая сцепленная система $\sigma \subset \Sigma$ состоит из не более чем счетного числа элементов.
\end{proposition}

\begin{proof}
    Зафиксируем $M_0 \in \sigma$. Для каждого $k \in \mathbb N$ обозначим через $\sigma_k$ подсистему системы $\sigma$, состоящую из всех множеств $M\in\sigma$, которые могут быть связаны с $M_0$ цепочками не больше $k$. Тогда:
    \begin{itemize}
        \item Система $\sigma_1$ состоит из одного множества $\{M_0\}$.
        
        \item Для каждого $k \in \N$ система $\sigma_{k+1}$ состоит из всех множеств из $\sigma$, пересекающихся хотя бы с одним элементом системы $\sigma_k$, то есть пересекающихся с $\widetilde{\sigma_k}$. Тогда, в частности, $\sigma_k \subset \sigma_{k+1}$.
        
        \item В силу сцепленности системы $\sigma$, выполнено $\sigma = \cup_{k=1}^{\infty}\sigma_k$.
    \end{itemize}

    Значит, остается показать, что для каждого $k \in \N$ система $\sigma_k$ не более чем счетна. Проведем индукцию по $k$. База, $k = 1$, тривиальна. Пусть теперь для некоторого $k \in \N$ система $\sigma_k$ не более чем счетна, тогда, поскольку система $\sigma \subset \Sigma$ является звездно счетной, система $\sigma_{k+1}$ тоже не более чем счетна по построению.
\end{proof}

\begin{definition}
    Пусть $\sigma, \tau$ "--- покрытия множества $X$.
    \begin{itemize}
        \item Покрытие $\sigma$ \textit{вписано} в покрытие $\tau$, если каждый элемент покрытия $\sigma$ содержится в некотором элементе покрытия $\tau$. Обозначение "--- $\sigma \preceq \tau$.

        \item Покрытие $\tau$ \textit{следует} за покрытием $\sigma$, если $\sigma$ вписано в $\tau$ и при этом $\tau \ne \sigma$. Обозначение "--- $\sigma \prec \tau$.

        \item Покрытие $\sigma$ \textit{звездно вписано} в покрытие $\tau$, если система $\sigma^*$ вписана в $\tau$.
    \end{itemize}
\end{definition}

\begin{note}
    Любое подпокрытие покрытия $\sigma$, в частности, вписано в $\sigma$.
\end{note}

\begin{proposition}[Лефшеца, об ужатии точечно конечных покрытий]\label{propositionLef}
    Пусть $X$ "--- нормальное пространство, $u = \{ U_{\alpha}\}_{\alpha\in\mathfrak{A}}$ "--- его точечно конечное открытое покрытие. Тогда существует такое открытое покрытие $v = \{V_{\alpha}\}_{\alpha\in\mathfrak{A}}$ пространства $X$, что для каждого $\alpha \in \mf A$ выполнено $\overline{V_{\alpha}}\subset U_{\alpha}$.
\end{proposition}

\begin{proof}
    Вполне упорядочим множество $\mathfrak{A}$ по теореме Цермело и проведем трансфинитную индукцию. Положим $F_1 := X\backslash \bigcup_{\alpha \geq 2}U_{\alpha} \subset U_1$. Поскольку пространство $X$ нормально, то существует окрестность $V_1$ множества $F_1$ такая, что $\overline{V_1} \subset U_1$. Система $v_1 = \{ V_1\}\cup\{U_{\alpha}\}_{\alpha\geq 2}$ является покрытием пространства $X$.
    
    Зафиксируем теперь $\alpha > 1$ и предположим, что для каждого $\alpha' < \alpha$ построено открытое множество $V_{\alpha'}$ такое, что $\overline{V_{\alpha'}} \subset U_{\alpha'}$, и система $v_{\alpha'} := \{ V_{\alpha''}\}_{\alpha'' \le \alpha'} \cup \{U_{\alpha''}\}_{\alpha'' > \alpha'}$ является покрытием пространства $X$. Покажем, что тогда покрытием пространства $X$ также является следующая система:
    \[v'_{\alpha} := \{ V_{\alpha'}\}_{\alpha'<\alpha}\cup\{U_{\alpha'}\}_{\alpha'\geq\alpha}\]
    \begin{itemize}
        \item Пусть $\alpha$ "--- изолированное порядковое число, тогда система $v'_{\alpha}$ совпадает с $v_{\alpha-1}$ и является покрытием по предположению индукции.
        
        \item Пусть $\alpha$ "--- предельное порядковое число. Зафиксируем $x \in
        X\backslash \bigcup_{\alpha'\geq \alpha}U_{\alpha'}$. В силу точечной конечности, существует лишь конечный набор множеств $U_{\alpha_1},\ldots, U_{\alpha_k} \in u$, содержащих точку $x$, причем $\alpha_0 := \max\{\alpha_1, \dotsc, \alpha_k\} < \alpha$. По предположению индукции, система $v_{\alpha_0}$ является покрытием пространства $X$, но $x\notin \cup_{\alpha'>\alpha_0} U_{\alpha'}$ по выбору $\alpha_0$, поэтому $x \in \bigcup_{\alpha' \le \alpha_0} V_{\alpha'} \subset \bigcup_{\alpha'<\alpha}V_{\alpha'}$. Значит, система $v'_{\alpha}$ действительно
        является покрытием. Рассмотрим следующее множество:
        \[F_{\alpha} := X\backslash \left(\,\bigcup_{\alpha'<\alpha}V_{\alpha'}\cup
        \bigcup_{\alpha'>\alpha}U_{\alpha'}\right)\]

        Множество $F_{\alpha}$ замкнуто и содержится в $U_{\alpha}$, поэтому существует такая окрестность $V_{\alpha}$ множества $F_{\alpha}$, что $\overline{V_{\alpha}}\subset U_{\alpha}$. Таким образом, для $\alpha \in \mf A$ построено покрытие $v_\alpha := \{ V_{\alpha'}\}_{\alpha' \le \alpha} \cup \{U_{\alpha'}\}_{\alpha' > \alpha}$.
    \end{itemize}

    Наконец, положим $\beta := \sup_{\alpha \in \mf A} \alpha$, тогда покрытие $v = v'_{\beta}$ является искомым.
\end{proof}

\begin{corollary}[лемма Чеха, об ужатии конечных покрытий]
    Пусть $X$ "--- нормальное пространство, $\{ U_1, \dotsc, U_N\}$ "--- его  открытое покрытие. Тогда существует такое открытое покрытие $\{V_1, \dotsc, V_n\}$ пространства $X$, что для каждого $i \in \{1, \dotsc, n\}$ выполнено $\overline{V_i}\subset U_i$.
\end{corollary}

\begin{definition}
    Пусть $X$ "--- топологическое пространство, $\alpha$ "--- система подмножеств множества $X$. Система $\alpha$ называется:
    \begin{itemize}
        \item \textit{Локально конечной}, если у каждой точки $x \in X$ существует окрестность $U(x)$, пересекающаяся лишь с конечным числом элементов системы $\alpha$
        
        \item \textit{Дискретной}, если у каждой точки $x\in X$ существует окрестность $U(x)$, пересекающаяся не более чем с одним элементом системы $\alpha$
        
        \item \textit{Консервативной}, если для любой подсистемы $\alpha_0 \subset \alpha$ выполнено следующее равенство:
        \[\overline{\bigcup_{A\in\alpha_0}A} = \bigcup_{A\in \alpha_0}\overline{A}\]
    \end{itemize}
\end{definition}

\begin{note}
    Справедливы следующие свойства:
    \begin{enumerate}
        \item Любая дискретная система локально конечна.
        
        \item Любое звездно конечное открытое покрытие пространства $X$ локально конечно.
        
        \item Любое дизъюнктное открытое покрытие пространства $X$ является
        дискретной системой. Кроме того, если дискретная система является покрытием пространства $X$, то ее элементы являются открыто-замкнутыми множествами.
        
        \item Система всех одноточечных множеств пространства $X$ звездно конечна, но, вообще говоря, не локально конечна.
    \end{enumerate}
\end{note}

\begin{proposition}
    Пусть $X$ "--- топологическое пространство, $\sigma$ "--- система замкнутых подмножеств множества $X$. Тогда $\sigma$ дискретна $\lra$ $\sigma$ дизъюнктна и консервативна.
\end{proposition}

\begin{proof}
    Можно считать, что система $\sigma$ дизъюнктна, поскольку иначе ни одно из свойств выше не выполнено. Консервативность системы замкнутых множеств эквивалентна тому, что объединение любого набора множеств системы также является замкнутым множеством.

    \begin{itemize}
        \item[$\ra$] Пусть $\sigma_0 \subset \sigma$, и $x \in X$ "--- точка прикосновения множества $\widetilde\sigma_0$. Любая окрестность точки $x$ пересекается с $\widetilde\sigma_0$, но, в силу дискретности, пересекается она только с одним множеством из $\sigma_0$, причем одним и тем же для любой окрестности. Обозначим это множество через $F_x$, тогда $x \in \overline{F_x} = F_x \subset \widetilde\sigma_0$.
        
        \item[$\la$] Пусть $x$ "--- точка прикосновения множества $\widetilde\sigma$. В силу консервативности, множество $\widetilde\sigma$ замкнуто, поэтому $x \in \widetilde\sigma$, тогда, в силу дизъюнктности, $x$ принадлежит единственному множеству $F \in \sigma$. Но тогда точка $x$ отделена от $X \bs F$ некоторой окрестностью $U(x)$, и эта окрестность не пересекается со множествами из системы $\sigma$.\qedhere
    \end{itemize}
\end{proof}

\begin{proposition}
    Пусть $X$ "--- топологическое пространство, $\sigma$ "--- локально конечная система подмножеств множества $X$. Тогда система $\sigma$ консервативна.
\end{proposition}

\begin{proof}
    Пусть $\sigma_0\subset \alpha$. В силу монотонности оператора замыкания, выполнено включение $\overline{\bigcup_{A\in\alpha_0}A} \supset \bigcup_{A\in \alpha_0}\overline{A}$, докажем обратное включение. Пусть $x \notin \bigcup_{A\in\sigma_0}\overline{A}$ и пусть $U(x)$ "--- окрестность точки $x$, пересекающаяся лишь с конечным числом элементов $A_1, \dotsc, A_n \in \sigma_0$. Поскольку $x\notin\cup_{A\in\alpha_0}\overline{A}$, то $x \notin \overline{A_1}\cup\ldots\cup\overline{A_n}$, поэтому множество $V(x) = U(x)\backslash (\overline{A_1}\cup\dotsb\cup\overline{A_n})$ является окрестностью точки $x$, не пересекающейся ни с одним элементом системы $\alpha_0$. Следовательно, $x\notin\overline{\cup_{A\in\alpha_0}A}$, что и требовалось.
\end{proof}

\subsection{Свойства типа компактности}

\begin{definition}
    Пусть $X$ "--- топологическое пространство, $\mathfrak{A}$ "--- система открытых покрытий пространства $X$, $\mf B$ "--- система покрытий пространства $X$. Пространство $X$ назыается \textit{$(\mathfrak{A},\mathfrak{B})$-компактным}, если в каждое покрытие $\alpha\in\mathfrak{A}$ вписано некоторое покрытие $\beta \in \mathfrak{B}$. Свойство $(\mathfrak{A},\mathfrak{B})$-компактности называется \textit{свойством типа компактности}.
\end{definition}

\begin{example}
    Рассмотрим несколько важных частных случаев свойства $(\mathfrak{A},\mathfrak{B})$-компактности:
    \begin{enumerate}
        \item Бикомпактность: $\mathfrak{A}$ "--- система всех открытых покрытий, $\mathfrak{B}$ "--- система всех конечных открытых покрытий
        
        \item \textit{Линделефовость}, или \textit{финальная компактность}: $\mathfrak{A}$ "--- система всех открытых покрытий, $\mathfrak{B}$ -- система всех не более чем счетных открытых покрытий.
        
        \item \textit{Паракомпактность}: $\mathfrak{A}$ "--- система всех открытых покрытий, $\mathfrak{B}$ "--- система всех локально конечных открытых покрытий.
        
        \item \textit{Сильная паракомпактность}: $\mathfrak{A}$ "--- система всех открытых покрытий, $\mathfrak{B}$ "--- система всех звездно конечных открытых покрытий.

        \item \textit{Счетная компактность}: $\mathfrak{A}$ "--- система всех счетных открытых покрытий, $\mathfrak{B}$ "--- система всех конечных открытых покрытий.
    \end{enumerate}
\end{example}

\begin{note}
    В свойствах $(1)$, $(2)$ и $(5)$ можно требовать, чтобы покрытие $\beta\in\mathfrak{B}$ не просто было вписано в покрытие $\alpha\in\mathfrak{A}$, а содержалось в покрытии $\alpha$.
\end{note}

\begin{note}
    Любое бикомпактное пространство является линделефовым, счетно компактным и сильно паракомпактным, а любое сильно паракомпактное пространство является паракомпактным.
\end{note}

\begin{proposition}
    Свойства $(1)$--$(5)$ наследуются замкнутыми подмножествами: если топологическое пространство $X$ обладает соответствующим свойством, то этим же свойством обладает любое его замкнутое подмножество $\Phi \subset X$.
\end{proposition}

\begin{proof}
    Пусть $\{U_n\}$ -- (счетное) открытое покрытие подпространства $\Phi$. Для каждого множества $U_n$ существует такое открытое множество $V_n \subset X$, что $V_n \cap \Phi = U_n$. Система $\{X\backslash \Phi\}\cup\{V_n\}$ образует (счетное) открытое покрытие пространства $X$. По предположению, существует покрытие $\{G_k\}$, вписанное в $\{V_n\}$ и принадлежащее соответствующему классу покрытий. Тогда система $\{G_k \cap \Phi\}$ образует искомое вписанное в $\{U_n\}$ покрытие.
\end{proof}