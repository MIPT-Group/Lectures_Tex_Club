\begin{corollary}
	$|\sin x| \le |x|\ \forall x \in \R$
\end{corollary}

\begin{proof}
	Если мы положим $x \in (-\frac{\pi}{2}; 0]$, то получим
	\begin{align*}
		&(-x) \in [0; \frac{\pi}{2})
		\\
		&\sin x = -\sin (-x)
		\\
		&x = -(-x)
	\end{align*}
	По уже доказанной лемме \ref{for_trig} имеем
	\[
		\sin (-x) \le (-x) \Ra -\sin (-x) \ge -(-x) \lra \sin(x) \ge x
	\]
	Но при этом мы имеем дело с отрицательными числами. А стало быть
	\[
		|\sin x| \le |x|,\ x \in \left(-\frac{\pi}{2},\ \frac{\pi}{2}\right)
	\]
	За рамками данного интервала $x$ уже точно по модулю за областью значений $\sin x$. Поэтому утверждение верно $\forall x \in \R$.
\end{proof}

\subsubsection*{Непрерывность $\sin$, $\cos$}

Докажем, что $\liml_{x \to a} \sin x = \sin a$ при $\forall a \in \R$
\begin{multline*}
	|\sin x - \sin a| = \left|2 \sin \frac{x - a}{2} \cos \frac{x + a}{2}\right| \le 2 \cdot \left|\sin \frac{x - a}{2}\right| \cdot \left|\cos \frac{x + a}{2}\right| \le \\
	2 \cdot \left|\frac{x - a}{2}\right| \cdot 1 = |x - a|
\end{multline*}
Для доказательства предела достаточно взять $\delta := \eps / 2$. Следовательно, $\sin x$ - непрерывная на всей области определения.

Теперь докажем, что $\liml_{x \to a} \cos x = \cos a$ при $\forall a \in \R$
\[
	|\cos x - \cos a| = \left|-2 \cdot \sin \frac{x + a}{2} \cdot \sin \frac{x - a}{2} \right| \le 2 \cdot 1 \cdot \left|\frac{x - a}{2}\right| = |x - a|
\]
Снова достаточно взять $\delta := \eps / 2$ и доказательство получено.


\begin{theorem} (Первый замечательный предел) Предел $\liml_{x \to 0} \frac{\sin x}{x}$ существует и равен $1$.
\end{theorem}

\begin{proof}
	Рассмотрим $x \in (0; \frac{\pi}{2})$. Тогда, по лемме \ref{for_trig} получим:
	$$
		1 < \frac{x}{\sin x} < \frac{1}{\cos x}
	$$
	В предельном переходе
	$$
		1 \le \liml_{x \to 0+} \frac{x}{\sin x} \le 1
	$$
	Следовательно $\liml_{x \to 0+} \frac{\sin x}{x} = 1$
\end{proof}

\subsubsection*{Непрерывность показательной функции}

\begin{lemma}
	\[
		\liml_{n \to \infty} \sqrt[n]{a} = 1,\ \forall a > 0
	\]
\end{lemma}

\begin{proof}
	Рассмотрим 3 случая:
	\begin{enumerate}
		\item $a = 1$ - тривиально
		
		\item $a > 1$. В таком случае, $\sqrt[n]{a} > 1$. А значит
		\[
			\sqrt[n]{a} = 1 + \alpha_n, \text{ где } \alpha_n \text{ - просто какая-то последовательность, причем } \alpha_n > 0
		\]
		Возведём все в $n$-ю степень и применим неравенство Бернулли:
		\[
			a = (1 + \alpha_n)^n \ge 1 + n\alpha_n
		\]
		Отсюда имеем
		\[
			0 < \alpha_n < \frac{a - 1}{n}
		\]
		Несложно увидеть, что $\liml_{n \to \infty} \frac{a - 1}{n} = 0$. Ну а значит
		\[
			\liml_{n \to \infty} \alpha_n = 0 \text{ - бесконечно малая последовательность}
		\]
		Используя предельный переход в равенстве, имеем
		\[
			\liml_{n \to \infty} \sqrt[n]{a} = \liml_{n \to \infty} (1 + \alpha_n) = 1 + 0 = 1
		\]
		
		\item $a < 1$. Теперь $\sqrt[n]{a} < 1$. Но всё равно можно применить тот же трюк:
		\[
			\sqrt[n]{a} = \frac{1}{1 + \alpha_n}, \text{ где } \alpha_n > 0
		\]
		Тогда получим
		\begin{align*}
			&a = (1 + \alpha_n)^{-n} \ge 1 - n\alpha_n
			\\
			&0 < \alpha_n \le \frac{1 - a}{n} \Ra \liml_{n \to \infty} \alpha_n = 0
			\\
			&\Ra \liml_{n \to \infty} \sqrt[n]{a} = \liml_{n \to \infty} \frac{1}{1 + \alpha_n} = \frac{1}{1 + 0} = 1
		\end{align*}
	\end{enumerate}
\end{proof}

Считая, что все рациональные степени уже определены, дадим определение $a^x$ в общем случае

\begin{definition}
	$a^x$ при $\forall x \in \R$ определяется как
	\begin{enumerate}
		\item $a > 1$
		\item $a = 1 \Ra a^1 = a$
		\item $0 < a < 1$
	\end{enumerate}
\end{definition}

\begin{lemma}
	$(\forall x \in \R)\ (\forall \{r_n\} \subset \Q,\ \liml_{n \to \infty} r_n = x) \Ra \liml_{n \to \infty} a^{r_n} = a^x$
\end{lemma}

\begin{proof}
	Доказательство проводится для $a > 1$. Для другого случая очевидно.
	
	Для доказательства понадобится утверждение:
	$$
		\liml_{n \to \infty} \sqrt[n]{a} = 1
	$$
	
	Заметим, что обе последовательности $a^{-\frac{1}{n}} - 1$ и $a^{\frac{1}{n}} - 1$ стремятся к нулю. То есть
	$$
		\forall \eps > 0\ \exists K \in \N \such \forall n > N\ \max(|a^{-\frac{1}{K}} - 1|, a^{\frac{1}{K}} - 1) < \eps
	$$
	Докажем, что $\{a^{r_n}\}$ - фундаментальная последовательность.
	$$
		a^{r_{n + p}} - a^{r_n} = a^{r_n} \cdot (a^{r_{n + p} - r_n} - 1)
	$$
	Так как $\{r_n\}$ сходится, то $\exists M \such \forall n \in \N\ r_n \le M$. Отсюда
	$$
		a^{r_n} \le a^M
	$$
	Сама $\{r_n\}$ фундаментальна. То есть
	$$
		\forall \eps > 0\ \exists N \in \N \such \forall n > N, p \in \N\ |r_{n + p} - r_n| < \frac{1}{K} \Ra -\frac{1}{K} < r_{n + p} - r_n < \frac{1}{K}
	$$
	Следовательно
	$$
		a^{-\frac{1}{K}} - 1 < a^{r_{n + p} - r_n} - 1 < a^{\frac{1}{K}} - 1
	$$
	Ну а отсюда уже
	$$
		|a^{r_{n + p} - r_n} - 1| \le \max(|a^{-\frac{1}{K}} - 1|, |a^{\frac{1}{K}} - 1|) < \frac{\eps}{a^M}
	$$
	Покажем, что у $\{a^{(x)_n}\}$ и $\{a^{r_n}\}$ одинаковые пределы. Для этого рассмотрим последовательность, у которой на чётных местах стоит элементы одной последовательности, а на нечётных - элементы другой.
\end{proof}

\subsubsection*{Свойства показательной функции}

\begin{enumerate}
	\item $a^{x_1 + x_2} = a^{x_1} \cdot a^{x_2}$
	\item $a^x \cdot b^x = (a \cdot b)^x$
	\item $(a^x)^y = a^{x \cdot y}$
\end{enumerate}

\begin{proof}
	Докажем свойство суммы:
	\begin{align*}
		\liml_{n \to \infty} (x_1)_n = x_1
		\\
		\liml_{n \to \infty} (x_2)_n = x_2
	\end{align*}
	При этом
	\begin{align*}
		\liml_{n \to \infty} a^{(x_1)_n} = a^{x_1}
		\\
		\liml_{n \to \infty} a^{(x_2)_n} = a^{x_2}
	\end{align*}
	Тогда
	$$
		\liml_{n \to \infty} a^{(x_1)_n + (x_2)_n} = a^{x_1 + x_2}
	$$
	Второе свойство доказывается аналогично.
	
	Третье свойство уже сложнее. Пусть $x, y > 0$,
	\begin{align*}
		\{r'_n\} - \text{ возрастает к } x
		\\
		\{r''_n\} - \text{ убывает к } x
		\\
		\{p'_n\} - \text{ возрастает к } y
		\\
		\{p''_n\} - \text{ убывает к } y
	\end{align*}
	Отсюда цепочка неравенств:
	$$
		a^{r'_n \cdot p'_n} = (a^{r'_n})^{p'_n} \le (a^x)^{p'_n} \le (a^x)^y \le (a^x)^{p''_n} \le (a^{r''_n})^{p''_n} = a^{r''_n \cdot p''_n}
	$$
\end{proof}

\begin{theorem}
	$a^x$ - непрерывная функция на $\R$ $\forall a \in (0; 1) \cup (1; \infty)$
\end{theorem}

\begin{proof}
	$a^x - a^{x_0} = a^{x_0}(a^{x - x_0} - 1)$
	
	Это значит, что достаточно установить факт
	$$
		\liml_{x \to 0} a^x = 1
	$$
	Рассмотрим $|x| < \frac{1}{K}$ для произвольного $K$. Тогда
	$$
		a^{-\frac{1}{K}} - 1 < a^x - 1 < a^{\frac{1}{K}} - 1
	$$
	Значит
	$$
		\forall \eps > 0\ \exists K \in \N \such \max(|a^{\frac{1}{K}} - 1|, |a^{-\frac{1}{K}} - 1|) < \eps
	$$
	Отсюда
	$$
		\exists \delta := \frac{1}{K} \such \forall x, |x| < \delta\ |a^x - 1| < \eps
	$$
\end{proof}

\begin{corollary} (Непрерывность логарифма)
	$\log_a x$ - непрерывная функция на $(0; +\infty)$
\end{corollary}

\begin{proof}
	Будем считать $a > 1$. Начнём с доказательнства, что
	\begin{align*}
		\inf\limits_{x \in \R} a^x = 0
		\\
		\sup\limits_{x \in \R} a^x = +\infty
	\end{align*}
	По неравенству Бернулли
	$$
		a = 1 + \alpha;\ (1 + \alpha)^n \ge 1 + n\alpha
	$$
	$$
		(1 + \alpha)^{-n} = \frac{1}{(1 + \alpha)^n}
	$$
	Отсюда будет следовать, что 
\end{proof}

\begin{theorem} (Второй замечательный предел) \\
	$$
		\liml_{x \to 0} (1 + x)^{1 / x} = e
	$$
\end{theorem}

\begin{proof}
	Положим $0 < x < 1$.
	$$
		n_x := \left\lfloor\frac{1}{x}\right\rfloor
	$$
	$$
		n_x \le \frac{1}{x} < n_x + 1 \Ra \frac{1}{n_x + 1} < x \le \frac{1}{n_x}
	$$
	Рассмотрим функцию
	$$
		f(x) = \left(1 + \frac{1}{n_x}\right)^{n_x + 1}
	$$
	Положим $x_1 < x_2$. Следовательно
	$$
		\frac{1}{x_2} < \frac{1}{x_1} \Ra n_{x_2} \le n_{x_1} \Ra f(x_2) \ge f(x_1) \ge 1 \text{ (в силу последовательности числа Эйлера)}
	$$
	По теореме Вейерштрасса предел $f(x)$ существует.
	
	Сделаем замечательное наблюдение:
	$$
		f\left(\frac{1}{n}\right) = \left(1 + \frac{1}{n}\right)^n \Ra \liml_{n \to \infty} f(1/n) = e
	$$
	Напишем цепочку неравенств:
	\begin{align*}
		&(1 + x)^{1 / x} \ge (1 + x)^{n_x} > \left(1 + \frac{1}{n_x + 1}\right)^{n_x}
		\\
		&(1 + x)^{1 / x} \le \left(1 + \frac{1}{n_x}\right)^{1 / x} < \left(1 + \frac{1}{n_x}\right)^{n_x + 1}
	\end{align*}
	Крайняя левая и крайняя правая оценки стремятся к $e$. А значит
	$$
		\liml_{x \to 0+} f(x) = e
	$$
	Осталось доказать левый предел. Сделаем замену $x = -y$:
	\begin{multline*}
		\liml_{x \to 0-} (1 + x)^{1 / x} = \liml_{y \to 0+} (1 - y)^{-1/y} = \\
		\liml_{y \to 0+} \frac{1}{(1 - y)^{1/y}} = \liml_{y \to 0+} \left(\frac{1}{1 - y}\right)^{1 / y} = \\
		\liml_{y \to 0+} \left(1 + \frac{y}{1 - y}\right)^{1 / y} = \liml_{t \to 0+} (1 + t)^{\frac{1 - t}{t}} = \\
		\liml_{t \to 0+} (1 + t)^{1 / t} \cdot \frac{1}{1 + t} = e \cdot 1 = e
	\end{multline*}
	По теореме о связи предела с односторонними пределами, в итоге получаем
	\[
		\liml_{x \to 0} (1 + x)^{1 / x} = e
	\]
\end{proof}

\begin{lemma}
	$\forall a \in (0; 1) \cup (1; +\infty)$
	\begin{enumerate}
		\item $\liml_{x \to \infty} \left(1 + \frac{1}{x}\right)^x = e$
		\item $\liml_{x \to 0} \frac{\log_a (1 + x)}{x} = \frac{1}{\ln_a}$
		\item $\liml_{x \to 0} \frac{a^x - 1}{x} = \ln a$
	\end{enumerate}
\end{lemma}

\begin{proof}
	\begin{enumerate}
		\item Введём $g(y)$:
		$$
			g(y) = \System{
			&{(1 + y)^{1 / y},\ y \neq 0}
			\\
			&{e, y = 0}
			}
		$$
		В силу второго замечательного предела
		\[
			\liml_{y \to 0} (1 + y)^{1 / y} = e = g(0)
		\]
		То есть, $g(y)$ непрерывна в 0. Дополнительно введём $y = f(x) = 1 / x$. Для этой функции есть предел
		\[
			\liml_{x \to \infty} f(x) = 0
		\]
		По теореме о пределе композиции функций получим
		\[
			\liml_{x \to \infty} \left(1 + \frac{1}{x}\right)^x = \liml_{x \to \infty} g(f(x)) = g(0) = e
		\]
		
		\item В силу непрерывности логарифма в 0 применим композицию:
		$$
			\liml_{x \to 0} \frac{\log_a (1 + x)}{x} = \liml_{x \to 0} \log_a((1 + x)^{1 / x}) = \log_a(\liml_{x \to 0} (1 + x)^{1 / x}) = \log_a e = \frac{1}{\ln a}
		$$
		
		\item Положим $f(x) = a^x - 1$. Если выразить $x$ через $y = f(x)$, то получится
		\[
			x = \log_a (y + 1)
		\]
		А выражение предела получает вид
		\[
			\frac{a^x - 1}{x} = \frac{y}{\log_a (y + 1)}
		\]
		Если подставить в такое выражение $y = 0$, то мы получим неопределённость. Это можно исправить, определив $g(y)$ как
		\[
			g(y) := \System{
				&{\frac{y}{\log_a (y + 1)},\ y \neq 0}
				\\
				&{\ln a,\ y = 0}
			}
		\]
		Посчитаем предел
		\[
			\liml_{y \to 0} g(y) = \liml_{y \to 0} \frac{1}{\frac{1}{g(y)}} = \liml_{y \to 0} \frac{1}{\frac{\log_a (y + 1)}{y}} = \frac{1}{\frac{1}{\ln a}} = \ln a
		\]
		То есть $g(y)$ непрерывно в 0. При этом
		\[
			\liml_{x \to 0} f(x) = 1 - 1 = 0
		\]
		По теореме о композиции функций получим
		\[
			\liml_{x \to 0} \frac{a^x - 1}{x} = \liml_{x \to 0} g(f(x)) = g(0) = \ln a
		\]
	\end{enumerate}
\end{proof}

\subsubsection*{Гиперболические функции}

\begin{definition}
	\begin{align*}
		&{\sh x := \frac{e^x - e^{-x}}{2} - \text{ гиперболический синус}}
		\\
		&{\ch x := \frac{e^x + e^{-x}}{2} - \text{ гиперболический косинус}}
		\\
		&{\th x := \frac{\sh x}{\ch x}}
		\\
		&{\cth x := \frac{\ch x}{\sh x}}
	\end{align*}
\end{definition}

Все гиперболические функции непрерывны на своих областях определения.

\subsubsection*{Свойства гиперболических функций}

\begin{enumerate}
	\item $\ch^2 x - \sh^2 x = 1$ - основное гиперболическое тождество
	\item $\ch^2 x + \sh^2 x = \ch 2x$
	\item $2 \cdot \sh x \cdot \ch x = \sh 2x$
\end{enumerate}

\subsubsection*{Обратные функции}

$$
	y = \sh x \lra x = \Arsh y
$$
В этом нет смысла, так как можно решить уравнение явно
$$
	y = \frac{e^x - e^{-x}}{2} \lra e^{2x} - 2ye^x - 1 = 0 \lra x = \ln (y + \sqrt{y^2 + 1})
$$

\begin{addition}
	$\liml_{x \to 0} \frac{\sh x}{x} = \liml_{x \to 0} \frac{e^x - e^{-x}}{2x} = \liml_{x \to 0} \frac{e^x - 1}{2x} - \frac{e^{-x} - 1}{2z} = \frac{1}{2} + \liml_{t \to 0} \frac{e^t - 1}{2t} = 1$
\end{addition}