\begin{definition}
	Пусть $f$ определена на $(a; b)\ |\ -\infty < a < b < +\infty$
	
	\textit{Левосторонним пределом} в точке $b$ называется $B \in \bar{\R} \cup \{\infty\}$ такое, что
	\begin{enumerate}
		\item $\forall \eps > 0\ \exists \delta > 0\ |\ \forall x,\ b - \delta < x < b\ \ f(x) \in U_{\eps}(B)$
		
		\item $\forall \{x_n\}_{n = 1}^\infty \subset (a; b),\ \liml_{n \to \infty} x_n = b\ \ \liml_{n \to \infty} f(x_n) = B$
	\end{enumerate}
	Обозначается как
	\[
		f(b - 0) := \liml_{x \to b-0} f(x) = B
	\]

	\textit{Правосторонним пределом} в точке $a$ называется $A \in \bar{\R} \cup \{\infty\}$ такое, что
	\begin{enumerate}
		\item $\forall \eps > 0\ \exists \delta > 0\ |\ \forall x,\ a < x < a + \delta\ \ f(x) \in U_{\eps}(A)$
		
		\item $\forall \{x_n\}_{n = 1}^\infty \subset (a; b),\ \liml_{n \to \infty} x_n = a\ \ \liml_{n \to \infty} f(x_n) = A$
	\end{enumerate}
	Обозначается как
	\[
	f(b + 0) := \liml_{x \to a+0} f(x) = A
	\]	
\end{definition}

\begin{definition}
	$(b - \delta; b)$ называется \textit{левосторонней} окрестностью точки $b$.
	
	$(a; a + \delta)$ называется \textit{правосторонней} окрестностью точки $a$.
\end{definition}

\begin{theorem} (Связь предела и односторонних пределов)
	Пусть $f$ ограничена в $U_{\delta}(a)$, $a \in \R$. Тогда
	$$
		\exists \liml_{x \to a} f(x) \lra \exists \liml_{x \to a-0} f(x) = \liml_{x \to a+0} f(x)
	$$
\end{theorem}

\begin{proof}
\begin{enumerate}
	\item Пусть $\exists \liml_{x \to a} f(x) = A$, тогда
	$$
		\forall \eps > 0\ \exists \delta > 0\ |\ \forall x,\ 0 < |x - a| < \delta,\ f(x) \in U_{\eps}(A)
	$$
	Отсюда следует, что $\forall x\ |\ a < x < a + \delta,\ f(x) \in U_{\eps}(A)$ и $\forall x\ |\ a - \delta < x < a,\ f(x) \in U_{\eps}(A)$, что равносильно утверждению справа.
	
	\item Пусть $\exists \liml_{x \to a-0} f(x) = \liml_{x \to a+0} f(x) = A$. Тогда
	\begin{align*}
		\forall \eps > 0\ \exists \delta_1 > 0 \such \forall x,\ a - \delta_1 < x < a\ \ f(x) \in U_{\eps}(A)
		\\
		\forall \eps > 0\ \exists \delta_2 > 0 \such \forall x,\ a < x < a + \delta_2\ \ f(x) \in U_{\eps}(A)
	\end{align*}
	Выберем $\delta := \min(\delta_1, \delta_2)$, получим
	\begin{align*}
		\delta_1 \ge \delta \Ra a - \delta_1 \le a - \delta
		\\
		\delta_2 \ge \delta \Ra a + \delta_2 \ge a + \delta
	\end{align*}
	Рассмотрим $\forall x \in \mc{U}_\delta(a)$:
	\begin{align*}
		a < x < a + \delta \Ra a < x < a + \delta_2
		\\
		a - \delta < x < a \Ra a - \delta_1 < x < a
	\end{align*}
	Любой из этих случаев ведёт к тому, что $f(x) \in U_{\eps}(A)$. А значит
	\[
		\forall \eps > 0 \exists \delta > 0 \such \forall x,\ x \in \mc{U}_{\delta}(a)\ \ f(x) \in U_{\eps}(A)
	\]
	Что равносильно левой стороне утверждения.
\end{enumerate}

\end{proof}

\begin{definition}
	Функция $f$ называется
	\begin{itemize}
		\item \textit{неубывающей} на $X$, если $\forall x_1, x_2 \in X, x_1 < x_2 \Ra f(x_1) \le f(x_2)$
		\item \textit{невозрастающей} на $X$, если $\forall x_1, x_2 \in X, x_1 < x_2 \Ra f(x_1) \ge f(x_2)$
		\item \textit{убывающей} на $X$, если $\forall x_1, x_2 \in X, x_1 < x_2 \Ra f(x_1) > f(x_2)$
		\item \textit{возрастающей} на $X$, если $\forall x_1, x_2 \in X, x_1 < x_2 \Ra f(x_1) < f(x_2)$
	\end{itemize}
	В любом из этих случаев $f$ монотонна на $X$, в 2х последних $f$ строго монотонна на $X$.
\end{definition}

\begin{theorem} (Существование односторонних пределов монотонной функции)
	Если $f$ монотонна на $(a; b)$, $-\infty < a < b < +\infty$, то
	$$
		\exists \liml_{x \to a+0} f(x) \in \bar{\R},\ \liml_{x \to b-0} f(x) \in \bar{\R}
	$$
	причём если $f$ неубывающая, то
	$$
		\liml_{x \to a+0} f(x) = \inf\limits_{x \in (a; b)} f(x),\ \liml_{x \to b-0} f(x) = \sup\limits_{x \in (a; b)} f(x)
	$$
	если $f$ невозрастающая, то
	$$
		\liml_{x \to a+0} f(x) = \sup\limits_{x \in (a; b)} f(x),\ \liml_{x \to b-0} f(x) = \inf\limits_{x \in (a; b)} f(x)
	$$
\end{theorem}

\begin{proof}
	Пусть $f$ неубывающая. Положим $\sup\limits_{x \in (a; b)} f(x) := M$
	\begin{enumerate}
		\item $M = +\infty$. Тогда
		\[
			\forall \eps > 0\ \exists x_0 \in (a; b)\ |\ f(x_0) > \frac{\dse 1}{\dse \eps}
		\]
		Отсюда $\exists \delta := b - x_0 > 0\ |\ \forall x, b - \delta < x < b \Ra \frac{\dse 1}{\dse \eps} < f(x) \le f(x)$, то есть $f(x_0) \in U_{\eps}(+\infty)$. В итоге
		\[
			\forall \eps > 0\ \exists \delta > 0 \such \forall x,\ b - \delta < x < b\ \ f(x) \in U_{\eps}(M) \lra \liml_{x \to b-0} f(x) = +\infty = M
		\]
		
		\item $M < +\infty$. Тогда
		\[
			\forall \eps > 0\ \exists x_0 \in (a; b)\ |\ f(x_0) \in (M - \eps; M]
		\]
		Отсюда уже аналогично получим, что
		\[
			\forall \eps > 0\ \exists \delta > 0 \such \forall x,\ b - \delta < x < b\ f(x) \in U_{\eps}(M) \lra \liml_{x \to b-0} f(x) = M
		\]
		Если $a = -\infty$, то $\liml_{x \to -\infty} f(x)$ вместо $\liml_{x \to a+0} f(x)$
		
		Если $b = +\infty$, то $\liml_{x \to +\infty} f(x)$ вместо $\liml_{x \to b-0} f(x)$
		
		
	\end{enumerate}
\end{proof}

\subsection{Непрерывность}

\subsubsection*{Непрерывность в точке}

\begin{definition}
	Если $f$ определена в некоторой окрестности точки $x_0$ и $\liml_{x \to x_0} f(x) = f(x_0)$, то функция называется \textit{непрерывной} в точке $x_0$.
\end{definition}

\begin{definition}
	Если $f$ определена на $[x_0; x_0 + \delta_0]$, где $\delta_0 > 0$ и $\liml_{x \to x_0 + 0} f(x) = f(x_0)$, то $f$ называется \textit{непрерывной справа} в точке $x_0$.
\end{definition}

\begin{definition}
	Если $f$ определена на $[x_0 - \delta_0; x_0]$, где $\delta_0 > 0$ и $\liml_{x \to x_0 - 0} f(x) = f(x_0)$, то $f$ называется \textit{непрерывной слева} в точке $x_0$.
\end{definition}

\begin{theorem}
	Пусть $f$ определена в некоторой окрестности точки $x_0$. Тогда, следующие утверждения эквивалентны:
	\begin{enumerate}
		\item $f$ непрерывна в точке $x_0$
		\item $\forall \eps > 0\ \exists \delta > 0 \such \left(\forall x,\ |x - x_0| < \delta\right) \Ra |f(x) - f(x_0)| < \eps$
		\item $\left(\forall \{x_n\}_{n = 1}^\infty,\ \liml_{n \to \infty} x_n = x_0\right) \liml_{n \to \infty} f(x_n) = f(x_0)$
	\end{enumerate}
\end{theorem}

\subsubsection*{Точки разрыва}

\begin{definition}
	Пусть $f$ определена в проколотой окрестности точки $x_0$. Если $\liml_{x \to x_0} f(x) \neq f(x_0)$, то $x_0$ называется \textit{точкой разрыва} функции $f(x)$.
\end{definition}

\begin{note}
	Неравенство полагается верным также и в тех случаях, когда хоть одна из частей не определена.
\end{note}

\begin{definition}
	Если $\exists \liml_{x \to x_0-0} f(x),\ \liml_{x \to x_0+0} f(x) \in \R$, то точка разрыва называется \textit{точкой разрыва первого рода}.
	
	В противном случае \textit{точкой разрыва второго рода}.
\end{definition}

\begin{definition}
	Если $\liml_{x \to x_0-0} f(x) = \liml_{x \to x_0+0} f(x) \in \R$ и $\neq f(x_0)$, то $x_0$ называется \textit{точкой устранимого разрыва}.
\end{definition}

\begin{definition}
	Если хотя бы 1 из односторонних пределов бесконечен, то $x_0$ называется \textit{точкой бесконечного разрыва}.
\end{definition}

\begin{definition}
	Величину $\liml_{x \to x_0+0} f(x) - \liml_{x \to x_0-0} f(x)$ называется \textit{скачком функции} в точке $x_0$.
\end{definition}

\begin{example}
	$f(x) = \sgn x = \System{&{1,\ x > 0} \\ &{0,\ x = 0} \\ &{-1,\ x < 0}}$
	\begin{align*}
		&\liml_{x \to 0 + 0} f(x) = 1
		\\
		&\liml_{x \to 0 - 0} f(x) = -1
	\end{align*}
\end{example}

\begin{example}
	$f(x) = \sgn^2 x$
	$$
		\liml_{x \to 0} \sgn^2 x = 1 \neq \sgn^2 0 = 0
	$$
\end{example}

\begin{example}
	$f(x) = \frac{1}{x}$
	\begin{align*}
		\liml_{x \to -0} \frac{\dse 1}{\dse x} = -\infty
		\\
		\liml_{x \to +0} \frac{\dse 1}{\dse x} = +\infty
	\end{align*}
\end{example}

\begin{example}
	$f(x) = \sin \frac{\dse 1}{\dse x}$
	
	Рассмотрим
	$$
		\System{
		&{x'_n = \frac{1}{\frac{\pi}{2} + 2\pi n},\ n \in \N}
		\\
		&{x''_n = \frac{1}{-\frac{\pi}{2} + 2\pi n}}
		}
		\Ra
		\System{
		&{\liml_{n \to \infty} f(x'_n) = 1}
		\\
		&{\liml_{n \to \infty} f(x''_n) = -1}
		}
	$$
\end{example}

\begin{theorem} (О точках разрыва монотонной функции)
	Если $f(x)$ монотонна на $(a; b),\ -\infty \le a < b \le +\infty$, то она может иметь на $(a; b)$ лишь точки разрыва 1го рода, причём неустранимого разрыва, и число таких точек разрыва не более чем счётно.
\end{theorem}

\begin{proof}
	$\forall x_0 \in (a; b)\ \exists \liml_{x \to x_0-0} f(x),\ \liml_{x \to x_0+0} f(x) \in \R$
	
	Считая $f$ неубывающей функцией, то 
	$$
		\forall x < x_0 \Ra f(x) \le f(x_0) \Ra f(x_0 - 0) \le f(x_0) \le f(x_0 + 0)
	$$
	$f(x_0 - 0) \neq f(x_0 + 0)$, иначе бы точки разрыва не было.
	
	$x_1 < x_2 \Ra f(x_1 - 0) < f(x_1 + 0) \le f(x_2 - 0) < f(x_2 + 0)$
	
	Отсюда $(f(x_1 - 0); f(x_1 + 0)) \cap (f(x_2 - 0); f(x_2 + 0)) = \emptyset$. То есть, мы получили систему непересекающихся отрезков на прямой действительных чисел, которая является не более чем счётным множеством (каждому отрезку можно сопоставить рациональное число внутри него).
\end{proof}

\begin{example} (Функция Римана)
	\[
		f(x) = \System{&{\frac{1}{n}, \text{ если } x = \frac{m}{n}} \\ &{0, \text{ если } x \in \R \bs \Q}}
	\]
	
	Докажем, что $f$ непрерывна в $x_0 \in \R \bs \Q$: зафиксируем произвольный $\eps > 0$ и рассмотрим множество
	\[
		M = \{x \such f(x) \ge \eps\}
	\]
	Так как $\eps > 0$ и $f(x) = 0\ \forall x \in \R \bs \Q$, то любой элемент $M$ - рациональное число, имеющее вид в несократимой дроби $\frac{m}{n}, m \in \Z, n \in \N$.
	\[
		f(x) = \frac{1}{n} \ge \eps \Ra n \le \frac{1}{\eps}
	\]
	То есть число таких $n$ конечно. Это значит, что число рациональных точек, попавших в $U_\delta(x_0) \cap M$, конечно (в самом деле, бесконечность может достигаться только за счёт $m$, а это мы ограничили пересечением). Ну а раз так, то найдётся $\delta > 0$ такое, что $U_\delta(x_0) \cap M = \emptyset$. Иными словами,
	\[
		\forall \eps > 0\ \exists \delta > 0 \such \forall x \in U_\delta(x_0)\ \ f(x) < \eps
	\]
	это означает непрерывность функции Римана в любой иррациональной точке.
	
	Теперь докажем, что $f(x)$ разрывна во всех рациональных точках. Пусть $x_0 \in \Q$ и мы снова зафиксировали $\eps > 0$. Какую $\delta$-окрестность точки $x_0$ ни взять, там найдётся иррациональное число, для которого $f(x) = 0 \Ra$ получим разрывность.
	
	Таким образом, функция Римана непрерывна $\forall x \in \R \bs \Q$ и разрывна $\forall x \in \Q$.
\end{example}

\subsubsection*{Непрерывность на множестве}

\begin{definition}
	Функция называется \textit{непрерывной на множестве} $X$, если 
	\[
		\forall x_0 \in X\ \ \left(\forall \{x_n\} \subset X, \liml_{x \to \infty} x_n = x_0 \Ra \liml_{n \to \infty} f(x_n) = f(x_0)\right)
	\]
	или по Коши
	\[
		\forall x_0 \in X\ \ \left(\forall \eps > 0\ \exists \delta > 0 \such \forall x \in X \cap U_\delta(x_0)\ \left|f(x) - f(x_0)\right| < \eps\right)
	\]
\end{definition}

\begin{note}
	Не стоит думать, что непрерывность на множестве - это непрерывность в каждой точке этого множества. Это не так. Как минимум потому, что мы не требуем определённость функции в некоторой окрестности точки из $X$.
\end{note}

\begin{theorem} (Первая теорема Вейерштрасса о непрерывных на отрезке функциях)
	Если $f$ непрерывна на $[a; b]$, то она ограничена на $[a; b]$
\end{theorem}

\begin{proof}
	Докажем от противного. Пусть $f$ - неограничена сверху (снизу аналогично). Это означает
	$$
		\forall \eps > 0\ \exists x_{\frac{1}{\eps}} \in [a; b]\ |\ f(x_{\frac{1}{\eps}}) > \frac{1}{\eps}
	$$
	Последовательно будем брать $\eps := 1, \frac{1}{2}, \dots, \frac{1}{n}, \dots$
	
	Получим $\{x_n\}_{n = 1}^\infty \subset [a; b],\ f(x_n) > n$. По теореме Больцано-Вейерштрасса
	$$
		\exists \{x_{n_k}\}_{k = 1}^\infty,\ \liml_{k \to \infty} x_{n_k} = x_0 \in [a; b]
	$$
	А из этого следует, что $\liml_{k \to \infty} f(x_{n_k}) = f(x_0)$, что неверно ($f(x_n) > n$).
\end{proof}