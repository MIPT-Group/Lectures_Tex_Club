\begin{definition}
	Пусть $f$ определена на $(a; b)\ |\ -\infty < a < b < +\infty$
	
	\textit{Левосторонним пределом} в точке $b$ называется $B \in \bar{\R} \cup \{\infty\}$ такое, что
	\begin{enumerate}
		\item $\forall \eps > 0\ \exists \delta > 0\ |\ \forall x\ b - \delta < x < b,\ f(x) \in U_{\eps}(B)$
		
		\item $\forall \{x_n\}_{n = 1}^\infty \subset (a; b),\ \liml_{n \to \infty} x_n = b\ \ \liml_{n \to \infty} f(x_n) = B$
	\end{enumerate}

	\textit{Правосторонним пределом} в точке $a$ называется $A \in \bar{\R} \cup \{\infty\}$ такое, что
	\begin{enumerate}
		\item $\forall \eps > 0\ \exists \delta > 0\ |\ \forall x\ a < x < a + \delta,\ f(x) \in U_{\eps}(A)$
		
		\item $\forall \{x_n\}_{n = 1}^\infty \subset (a; b),\ \liml_{n \to \infty} x_n = a\ \ \liml_{n \to \infty} f(x_n) = A$
	\end{enumerate}
\end{definition}

\begin{definition}
	$(b - \delta; b)$ называется \textit{левосторонней} окрестностью точки $b$.
	
	$(a; a + \delta)$ называется \textit{правосторонней} окрестностью точки $a$.
\end{definition}

\begin{theorem} (Связь предела и односторонних пределов)
	Пусть $f$ ограничена в $U_{\delta}(a)$, $a \in \R$. Тогда
	$$
		\exists \liml_{x \to a} f(x) \lra \exists \liml_{x \to a-0} f(x) = \liml_{x \to a+0} f(x)
	$$
\end{theorem}

\begin{proof}
\begin{enumerate}
	\item Пусть $\exists \liml_{x \to a} f(x) = A$, тогда
	$$
		\forall \eps > 0\ \exists \delta > 0\ |\ \forall x,\ 0 < |x - a| < \delta,\ f(x) \in U_{\eps}(A)
	$$
	Отсюда следует, что $\forall x\ |\ a < x < a + \delta,\ f(x) \in U_{\eps}(A)$ и $\forall x\ |\ a - \delta < x < a,\ f(x) \in U_{\eps}(A)$, что равносильно утверждению справа.
	
	\item Пусть $\exists \liml_{x \to a-0} f(x) = \liml_{x \to a+0} f(x)$. Тогда
	$$
		\forall \eps > 0\ \exists \delta_1 > 0\ |\ \forall x,\ a - \delta < x < a \Ra f(x) \in U_{\eps}(A)
		\forall \eps > 0\ \exists \delta_2 > 0\ |\ \forall x,\ a < x < a + \delta \Ra f(x) \in U_{\eps}(A)
	$$
	Выберем $\delta := \min(\delta_1, \delta_2)$
\end{enumerate}

%%%%%%%%%%%%%%%%%%%%%%%%%%% ДОПИСАТЬ
\end{proof}

\begin{definition}
	Функция $f$ называется
	\begin{itemize}
		\item \textit{неубывающей} на $X$, если $\forall x_1, x_2 \in X, x_1 < x_2 \Ra f(x_1) \le f(x_2)$
		\item \textit{невозрастающей} на $X$, если $\forall x_1, x_2 \in X, x_1 < x_2 \Ra f(x_1) \ge f(x_2)$
		\item \textit{убывающей} на $X$, если $\forall x_1, x_2 \in X, x_1 < x_2 \Ra f(x_1) > f(x_2)$
		\item \textit{возрастающей} на $X$, если $\forall x_1, x_2 \in X, x_1 < x_2 \Ra f(x_1) < f(x_2)$
	\end{itemize}
	В любом из этих случаев $f$ монотонна на $X$, в 2х последних $f$ строго монотонна на $X$.
\end{definition}

\begin{theorem} (Существование односторонних пределов монотонной функции)
	Если $f$ монотонна на $(a; b)$, $-\infty < a < b < +\infty$, то
	$$
		\exists \liml_{x \to a+0} f(x) \in \bar{\R},\ \liml_{x \to b-0} f(x) \in \bar{\R}
	$$
	причём если $f$ неубывающая, то
	$$
		\liml_{x \to a+0} f(x) = \inf\limits_{x \in (a; b)} f(x),\ \liml_{x \to b-0} f(x) = \sup\limits_{x \in (a; b)} f(x)
	$$
	если $f$ невозрастающая, то
	$$
		\liml_{x \to a+0} f(x) = \inf\limits_{x \in (a; b)} f(x),\ \liml_{x \to b-0} f(x) = \sup\limits_{x \in (a; b)} f(x)
	$$
	Если 
\end{theorem}

\begin{proof}
	Пусть $f$ неубывающая. Положим $\sup\limits_{x \in (a; b)} f(x) := M$
	\begin{enumerate}
		\item $M = +\infty$. Тогда
		$$
			\forall \eps > 0\ \exists x_0 \in (a; b)\ |\ f(x_0) > \frac{\dse 1}{\dse \eps}
		$$
		Отсюда $\exists \delta := b - x_0 > 0\ |\ \forall x, b - \delta < x < b \Ra \frac{\dse 1}{\dse \eps} < f(x_0) \le f(x)$, то есть $f(x_0) \in U_{\eps}(+\infty)$.
		Далее
		$$
			M - \eps < f(x_0) \le f(x) \le M < M + \eps \Ra f(x) \in U_{\eps}(M)
		$$
		
		\item $M < +\infty$. Тогда
		$$
			\forall \eps > 0\ \exists x_0 \in (a; b)\ |\ f(x_0) \in [M - \eps; M]
		$$
		Если $a = -\infty$, то $\liml_{x \to -\infty} f(x)$ вместо $\liml_{x \to a+0} f(x)$
		
		Если $b = +\infty$, то $\liml_{x \to +\infty} f(x)$ вместо $\liml_{x \to b-0} f(x)$
		
		
	\end{enumerate}
\end{proof}

\section{Непрерывность}

\subsection{Непрерывность в точке}

\begin{definition}
	Если $f$ определена в некоторой окрестности точки $x_0$ и $\liml_{x \to x_0} f(x) = f(x_0)$, то функция называется \textit{непрерывной} в точке $x_0$.
\end{definition}

\begin{definition}
	Если $f$ определена на $[x_0; x_0 + \delta_0]$, где $\delta_0 > 0$ и $\liml_{x \to x_0 + 0} f(x) = f(x_0)$, то $f$ называется \textit{непрерывной справа} в точке $x_0$.
\end{definition}

\begin{theorem}
	Пусть $f$ определена в некоторой окрестности точки $x_0$. Тогда, следующие утверждения эквивалентны:
	\begin{enumerate}
		\item $f$ непрерывна в т. $x_0$
		\item $\forall \eps > 0\ \exists \delta > 0\ |\ \forall x, |x - x_0| < \delta \Ra |f(x) - f(x_0)| < \eps$
		\item $\forall \{x_n\}_{n = 1}^\infty\ |\ \liml_{x \to \infty} f(x_n) = f(x_0)$
	\end{enumerate}
\end{theorem}

\subsection{Точки разрыва}

\begin{definition}
	Пусть $f$ определена в проколотой окрестности точки $x_0$. Если $\liml_{x \to x_0} f(x) \neq f(x_0)$, то $x_0$ называется \textit{точкой разрыва} функции $f(x)$.
\end{definition}

\begin{note}
	Неравенство полагается верным также и в тех случаях, когда хоть одна из частей не определена.
\end{note}

\begin{definition}
	Если $\exists \liml_{x \to x_0-0} f(x),\ \liml_{x \to x_0+0} f(x) \in \R$, то точка разрыва называется \textit{точкой разрыва первого рода}.
	
	В противном случае \textit{точкой разрыва второго рода}.
\end{definition}

\begin{definition}
	Если $\liml_{x \to x_0-0} f(x) = \liml_{x \to x_0+0} f(x) \in \R$ и $\neq f(x_0)$, то $x_0$ называется \textit{точкой устранимого разрыва}.
\end{definition}

\begin{definition}
	Если хотя бы 1 из односторонних пределов бесконечен, то $x_0$ называется \textit{точкой бесконечного разрыва}.
\end{definition}

\begin{note}
	Для простоты вводят обозначения:
	\begin{align*}
		f(x_0 - 0) := \liml_{x \to x_0-0} f(x)
		\\
		f(x_0 + 0) := \liml_{x \to x_0+0} f(x)
	\end{align*}
\end{note}

\begin{definition}
	Величину $\liml_{x \to x_0+0} f(x) - \liml_{x \to x_0-0} f(x)$ называется \textit{скачком функции} в точке $x_0$.
\end{definition}

\begin{example}
	$f(x) = \sgn x = \System{&{1,\ x > 0} \\ &{0,\ x = 0} \\ &{-1,\ x < 0}}$
	$$
		\liml_{x \to 0+0} f(x) - \liml_{x \to 0-0} f(x) = 1 - (- 1) = 2
	$$
\end{example}

\begin{example}
	$f(x) = \sgn^2 x\ \liml_{x \to 0} \sgn^2 x = 1 \neq \sgn^2 0 = 0$
\end{example}

\begin{example}
	$f(x) = \liml_{x \to -0} \frac{\dse 1}{\dse x} = -\infty$, но $f(x) = \liml_{x \to +0} \frac{\dse 1}{\dse x} = +\infty$
\end{example}

\begin{example}
	$f(x) = \sin \frac{\dse 1}{\dse x}$
	
	Рассмотрим $x'_n = \frac{1}{\frac{\pi}{2} + 2\pi n},\ n \in \N$ и
	
	$x''_n = \frac{1}{-\frac{\pi}{2} + 2\pi n}$
	
%%%%%%%%%%%%%%%%%%%%%%%%% ДОПИСАТЬ
\end{example}

\begin{theorem} (О точках разрыва монотонной функции)
	Если $f(x)$ монотонна на $(a; b),\ -\infty \le a < b \le +\infty$, то она может иметь на $(a; b)$ лишь точки разрыва 1го рода, причём неустранимого разрыва, и число таких точек разрыва не более чем счётно.
\end{theorem}

\begin{proof}
	$\forall x_0 \in (a; b)\ \exists \liml_{x \to x_0-0} f(x),\ \liml_{x \to x_0+0} f(x) \in \R$
	
	Считая $f$ неубывающей функцией, то 
	$$
		\forall x < x_0 \Ra f(x) \le f(x_0) \Ra f(x_0 - 0) \le f(x_0) \le f(x_0 + 0)
	$$
	$f(x_0 - 0) \neq f(x_0 + 0)$, иначе бы точки разрыва не было.
	
	$x_1 < x_2 \Ra f(x_1 - 0) < f(x_1 + 0) \le f(x_2 - 0) < f(x_2 + 0)$
	
	Отсюда $(f(x_1 - 0); f(x_1 + 0)) \cap (f(x_2 - 0); f(x_2 + 0)) = \emptyset$. То есть, мы получили систему непересекающихся отрезков на прямой действительных чисел, которая является не более чем счётным множеством (каждому отрезку можно сопоставить рациональное число внутри него).
\end{proof}

\begin{example} (Функция Римана)
	$$
		f(x) = \System{&{\frac{1}{n}, \text{ если } x = \frac{m}{n}} \\ &{0, \text{ если } x \in \R \bs \Q}}
	$$
	Докажем, что $f$ непрерывна в $x_0 \in \R \bs \Q$
	$$
		\forall \eps > 0\ \exists \delta := \min\left(\left|x_0 - \frac{\lfloor nx_0 \rfloor}{n}\right|, \left|x_0 - \frac{\lfloor nx_0 + 1 \rfloor}{n}\right|, \frac{1}{n} \ge \eps\right) > 0
	$$
	Это означает, что
	$$
		\forall x, |x - x_0| < \delta \Ra |f(x)| < \eps
	$$
\end{example}

\subsection{Непрерывность на множестве}

\begin{definition}
	Функция называется \textit{непрерывной на множестве} $X$, если $\forall x_0 \in X\ \ \forall \{x_n\} \subset X, \liml_{x \to \infty} x_n = x_0 \Ra \liml_{n \to \infty} f(x_n) = f(x_0)$
\end{definition}

\begin{theorem} (Первая теорема Вейерштрасса о непрерывных на отрезке функциях)
	Если $f$ непрерывна на $[a; b]$, то она ограничена на $[a; b]$
\end{theorem}

\begin{proof}
	Докажем от противного. Пусть $f$ - неограничена сверху (снизу аналогично). Это означает
	$$
		\forall \eps > 0\ \exists x_{\frac{1}{\eps}} \in [a; b]\ |\ f(x_{\frac{1}{\eps}}) > \frac{1}{\eps}
	$$
	Последовательно будем брать $\eps := 1, \frac{1}{2}, \dots, \frac{1}{n}, \dots$
	
	Получим $\{x_n\}_{n = 1}^\infty \subset [a; b],\ f(x_n) > n$. По теореме Больцано-Вейерштрасса
	$$
		\exists \{x_{n_k}\}_{k = 1}^\infty,\ \liml_{k \to \infty} x_{n_k} = x_0 \in [a; b]
	$$
	А из этого следует, что $\liml_{k \to \infty} f(x_{n_k}) = f(x_0)$, что неверно ($f(x_n) > n$).
\end{proof}