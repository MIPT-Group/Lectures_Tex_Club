\begin{theorem} (Свойства предела вектор-функций)
	\begin{enumerate}
		\item (Единственность предела) Если $\liml_{t \to t_0} \vec{a}(t) = \vec{l}_1$ и $\liml_{t \to t_0} \vec{a}(t) = \vec{l}_2$, то $\vec{l}_1 = \vec{l}_2$
		
		\item (Ограниченность предела) Если существует предел $\liml_{t \to t_0} \vec{a}(t)$, то существует $C > 0$ такое, что
		\[
			\exists r > 0 \such \forall t, 0 < |t - t_0| < r\ \ |\vec{a}(t)| < C
		\]
		
		\item (Отделимость от нуля) Если предел $\liml_{t \to t_0} \vec{a}(t) = \vec{a}_0 \neq \vec{0}$, то
		\[
			\exists r > 0 \such \forall t, 0 < |t - t_0| < r\ \ |\vec{a}(t)| > \frac{|\vec{a}_0|}{2}
		\]
	\end{enumerate}
\end{theorem}

\begin{proof}
	Во всех случаях пользуемся леммой \ref{vfeq} и переписываем задачу покоординатно: в каждом случае она уже решена.
\begin{enumerate}
	\item Из покоординатного равенства напрямую следует равенство $\vec{l}_1 = \vec{l}_2$
	
	\item Для каждой $a_i(t)$ существует нужная окрестность, в которой она ограничена:
	\begin{align*}
		&{\exists r_1 > 0 \such \forall t, 0 < |t - t_0| < r_1\ \ |a_1(t)| < C_1}
		\\
		\vdots
		\\
		&{\exists r_n > 0 \such \forall t, 0 < |t - t_0| < r_n\ \ |a_n(t)| < C_n}
	\end{align*}
	По определению
	\[
		|\vec{a}(t)| = \sqrt{|a_1(t)|^2 + \ldots + |a_n(t)|^2} < \sqrt{C_1^2 + \ldots + C_n^2} = C
	\]
	
	\item Аналогично предыдущему пункту
	\[
		|\vec{a}(t)| = \sqrt{|a_1(t)|^2 + \ldots + |a_n(t)|^2} > \sqrt{\left(\frac{|a_{0, 1}|}{2}\right)^2 + \ldots + \left(\frac{|a_{0, n}|}{2}\right)^2} = \frac{|\vec{a_0}|}{2}
	\]
\end{enumerate}
\end{proof}

\begin{theorem} (Арифметические свойства предела вектор-функций)
	\begin{enumerate}
		\item Если $\liml_{t \to t_0} \vec{a}(t) = \vec{a}_0$ и $\liml_{t \to t_0} \vec{b}(t) = \vec{b}_0$, то
		\[
			\liml_{t \to t_0} (\vec{a}(t) + \vec{b}(t)) = \vec{a}_0 + \vec{b}_0
		\]
		
		\item Если $\liml_{t \to t_0} \vec{a}(t) = \vec{a}_0$ и $\liml_{t \to t_0} \alpha(t) = \alpha_0$, то
		\[
			\liml_{t \to t_0} \alpha(t) \cdot \vec{a}(t) = \alpha_0 \cdot \vec{a}_0
		\]
		
		\item (Предел скалярного произведения) Если $\liml_{t \to t_0} \vec{a}(t) = \vec{a}_0$, $\liml_{t \to t_0} \vec{b}(t) = \vec{b}_0$, то
		\[
		\liml_{t \to t_0} (\vec{a}(t), \vec{b}(t)) = (\vec{a}_0, \vec{b}_0)
		\]
		
		\item (Предел векторного произведения) Если $n = 3$, $\liml_{t \to t_0} \vec{a}(t) = \vec{a}_0$, $\liml_{t \to t_0} \vec{b}(t) = \vec{b}_0$, то
		\[
		\liml_{t \to t_0} [\vec{a}(t), \vec{b}(t)] = [\vec{a}_0, \vec{b}_0]
		\]
	\end{enumerate}
\end{theorem}

\begin{proof}
	Все свойства доказываются через определение Гейне и уже доказанные свойства предела последовательностей в $\R^n$
\end{proof}

\begin{definition}
	Пусть $x_0 \in D$ - предельная точка множества $D \subset \trbr{X, \rho}$ и $f: D \to Y$. Тогда $f$ \textit{непрерывна} в точке $x_0$, если
	\[
		\liml_{x \to x_0} f(x) = f(x_0)
	\]
\end{definition}

\begin{definition} (Топологическое определение непрерывности на множестве)
	Функция $f: D \to Y,\ D \subset X$ называется \textit{непрерывной на множестве} $D$, если для любого открытого множества $G \subset Y$ его прообраз $f^{-1}(G) = \{x \in D \such f(x) \in G\}$ относительно открыт в $D$, то есть
	\begin{align*}
		&{\exists \widetilde{G} \subset X \text{ - открытое}}
		\\
		&{f^{-1}(G) = D \cap \widetilde{G}}
	\end{align*}
\end{definition}

\begin{definition} (Поточечное определение непрерывности на множестве)
	Функция $f: D \to Y, D \subset X$ называется \textit{непрерывной на множестве} $D$, если $\forall x_0 \in D$ такой, что $x_0$ - предельная точка $D$ верно 
	\[
		\liml_{x \to x_0} f(x) = f(x_0)
	\]
\end{definition}

\begin{theorem}
	Два определения непрерывности на множестве эквивалентны.
\end{theorem}

\begin{proof}
\begin{itemize}
	\item (1 $\Ra$ 2) Пусть $x_0 \in D$ и $x_0$ - предельная точка. При этом $f(x_0) \in Y$. Тогда
	\[
		\forall \eps > 0\ \ U_\eps(f(x_0)) \text{ - открытое множество}
	\]
	Согласно топологическому определению непрерывности
	\[
		f^{-1}(U_\eps(f(x_0))) = D \cap \widetilde{G},\ \widetilde{G} \text{ - тоже открытое}
	\]
	Значит \(x_0 \in f^{-1}(U_\eps(f(x_0))) \Ra x_0 \in \widetilde{G} \Ra \exists \delta > 0 \such U_\delta(x_0) \subset \widetilde{G}\). При этом
	\[
		\forall x \in U_\delta(x_0) \cap D \Ra \left(x \in U_\delta(x_0) \cap D \Ra x \in \widetilde{G} \cap D = f^{-1}(U_\eps(f(x_0)))\right)
	\]
	А это в итоге даёт, что \(f(x) \in U_\eps(f(x_0))\). Мы получили утверждение:
	\[
		\forall \eps > 0\ \exists \delta > 0 \such \forall x \in U_\delta(x_0) \cap D\ \ f(x) \in U_\eps(f(x_0))
	\]
	которое в точности является вторым определением непрерывности
	
	\item (2 $\Ra$ 1) Пусть $G$ - произвольное открытое множество в $Y$, $x \in f^{-1}(G)$. Если $x$ - изолированная точка множества $D$, то по определению это означает
	\[
		\exists \delta_x > 0 \such U_{\delta_x}(x) \cap D = \{x\}
	\]
	Теперь пусть $x$ - предельная точка множества $D$. Тогда по определению $\liml_{y \to x} f(y) = f(x)$. То есть
	\[
		\forall \eps > 0\ \exists \delta > 0 \such \forall y \in U_\delta(x) \cap D\ \ f(y) \in U_\eps(f(x))
	\]
	Раз $f(x) \in G$, то
	\[
		\exists \eps > 0 \such U_\eps(f(x)) \subset G
	\]
	Значит
	\[
		\exists \delta_x > 0 \such \forall y \in U_{\delta_x}(x) \cap D\ \ f(y) \in G
	\]
	Положим $\widetilde{G} := \bigcup\limits_{x \in f^{-1}(G)} U_{\delta_x}(x)$. Хочется доказать, что $f^{-1}(G) = D \cap \widetilde{G}$. Включение $\subset$ очевидно: если точка попадает в прообраз, то она попадает в $D$. Для обратного включения рассмотрим $\forall y \in D \cap \widetilde{G}$. То есть $y$ попало в $D$ и в какую-то из окрестностей $x$, которые образуют $\widetilde{G}$: если это была окрестность изолированной точки, то $y = x \Ra y \in f^{-1}(G)$. Иначе он попал в окрестность предельной точки, то $f(y) \in G \lra y \in f^{-1}(G)$.
\end{itemize}
\end{proof}

\begin{theorem} (Непрерывный образ компактного множества)
	Если $f: X \to Y$ непрерывна на компактном множестве $K \subset X$, то $f(K)$ компактно.
\end{theorem}

\begin{proof}
	Пусть $\bigcup\limits_{\alpha in A} G_\alpha$ - открытое покрытие множества $f(K)$. Согласно топологическому определению непрерывности
	\[
		\forall \alpha \in A\ \exists \widetilde{G}_\alpha \text{ - открытое в }X \such f^{-1}(G_\alpha) = K \cap \widetilde{G}_\alpha
	\]
	Если $x \in K$, то
	\[
		\exists \alpha \in A \such f(x) \in G_\alpha \lra x \in f^{-1}(G_\alpha) \Ra x \in \widetilde{G}_\alpha
	\]
	Отсюда $\bigcup\limits_{\alpha \in A} \widetilde{G}_\alpha \supset K$ и при этом $K$ - компактное множество
	\[
		\Ra \exists \{\alpha_1, \ldots, \alpha_N\} \subset A \such \bigcup\limits_{i = 1}^N \widetilde{G}_\alpha \supset K
	\]
	В итоге имеем, что для $\forall y \in f(K)$ существует $x \in K \such f(x) = y$. Следовательно
	\[
		\exists i \such x \in K \cap \widetilde{G}_{\alpha_i} = f^{-1}(G_{\alpha_i}) \Ra y \in G_{\alpha_i}
	\]
	То есть $f(K) \subset \bigcup\limits_{i = 1}^N G_{\alpha_i}$, мы выделили конечное открытое подпокрытие. Что и требовалось доказать.
\end{proof}

\begin{corollary} (Теорема Вейерштрасса в $\R^n$)
	Если $f: \R^n \to \R$ непрерывна на ограниченном замкнутом множестве $K \subset \R^n$, то она ограничена на $K$ и достигает своих верхней и нижней граней.
\end{corollary}

\begin{proof}
	По критерию компактности в $\R^n$ $K$ - компактное множество. По недавно доказанной теореме $f(K)$ - тоже компактно, при этом $f(K) \subset \R \Ra f(K)$ - ограниченное и замкнутое множество.
	
	Пусть $M := \sup\limits_{\vec{x} \in K} f(\vec{x})$. По определению точной верхней грани
	\[
		\forall m \in \N\ \exists \vec{x}_m \in K \such M - \frac{1}{m} < f(\vec{x}_m) \le M
	\]
	Соберём как раз такую последовательность $\{\vec{x}_m\} \subset K$. Из эквивалентного определения компактности:
	\[
		\exists \{\vec{x}_{m_k}\}_{k = 1}^\infty \such \liml_{k \to \infty} \vec{x}_{m_k} = \vec{x}_0 \in K
	\]
	По определению Гейне для непрерывности получаем, что
	\[
		\liml_{k \to \infty} f(\vec{x}_{m_k}) = f(\vec{x}_0) = M
	\]
\end{proof}

\subsection{Геометрия кривой в $\R^n$}

\begin{definition}
	\textit{Кривой} $\Gamma$ в пространстве $\R^n$ называется множество значений (\textit{годограф}) непрерывной вектор-функции $\vec{r}(t)$, $a \le t \le b$ (вместе с $\vec{r}(t)$)
\end{definition}

\begin{note}
	То есть кривая это не только множество точек, но и сама функция.
\end{note}

\begin{corollary}
	Согласно критерию компактности в $\R^n$, кривая - это ограниченное замкнутое множество.
\end{corollary}

\begin{definition}
	Кривая $\Gamma$ называется \textit{простой}, если верно утверждение
	\[
		\vec{r}(t_1) = \vec{r}(t_2) \Ra \left((t_1 = t_2) \vee (\{t_1, t_2\} = a, b)\right)
	\]
	То есть кривая без самопересечений
\end{definition}

\begin{definition}
	Кривая $\Gamma = \{\vec{r}(t) \such a \le t \le b\}$ называется \textit{замкнутой}, если $\vec{r}(a) = \vec{r}(b)$
\end{definition}

\begin{definition}
	Простая замкнутая кривая называется \textit{жордановой}.
\end{definition}

\begin{definition}
	\textit{Производной вектор-функции} называется предел (если он существует):
	\[
		\vec{r'}(t_0) := \liml_{\Delta t \to 0} \frac{\vec{r}(t_0 + \Delta t) - \vec{r}(t_0)}{\Delta t}
	\]
	При этом $t_0 \in (a; b)$ - область определения $\vec{r}(t)$.
\end{definition}

\begin{note}
	Далее принято соглашение, что $\vec{r}(t) = (x_1(t), \ldots, x_n(t))$
\end{note}

\begin{lemma}
	$\vec{r'}(t_0)$ существует тогда и только тогда, когда $x'_j(t_0)$ существует $\forall j \in \range{n}$
\end{lemma}

\begin{proof}
	\[
		\exists \vec{r'}(t_0) := \liml_{\Delta t \to 0} \frac{\vec{r}(t_0 + \Delta t) - \vec{r}(t_0)}{\Delta t} \lra \left(\forall j \in \range{n}\ \exists \liml_{\Delta t \to 0} \frac{x_j(t_0 + \Delta t) - x_j(t_0)}{\Delta t} = x'_j(t_0)\right)
	\]
\end{proof}

\begin{corollary}
	Чтобы взять производную от вектор-функции, нужно продифференцировать функцию каждой координаты, то есть
	\[
		\vec{r}(t_0) = (x'_1(t_0), \ldots, x'_n(t_0))
	\]
\end{corollary}

\begin{definition}
	Вектор-функция $\vec{r}: (a; b) \to \R^n$ называется \textit{дифференцируемой} в $t_0 \in (a; b)$, если вектор $\dvec{r}(t_0) = \vec{r}(t_0 + \Delta t) - \vec{r}(t_0)$ может быть записан в виде
	\[
		\dvec{r}(t_0) = \Delta t \vec{A} + \Delta t \cdot \vec{\eps}(\Delta t)
	\]
	где $\vec{A} \in \R^n$ и $\liml_{\Delta t \to 0} \vec{\eps}(\Delta t) = \vec{0}$
\end{definition}

\begin{note}
	Величину $\Delta t \cdot \vec{\eps}(\Delta t)$ естественно называть \textit{о-маленьким от} $\Delta t$:
	\[
		\Delta t \cdot \vec{\eps}(\Delta t) = \vec{o}(\Delta t)
	\]
	Сравнение функций можно определить аналогичным образом на вектор-функциях.
\end{note}

\begin{lemma}
	Вектор-функция $\vec{r}: (a; b) \to \R^n$ является дифференцируемой в $t_0 \in (a; b)$ тогда и только тогда когда существует производная $\vec{r'}(t_0)$. При этом $\vec{A} = \vec{r'}(t_0)$
\end{lemma}

\begin{proof}~
\begin{itemize}
	\item $\Ra$ $\vec{r}(t)$ - дифференцируема. Значит
	\[
		\frac{\dvec{r}(t_0)}{\Delta t} = \frac{\vec{r}(t_0 + \Delta t) - \vec{r}(t_0)}{\Delta t} = \vec{A} + \vec{\eps}(\Delta t) \xrightarrow[\Delta t \to 0]{} \vec{A}
	\]
	
	\item $\La$ $\exists \vec{r'}(t_0)$. По определению производной
	\[
		\vec{r'}(t_0) = \liml_{\Delta t \to 0} \frac{\vec{r}(t_0 + \Delta t) - \vec{r}(t_0)}{\Delta t}
	\]
	Это эквивалентно пределу $\liml_{\Delta t \to 0} \left(\frac{\vec{r}(t_0 + \Delta t) - \vec{r}(t_0)}{\Delta t} - \vec{r'}(t_0)\right) = 0$. То есть исходную дробь можно записать в виде:
	\[
		\frac{\vec{r}(t_0 + \Delta t) - \vec{r}(t_0)}{\Delta t} = \frac{\dvec{r}(t_0)}{\Delta t} = \vec{r'}(t_0) + \vec{\eps}(\Delta t)
	\]
	где $\vec{\eps}(\Delta t) \xrightarrow[\Delta t \to 0]{} 0$ $\Ra$ получили нужную запись приращения (нужно ещё на $\Delta t$ домножить).
\end{itemize}
\end{proof}

\begin{lemma}
	Если $\vec{r}(t)$ дифференцируема в точке $t_0$, то она непрерывна в $t_0$
\end{lemma}

\begin{proof}
	Раз функция дифференцируема, то приращение функции у точки $t_0$ записывается в виде
	\[
		\dvec{r}(t_0) = \vec{r}(t_0 + \Delta t) - \vec{r}(t_0) = \Delta t \cdot (\vec{A} + \vec{\eps}(\Delta t))
	\]
	Отсюда
	\[
		\liml_{\Delta t \to 0} \dvec{r}(t_0) = \liml_{\Delta t \to 0} \vec{r}(t_0 + \Delta t) - \vec{r}(t_0) = 0
	\]
	Применив равносильность имеем
	\[
		\liml_{\Delta t \to 0} \vec{r}(t_0 + \Delta t) = \vec{r}(t_0) = \liml_{t \to t_0} \vec{r}(t)
	\]
	Что и требовалось доказать.
\end{proof}