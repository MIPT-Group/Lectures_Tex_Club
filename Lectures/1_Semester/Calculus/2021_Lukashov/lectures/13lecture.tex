\subsection{Сравнение функций}

\begin{definition}
	Пусть $f(x) = \lambda(x) \cdot g(x)$.
	\begin{enumerate}
		\item Если $\lambda(x)$ ограничена в некоторой проколотой окрестности, то $f(x) = O(g(x))$ при $x \to a$
		
		\item Если $\liml_{x \to a} \lambda(x) = 0$, то $f(x) = o(g(x))$ при $x \to a$
		
		\item Если $\liml_{x \to a} \lambda(x) = 1$, то $f(x) \sim g(x)$ при $x \to a$. (эквивалентны)
	\end{enumerate}
\end{definition}

\begin{theorem}
	Если $g(x) \neq 0$ в некоторой проколотой окрестности точки $a$, то
	\begin{enumerate}
		\item $\frac{f(x)}{g(x)}$ ограничена в некоторой проколотой окрестности точки $a$ $\lra$ $f(x) = O(g(x))$ при $x \to a$
		
		\item $\liml_{x \to a} \frac{f(x)}{g(x)} = 0 \lra f(x) = o(g(x))$ при $x \to a$
		
		\item $\liml_{x \to a} \frac{f(x)}{g(x)} = 1 \lra f(x) \sim g(x)$ при $x \to a$
	\end{enumerate}
\end{theorem}

\begin{proof}
	Положим $\lambda(x) := \frac{f(x)}{g(x)}$. А дальше всё уже следует из сказанного выше.
\end{proof}

\begin{example}
	$\sin x \cdot \sin \frac{1}{x} = O(\sin \frac{1}{x})$ при $x \to 0$.
\end{example}

\begin{example}
	$x \cdot \sin \frac{1}{x} = o(\sin \frac{1}{x})$ при $x \to 0$.
\end{example}

\begin{theorem}
	$f(x) \sim g(x)$ при $x \to a \lra f(x) - g(x) = o(g(x))$ при $x \to a$.
\end{theorem}

\begin{proof}
	Положим $f(x) \sim g(x)$ при $x \to a$. Тогда
	$$
		f(x) = \lambda(x) \cdot g(x),\ \liml_{x \to a} \lambda(x) = 1
	$$
	Следовательно,
	$$
		f(x) - g(x) = (\lambda(x) - 1) \cdot g(x),\ \liml_{x \to a} (\lambda(x) - 1) = 0 \Ra f(x) - g(x) = o(g(x))
	$$
\end{proof}

\begin{theorem} (Использование эквивалентных при вычислении пределов)
	Если $f_1(x) \sim f_2(x)$ при $x \to a$, то
	$$
		\liml_{x \to a} f_1(x) \cdot g(x) = \liml_{x \to a} f_2(x) \cdot g(x)
	$$
	А также
	$$
		\liml_{x \to a} \frac{g(x)}{f_1(x)} = \liml_{x \to a} \frac{g(x)}{f_2(x)}
	$$
	при условии, что хотя бы один из пределов в каждом равенстве существует
\end{theorem}

\begin{proof}
	По условию
	$$
		f_1(x) = \lambda(x) \cdot f_2(x)
	$$
	Отсюда если существует предел одной, то автоматически существует предел и второй.
	
	Для дробей
	$$
		\frac{g(x)}{f_1(x)} = \frac{1}{\lambda(x)} \cdot \frac{g(x)}{f_2(x)}
	$$
\end{proof}

\begin{proposition}
	$\sin x \sim \tg x \sim (e^x - 1) \sim \ln(1 + x) \sim \sh x \sim \th x \sim \arcsin x \sim \arctg x \sim x$ при $x \to 0$
\end{proposition}

\begin{note}
	Мы можем писать $o(f) = O(f)$ и подобное, подразумевая, что на самом деле мы рассматриваем некоторое $g = o(f)$
\end{note}

\begin{definition}
	Если $f = O(g)$ и $g = O(f)$ при $x \to a$, то говорят, что
	$$
		f \asymp g
	$$
\end{definition}


\section{Дифференциальное исчисление функций одной переменной}

\subsection{Производная}

\begin{definition}
	Пусть $y = f(x)$ определена в некоторой окрестности точки $a \in \R$.
	
	\textit{Приращением} $\delta y$ этой функции в точке $a$, соответствующим приращению аргумента $\delta x$, называется $\delta y$ = $f(a + \delta x) - f(a)$.
	
	\textit{Производной} функции $y = f(x)$ в точке $a$ называется предел (если он существует и конечен)
	$$
		\liml_{\Delta x \to 0} \frac{\Delta y}{\Delta x} = \liml_{x \to a} \frac{f(x) - f(a)}{x - a} =: f'(a)
	$$
\end{definition}

\begin{theorem}
	Если функция имеет производную в точке $a$, то она непрерывна в этой точке.
\end{theorem}

\begin{proof}
	По условию,
	$$
		\liml_{x \to a} \frac{f(x) - f(a)}{x - a} = f'(a) + \alpha(x)
	$$
	Если совершить предельный переход, то увидим, что
	$$
		\liml_{x \to a} \alpha(x) = 0
	$$
	Следовательно,
	$$
		f(x) - f(a) = f'(a)(x - a) + \alpha(x)(x - a)
	$$
	при $x \to a$. Перенесём $f(a)$ в другую сторону и в предельном переходе получим, что
	$$
		\liml_{x \to a} f(x) = f(a) + 0 + 0 = f(a)
	$$
\end{proof}

\begin{theorem} (Арифметические операции и производные)
	Если $\exists f'(a)$ и $g'(a)$, то
	\begin{enumerate}
		\item $(f \pm g)'(a) = f'(a) \pm g'(a)$
		
		\item $(f \cdot g)'(a) = f'(a) \cdot g(a) + f(a) \cdot g'(a)$
		
		\item Если $g(a) \neq 0$, то $\left(\frac{f}{g}\right)'(a) = \frac{f'(a) \cdot g(a) - f(a) \cdot g'(a)}{g^2(a)}$
	\end{enumerate}
\end{theorem}

\begin{proof}~
	\begin{enumerate}
		\item 
		$$
			f'(a) = \liml_{x \to a} \frac{f(x) - f(a)}{x - a},\ g'(a) = \liml_{x \to a} \frac{g(x) - g(a)}{x - a}
		$$
		Отсюда
		$$
			\liml_{x \to a} \frac{(f(x) \pm g(x)) - (f(a) \pm g(a))}{x - a} = \liml_{x \to a} \frac{f(x) - f(a)}{x - a} \pm \liml_{x \to a} \frac{g(x) - g(a)}{x - a}
		$$
		
		\item Аналогично первому,
		\begin{multline*}
			\frac{(f \cdot g)(x) - (f \cdot g)(a)}{x - a} = \frac{f(x)g(x) - f(a)g(x)}{x - a} + \frac{f(a)g(x) - f(a)g(a)}{x - a} = \\
			\frac{f(x) - f(a)}{x - a} \cdot g(x) + f(a) \cdot \frac{g(x) - g(a)}{x - a}
		\end{multline*}
		Что в предельном переходе даёт
		$$
			f'(a) \cdot g(a) + f(a) \cdot g'(a).
		$$
		
		\item Аналогично первому и второму,
		$$
			\frac{(\frac{f}{g})(x) - (\frac{f}{g})(a)}{x - a} = \frac{f(x)g(a) - f(a)g(x)}{g(x)g(a)(x - a)} = \frac{f(x)g(a) - f(a)g(a)}{g(x)g(a)(x - a)} + \frac{f(a)g(a) - f(a)g(x)}{g(x)g(a)(x - a)}
		$$
		В предельном переходе получим
		$$
			\frac{g(a)f'(a)}{g^2(a)} - \frac{f(a)g'(a)}{g^2(a)}
		$$
	\end{enumerate}
\end{proof}

\begin{theorem} (Производные элементарных функций)
	Для всех $a$ из областей определения соответствующих функций справедливы равенства:
	\begin{enumerate}
		\item $(\sin x)' \big|_{x = a} = \cos a$
		\item $(\cos x)' \big|_{x = a} = -\sin a$
		\item $(\tg x)' \big|_{x = a} = \frac{1}{\cos^2 a}$
		\item $(\ctg x)' \big|_{x = a} = -\frac{1}{\sin^2 a}$
		\item $(x^b)' \big|_{x = a} = b \cdot a^{b - 1} (x > 0)$
		\item $(b^x)' \big|_{x = a} = b^a \ln b$
		\item $(\sh x)' \big|_{x = a} = \ch a$
		\item $(\ch x)' \big|_{x = a} = \sh a$
		\item $(\th x)' \big|_{x = a} = \frac{1}{\ch^2 a}$
		\item $(\cth x)' \big|_{x = a} = -\frac{1}{\sh^2 a}$
	\end{enumerate}
\end{theorem}

\begin{proof}
	Рутинно
	\begin{enumerate}
		\item $\liml_{x \to a} \frac{\sin x - \sin a}{x - a} = \liml_{x \to a} \frac{2 \sin \frac{x - a}{2} \cos \frac{x + a}{2}}{x - a} = \liml_{x \to a} \cos \frac{x + a}{2} = \cos a$, за счёт $\sin \frac{x - a}{2} \sim \frac{x - a}{2}$ при $x \to a$
		
		\item $\liml_{x \to a} \frac{\cos x - \cos a}{x - a} = \liml_{x \to a} \frac{-2 \sin \frac{x - a}{2} \sin \frac{x + a}{2}}{x - a} = \liml_{x \to a} (-\sin \frac{x + a}{2}) = -\sin a$
		
		\item Аналогично
		\item Аналогично
		\item 
		\begin{multline*}
		\liml_{x \to a} \frac{x^b - a^b}{x - a} = a^b \liml_{x \to a} \frac{\left(\frac{x}{a}\right)^b - 1}{x - a} = \\
		a^b \liml_{x \to a} \frac{e^{b \ln a} - 1}{x - a} = a^b \liml_{x \to a} \frac{b \ln \frac{x}{a}}{x - a} = \\
		a^b \cdot b \cdot \liml_{x \to a} \frac{\ln(1 + \frac{x - a}{a})}{x - a} = a^b \cdot b \cdot \liml_{x \to a} \frac{x - a}{a(x - a)} = b \cdot a^{b - 1}
		\end{multline*}
		Верно из-за $\ln (1 + t) \sim t, t \to 0$.
		
		\item
		$$
			\liml_{x \to a} \frac{b^x - b^a}{x - a} = b^a \liml_{x \to a} \frac{b^{x - a} - 1}{x - a} = b^a \cdot \ln b
		$$
		В силу одной из форм второго замечательного предела
		
		\item
		$$
			(\sh x)' \big|_{x = a} = \left(\frac{e^x - e^{-x}}{2}\right)' = \frac{1}{2}\left(e^a - \left(\frac{1}{e^x}\right)'\right) = \frac{1}{2}\left(e^a - \frac{e^{-a}}{e^{2a}}\right) = \frac{e^a + e^{-a}}{2} = \ch a
		$$
		
		\item Аналогично
		\item Аналогично
		\item $(\cth x)' \big|_{x = a} = \left(\frac{\ch x}{\sh x}\right)' = \frac{(\ch x)' \cdot \sh a - \ch a \cdot (\sh x)'}{\sh^2 a} = \frac{\sh^2 a - \ch^2 a}{\sh^2 a} = -\frac{1}{\sh^2 a}$
	\end{enumerate}
\end{proof}

\begin{theorem} (Производная обратной функции)
	Если $f(x)$ непрерывна и строго монотонна на $U_{\delta}(a),\ \delta > 0$ и $\exists f'(a) \neq 0$, то обратная функция $f^{-1}$ имеет производную в точке $f(a)$, равную
	$$
		(f^{-1})'(f(a)) = \frac{1}{f'(a)}
	$$
\end{theorem}

\begin{proof}
	Во первых, обратная функция определена, непрерывна и строго монотонна на интервале $f(U_{\delta}(a))$. Для краткости обозначим $\varphi = f^{-1}$. Рассмотрим $[a - \delta; a + \delta]$. Для определённости будем считать $f$ - возрастающей функцией. Тогда, $\varphi$ определена на $y \in [f(a - \delta); f(a + \delta)]$. По определению производной, нам надо найти предел
	\[
		\liml_{\Delta y \to 0} \frac{\varphi(f(a) + \Delta y) - \varphi(f(a))}{\Delta y}
	\]
	Обозначим
	\[
		\Delta x := \varphi(f(a) + \Delta y) - \varphi(f(a))
	\]
	Тогда
	\[
		\liml_{\Delta y \to 0} \Delta x = 0
	\]
	В силу непрерывности $\varphi(y)$. Дополнительно имеем, что
	\[
		a + \Delta x = \varphi(f(a) + \Delta y)
	\]
	То есть
	\[
		f(a + \Delta x) - f(a) = f(a) + \Delta y - f(a) = \Delta y
	\]
	Отсюда исходный предел выражается как
	\[
		\liml_{\Delta y \to 0} \frac{\varphi(f(a) + \Delta y) - \varphi(f(a))}{\Delta y} = \liml_{\Delta y \to 0} \frac{\Delta x}{\Delta y} = \liml_{\Delta y \to 0} \frac{1}{\frac{\Delta y}{\Delta x}} = \frac{1}{\liml_{\Delta x \to 0} \frac{\Delta y}{\Delta x}} = \frac{1}{f'(a)}
	\]
\end{proof}