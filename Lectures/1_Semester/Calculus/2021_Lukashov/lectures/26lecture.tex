\begin{lemma} \label{6lemma}
	$\vec{f} = (f_1, \ldots, f_n): [a; b] \to \R^n$ является функцией ограниченной вариации тогда и только тогда, когда $\forall j \in \range{n}$ функции $f_j$ являются функциями ограниченной вариации на $[a; b]$. При этом справедливо неравенство:
	\[
		V(f_j) \le V(\vec{f}) \le \suml_{i = 1}^n V(f_i)
	\]
\end{lemma}

\begin{proof}
	Начнём с замечания, что при вычислении вариации $\forall j \in \range{n}$:
	\[
		|f_j(t_k) - f_j(t_{k - 1})| \le |\vec{f}(t_k) - \vec{f}(t_{k - 1})| \le \suml_{i = 1}^n |f_i(t_k) - f_i(t_{k - 1})|
	\]
	Первое неравенство в цепочке следует из уже известного соотношения между модулем вектора и его координаты, а второе уже из того, чем является модуль вектора. Теперь просуммируем неравенства для всего разбиения $P: a = t_0 < t_1 < \ldots < t_N = b$:
	\[
		V(P, f_j) \le V(P, \vec{f}) \le \suml_{i = 1}^n V(P, f_i)
	\]
	
	Если $\vec{f}$ - функция ограниченной вариации, то для любого разбиения $P$ верно неравенство:
	\[
		\forall j \in \range{n}\ \ V(P, f_j) \le V(P, \vec{f}) \le V(\vec{f})
	\]
	То есть $\forall j \in \range{n}\ f_j$ - функция ограниченной вариации. При этом
	\[
		\forall j \in \range{n}\ \ V(f_j) \le V(\vec{f})
	\]
	как раз таки из-за верности при любом $P$
	
	Если же все $f_j$ являются функциями ограниченной вариации, то тогда для любого разбиения $P$ выполнено схожее неравенство:
	\[
		\forall j \in \range{n}\ \ V(P, \vec{f}) \le \suml_{i = 1}^n V(\vec{P}, f_i) \le \suml_{i = 1}^n V(f_i)
	\]
	Значит и $\vec{f}$ является функцией ограниченной вариации и выполнено утверждение
	\[
		V(\vec{f}) \le \suml_{i = 1}^n V(f_i)
	\]
\end{proof}

\begin{lemma} \label{7lemma}
	$\vec{r}: [a; b] \to \R^n$ является функцией ограниченной вариации на $[a; b]$ тогда и только тогда, когда $\vec{r}$ является функцией ограниченной вариации на $[a; c]$ и $[c; b]$ для $\forall c \in (a; b)$. При этом верно равенство:
	\[
		V(\vec{r}, [a; b]) = V(\vec{r}, [a; c]) + V(\vec{r}, [c; b])
	\]
\end{lemma}

\begin{proof}
	Зафиксируем $\forall c \in (a; b)$. Возьмём произвольное разбиение $P: a = t_0 < t_1 < \ldots < t_N = b$ и создадим из него 2 других разбиения:
	\begin{align*}
		&{P_1: a = t_0 < t_1 < \ldots < t_m \le c}
		\\
		&{P_2: c \le t_{m + 1}  < t_{m + 2} < \ldots < t_N}
	\end{align*}
	Теперь имеет место следующее неравенство треугольника:
	\[
		|\vec{r}(t_{m + 1}) - \vec{r}(t_m)| \le |\vec{r}(c) - \vec{r}(t_m)| + |\vec{r}(t_{m + 1}) - \vec{r}(c)|
	\]
	Так как эти 3 модуля - единственное, в чём отличается $V(P, \vec{r})$ от $V(P_1, \vec{r}) + V(P_2, \vec{r})$, то верно неравенство:
	\[
		V(P, \vec{r}) \le V(P_1, \vec{r}) + V(P_2, \vec{r}) = V(\wt{P}, \vec{r})
	\]
	где $\wt{P} = P \cup \{c\}$. А теперь обратимся к полной вариации $\vec{r}$ на $[a; b]$:
	\[
		\forall \eps > 0\ \exists P \such V(\vec{r}, [a; b]) - \eps < V(P, \vec{r}) \le V(P_1, \vec{r}) + V(P_2, \vec{r}) = V(\wt{P}, \vec{r}) \le V(\vec{r}, [a; b])
	\]
	Так как неравенство имеет место для произвольного $P$, то
	\[
		\forall c \in (a; b)\ \forall \eps > 0\ \ V(\vec{r}, [a; b]) - \eps < V(\vec{r}, [a; c]) + V(\vec{r}, [c; b]) \le V(\vec{r}, [a; b])
	\]
	Значит, имеет место равенство
	\[
		V(\vec{r}, [a; c]) + V(\vec{r}, [c; b]) = V(\vec{r}, [a; b])
	\]
	
	Аналогично предыдущей лемме показывается, что если $\vec{r}$ имеет ограниченную вариацию на отрезках $[a; c]$ и $[c; b]$, то в силу неравенства
	\[
		\forall P\ \ V(P, \vec{r}) \le V(P_1, \vec{r}) + V(P_2, \vec{r})
	\]
	$\vec{r}$ будет функцией ограниченной вариации на всём $[a; b]$. Обратное утверждение имеет место из-за доказанного равенства вариаций.
\end{proof}

\begin{definition}
	Пусть $\vec{f}: [a; b] \to \R^n$ - функция ограниченной вариации. \textit{Функцией полной вариации} называется
	\[
		v_{\vec{f}}(x) := V(\vec{f}, [a; x])
	\]
\end{definition}

\begin{corollary} (из леммы \ref{7lemma})
	Функция полной вариации - неубывающая для $\forall \vec{f}$ ограниченной вариации
\end{corollary}

\begin{proof}
	Пусть $a \le x < y \le b$. Тогда по доказанному утверждению
	\[
		V(\vec{f}, [a; y]) = V(\vec{f}, [a; x]) + V(\vec{f}, [x; y])
	\]
	При этом все слагаемые неотрицательные. Значит
	\[
		V(\vec{f}, [a; y]) \ge V(\vec{f}, [a; x]) \lra v_{\vec{f}}(y) \ge v_{\vec{f}}(x)
	\]
\end{proof}

\begin{theorem} (Непрерывность функции полной вариации)
	Пусть $\vec{f}: [a; b] \to \R^n$ - функция ограниченной вариации. Тогда $\vec{f}$ непрерывна справа (слева) в точке $y \in [a; b)$ ($y \in (a; b]$) тогда и только тогда, когда $v_{\vec{f}}$ непрерывна справа (слева) в той же точке.
\end{theorem}

\begin{proof}
	Доказательство проведём для левой непрерывности
	\begin{itemize}
		\item $\Ra$ От противного. Пусть $v_{\vec{f}}$ не непрерывна слева в $y \in (a; b]$. При этом точно существует левый предел, так как $v_{\vec{f}}$ ограничена (ибо $\vec{f}$ - функция ограниченной вариации) и при этом $v_{\vec{f}}$ - неубывающая. Значит выполнено утверждение
		\[
			\liml_{x \to y-} v_{\vec{f}}(x) \neq v_{\vec{f}}(y) = V(\vec{f}, [a; y]) = V(\vec{f}, [a; x]) + V(\vec{f}, [x; y])
		\]
		То есть $V(\vec{r}, [x; y])$ не стремится к нулю при $x \to y-$. Это можно записать так:
		\[
			\exists \delta > 0 \such \forall x \in [a; y)\ V(\vec{f}, [x; y]) > \delta
		\]
		Посмотрим на $x := a$. Так как $V(\vec{f}, [x; y]) := \sup\limits_{P} V(P, \vec{f})$ где $P$ - разбиения $[x; y]$, то:
		\[
			\exists P \such V(P, \vec{f}) > \delta
		\]
		Распишем вариацию через сумму:
		\[
			V(P, \vec{f}) = \suml_{i = 1}^{m - 1} |\vec{f}(t_i) - \vec{f}(t_{i - 1})| + |\vec{f}(y) - \vec{f}(t_{m - 1})| > \delta
		\]
		Эту сумму можно считать некоторой функцией $h(y)$, которая является непрерывной слева в точке $y$ (из-за $\vec{f}$). Теперь дополнительно зафиксируем $a < a_1 < y$ (мы это можем сделать из-за свойства $\delta$ по определению выше):
		\[
			\exists a_1 \such a < a_1 < y,\ \ \suml_{i = 1}^{m - 1} |\vec{f}(t_i) - \vec{f}(t_{i - 1})| + |\vec{f}(a_1) - \vec{f}(t_{m - 1})| > \delta
		\]
		Эта же сумма соответствует разбиению $P_1: a = t_0 < t_1 < \ldots < t_{m - 1} < a_1 = \wt{t}_m$. То есть
		\[
			V(P_1, \vec{f}) > \delta \Ra V(\vec{f}, [a; a_1]) > \delta
		\]
		Теперь в качестве $x$ положим $a_1$ и повторим рассуждения рекурсивно. В итоге имеем:
		\begin{align*}
			&{\forall N \in \N\ \ a < a_1 < \ldots < a_N < y}
			\\
			&{\forall i \in \range{N}\ \ V(\vec{f}, [a; a_i]) > \delta}
		\end{align*}
		Из этого для полной вариации $\vec{f}$ на $[a; b]$ заключаем (считая $a_0 := a$):
		\[
			\forall N \in \N\ \ V(\vec{f}, [a; b]) \ge \suml_{i = 1}^N V(f, [a_{i - 1}; a_i]) > N\delta
		\]
		Значит $V(\vec{f}, [a; b])$ - бесконечность. Противоречие.
		
		\item $\La$ Теперь $v_{\vec{f}}$ непрерывна слева в $y$. Отсюда следует, что
		\[
			\liml_{x \to y-} V(\vec{f}, [x; y]) = 0
		\]
		При этом мы можем рассмотреть разбиение отрезка $[x; y]$, состоящее только из точек $x$ и $y$. Тогда
		\[
			0 \le |\vec{f}(x) - \vec{f}(y)| \le V(\vec{f}, [x; y])
		\]
		В итоге имеем непрерывность $\vec{f}$ слева в точке $y$.
	\end{itemize}
\end{proof}

\begin{corollary}
	$\vec{f}: [a; b] \to \R^n$ - функция ограниченной вариации тогда и только тогда, когда
	\begin{align*}
		&{\vec{f} = (f_1, \ldots, f_n)}
		\\
		&{\forall j \in \range{n}\ \ f_j(t) = h_j(t) - g_j(t) + f_j(a)}
	\end{align*}
	где $h_j, g_j$ - неубывающие функции на $[a; b]$. При этом $h_j(a) = g_j(a) = 0$
\end{corollary}

\begin{note}
	Любая монотонная функция автоматом является функцией ограниченной вариации: внутри суммы мы можем раскрыть все модули однозначно и получить просто разность значений на концах.
\end{note}

\begin{proof}
	По лемме \ref{6lemma} уже имеем, что $\vec{f}$ - функция ограниченной вариации тогда и только тогда, когда $\forall j \in \range{n}\ f_j$ - функция ограниченной вариации. Зададим $h_j$ и $g_j$ системой уравнений:
	\[
		\System{
			&{f_j(t) - f_j(a) = h_j(t) - g_j(t)}
			\\
			&{v_{\vec{f}} = h_j(t) + g_j(t)}
		}
	\]
	Тогда $h_j$ и $g_j$ можно выписать так:
	\begin{align*}
		&{2h_j(t) = v_{f_j}(t) + f_j(t) - f_j(a)}
		\\
		&{2g_j(t) = v_{f_j}(t) - (f_j(t) - f_j(a))}
	\end{align*}
	Равенство $h_j(a) = g_j(a) = 0$ очевидно. Теперь докажем, что они неубывающие. Рассмотрим $a \le x < y \le b$:
	\begin{align*}
		&{2h_j(x) = v_{f_j}(x) + f_j(x) - f_j(a)}
		\\
		&{2h_j(y) = v_{f_j}(y) + f_j(y) - f_j(a)}
	\end{align*}
	Вычтем первое из второго. Получим
	\[
		2(h_j(y) - h_j(x)) = V(\vec{f}, [x; y]) + \left(f_j(y) - f_j(x)\right)
	\]
	При этом из доказательства теоремы выше известен факт
	\[
		0 \le |f_j(x) - f_j(y)| \le |\vec{f}(x) - \vec{f}(y)| \le V(\vec{f}, [x; y])
	\]
	А значит, знак выражения справа в равенстве полностью определяется первым слагаемым, которое положительно $\Ra$ $h_j(y) \ge h_j(x)$ - функция неубывающая. Аналогично доказывается $g_j$.
	
	В обратную сторону утверждение верно, так как сумма и разность двух функций ограниченной вариации - это также функция ограниченной вариации.
\end{proof}

\begin{note}
	Если $\vec{f}$ - непрерывная функция, то и все $h_j, g_j$ также непрерывны и наоборот.
\end{note}

\begin{lemma} (Признак спрямляемости)
	Если вектор-функция $\vec{r}$ дифференцируема на $[a; b]$ и её производная ограничена на $[a; b]$, то она является функцией ограниченной вариации на $[a; b]$, причём
	\[
		V(\vec{r}, [a; b]) \le \sup\limits_{t \in [a; b]} |\vec{r'}(t)| \cdot (b - a)
	\]
\end{lemma}

\begin{proof}
	Рассмотрим произвольное разбиение $P: a = t_0 < t_1 < \ldots < t_m = b$. Распишем вариацию на данном разбиении:
	\begin{multline*}
		V(P, \vec{r}) = \suml_{j = 1}^m |\vec{r}(t_j) - \vec{r}(t_{j - 1})| \le \suml_{j = 1}^m |\vec{r'}(c_j)|(t_j - t_{j - 1}) \le \sup\limits_{t \in [a; b]} |\vec{r'}(t)| \suml_{j = 1}^m (t_j - t_{j - 1}) =
		\\
		\sup\limits_{t \in [a; b]} |\vec{r'}(t)| \cdot (b - a)
	\end{multline*}
\end{proof}

\begin{example} (неспрямляемой кривой) (непрерывной функции, не являющейся функцией ограниченной вариации)
	\[
		\vec{r}(t) = (t, f(t))
	\]
	где $f(t)$ определена следующим образом:
	\[
		f(t) = \System {
			&{t\cos \frac{\pi}{t},\ 0 < t \le 1}
			\\
			&{0,\ t = 0}
		}
	\]
	Понятно, что спрямляемость $\vec{r}(t)$ будет определяться $f(t)$. Рассмотрим разбиение $P: 0 = t_0 < t_1 < \ldots < t_m = 1$, при этом $t_k = \frac{1}{m + 1 - k}, k \in \{0, \ldots, m\}$:
	\[
		V(P, f) = \left|\frac{1}{m} \cos \pi m\right| + \suml_{k = 2}^m \left|\frac{1}{k}\cos \pi k - \frac{1}{k - 1} \cos \pi (k - 1)\right| = 1 + 2\suml_{k = 2}^m \frac{1}{k}
	\]
	Уже доказывалось, что ряд $a_m = \suml_{k = 1}^m \frac{1}{k}$ расходится, то есть $V(P, f) \xrightarrow[m \to \infty]{} +\infty \Ra$ функция имеет неограниченную вариацию.
\end{example}