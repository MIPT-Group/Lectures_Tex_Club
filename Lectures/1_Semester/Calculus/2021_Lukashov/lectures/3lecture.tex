\begin{theorem}
    Если некоторое $r$ является рациональным числом, то оно представимо в виде периодической десятичной дроби и наоборот.
     $$
     r \in \Q \lra (r = \alpha_0,\alpha_1 \dots \alpha_k (\beta_1 \dots \beta_t)), \alpha_0 \in \N \cup \{0\}, \forall i > 0\ \alpha_i \in \{0, 1, \dots, 9\},\ \forall \beta_i \in \{0, 1, \dots, 9\}
     $$
\end{theorem}

\begin{proof}
    Пусть есть число $r \in \Q_+$ (не умаляя общности). Покажем, что оно представимо в виде периодической десятичной дроби:
    
    Пусть $[r]$ - целая часть числа $r$. Тогда понятно, что $[r] := \alpha_0$.
    Рассмотрим $r - [r] = \frac{m}{n}, 0 \le m < n$ (несложно доказать, что это верно. и по-другому быть не может).
    
    Далее возможно только 2 случая:
    $$
    \System{&{m = 0 \Ra r = \alpha_0,(0) \Ra \text{ периодическая десятичная дробь}} \\ &m \neq 0}
    $$
    Продолжим рассуждения для второго случая. Согласно свойству Архимеда,
    $$
    \exists p \in \N\ |\ 10^p \cdot m \ge n
    $$
    Если взять $p = p_{\min}$, то будет также выполнено
    $$
    10^{p - 1} \cdot m < n
    $$
    Рассмотрим $r_1 = \frac{10^p \cdot m}{n} = \psi_1,\alpha'_1 \dots$, где $\psi_1$ тоже цифра (это можно доказать из неравенств выше).
    
    Повторим алгоритм для $r_1$ и получим некоторое $m_1$. Если оно тоже не $0$, то продолжаем дальше.
    
    Если на каждом шаге $m_i \neq 0$, то сделав $n$ шагов, мы гарантированно получим период дроби. Действительно, если на каком-то из $n - 1$ шага у нас совпало текущее $m_i$ с некоторым $m_j, j < i$, то мы снова нашли период. А на $n$-м шаге $m_j$ покроют все числа от $1$ до $n - 1$. То есть, мы либо получим $0$, либо попадём в уже какое-то $m_j \Ra$ нашли период. $\Ra$ получили периодическую десятичную дробь, что и требовалось показать.
    
    Теперь покажем, что если есть периодическая десятичная дробь $\alpha_0, \alpha_1 \dots \alpha_k (\beta_1 \dots \beta_t)$, то она представима в виде $\frac{m}{n},\ m \in \Z, n \in \N$:
    
    Обозначим $r = \alpha_0, \alpha_1 \dots \alpha_k (\beta_1 \dots \beta_t)$, тогда
    \begin{align*}
        &r \cdot 10^{k + t}= \alpha_0 \alpha_1 \dots \alpha_k \beta_1 \dots \beta_t,(\beta_1 \dots \beta_t) \\
        &r \cdot 10^k= \alpha_0 \alpha_1 \dots \alpha_k,(\beta_1 \dots \beta_t) \\
        &r \cdot 10^k \cdot (10^t - 1) = \alpha_0 \alpha_1 \dots \alpha_k \beta_1 \dots \beta_t - \alpha_0 \alpha_1 \dots \alpha_k = [r \cdot 10^{k + t}] - [r \cdot 10^k] \\
        &\Ra r = \frac{[r \cdot 10^{k + t}] - [r \cdot 10^k]}{10^k \cdot (10^t - 1)}
    \end{align*}
    
    Числитель целое число, а знаменатель - натуральное $\Ra$ периодическая десятичная дробь представима как $\frac{m}{n}$, что и требовалось показать.
\end{proof}

\subsection{Действительные числа}


%%%\begin{definition}
%%%    \textit{Последовательностью рациональных чисел} называется отображение из $\N$ в $\Q$. 
%%%\end{definition}

\begin{definition}
    \textit{Рациональным отрезком} называется $\{r \in \Q\ |\ p \le r \le q\} := [p; q]_\Q$, где $p, q \in \Q$
\end{definition}

\begin{definition}
    \textit{Системой вложенных рациональных отрезков} называется 
    
    $\{[p_n;q_n]_\Q\}_{n = 1}^\infty$, если $\forall n \in \N\ [p_{n + 1}; q_{n + 1}]_\Q \subset [p_n; q_n]_\Q$
\end{definition}

\begin{definition}
    Систему вложенных рациональных отрезков называют \textit{стягивающейся}, если $\forall \veps \in \Q_+\ \ \exists N \in \N\ |\ \forall n > N\ q_n - p_n < \veps$
    
    Далее будем опускать слово \textit{вложенных} и писать вместо него \textit{стягивающихся}.
\end{definition}

\subsubsection{Отношение эквивалентности на множестве систем стягивающихся отрезков}

\begin{definition}
    Две системы стягивающихся рациональных отрезков $\{[p_n; q_n]_\Q\}_{n = 1}^\infty$ и $\{[r_n; s_n]_\Q\}_{n = 1}^\infty$ называются \textit{эквивалентными}, если:
    
    $\{[min(p_n, r_n); max(q_n, s_n)]_\Q\}_{n = 1}^\infty$ - тоже стягивающаяся последовательность
\end{definition}

\begin{proposition}
    Определение эквивалентности систем стягивающихся рациональных отрезков удовлетворяет всем свойствам отношения эквивалентности.
\end{proposition}

\begin{proof}
    Рефлексивность очевидна, так как $min(p_n, p_n) = p_n$ ну и $max(q_n, q_n) = q_n$, что эквивалентно изначальной стягивающейся последовательности.
    
    Симметричность тоже очевидна, так как минимум и максимум - инвариант относительно перестановки.
    
    Транзитивность
    $$
    \System{\{[p_n; q_n]_\Q\}_{n = 1}^\infty \sim \{[r_n; s_n]_\Q\}_{n = 1}^\infty \\ \{[r_n; s_n]_\Q\}_{n = 1}^\infty \sim \{[t_n; u_n]_\Q\}_{n = 1}^\infty} \Ra \{[p_n; q_n]_\Q\}_{n = 1}^\infty \sim \{[t_n; u_n]_\Q\}_{n = 1}^\infty
    $$
    Из условия следует, что
    \begin{align*}
        \{[min(p_n, r_n); max(q_n, s_n)]_\Q\}_{n = 1}^\infty \text{ - стягивающаяся, то есть } \\
        \forall \veps \in \Q\ \exists N_1 \in \N\ |\ \forall n > N_1\ max(q_n, s_n) - min(p_n, r_n) < \frac{\veps}{2} \\
        \{[min(r_n, t_n); max(s_n, u_n)]_\Q\}_{n = 1}^\infty \text{ - стягивающаяся, то есть } \\
        \forall \veps \in \Q\ \exists N_2 \in \N\ |\ \forall n > N_2\ max(s_n, u_n) - min(r_n, t_n) < \frac{\veps}{2}
    \end{align*}
    Если положить $\forall n > max(N_1, N_2)$, то оба неравенства будут верны одновременно. Тогда, нужно доказать следующее:
    $$
        max(q_n, u_n) - min(p_n, t_n) < \veps
    $$
    Для этого рассмотрим по отдельности 4 случая:
    \begin{enumerate}
        \item $max(q_n, u_n) - min(p_n, t_n) = q_n - p_n$ \\ 
        Заметим, что $q_n - p_n \le  max(q_n, s_n) - p_n \le max(q_n, s_n) - min(p_n, r_n) < \veps$.
        Следовательно $max(q_n, u_n) - min(p_n, t_n) = q_n - p_n < \veps$
        \item $max(q_n, u_n) - min(p_n, t_n) = u_n - t_n$ - аналогично 1му случаю
        \item $max(q_n, u_n) - min(p_n, t_n) = q_n - t_n$ \\
        Заметим, что $q_n - t_n = (q_n - r_n) + (r_n - t_n)$. Далее $q_n - r_n \le max(q_n, s_n) - r_n \le max(q_n, s_n) - min(p_n, r_n) < \frac{\veps}{2}$. А также $r_n - t_n \le s_n - t_n \le max(s_n, u_n) - t_n \le max(s_n, u_n) - min(r_n, t_n) < \frac{\veps}{2}$.
        
        Складывая два выражения, получим:
        $max(q_n, u_n) - min(p_n, t_n) = q_n - t_n < \frac{\veps}{2} + \frac{\veps}{2} = \veps$
        \item $max(q_n, u_n) - min(p_n, t_n) = u_n - p_n$ - аналогично 3му случаю
    \end{enumerate}
    Таким образом, $\{[p_n; q_n]_\Q\}_{n = 1}^\infty = \{[t_n; u_n]_\Q\}_{n = 1}^\infty \Ra$ транзитивность верна.
\end{proof}

\begin{definition}
    \textit{Действительным числом} называется класс эквивалентности систем стягивающихся рациональных отрезков.
\end{definition}

\begin{proposition}
    Множество рациональных чисел вложено в множество действительных $\Q \subset \R$.
\end{proposition}

\begin{proof}
    Действительно, $r \in \Q \lra \{[r;r]_\Q\}_{n = 1}^\infty$ - система стягивающихся рациональных отрезков.
\end{proof}

\subsubsection{Сложение}

\begin{definition}
    \textit{Суммой двух действительных чисел} с представлениями \\$\{[p_n; q_n]_\Q\}_{n = 1}^\infty$ и $\{[r_n; s_n]_\Q\}_{n = 1}^\infty$ называется число с представлением $\{[p_n + r_n; q_n + s_n]_\Q\}_{n = 1}^\infty$
\end{definition}

\begin{proposition}
    Определение сложения двух действительных чисел корректно, то есть сложение не зависит от того, каких представителей классов эквивалентностей мы складываем.
    $$
    \System{\{[p_n; q_n]_\Q\}_{n = 1}^\infty \sim \{[p'_n; q'_n]_\Q\}_{n = 1}^\infty \\ 
            \{[r_n; s_n]_\Q\}_{n = 1}^\infty \sim \{[r'_n; s'_n]_\Q\}_{n = 1}^\infty}
    \Ra
    \{[p_n + r_n; q_n + s_n]_\Q\}_{n = 1}^\infty \sim \{[p'_n + r'_n; q'_n + s'_n]_\Q\}_{n = 1}^\infty
    $$
\end{proposition}

\begin{proof}
    По условию:
    \begin{align*}
        \forall \veps \in \Q_+\ \exists N_1 \in \N\ |\ \forall n > N_1\ \max(q_n, q'_n) - \min(p_n, p'_n) < \frac{\veps}{2} \\
        \forall \veps \in \Q_+\ \exists N_2 \in \N\ |\ \forall n > N_2\ \max(s_n, s'_n) - \min(r_n, r'_n) < \frac{\veps}{2}
    \end{align*}
    А нам необходимо показать, что
    $$
        \forall \veps \in \Q_+\ \exists N = \max(N_1, N_2) \in \N\ |\ \forall n > N\ \max(q_n + s_n, q'_n + s'_n) - \min(p_n + r_n, p'_n + r'_n)
    $$
    Покажем цепочку неравенств:
    \begin{multline}
        \max(q_n + s_n, q'_n + s'_n) - \min(p_n + r_n, p'_n, + r'_n) \le \\
        (\max(q_n, q'_n) + \max(s_n, s'_n)) - (\min(p_n, p'_n) + \min(r_n, r'_n)) = \\
        (\max(q_n, q'_n) - \min(p_n, p'_n)) + (\max(s_n, s'_n) - \min(r_n, r'_n)) < \frac{\veps}{2} + \frac{\veps}{2} = \veps
    \end{multline}
    Следовательно,
    $$
        \{[p_n + r_n; q_n + s_n]_\Q\}_{n = 1}^\infty \sim \{[p'_n + r'_n; q'_n + s'_n]_\Q\}_{n = 1}^\infty
    $$
\end{proof}

\subsubsection{Свойства сложения}

Для сложения действительных чисел верны следующие свойства:

$\forall x, y, z \in \R$
\begin{itemize}
    \item $x + y = y + x$ (коммутативность)
    \item $(x + y) + z = x + (y + z)$ (ассоциативность)
    \item $x + 0 = x$ (существование нейтрального элемента относительно сложения)
    \item $x + (-x) = 0$, где $-x$ - представитель $\{[-q_n; -p_n]_\Q\}_{n = 1}^\infty$ (существование обратного элемента относительно сложения)
\end{itemize}