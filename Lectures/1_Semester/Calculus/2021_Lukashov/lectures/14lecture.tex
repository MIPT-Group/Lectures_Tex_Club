\begin{corollary} (Производные обратных тригонометрических и логарифмических функций)
	Для всех $a$ из интервалов, входящих в область определения, справедливы равенства:
	\begin{align*}
		&(\arcsin x)' \big|_{x = a} = \frac{1}{\sqrt{1 - a^2}}
		\\
		&(\arccos x)' \big|_{x = a} = -\frac{1}{\sqrt{1 - a^2}}
		\\
		&(\arctg x)' \big|_{x = a} = \frac{1}{1 + a^2}
		\\
		&(\arcctg x)' \big|_{x = a} = -\frac{1}{1 + a^2}
		\\
		&(\log_b x)' \big|_{x = a} = \frac{1}{a \cdot \ln b},\ b \in (0; 1) \cup (1; +\infty)
	\end{align*}
\end{corollary}

\begin{proof}
	$$
		(\arcsin x)' \big|_{x = a} = \frac{1}{(\sin y) \big|_{y = \arcsin a}} = \frac{1}{\cos(\arcsin a)}
	$$
	Так как $\arcsin a \in (-\frac{\pi}{2}; \frac{\pi}{2})$, то
	$$
		(\arcsin x)' \big|_{x = a} = \frac{1}{\sqrt{1 - \sin^2(\arcsin a)}} = \frac{1}{\sqrt{1 - a^2}}
	$$
	
	$$
		(\arctg x)' \big|_{x = a} = \frac{1}{(\tg y)' \big|_{y = \arctg a}} = \cos^2 (\arctg a) = \frac{1}{\tg^2 (\arctg a) + 1} = \frac{1}{1 + a^2}
	$$
	С $\arcctg$ аналогично.
	$$
		(\log_b x)' \big|_{x = a} = \frac{1}{(b^y)' \big|_{y = \log_b a}} = \frac{1}{b^{\log_b a} \cdot \ln b} = \frac{1}{a \cdot \ln b}
	$$
\end{proof}

\begin{note}
	Предположение непрерывности функции в окрестности точки $a$ существенно.
\end{note}

\begin{example}
	Определим $y = f(x)$ как
	$$
		f\left(\frac{1}{n}\right) := \frac{1}{2n - 1},\ \forall n \in \N
	$$
	Будем считать, что
	$$
		f\left(\frac{1}{n} + 0\right) := f\left(\frac{1}{n}\right)
	$$
	При этом
	$$
		f\left(\frac{1}{n} - 0\right) := \frac{1}{2n}
	$$
	И дополнительно
	\begin{align*}
		&f(0) := 0
		\\
		&f(-x) := -f(x),\ \forall x \in (0; 1]
	\end{align*}
	Посчитаем предел $\liml_{\Delta x \to 0+} \frac{\Delta y}{\Delta x}$ в нуле.
	
	Рассмотрим случай $\Delta x \in [\frac{1}{n + 1}; \frac{1}{n})$. Тогда
	$$
		\Delta y = f(0 + \Delta x) - f(0)  = f(\Delta x)
	$$
	Отсюда
	$$
		\frac{1 / (2n + 1)}{1 / n} \le \frac{\Delta y}{\Delta x} \le \frac{1 / 2n}{1 / (n + 1)}
	$$
	В предельном переходе имеем
	$$
		\frac{1}{2} \le \liml_{\Delta x \to 0+} \frac{\Delta y}{\Delta x} \le \frac{1}{2}
	$$
	А отсюда
	$$
		f([-1; 1]) = [-1; 1] \bs \bigcup\limits_{n = 1}^{\infty} \left(\left(\frac{1}{2n}; \frac{1}{2n - 1}\right) \cup \left(-\frac{1}{2n - 1}; -\frac{1}{2n}\right)\right)
	$$
	То есть окрестность нуля не включена в область определения обратной функции $f^{-1}$ и мы не можем говорить о непрерывности и производной в нуле для неё.
\end{example}

\subsection{Дифференцируемость}

\begin{definition}
	Функция $y = f(x)$ называется \textit{дифференцируемой} в точке $a \in \R$, если её приращение в этой точке может быть записано в виде
	$$
		\Delta y = A\Delta x + o(\Delta x),\ \Delta x \to 0
	$$
	где $A \in \R$.
	
	Выражение $A\Delta x$ называется \textit{дифференциалом} функции $y = f(x)$ в точке $a$. Обозначается как $dy := A \Delta x$
\end{definition}

\begin{theorem}
	(Дифференцируемость и производная) Функция $y = f(x)$ дифференцируема в точке $a$ тогда и только тогда, когда она имеет производную в этой точке. При этом $A$ в точности равно $f'(a)$.
\end{theorem}

\begin{proof}
	Пусть $f$ дифференцируема в точке $a$. То есть
	$$
		\Delta y = A \Delta x + o(\Delta x),\ \Delta x \to 0
	$$
	Так как функция определена в окрестности нуля, то мы можем записать
	$$
		\frac{\Delta y}{\Delta x} = A + o(1),\ \Delta x \to 0
	$$
	Отсюда имеем
	$$
		\liml_{\Delta x \to 0} \frac{\Delta y}{\Delta x} = A = f'(a)
	$$
	В обратную сторону доказывается аналогично.
\end{proof}

\begin{corollary}
	Если $f$ дифференцируема в точке $a$, то она непрерывна в точке $a$.
\end{corollary}

\begin{note}
	Утверждение верно лишь в одну сторону
\end{note}

\begin{example}
	$y = |x|$. Тогда если рассмотреть $a = 0$
	$\liml_{\Delta x \to 0} \Delta y = 0,\ \Delta y = |\Delta x|$
	При этом рассмотрим односторонние пределы:
	\begin{align*}
		\liml_{\Delta x \to 0+} \frac{\Delta y}{\Delta x} = \liml_{\Delta x \to 0+} \frac{\Delta x}{\Delta x} = 1
		\\
		\liml_{\Delta x \to 0-} \frac{\Delta y}{\Delta x} = \liml_{\Delta x \to 0-} -\frac{\Delta x}{\Delta x} = -1
	\end{align*}
\end{example}

\begin{note}
	Если смотреть предел производной лишь с одной стороны, то можно говорить о правосторонней и левосторонней производной.
\end{note}

\begin{example}
	$$
		y = \System{&{x \sin \frac{1}{x},\ x \neq 0} \\ &{0,\ x = 0}}
	$$
	Здесь просто предела нет вообще
	$$
		\liml_{\Delta x \to 0} \frac{\Delta y}{\Delta x} = \liml_{\Delta x \to 0} \sin \frac{1}{\Delta x}
	$$
\end{example}

\begin{theorem} (Дифференцируемость сложной функции)
	Если $u = f(y)$ дифференцируема в точке $g(a)$, функция $y = f(x)$ дифференцируема в точке $a$, то композиция $u = h(x) = f(g(x))$ дифференцируема в точке $a$, причём $h'(a) = f'(g(a)) \cdot g'(a)$
\end{theorem}

\begin{proof}
	По условию
	$$
		\Delta u = f'(g(a))\Delta y + o(\Delta y),\ \Delta y \to 0
	$$
	Где 
	$$
		\Delta u = f(g(a) + \Delta y) - f(g(a))
	$$
	А также
	$$
		\Delta y = g'(a)\Delta x + o(\Delta x),\ \Delta x \to 0
	$$
	С другой стороны
	$$
		\Delta y = g(a + \Delta x) - g(a)
	$$
	Отсюда получим
	$$
		\Delta u = f(g(a) + g(a + \Delta x) - g(a)) - f(g(a)) = h(a + \Delta x) - h(a)
	$$
	Подставим всё в выражение $\Delta u$:
	$$
		\Delta u = f'(g(a))g'(a)\Delta x + f'(g(a)) \cdot o(\Delta x) + o(g(a + \Delta x) - g(a)),\ \Delta x \to 0
	$$
	В силу определения $o$-маленького
	\begin{multline*}
		o(g(a + \Delta x) - g(a)) = \lambda(\Delta x) \cdot \left(g(a + \Delta x) - g(a)\right) = \\
		\lambda(\Delta x)g'(a)\Delta x + \lambda(x)o(\Delta x) = o(\Delta x) + \lambda(x) \cdot o(\Delta x),\ \Delta x \to 0
	\end{multline*}
	В итоге имеем
	$$
		\Delta u = f'(g(a))g'(a)\Delta x + o(\Delta x),\ \Delta x \to 0
	$$
\end{proof}

\begin{note}
	По определению считается, что $dx := \Delta x$. Это можно также получить из функции $y = x$. Отсюда получаем, что
	$$
		dy = f'(a)dx,\ \ y' = \frac{dy}{dx} 
	$$
\end{note}

\begin{corollary} (Инвариантность формы первого дифференциала)
	Формула для дифференциала $dy = f'(a)dx$ справедлива как в случае, когда $x$ - независимая переменная, так и в случае, когда $x$ является функцией от другой переменной.
\end{corollary}

\begin{proof}
	Пусть $x = g(t),\ y = f(x)$. Тогда
	$$
		y = f(x) = f(g(t)) =: h(t)
	$$
	Положим $a = g(b)$.
	\begin{align*}
		&h'(b) = f'(a) \cdot g'(b)
		\\
		&dx = g'(b)dt
	\end{align*}
	Распишем $dy$:
	$$
		dy = h'(b) \cdot dt = f'(a) \cdot g'(h) dt = f'(a) dx
	$$
\end{proof}

\begin{definition}
	\textit{Касательной} к графику функции $y = f(x)$ в точке $a; f(a)$ называется предельное положение секущей, то есть прямой, проходящей через точки $(a; f(a))$ и $(a + \Delta x; f(a + \Delta x))$ при $\Delta x \to 0$
\end{definition}

\begin{proof}
	Уравнение секущей имеет вид
	$$
		\frac{y - f(a)}{x - a} = \frac{f(a + \Delta x) - f(a)}{\Delta x}
	$$
	То есть
	$$
		y = f(a) + \frac{f(a + \Delta x) - f(a)}{\Delta x} (x - a)
	$$
\end{proof}

\begin{definition}
	Если предел $\liml_{\Delta x \to 0} \frac{f(a + \Delta x) - f(a)}{\Delta x} = +\infty$ или $-\infty$ и $f(x)$ непрерывна в точке $a$, то будем говорить, что $f'(a)$ равна $+\infty$ или $-\infty$ соответственно.
\end{definition}

\begin{example}
	$f(x) = \sqrt[3]{x}$. Посчитаем $f'(0)$:
	$$
		f'(0) = \liml_{\Delta x \to 0} \frac{\sqrt[3]{\Delta x}}{\Delta x} = \liml_{\Delta x \to 0} \frac{1}{\sqrt[3]{(\Delta x)^2}} = +\infty
	$$
\end{example}

\begin{example}
	$f(x) = \sqrt[3]{|x|}$. Если посмотреть на график, то касательная в нуле вроде есть. Но предел будет
	$$
		\liml_{\Delta x \to 0} \frac{\Delta y}{\Delta x} = \infty
	$$
	Что не соответсвует нашему определению.
\end{example}

\begin{theorem} (Геометрический смысл производной и дифференциала)
	Пусть $f(x)$ непрерывна в некоторой окрестности точки $a$. Тогда, касательная к графику $y = f(x)$ в точке $(a; f(a))$ существует тогда и только тогда, когда существует предел $\liml_{\Delta x \to 0} \frac{\Delta y}{\Delta x} \in \bar{\R}$. 
	
	При этом уравнение касательной в случае дифференцируемости в точке $a$:
	$$
		y = f(a) + f'(a)(x - a)
	$$
	В случае бесконечной производной в точке $a$:
	$$
		x = a
	$$
	
	Дифференциал представляет приращение ординаты касательной, соответствующее приращению $\Delta x$.
\end{theorem}

%%%%%%%%%%%%%%%% Здесь нужен легендарный график производной

\subsection{Производные и дифференциалы высших порядков}

\begin{definition}
	Если рассмотреть производную как функцию и от неё можно тоже взять производную, то мы получим \textit{производную второго порядка}:
	$$
		f''(x) := (f'(x))' \big|_{x = a}
	$$
	Индуктивно определяется так:
	\begin{align*}
		&f^{(0)}(x) := f(x)
		\\
		&f^{(n)}(a) := (f^{(n - 1)}(x))' \big|_{x = a}
	\end{align*}
\end{definition}

\begin{example}
	\begin{align*}
		&{(\sin x)' = \cos x}
		\\
		&{(\cos x)' = -\sin x}
		\\
		&{(-\sin x)' = \cos x}
		\\
		\vdots
	\end{align*}
	Несложно понять, что
	$$
		(\sin x)^{(n)} = \sin (x + \frac{n\pi}{2}),\ n \in \N \cup \{0\}
	$$
	Доказательство по индукции
	$$
		(\sin x)^{(n + 1)} = \left((\sin x)^{(n)}\right)' = \left(\sin(x + \frac{n\pi}{2})\right)' = \cos(x + \frac{n\pi}{2}) = \sin\left(x + \frac{(n + 1)\pi}{2}\right)
	$$
\end{example}