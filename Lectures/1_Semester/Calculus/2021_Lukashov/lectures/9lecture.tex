\subsection{Предел функции}

\begin{definition}
	\textit{Проколотой $\delta$-окрестностью} точки $a \in \R$ называется множество
	$$
		\mc{U}_{\delta}(a) := (a - \delta; a) \cup (a; a + \delta)
	$$
\end{definition}

\begin{note}
	Далее, если не оговорено иного, будем считать, что $f : X \ra \R$, $X \subset \R$ определена в некоторой $\mc{U}_{\delta_0}(a) \subset X$, $\delta_0 > 0$
\end{note}

\begin{definition} (Предел по Коши)
	$$
		\liml_{x \to a} f(x) = A \lra \forall \eps > 0\ \exists \delta > 0\ |\ \forall x \in \mc{U}_{\delta}(a)\ f(x) \in U_{\eps}(A)
	$$
\end{definition}

\begin{definition} (Предел по Гейне)
	$$
		\liml_{x \to a} f(x) = A \lra \left(\forall \{x_n\} \subset X \bs \{a\}\ \liml_{n \to \infty} x_n = a\right)\ \liml_{n \to \infty} f(x_n) = A
	$$
\end{definition}

\begin{theorem}
	Определения предела функции по Коши и по Гейне эквивалентны.
\end{theorem}

\begin{proof}
	\begin{enumerate}
		\item (К $\Ra$ Г)
		
		Рассмотрим $\forall \{x_n\} \subset X \bs \{a\} \such \liml_{n \to \infty} x_n = a$. По определению предела
		$$
			\forall \delta > 0\ \exists N \in \N \such \forall n > N\ |x_n - a| < \delta
		$$
		Так как $\forall n \in \N\ x_n \in X \bs \{a\}$, то отсюда следует
		$$
			\forall \delta > 0\ \exists N \in \N \such \forall n > N\ x_n \in \mc{U}_{\delta}(a)
		$$
		По условию выполнено утверждение:
		$$
			\liml_{x \to a} f(x) = A \lra \forall \eps > 0\ \exists \delta > 0 \such \forall x \in \mc{U}_{\delta}(a)\ f(x) \in U_{\eps}(A)
		$$
		То есть для любого $\eps > 0$ найдётся $\delta > 0$, для которого верно 2 условия:
		$$
		\System{
			&{\exists N \in \N \such \forall n > N\ x_n \in \mc{U}_{\delta}(a)}
			\\
			&{\forall x \in \mc{U}_{\delta}(a)\ f(x) \in U_{\eps}(A)}
		}
		$$
		В итоге имеем:
		$$
			\forall \eps > 0\ \exists N \in \N \such \forall n > N\ f(x_n) \in U_{\eps}(A) \lra \liml_{n \to \infty} f(x_n) = A
		$$
		
		\item (Г $\Ra$ К)
		
		Докажем от противного, то есть при выполненности определения Гейне неверно определение Коши:
		
		$\exists \eps > 0\ |\ \forall \delta > 0\ \exists x \in \mc{U}_{\delta}(a)\ |\ f(x) \notin U_{\eps}(A)$
		
		Зафиксируем $\eps$ и подставим разные $\delta$:
		\begin{align*}
			&\delta := 1 & &{\exists x_1 \in \mc{U}_{1}(a)} & &{f(x_1) \notin U_{\eps}(A)}
			\\
			&\delta := 1/2 & &{\exists x_2 \in \mc{U}_{1/2}(a)} & &{f(x_2) \notin U_{\eps}(A)}
			\\
			&\dots & &\dots & &\dots
			\\
			&\delta := 1/n & &{\exists x_n \in \mc{U}_{1/n}(a)} & &{f(x_n) \notin U_{\eps}(A)}
			\\
			&\dots & &\dots & &\dots
		\end{align*}
		
		Получили последовательность $\{x_n\}_{n = 1}^\infty \such \forall n \in \N\ x_n \in \mc{U}_{1/n}(a),\ f(x_n) \notin U_{\eps}(A)$
		
		Но при этом, для этой последовательности верно утверждение:
		$$
			\forall \eps > 0\ \exists N \in \N \such \forall n > N\ x_n \in \mc{U}_{\eps}(a) \lra \liml_{n \to \infty} x_n = a
		$$
		А из определения предела по Гейне это будет означать, что
		$$
			\liml_{n \to \infty} f(x_n) = A \lra \forall \eps > 0\ \exists N \in \N \such \forall n > N\ f(x_n) \in U_{\eps} (A)
		$$
		Получили противоречие. (Потому что хотя бы для одного $\eps$, которое мы зафиксировали для последовательности, это выполнено не будет)
	\end{enumerate}
\end{proof}

\subsubsection*{Геометрический смысл предела функции}

$$
	\liml_{x \ra a} f(x) = A \lra \forall \eps > 0\ \exists \delta > 0\ |\ \forall x,\ 0 < |x - a| < \delta \Ra |f(x) - A| < \eps
$$

\begin{center}
 \begin{tikzpicture}[scale=1]
	% Axis
	\coordinate (y) at (0,5);
	\coordinate (x) at (6,0);
	\draw[<->] (y) node[above] {$y$} -- (0,0) --  (x) node[right] {$x$};
	\draw (-0.4,0) -- (0,0) --  (0,-0.4);
	
	\path
	coordinate (start) at (1,0.5)
	coordinate (c1) at +(2,1.7)
	coordinate (c2) at +(4,2.5)
	coordinate (top) at (4.8,3.6);
	
	\draw[style={thick}] plot [smooth, tension=0.6] coordinates {(start) (c1) (c2) (top)};
	
	\draw[style={dotted}] (c1) -- (2,0) node[circle, fill, inner sep = 1pt, label={below:$a - \delta$}] {};
	\draw[style={dotted}] (c2) -- (4,0) node[circle, fill, inner sep = 1pt, label={below:$a + \delta$}] {};
	\draw[style={dotted}] (3,2.1) -- (3,0) node[circle, fill, inner sep = 1pt, label={below:$a$}] {};
	
	\draw[style={dotted}] (c1) -- (0,1.7) node[circle, fill, inner sep = 1pt, label={left:$A - \eps$}] {};
	\draw[style={dotted}] (c2) -- (0,2.5) node[circle, fill, inner sep = 1pt, label={left:$A + \eps$}] {};
	\draw[style={dotted}] (3,2.1) -- (0,2.1) node[circle, fill, inner sep = 1pt, label={left:$A$}] {};
	
	\filldraw [black] (c1) circle (1.2pt);
	\filldraw [black] (c2) circle (1.2pt);
	\filldraw [black] (3, 2.1) circle (1.2pt);
	
	\draw (top) node[below right, black] {$y = f(x)$};
\end{tikzpicture}
\end{center}

\begin{example}
	Почему мы не берём сам предел в окрестность? А потому, что мы это используем при расчёте пределов:
	$$
		\liml_{x \ra 1} \frac{x^2 - 1}{x - 1} = \liml_{x \ra 1} \frac{(x - 1)(x + 1)}{x - 1} = \liml_{x \ra 1} (x + 1) = 2
	$$
	
	Проверка:
	$\forall \eps > 0\ \exists \delta > 0\ |\ \forall x,\ 0 < |x - 1| < \delta\ \ \left|\frac{x^2 - 1}{x - 1} - 2\right| < \eps$
	
	Примем $\delta := \eps$: $0 < |x - 1| < \eps \Ra \left|\frac{(x - 1)(x + 1)}{x - 1} - 2\right| = |x - 1| < \eps$
	
	Если мы бы допустили, что $a$ включено в $\delta$-окрестность, то никакое бы $\delta$ не подошло - для значения $x = a = 1$ было бы неверно, что $f(1) \in U_{\delta}(2)$
\end{example}

\begin{example} (Функция Дирихле)
	\[
		f(x) = \System{&{1, x \in \Q} \\ &{0, x \in \R \bs \Q}} = \mathbbm{1}
	\]
	Докажем, что $\forall a \in \R\ \not\exists x\ |\ \liml_{x \ra a} f(x) = A$:
	
	\begin{enumerate}
		\item $a \in \Q$
		
		$x'_n = a - \frac{1}{n} \in \Q \Ra f(x'_n) = 1$
		
		$x''_n = a - \frac{\sqrt{2}}{n} \in \R \bs \Q \Ra f(x''_n) = 0$
		
		\item $a \in \R \bs \Q$
		
		$x'_n = a - \frac{1}{n} \in \R \bs \Q \Ra f(x'_n) = 0$
		
		$x''_n = (a)_n \in \Q \Ra f(x''_n) = 1$ (десятичное представление $a$ до $n$-го знака)
	\end{enumerate}
\end{example}

\subsubsection*{Свойства предела функции, связанные с неравенствами}

\begin{enumerate}
	\item (Ограниченность) Если $\liml_{x \ra a} f(x) = A \in \R$, то $f(x)$ ограничена в некоторой проколотой окрестности точки $a$.
	
	\item (Отделимость от 0 и сохранение знака) Если $\liml_{x \ra a} f(x) = A \in \bar{\R}$, то $\exists C > 0$ такое, что в некоторой проколотой окрестности точки $a$ $|f(x)| \ge C$ и знак $f(x)$ тот же, что и у $A$.
	
	\item (Переход к пределу в неравенствах) Если 
	$$
	\System{
		&{\exists \delta > 0\ |\ \forall x \in \mc{U}_{\delta}(a)\ f(x) \le g(x)}
		\\
		&{\exists \liml_{x \ra a} f(x), \liml_{x \ra a} g(x) \in \bar{\R}}
	}
	$$
	то $A \le B$.
	
	\item (Теорема о трёх функциях) Если $\exists \delta > 0\ |\ \forall x \in \mc{U}_{\delta}(a),\ f(x) \le g(x) \le h(x)$ и $\liml_{x \ra a} f(x) = \liml_{x \ra a} h(x) = A \in \bar{\R}$, то $\liml_{x \ra a} g(x) = A$.
\end{enumerate}

\begin{proof}
\begin{enumerate}
	\item $\forall \eps > 0\ \exists \delta > 0\ |\ \forall x \in \mc{U}_{\delta}(a)\ |f(x) - A| < \eps$

	Рассмотрим $\eps := 1$: $\exists \delta > 0\ |\ \forall x \in \mc{U}_{\delta}(a)\ A - 1 < f(x) < A + 1$
	
	\item $\forall \eps > 0\ \exists \delta > 0\ |\ \forall x \in \mc{U}_{\delta}(a)\ f(x) \in U_{\eps}(A)$
	
	Первый случай $A = \pm \infty,\ \eps := 1 \Ra f(x) \in U_1 (\pm \infty) \Ra \sgn f(x) = \pm 1$
	
	Если же $A \in \R \bs \{0\}$, то $\eps := \frac{|A|}{2} > 0$, $f(x) \in U_{\eps}(A) \lra |f(x) - A| < \frac{|A|}{2}$
	
	Раскроем модуль: $A - \frac{|A|}{2} < f(x) < A + \frac{|A|}{2}$
	
	Если $A > 0 \Ra A - \frac{|A|}{2} = \frac{A}{2} > 0$. Иначе $A + \frac{|A|}{2} = \frac{A}{2} < 0$.
	
	\item Рассмотрим $\{x_n\}_{n = 1}^\infty\ |\ x_n \neq a, \liml_{n \ra \infty} x_n = a \text{ и } \liml_{n \ra \infty} f(x_n) = A,\ \liml_{n \ra \infty} g(x_n) = B$
	
	Так как последовательность сходится к $a$, то с какого-то номера $N \in \N$ она будет полностью в проколотой $\delta$-окрестности $a$. А для элементов из неё будет выполняться
	\[
		f(x_n) \le g(x_n)
	\]
	
	А значит, мы можем сделать предельный переход в неравенстве для последовательностей и получить
	\[
		\liml_{n \to \infty} f(x_n) \le \liml_{n \to \infty} g(x_n) \Ra A \le B
	\]
	
	\item Доказательство аналогично третьему, через предел по Гейне и теорему о трёх последовательностях
\end{enumerate}
\end{proof}

\subsubsection*{Свойства предела функции, связанные с арифметическими операциями}

Пусть $\liml_{x \ra a} f(x) = A,\ \liml_{x \ra a} g(x) = B,\ A, B \in \R$. Тогда
\begin{enumerate}
	\item $\liml_{x \ra a} (f \pm g)(x) = A \pm B$
	\item $\liml_{x \ra a} (f \cdot g)(x) = A \cdot B$
	\item Если $B \neq 0$, то $\liml_{x \ra a} \left(\frac{f}{g}\right)(x) = \frac{A}{B}$
\end{enumerate}

\begin{proof}
	Доказательство сводится к свойствам последовательностей. Небольшое отличие есть только в доказательстве третьего пункта:
	\[
		B \neq 0 \Ra \exists \delta > 0\ |\ \forall x \in \mc{U}_{\delta}(a)\ g(x) \neq 0
	\]
	Рассмотрим $\forall \{x_n\}_{n = 1}^\infty, x_n \neq a, \liml_{n \ra \infty} x_n = a$ \\
	Мы знаем, что $\System{&{\liml_{n \ra \infty} f(x_n) = A} \\ &{\liml_{n \ra \infty} g(x_n) = B}} \Ra \liml_{n \ra \infty} \frac{f(x_n)}{g(x_n)} = \frac{A}{B}$
\end{proof}

\subsubsection*{Критерий Коши существования предела функции}

\begin{theorem}
	$\exists \liml_{x \ra a} f(x) \in \R \lra \underbrace{\forall \eps > 0\ \exists \delta > 0\ |\ \forall x_1, x_2 \in \mc{U}_{\delta}(a)\ |f(x_1) - f(x_2)| < \eps}_{\text{\normalfont Условие Коши}}$
\end{theorem}

\begin{proof}
	Докажем необходимость: \\
	Из определения предела:
	\[
		\liml_{x \ra a} f(x) = A \in \R \lra \forall \eps > 0\ \exists \delta > 0\ |\ \forall x \in \mc{U}_{\delta}(a)\ |f(x) - A| < \frac{\eps}{2}
	\]
	По неравенству треугольника: $\forall x_1, x_2 \in \mc{U}_{\delta}(a)\ |f(x_1) - f(x_2)| \le |f(x_1) - A| + |A - f(x_2)| < \eps$
	
	Докажем достаточность: \\
	Рассмотрим $\forall \{x_n\} \subset X \bs \{a\} \such \liml_{n \to \infty} x_n = a$. Из определения предела:
	$$
		\liml_{n \to \infty} x_n = a \lra \forall \delta > 0\ \exists N \in \N \such \forall n > N\ x_n \in \mc{U}_{\delta}(a)
	$$
	Согласно этому утверждению и условию Коши, мы получаем
	$$
		\forall \eps > 0\ \exists N \in \N \such \forall n, m > N\ |f(x_n) - f(x_m)| < \eps
	$$
	Что в точности означает фундаментальность последовательности $f(x_n)$, то есть она сходящаяся.
\end{proof}