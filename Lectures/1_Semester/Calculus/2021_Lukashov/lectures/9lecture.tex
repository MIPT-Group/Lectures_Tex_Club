\section{Предел функции}

\begin{definition}
	\textit{Проколотой $\delta$-окрестностью} точки $a \in \R$ называется множество
	$$
		U_{\delta}(a) := (a - \delta; a) \cup (a; a + \delta)
	$$
\end{definition}

\begin{note}
	Будем считать, что $f : X \ra \R$, $X \subset \R$ определена в некоторой $U_{\delta}(a) \subset X$, $\delta > 0$
\end{note}

\begin{definition} (Предел по Коши)
	$$
		\liml_{x \ra a} f(x) = A \lra \forall \eps > 0\ \exists \delta > 0\ |\ \forall x \in U_{\delta}(a)\ f(x) \in U_{\eps}(A)
	$$
\end{definition}

\begin{definition} (Предел по Гейне)
	$$
		\liml_{x \ra a} f(x) = A \lra \forall \{x_n\} \subset X \bs \{a\}\ |\  \liml_{n \ra \infty} x_n = a \Ra \liml_{n \ra \infty} f(x_n) = A
	$$
\end{definition}

\begin{theorem}
	Определения предела функции по Коши и по Гейне эквивалентны.
\end{theorem}

\begin{proof}
	\begin{enumerate}
		\item (К $\Ra$ Г)
		
		Распишем определение Коши для каждого члена последовательности: $\forall \{x_n\} \subset X \bs \{a\}\ |\  \liml_{n \ra \infty} x_n = a$
		
		Из предел равносилен утверждению: $\forall \delta > 0\ \exists N \in \N\ |\ \forall n > N\ x_n \in U_{\delta}(a)$
		
		Отсюда напрямую следует, что $f(x_n) \in U_{\eps}(A)$
		
		\item (Г $\Ra$ К)
		
		Докажем от противного:
		
		$\exists \eps > 0\ |\ \forall \delta > 0\ \exists x \in U_{\delta}(a)\ |\ f(x) \notin U_{\eps}(A)$
		
		Подставим разные $\delta$:
		\begin{align*}
			\delta := 1 & x_1 \in U_{1}(a) & f(x_1) \notin U_{\eps}(A)
			\\
			\delta := 1/2 & x_1 \in U_{1/2}(a) & f(x_2) \notin U_{\eps}(A)
			\\
			\dots & \dots & \dots
			\\
			\delta := 1/n & x_1 \in U_{1/n}(a) & f(x_n) \notin U_{\eps}(A)
		\end{align*}
		
		Получили последовательность $\{x_n\}_{n = 1}^\infty\ |\ \forall \eps > 0\ \exists N = [\frac{1}{\eps}] + 1\ |\ \forall n > N\ x_n \in U_{\eps}(a)$
		
		То есть $\liml_{n \ra \infty} x_n = a$
%%%%%%%%%%%%%%%%%%%%%%%%%%%% ДОПИСАТЬ
	\end{enumerate}
\end{proof}

\subsection{Геометрический смысл предела функции}

$$
	\liml_{x \ra a} f(x) = A \lra \forall \eps > 0\ \exists \delta > 0\ |\ \forall x\ 0 < |x - a| < \delta \Ra |f(x) - A| < \eps
$$

%%%%%%%%%%%%%% Тут нужна картиночка, я её сфоткал на телефон

\begin{example}
	Почему мы не берём сам предел в окрестность? А потому, что мы это используем при расчёте пределов:
	$$
		\liml_{x \ra 1} \frac{x^2 - 1}{x - 1} = \liml_{x \ra 1} \frac{(x - 1)(x + 1)}{x - 1} = \liml_{x \ra 1} (x + 1) = 2
	$$
	Проверка:
	$\forall \eps > 0\ \exists \delta > 0\ |\ \forall x,\ 0 < |x - 1| < \delta\ \ \left|\frac{x^2 - 1}{x - 1} - 2\right| < \eps$
	
	Примем $\delta := \eps$: $0 < |x - 1| < \eps \Ra $ всё остальное верно
\end{example}

\begin{example} (Функция Дирихле)
	$$
		f(x) = \System{&{1, x \in \Q} \\ &{0, x \in \R \bs \Q}} = \mathbb{1}
	$$
	Докажем, что $\forall a \in R\ \not\exists x\ |\ \liml_{x \ra a} f(x) = a$:
	
	\begin{enumerate}
		\item $a \in \Q$
		
		$x'_n = a - \frac{1}{n} \in \Q \Ra f(x'_n) = 1$
		
		$x''_n = a - \frac{\sqrt{2}}{n} \in \R \bs \Q \Ra f(x''_n) = 0$
		
		\item $a \in \R \bs \Q$
		
		$x'_n = a - \frac{1}{n} \in \R \bs \Q \Ra f(x'_n) = 1$
		
		$x''_n = (a)_n \in \Q \Ra f(x''_n) = 1$ (десятичное представление $a$ до $n$-го знака)
	\end{enumerate}
\end{example}

\subsection{Свойства предела функции, связанные с неравенствами}

\begin{enumerate}
	\item (Ограниченность) Если $\liml_{x \ra a} f(x) = A \in \R$, то $f(x)$ ограничена в некоторой проколотой окрестности точки $a$.
	
	\item (Отделимость от 0 и сохранение знака) Если $\liml_{x \ra a} f(x) = A \in \bar{\R}$, то $\exists C > 0$ такое, что в некоторой проколотой окрестности точки $a$ $|f(x)| \ge C$ и знак $f(x)$ тот же, что и у $A$.
	
	\item (Переход к пределу в неравенствах) Если $\exists \delta > 0\ |\ \forall x \in U_{\delta}(a)\ f(x) \le g(x)$ и $\exists \liml_{x \ra a} f(x), \liml_{x \ra a} g(x) \in \bar{\R}$, то $A \le B$.
	
	\item (Теорема о трёх функциях) Если $\exists \delta > 0\ |\ \forall x \in U_{\delta}(a),\ f(x) \le g(x) \le h(x)$ и $\liml_{x \ra a} f(x) = \liml_{x \ra a} h(x) = A \in \bar{\R}$, то $\liml_{x \ra a} g(x) = A$.
\end{enumerate}

\begin{proof}
\begin{enumerate}
	\item $\forall \eps > 0\ \exists \delta > 0\ |\ \forall x \in U_{\delta}(a)\ |f(x) - A| < \eps$
	
	Рассмотрим $\eps := 1$: $\exists \delta > 0\ |\ \forall x \in U_{\delta}(a)\ A - 1 < f(x) < A + 1$
	
	\item $\forall \eps > 0\ \exists \delta > 0\ |\ \forall x \in U_{\delta}(a)\ f(x) \in U_{\eps}(A)$
	
	Первый случай $A = \pm \infty,\ \eps := 1 \Ra f(x) \in U_1 (\pm \infty) \Ra \sgn f(x) = \pm 1$
	
	Если же $A \in R \bs \{0\}$, то $\eps := \frac{|A|}{2} > 0$, $f(x) \in U_{\eps}(A) \lra |f(x) - A| < \frac{|A|}{2}$
	
	Раскроем модуль: $A - \frac{|A|}{2} < f(x) < A + \frac{|A|}{2}$
	
	Если $A > 0 \Ra A - \frac{|A|}{2} = \frac{A}{2} > 0$. Иначе $A + \frac{|A|}{2} = \frac{A}{2} < 0$.
	
	\item Рассмотрим $\{x_n\}_{n = 1}^\infty\ |\ x_n \neq a, \liml_{n \ra \infty} x_n = a \text{ и } \liml_{n \ra \infty} f(x_n) = A,\ \liml_{n \ra \infty} g(x_n) = B$
	
	Из анализа последовательностей $f(x_n) \le g(x_n) \Ra A \le B$.
%%%%%%%%%%%%%%%%%%%%%%%%%%%%%%% ДОПИСАТЬ
	\item Доказательство аналогично третьему, через предел по Гейне и теорему о трёх последовательностях
\end{enumerate}
\end{proof}

\subsection{Свойства предела функции, связанные с арифметическими операциями}

Пусть $\liml_{x \ra a} f(x) = A,\ \liml_{x \ra a} g(x) = B,\ A, B \in \R$. Тогда
\begin{enumerate}
	\item $\liml_{x \ra a} (f \pm g)(x) = A \pm B$
	\item $\liml_{x \ra a} (f \cdot g)(x) = A \cdot B$
	\item Если $B \neq 0$, то $\liml_{x \ra a} \left(\frac{f}{g}\right)(x) = \frac{A}{B}$
\end{enumerate}

\begin{proof}
	Доказательство сводится к свойствам последовательностей. Небольшое отличие есть только в доказательстве третьего пункта:
	
	$B \neq 0 \Ra \exists \delta > 0\ |\ \forall x \in U_{\delta}(a)\ g(x) \neq 0$
	
	Рассмотрим $\forall \{x_n\}_{n = 1}^\infty, x_n \neq a, \liml_{n \ra \infty} x_n = a$
	
	Мы знаем, что $\System{&{\liml_{n \ra \infty} f(x_n) = A} \\ &{\liml_{n \ra \infty} g(x_n) = B}} \Ra \liml_{n \ra \infty} \frac{f(x_n)}{g(x_n)} = \frac{A}{B}$
\end{proof}

\subsection{Критерий Коши существования предела функции}

\begin{theorem}
	$\exists \liml_{x \ra a} f(x) \in \R \lra \forall \eps > 0\ \exists \delta > 0\ |\ \forall x \in U_{\delta}(a)\ |f(x_1) - f(x_2)| < \eps$
	
	Всё кванторное высказывание называется \textit{условием Коши}.
\end{theorem}

\begin{proof}
	Докажем необходимость: $\liml_{x \ra a} f(x) = A \in \R \lra \forall \eps > 0\ \exists \delta > 0\ |\ \forall x \in U_{\delta}(a)\ |f(x) - A| < \frac{\eps}{2}$
	
	Из неравенства треугольника: $|f(x_1) - f(x_2)| \le |f(x_1) - A| + |A - f(x_2)| < \eps$
	
	Докажем достаточность: $\forall \{x_n\}_{n = 1}^\infty \subset X \bs \{a\}, \liml_{n \ra \infty} x_n = a \Ra \liml_{n \ra \infty} f(x_n) = A \in \R$
	
	$\exists N \in \N\ |\ \forall n > N\ x_n \in U_{\delta}(a)$
	
	$\forall p \in \N \Ra |f(x_{n + p}) - f(x_n)| < \eps \Ra f(x_n)$ - фундаментальна $\Ra$ она сходится.
	
	Докажем, что пределы всех последовательностей будут одинаковы: $\{x'_n\} \subset X \bs \{a\},\ \liml_{n \ra \infty} x'_n = a$ и $\{x''_n\} \subset X \bs \{a\},\ \liml_{n \ra \infty} x''_n = a$.
	
	Положим $\liml_{n \ra \infty} f(x'_n) = A',\ \liml_{n \ra \infty} f(x''_n) = A''$, но при этом $A' \neq A''$
	
	Рассмотрим $\{x_n\}_{n = 1}^\infty := \{x'_1, x''_1, x'_2, x''_2, \dots\}$. Последовательность $\{x_n\}$ сходится к $a$, но при этом её $f(x_n)$ расходится $\Ra$ противоречие, предел единственен.
\end{proof}