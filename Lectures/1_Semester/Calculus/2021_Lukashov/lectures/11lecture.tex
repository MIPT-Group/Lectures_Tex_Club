\begin{theorem} (Вторая теорема Вейерштрасса о непрерывных на отрезке функциях) \\
	Если $f$ непрерывна на $[a; b]$, то она достигает своих точных верхней и нижней граней. То есть
	$$
		\exists x', x'' \in [a; b] \such f(x') = \inf\limits_{x \in [a; b]} f(x),\ f(x'') = \sup\limits_{x \in [a; b]} f(x)
	$$
\end{theorem}

\begin{proof}
	По определению минимума
	$$
		m := \inf\limits_{x \in [a; b]} f(x) \Ra \forall \eps > 0\ \exists x \in [a; b] \such m \le f(x) < m + \eps
	$$
	Построим подпоследовательность через выбор $\eps$:
	\begin{align*}
		&\eps := 1 & &m \le f(x_1) < m + 1
		\\
		&\eps := 1/2 & &m \le f(x_2) < m + 1/2
		\\
		&\dots & &\dots
		\\
		&\eps := 1/n & &m \le f(x_n) < m + 1/n
		\\
		&\dots & &\dots
	\end{align*}
	Получили ограниченную последовательность $\{x_n\}_{n = 1}^\infty$. По теореме Больцано-Вейерштрасса:
	$$
		\exists \{x_{n_k}\}_{k = 1}^\infty \subset \{x_n\}_{n = 1}^\infty \such \liml_{k \to \infty} x_{n_k} = x' \in [a; b] 
	$$
	Так как для $\forall n \in \N$ верно
	$$
		m \le f(x_n) < m + 1/n
	$$
	то в силу непрерывности $f$, можно совершить предельный переход:
	$$
		m \le f(x') \le m \lra f(x') = m
	$$
\end{proof}

\begin{theorem} (Больцано-Коши о промежуточных значениях) \\
	Если $f$ непрерывна на $[a; b]$, то $\forall c = f(x_1) < d = f(x_2)$, где $\{x_1, x_2\} \subset [a; b]$ $\forall u \in (c; d)\ \exists \gamma \in [a; b] \such f(\gamma) = u$
\end{theorem}

\begin{anote}
	В классической версии данной теоремы утверждается, что $\exists \gamma$ не просто в $[a; b]$, а в $[\min(x_1, x_2); \max(x_1, x_2)]$. Из доказательства лектора это следует.
\end{anote}

\begin{proof}
	Рассмотрим $c < u = 0 < d$. Положим $\{a_1, b_1\} := \{x_1, x_2\}$. Не умаляя общности будем считать $a_1 < b_1$. В силу условия имеем
	\[
		f(a_1) \cdot f(b_1) < 0
	\]
	Посмотрим на $f(\frac{a_1 + b_1}{2})$. Если оно равно 0, то мы нашли подходящее нам $\gamma := \frac{a_1 + b_1}{2}$. Иначе рассмотрим одну из половин $[a_2; b_2]$ изначального отрезка такую, что на её концах функция тоже принимает разные значения (то есть \(\{\frac{a_1 + b_1}{2}\} \subset \{a_2, b_2\}\))
	\[
		f(a_2) \cdot f(b_2) < 0
	\]
	Продолжим рассуждения рекурсивно. Если мы так и не дошли до конкретного $\gamma$, то мы получили систему вложенных отрезков $\{[a_n; b_n]\}_{n = 1}^\infty$. Несложно заметить, что
	\begin{align*}
		&{[a_n; b_n] \supset [a_{n + 1}; b_{n + 1}]}
		\\
		&{b_n - a_n = \frac{b_1 - a_1}{2^{n - 1}}}
	\end{align*}
	То есть полученная система - стягивающаяся. А по принципу Кантора вложенных отрезков это нам даёт
	\[
		\exists \{\gamma\} = \bigcap\limits_{n = 1}^\infty [a_n; b_n]
	\]
	Равенство имеет место, потому что $\liml_{n \to \infty} a_n = \liml_{n \to \infty} b_n = \gamma$ по построению.\\
	В силу непрерывности функции и принципа Кантора
	\[
		\forall n \in \N\ a_n \le \gamma \le b_n \Ra f(\gamma) = \liml_{n \to \infty} a_n = \liml_{n \to \infty} b_n
	\]
	Так как по построению $f(a_n) \cdot f(b_n) < 0$, то предельный переход даёт неравенство
	\[
		f^2(\gamma) \le 0 \lra f(\gamma) = 0
	\]
	
	При любом другом $u$ мы можем рассмотреть вспомогательную функцию $F(x) = f(x) - u$, для которой верно доказанное утверждение, а значит получим и нужное:
	\[
		F(\gamma) = 0 = f(\gamma) - u \Ra f(\gamma) = u
	\]
\end{proof}

\subsubsection*{Следствия свойств предела функции}

\begin{enumerate}
	\item (Ограниченность непрерывной функции) Если $f$ непрерывна в $x_0$, то она ограничена в некоторой окрестности точки $x_0$.
	
	\item (Отделимость от нуля и сохранение знака непрерывной функции) Если $f$ непрерывна в $x_0$ и $f(x_0) \neq 0$, то в некоторой окрестности $x_0$ $|f(x)| > \frac{|f(x_0)|}{2}$ и $\sgn f(x) = \sgn f(x_0)$.
	
	\item (Арифметические операции с непрерывными функциями) Если $f$ и $g$ непрерывны в $x_0$, то $f \pm g$, $f \cdot g$ и (если $g(x_0) \neq 0$) $\frac{f}{g}$ непрерывны в $x_0$.
\end{enumerate}

\begin{definition}
	Композицией функций $f$ и $g$ называется 
	$$
		(g \circ f)(x) := g(f(x))
	$$
\end{definition}

\begin{theorem} (Переход к пределу под знаком непрерывной функции) \\
	Если $\liml_{x \to a} f(x) = b$ и $g$ непрерывна в точке $b$, то $\liml_{x \to a} (g \circ f)(x) = g(b)$
\end{theorem}

\begin{proof}
	Рассмотрим $\left(\forall \{x_n\}_{n = 1}^\infty,\ x_n \neq a,\ \liml_{n \to \infty} x_n = a\right)\ \liml_{n \to \infty} f(x_n) = b$
	
	Положим $y_n := f(x_n)$
	$$
		\{y_n\}_{n = 1}^\infty,\ \liml_{n \to \infty} y_n = b \Ra \liml_{n \to \infty} g(y_n) = g(b) 
	$$
\end{proof}

\begin{addition} (Следствие теоремы выше. Непрерывность сложной функции)
	Если $f$ непрерывна в $a$, $g$ непрерывна в $f(a)$, то $g \circ f$ непрерывна в $a$.
\end{addition}

\begin{note} (Предел сложной функции)
	Для того, чтобы из $\liml_{x \to a} f(x) = b$ и $\liml_{y \to b} g(y) = l$ следовало $\liml_{x \to a} (g \circ f)(x) = l$, достаточно потребовать, чтобы $f(x) \neq b$ в некоторой проколотой окрестности точки $a$.
\end{note}

\subsubsection*{Промежутки}

\begin{definition}
	Множество $I \subset \bar{\R}$ называется \textit{промежутком}, если $\forall x_1 < x_2,\ \{x_1, x_2\} \subset I \Ra [x_1; x_2] \subset I$. Если таких $x_1, x_2$ не существует (то есть $I = \emptyset$ либо один элемент), то $I$ называется \textit{вырожденным} промежутком
\end{definition}

\begin{lemma}
	$I$ - невырожденный промежуток $\lra$ $\left(\exists a < b,\ \{a, b\} \subset \R\right)$ такие, что
	\[
		I = \left[
		\begin{aligned}
			&(-\infty; +\infty)
			\\
			&(-\infty; a)
			\\
			&(b; +\infty)
			\\
			&(-\infty; a]
			\\
			&[b; +\infty)
			\\
			&[a; b]
			\\
			&(a; b]
			\\
			&[a; b)
			\\
			&(a; b)
		\end{aligned}
		\right.
	\]
\end{lemma}

\begin{proof}
	Здесь приведено доказательство двух возможных случаев. Остальные - аналогично.
	
	Пусть $I$ - невырожденный промежуток, который неограничен сверху и ограничен снизу. Тогда $\exists a := \inf I$. Сам по себе $I$ может оказаться каким угодно, но точно верно, что (так как $I \subset \bar{\R}$)
	$$
		I \subset [a; +\infty)
	$$
	Выберем $\forall x_0 \in (a; +\infty)$. Из этого следует, что
	\begin{align*}
		\exists x_2 \in I \cap (x_0; +\infty) \text{, иначе } I \text{ ограничено сверху}
		\\
		\exists x_1 \in I \cap (a; x_0) \text{, иначе } \inf I \text{ определен неверно}
	\end{align*}
	А значит и $[x_1; x_2] \subset I$, то есть $x_0 \in I$. Следовательно,
	$(a;+\infty) \subset I$. Ну а из этого уже либо $I = (a; +\infty)$, либо $I = [a; +\infty)$, в зависимости от достижимости инфинума.
\end{proof}

\begin{lemma}
	Если $f$ непрерывна на промежутке $I$, то $f(I)$ - промежуток
\end{lemma}

\begin{proof}
	Будем считать, что $f$ - непостоянна. (Иной случай тривиален)
	
	Рассмотрим $\forall y_1 < y_2,\ \{y_1, y_2\} \subset f(I)$
	Это значит, что
	$$
		\exists x_1 < x_2,\ \{x_1, x_2\} \subset I \such \{f(x_1), f(x_2)\} = \{y_1, y_2\}
	$$
	Так как мы работаем с промежутком, то $[x_1; x_2] \subset I$, $f$ непрерывна на $[x_1; x_2] \Ra \forall c \in (y_1; y_2)\ \exists d \in (x_1; x_2) \such f(d) = c$ по теореме Больцано-Коши. А это означает, что
	$$
		[y_1; y_2] \subset f(I)
	$$
\end{proof}

\begin{lemma} \label{for_back}
	Пусть $f$ - строго монотонна на промежутке $I$. Тогда $f$ непрерывна на $I$ тогда и только тогда, когда $f(I)$ - промежуток.
\end{lemma}

\begin{proof}
	Нужно доказать только достаточность, остальное следует из предыдущей леммы.
	
	Пусть $f(I)$ - промежуток. Предположим, что $f$ - разрывная, $x_0$ - не концевая точка $I$ и $f$ не непрерывна в $x_0$. Будем считать, что $f$ - возрастающая. Тогда
	$$
		f(x_0 - 0) < f(x_0 + 0)
	$$
	Рассмотрим $\forall x < x_0 \Ra f(x) \le f(x_0 - 0)$. Аналогично $\forall x > x_0 \Ra f(x) \ge f(x_0 + 0)$. Отсюда следует, что
	\[
		f(I) \subset (-\infty; f(x_0 - 0)] \cup \{f(x_0)\} \cup [f(x_0 + 0); +\infty)
	\]
	Так как $x_0$ - не концевая точка $I$, то из каждого полуинтервала можно выбрать $y_1, y_2 \in f(I)$ соответственно. Но по определению промежутка $[y_1; y_2] \subset f(I)$, то есть $(f(x_0 - 0); f(x_0 + 0)) \in [y_1; y_2] \Ra (f(x_0 - 0); f(x_0 + 0)) \in f(I)$, что противоречит условию.
	
	Пусть теперь $x_0$ - концевая точка, например, $x_0 = \inf I$. Раз $f$ - возрастающая, то
	\[
		\exists f(x_0 + 0) > f(x_0)
	\]
	Получаем $f(I) \subset \{f(x_0)\} \cup [f(x_0 + 0); +\infty)$. Взяв $y_1$ и $y_2$ из разных частей, снова получим противоречие.
\end{proof}

\begin{definition}
	Если $f$ инъективна на $X$, то на $f(X)$ определена обратная функция $f^{-1}$:
	$$
		y = f(x) \lra x = f^{-1}(y)
	$$
\end{definition}

\begin{theorem} (Теорема об обратной функции)
	Если $f$ непрерывна и строго монотонна на промежутке $I$, то на промежутке $f(I)$ определена обратная функция $f^{-1}$, строго монотонная в том же смысле, что и $f$, и непрерывная на $f(I)$.
\end{theorem}

\begin{proof}
	Будем рассматривать такую $f$, что $\forall x_1 < x_2 \Ra f(x_1) < f(x_2)$. Положим
	\begin{align*}
		y_1 := f(x_1)
		\\
		y_2 := f(x_2)
	\end{align*}
	То есть
	\begin{align*}
		f^{-1}(y_1) := x_1
		\\
		f^{-1}(y_2) := x_2
	\end{align*}
	
	$f^{-1}(y_1) < f^{-1}(y_2)$, то есть $f^{-1}$ монотонно возрастает.
	
	По лемме \ref{for_back} $f(I)$ - промежуток. А значит, $f^{-1}$ определена на промежутке и при этом строго монотонна. Следовательно, по той же лемме, $f^{-1}$ - непрерывна на $f(I)$.
\end{proof}

\subsection{Непрерывность элементарных функций}

\begin{enumerate}
	\item $y = x^n,\ n \in \N,\ n$ - нечётное
	
	Возрастает на $(-\infty; +\infty)$, непрерывна
	
	Обратная: $f^{-1}(y) := \sqrt[n]{x}$
	
	\item $y = x^n,\ n \in \N,\ n$ - чётное
	
	Возрастает на $[0; +\infty)$, непрерывна
	
	$f^{-1}(y) = \sqrt[n]{x}$
	
	\item $y = x^r,\ r \in \Q$
	
	Определена и непрерывна на $(0; +\infty)$
\end{enumerate}

\subsubsection*{Тригонометрические функции}

\begin{lemma} \label{for_trig}
	$\forall x \in (0; \frac{\pi}{2}) \Ra \sin x < x < \tg x$
\end{lemma}

%%%%%%%%% Здесь нужна картинка, я её сфоткал

\begin{proof}
	$\sin x = MA,\ \tg x = BC$
	
	$S_{\triangle OMC} < S_{OMC} < S_{\triangle OBC}$, где $S_{\triangle OAM} = \frac{\sin x}{2}$, $S_{\triangle OMC} = \frac{x}{2}$, $S_{\triangle OBC} = \frac{\tg x}{2}$
\end{proof}