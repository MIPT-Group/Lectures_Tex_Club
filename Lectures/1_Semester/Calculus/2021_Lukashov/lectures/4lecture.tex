\subsubsection*{Положительное действительное число}

\begin{definition}
    Действительное число $a$ называется положительным ($a > 0$), если для любого представителя его класса систем стягивающихся рациональных отрезков найдётся такое $n \in \N$, что $p_n > 0$.
\end{definition}

\subsubsection*{Отношение строгого порядка на множестве действительных чисел}

\begin{definition}
    Действительное число $a$ состоит в отношении $>$ с числом $b$, если $a + (-b) > 0$. Для $<$ аналогично.
\end{definition}

\subsubsection*{Отношение порядка на множестве действительных чисел}

\begin{definition}
    Отношение порядка на множестве действительных чисел задаётся как
    $$
        a \le b := (a = b) \vee (a < b)
    $$
\end{definition}

\begin{proposition}
    Данное определение отношения порядка на $\R$ удовлетворяет всем условиям отношения порядка.
\end{proposition}

\begin{proof}
    Рефлексивность верна, так как $a = a$.
    
    Симметричность. По условию имеем:
    \begin{align*}
        &(a \le b) := (a < b) \vee (a = b) \lra (b - a > 0) \vee (a = b) \\
        &(b \le a) := (b < a) \vee (a = b) \lra (a - b > 0) \vee (a = b)
    \end{align*}
    Заметим, что случаи, где мы берём хотя бы 1 скобку с $<$ ($>$) приводят к закономерному противоречию $0 > 0$. Поэтому единственно возможным вариантом остаётся утверждение, что $a = b$.
    
    Транзитивность. По условию имеем:
    \begin{align*}
        &(a \le b) := (a < b) \vee (a = b) \lra (b - a > 0) \vee (a = b) \\
        &(b \le c) := (b < c) \vee (b = c) \lra (c - b > 0) \vee (b = c)
    \end{align*}
    Необходимо показать, что верно высказывание
    $$
        ((c - a > 0) \vee (a = c)) \equiv 1
    $$
    Так как одновременно из одного неравенства оказаться верной может только одна скобка, то рассморим все случаи:
    \begin{enumerate}
        \item $(a = b) \wedge (b = c) \Ra a = c$
        \item $(a = b) \wedge (c - b > 0) \Ra c - a > 0$
        \item $(b - a > 0) \wedge (b = c) \Ra c - a > 0$
        \item $(b - a > 0) \wedge (c - b > 0) \Ra (c - b) + (b - a) = c - a > 0$
    \end{enumerate}
\end{proof}

\begin{proposition}
    Множество $\R$ является линейно упорядоченным относительно отношения $\le$ ($\ge$)
\end{proposition}

\begin{note}
    Действительное число $\{[p_n; q_n]_\Q\}_{n = 1}^\infty \sim \{[p_n; q_n]_\Q\}_{n = m}^\infty$, где $m \in \N$
\end{note}

\begin{proof}
    Нам необходимо доказать, что для любых $a, b \in \R$ верно выражение:
    $$
        (a \le b) \vee (b \le a)
    $$
    Исходя из определения данного отношения, его можно переписать в виде
    $$
        (a < b) \vee (a = b) \vee (a > b)
    $$
    В силу корректности операции сложения и определения отношения $<$ ($>$), равносильной формой записи является
    $$
        (a - b < 0) \vee (a - b = 0) \vee (a - b > 0)
    $$
    Или же положим $c = a - b := \{[p_n; q_n]_\Q\}_{n = 1}^\infty$:
    $$
        (c < 0) \vee (c = 0) \vee (c > 0)
    $$
    Тогда, рассмотрим случай, когда $\forall n \in \N\ p_n \le 0 \le q_n$.
    Покажем, что в таком случае $\{[p_n;q_n]_\Q\}_{n = 1}^\infty \sim \{[0;0]_\Q\}_{n = 1}^\infty$
    
    По определению $\sim$ нужно проверить, что
    $$
        \forall \veps \in \Q_+\ \exists N \in \N\ |\ \forall n > N\ \max(q_n, 0) - \min(p_n, 0) < \veps
    $$
    Из условия следует, что $\max(q_n, 0) - \min(p_n, 0) = q_n - p_n < \veps$. А значит, эквивалентность верна и $c = 0$.
    
    В силу линейной упорядоченности $\Q$, остаются лишь случаи, когда $q_n < 0$ и $p_n > 0$. А они прямо по определению соответствуют случаям, когда $c < 0$ и $c > 0$ соответственно.
\end{proof}

\begin{lemma}
    Если $\{[p_n; q_n]_\Q\}_{n = 1}^\infty$ представляет число $c \in \R$, то $\forall n \in \N\ p_n \le c \le q_n$
\end{lemma}

\begin{proof}
    Предположим обратное, то есть
    $$
        \exists N_0 \in \N\ |\ \forall n_0 > N_0\ p_{n_0} > c
    $$
    Из этого следует, что
    $$
        p_{n_0} - c > 0
    $$
    Выражение слева является числом, поэтому сопоставим ему систему
    $$
        p_{n_0} - c := \{[r_n; s_n]_\Q\}_{n = 1}^\infty
    $$
    Так как $p_{n_0} - c > 0$, то по определению
    $$
        \exists N_1 \in \N\ |\ \forall n_1 > N_1\ r_{n_1} > 0
    $$
    А число $p_{n_0}$ будет представлять система
    $$
        p_{n_0} := \{[p_{n_0}; p_{n_0}]_\Q\}_{n = 1}^\infty
    $$
    Рассмотрим $n > \max(N_0, N_1)$. Тогда, разность $p_{n_0} - (p_{n_0} - c)$ с одной стороны, равна
    $$
        p_{n_0} - (p_{n_0} - c) = p_{n_0} - p_{n_0} + c = c
    $$
    А с другой стороны,
    $$
        p_{n_0} - (p_{n_0} - c) := \{[p_{n_0} - s_n; p_{n_0} - r_n]_\Q\}_{n = 1}^\infty
    $$
    Стало быть,
    $$
        \{[p_{n_0} - s_n; p_{n_0} - r_n]_\Q\}_{n = 1}^\infty \sim \{[p_n; q_n]_\Q\}_{n = 1}^\infty
    $$
    Выясним отношения между границами отрезков:
    $$
        p_{n_0} - s_n \le p_{n_0} - r_n < p_{n_0} \le p_n \le q_n
    $$
    А если системы эквивалентны, то
    $$
        \max(q_n, p_{n_0} - r_n) - \min(p_n, p_{n_0} - s_n) = q_n - p_{n_0} + s_n > 0
    $$
\end{proof}

\subsubsection*{Произведение}

\begin{definition}
    \textit{Произведением} двух положительных действительных чисел с представлениями $\{[p_n;q_n]_\Q\}_{n = 1}^\infty,\ p_1 > 0$ и $\{[r_n;s_n]_\Q\}_{n = 1}^\infty,\ r_1 > 0$ называют действительное число $\{[p_n \cdot r_n;q_n \cdot s_n]_\Q\}_{n = 1}^\infty$
\end{definition}

Доопределим произведение на всё множество $\R$:
\[
	a \cdot b := \System{
		&{a \cdot (-b),\ a > 0,\ b < 0}
		\\
		&{(-a) \cdot (-b),\ a < 0,\ b < 0}
	}
\]
А также $a \cdot 0 = 0 \cdot a = 0$, $\forall a \in \R$

\begin{proposition}
    Произведение двух действительных чисел определено корректно.
    $$
        \System{\{[p_n; q_n]_\Q\}_{n = 1}^\infty \sim \{[p'_n; q'_n]_\Q\}_{n = 1}^\infty \\ 
            \{[r_n; s_n]_\Q\}_{n = 1}^\infty \sim \{[r'_n; s'_n]_\Q\}_{n = 1}^\infty}
    \Ra
    \{[p_n \cdot r_n; q_n \cdot s_n]_\Q\}_{n = 1}^\infty \sim \{[p'_n \cdot r'_n; q'_n \cdot s'_n]_\Q\}_{n = 1}^\infty
    $$
\end{proposition}

\begin{anote}
    Стоит отметить, что $p'_1 > 0$, $q'_1 > 0$ (нам нужно доказать корректность основного определения)
\end{anote}

\begin{proof}
    Для начала покажем, что просто произведение положительных действительных чисел вообще является системой стягивающейся рациональных отрезков:
    \begin{align*}
        &\System{
        &p_n r_n \le q_n s_n \\ 
        &p_n \le q_n
        } 
        \Ra p_n r_n \le q_n r_n \le q_n s_n \\
        &q_n s_n - p_n r_n \le q_n (s_n - r_n) + r_n (q_n - p_n) \le q_1 (s_n - r_n) + r_1 (q_n - p_n)
    \end{align*}
    $\{[p_n; q_n]_\Q\}_{n = 1}^\infty$ - система стягивающихся рациональных отрезков
    $$
    \Ra \forall \veps \in \Q_+\ \exists N_1 \in \N\ |\ \forall n > N_1\ q_n - p_n < \frac{\veps}{2s_1}
    $$
    
    Аналогично для $\{[r_n; s_n]_\Q\}_{n = 1}^\infty$
    $$
    \Ra \forall \veps \in \Q_+\ \exists N_2 \in \N\ |\ \forall n > N_2\ s_n - r_n < \frac{\veps}{2q_1}
    $$
    
    $\Ra q_n s_n - p_n r_n \le q_1 (s_n - r_n) + r_1 (q_n - p_n) < q_1 \cdot \frac{\veps}{2q_1} + r_1 \cdot \frac{\veps}{2r_1} = \veps$, что и требовалось доказать.
    
    Далее покажем, что произведения разных представителей классов эквивалентны:
    \begin{multline}
        \max(q_n s_n, q'_n s'_n) - \min(p_n r_n, p'_n r'_n) \le \\
        \max(q_n, q'_n) \cdot \max(s_n, s'_n) - \min(p_n, p'_n) \cdot \min(r_n, r'_n) = \\
        \max(q_n, q'_n) \cdot \max(s_n, s'_n) - \max(s_n, s'_n) \cdot \min(p_n, p'_n) + \\
        \max(s_n, s'_n) \cdot \min(p_n, p'_n) - \min(p_n, p'_n) \cdot \min(r_n, r'_n) = \\
        \max(s_n, s'_n) \cdot (\max(q_n, q'_n) - \min(p_n, p'_n)) + \min(p_n, p'_n) \cdot (\max(s_n, s'_n) - \min(r_n, r'_n)) \le \\
        \max(s_1, s'_1) \cdot (\max(q_n, q'_n) - \min(p_n, p'_n)) + \max(p_1, p'_1) \cdot (\max(s_n, s'_n) - \min(r_n, r'_n))
    \end{multline}
    
    Из условия следует, что
    \begin{align*}
        \forall \veps \in \Q_+\ \exists N_1 \in \N\ |\ \forall n > N_1\ \max(q_n, q'_n) - \min(p_n, p'_n) < \frac{\veps}{2 \cdot \max(s_1, s
        _1)}
        \\
        \forall \veps \in \Q_+\ \exists N_2 \in \N\ |\ \forall n > N_2\ \max(s_n, s'_n) - \min(r_n, r'_n) < \frac{\veps}{2 \cdot \max(p_1, p
        _1)}
    \end{align*}
    
    А значит
    $$
        \max(q_n s_n, q'_n s'_n) - \min(p_n r_n, p'_n, r'_n) < \veps
    $$
\end{proof}

\subsubsection*{Свойства произведения}

Для произведения действительных чисел верны следующие свойства:

$\forall x, y, z \in \R$
\begin{itemize}
    \item $x \cdot y = y \cdot x$
    \item $x \cdot (y \cdot z) = (x \cdot y) \cdot z$
    \item $x \cdot 1 = x$
\end{itemize}

\subsubsection*{Обратное действительное число по произведению}

\begin{definition}
    Если $a \in \R,\ a > 0$, представимое как $\{[p_n; q_n]_\Q\}_{n = 1}^\infty,\ p_1 > 0$, то \textit{числом, обратным к данному}, называется $\frac{1}{a} = a^{-1} > 0$, равное $\{[\frac{1}{q_n};\frac{1}{p_n}]_\Q\}_{n = 1}^\infty$
\end{definition}

Данное определение дополняется для всех ненулевых действительных чисел как
$$
    a < 0 \Ra \frac{1}{a} = \frac{1}{(-a)}
$$

\begin{proposition}
    Определение обратного действительного числа по произведению корректно.
    $$
        \{[p_n; q_n]_\Q\}_{n = 1}^\infty \sim \{[p'_n; q'_n]_\Q\}_{n = 1}^\infty \Ra \left\{\left[\frac{1}{q_n}; \frac{1}{p_n}\right]_\Q\right\}_{n = 1}^\infty \sim \left\{\left[\frac{1}{s_n}; \frac{1}{r_n}\right]_\Q\right\}_{n = 1}^\infty
    $$
\end{proposition}

\begin{proof}
    Докажем, что такая система стягивающихся отрезков вообще будет стягиваться:
    $$
        \frac{1}{p_n} - \frac{1}{q_n} = \frac{q_n - p_n}{p_n \cdot q_n} \le \frac{q_n - p_n}{p^2_n} \le \frac{q_n - p_n}{p^2_1}
    $$
    По условию стягивания изначального числа:
    $$
        \forall \veps \in \Q_+\ \exists N \in \N\ |\ \forall n > N\ q_n - p_n < \veps * p^2_1
    $$
    $$
        \Ra \frac{1}{p_n} - \frac{1}{q_n} \le \frac{q_n - p_n}{p^2_1} < \frac{\veps \cdot p^2_1}{p^2_1} = \veps
    $$
    
    Теперь докажем, что определение не зависит от представителя класса:
    $$
        \System{\{[p_n; q_n]_\Q\}_{n = 1}^\infty \sim \{[p'_n; q'_n]_\Q\}_{n = 1}^\infty \\ 
            \{[r_n; s_n]_\Q\}_{n = 1}^\infty \sim \{[r'_n; s'_n]_\Q\}_{n = 1}^\infty}
    $$
\end{proof}