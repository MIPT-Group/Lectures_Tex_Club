\begin{definition}
	Последовательность $\{x_n\}_{n = 1}^\infty$ называется \textit{бесконечно малой}, если $\liml_{n \ra \infty} x_n = 0$ 
\end{definition}

\begin{theorem} (Предел произведения б.м. и ограниченной последовательностей)
	Если $\{x_n\}_{n = 1}^\infty$ - бесконечно малая, а $\{y_n\}_{n = 1}^\infty$ ограничена, то $\{x_ny_n\}_{n = 1}^\infty$ - бесконечно малая последовательность.
\end{theorem}

\begin{proof}
	$\{y_n\}_{n = 1}^\infty$ - ограниченная $\Ra \exists M > 0\ |\ \forall n \in \N\ |y_n| \le M$
	
	$\{x_n\}_{n = 1}^\infty$ - бесконечно малая $\Ra \forall \eps > 0\ \exists N \in \N\ |\ \forall n > N\ |x_n| < \frac{\eps}{M}
\end{proof}

\begin{definition}
	$\eps$-окрестностью числа $l \in \R$ называется $U_{\eps}(l) := (l - \eps, l + \eps)$. При этом $\eps > 0$.
\end{definition}

\begin{definition}
	Предел последовательности через $\eps$-окрестность определяется как $\liml_{n \ra \infty} x_n = l \lra \forall \eps > 0\ \exists N \in \N\ |\ \forall n > N\ x_n \in U_{\eps}(l)$
\end{definition}

\begin{definition}
	Отрицательной бесконечностью называется объект, для которого верно высказывание
	$$
		\forall x \in \R \Ra -\infty < x
	$$
\end{definition}

\begin{definition}
	Положительной бесконечностью называется объект, для которого верно высказывание
	$$
		\forall x \in \R \Ra x < +\infty
	$$
\end{definition}

\begin{definition}
	\textit{Расширенным действительным множеством} называется множество
	$$
		\bar{\R} = \R \cup \{-\infty, +\infty\}
	$$
\end{definition}

\begin{definition}
	$\eps$-окрестностью 
\end{definition}

\begin{definition}
	Последовательностью $\{x_n\}_{n = 1}^\infty$ называется бесконечно большой, если
	$$
		\liml_{n \ra \infty} x_n = -\infty, +\infty \text{ или } \infty
	$$
\end{definition}

\begin{theorem} (Связь б.м. и б.б. последовательностей)
	Если $x_n \neq 0\ \forall n \in \N$, то $\{x_n\}_{n = 1}^\infty$ - б.м. $\lra \{\frac{1}{x_n}\}_{n = 1}^\infty$ - б.б.
\end{theorem}

\begin{proof}
\begin{enumerate}
	\item $\{x_n\}_{n = 1}^\infty$ - б.м. $\Ra \forall \eps > 0\ \exists N \in \N\ |\ \forall n > N\ |x_n| < \eps$. Отсюда следует, что $\left|\frac{1}{x_n}\right| > \frac{1}{\eps} \lra \frac{1}{x_n} \in U_{\eps}(\infty)$
	
	\item $\liml_{n \ra \infty} \frac{1}{x_n} = +\infty \lra \forall \eps > 0\ \exists N \in \N\ |\ n > N\ \left|\frac{1}{x_n}\right| > \frac{1}{\eps} \Ra 0 < x_n < \eps \Ra |x_n| < \eps$
\end{enumerate}
\end{proof}

\begin{definition}
	Последовательность $\{x_n\}_{n = 1}^\infty$ называется \textit{неубывающей}, если
	$$
		\forall n \in \N\ x_n \le x_{n + 1}
	$$
\end{definition}

\begin{definition}
	Последовательность $\{x_n\}_{n = 1}^\infty$ называется \textit{невозрастающей}, если
	$$
		\forall n \in \N\ x_n \ge x_{n + 1}
	$$
\end{definition}

\begin{definition}
	Последовательность $\{x_n\}_{n = 1}^\infty$ называется \textit{убывающей}, если
	$$
		\forall n \in \N\ x_n > x_{n + 1}
	$$
\end{definition}

\begin{definition}
	Последовательность $\{x_n\}_{n = 1}^\infty$ называется \textit{неубывающей}, если
	$$
		\forall n \in \N\ x_n < x_{n + 1}
	$$
\end{definition}

\begin{theorem} (Вейерштрасса о монотонных последовательностях)
	Если $\{x_n\}_{n = 1}^\infty$ ограниченная и неубывающая последовательность, то $\exists \liml_{n \ra \infty} x_n = \sup \{x_n\}$. Если же невозрастающая, то $\liml_{n \ra \infty} x_n = \inf \{x_n\}$
\end{theorem}

\begin{proof}
	$l := \sup \{x_n\} \lra \System{\forall n \in \N\ x_n \le l \\ \forall \eps > 0\ \exists N \in \N\ l - \eps < x_N \le l}$
	
	Тогда, $\forall n > N \Ra l \ge x_n \ge x_{n - 1} \ge \dots \ge x_N > l - \eps \Ra |x_n - l| < \eps$
\end{proof}

\begin{addition}
	Каждая монотонная последовательность имеет предел в $\bar{\R}$
\end{addition}

\begin{proof}
	Пусть $\{x_n\}_{n = 1}^\infty$ - неубывающая неограниченная сверху $\Ra \forall \eps > 0\ \exists N \in \N\ |\ x_N > \frac{1}{\eps} \forall n > N\ x_n \ge x_{n - 1} \ge \dots \ge x_N > \frac{1}{\eps}$
\end{proof}

\begin{definition}
	Последовательность вложенных отрезков - это $\{[a_n; b_n]\}_{n = 1}^\infty$, $a_n \le b_n\ \forall n \in \N$ такая, что $\forall n \in \N\ [a_{n + 1}; b_{n + 1}] \subset [a_n; b_n]$
\end{defintion}

\begin{theorem} (Принцип Кантора вложенных отрезков)
	Каждая система вложенных отрезков имеет непустое пересечение.
\end{theorem}

\begin{proof}
	$[a_{n + 1}; b_{n + 1}] \subset [a_n; b_n] \Ra \left((a_n \le a_{n + 1}) \wedge (b_n \ge b_{n + 1})\right)$
	
	$a_n \le b_n \le b_1$
	
	$\exists a \liml_{n \ra \infty} a_n = \sup \{a_n\}$
	
	$\exists b = \liml_{n \ra \infty} b_n = \inf \{b_n\} \Ra b \le b_n$
	
%%%%%%%%%%%%%%%%%%%%%	ДОПИСАТЬ
\end{proof}

\begin{definition}
	Стягивающейся системой отрезков называется система вложенных отрезков, длины которых образуют б.м. последовательность.
\end{definition}

\begin{addition}
	Система стягивающихся отрезков имеет пересечение, состоящее из одной точки.
\end{addition}

\begin{proof}
	$a_n \le a \le b \le b_n \Ra 0 \le b - a \le b_n - a_n \Ra a = b$
\end{proof}

\begin{definition}
	Подпоследовательностью последовательности $\{x_n\}_{n = 1}^\infty$ называется $\{x_{n_k}\}_{k = 1}^\infty$, где $\{n_k\}_{k = 1}^\infty$ - возрастающая последовательность натуральных чисел
\end{definition}

\begin{definition}
	Частичным пределом последовательности $\{x_n\}_{n = 1}^\infty$ называется предел её подпоследовательности.
\end{definition}

\begin{theorem} (Эквивалентное определение частичного предела)
	Число $l \in \R$ является частичным пределом $\{x_n\}_{n = 1}^\infty$ тогда и только тогда, когда $\forall \eps > 0\ \forall N \in \N\ \exists n > N\ |\ |x_n - l| < \eps$
\end{theorem}

\begin{proof}
\begin{enumerate}
	\item $l$ - частичный предел. То есть $l = \liml_{n \ra +\infty} x_{n_k} \lra \forall \eps > 0\ \exists K \in \N\ |\ \forall k > K\ |x_{n_k} - l| < \eps$
	
	$\forall N \in \N\ \exists K_1 \in \N\ |\ n_{K_1} > N \Ra \forall k > \max(K, K_1), $
	
%%%%%%%%%%%%%%%%%%%%%	ДОПИСАТЬ
\end{enumerate}
\end{proof}

\begin{theorem} (Больцано-Вейерштрасса)
	Из каждой ограниченной последовательности можно выделить сходящуюся подпоследовательность
\end{theorem}

\begin{proof}
	$\{x_n\}_{n = 1}^\infty$ - ограниченная, то есть $\exists [a_1; b_1] \supset \{x_n\}_{n = 1}^\infty$
	
	Разделим отрезок пополам. Утверждение: хотя бы 1 из половин содержит бесконечное число членов последовательности.
	
	Пусть $[a_2; b_2]$ - та из половин $[a_1; b_1]$, которая содержит бесконечно много членов последовательности $\{x_n\}$. Продолжая, получим последовательность вложенных отрезков $\{[a_n; b_n]\}_{n = 1}^\infty$. Так как $b_n - a_n = \frac{b_1 - a_1}{2^{n - 1}}$.
	
	Следовательно, $\{[a_n; b_n]\}_{n = 1}^\infty$ - стягивающаяся система, пусть $c = \bigcap\limits_{n = 1}^\infty [a_n; b_n]$. Докажем, что $c$ - частичный предел.
	
	$x_{n_1} = x_1\ ;\ x_{n_2} \in [a_2; b_2]\ ;\ \dots\ ;\ x_{n_k} \in [a_k; b_k]$. Отсюда $0 \le |c - x_{n_k}| \le \frac{b_1 - a_1}{2^{n - 1}} \ra 0$.
\end{proof}

\begin{addition}
	$\forall \{x_n\}_{n = 1}^\infty\ \exists \{x_{n_k}\}_{k = 1}^\infty\ \liml_{x \ra \infty} x_{n_k} = l \in \bar{\R}$
\end{addition}

\begin{proof}
	Предположим, $\{x_n\}_{n = 1}^\infty$ неограничена сверху. (Если ограничено, то смотрим теорему Больцано-Вейерштрасса)
%%%%%%%%%%%%%%%%%%%%%	ДОПИСАТЬ
\end{proof}

