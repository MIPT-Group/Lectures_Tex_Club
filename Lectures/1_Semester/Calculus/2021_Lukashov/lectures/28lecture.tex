\begin{definition}
	Давайте посмотрим, что из себя представляет $\frac{d\vec{\beta}}{ds}$:
	\[
		\frac{d\vec{\beta}}{ds} = \left[\frac{d\vec{\tau}}{ds}, \vec{\nu}\right] + \left[\vec{\tau}, \frac{d\vec{\nu}}{ds}\right] = \left[\frac{d^2\vec{r}}{ds^2}, \frac{1}{k} \cdot \frac{d^2\vec{r}}{ds^2}\right] + \left[\vec{\tau}, \frac{d\vec{\nu}}{ds}\right]
	\]
	Первое слагаемое обнуляется. Для второго вектора во втором слагаемом мы знаем, что он будет перпендикулярен самому $\vec{\nu} \Ra $ будет находиться в плоскости $\vec{\tau}$ и $\vec{\beta}$. То есть
	\[
		\exists \alpha_1, \alpha_2 \in \R \such \frac{d\vec{\nu}}{ds} = \alpha_1 \vec{\tau} + \alpha_2 \vec{\beta}
	\]
	Отсюда получаем, что
	\[
		\frac{d\vec{\beta}}{ds} = \alpha_2\left[\vec{\tau}, \vec{\beta}\right] = -\ae \vec{\nu}
	\]
	Коэффициент $\ae$ называется \textit{кручением пространственной кривой}. Его можно выразить как
	\begin{multline*}
		\ae = -\trbr{\frac{d\vec{\beta}}{ds}, \vec{\nu}} = -\frac{1}{k} \trbr{\left[\vec{\tau}, \frac{d\vec{\nu}}{ds}\right], \frac{d^2\vec{r}}{ds^2}} = \frac{1}{k^2} \trbr{\frac{d\vec{r}}{ds}, \frac{d^2\vec{r}}{ds^2}, \frac{d^3\vec{r}}{ds^3}} = \frac{1}{k^2} \cdot \frac{1}{|\vec{r'}|^6} \cdot \trbr{\vec{r'}, \vec{r''}, \vec{r'''}} =
		\\
		\frac{(\vec{r'}, \vec{r''}, \vec{r'''})}{|[\vec{r'}, \vec{r''}]|^2}
	\end{multline*}
\end{definition}

\begin{lemma} (Физический смысл кривизны и кручения)
	Если $\Gamma = \{\vec{r}(t)\}$ (т.е. любая) - трижды дифференцируемая кривая, то $k$ и $|\ae|$ - угловые скорости вращения векторов $\vec{\tau}$ и $\vec{\beta}$ соответственно.
\end{lemma}

\begin{proof}
	Пусть $\vec{a}(s)$ - вектор-функция такая, что $\forall s\ \ |\vec{a}(s)| = 1$. Распишем модуль производной этой функции:
	%%% Нарисовать. Внутрь доказательства поместить, наверное. Трансляция Мат.анализ 08.12 время 28:50
	\[
		\left|\frac{d\vec{a}}{ds}\right| = \liml_{\Delta s \to 0} \left|\frac{\vec{a}(s_0 + \Delta s) - \vec{a}(s_0)}{\Delta s}\right|
	\]
	В стремления к нулю $\Delta s$ справедливо приближение:
	\[
		|\Delta \vec{a}(s_0)| = |\vec{a}(s_0 + \Delta s) - \vec{a}(s_0)| = 2\left|\sin \frac{\Delta \phi}{2}\right|
	\]
	То есть предел на самом деле
	\[
		\left|\frac{d\vec{a}}{ds}\right| = \liml_{\Delta s \to 0} \left|\frac{\vec{a}(s_0 + \Delta s) - \vec{a}(s_0)}{\Delta s}\right| = \liml_{\Delta s \to 0} \left|\frac{2\sin \frac{\Delta \phi}{2}}{\Delta s}\right| = \liml_{\Delta s \to 0} \left|\frac{\Delta \phi}{\Delta s}\right|
	\]
	В итоге получаем, что производная по модулю совпадает со скоростью вращения единичного вектора, что и требовалось доказать.
\end{proof}

\begin{theorem} (Формулы Френе)
	Из проведённого анализа кривых можно выделить 3 формулы, носящие имя французского математика Жана Фредерика Френе.
	\begin{enumerate}
		\item \[
			\frac{d\vec{\tau}}{ds} = k\vec{\nu}
		\]
		
		\item \[
			\frac{d\vec{\nu}}{ds} = -k\vec{\tau} + \ae\vec{\beta}
		\]
		
		\item \[
			\frac{d\vec{\beta}}{ds} = -\ae\vec{\nu}
		\]
	\end{enumerate}
	Их все можно записать в матричном виде:
	\[
		\frac{d}{ds}\Matrix{\vec{\tau} \\ \vec{\nu} \\ \vec{\beta}} = \Matrix{0& & k& & 0 \\ -k& & 0& & \ae \\ 0& & -\ae& & 0} \cdot \Matrix{\vec{\tau} \\ \vec{\nu} \\ \vec{\beta}}
	\]
\end{theorem}

\begin{proof}
	Первая и последняя уже доказаны, осталось разобраться со второй:
	\[
		\frac{d\vec{\nu}}{ds} = \frac{d}{ds}[\vec{\beta}, \vec{\tau}] = \left[\frac{d\vec{\beta}}{ds}, \vec{\tau}\right] + \left[\vec{\beta}, \frac{d\vec{\tau}}{ds}\right] = -\ae [\vec{\nu}, \vec{\tau}] + k[\vec{\beta}, \vec{\nu}] = \ae \vec{\beta} - k\vec{\tau}
	\]
\end{proof}

\begin{definition}
	В случае плоской кривой ($\R^2$), у нас всего 2 формулы Френе:
	\begin{enumerate}
		\item \[
			\frac{d\vec{\tau}}{ds} = k\vec{\nu}
		\]
		
		\item \[
			\frac{d\vec{\nu}}{ds} = -k\vec{\tau}
		\]
	\end{enumerate}
	\textit{Эволютой} плоской кривой называется геометрическое место центров кривизны кривой. Исходная кривая называется \textit{эвольвентой} для эволюты.
\end{definition}

\begin{theorem} (Уравнения эволюты плоской кривой)
	Уравнения эволюты в случае $\R^2$ имеет вид:
	\[
		\vec{\rho}(s) = \xi(s)\vec{i} + \eta(s)\vec{j}
	\]
	\begin{align*}
		&{\xi = x - \frac{(x')^2 + (y')^2}{x'y'' - x''y'}y'}
		\\
		&{\eta = y + \frac{(x')^2 + (y')^2}{x'y'' - x''y'}x'}
	\end{align*}
\end{theorem}

\begin{proof}
	Движение вдоль кривой - это с точностью до бесконечно малых второго порядка движение в каждый момент по какой-то окружности. Радиус окружности - это радиус кривизны. Вооружившись полученными формулами, мы можем записать уравнение эволюты:
	\[
	\vec{\rho}(s) = \vec{r}(s) + R(s) \vec{\nu}(s)
	\]
	где $\vec{\rho}(s)$ - это радиус-вектор текущей касательной окружности.
	
	Подставим $\vec{\nu}$ в уже известное уравнение. Что получим?
	\[
		\vec{\rho}(s) = \vec{r}(s) + R^2(s) \cdot \frac{d^2\vec{r}}{ds^2}(s)
	\]
	При этом
	\[
		\frac{d^2\vec{r}}{ds^2} = \frac{\vec{r''}}{|\vec{r'}|^2} - \frac{1}{|\vec{r'}|^2} \frac{d}{ds}(|\vec{r'}|) \cdot \vec{r'} = \frac{\vec{r''}}{(s')^2} - \frac{s''}{(s')^3}\vec{r'} = \frac{(s')^2 \vec{r''} - s's''\vec{r'}}{(s')^4}
	\]
	Если $\vec{i}, \vec{j}$ - это орты плоскости, то тогда имеем
	\begin{align*}
		&{\vec{r} = x\vec{i} + y\vec{j}}
		\\
		&{\vec{r'} = x'\vec{i} + y'\vec{j}}
		\\
		&{\vec{s'} = \sqrt{(x')^2 + (y')^2}}
		\\
		&{s'' = \frac{x'x'' + y'y''}{\sqrt{(x')^2 + (y')^2}}}
	\end{align*}
	Подставим всё, что выписали в выражение выше:
	\begin{multline*}
		\frac{d^2\vec{r}}{ds^2} = \frac{\left((x')^2 + (y')^2\right)(x''\vec{i} + y''\vec{j}) - (x'x'' + y'y'')(x'\vec{i} + y'\vec{j})}{(s')^4} =
		\\
		\frac{(x''(y')^2 - x'y'y'')\vec{i} + ((x')^2y'' - x'x''y')\vec{j}}{(s')^4} = \frac{x'y'' - x''y'}{(s')^4}(-y'\vec{i} + x'\vec{j})
	\end{multline*}
	Теперь распишем $R^2$:
	\[
		R^2 = \frac{1}{k^2} = \frac{(s')^6}{(x'y'' - x''y')^2}
	\]
	где последнее было получено по лемме о выражении кривизны.
	
	Эволюту можно записать через орты как
	\[
		\vec{\rho}(s) = \xi(s)\vec{i} + \eta(s)\vec{j}
	\]
	Если подставим полученные для всех частей формулы в исходное векторное уравнение, то сможем выделить $\xi$ и $\eta$ в том виде, в котором они даны в теореме.
\end{proof}

\begin{theorem} (Свойства эволюты и эвольвенты)
	Если $\Gamma = \{\vec{r}(s), 0 \le s \le L\}$ - трижды непрерывнод дифференцируемая кривая с натуральным параметром $s$, причём $\frac{dR}{ds} \neq 0$, то
	\begin{enumerate}
		\item Касательная к эволюте в любой точке является нормалью к эвольвенте
		
		\item Длина дуги эволюты равна соответствующему изменению радиуса кривизны эвольвенты.
	\end{enumerate}
\end{theorem}

%%% Нарисовать. Тут точно стоит пояснение рисунком сделать, с эволютой/эвольвентой любой кривой

\begin{proof}	
	Для исходной кривой $s$ - это натуральный параметр, но для эволюты - совершенно не обязательно. Поэтому штрих будет обозначать $\frac{d}{ds}$:
	\[
		\vec{\rho'}(s) = \frac{d\vec{r}}{ds} + \frac{dR}{ds}(s) \cdot \vec{\nu}(s) + R(s) \cdot \frac{d\vec{\nu}}{ds}(s)
	\]
	Для удобства опустим $(s)$:
	\[
		\vec{\rho'} = \vec{\tau} + \frac{dR}{ds}\vec{\nu} - kR\vec{\tau} = \frac{dR}{ds} \vec{\nu}
	\]
	Этим мы доказали первый пункт: касательная к эволюте в любой точке выражается через $\vec{\nu}$ - нормаль к эвольвенте.
	
	Обозначим за $\sigma$ - натуральный параметр для эволюты. Тогда
	\[
		\sigma' = |\vec{\rho'}| = \left|\frac{dR}{ds}\right| = \pm \frac{dR}{ds}
	\]
	Последний переход имеет место из-за того, что мы имеем дело с непрерывной функцией, которая не обращается в 0. В итоге получили, что производная длины эволюты с точностью до знака совпадает во всех точках с производной радиуса кривизны. По теореме Лагранжа можем заключить, что
	\[
		\sigma(s_2) - \sigma(s_1) = \pm(R(s_2) - R(s_1))
	\]
	Это и требовалось доказать.
\end{proof}

\begin{note}
	Эвольвента - это развёртка эволюты. Представить себе это можно, если взять нить, натянуть её вдоль эволюты и постепенно отодвигать её от кривой: она будет идти по касательной и отклонение будет изменяться так же, как и радиус кривизны.
\end{note}