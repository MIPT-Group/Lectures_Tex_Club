\section{О числах}

\subsection{Натуральные числа}

\begin{definition}
    $Sc(n)$ - следующий элемент за $n$ (от английского слова \textit{successor} - преемник)
\end{definition}

\subsubsection*{Аксиомы Пеано}

\begin{definition}
    Аксиомы Пеано - это набор бездоказательных высказываний, позволяющих на своей основе построить всю систему натуральных чисел (т.е. определить все элементы, отношения и операции).
\end{definition}

\begin{enumerate}
    \item $\mathbf{1}$ есть натуральное число, то есть $\mathbf{1} \in \N$
    \item $\forall n \in \N\ \exists Sc(n) \in \N$ - для любого числа существует следующее за ним
    \item $\forall n \in \N\ 1 \neq Sc(n)$ - число $1$ не является чьим-либо преемником
    \item $Sc(n) = Sc(m) \Ra n = m$ - то есть $Sc$ инъективна
    \item (Аксиома индукции) $\forall \goth{M} \subset \N\ |\ (1 \in \goth{M}) \wedge (\forall n \in \goth{M} \Ra Sc(n) \in \goth{M}) \Ra \goth{M} = \N$
\end{enumerate}

\begin{example}
    Понятно, что данным аксиомам могут соответствовать разные модели. Одной из таких является модель Фреге-Рассела:
    
    $\{\emptyset\} := 1$, $Sc(n) := n \cup \{n\}$
    
    $\Ra \{\emptyset, \{\emptyset\}\} = 2$, $\{\emptyset, \{\emptyset\}, \{\emptyset, \{\emptyset\}\}\} = 3$
\end{example}

\subsubsection*{Сложение}

Чтобы определить операцию сложения, нам необходимо ввести 2 аксиомы:
\begin{enumerate}
    \item $m + 1 := Sc(m)$
    \item $m + Sc(n) := Sc(m + n)$
    \begin{note}
        Эта аксиома не интуитивна (ведь мы не определили $m + n$), но верна за счёт первой.
        
        База $n = 1$: $m + Sc(1) := Sc(m + 1)$ - всё верно и определено.
        Пусть мы доказали, что $m + Sc(n) = Sc(m + n)$. Тогда поймём, что и
        $m + Sc(Sc(n)) = Sc(m + Sc(n))$ - также определено и верно, в силу предыдущего шага.
        $\Ra$ аксиома верна и для любых $m$ и $n$ из натуральных. (Несмотря на наличие "доказательной"\ части, данное выражение остаётся аксиомой, иначе базу не обосновать)
    \end{note}
\end{enumerate}

Из этих двух аксиом следует, что операция сложения существует и единственна.

\subsubsection*{Умножение}

Чтобы определить операцию умножения, нам необходимо ввести 2 аксиомы:
\begin{enumerate}
    \item $m \cdot 1 := m$
    \item $m \cdot Sc(n) := m \cdot n + m$
\end{enumerate}

Из этих двух аксиом следует, что операция сложения существует и единственна.

\begin{example}
    Доказать: $2 \cdot 2 = 4$
    
    Что такое $2$? По определению, $2 := Sc(1)$. Аналогично $3 := Sc(2), 4 := Sc(3)$
    То есть, имеем:
    
    $2 \cdot 2 = 2 \cdot Sc(1) = 2 \cdot 1 + 2 = 2 + Sc(1) = Sc(2 + 1) = Sc(Sc(2)) = Sc(3) = 4$, что и требовалось доказать.
\end{example}

\subsubsection*{Отношение строгого порядка на множестве натуральных чисел(не материал лектора)}

\begin{definition}
    Отношение строгого порядка $<$ на множестве $\N$ определяется как
    \[
        (a < b) := (\exists p \in \N\ |\ a + p = b)
    \]
    Для $>$ аналогично.
\end{definition}

\subsubsection*{Отношение порядка на множестве натуральных чисел}

\begin{definition}
	Отношение порядка $\le$ на множестве $\N$ определяется как
	\[
		(a \le b) := \big((a = b) \vee (\exists p \in \N\ |\ a + p = b)\big)
	\]
	Для $\ge$ аналогично.
\end{definition}

\begin{anote}
    Отношение порядка также можно задать следующим образом:
    \[
    	(a \le b) := \big((a = b) \vee (a < b)\big)
    \]
\end{anote}

Из определения становится возможным доказать теорему, что отношения $\le$ и $\ge$ являются отношениями порядка. (Для этого необходимо доказать 3 свойства).

\subsubsection*{Свойства операций и отношений на натуральных числах}

Для $\forall m, n, p \in \N$ верно следующее:

\begin{itemize}
    \item Сложение
    \begin{itemize}
        \item $m + n = n + m$ (коммутативность)
        \item $(m + n) + p = m + (n + p)$ (ассоциативность)
    \end{itemize}
    \item Умножение
    \begin{itemize}
        \item $m \cdot n = n \cdot m$ (коммутативность)
        \item $(m \cdot n) \cdot p = m \cdot (n \cdot p)$ (ассоциативность)
    \end{itemize}
    \item Дистрибутивность сложения и умножения
    \[
    	m \cdot (n + p) = m \cdot n + m \cdot p
    \]
    \item Отношение порядка
    \begin{itemize}
        \item $m \le m$ (рефлексивность)
        \item $(m \le n) \wedge (n \le m) \Ra (m = n)$ (антисимметричность)
        \item $(m \le n) \wedge (n \le p) \Ra (m \le p)$ (транзитивность)
        \item $(m \le n) \vee (n \le m)$ - всегда истинное выражение (множество $\N$ линейно упорядочено)
        \item $(m \le n) \Ra (m + p \le n + p)$
        \item $(m \le n) \Ra (m \cdot p \le n \cdot p)$
        \item $m \le p \Ra \exists n\ |\ m \cdot n \ge p$ (свойство Архимеда)
    \end{itemize}
\end{itemize}


\subsection{Целые числа}

\begin{definition}
    \textit{Множеством целых чисел} называется множество $\Z = \N \cup \{0\} \cup \{-n\ |\ n \in \N\}$
\end{definition}

Из определения сразу следует, что $\N \subset \Z$

\subsubsection*{Сложение}

Рассмотрим $m, n \in \N$:

\begin{enumerate}
    \item $m + n \Ra$ сложение происходит также, как и с натуральными числами.
    \item $(-m) + (-n) = -(m + n)$
    \item $m + (-n) = \System{&{p \in \N\ |\ n + p = m, \text{ если } m > n} \\ 
                              &{0, \text{ если } m = n} \\ 
                              &{-p, p \in \N\ |\ m + p = n, \text{ если } m < n}}$
\end{enumerate}

\subsubsection*{Умножение}

Рассмотрим $m, n \in \N$:

\begin{enumerate}
    \item $m \cdot n \Ra$ умножение происходит также, как и с натуральными числами.
    \item $(-m) \cdot (-n) := m \cdot n$
    \item $m \cdot (-n) := -(m \cdot n)$
\end{enumerate}

\subsubsection*{Свойства операций и отношений на множестве целых чисел}

\begin{itemize}
    \item Сложение
    \begin{itemize}
        \item Все свойства сложения натуральных чисел верны и для целых.
        \item $m + 0 := m$ (существование нейтрального к сложению числа)
        \item $\forall m \in \Z\ \exists (-m) \in \Z\ |\ m + (-m) = 0$ (существование обратного числа по сложению)
    \end{itemize}
    \item Умножение
    \begin{itemize}
        \item Все свойства умножения натуральных чисел верны и для целых.
        \item $m \cdot 0 := 0$
    \end{itemize}
    \item Отношение порядка
    \begin{itemize}
        \item $(\forall m \in \Z) (\forall n \in \Z) (\forall p \in \N)\ (m \le n) \Ra (m \cdot p \le n \cdot p)$
        \item $(\forall m_1 \in \Z)(\forall m_2 \in \Z \bs \{0\})\ \exists n \in \Z\ |\ n \cdot m_2 \ge m_1$ (свойство Архимеда)
    \end{itemize}
\end{itemize}


\subsection{Рациональные числа}

\begin{definition}
    \textit{Рациональным числом} называется число вида $\frac{m}{n}$, где $m \in \Z, n \in \N$. Несложно понять, что упорядоченная пара $(m, n)$ будет полностью задавать такое число.
    
    Множество рациональных чисел обозначают как $\Q$.
\end{definition}

\subsubsection*{Отношение эквивалентности на множестве рациональных чисел}

\begin{definition}
    Скажем, что $(m, n)R(p, q) := (mq = np)$
\end{definition}

\begin{proposition}
    Отношение $R$ является отношением эквивалентности на множестве рациональных чисел
\end{proposition}

\begin{proof}
    Для доказательства необходимо проверить, что выполнены все свойства отношения эквивалентности:
    \begin{enumerate}
        \item $(m, n)R(m, n) := (mn = nm)$ - верно.
        \item $(m, n)R(p, q) \Ra (p, q)R(m, n)$. $(m, n)R(p, q) := (mq = np)$, а $(p, q)R(m, n) := (pn = qm)$. Так как $(mq = np) \lra (pn = qm)$, то симметричность также верна.
        \item $(m, n)R(p, q) \wedge (p, q)R(r, s) \Ra (m, n)R(r, s)$. Опуская формальности, имеем $2$ равенства: $mq = np$ и $ps = qr$. Домножим первое на $s$, а второе на $n$: $mqs = nps = psn = qrn \Ra mqs = qrn \Ra ms = rn = nr$ - верно.
    \end{enumerate}
    $\Ra$ все $3$ свойства верны, а значит $R$ является отношением эквивалентности, что и требовалось доказать.
\end{proof}

\subsubsection*{Положительное рациональное число}

\begin{definition}
    \textit{Положительным рациональным числом} называется класс эквивалентности в $\N^2$ по отношению $R$ на множестве $\Q$.
\end{definition}
    
Множество всех положительных чисел обозначается за $\Q_+$
    
Используя это определение, множество всех рациональных чисел задаётся как $\Q = \Q_+ \cup \{0\} \cup \{-r\ |\ r \in \Q_+\}$
    
Отсюда также следует, что $\N \subset \Z \subset \Q$ 

\begin{anote}
    Рациональное число определяется как класс эквивалентности из-за того факта, что $\frac{1}{2} = \frac{2}{4}$ и так далее. Определение положительного рационального числа через $N^2$ справедливо, так как $\N \subset \Z \Ra \N^2 \subset \Z \times \N$
\end{anote}

\subsubsection*{Отношение строгого порядка на множестве рациональных чисел(не материал лектора)}

\begin{definition}
    Отношение строгого порядка $<$ ($>$) на множестве $\Q$ задаётся как
    \[
        \left(\frac{p}{q} < \frac{m}{n}\right) := (p \cdot n < m \cdot q)
    \]
\end{definition}

\begin{anote}
    В данном определении всё верно, так как $q, n \in \N$ по определению рациональных чисел.
\end{anote}

\subsubsection*{Отношение порядка на множестве рациональных чисел(не материал лектора)}

\begin{definition}
    Отношение порядка $\le$ ($\ge$) на множестве $\Q$ задаётся как и на предыдущих:
    \[
        (a \le b) := \big((a = b) \vee (a < b)\big)
    \]
\end{definition}

\subsubsection*{Свойства операций и отношений на множестве рациональных чисел}

\begin{itemize}
    \item Сложение
    \begin{itemize}
        \item Все свойства сложения целых чисел верны и для рациональных.
    \end{itemize}
    \item Умножение
    \begin{itemize}
        \item Все свойства умножения целых чисел верны и для рациональных.
        \item $\forall p \in \Q\ \exists p^{-1}\ |\ p \cdot p^{-1} = 1$ (существование обратного числа по умножению)
    \end{itemize}
    \item Отношение порядка
    \begin{itemize}
        \item Все свойства отношения порядка целых чисел верны и для отношения порядка рациональных чисел (хоть определение и отличается от того, что используется на целых числах).
    \end{itemize}
\end{itemize}