\subsection{Компактные множества}

\begin{definition}
	\textit{Компактным множеством в метрическом пространстве} $X$ называется такое множество $K$, что из любого его открытого покрытия множествами с индексами из $A$ можно выделить конечное подпокрытие:
	\[
		\exists \{\alpha_1, \ldots, \alpha_n\} \subset A \such \bigcup_{i = 1}^n G_{\alpha_i} \supset K
	\]
\end{definition}

\begin{example} (счётное покрытие)
	\[
		\bigcup_{n = 1}^\infty \left(\frac{1}{n}; 1\right) \supset (0; 1)
	\]
\end{example}

\begin{theorem}
	Любое компактное множество замкнуто
\end{theorem}

\begin{proof}
	$K$ - замкнутое $\lra$ $X \bs K$ - открытое. Рассмотрим $\forall x \in X \bs K, y \in K$. Тогда сразу
	\[
		\rho(y, x) > 0
	\]
	Положим $r_y := \frac{1}{2}\rho(y, x)$. Понятно, что $U_{r_y}(y) \cap U_{r_y}(x) = \emptyset$. Также несложно заметить открытое покрытие:
	\[
		\bigcup_{y \in K} U_{r_y} (y) \supset K
	\]
	При этом $K$ - компактное. Следовательно
	\[
		\exists \{y_1, \ldots, y_N\} \subset K \such \bigcup_{i = 1}^N U_{r_{y_i}}(y_i) \supset K
	\]
	Рассмотрим пересечение следующего вида:
	\[
		\bigcap_{i = 1}^N U_{r_{y_i}}(x) = U_{\min(r_{y_1}, \ldots, r_{y_N})}(x)
	\]
	Дополнительно $\forall i \in \range{N}$ выполнено
	\begin{align*}
		&{U_{\min(r_{y_1}, \ldots, r_{y_N})}(x) \subset U_{r_{y_i}}(x)}
		\\
		&{U_{r_{y_i}}(x) \cap U_{r_{y_i}}(y) = \emptyset}
	\end{align*}
	Тогда
	\[
		U_{\min(r_{y_1}, \ldots, r_{y_N})}(x) \cap \bigcup_{i = 1}^N U_{r_{y_i}}(y_i) = \emptyset
	\]
	Но при этом $K \subset \bigcup\limits_{i = 1}^N U_{r_{y_i}}(y_i)$. Значит
	\[
		U_{\min(r_{y_1}, \ldots, r_{y_N})}(x) \cap K = \emptyset
	\]
	Отсюда заключаем, что
	\[
		U_{\min(r_{y_1}, \ldots, r_{y_N})}(x) \subset X \bs K
	\]
	Итак, для $\forall x \in X \bs K$ мы нашли окрестность, которая находится в том же множестве $\Ra$ любая точка $X \bs K$ - внутренняя $\Ra$ $X \bs K$ - открытое множество $\lra$ $K$ - замкнутое множество.
\end{proof}

\begin{theorem}
	Каждое замкнутое подмножество компактного множества компактно.
\end{theorem}

\begin{proof}
	Пусть $K$ - компактное множество, а $F \subset K$ - замкнутое. Тогда, рассмотрим произвольное покрытие $F$ открытыми множествами:
	\[
		\bigcup\limits_{\alpha \in A} G_\alpha \supset F\ \ \ (G_\alpha \text{ - открытое})
	\]
	Так как $F$ - замкнутое, то $X \bs F$ - открытое. А значит
	\[
		(\bigcup\limits_{\alpha \in A} G_\alpha) \cup (X \bs F) \supset X \supset K
	\]
	получили открытое покрытие компактного множества $K$. По компактности получаем
	\[
		\exists \{\alpha_1, \ldots, \alpha_N\} \subset A \such \left(\bigcup\limits_{i = 1}^N G_{\alpha_i}\right) \cup (X \bs F) \supset K \supset F
	\]
	Отсюда
	\[
		F \subset \bigcup\limits_{i = 1}^N G_{\alpha_i}
	\]
\end{proof}

\begin{definition}
	$n$-мерным кубом назовём декартово произведение отрезков, каждый из которых имеет длину $d$:
	\[
		I = \prodl_{j = 1}^n [a^{(j)}; b^{(j)}] \such \forall j \in \range{n}\ \ b^{(j)} - a^{(j)} = d
	\]
\end{definition}

\begin{theorem}
	$n$-мерный куб компактен.
\end{theorem}

\begin{proof}
	От противного. Предположим, что существует покрытие такое, что из него нельзя выделить конечное подпокрытие куба:
	\[
		\exists \bigcup\limits_{\alpha \in A} G_\alpha \supset I
	\]
	Мысленно поделим каждую сторону куба на 2 части. Получится $2^n$ меньших кубов. Хотя бы 1 из этих кубов нельзя покрыть конечным подпокрытием (в противном случае теорема оказалась бы верна). Обозначим такой подкуб за $I_1$. Рекурсивно продолжим выбирать кубики и получим цепочку включений:
	\[
		I \supset I_1 \supset I_2 \supset I_3 \supset \ldots
	\]
	При этом куб $I_i$ будет обладать стороной
	\[
		d_i = \frac{d}{2^i}
	\]
	А также
	\[
		I_i = \prodl_{j = 1}^n [a_i^{(j)}; b_i^{(j)}]
	\]
	Отсюда имеем другую цепочку включений:
	\[
		[a^{(j)}; b^{(j)}] \supset [a_1^{(j)}; b_1^{(j)}] \supset [a_2^{(j)}; b_2^{(j)}] \supset \ldots
	\]
	По принципу Кантора это нам даёт, что
	\[
		\exists x_0^{(j)} \in \bigcap\limits_{i = 1}^\infty [a_i^{(j)}; b_i^{(j)}]
	\]
	Значит, $\vec{x}_0$ вообще попало в пересечение всех кубов:
	\[
		\vec{x}_0 = (x_0^{(1)}, \ldots, x_0^{(n)}) \in \bigcap\limits_{i = 1}^\infty I_i \subset I
	\]
	Коль скоро изначальный куб $I$ был покрыт открытыми множествами, то
	\begin{align*}
		&{\exists \alpha_0 \in A \such G_{\alpha_0} \ni \vec{x}_0}
		\\
		&{\exists r > 0 \such U_r(\vec{x}_0) \subset G_{\alpha_0}}
	\end{align*}
	Самые дальние точки куба $I_i$ удалены на расстояние $d_i \cdot \sqrt{n}$. Найдём такое $i \in \N$, что
	\[
		d_i \cdot \sqrt{n} = \frac{d}{2^i} \cdot \sqrt{n} < r \Ra I_i \subset G_{\alpha_0}
	\]
	Оно точно найдётся, так как $2^i$ возрастает. Но в итоге это приводит к противоречию, так как мы покрыли куб конечным подпокрытием.
\end{proof}

\begin{corollary}
	При $n = 1$ получается утверждение, что из любого покрытия отрезка $[a; b]$ открытыми множествами можно выделить конечное подпокрытие. Это утверждение также известно как \textit{теорема Гейне-Бореля}
\end{corollary}

\begin{theorem} (Критерий компактности в $\R^n$)
	В пространстве $\R^n$ следующие утверждения эквивалентны:
	\begin{enumerate}
		\item $K$ - ограниченное замкнутое множество
		
		\item $K$ - компактное множество
		
		\item \(\forall \{\vec{x}_n\}_{n = 1}^\infty \subset K\ \ \exists \left(\{\vec{x}_{m_k}\}_{k = 1}^\infty,\ \liml_{k \to \infty} \vec{x}_{m_k} = \vec{x}_0 \in K \right)\)
	\end{enumerate}
\end{theorem}

\begin{note}
	Утверждения 2 и 3 эквивалентны в любом метрическом пространстве, а вот эквивалентность с 1м - специфично для $\R^n$
\end{note}

\begin{proof}~
\begin{itemize}
	\item (1 $\Ra$ 2) $K$ ограничено. Следовательно
	\[
		\exists I \text{ - куб} \such K \subset I
	\]
	$I$ - компактное множество. Отсюда сразу следует, что и $K$ как замкнутое подмножество тоже компактно.
	
	\item (2 $\Ra$ 3) От противного. Предположим, что
	\[
		\exists \{\vec{x}_m\}_{m = 1}^\infty \subset K \such \forall \{\vec{x}_{m_k}\}_{k = 1}^\infty,\ \liml_{k \to \infty} \vec{x}_{m_k} = \vec{x}_0 \notin K
	\]
	Отсюда следует утверждение
	\[
		\forall \vec{x} \in K\ \exists U_{r_{\vec{x}}}(\vec{x}) \such  U_{r_{\vec{x}}}(\vec{x}) \cap \{\vec{x}_m\} \subset \{\vec{x}\}
	\]
	То есть \(\bigcup\limits_{\vec{x} \in K} U_{r_{\vec{x}}}(\vec{x}) \supset K\) - открытое покрытие. В силу компактности $K$
	\[
		\exists \{\vec{x}^{(1)}, \ldots, \vec{x}^{(N)}\} \such \bigcup\limits_{i = 1}^N U_{r_{\vec{x}^{(i)}}}(\vec{x}^{(i)}) \supset K \supset \{\vec{x}_m\}_{m = 1}^\infty
	\]
	Но это означает, что в нашей последовательности не более $N$ точек, а это невозможно.
	
	\item (3 $\Ra$ 1) От противного. Предположим, что $K$ неограничено:
	\[
		\forall m \in \N\ \exists \vec{x}_m \in K \such |\vec{x}_m| > m
	\]
	По третьему утверждению теоремы:
	\[
		\exists \{\vec{x}_{m_k}\}_{k = 1}^\infty \such \liml_{k \to \infty} \vec{x}_{m_k} = \vec{\lambda}_0
	\]
	Но в таком случае $\{\vec{x}_{m_k}\}_{k = 1}^\infty$ ограничена, что противоречит построению последовательности. Теперь докажем замкнутость $K$ (снова от противного). Предположим, что $K$ не замкнуто. Тогда
	\[
		\exists \{\vec{x}_m\} \subset K \such \liml_{m \to \infty} \vec{x}_m = \vec{x}_0 \notin K
	\]
	Но по условию у любой последовательности есть подпоследовательность, которая сходится к точке в множестве $K$. Противоречие.
\end{itemize}
\end{proof}

\begin{note}
	Дальнейшие определения даны для функций $f: X \to Y$, где $X, Y$ - метрические пространства.
\end{note}

\begin{definition}
	$x_0$ называется \textit{изолированной точкой} множества $E \subset \trbr{X, \rho}$, если $x_0 \in E$ и $\exists U_r(x_0) \such U_r(x_0) \cap E = \{x_0\}$
\end{definition}

\begin{definition}
	Точка прикосновения множества $E$, не являющаяся изолированной, называется \textit{предельной точкой} множества $E$.
\end{definition}

\begin{definition} (Предел функции по Коши)
	Пусть $x_0$ - предельная точка множества $D \subset X$ и $f: D \to Y$.
	
	$l \in Y$ называется \textit{пределом функции (отображения)} $f$ в $x_0$, если
	\[
		\forall \eps > 0\ \exists \delta > 0 \such \forall x \in D, 0 < \rho_X(x, x_0) < \delta\ \ \rho_Y(f(x), l) < \eps
	\]
\end{definition}

\begin{definition} (Предел функции по Гейне)
	Пусть $x_0$ - предельная точка множества $D \subset X$ и $f: X \to Y$.
	
	$l \in Y$ называется \textit{пределом функции (отображения)} $f$ в $x_0$, если
	\[
		\left(\forall \{x_n\}_{n = 1}^\infty \subset D \bs \{x_0\},\ \liml_{n \to \infty} x_n = x_0\right)\ \liml_{n \to \infty} f(x_n) = l
	\]
\end{definition}

\begin{theorem} (Эквивалентность определений предела функции)
	Определения предела по Коши и по Гейне эквивалентны.
\end{theorem}

\begin{proof}~
\begin{itemize}
	\item (К $\Ra$ Г) Рассмотрим произвольную последовательность $\{x_n\} \subset D \bs \{x_0\}$, у которой $\exists \liml_{n \to \infty} x_n = x_0$. По определению это означает, что
	\[
		\forall \delta > 0\ \exists N \in \N \such \forall n > N\ 0 < \rho_X(x_n, x_0) < \delta
	\]
	Отсюда следует, что $\rho_Y(f(x_n), l) < \eps$ для фиксированных $\eps, \delta > 0$ из предела по Коши. То есть
	\[
		\liml_{n \to \infty} f(x_n) = l
	\]
	
	\item (Г $\Ra$ К) От противного. Предположим, что условие Коши не выполнено:
	\[
		\exists \eps_0 > 0 \such \forall \delta > 0\ \exists x \in D,\ 0 < \rho_X(x, x_0) < \delta\ \ \rho_Y(f(x), l) \ge \eps_0
	\]
	Построим последовательность, рассмотрев $\delta := 1, \frac{1}{2}, \ldots, \frac{1}{n}, \ldots$:
	\begin{align*}
		&{\forall n \in \N\ \ \rho_X(x_n, x_0) < \frac{1}{n}}
		\\
		&{\forall n \in \N\ \ \rho_Y(f(x_n), l) \ge \eps_0}
	\end{align*}
	По первому утверждению получается, что последовательность сходится к $x_0$ и при этом $x_n \neq x_0$. Но в таком случае по Гейне последовательность $f(x_n)$ тоже должна сходится, чего не происходит. Противоречие.
\end{itemize}
\end{proof}

\subsection{Вектор-функции в $\R^n$}

\begin{definition}
	Вектор-функцией будем называть $\vec{a}(t): \R \to \R^n$, которую можно также записать как
	\[
		\vec{a}(t) = (a_1(t), \ldots, a_n(t))
	\]
\end{definition}

\begin{lemma} \label{vfeq}
	Следующие утверждения эквивалентны:
	\begin{enumerate}
		\item \(\liml_{t \to t_0} \vec{a}(t) = \vec{a}_0 = (a_{0, 1}, \ldots, a_{0, n})\)
		
		\item \(\liml_{t \to t_0} |\vec{a}(t) - \vec{a}_0| = 0\)
		
		\item \(\forall i \in \range{n}\ \liml_{t \to t_0} a_i(t) = a_{0, i}\)
	\end{enumerate}
\end{lemma}

\begin{proof}~
\begin{enumerate}
	\item (1 $\lra$ 2) Запишем определение Коши для первого утверждения:
	\[
		\forall \eps > 0\ \exists \delta > 0 \such \forall t, 0 < |t - t_0| < \delta\ |\vec{a}(t) - \vec{a}_0| < \eps
	\]
	Эта запись в точности совпадает с определением предела из второго утверждения
	
	\item (1 $\lra$ 3) Запишем определение Гейне для первого утверждения:
	\[
		\left(\forall \{t_k\}_{k = 1}^\infty \such t_k \neq t_0, \liml_{k \to \infty} t_k = t_0\right)\ \liml_{k \to \infty} \vec{a}(t_k) = \vec{a}_0
	\]
	Пользуясь критерием сходимости в $\R^n$, заменим запись выше эквивалентной:
	\[
		\forall i \in \range{n}\ \left(\forall \{t_k\}_{k = 1}^\infty \such t_k \neq t_0, \liml_{k \to \infty} t_k = t_0\right)\ \liml_{k \to \infty} a_i(t_k) = a_{0, i}
	\]
	Которая и является утверждением 3.
\end{enumerate}
\end{proof}