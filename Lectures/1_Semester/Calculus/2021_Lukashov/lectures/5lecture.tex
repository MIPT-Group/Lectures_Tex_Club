\subsubsection{Свойство полноты}

\begin{definition}
    Любое действительное число разбивает по отношению порядка множество $\R$ на две части
    \begin{align*}
        &A \neq \emptyset,\ B \neq \emptyset,\ A \cap B = \emptyset, A \cup B = \R \\
        &\forall a \in A, b \in B\ \ a \le b \\
        &\exists c \in \R\ |\ \forall a \in A, b \in B\ \ a \le c \le b
    \end{align*}
\end{definition}

\begin{proof}
    Построим $\{[p_n; q_n]_\Q\}_{n = 1}^\infty$ - система стягивающихся отрезков
    $$
        p_n - \text{ наибольшее } \frac{1}{10^{n - 1}}\Z \cap A,\ q_n - \text{ наименьшее } \frac{1}{10^{n - 1}}\Z \cap B
    $$
    
    Предположим, что $\exists b \in B\ |\ c > b \ra c + (-b) > 0$
    \begin{align*}
        \frac{1}{10^{n - 1}} < c - b \\
        c - p_n \le q_n - p_n < c - b \ra p_n > b
    \end{align*}
    А так как $p_n \in A,\ b \in B$, то мы получили противоречие.
\end{proof}

\subsection{Комплексные числа}

\begin{definition}
    \textit{Множеством комплексных чисел} называют множество $\Cm = \R^2$.
\end{definition}

\subsubsection{Сложение}

\begin{definition}
    \textit{Суммой} двух комплексных чисел $(a, b)$ и $(c, d)$ называют число
    $$
        (a, b) + (c, d) := (a + c, b + d)
    $$
\end{definition}

\subsubsection{Умножение}

\begin{definition}
    \textit{Произведением} двух комплексных чисел $(a, b)$ и $(c, d)$ называют число
    $$
        (a, b) \cdot (c, d) := (ac - bd, ad + bc)
    $$
\end{definition}

\subsubsection{Мнимая единица}

\begin{definition}
    \textit{Мнимой единицей} $i$ называют комплексное число $(0, 1)$, которое из определения выше имеет свойство:
    $$
        i^2 := -1
    $$
\end{definition}

\begin{proposition}
    Множество $\R$ вложено в множество $\Cm$.
\end{proposition}

\begin{proof}
    Действительно, если $a \in R$, то $a = (a, 0)$. Несложно проверить, что все операции будут точно такими же, как и с обычными действительными числами.
\end{proof}

\subsubsection{Алгебраическая форма комплексного числа}

\begin{definition}
    Заметим, что $(a, b) = a \cdot (1, 0) + b \cdot (0, 1)$. То есть:
    $$
        (a, b) = a + bi
    $$
    Запись числа $z$ в виде $a + bi$ называется \textit{алгебраической формой комплексного числа}
\end{definition}

\begin{center}
	\scalebox{1}{
		\begin{tikzpicture}
			\clip (-1.4, -1.3) rectangle (3.4, 3.3);
			\draw [->] (-1, 0) -- (3, 0) node [above, black] {$\re z$};
			\draw [->] (0, -1) -- (0, 3) node [right, black] {$\im z$};
			
			\draw [line width = 1pt, black!15!blue] (1.4,3pt) -- (1.4,-3pt) node [below, black] {$1$};
			\draw [line width = 1pt, black!15!blue] (3pt,1.4) -- (-3pt,1.4) node [left, black] {$i$};
			
			\draw [->, black!15!blue] (0, 0) -- (2, 1.5) node [black, above right, scale = 1.2] {$z$};
			\node[draw, circle, inner sep=1.4pt, fill, black!15!blue] at (2.06, 1.54) {};
			
			\coordinate (a1) at (2, 1.5);
			\coordinate (b) at (0, 0);
			\coordinate (c) at (1, 0);
			
			\pic [draw, ->] {angle = c--b--a1};
			\node [] at (0.75, 0.25) {$\phi$};
		\end{tikzpicture}
	}
\end{center}

\begin{definition}
    \textit{Модулем} комплексного числа нызывают число:
    $$
        |z| := \sqrt{a^2 + b^2}
    $$
\end{definition}

\begin{definition}
    \textit{Реальной частью} комплексного числа называют число $a$ в его алгебраической форме:
    $$
        \re(a + bi) := a
    $$
\end{definition}

\begin{definition}
    \textit{Мнимой частью} комплексного числа называют число $a$ в его алгебраической форме:
    $$
        \im(a + bi) := b
    $$
\end{definition}

\subsubsection{Неравенство треугольника}

\begin{center}
	\scalebox{1}{
		\begin{tikzpicture}
			\clip (-1.4, -1.3) rectangle (3.4, 3.3);
			\draw [->] (-1, 0) -- (3, 0) node [above, black] {$\re z$};
			\draw [->] (0, -1) -- (0, 3) node [right, black] {$\im z$};
			
			\draw [line width = 1pt, black!15!blue] (1,3pt) -- (1,-3pt) node [below, black] {$1$};
			\draw [line width = 1pt, black!15!blue] (3pt,1) -- (-3pt,1) node [left, black] {$i$};
			
			\draw [->, black!15!blue] (0, 0) -- (1.8, 0.8) node [black, below, scale = 1] {};
			\node[draw, circle, inner sep=1pt, fill, black!15!blue] at (1.85, 0.82) {};
			\node [] at (1.25, 0.3) {$z_1$};
			
			\draw [->, black!15!blue] (1.85, 0.82) -- (2.3, 2.6) node [black, below, scale = 1] {};
			\node[draw, circle, inner sep=1pt, fill, black!15!blue] at (2.34, 2.63) {};
			\node [] at (2.4, 2) {$z_2$};
			
			\draw [->, black!15!blue] (0.0, 0.0) -- (2.3, 2.6) node [black, below, scale = 1] {};
			\node [] at (0.95, 2.05) {$z_1 + z_2$};
		\end{tikzpicture}
	}
\end{center}

Геометрически очевидны следующие неравенства:
\begin{align*}
    &|z_1 + z_2| \le |z_1| + |z_2| \\
    &|z_1 - z_2| \ge ||z_1| - |z_2||
\end{align*}

\subsubsection{Деление комплексных чисел}

\begin{definition}
    Комплексное число $z_3$ называется \textit{частным} от деления числа $z_1$ на число $z_2$, если верно равенство:
    $$
        z_2 \cdot z_3 = z_1 \lra z_3 := \frac{z_1}{z_2} := z_1 / z_2
    $$
\end{definition}

\begin{corollary}
    Выведем действительную и мнимую часть частного, если $z_1 = a + bi$, а $z_2 = c + di$. При этом обозначим $z_3 = x + yi$:
    \begin{align*}
        &(c + di) \cdot (x + yi) = a + bi
        \\
        &cx - dy + (cy + dx)i = a + bi
        \\
        &\ra \System{a = cx - dy \\ b = cy + dx}
        \ra \System{x = \frac{\dse ac + bd}{\dse c^2 + d^2} \\ y = \frac{\dse bc - ad}{\dse c^2 + d^2}}
    \end{align*}
\end{corollary}

\subsubsection{Комплексно сопряженное число}

\begin{definition}
    Число $\bar{z}$ называется \textit{комплексно сопряжённым} к числу $z$, если
    $$
        z = a + bi \ra \bar{z} = a - bi
    $$
\end{definition}

\begin{proposition}
    Произведение комплексного числа $z$ на своё сопряженное является квадратом модуля
\end{proposition}

\begin{proof}
    Пусть $z = a + bi$. Тогда:
    $$
        z \cdot \bar{z} = (a + bi) \cdot (a - bi) = a^2 + b^2 = |z|^2
    $$
\end{proof}

\subsubsection{Аргумент комплексного числа}

\begin{definition}
    \textit{Аргументом} комплексного числа $z = a + bi$ называется угол $\phi$, отсчитываемый от положительного направления оси $\re$, с точностью до $2\pi k$, $k \in \Z$
    $$
        \arg z = \phi + 2\pi k, k \in \Z
    $$
    Угол считается положительным, если отсчитывается против часовой стрелки, и отрицательным, если наоборот.
\end{definition}

\begin{note}
    Аргумент определен только для комплексного числа, не равного нулю.
\end{note}

\subsubsection{Комплексное число в полярной записи}

\begin{definition}
    Несложно заметить, что
    \begin{align*}
        a = |z| \cdot \cos \phi \\
        b = |z| \cdot \sin \phi
    \end{align*}
    $$
        \ra z = |z|(\cos \phi + i \sin \phi)
    $$
\end{definition}

\subsubsection{Умножение чисел в полярных координатах}

Пусть есть 2 комплексных числа $z_1$ и $z_2$:

\begin{align*}
    z_1 = |z_1|(\cos\phi + i \sin\phi)
    \\
    z_2 = |z_2|(\cos\psi + i \sin\psi)
\end{align*}

Найдём их произведение в виде комплексного числа, записанного в полярных координатах:
\begin{multline}
    z_1 \cdot z_2 = |z_1||z_2|(\cos\phi + i \sin\phi)(\cos\psi + i \sin\psi) = \\
    |z_1||z_2|(\cos\phi \cdot \cos\psi - \sin\phi \cdot \sin\psi + i(\sin\phi \cdot \cos\psi + \sin\psi \cdot \cos\phi)) = \\
    |z_1||z_2|(\cos(\phi + \psi) + i \sin(\phi + \psi))
\end{multline}

Таким образом,
$$
    \System{
    &\arg(z \cdot w) = \arg(z) + \arg(w)
    \\
    &|z_1 \cdot z_2| = |z_1| \cdot |z_2|
    }
$$

\subsubsection{Показательная форма комплексного числа}

\begin{definition}
    Комплексное число можно записать как степень по натуральному основанию
    $$
        \cos \phi + i \sin \phi = e^{i \phi}
    $$
    Также это выражение называется \textit{формулой Эйлера}. С её помощью, комплексное число можно записать в \textit{показательной форме}.
    $$
        z = |z| \cdot e^{i \phi}
    $$
\end{definition}

\begin{note}
    Сейчас формулу Эйлера нужно принять "на веру". В будущем её можно и нужно доказать.
\end{note}

\subsubsection{Комплексное расширение тригонометрических функций}

Имея на руках формулу Эйлера, можно вывести интересные выражения для тригонометрических функций:
\begin{align*}
    e^{i\phi} &= \cos\phi + i \sin\phi
    \\
    e^{-i\phi} &= \cos\phi - i \sin\phi
    \\
    \ra \cos\phi &= \frac{\dse e^{i\phi} + e^{-i\phi}}{2}
    \\
    \sin\phi &= \frac{\dse e^{i\phi} - e^{-i\phi}}{2}
    \\
    \tg\phi &= \frac{\sin\phi}{\cos\phi}
    \\
    \ctg\phi &= \frac{\cos\phi}{\sin\phi}
\end{align*}

\subsubsection{Формула Муавра}

\begin{definition}
    \textit{Формулой Муавра} называется выражение:
    $$
        (\cos \phi + i \sin \phi)^n = \cos n\phi + i \sin n\phi,\ n \in \Z
    $$
\end{definition}

С помощью формулы Муавра можно находить натуральную степень комплексного числа:
$$
    z^n = |z|^n (\cos n\phi + i \sin n\phi) = (|z|(\cos \phi + i \sin \phi))^n
$$

\subsubsection{Натуральный корень из комплексного числа}

Решим уравнение $z^n = w$

\begin{enumerate}
    \item $w = 0 \ra z = 0$
    \item \begin{align*}
    &w \neq 0 \ra w = |w|(\cos \psi + i \sin \psi)
    \\
    &z = |z|(\cos \phi + i \sin \phi)
    \\
    &z^n = |z|^n(\cos n\phi + i \sin n\phi)
    \\
    &\ra |z| = \sqrt[n]{|w|},\ n\phi = \psi + 2\pi k,\ k \in \Z
    \\
    &\phi = \frac{\psi + 2\pi k}{n},\ k = \{0, 1, \dots n - 1\}
    \end{align*}
\end{enumerate}

\section{Пределы}

\subsection{Дополнительные свойства действительных чисел}

\subsubsection{Плотность множества рациональных чисел в множестве действительных}

\begin{proposition}
    Между любыми двумя неравными числами найдётся рациональное
    $$
    (\forall a, b \in \R\ |\ a < b)(\exists r \in \Q)\ |\ a < r < b
    $$
\end{proposition}

\begin{proof}
    По определению действительных чисел
    $$
    a := \{[p_n; q_n]_\Q\}_{n = 1}^\infty,\ b := \{[r_n; s_n]_\Q\}_{n = 1}^\infty
    $$
    $q_n - p_n = s_n - r_n = \frac{1}{10^{n - 1}} \text{ (такое } $n$ \text{ найдётся) } < \frac{b - a}{2} \ra q_n < r_n$
    
    $r := \frac{q_n + r_n}{2}$, так как $a \in [p_n; q_n]$ и $b \in [r_n; s_n]$
\end{proof}

\subsubsection{Равномощность}

\begin{definition}
    Множества $A$ и $B$ называются \textit{равномощными}, если существует биекция из $A$ в $B$
\end{definition}

\subsubsection{Счётность}

\begin{definition}
    Множство $A$ называется счётным, если оно равномощно $\N$
\end{definition}

\subsubsection{Теорема Кантора}

\begin{proposition}
    $\Q$ счётно, $R$ - несчётно
\end{proposition}

\begin{proof}
    По определению рационального числа, $(\forall r \in \Q)\ r = \frac{m}{n}, m \in \Z, n \in \N$
    
    Здесь надо как-то прикрепить табличку чисел и показать, что она счётна (нумерация по диагоналям)
    
    Предположим, что $[0; 1)$ - счётно, то есть $[0, 1) = \{x_1, x_2, \dots\}$
    Запишем каждое число в виде десятичной дроби:
    \begin{align*}
        x_1 = 0, \alpha_{11} \alpha_{12} \alpha{13} \\
        x_2 = 0, \alpha_{21} \alpha_{22} \alpha{23} \\
        x_3 = 0, \alpha_{31} \alpha_{32} \alpha{33} \\
        \vdots
    \end{align*}
    Рассмотрим число $\gamma = 0, \alpha_{11} \alpha_{22} \alpha_{33} \dots$. Заметим, что данного числа не было в таблице, потому что потому.
\end{proof}

\begin{definition}
    Если $A$ - ограниченное сверху множество действительных чисел, то число $b \in \R\ |\ (\forall a \in A)\ a \le b$ называется \textit{верхней гранью} множества $A$.
    
    Наименьшая из верхних граней называется \textit{точной верхней гранью}, обозначаемая как $\sup A$ (supremum)
\end{definition}

\begin{definition}
    Если $A$ - ограниченное снизу множество действительных чисел, то число $b \in \R\ |\ (\forall a \in A)\ a \ge b$ называется \textit{нижней гранью} множества $A$.
    
    Наибольшая из нижних граней называется \textit{точной нижней гранью}, обозначаемая как $\inf A$ (infinum)
\end{definition}

\begin{proposition}
    Любое непустое ограниченное сверху (снизу) множество действительных чисел имеет точную верхнюю (нижнюю) грань.
\end{proposition}

\begin{proof}
    Пусть $E \subset R$ - ограниченное сверху множество. Обозначим через $B$ множество всех верхних граней множества $E$. Тогда $A := \R \backslash B$.
    
    \begin{align*}
        x \in E,\ \forall a \in \R,\ a < x, a \in A \ra A \neq \emptyset, B \neq \emptyset, A \cup B = \R
        \\
        A \cap B = \emptyset,\ \forall a \in A, b \in B\ a < b
        \\
        \forall b \in B, x > b\ x \in B \ra \exists c \in R\ |\ \forall a \in A, b \in B\ a \le c \le b
    \end{align*}
    Предположим, что
    $$
        c \in B \ra c = \sup E
    $$
    А если предположить обратное, то
    $$
        c \notin B \ra \exists x \in E\ |\ x > c
    $$
    Рассмотрим
    $$
        \frac{x + c}{2} < x \ra \frac{x + c}{2} \in A \ra \frac{x + c}{2} > c, \text{ противоречие}
    $$
\end{proof}