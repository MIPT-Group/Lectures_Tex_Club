\section{Основы теории групп}
	
\subsection{Изоморфизмы групп}

\begin{proposition}
	Невырожденные матрицы порядка $n$ над полем $F$ образуют группу по умножению.
\end{proposition}

\begin{proof}
	Проверим непосредственно свойства группы:
	\begin{itemize}
		\item Умножение матриц в $M_n(F)$ ассоциативно.
		\item Матрица $E_n \in M_n(F)$ "--- невырожденная, и она является нейтральным элементом по умножению.
		\item Если $A, B$ "--- невырожденные матрицы порядка $n$, то $AB$ "--- тоже, поскольку $\det{AB} = \det{A}\det{B} \ne 0$. Кроме того, матрицы $A$ и $B$ обратимы, и матрицы $A^{-1}$, $B^{-1}$ "--- тоже невырожденные.\qedhere
	\end{itemize}
\end{proof}

\begin{definition}
	Пусть $F$ "--- поле, $n \in \N$.
	\begin{itemize}
		\item Группа невырожденных матриц порядка $n$ над $F$ обозначается через $\GL_n(F)$
		\item Группа невырожденных матриц порядка $n$ над $F$ с определителем, равным $1$, обозначается через $\SL_n(F)$
	\end{itemize}
\end{definition}

\begin{proposition}
	$\GL_n(\mathbb{Z}) := \{A \in M_n(\mathbb{Z}): \exists A^{-1} \in M_n(\mathbb{Z})\}$ "--- это группа матриц из $M_n(\Z)$ с определителем, равным $\pm1$.
\end{proposition}

\begin{proof}~
	\begin{itemize}
		\item Если $A, A^{-1} \in \GL_n(\mathbb{Z})$, то $\det{A}\det{A^{-1}} = \det{E} \hm{=} 1$, откуда $\det{A} = \det{A^{-1}} = \pm1$
		\item Если $\det{A} = \pm 1$, то, по формуле Крамера, $A^{-1} \in M_n(\mathbb{Z})$, и, аналогично, $\det{A^{-1}} = \pm1$, тогда $A^{-1} \in \GL_n(\mathbb{Z})$\qedhere
	\end{itemize}
\end{proof}

\begin{definition}
	\textit{Порядком} группы $G$ называется мощность множества $G$.
\end{definition}

\begin{definition}
	\textit{Гомоморфизмом групп} $G$ и $H$ называется отображение $\phi : G \rightarrow H$ такое, что для любых $a,b \in G$ выполнено равенство $\phi(ab) = \phi(a)\phi(b)$. \textit{Изоморфизмом групп} $G$ и $H$ называется биективный гомоморфизм $\phi : G \rightarrow H$. Пространства $G$ и $H$ называются \textit{изоморфными}, если между ними существует изоморфизм. Обозначение "--- $G \cong H$.
\end{definition}

\begin{note}
	Аналогичным образом можно определить изоморфизм и для других алгебраических структур, таких как кольцо или алгебра.
\end{note}

\begin{proposition}
	Пусть $\phi: G \rightarrow H$ "--- гомоморфизм групп, тогда:
	\begin{itemize}
		\item $\phi(e) = e$
		\item $\forall a \in G: \phi(a^{-1}) = (\phi(a))^{-1}$
	\end{itemize}
\end{proposition}

\begin{proof}~
	\begin{itemize}
		\item $\phi(e) = \phi(ee) = \phi(e)\phi(e) \Rightarrow \phi(e) = e$
		\item $\phi(e) = \phi(aa^{-1}) = \phi(a)\phi(a^{-1}) \Rightarrow \phi(a^{-1}) = (\phi(a))^{-1}\phi(e) \hm{=} (\phi(a))^{-1}$\qedhere
	\end{itemize}
\end{proof}

\begin{example}
	Гомоморфизмом групп $(\mathbb{Z}, +)$ и $(\mathbb{Z}_n, +)$ является отображение $\phi: \mathbb{Z} \rightarrow \mathbb{Z}_n$ такое, что для любого $a \in \Z$ выполнено $\phi(a) := \overline{a}$.
\end{example}

\begin{theorem}[Кэли]
	Пусть $G$ "--- конечная группа, $|G| = n$. Тогда существует подгруппа $H \le S_n$ такая, что $H \cong G$, то есть группа $G$ вкладывается в группу $S_n$.
\end{theorem}

\begin{proof}
	Рассмотрим группу $S(G)$ перестановок множества $G$, тогда $S(G) \cong S_n$, поскольку имеет место биекция между $G$ и $\{1, \dots, n\}$. Найдем требуемую подгруппу в $S(G)$. Для каждого элемента $a \in G$ определим перестановку $\sigma_a \in S(G)$ такую, что для любого $b \in G$ выполнено $\sigma_a(b) := ab$. Положим $H := \{\sigma_a \in S(G): a \in G\}$. Проверим, что $H \le S(G)$:
	\begin{itemize}
		\item $H \ne \emptyset$, поскольку $\sigma_e = \id \in H$
		\item $\forall a, b \in G: \sigma_a \circ \sigma_b = \sigma_{ab} \in H$
		\item $\forall a \in G: (\sigma_a)^{-1} = \sigma_{a^{-1}} \in H$
	\end{itemize}
	
	Определим отображение $\phi: G \rightarrow H$ для каждого $a \in G$ как $\phi(a) := \sigma_a$. Очевидно, это гомоморфизм, причем сюръективный. Он также инъективен, поскольку для различных $a, b \in G$ выполнено $\sigma_a(e) \ne \sigma_b(e)$. Таким образом, $G \cong H \le S(G) \cong S_n$.
\end{proof}

\subsection{Циклические группы}

\begin{proposition}
	Пусть $G$ "--- группа, $\{H_\alpha\}_{\alpha \in A}$ "--- произвольное семейство подгрупп в $G$. Тогда:
	\[\bigcap_{\alpha \in A} H_\alpha \le G\]
\end{proposition}

\begin{proof}
	Положим $K := \bigcap_{\alpha} H_\alpha$ и проверим свойства подгруппы:
	\begin{itemize}
		\item $K \ne \emptyset$, поскольку $e \in K$
		\item Если $a, b \in K$, то $a, b \in H_\alpha$ для любого $\alpha \in A$, тогда $ab \in H_\alpha$ для любого $\alpha \in A$, откуда $ab \in K$
		\item Аналогично предыдущему пункту, если $a\in K$, то $a^{-1}\in K$\qedhere
	\end{itemize}
\end{proof}

\begin{definition}
	Пусть $G$ "--- группа, $X \subset G$. \textit{Подгруппой, порожденной множеством} $X$, называется следующая подгруппа:
	\[\langle X\rangle := \bigcap_{H \le G~(X \subset H)}H\]
\end{definition}

\begin{note}
	$\langle X\rangle$ "--- наименьшая по включению подгруппа в $G$, содержащая множество $X$.
\end{note}

\begin{example}
	Рассмотрим несколько примеров порождающих множеств групп:
	\begin{itemize}
		\item $\mathbb{Z} = \langle 1 \rangle$
		\item $2\mathbb{Z} = \langle 2 \rangle$
		\item $\{e\} = \langle \emptyset \rangle$ для любой группы $G$ с нейтральным элементом $e$
	\end{itemize}
\end{example}

\begin{proposition}
	Для любого $n \in \N$ выполнено равенство $S_n = \langle (1, 2), (1, \dots, n)\rangle$.
\end{proposition}

\begin{proof}
	Уже было доказано, что любую перестановку $\sigma \in S_n$ можно представить в виде произведения транспозиций вида $(i, i+1)$. В то же время, любую такую транспозицию можно представить в виде произведения двух перестановок из условия. Для этого сначала циклическими сдвигами следует поместить элементы $i, i+1$ на позиции $1, 2$, затем поменять их местами и циклическими сдвигами вернуть на свои позиции.
\end{proof}

\begin{proposition}
	Пусть $G$ "--- группа, $X \subset G$. Тогда выполнено следующее равенство:
	\[\langle X\rangle = \{x_1\dots x_n: x_i \in X \text{ или } x_i^{-1} \in X\}\]
\end{proposition}

\begin{proof}
	Обозначим правую часть равенства через $K$.
	\begin{itemize}
		\item[$\supset$] Для любой подгруппы $H \le G$, такой, что $X \subset H$, каждый из множителей $x_i$ содержится в $H$, поэтому и $x_1\dots x_n \in H$. Значит, $K \subset \langle X\rangle$.
		\item[$\subset$] Множество $K$ непусто потому, что пустым произведением считается элемент $e$, а свойства замкнутости множества $K$ относительно умножения и взятия обратного элемента, очевидно, выполнены. Значит, $K \le G$, причем $X \subset K$, поэтому $\langle X\rangle \subset K$.\qedhere
	\end{itemize}
\end{proof}

\begin{note}
	В любой группе $G$ можно определить степень $n \in \Z$ произвольного элемента $a \in G$, отличную от нулевой и первой:
	\begin{itemize}
		\item Если $n > 0$, то $a^n$ "--- это произведение $n$ элементов $a$
		\item Если $n < 0$, то $a^n$ "--- это произведение $|n|$ элементов $a^{-1}$
	\end{itemize}

	Справедливо свойство, что для любых $k, n \in \Z$ выполнено $a^{k}a^{n} = a^{k + n}$. В этом можно убедиться непосредственной проверкой, перебрав все случаи знаков чисел $k$ и $n$.
\end{note}

\begin{definition}
	\textit{Порядком} элемента $a$ называется наименьшее $n \in \mathbb{N}$ такое, что $a^n = e$. Если такого $n$ не существует, то порядок считается равным $\infty$. Обозначение "--- $\ord{a}$.
\end{definition}

\begin{proposition}
	Пусть $G$ "--- группа, $a \in G$, $\ord{a} = n$. Тогда $a^k = e \Leftrightarrow n \mid k$.
\end{proposition}

\begin{proof}
	Разделим $k$ на $n$ с остатком, то есть представим его в виде $k = qn + r$, $r \hm{\in} \{0, \dots, n-1\}$, $q \in \Z$. Тогда $a^k = a^{qn + r} = (a^n)^qa^r = a^r$. Если $r \ne 0$, то $a^r \ne e$, иначе бы порядок $a$ был меньше $n$, что противоречит условию. Значит, $a^k = e \Leftrightarrow r = 0 \Leftrightarrow n \mid k$.
\end{proof}

\begin{proposition}
	Пусть $G$ "--- группа, $a \in G$. Тогда $\ord{a} = |\langle a\rangle|$.
\end{proposition}

\begin{proof}
	Если $\ord{a} = n \in \mathbb{N}$, то $\langle a\rangle = \{a^k: k \in \mathbb{Z}\} = \{e, a, \dots, a^{n-1}\}$, поэтому $|\langle a\rangle| \le n$. Кроме того, все элементы различны $e, a, \dots, a^{n-1}$. \pagebreak Действительно, если для некоторых $r, s \in \{1, \dots, n-1\}$, $r < s$ выполнено $a^r = a^s$, то $a^{s - r} = e$, откуда $s - r = 0$ в силу минимальности порядка $n$. Значит, $|\langle a\rangle| = n$. Если же $\ord{a} = \infty$, то для любых $\forall r, s \in \mathbb{Z}$, $r < s$, выполнено $a^r \ne a^s$ из аналогичных соображений, тогда $|\langle a\rangle| = \infty$.
\end{proof}

\begin{definition}
	Группа $G$ называется \textit{циклической}, если существует элемент $\exists a \in G$ такой, что $\langle a\rangle = G$.
\end{definition}

\begin{example}
	Рассмотрим несколько примеров циклических групп:
	\begin{itemize}
		\item $\mathbb{Z} = \langle 1 \rangle$
		\item $\mathbb{Z}_n = \langle \overline{1} \rangle$
	\end{itemize}
\end{example}

\begin{theorem}
	Любые две циклических группы одного порядка изоморфны.
\end{theorem}

\begin{proof}
	Пусть $G$ "--- циклическая группа, $a \in G$, $G = \langle a\rangle$, $|G| = n \in \mathbb{N}\cup\{\infty\}$. Рассмотрим сначала случай, когда $n = \infty$. Докажем, что тогда $G \cong \mathbb{Z}$ (и из этого будет следовать, что любые две таких группы изоморфны друг другу). Рассмотрим отображение $\phi: \mathbb{Z} \rightarrow G$, $\phi(k) = a^k$. $\phi(k + l) = a^{k + l} = a^ka^l = \phi(k)\phi(l)$, значит, $\phi$ "--- гомоморфизм. Поскольку $G = \{a^k~|~k \in \mathbb{Z}\}$, $\phi$ сюръективно. Наконец, $\phi$ инъективно из соображений деления с остатком.
	
	Теперь $n \in \mathbb{N}$. Докажем, что тогда $G \cong \mathbb{Z}_n$ (и из этого будет следовать, что любые две таких группы изоморфны друг другу). Рассмотрим отображение $\phi: \mathbb{Z}_n \rightarrow G$, $\phi(\overline{k}) = a^k$. $\phi$ определено корректно, поскольку $\forall k, l \in \mathbb{Z}: a^k = a^l \Leftrightarrow a^{k - l} = e \Leftrightarrow n~|~(k - l) \hm{\Leftrightarrow} \overline{k} = \overline{l}$ (и из этого также следует инъективность $\phi$). $\phi(\overline{k + l}) \hm{=} a^{k + l} = a^ka^l = \phi(\overline{k})\phi(\overline{l})$, значит, $\phi$ "--- гомоморфизм. Поскольку $G \hm{=} \{a^k~|~k \in \{0, \dots, n - 1\}\}$, $\phi$ сюръективно.
\end{proof}

\begin{note}
	Циклическая группа не более, чем счетна. Более того, любая конечно порожденная группа не более, чем счетна, поскольку $\mathbb{N}^k$ равномощно $\mathbb{N}$.
\end{note}

\begin{theorem}~
	\begin{enumerate}
		\item Подгруппа циклической группы тоже является циклической.
		\item Если $G = \langle a \rangle$ "--- циклическая группа, $|G| = n \in \mathbb{N}$, то $\forall k~|~n$ в $G$ есть ровно одна подгруппа порядка $k$, и других подгрупп в $G$ нет.
	\end{enumerate}
	
	$(\forall n \in \mathbb{N}: n\mathbb{Z} \le \mathbb{Z}$, $\forall k~|~n: \frac{n}{k}\mathbb{Z}_n \le \mathbb{Z}_n)$
\end{theorem}

\begin{proof}
	Рассмотрим сначала случай, когда $|G| = \infty$. Тогда можно считать, что это группа $\mathbb{Z}$. Пусть $H \le \mathbb{Z}$. Если $H = \{0\}$, утверждение уже доказано. Иначе "--- $H$ содержит ненулевые элементы, в частности, натуральные числа. Пусть $n$ "--- наименьшее из них. Поскольку $H$ "--- группа, то $\langle n \rangle = n\mathbb{Z} \le H$. Теперь рассмотрим $k \in H$. Разделим $k$ с остатком на $n$: $k = qn + r$, тогда $r = k - qn \in H$. Если $r \ne 0$, то $n$ "--- не наименьшее натуральное число, значит, $r = 0$ и $n~|~k$. Итак, $n\mathbb{Z} \le H$, при этом $\forall k \in H: k \in n\mathbb{Z}$, тогда $H = n\mathbb{Z}$: все подгруппы $\mathbb{Z}$ имеют вид циклических групп $n\mathbb{Z}$.
	
	Теперь пусть $|G| = n \in \mathbb{N}$. Тогда можно считать, что это группа $\mathbb{Z}_n$. Пусть $H \le \mathbb{Z}$. Если $H = \{\overline{0}\}$, утверждение уже доказано. Иначе "--- $H$ содержит ненулевые элементы. Выберем наименьшее $l > 0$ такое, что $\overline{l} \in H$. Разделим $n$ с остатком на $l$: $n = ql + r$, тогда $r = n - ql$ и $\overline{r} = -q\overline{l} \in H$. Если $r \ne 0$, то $l$ "--- не наименьшее натуральное число, значит, $r = 0$ и $l~|~n$. Тогда положим $k = \frac{n}{l}$ и заметим, что $\ord{\overline{l}} = k$ (поскольку $k\overline{l} = \overline{0}$ и $\forall m \in \{1, \dots, k - 1\}: 0 < ml < n$). Покажем теперь, что $H = \langle\overline{l}\rangle = l\mathbb{Z}_n$. $H$ "--- группа, поэтому $\langle\overline{l}\rangle \le H$. Аналогично, из соображений деления с остатком, $\forall \overline{d} \in H: l~|~d$, значит, $\forall \overline{d} \in H: \overline{d} \in l\mathbb{Z}_n$, тогда $H = l\mathbb{Z}_n$: все подгруппы $\mathbb{Z}_n$ имеют вид циклических групп $\frac{n}{k}\mathbb{Z}_n$ порядка $k$.
\end{proof}

\begin{proposition}
	Пусть $G$ "--- группа, $a \in G$, $\ord{a} = n$. Тогда $\ord{a^k} = \frac{n}{(n, k)}$.
\end{proposition}

\begin{proof}
	Пусть $(a^k)^l = a^{kl} = e$. Разделим $kl$ с остатком на $n$: $kl = qn + r$, тогда $a^{kl} = (a^{n})^qa^r = a^r$, поэтому если $r \ne 0$, то порядок $a$ меньше $n$. Значит, $r = 0$ и $n~|~kl$. Наименьшее число, одновременно кратное $n$ и $k$ "--- это $[n, k]$, тогда $l = \frac{[n, k]}{k} = \frac{n}{(n, k)}$.
\end{proof}

\begin{note}
	Одна и та же группа может быть порождена множествами различной мощности. Например, $\mathbb{Z} = \langle 1\rangle = \langle 2, 3\rangle = \langle 6, 10, 15\rangle$.
\end{note}

\subsection{Смежные классы}

\begin{definition}
	Пусть $G$ "--- группа, $A, B \subset G$. Определим следующие операции с множествами:
	\begin{itemize}
		\item $AB = \{ab~|~a \in A, b \in B\}$
		\item $A^{-1} = \{a^{-1}~|~a\in A\}$
	\end{itemize}
\end{definition}

\begin{note}~
	\begin{itemize}
		\item Умножение множеств ассоциативно как следствие ассоциативности операции в группе.
		\item В общем случае неверно, что $A^{-1}$ "--- обратный к $A$, поскольку не всегда $AA^{-1} = \{e\}$.
	\end{itemize}
\end{note}

\begin{definition}
	Пусть $G$ "--- группа, $H \le G$, $a \in G$. Тогда для элемента $a$ определим \textit{левый смежный класс} по подгруппе $H$ как $aH$, \textit{правый смежный класс} "--- как $Ha$.
\end{definition}

\begin{definition}
	\textit{Множество левых смежных классов} по подгруппе $H$ в группе $G$ обозначается как $G/H$, \textit{множество правых смежных классов} "--- $H\backslash G$.
\end{definition}

\begin{note}
	Если $a \in H$, то $aH = H$, поскольку $H$ "--- группа.
\end{note}

\begin{proposition}
	Пусть $G$ "--- группа, $H \le G$, $a, b \in G$. Тогда следующие утверждения эквивалентны:
	\begin{enumerate}
		\item $aH \cap bH \ne \emptyset$
		\item $b^{-1}a \in H$
		\item $aH = bH$
		\item $a \in bH$
	\end{enumerate}
\end{proposition}

\begin{proof}
	Доказательство $1 \Rightarrow 2$. По условию, $\exists h_1, h_2 \in H: ah_1 = bh_2$. Тогда $b^{-1}a = h_2h_1^{-1} \in H$.
	
	Доказательство $2 \Rightarrow 3$. $b^{-1}a \in H \Rightarrow b^{-1}aH = H$. Умножив обе части равенства на $b$, получим $aH = bH$.
	
	Доказательство $3 \Rightarrow 4$. $a = ae \in aH$, тогда, поскольку $aH = bH$, то $a \in bH$.
	
	Доказательство $4 \Rightarrow 1$. $a \in aH$, $a \in bH$, поэтому $a \in aH \cap bH$.
\end{proof}

\begin{note}
	Аналогичное утверждение спрведливо и для правых смежных классов с одним отличием: $ab^{-1} \in H$ вместо $b^{-1}a \in H$.
\end{note}

\begin{corollary}
	Если смежные классы (по одной и той же подгруппе) пересекаются, то они совпадают, значит, вся группа $G$ разбивается на непересекающиеся смежные классы.
\end{corollary}

\begin{theorem}[Лагранжа]
	Пусть $G$ "--- конечная группа, $H \le G$. Тогда:
	\[|G| = |H|\cdot|G/H|\]
	
	В частности, $|H|~\big|~|G|$.
\end{theorem}

\begin{proof}
	$G$ разбивается на $|G/H|$ непересекающихся смежных классов, и мощность каждого из них равна $H$ (поскольку умножение на фиксированный элемент $a$ инъективно). Следовательно, $|G| = |H|\cdot|G/H|$.
\end{proof}

\begin{corollary}
	Пусть $G$ "--- конечная группа, $a \in G$. Тогда $\ord{a}~\big|~|G|$.
\end{corollary}

\begin{proof}
	По теореме Лагранжа, $\ord{a} = |\langle a\rangle|~\big|~|G|$.
\end{proof}

\begin{corollary}
	Пусть $G$ "--- конечная группа, $a \in G$. Тогда $a^{|G|} = e$.
\end{corollary}

\begin{proof}
	Пусть $\ord{a} = k$, при этом $k~\big|~|G|$, следовательно, $a^{|G|} = e$.
\end{proof}

\begin{corollary}[Малая теорема Ферма]
	Пусть $p$ "--- простое число, $\overline{a} \hm{\in} \mathbb{Z}_p$, $\overline{a} \ne \overline{0}$. Тогда $\overline{a}^{p-1} = \overline{1}$.
\end{corollary}

\begin{proof}
	Рассмотрим группу $(\mathbb{Z}_p\backslash\{\overline{0}\}, \cdot)$, $|\mathbb{Z}_p\backslash\{\overline{0}\}| = p - 1$. Применяя предыдущее следствие, получим $\overline{a}^{p - 1} = \overline{1}$.
\end{proof}

\begin{definition}
	\textit{Функцией Эйлера} называется следующая функция: $\phi(n) = |\{a \in \mathbb{N}, a \le n, (a, n) = 1\}|, n \in \mathbb{N}$. Если $n = p_1^{\alpha_1}\dots p_k^{\alpha_k}$, то $\phi(n) = n(1 - \frac{1}{p_1})\dots(1 - \frac{1}{p_k})$.
\end{definition}

\begin{theorem}[Эйлера]
	Пусть $\overline{a} \in \mathbb{Z}_n$, $(a, n) = 1$. Тогда $\overline{a}^{\phi(n)} = \overline{1}$.
\end{theorem}

\begin{proof}
	Рассмотрим $(\mathbb{Z}_n^*, \cdot)$, $\mathbb{Z}_n^* = \{\overline{b} \in \mathbb{Z}_n~|~(b, n) = 1\}$, $|\mathbb{Z}_n^*| = \phi(n)$. Тогда для $\overline{a} \in \mathbb{Z}_n^*$ верно, что $\overline{a}^{|\mathbb{Z}_n^*|} = \overline{a}^{\phi(n)} = \overline{1}$.
\end{proof}

\begin{note}
	Из аналогичных соображений можно показать, что $|G| = |H|\cdot|H\backslash G|$, тогда в случае, когда $G$ "--- конечная группа, верно, что $|G / H| = |H\backslash G|$.
\end{note}

\begin{proposition}
	Пусть $H$ "--- подгруппа в $G$. Тогда $|G / H| \hm{=} |H\backslash G|$, даже когда $|G| = \infty$.
\end{proposition}

\begin{proof}
	Определим отображение $\phi$ следующим образом: $aH \mapsto (aH)^{-1} = H^{-1}a^{-1} = Ha^{-1}$. $\phi$ определено корректно: если $aH = bH$, то и их образы совпадают. Более того, $\phi$ является биекцией, поскольку существует обратное отображение $\phi^{-1}$: $Ha \hm{\mapsto} (Ha)^{-1} = a^{-1}H$, что равносильно биективности.
\end{proof}

\begin{definition}
	Пусть $G$ "--- группа, $H \le G$. Тогда величина $|G / H| = |H\backslash G|$ называется \textit{индексом} $H$ в $G$. Обозначение "--- $|G : H|$.
\end{definition}