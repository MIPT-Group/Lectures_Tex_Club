\section{Уравнения прямых и плоскостей}

\subsection{Прямая в плоскости}

\begin{definition}
	\textit{Направляющим вектором} прямой $l \subset P_3$ называется вектор $\overline{a} \in V_3$, $\overline{a} \ne \overline 0$, представителем которого является направленный отрезок, лежащий в $l$.
\end{definition}

\begin{definition}
	Пусть $l \subset P_2$ "--- прямая с направляющим вектором $\overline{a} \in V_2$, $M \in L$, и в декартовой системе координат $(O, e)$ в $P_2$ выполнены соотношения $\overline{a} \leftrightarrow_{e} (\alpha_1, \alpha_2)^T$, $M \leftrightarrow_{(O, e)} (x_0, y_0)^T$, $\overline{r_0} = \overline{OM} \leftrightarrow_{e} (x_0, y_0)^T$.
	\begin{itemize}
		\item \textit{Векторно-параметрическим уравнением прямой} называется следующее уравнение:
		\[\overline{r} = \overline{r_0} + t\overline{a},~t \in \mathbb{R}\]
		\item \textit{Параметрическим уравнением прямой} называется следующая система:
		\[\left\{
		\begin{aligned}
			x = x_0 + t\alpha_1\\
			y = y_0 + t\alpha_2
		\end{aligned}
		\right.,~t \in \R
		\]
		\item \textit{Каноническим уравнением прямой} называется следующее уравнение:
		\[\frac{x - x_0}{\alpha_1} = \frac{y - y_0}{\alpha_2}\]
	\end{itemize}
\end{definition}

\begin{note}
	Множество точек $X \in P_2$ таких, что $X \leftrightarrow_{(O, e)} (x, y)^T$, $\overline{r} := \overline{OX}$, являющихся решениями любого из уравнений прямой выше, совпадает с прямой $l$. Действительно, $X \in L \lra MX \parallel l \lra \overline{MX} \parallel \overline{a}$.
\end{note}

\begin{note}
	В случае канонического уравнения прямой, если без ограничения общности $\alpha_1 = 0$, то тогда $\alpha_2 \ne 0$, и следует считать, что исходное уравнение эквивалентно условию $x = x_0$. Отметим также, что каноническое уравнение прямой эквивалентно следующему такому уравнению:
	\[\alpha_2 x - \alpha_1 y + (\alpha_1 y_0 - \alpha_2 x_0) = 0\]
\end{note}

\begin{definition}
	Пусть $A, B, C \in \R$, $A^2+B^2 \ne 0$. \textit{Общим уравнением прямой} называется следующее уравнение:
	\[Ax+By+C = 0\]
\end{definition}

\begin{proposition}
	Пусть прямая $l \subset P_2$ задана в декартовой системе координат $(O, e)$ общим уравнением прямой $Ax+By+C=0$, $\overline{b} \in V_2$, $\overline{b} \leftrightarrow_{e} \beta$. Тогда $\overline{b} \parallel l \Leftrightarrow A\beta_1 + B\beta_2 = 0$.
\end{proposition}

\begin{proof}
	Пусть $M \in l$, точка $N \in P_2$ такова, что $\overline{MN} \hm{=} \overline{b}$, и $M \leftrightarrow_{(O, e)} (x_0, y_0)^T$, тогда $N \leftrightarrow_{(O, e)} (x_0 + \beta_1, y_0 + \beta_2)^T$. Поскольку $M \in l$, выполнены следующие эквивалентности:
	\[\overline{b} \parallel l \lra N \in L \lra A(x_0 + \beta_1) + B(y_0 + \beta_2) + C = 0 \lra A\beta_1 + B\beta_2 = 0\qedhere\]
\end{proof}

\begin{corollary}
	Пусть прямая $l \subset P_2$ задана в декартовой системе координат $(O, e)$ общим уравнением прямой $Ax+By+C=0$, $\overline{b} \in V_2$, $\overline{b} \leftrightarrow_{e} \beta$. Тогда направляющим вектором прямой $l$ является вектор $\overline{a} \in V_2$ такой, что $\overline a \leftrightarrow_{e} (-B, A)^T$.
\end{corollary}

\begin{definition}
	\textit{Вектором нормали} прямой $l \subset P_3$ называется вектор $\overline{n} \in V_3$, $\overline{n} \ne \overline 0$, представителем которого является направленный отрезок, ортогональный прямой $l$.
\end{definition}

\begin{corollary}
	Пусть прямая $l \subset P_2$ задана в декартовой системе координат $(O, e)$ общим уравнением прямой $Ax+By+C=0$, $\overline{b} \in V_2$, $\overline{b} \leftrightarrow_{e} \beta$. Тогда вектором нормали прямой $l$ является вектор $\overline{n} \in V_2$ такой, что $\overline n \leftrightarrow_{e} (A, B)^T$.
\end{corollary}

\begin{definition}
	Пусть $l \subset P_2$ "--- прямая с вектором нормали $\overline{n} \in V_2$, $M \in L$, $\overline{r_0} = \overline{OM}$. \textit{Нормальным уравнением прямой} называется следующее уравнение:
	\[(\overline{r} - \overline{r_0}, \overline{n}) = 0\]
\end{definition}

\begin{note}
	Множество точек $X \in P_2$, $\overline{r} := \overline{OX}$, являющихся решениями нормального уравнения прямой, совпадает с прямой $l$. Кроме того, это уравнение можно переписать в следующем виде при $\alpha := (\overline{r_0}, \overline{n})$:
	\[(\overline{r}, \overline{n}) = (\overline{r_0}, \overline{n})\]
\end{note}

\begin{note}
	Различные типы уравнений, задающих прямую, эквивалентны: из каждого из них можно получить любое другое.
\end{note}

\begin{note}
	Рассмотренные способы задания прямой позволяют определить \textit{взаимное расположение прямых на плоскости}. Пусть прямые $l_1, l_2$ заданы векторно-параметричес\-кими уравнениями $\overline{r} \hm{=} \overline{r_1} + t\overline{a_1}$, $\overline{r} = \overline{r_2} + t\overline{a_2}$. Тогда:
	\begin{itemize}
		\item $l_1 \cap l_2 \ne \emptyset$ и $l_1 \ne l_2 \lra \overline{a_1} \nparallel \overline{a_2}$
		\item $l_1 \parallel l_2$ и $l_1 \ne l_2 \Leftrightarrow \overline{a_1} \parallel \overline{a_2}$ и $(\overline{r_1} - \overline{r_2}) \nparallel \overline{a_1}$ 
		\item $l_1 = l_2 \Leftrightarrow (\overline{r_1} - \overline{r_2}) \parallel \overline{a_1} \parallel \overline{a_2}$
	\end{itemize}
\end{note}

\begin{note}
	Прямая $l \subset P_2$ в плоскости делит ее на две полуплоскости. Выделим одну из открытых полуплоскостей $S \subset P_2$. Пусть в прямоугольной декартовой системе координат прямая $l \subset P_2$ задана нормальным уравнением $(\overline{r} \hm{-} \overline{r_0}, \overline{n}) = 0$, причем вектор нормали $\overline{n}$ направлен в полуплоскость $S$. Тогда точка $X \in P_2$, $\overline{OX} = \overline{r}$, лежит в $S$ $\Leftrightarrow (\overline{r} \hm{-} \overline{r_0}, \overline{n}) > 0$. Противоположная полуплоскость задается противоположным неравенством.
\end{note}

\begin{definition}
	\textit{Пучком прямых} называется либо множество всех прямых в $P_2$, проходящих через фиксированную точку $P \in P_2$, либо множество всех прямых, параллельных фиксированной прямой $l \subset P_2$.
\end{definition}

\begin{note}
	Любые две прямые лежат ровно в одном пучке.
\end{note}

\begin{theorem}
	Пусть в декартовой системе координат $(O, e)$ в $P_2$ прямые $l_1, l_2$, $l_1 \ne l_2$, заданы уравнениями $A_1x\hm{+}B_1y+C_1=0$, $A_2x+B_2y+C_2=0$. Тогда прямая $l \subset P_2$ лежит в одном пучке с прямыми $l_1$ и $l_2$ $\lra$ прямая $l$ задается уравнением следующего вида при некоторых $\alpha_1, \alpha_2 \in \R$:
	\[\alpha_1(A_1x+B_1y+C_1) + \alpha_2(A_2x+B_2y+C_2) = 0\]
\end{theorem}

\begin{proof}~
	\begin{itemize}
		\item[$\la$] Возможны два случая:
		\begin{enumerate}
			\item Если $l_1 \cap l_2 = \{P\}$, $P \in P_2$, то координаты точки $P$ удовлетворяют требуемому уравнению, то есть $P \in l$.
			\item Если $l_1 \parallel l_2$, то из требуемого уравнения направляющий вектор прямой $l$ параллелен направляющим векторам $l_1$ и $l_2$. В этом случае уравнение задает прямую не при всех $\alpha_1, \alpha_2 \in \R$, но если задает, то лежащую в данном пучке.
		\end{enumerate}
		
		\item[$\ra$] Возможны два случая:
		\begin{enumerate}
			\item Если $l \cap l_1 \cap l_2 = \{P\}$, $P \in P_2$, то выберем на $l$ точку $Q \ne P$, $Q \leftrightarrow_{(O, e)} (x_0, y_0)^T$. Тогда $Q$ удовлетворяет уравнению с коэффициентами $\alpha_1 = A_2x_0+B_2y_0+C_2$, $\alpha_1 = -(A_1x_0+B_1y_0+C_1)$. Хотя бы один из коэффициентов ненулевой, поскольку $Q$ лежит не более, чем на одной из прямых $l_1$, $l_2$. Значит, такое уравнение задает $l$, так как ему удовлетворяют две различных точки этой прямой.
			
			\item Если $l \parallel l_1 \parallel l_2$, то аналогичным образом выберем любую точку $Q \in l$ и соответствующие коэффициенты, тогда полученное уравнение задает $l$ при условии, что оно задает прямую. Но оно всегда задает прямую, поскольку множество его решений непусто и не содержит хотя бы одну из прямых $l_1, l_2$.\qedhere
		\end{enumerate}
	\end{itemize}
\end{proof}

\begin{proposition}
	Пусть в прямоугольной декартовой системе координат $(O, e)$ в $P_2$ прямые $l_1, l_2$ заданы  уравнениями $A_1x\hm{+}B_1y+C_1=0$, $A_2x\hm{+}B_2y\hm{+}C_2=0$. Тогда угол $\phi$ между ними удовлетворяет следующему равенству:
	\[\cos{\phi} = \frac{|A_1A_2+B_1B_2|}{\sqrt{A_1^2+B_1^2}\sqrt{A_2^2+B_2^2}}\]
\end{proposition}

\begin{proof}
	Пусть $\overline{n_1}, \overline{n_2} \in V_2$, $\overline{n_1} \leftrightarrow_{e} (A_1, B_1)^T$, $\overline{n_2} \leftrightarrow_{e} (A_2, B_2)^T$, "--- нормальные векторы прямых $l_1, l_2$, $\alpha := \angle(\overline{n_1}, \overline{n_2})$. Тогда угол $\phi$ равен меньшему из углов $\alpha$ и $\pi - \alpha$. В каждом из случаев выполнено следующее:
	\[\cos{\phi} = |\cos{\alpha}| = \frac{|A_1A_2+B_1B_2|}{\sqrt{A_1^2+B_1^2}\sqrt{A_2^2+B_2^2}}\qedhere\]
\end{proof}

\begin{proposition}
	Пусть в прямоугольной декартовой системе координат $(O, e)$ в $P_2$ прямая $l$ задана  уравнением $Ax\hm{+}By+C=0$, $M \in P_2$, $M \leftrightarrow_{(O, e)} (x_0, y_0)^T$. Тогда расстояние $\rho$ от точки $M$ до прямой $l$ равно следующей величине:
	\[\rho = \frac{|Ax_0 + By_0 + C|}{\sqrt{A^2 + B^2}}\]
\end{proposition}

\begin{proof}
	Пусть $\overline{n} \in V_2$, $\overline n \leftrightarrow_{e} (A, B)^T$ "--- вектор нормали прямой $l$, $\overline{r_0} := \overline{OM}$, и пусть $X \in L$, $\overline{r} := \overline{OX}$. Тогда:
	\[\rho = |\pr_{\overline{n}}(\overline{r_0} - \overline{r})|
	=
	\left|\frac{(\overline{r_0} - \overline{r}, \overline{n})}{|\overline{n}|^2}\overline{n}\right|
	=
	\frac{|(\overline{r_0} - \overline{r}, \overline{n})|}{|\overline{n}|^2}|\overline{n}|
	=
	\left|\frac{(\overline{r_0} - \overline{r}, \overline{n})}{|\overline{n}|}\right|
	=
	\frac{|Ax_0 + By_0 + C|}{\sqrt{A^2 + B^2}}\qedhere\]
\end{proof}

\subsection{Плоскость в простанстве}

Рассмотрим различные способы задания плоскости $\nu$ в пространстве. Пусть $M (\overline{r_0})$ "--- точка на плоскости $\nu$, $\overline{a}$ $\overline{b}$ "--- векторы в плоскости $\nu$ ($\overline{a} \nparallel \overline{b}$, $\overline{a}, \overline{b} \in \nu$). Тогда $X (\overline{r}) \in \nu \Leftrightarrow \overline{MX} \text{ компланарен } \overline{a}, \overline{b}$. Введем базис $e$ и декартову систему координат $(O, e)$:
\[M \leftrightarrow_{(O, e)}
\begin{pmatrix}
x_0\\y_0\\z_0
\end{pmatrix},~
X \leftrightarrow_{(O, e)}
\begin{pmatrix}
x\\y\\z
\end{pmatrix},~
\overline{a} \leftrightarrow_{e}
\begin{pmatrix}
\alpha_1\\\alpha_2\\\alpha_3
\end{pmatrix},~
\overline{b} \leftrightarrow_{e}
\begin{pmatrix}
\beta_1\\\beta_2\\\beta_3
\end{pmatrix}\]

\begin{definition} \textit{Векторно-параметрическим уравнением плоскости} называется следующее уравнение:
	\[\overline{r} = \overline{r_0} + t\overline{a} + s\overline{b}~(t,s \in \mathbb{R})\]
	
	Данное уравнение можно также переписать в виде:
	\[(\overline{r} - \overline{r_0}, \overline{a}, \overline{b}) = 0 \Leftrightarrow (\overline{r}, \overline{a}, \overline{b}) = \alpha\]
\end{definition}

\begin{definition} \textit{Параметрическим уравнением плоскости} называется следующая система:
	\[\left\{
	\begin{aligned}
	x = x_0 + t\alpha_1 + s\beta_1\\
	y = y_0 + t\alpha_2 + s\beta_2\\
	z = z_0 + t\alpha_3 + s\beta_3
	\end{aligned}
	\right.
	\]
\end{definition}

Если в явном виде расписать смешанное произведение $(\overline{r} - \overline{r_0}, \overline{a}, \overline{b})$ как определитель соответствующей матрицы, можно получить еще одно уравнение плоскости.
\begin{definition}
	\textit{Общим уравнением плоскости} называется следующее уравнение:
	\[Ax + By + Cz + D = 0~(A^2+B^2+C^2 \ne 0)\]
\end{definition}

\begin{proposition}
	Пусть $\nu$ задана в $(O, e)$ общим уравнением плоскости $Ax+By+Cz+D=0$, вектор $\overline{b} \leftrightarrow_{(O, e)} \beta$. Тогда $\overline{b} \hm{\parallel} \nu \Leftrightarrow A\beta_1 + B\beta_2 + C\beta_3 = 0$.
\end{proposition}

\begin{proof}
	Пусть точка $M \in \nu$, а точка $N$ такова, что $\overline{MN} \hm{=} \overline{b}$. Тогда:
	\begin{gather*}
	M \leftrightarrow_{(O, e)}
	\begin{pmatrix}
	x_0\\y_0\\z_o
	\end{pmatrix},~
	N \leftrightarrow_{(O, e)}
	\begin{pmatrix}
	x_0 + \beta_1\\y_0 + \beta_2\\z_o+\beta_3
	\end{pmatrix}\\
	\overline{b} \parallel \nu \Leftrightarrow N \in \nu \Leftrightarrow A(x_0 + \beta_1) + B(y_0 + \beta_2) + C(z_0 + \beta_3) + D = 0
	\end{gather*}
	
	Теперь, так как $M \in \nu$:
	\[A(x_0 + \beta_1) + B(y_0 + \beta_2) + C(z_0 + \beta_3) + D = 0 \Leftrightarrow A\beta_1 + B\beta_2 + C\beta_3= 0\]
\end{proof}

\begin{proposition}
	Уравнение вида $Ax+By+Cz+D=0$ при $A^2+B^2+C^2\ne0$ всегда задает плоскость в декартовой системе координат $(O, e)$.
\end{proposition}

\begin{proof}
	Пусть без ограничения общности $A \ne 0$. Выберем пару векторов $\overline{a}$, $\overline{b}$ и точку $M$ такие, что: \[\overline{a} \leftrightarrow_{e} \begin{pmatrix}-B\\A\\0\end{pmatrix},~\overline{b} \leftrightarrow_{e} \begin{pmatrix}-C\\0\\A\end{pmatrix},~M \leftrightarrow_{(O, e)} \begin{pmatrix}-\frac{D}{A}\\0\\0\end{pmatrix}\]
	
	Векторы $\overline{a}$ и $\overline{b}$ неколлинеарны, поскольку их координаты непропорциональны. Если отложить их от точки $M$, то им будет соответствовать некоторая плоскость $\nu$. Рассмотрим произвольную точку $X(\overline{r})$:
	\[X \in \nu \Leftrightarrow (\overline{r} - \overline{r_0}, \overline{a}, \overline{b}) = 0\]
	\[\begin{vmatrix}
	x + \frac{D}{A} & -B & -C\\
	y & A & 0\\
	z & 0 & A
	\end{vmatrix} = (x + \frac{D}{A})A^2 - y(-B)A - z(-C)A = 0\]
	\[A^2x + ABy + ACz + AD = 0 \Leftrightarrow Ax + By + Cz + D = 0\]
	
	Таким образом, мы получили, что данное уравнение задает в точности плоскость $\nu$.
\end{proof}

Пусть $\overline{n}$ "--- вектор нормали к плоскости $\nu$. Тогда $X (\overline{r}) \in \nu \hm{\Leftrightarrow} \overline{MX} \perp \overline{n}$.

\begin{definition}
	\textit{Нормальным уравнением плоскости} называется следующее уравнение:
	\[(\overline{r} - \overline{r_0}, \overline{n}) = 0 \Leftrightarrow (\overline{r}, \overline{n}) = \alpha~(\overline{n} \ne \overline{0})\]
\end{definition}

\begin{note}
	Непосредственная проверка позволяет убедиться, что различные уравнения, задающие плоскость, эквивалентны, поскольку из каждого из них можно получить любое другое.
\end{note}

\begin{note}
	В прямоугольной декартовой системе координат нормальное уравнение плоскости преобразуется к виду $Ax+By+Cz\hm{+}D=0$, и это означает, что в такой системе координат нормальный вектор плоскости имеет вид:
	\[\overline{n} \leftrightarrow_{e} \begin{pmatrix}A\\B\\C\end{pmatrix}\]
\end{note}

\begin{definition}
	Пусть в произвольной декартовой системе координат плоскость $\nu$ задана уравнением $Ax+By+Cz\hm{+}D=0$. Тогда \textit{сопутствующим вектором} плоскости $\nu$ в данной системе координат называется следующий вектор:
	\[\overline{n} \leftrightarrow_{e} \begin{pmatrix}A\\B\\C\end{pmatrix}\]
\end{definition}

Рассмотренные способы задания плоскости позволяют определить \textit{взаимное расположение плоскостей в пространстве}.

\begin{proposition}
	Пусть плоскости $\nu_1, \nu_2$ заданы общими уравнениями $A_1x+B_1y+C_1z+D_1 = 0$, $A_2x+B_2y+C_2z+D_2 = 0$. Тогда:
	\begin{itemize}
		\item $\nu_1 \cap \nu_2 \Leftrightarrow \overline{n_1} \nparallel \overline{n_2}$
		\item $\nu_1 \parallel \nu_2 \text{ и } \nu_1 \ne \nu_2 \Leftrightarrow \overline{n_1} \parallel \overline{n_2} \text{, но уравнения непропорциональны}$
		\item $l_1 = l_2 \Leftrightarrow  \text{уравнения пропорциональны}$
	\end{itemize}
\end{proposition}

\begin{proof}~
	\begin{itemize}
		\item По условию $\overline{n_1} \nparallel \overline{n_2}$, тогда без ограничения общности $\begin{vmatrix}A_1 & A_2\\B_1 & B_2\end{vmatrix} \hm{\ne} 0$. Рассмотрим систему уравнений относительно $x$ и $y$:
		\[\left\{
		\begin{aligned}
		A_1x + B_1y = -C_1z -D_1\\
		A_2x + B_2y = -C_2z -D_2
		\end{aligned}
		\right.
		\]
		Согласно правилу Крамера, эта система имеет единственное решение $\forall z$. Значит, плоскости имеют общие точки, но не все их точки общие, и это означает, что они пересекаются.
		\item Коэффициенты $A_i$, $B_i$, $C_i$ пропорциональны из параллельности $\overline{n_1}$ и $\overline{n_2}$, а уравнения "--- нет, поэтому можно считать, что они отличаются только коэффициентом $D_i$. Тогда они не имеют общих решений, то есть $\nu_1 \parallel \nu_2 \text{ и } \nu_1 \ne \nu_2$.
		\item Уравнения пропорциональны, поэтому можно считать, что они совпадают. Тогда совпадают и их множества решений, то есть $\nu_1 = \nu_2$.
	\end{itemize}
\end{proof}

\begin{proposition}
	Пусть в произвольной декартовой системе координат $(O, e)$ пересекающиеся плоскости $\nu_1$ и $\nu_2$ заданы уравнениями $A_1x+B_1y+C_1z+D_1 = 0$ и $A_2x+B_2y+C_2z+D_2 = 0$. Тогда направляющий вектор их прямой пересечения имеет координаты:
	\[\overline{v} \leftrightarrow_{e}
	\begin{pmatrix}
	\det\begin{pmatrix}
	B_1&C_1\\
	B_2&C_2
	\end{pmatrix}\\
	\det\begin{pmatrix}
	C_1&A_1\\
	C_2&A_2
	\end{pmatrix}\\~
	\det\begin{pmatrix}
	A_1&B_1\\
	A_2&B_2
	\end{pmatrix}
	\end{pmatrix}\]
\end{proposition}

\begin{proof}
	Докажем, что $\overline{v}$ "--- направляющий вектор прямой пересечения. Во-первых, он ненулевой, поскольку из непараллельности $\nu_1$ и $\nu_2$ следует, что хотя бы один из определителей, указанных в координатном столбце $\overline{v}$, ненулевой. Во-вторых, $\overline{v} \parallel \nu_1$ и $\overline{v} \parallel \nu_2$. Покажем это на примере $\nu_1$:
	\[A_1\begin{vmatrix}B_1&C_1\\B_2&C_2\end{vmatrix}+
	B_1\begin{vmatrix}C_1&A_1\\C_2&A_2\end{vmatrix}+
	A_1\begin{vmatrix}A_1&B_1\\A_2&B_2\end{vmatrix} = 
	\begin{vmatrix}A_1&B_1&C_1\\A_1&B_1&C_1\\A_2&B_2&C_2\end{vmatrix}\]
	
	Заметим теперь, что определитель матрицы не меняется при транспонировании:
	\[\begin{vmatrix}A_1&B_1&C_1\\A_1&B_1&C_1\\A_2&B_2&C_2\end{vmatrix} = \begin{vmatrix}A_1&A_1&A_2\\B_1&B_1&B_2\\C_1&C_1&C_2\end{vmatrix} = 0\]
	
	Данный определитель равен нулю, поскольку он соответствует ориентированному объему от тройки векторов, среди которых есть два одинаковых. Значит, $\overline{v}$ удовлетворяет критерию параллельности $\nu_1$. Аналогично показывается, что $\overline{v} \parallel \nu_2$.
\end{proof}

\begin{note}
	Каждая плоскость в пространстве делит его на два \textit{полупространства}. Пусть в прямоугольной декартовой системе координат плоскость $\nu$ задана уравнением $Ax+By+Cz+D = 0 \Leftrightarrow (\overline{r} \hm{-} \overline{r_0}, \overline{n}) = 0$, причем вектор $\overline{n}$ направлен в полупространство $V$ относительно плоскости $\nu$ (считаем, что $V$ "--- открытое, то есть не содержит саму плоскость $\nu$). Тогда $X(\overline{r}) \in V \Leftrightarrow (\overline{r} \hm{-} \overline{r_0}, \overline{n}) > 0 \hm{\Leftrightarrow} Ax+By+Cz+D > 0$. Противоположное полупространство задается противоположным неравенством.
	
	В произвольной декартовой системе координат неравенство $A'x'\hm{+}B'y'+C'z' +D' > 0$ также задает одно из полупространств относительно $\nu$, поскольку при переходе в исходную декартову систему координат выражение $A'x'+B'y'+C'z+D'$ переходит в $Ax\hm{+}By+Cz+D$ (поскольку оба они равны одному и тому же скалярному произведению).
\end{note}

\begin{definition}
	\textit{Пучком пересекающихся плоскостей} называется множество всех плоскостей, проходящих через данную прямую.
\end{definition}

\begin{definition}
	\textit{Пучком параллельных плоскостей} называется множество всех плоскостей, параллельных данной плоскости.
\end{definition}

\begin{note}
	Любые две прямые плоскости ровно в одном пучке.
\end{note}

\begin{theorem}
	Пусть в произвольной декартовой системе координат плоскости $\nu_1 \ne \nu_2$ заданы уравнениями $A_1x+B_1y+C_1z+D_1=0$, $A_2x+B_2y+C_2z+D_2=0$. Тогда плоскости, лежащие в том же пучке, что и $\nu_1$ и $\nu_2$, "--- это плоскости, задаваемые уравнениями вида:
	\[\alpha_1(A_1x+B_1y+C_1z+D_1) + \alpha_2(A_2x+B_2y+C_2z+D_2) = 0\]
\end{theorem}

\begin{proof}
	Пусть плоскость $\nu$ задается данным уравнением. Если $\nu_1 \cap \nu_2 = l$, то координаты точки каждой точки на прямой $l$ удовлетворяют данному уравнению, то есть $l \subset \nu$. Если же $\nu_1 \parallel \nu_2$, то из данного уравнения сопутствующий вектор $\nu$ параллелен сопутствующим векторам $\nu_1$ и $\nu_2$ (в этом случае уравнение выше задает плоскость не всегда, но если и задает, то лежащую в данном пучке).
	
	Теперь пусть плоскость $\nu$ лежит в данном пучке. Если $\nu_1 \cap \nu_2 = l$, то выберем на $\nu$ точку $Q \not\in l$ с координатами $\begin{pmatrix}x_0\\y_0\\z_0\end{pmatrix}$. Тогда $Q$ удовлетворяет уравнению с коэффициентами $\alpha_1 = A_2x_0+B_2y_0+C_2z_0+D_0$, $\alpha_1 = -(A_1x_0+B_1y_0+C_1z_0+D_2)$, причем хотя бы один из коэффициентов ненулевой (так как $Q$ лежит не более, чем на одной из плоскостей $\nu_1$, $\nu_2$). Тогда данное уравнение задает $\nu$, так как ему удовлетворяют все точки прямой и точка, не лежащая на этой прямой. Если же $\nu \parallel \nu_1 \parallel \nu_2$, то аналогичным образом можно выбрать любую точку $Q \in \nu$ и соответствующие коэффициенты, тогда данное уравнение задает $\nu$ при условии, что оно задает плоскость, но это верно всегда, поскольку множество его решений содержит $Q$ (то есть непусто) и не содержит хотя бы одну из плоскостей $\nu_1$, $\nu_2$.
\end{proof}

\begin{proposition}
	Пусть плоскости $\nu_1$, $\nu_2$ заданы в прямоугольной системе координат $(O, e)$ уравнениями $A_1x\hm{+}B_1y+C_1z+D_1=0$, $A_2x\hm{+}B_2y\hm{+}C_2z+D_2=0$. Тогда угол $\phi$ между ними удовлетворяет равенству:
	\[\cos{\phi} = \frac{|A_1A_2+B_1B_2+C_1C_2|}{\sqrt{A_1^2+B_1^2+C_1^2}\sqrt{A_2^2+B_2^2+C_2^2}}\]
\end{proposition}

\begin{proof}
	Если $\alpha$ "--- угол между нормальными векторами данных плоскостей $\overline{n_1} \leftrightarrow_{e} \begin{pmatrix}A_1\\B_1\\C_1\end{pmatrix}$, $\overline{n_2} \leftrightarrow_{e} \begin{pmatrix}A_2\\B_2\\C_2\end{pmatrix}$, то $\phi$ равен меньшему из углов $\alpha$ и $\pi - \alpha$. В каждом из случаев справедливо следующее:
	\[\cos{\phi} = |\cos{\angle(\overline{n_1}, \overline{n_2})}| = \frac{|A_1A_2+B_1B_2+C_1C_2|}{\sqrt{A_1^2+B_1^2+C_1^2}\sqrt{A_2^2+B_2^2+C_2^2}}\]
\end{proof}

\begin{proposition}
	Пусть плоскость $\nu$ задана в прямоугольной системе $(O, e)$ координат уравнением $Ax+By+Cz+D=0$, точка $M(\overline{r_0})$ имеет координаты $\begin{pmatrix}x_0\\y_0\\z_0\end{pmatrix}$. Тогда расстояние $\rho$ от $M$ до $\nu$ равно:
	\[\rho = \frac{|Ax_0 + By_0 + Cz_0+D|}{\sqrt{A^2 + B^2+C^2}}\]
\end{proposition}

\begin{proof}
	Пусть $\overline{n}$ "--- вектор нормали к $\nu$, $X(\overline{r})$ "--- произвольная точка на $l$. Тогда:
	\[\rho = |\pr_{\overline{n}}(\overline{r_0} - \overline{r})| = \left|\frac{(\overline{r_0} - \overline{r}, \overline{n})}{|\overline{n}|^2}\overline{n}\right| = \left|\frac{(\overline{r_0} - \overline{r}, \overline{n})}{|\overline{n}|}\right| = \frac{|Ax_0 + By_0 + Cz_0+D|}{\sqrt{A^2 + B^2+C^2}}\]
\end{proof}

\begin{proposition}
	Пусть плоскости $\nu_1 \parallel \nu_2$ заданы в прямоугольной системе координат $(O, e)$ уравнениями $Ax+By+Cz\hm{+}D_1=0$, $Ax\hm{+}By\hm{+}Cz+D_2=0$. Тогда расстояние $\rho$ между ними равно:
	\[\rho = \frac{|D_1 - D_2|}{\sqrt{A^2 + B^2 + C^2}}\]
\end{proposition}

\begin{proof}
	Пусть $M_1(\overline{r_1})$, $M_2(\overline{r_2})$ "--- произвольные точки на $\nu_1$ и $\nu_2$ соответственно. Так как $\overline{n_1} = \overline{n_2} = \overline{n}$, $(\overline{r_1}, \overline{n_1}) = -D_1$, $(\overline{r_2}, \overline{n_2}) = -D_2$:
	\[\rho = |\pr_{\overline{n}}(\overline{r_1} - \overline{r_2})| = \left|\frac{(\overline{r_1}, \overline{n_1}) - (\overline{r_2}, \overline{n_2})}{|\overline{n}|}\right| = \frac{|D_1 - D_2|}{\sqrt{A^2 + B^2 + C^2}}\]
\end{proof}

\subsection{Прямая в пространстве}

Рассмотрим различные способы задания прямой $l$ в пространстве. Пусть $M (\overline{r_0})$ "--- точка на прямой $l$, $\overline{a}$ "--- направляющий вектор прямой ($\overline{a} \parallel l$, $\overline{a} \ne 0$). Тогда $X (\overline{r}) \in l \Leftrightarrow \overline{MX} \text{ коллинеарен } \overline{a}$. Введем базис $e$ и декартову систему координат $(O, e)$:
\[M \leftrightarrow_{(O, e)}
\begin{pmatrix}
x_0\\y_0\\z_0
\end{pmatrix},~
X \leftrightarrow_{(O, e)}
\begin{pmatrix}
x\\y\\z
\end{pmatrix},~
\overline{a} \leftrightarrow_{e}
\begin{pmatrix}
\alpha_1\\\alpha_2\\\alpha_3
\end{pmatrix}\]

\begin{definition} \textit{Векторно-параметрическим уравнением прямой} называется следующее уравнение:
	\[\overline{r} = \overline{r_0} + t\overline{a}~(t \in \mathbb{R})\]
\end{definition}

\begin{definition} \textit{Параметрическим уравнением прямой} называется следующая система:
	\[\left\{
	\begin{aligned}
	x = x_0 + t\alpha_1\\
	y = y_0 + t\alpha_2\\
	z = z_0 + t\alpha_3
	\end{aligned}
	\right.
	\]
\end{definition}

\begin{definition}
	\textit{Каноническим уравнением прямой} называется следующее уравнение:
	\[\frac{x - x_0}{\alpha_1} = \frac{y - y_0}{\alpha_2} = \frac{z - z_0}{\alpha_3}\]
\end{definition}

\begin{note}
	Если, например, $\alpha_1 = 0$ (тогда $\alpha_2 \ne 0$ или $\alpha_3 \ne 0$), то считаем, что $\forall t\in\mathbb{R}: x = x_0$.
\end{note}

\begin{definition}
	\textit{Векторным уравнением прямой} называется следующее уравнение:
	\[[\overline{r} - \overline{r_0}, \overline{a}] = \overline{0} \Leftrightarrow [\overline{r}, \overline{a}] = \overline{b}\]
\end{definition}

\begin{note}
	В пространстве прямую также можно задать как пересечение двух плоскостей.
\end{note}

\begin{note}
	Рассмотренные способы задания прямой позволяют определить \textit{взаимное расположение прямых в пространстве}. Пусть прямые $l_1, l_2$ заданы векторно-параметрическими уравнениями $\overline{r} \hm{=} \overline{r_1} + t\overline{a_1}$, $\overline{r} = \overline{r_2} + t\overline{a_2}$. Тогда:
	\begin{itemize}
		\item $l_1 \parallel l_2 \Leftrightarrow [\overline{a_1}, \overline{a_2}] = \overline{0} \text{ и } [\overline{a_1}, \overline{r_2} - \overline{r_1}] \ne \overline{0}$
		\item $l_1 = l_2 \Leftrightarrow [\overline{a_1}, \overline{a_2}] = \overline{0} \text{ и } [\overline{a_1}, \overline{r_2} - \overline{r_1}] = \overline{0}$
		\item $l_1 \cap l_2 \Leftrightarrow (\overline{a_1}, \overline{a_2}, \overline{r_2} - \overline{r_1}) = 0 \text{ и } [\overline{a_1}, \overline{a_2}] \ne \overline{0}$
		\item $l_1 \text{ и } l_2 \text{ скрещиваются} \Leftrightarrow (\overline{a_1}, \overline{a_2}, \overline{r_2} - \overline{r_1}) \ne 0$
	\end{itemize}
\end{note}

\begin{note}
	Рассмотренные способы задания прямой и плоскости позволяют определить \textit{взаимное расположение прямой и плоскости в пространстве}. Пусть плоскость $\nu$ задана общим уравнением $Ax + By + Cz + D = 0$, прямая $l$ задана векторно-параметрическим уравнением $\overline{r} \hm{=} \overline{r_0} + t\overline{a}$, $\overline{r_0} \leftrightarrow_{e} \begin{pmatrix}
	x_0\\y_0\\z_0\end{pmatrix}$, $\overline{a} \leftrightarrow_{e} \begin{pmatrix}
	\alpha_1\\\alpha_2\\\alpha_3\end{pmatrix}$. Тогда:
	\begin{itemize}
		\item $l \cap \nu \Leftrightarrow \overline{a} \nparallel \nu \Leftrightarrow A\alpha_1 + B\alpha_2 + C\alpha_3 \ne 0$
		\item $l \parallel \nu \text{ и } l \not\subset \nu \Leftrightarrow
		\left\{\begin{aligned}
		&A\alpha_1 + B\alpha_2 + C\alpha_3 = 0\\
		&Ax_0 + By_0 + Cz_0 + D \ne 0
		\end{aligned}\right.$
		\item $l \subset \nu \Leftrightarrow
		\left\{\begin{aligned}
		&A\alpha_1 + B\alpha_2 + C\alpha_3 = 0\\
		&Ax_0 + By_0 + Cz_0 + D = 0
		\end{aligned}\right.$
	\end{itemize}
\end{note}

\begin{proposition}
	Пусть прямая $l$ задана в прямоугольной системе координат $(O, e)$ векторно-параметрическим уравнением $\overline{r} = \overline{r_0} + \overline{a}t$, точка $A(\overline{r_A})$ не лежит на данной прямой. Тогда расстояние $\rho$ от $A$ до $l$ равно:
	\[\rho = \frac{|[\overline{r_A} - \overline{r_0}, \overline{a}]|}{|\overline{a}|}\]
\end{proposition}

\begin{proof}
	Искомое расстояние $\rho$ является высотой параллелограма, построенного на векторах $\overline{r_A} - \overline{r_0}$ и $\overline{a}$, что и дает требуемую формулу.
\end{proof}

\begin{proposition}
	Пусть скрещивающиеся прямые $l_1, l_2$ заданы в прямоугольной системе координат $(O, e)$ уравнениями $\overline{r} = \overline{r_1} \hm{+} \overline{a_1}t$, $\overline{r} = \overline{r_2} + \overline{a_2}t$. Тогда расстояние $\rho$ между ними равно:
	\[\rho = \frac{|(\overline{a_1}, \overline{a_2}, \overline{r_1} - \overline{r_2})|}{|[\overline{a_1}, \overline{a_2}]|}\]
\end{proposition}

\begin{proof}
	Искомое расстояние $\rho$ является высотой параллелепипеда, построенного на векторах $\overline{a_1}$, $\overline{a_2}$ и $\overline{r_1} - \overline{r_2}$ (эти векторы гарантированно некомпланарны), что и дает требуемую формулу.
\end{proof}