\section{Алгебраические кривые}

\subsection{Многочлены}

\begin{definition}
	\textit{Одночленом} (\textit{мономом}) от переменных $x, y, \dots$ называется функция вида:
	\[f(x, y, \dots) = \alpha x^ky^l\dots (\alpha \in \mathbb{R}, k, l \in \mathbb{N} \cup \{0\})\]
\end{definition}

\begin{definition}
	\textit{Многочленом} (\textit{полиномом}) от переменных $x, y, \dots$ называется линейная комбинация одночленов от $x, y, \dots$.
\end{definition}

\begin{definition}
	\textit{Несократимой записью} многочлена $P$ называется представление $P$ в виде линейной комбинации одночленов, которые попарно непропорциональны:
	\[P(x,y,\dots) = \sum_{i}\alpha_ix^{k_i}y^{l_i}\dots (\text{наборы } k_i, l_i, \dots \text{ попарно различны})\]
\end{definition}

\begin{proposition}
	Если несократимая запись многочлена $P$ от переменных $x_1, \dots, x_n$ содержит хотя бы один моном, то $P \ne 0$.
\end{proposition}

\begin{proof}
	Докажем утверждение индукцией по числу переменных $n$.
	
	База: если $n = 1$, то многочлен $P(x)$ имеет конечное число корней, тогда $\exists a \in \mathbb{R}: P(a) \ne 0$.
	
	Переход: сгруппируем в $P(x_1, \dots, x_n)$ мономы с одинаковой степенью $x_n$:
	\[P(x_1, \dots, x_n) = \sum_{j = 0}^dQ_j(x_1,\dots, x_{n-1})x_n^j\]
	
	Хотя бы один из $Q_j$ имеет ненулевую несократимую запись. Пусть это $Q_t$. Тогда по предположению индукции можно выбрать $a_1,\dots,a_{n-1}$ такие, что $Q_t(a_1, \dots, a_{n-1}) \ne 0$. Тогда:
	\[P(a_1, \dots, a_{n-1}, x_n) = \sum_{j = 0}^dQ_j(a_1,\dots,a_{n-1})x_n^j\]
	
	Рассматривая полученное выражение как многочлен от $x_n$ с ненулевой несократимой записью, можно снова применить базу индукции: $\exists a_n \in \mathbb{R}:P(a_1, \dots, a_{n-1}, a_n) \ne 0$.
\end{proof}

\begin{corollary}
	Несократимая запись многочлена единственна.
\end{corollary}

\begin{proof}
	Предположим, что у $P(x_1,\dots, x_n)$ есть две различных несократимых записи $P_1$ и $P_2$. Несократимая запись разности $P_1 - P_2$ по предположению содержит хотя бы один моном, но, т.\:к. это были равные выражения, эта запись должна равняться нулю "--- противоречие.
\end{proof}

\begin{definition}
	\textit{Степенью одночлена} $\alpha x^ky^l \dots (\alpha \ne 0)$ называется число $k + l + \dots$.
\end{definition}

\begin{definition}
	\textit{Степенью многочлена} называется наибольшая степень одночлена, входящего в его несократимую запись. Обозначение "--- $\deg{P}$. Будем также считать, что $\deg{0} = -\infty$.
\end{definition}

\begin{proposition}
	$\deg{(P + Q)} \le \max(\deg{P}, \deg{Q})$.
\end{proposition}

\begin{proof}
	Сложим несократимые записи $P$ и $Q$. Приводя подобные слагаемые, получим несократимую запись $P+Q$. В ней не будет мономов степени, превосходящей $\max(\deg{P}, \deg{Q})$.
\end{proof}

\begin{proposition}
	$\deg{(PQ)} = \deg{P} + \deg{Q}$.
\end{proposition}

\begin{proof}
	Перемножим несократимые записи $P$ и $Q$, получим сумму мономов степени, не превосходящей $d$, поэтому $\deg{(PQ)} \hm{\le} \deg{P} + \deg{Q}$. Далее рассмотрим в несократимой записи $P$ моном $ax_1^{\alpha_1}\dots x_n^{\alpha_n}$ $(a \ne 0)$, удовлетворяющий условиям:
	\begin{enumerate}
		\item $\alpha_1 + \dots + \alpha_n = \deg{P}$ (моном наибольшей степени).
		\item Среди всех мономов, удовлетворяющих пункту 1, показатель степени $\alpha_1$ у данного монома наибольший.
		\item Среди всех мономов, удовлетворяющих пункту 2, показатель степени $\alpha_2$ у данного монома наибольший, и т.\:д.
	\end{enumerate}
	
	Аналогичным образом выберем в $Q$ моном $bx_1^{\beta_1}\dots x_n^{\beta_n} (b \ne 0)$. Произведение выбранных мономов дает моном $abx_1^{\alpha_1+\beta_1}\dots x_n^{\alpha_n+\beta_n}$ $(ab \hm{\ne} 0)$. Если моном с такими же показателями степеней появился как произведение $cx_1^{\gamma_1}\dots x_n^{\gamma_n}$ $(c \ne 0)$ из $P$ и $dx_1^{\delta_1}\dots x_n^{\delta_n} (d \ne 0)$ из $Q$, то:
	\begin{enumerate}
		\item $\gamma_1 + \dots + \gamma_n \le \alpha_1 + \dots + \alpha_1$ и $\delta_1 + \dots + \delta_n \le \beta_1 + \dots + \beta_1$, тогда из равенства степеней мономов в обоих случаях имеет место равенство.
		\item $\gamma_1 \le \alpha_1$ и $\delta_1 \le \beta_1$, тогда из равенства степеней $x_1$ в обоих случаях имеет место равенство.
		\item $\gamma_2 \le \alpha_2$ и $\delta_2 \le \beta_2$, тогда из равенства степеней $x_2$ в обоих случаях имеет место равенство, и т.\:д.
	\end{enumerate}
	
	Рассуждая таким образом получим, что все степени в данных парах мономов совпадают, тогда, т.\:к. мы рассматривали несократимые записи, совпадают и эти мономы. Значит, при после приведения подобных слагаемых моном $abx_1^{\alpha_1+\beta_1}\dots x_n^{\alpha_n+\beta_n} (ab \hm{\ne} 0)$ степени $\deg{P} + \deg{Q}$ сократиться не мог.
\end{proof}

\begin{theorem}
	Рассмотрим многочлен $P(x, y, z)$ от координат точки в произвольной декартовой системе координат в пространстве $P_3$. Если при замене системы координат из $P(x, y, z)$ была получена функция $Q(x', y', z')$, то $Q$ "--- тоже многочлен, причем $\deg{Q} \hm{=} \deg{P}$.
\end{theorem}

\begin{proof}
	Воспользуемся формулой замены координат:
	\[\begin{pmatrix}x\\y\\z\end{pmatrix} = S\begin{pmatrix}x'\\y'\\z'\end{pmatrix} + \gamma\]
	
	Каждая из переменных $x, y, z$ заменяется на сумму линейной комбинации переменных $x', y', z'$ и некоторой константы. При подстановке этих выражений в $P(x, y, z) = Q(x', y', z')$ получится многочлен, причем степень каждого монома $\alpha x^iy^jz^k$ сохранится. Но при приведении подобных слагаемых степень выражения может уменьшиться, поэтому $\deg{Q} \le \deg{P}$. Но, т.\:к. можно перейти обратно к переменным $x, y, z$ и получить из $Q(x', y', z')$ многочлен $P(x, y, z)$, то $\deg{P} \le \deg{Q}$. Значит, $\deg{P} = \deg{Q}$.
\end{proof}

\begin{note}
	Аналогичное утверждение верно и для двумерного пространства $P_2$.
\end{note}

\begin{definition}
	\textit{Алгебраической кривой} называется множество всех точек в $P_2$, координаты которых в некоторой декартовой системе координат удовлетворяют уравнению $P(x, y) = 0$, где $P$ "--- фиксированный многочлен. \textit{Порядком кривой} называется наименьшая степень многочлена, задающего данную кривую.
	
	Аналогичным образом можно определить \textit{алгебраические поверхности} и их порядок в $P_3$.
\end{definition}

\begin{note}
	Порядок алгебраической кривой не зависит от выбора системы координат.
\end{note}

\begin{note}
	Алгебраическая кривая первого порядка "--- это прямая, алгебраическая поверхность первого порядка "--- это плоскость.
\end{note}

\begin{proposition}
	Объединение и пересечение алгебраических кривых также являются алгебраическими кривыми.
\end{proposition}

\begin{proof}
	Пусть две кривые задаются многочленами $P_1(x, y)$ и $P_2(x, y)$ соответственно. Тогда их объединение и пересечение задаются следующим образом:
	\[
	\left[
	\begin{aligned}
	P_1(x, y) = 0\\
	P_2(x, y) = 0
	\end{aligned}
	\right. \Leftrightarrow P_1(x, y)P_2(x, y) = 0\]
	\[
	\left\{
	\begin{aligned}
	P_1(x, y) = 0\\
	P_2(x, y) = 0
	\end{aligned}
	\right. \Leftrightarrow (P_1(x, y))^2 + (P_2(x, y))^2 = 0
	\]
	
	Видно, что оба полученных выражения также являются многочленами.
\end{proof}

\begin{proposition}
	Сечение алгебраической поверхности плоскостью "--- это алгебраическая кривая.
\end{proposition}

\begin{proof}
	Перейдем в такую систему координат, в которой плоскость будет задаваться уравнением $z = 0$. Тогда, если алгебраическая поверхность задана уравнением $P(x, y, z) = 0$, то уравнение сечения имеет вид $P(x, y, 0) = 0$, т.\:е. это алгебраическая кривая.
\end{proof}

\subsection{Кривые второго порядка}

\begin{definition}
	\textit{Кривой второго порядка} называется алгебраическая кривая, которая в некоторой прямоугольной декартовой системе координат задается уравнением:
	\[Ax^2 + 2Bxy + Cy^2 + 2Dx + 2Ey + F = 0~(A^2 + B^2 + C^2 \ne 0)\]
\end{definition}

\begin{theorem}
	Любое уравнение кривой второго порядка в некоторой прямоугольной декартовой системой координат имеет один из девяти следующих видов:
	\begin{itemize}
		\item Кривые эллиптического типа:
		\begin{itemize}
			\item $\frac{x^2}{a^2} + \frac{y^2}{b^2} = 1$ "--- \textit{эллипс}
			\item $\frac{x^2}{a^2} + \frac{y^2}{b^2} = 0$ "--- \textit{точка}
			\item $\frac{x^2}{a^2} + \frac{y^2}{b^2} = -1$ "--- \textit{мнимый эллипс}
		\end{itemize}
		$(a \ge b > 0)$
		\item Кривые гиперболического типа:
		\begin{itemize}
			\item $\frac{x^2}{a^2} - \frac{y^2}{b^2} = 1$ "--- \textit{гипербола}
			\item $\frac{x^2}{a^2} - \frac{y^2}{b^2} = 0$ "--- \textit{пара пересекающихся прямых}
		\end{itemize}
		$(a  > 0, b > 0)$
		\item Кривые параболического типа:
		\begin{itemize}
			\item $y^2 = 2px$ "--- \textit{парабола}
			\item $\frac{y^2}{a^2} = 1$ "--- \textit{пара параллельных прямых}
			\item $\frac{y^2}{a^2} = 0$ "--- \textit{пара совпадающих прямых}
			\item $\frac{y^2}{a^2} = -1$ "--- \textit{пара мнимых параллельных прямых}
		\end{itemize}
		$(p > 0, a > 0)$
	\end{itemize}
\end{theorem}

\begin{proof}
	Изначально имеется уравнение $Ax^2 + 2Bxy + Cy^2 \hm{+} 2Dx + 2Ey + F = 0$. Процесс перехода в требуемую систему координат можно разделить на три этапа:
	\begin{enumerate}
		\item Избавимся от монома $2Bxy$ (если он ненулевой). Для этого произведем поворот системы координат на угол $\alpha$ против часовой стелки. При этом матрица перехода $S$ ($e' = eS \Leftrightarrow \alpha = S\alpha'$) имеет вид:
		\[S = \begin{pmatrix}\cos{\alpha}&-\sin{\alpha}\\\sin{\alpha}&\cos{\alpha}\end{pmatrix} \Rightarrow \left\{\begin{aligned}
		&x = x'\cos{\alpha}-y'\sin{\alpha}\\
		&y = x'\sin{\alpha}+y'\cos{\alpha}
		\end{aligned}\right.\]
		Определим, чему должно быть равно $\alpha$, чтобы коэффициент при $x'y'$ стал нулевым:
		\[-2A\sin{\alpha}\cos{\alpha} + 2B(\cos^2{\alpha} - \sin^2{\alpha}) + 2C\sin{\alpha}\cos{\alpha} = 0\]
		\[2B\cos{2\alpha} = (A - C)\sin{2\alpha}\]
		Если $A = C$, выберем $\alpha = \frac{\pi}{4}$, иначе "--- такой $\alpha$, что $\tg{2\alpha} = \frac{2B}{A - C}$. В обоих случаях в новой системе координат гарантированно получаем выражение $A'x'^2 + C'y'^2 + 2D'x' +2E'y' + F' = 0$.
		\item Избавимся от монома вида $2Dx'$ (если $A' \ne 0$). Для этого сместим начало системы координат так, чтобы:
		\[\left\{\begin{aligned}
		&x'' = x' - \frac{D'}{A'}\\
		&y'' = y'
		\end{aligned}\right.\]
		После этого мы получим выражение $A''x''^2+C''y''^2+2E''y'' \hm{+} F'' = 0$.
		\item Аналогичным образом избавимся от монома вида $2E''y''$ (если $C'' \ne 0$), снова сместив начало системы координат:
		\[\left\{\begin{aligned}
		&x''' = x''\\
		&y''' = y'' - \frac{E''}{C''}\end{aligned}\right.\]
	\end{enumerate}
	
	Для удобства опустим штрихи в записи уравнения в полученной системе координат. После того, как произведены описанные выше операции, могут быть получены три различных результата:
	\begin{enumerate}
		\item $AC > 0$ (это также означает, что ни один из мономов $x^2$, $y^2$ не сократился). Тогда полученное уравнение имеет вид:
		\[Ax^2 + Cy^2 + F = 0\]
		Если $A, C < 0$, домножим уравнение на $-1$. Перенесем $F$ в другую часть и, если $F \ne 0$, разделим уравнение на $|F|$. Возможные результаты:
		\[\frac{x^2}{a^2} + \frac{y^2}{b^2} = 
		\left[\begin{aligned}
		1\\0\\-1
		\end{aligned}\right.~~\left(a = \sqrt{\frac{|F|}{|A|}}, b = \sqrt{\frac{|F|}{|C|}}\right)
		\]
		Если $a < b$, то поменяем координаты местами. Таким образом, получены уравнения кривых эллиптического типа.
		\item $AC < 0$ (это также означает, что ни один из мономов $x^2$, $y^2$ не сократился). Тогда полученное уравнение имеет вид:
		\[Ax^2 + Cy^2 + F = 0\]
		Аналогичными описанным в предыдущем пункте преобразованиями, получим один из результаты:
		\[\frac{x^2}{a^2} - \frac{y^2}{b^2} = 
		\left[\begin{aligned}
		1\\0\\
		\end{aligned}\right.
		\]
		Таким образом, получены уравнения кривых гиперболического типа.
		\item $AC = 0$. Т.\:к. рассматриваемая кривая "--- второго порядка, то одно из чисел $A$, $C$ осталось ненулевым. Заменой системы координат можно добиться того, чтобы это было число $C$. Тогда полученное уравнение имеет вид:
		\[Cy^2+2Dx+F = 0\]
		Если $D \ne 0$, то можно смещением начала координат избавиться от $F$ и получить уравнение:
		\[y^2 = 2px\]
		Если $D = 0$, то возможны три случая:
		\[\frac{y^2}{a^2} = 
		\left[\begin{aligned}
		1\\0\\-1
		\end{aligned}\right.
		\]
		Таким образом, получены уравнения кривых параболического типа.
	\end{enumerate}
	
	Итак, заменой системы координат и алгебраическими преобразованиями уравнение любой кривой второго порядка приводится к одному из девяти видов.
\end{proof}

\begin{definition}
	\textit{Канонической системой координат} для кривой второго порядка называется такая прямоугольная декартова система координат, в которой данная кривая задается одним из девяти описанных выше уравнений.
\end{definition}

\begin{definition}
	\textit{Центром многочлена} $P(x, y)$ в произвольной декартовой системе координат $(O, e)$ называется такая точка $A \in P_2$, $A \leftrightarrow_{(O, e)} \begin{pmatrix}\alpha\\\beta\end{pmatrix}$, что:
	\[\forall x \in \mathbb{R}, \forall y \in \mathbb{R} : P(\alpha - x, y -\beta) = P(\alpha + x, y + \beta)\]
\end{definition}

\begin{proposition}
	Если в некоторой декартовой системе координат $(O, e)$ точка $A \leftrightarrow_{(O, e)} \begin{pmatrix}\alpha\\\beta\end{pmatrix}$ "--- центр многочлена $P(x, y)$, то $A$ "--- центр симметрии кривой, заданной уравнением $P(x, y) = 0$.
\end{proposition}

\begin{proof}
	Из условия, что $A$ "--- центр многочлена $P(x, y)$ следует, что точки $A_1 \leftrightarrow_{(O, e)} \begin{pmatrix}\alpha - x\\\beta - y\end{pmatrix}$, $A_2 \leftrightarrow_{(O, e)} \begin{pmatrix}\alpha + x\\\beta + y\end{pmatrix}$, симметричные относительно $A$, или одновременно принадлежат данной кривой, или одновременно не принадлежат ей.
\end{proof}

\begin{note}
	Можно показать, что обратное утверждение тоже верно.
\end{note}

\begin{note}
	Начало координат в канонической системе координат некоторой кривой второго порядка является ее центром симметрии, если центр симметрии существует.
\end{note}

\begin{proposition}
	Пусть в произвольной декартовой системе координат $(O, e)$ дана точка $A \leftrightarrow_{(O, e)} \begin{pmatrix}\alpha\\\beta\end{pmatrix}$ и многочлен $P(x, y) \hm{=} Ax^2 + 2Bxy + Cy^2 + 2Dx + 2Ey + F$. Тогда:
	\[A \text{ "--- центр многочлена } P(x, y) \Leftrightarrow \left\{\begin{aligned}
	A\alpha + B\beta + D = 0\\
	B\alpha + C\beta + E = 0
	\end{aligned}\right.\]
\end{proposition}

\begin{proof}
	Доказательство производится непосредственной проверкой.
\end{proof}

\begin{definition}
	Кривая второго порядка называется \textit{центральной}, если у нее существует единственный центр. Этому условию удовлетворяют кривые эллиптического и гиперболического типов.
\end{definition}

\begin{note}
	Согласно правилу Крамера, кривая второго порядка центральна $\Leftrightarrow$ $\begin{vmatrix}
	A&B\\
	B&C
	\end{vmatrix} \ne 0$.
\end{note}

\subsection{Эллипс, гипербола и парабола}
\begin{definition}
	\textit{Эллипсом} называется кривая второго порядка, которая в некоторой прямоугольной декартовой системе координат задается уравнением $\frac{x^2}{a^2} + \frac{y^2}{b^2} = 1~(a \ge b > 0)$. Дальнейшие определения, связанные с эллипсом, даются именно в этой системе координат.
\end{definition}

\begin{definition}
	\textit{Вершинами} эллипса называются точки с координатами $\begin{pmatrix}\pm a\\0\end{pmatrix}$, $\begin{pmatrix}0\\\pm b\end{pmatrix}$. Число $a$ называется \textit{длиной большой полуоси} эллипса, $b$ "--- \textit{длиной малой полуоси} эллипса.
\end{definition}

\begin{definition}
	\textit{Фокусным расстоянием} эллипса называется величина:
	\[c = \sqrt{a^2 - b^2}\]
	
	Точки $F_1 \leftrightarrow_{(O, e)} \begin{pmatrix}c\\0\end{pmatrix}$, $F_2 \leftrightarrow_{(O, e)} \begin{pmatrix}-c\\0\end{pmatrix}$ называются \textit{фокусами} эллипса.
\end{definition}

\begin{definition}
	\textit{Эксцентриситетом} эллипса называется величина:
	\[\epsilon = \frac{c}{a} < 1\]
\end{definition}

\begin{theorem}
	Точка $A \leftrightarrow_{(O, e)}\begin{pmatrix}x\\y\end{pmatrix}$ лежит на эллипсе $\Leftrightarrow$ $AF_1 \hm{=} |a - \epsilon  x|$ $\Leftrightarrow$ $AF_2 = |a + \epsilon x|$.
\end{theorem}

\begin{proof}
	\[AF_1^2 - |a - \epsilon x|^2 = (x - c)^2 + y^2 - |a - \epsilon x|^2 = b^2\left(\frac{x^2}{a^2}+\frac{y^2}{b^2} - 1\right)\]
	Значит, $AF_1 = |a - \epsilon x| \Leftrightarrow
	\frac{x^2}{a^2}+\frac{y^2}{b^2} = 1$. Вторая равносильность показывается аналогично.
\end{proof}

\begin{definition}
	\textit{Директрисами} эллипса называются прямые $d_1: x = \frac{a}{\epsilon} = \frac{a^2}{c}$ и $d_2: x = -\frac{a}{\epsilon} = -\frac{a^2}{c}$.
\end{definition}

\begin{theorem}
	Эллипс "--- геометрическое место точек $A \leftrightarrow_{(O, e)} \begin{pmatrix}x\\y\end{pmatrix}$ таких, что:
	\[\frac{AF_1}{\rho(A, d_1)} = \frac{AF_2}{\rho(A, d_2)} = \epsilon\]
\end{theorem}

\begin{proof}
	\[\rho(A, d1) = |x - \frac{a}{\epsilon}| = \frac{1}{\epsilon}|a - \epsilon x|\]
	
	Уже было доказано, что A лежит на эллипсе $\Leftrightarrow$ $|a - \epsilon x| = AF_1$, поэтому утверждение данной теоремы также верно. Вторая равносильность показывается аналогично.
\end{proof}

\begin{theorem}
	Эллипс "--- геометрическое место точек $A \leftrightarrow_{(O, e)} \begin{pmatrix}x\\y\end{pmatrix}$ таких, что:
	\[AF_1 + AF_2 = 2a\]
\end{theorem}

\begin{proof}
	Если $A$ лежит на эллипсе, то $AF_1 = a - \epsilon x$, $AF_2 = a + \epsilon x$, тогда $AF_1 + AF_2 = 2a$.
	
	Напротив, если $AF_1 + AF_2 = 2a$, то $A$ лежит на эллипсе, поскольку при движении точки $X \leftrightarrow_{(O, e)} \begin{pmatrix}x_0\\0\end{pmatrix}$ вдоль прямой $x = x_0$ вверх (или вниз) величина $XF_1 + XF_2$ строго монотонно возрастает, тогда при $|x_0| < a$ таких точек, что $XF_1 + XF_2 = 2a$, на каждой прямой две, при $|x_0| = a$ "--- одна, а при $|x_0| \hm{>} a$ таких точек нет. Это множество точек совпадает с множеством точек эллипса.
\end{proof}

\begin{definition}
	\textit{Гиперболой} называется кривая второго порядка, которая в некоторой прямоугольной декартовой системе координат задается уравнением $\frac{x^2}{a^2} - \frac{y^2}{b^2} = 1~(a > 0, b > 0)$. Дальнейшие определения, связанные с гиперболой, даются именно в этой системе координат.
\end{definition}

\begin{definition}
	\textit{Вершинами} гиперболы называются точки с координатами $\begin{pmatrix}\pm a\\0\end{pmatrix}$. Число $a$ называется \textit{длиной действительной полуоси} гиперболы, $b$ "--- \textit{длиной мнимой полуоси} гиперболы.
\end{definition}

\begin{definition}
	\textit{Фокусным расстоянием} гиперболы называется величина:
	\[c = \sqrt{a^2 + b^2}\]
	
	Точки $F_1 \leftrightarrow_{(O, e)} \begin{pmatrix}c\\0\end{pmatrix}$, $F_2 \leftrightarrow_{(O, e)} \begin{pmatrix}-c\\0\end{pmatrix}$ называются \textit{фокусами} гиперболы.
\end{definition}

\begin{definition}
	\textit{Эксцентриситетом} гиперболы называется величина:
	\[\epsilon = \frac{c}{a} > 1\]
\end{definition}

\begin{theorem}
	Точка $A \leftrightarrow_{(O, e)}\begin{pmatrix}x\\y\end{pmatrix}$ лежит на гиперболе $\Leftrightarrow$ $AF_1 \hm{=} |a - \epsilon  x|$ $\Leftrightarrow$ $AF_2 = |a + \epsilon x|$.
\end{theorem}

\begin{proof}
	\[AF_1^2 - |a - \epsilon x|^2 = (x - c)^2 + y^2 - |a - \epsilon x|^2 = b^2\left(\frac{x^2}{a^2}-\frac{y^2}{b^2} - 1\right)\]
	
	Значит, $AF_1 = |a - \epsilon x| \Leftrightarrow \frac{x^2}{a^2}-\frac{y^2}{b^2} = 1$. Вторая равносильность показывается аналогично.
\end{proof}

\begin{definition}
	\textit{Директрисами} гиперболы называются прямые $d_1: x = \frac{a}{\epsilon} = \frac{a^2}{c}$ и $d_2: x = -\frac{a}{\epsilon} = -\frac{a^2}{c}$.
\end{definition}

\begin{theorem}
	Гипербола "--- геометрическое место точек $A \leftrightarrow_{(O, e)} \begin{pmatrix}x\\y\end{pmatrix}$ таких, что:
	\[\frac{AF_1}{\rho(A, d_1)} = \frac{AF_2}{\rho(A, d_2)} = \epsilon\]
\end{theorem}

\begin{proof}
	\[\rho(A, d1) = |x - \frac{a}{\epsilon}| = \frac{1}{\epsilon}|a - \epsilon x|\]
	
	Уже было доказано, что A лежит на гиперболе $\Leftrightarrow$ $|a - \epsilon x| = AF_1$, поэтому утверждение данной теоремы также верно. Вторая равносильность показывается аналогично.
\end{proof}

\begin{theorem}
	Гипербола "--- геометрическое место точек $A \leftrightarrow_{(O, e)} \begin{pmatrix}x\\y\end{pmatrix}$ таких, что:
	\[|AF_1 - AF_2| = 2a\]
\end{theorem}

\begin{proof}
	Если $A$ лежит на гиперболе (без ограничения общности, на правой ее ветви), то $AF_1 = \epsilon x - a$, $AF_2 = a + \epsilon x$, тогда $|AF_1 - AF_2| = 2a$.
	
	Напротив, если $|AF_1 - AF_2| = 2a$, то $A$ лежит на гиперболе, поскольку при движении точки $X \leftrightarrow_{(O, e)} \begin{pmatrix}x_0\\0\end{pmatrix}$ вдоль прямой $x = x_0$ вверх (или вниз) величина $|XF_1 - XF_2|$ строго монотонно убывает (если $x_0 \ne 0$), тогда при $|x_0| > a$ таких точек, что $|XF_1 - XF_2| = 2a$, на каждой прямой две, при $|x_0| = a$ "--- одна, а при $|x_0| \hm{<} a$ таких точек нет. Это множество точек совпадает с множеством точек эллипса.
\end{proof}

\begin{definition}
	\textit{Асимптотами} гиперболы называются прямые $l_1: \frac{x}{a} - \frac{y}{b} = 0$ и $l_2: \frac{x}{a} + \frac{y}{b} = 0$.
\end{definition}

\begin{proposition}
	Пусть $A$ "--- точка на гиперболе. Тогда:
	\[\rho(A, l_1)\rho(A, l_2) = \frac{a^2b^2}{a^2 + b^2}\]
\end{proposition}

\begin{proof}
	Пусть $A \leftrightarrow_{(O, e)} \begin{pmatrix}x\\y\end{pmatrix}$. По формуле расстояния от точки до прямой в плоскости:
	\begin{gather*}
	\rho(A, l_1) = \frac{\left|\frac{x}{a}-\frac{y}{b}\right|}{\sqrt{\frac{1}{a^2} + \frac{1}{b^2}}} = \frac{|bx - ay|}{\sqrt{a^2 + b^2}}\\
	\rho(A, l_2) = \frac{\left|\frac{x}{a}+\frac{y}{b}\right|}{\sqrt{\frac{1}{a^2} + \frac{1}{b^2}}} = \frac{|bx + ay|}{\sqrt{a^2 + b^2}}\\
	\rho(A, l_1)\rho(A, l_2) = \frac{b^2x^2 - a^2y^2}{a^2 + b^2} = \frac{a^2b^2\left(\frac{x^2}{a^2} - \frac{y^2}{b^2}\right)}{a^2 + b^2} = \frac{a^2b^2}{a^2 + b^2}
	\end{gather*}
\end{proof}

\begin{corollary}
	Если точка $A$ движется по одной полуветви гиперболы так, что $x \rightarrow \infty$, то расстояние от $A$ до одной из асимптот $\rightarrow 0$.
\end{corollary}

\begin{proof}
	Пусть без ограничения общности $A$ движется так, что $x \rightarrow +\infty$, $y \rightarrow +\infty$. Тогда $\rho(A, l_2) \rightarrow +\infty$. Т.\:к. $\rho(A, l_1)$ и $\rho(A, l_2)$ обратно пропорциональны, то $\rho(A, l_1) \rightarrow 0$.
\end{proof}

\begin{definition}
	\textit{Параболой} называется кривая второго порядка, которая в некоторой прямоугольной декартовой системе координат задается уравнением $y^2 = 2px~(p > 0)$. Дальнейшие определения, связанные с параболой, даются именно в этой системе координат.
\end{definition}

\begin{definition}
	\textit{Вершиной} параболы называется точка с координатами $\begin{pmatrix}0\\0\end{pmatrix}$.
\end{definition}

\begin{definition}
	\textit{Фокусом} гиперболы называется точка с координатами $\begin{pmatrix}\frac{p}{2}\\0\end{pmatrix}$.
\end{definition}

\begin{definition}
	\textit{Эксцентриситетом} гиперболы называется величина:
	\[\epsilon = 1\]
\end{definition}

\begin{definition}
	\textit{Директрисой} параболы называется прямая $d: x = - \frac{p}{2}$.
\end{definition}

\begin{theorem}
	Точка $A \leftrightarrow_{(O, e)}\begin{pmatrix}x\\y\end{pmatrix}$ лежит на параболе $\Leftrightarrow$ $AF \hm{=} \rho(A, d)$.
\end{theorem}

\begin{proof}
	\[AF^2 - \rho^2(A, d) = \left(x - \frac{p}{2}\right)^2 + y^2 - \left(x + \frac{p}{2}\right)^2 = y^2 - 2px\]
	
	Значит, $AF = \rho(A, d) = |x + \frac{p}{2}|$ $\Leftrightarrow$ $y^2 = 2px$.
\end{proof}

\subsection{Сопряженные диаметры и касательные}

\begin{theorem}
	Пусть $C$ "--- некоторая кривая второго порядка (эллипс, гипербола или парабола), заданная в своей канонической системе координат $(O, e)$, $\overline{v} \xleftrightarrow[e]{} \begin{pmatrix}\alpha\\\beta\end{pmatrix}$ "--- выбранное направление. Тогда центры всех хорд данной кривой, параллельных $\overline{v}$, лежат на одной прямой.
\end{theorem}

\begin{proof}
	Рассмотрим случай, когда $C$ "--- гипербола. Пусть $A \leftrightarrow_{(O, e)} \begin{pmatrix}x_0\\y_0\end{pmatrix}$ "--- середина некоторой хорды, параллельной $\overline{v}$. Рассмотрим прямую, заданную параметрическим уравнением, содержащую данную хорду:
	\[\left\{\begin{aligned}
	x = x_0 + \alpha t\\
	y = y_0 + \beta t
	\end{aligned}\right.~(t \in \mathbb{R})\]
	
	Точки пересечения этой прямой с заданной гиперболой удовлетворяют уравнению:
	\[\frac{(x_0 + \alpha t)^2}{a^2} - \frac{(y_0 + \beta t)^2}{b^2} = 1\]
	
	Т.\:к. точка $A$ является серединой хорды, то значения параметра $t$, удовлетворяющие уравнению, должны быть противоположными числами, т.\:е. если привести данное уравнение к виду квадратного относительно $t$, по теореме Виета коэффициент при $t$ должен быть равен нулю. Преобразовав уравнение, получим значение данного коэффициента:
	\[\alpha b^2 x_0 - \beta a^2 y_0 = 0\]
	
	Таким образом, центры всех хорд, параллельных $\overline{v}$ удовлетворяют уравнению прямой:
	\[\frac{\alpha x}{a^2} - \frac{\beta y}{b^2} = 0\]
	
	Аналогичным образом получаются уравнения соответствующих прямых для случев эллипса и параболы.
\end{proof}

\begin{definition}
	\textit{Диаметром}, сопряженным к направлению $\overline{v}$ относительно кривой $C$, называется прямая, содержащая середины всех хорд $C$, параллельных $\overline{v}$.
\end{definition}

\begin{note}
	Пусть в системе координат $(O, e)$, канонической для $C$, $\overline{v} \xleftrightarrow[e]{} \begin{pmatrix}\alpha\\\beta\end{pmatrix}$. Тогда уравнения диаметров, сопряженных к направлению $\overline{v}$, имеют вид:
	\begin{itemize}
		\item $C$ "--- эллипс: $\frac{\alpha x}{a^2} + \frac{\beta y}{b^2} = 0$, направляющий вектор $\overline{a} \xleftrightarrow[e]{} \begin{pmatrix}\frac{\beta}{b^2}\\-\frac{\alpha}{a^2}\end{pmatrix}$
		\item $C$ "--- гипербола: $\frac{\alpha x}{a^2} - \frac{\beta y}{b^2} = 0$, направляющий вектор $\overline{a} \xleftrightarrow[e]{} \begin{pmatrix}\frac{\beta}{b^2}\\\frac{\alpha}{a^2}\end{pmatrix}$
		\item $C$ "--- парабола: $\beta y = \alpha p$, направляющий вектор $\overline{a} \xleftrightarrow[e]{} \begin{pmatrix}1\\0\end{pmatrix}$
	\end{itemize}
\end{note}

\begin{proposition}
	Пусть $C$ "--- эллипс или гипербола. Если диаметр, сопряженный к $\overline{v}$, имеет направляющий вектор $\overline{u}$, то диаметр, сопряженный к $\overline{u}$, имеет направляющий вектор $\overline{v}$ (т.\:е. сопряжение обладает свойством двойственности).
\end{proposition}

\begin{proof}
	Рассмотрим случай, когда $C$ "--- гипербола. Пусть $\overline{v} \xleftrightarrow[e]{} \begin{pmatrix}\alpha\\\beta\end{pmatrix}$. Диаметр, сопряженный к направлению $\overline{v}$, имеет направляющий вектор $\overline{u} \xleftrightarrow[e]{} \begin{pmatrix}\frac{\beta}{b^2}\\\frac{\alpha}{a^2}\end{pmatrix}$. Диаметр, сопряженный к направлению $\overline{u}$, имеет направляющий вектор $\overline{w} \xleftrightarrow[e]{} \begin{pmatrix}\frac{\alpha}{a^2b^2}\\\frac{\beta}{a^2b^2}\end{pmatrix}$. Остается заметить, что $\overline{w} \parallel \overline{v}$. Аналогичным образом получается результат для случая эллипса.
\end{proof}

\begin{note}
	Для произвольной кривой второго порядка также можно определить сопряженные диаметры и получить соответствующие уравнения прямых.
\end{note}

\begin{definition}
	\textit{Касательной} к кривой в точке $A$ называется предельное положение секущей $AB$ при $B \rightarrow A$.
\end{definition}

\begin{proposition}
	Пусть $C$ "--- эллипс или гипербола. Касательная к $C$ в точке $A$ параллельна диаметру, сопряженному к направлению диаметра, содержащего $A$.
\end{proposition}

\begin{proof}
	Рассмотрим случай, когда $C$ "--- гипербола. Пусть $A \leftrightarrow_{(O, e)} \begin{pmatrix}x_0\\y_0\end{pmatrix}$. Когда точка $B$ на гиперболе стремится к $A$, середина хорды $AB$ также стремится к $A$. Поэтому диаметр, содержащий середину хорды $AB$, в предельном случае проходит через $A$. Диаметр, сопряженный к данному, параллелен $AB$ в силу двойственности, и в предельном случае его направление совпадает с направлением касательной. Рассуждения в случае эллипса аналогичны.
\end{proof}

\begin{corollary}
	Уравнения касательных к $C$ (эллипсу или гиперболе) в точке $A \leftrightarrow_{(O, e)} \begin{pmatrix}x_0\\y_0\end{pmatrix}$ имеют вид:
	\begin{itemize}
		\item $C$ "--- эллипс: $\frac{x_0x}{a^2} + \frac{y_0y}{b^2} = 1$
		\item $C$ "--- гипербола: $\frac{x_0x}{a^2} - \frac{y_0y}{b^2} = 1$
	\end{itemize}
\end{corollary}

\begin{proposition}
	Уравнение касательной к параболе $C$ в точке $A \leftrightarrow_{(O, e)} \begin{pmatrix}x_0\\y_0\end{pmatrix}$ имеет вид:
	\[y_0y = p(x_0 + x)\]
\end{proposition}

\begin{proof}
	Диаметр, проходящий через $A$, задается уравнением $y = y_0$. Если касательная в точке $A$ имеет направляющий вектор $\overline{v} \xleftrightarrow[e]{} \begin{pmatrix}\alpha\\\beta\end{pmatrix}$, то $\beta y_0 = \alpha p$. Если $X \leftrightarrow_{(O, e)} \begin{pmatrix}x\\y\end{pmatrix}$ "---произвольная точка на касательной, то:
	\[\frac{x - x_0}{y - y_0} = \frac{\alpha}{\beta} = \frac{y_0}{p}\]
	
	Преобразуя полученное равенство с учетом того, что $y_0^2 = 2px_0$, получим:
	\[y_0y = p(x_0 + x)\]
\end{proof}

\begin{note}
	В каждом из случаев эллипса, параболы и гиперболы диаметр, сопряженный к направлению касательной в точке $A$, проходит через $A$, поэтому достаточно из уравнения диаметра определить, какому направлению он соответствует и учесть, что касательная проходит через точку $A$.
\end{note}