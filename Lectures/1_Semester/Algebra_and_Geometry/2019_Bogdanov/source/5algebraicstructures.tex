\section{Алгебраические структуры}

\subsection{Группы}

\begin{definition}
	\textit{Группой} называется множество $G$ с введенной на нем бинарной операцией $\cdot: G \times G \rightarrow G$ ($(a, b) \mapsto a\cdot b = ab$), удовлетворяющей следующим свойствам:
	\begin{itemize}
		\item $\forall a, b, c \in G : (ab)c = a(bc)$ (ассоциативность)
		\item $\exists e \in G: \forall a \in G: ae = ea = a$ (существование нейтрального элемента)
		\item $\forall a \in G: \exists a^{-1} \in G: aa^{-1} = a^{-1}a = e$ (существование обратного элемента)
	\end{itemize}
\end{definition}

\begin{proposition}
	Пусть $(G, \cdot)$ "--- группа. Нейтральный элемент $e$ в группе единственен.
\end{proposition}

\begin{proof}
	Пусть $e$ и $e'$ "--- нейтральные элементы в $G$. Тогда:
	\[e = ee' = e'\]
\end{proof}

\begin{proposition}
	Обратный элемент в $(G, \cdot)$ единственен: $\forall a \hm{\in} G: \exists! a^{-1} \in G$. Более того, $\forall b, c \in G: ba = ac = e \rightarrow b = c$.
\end{proposition}

\begin{proof}
	Сначала докажем вторую часть утверждения:
	\[b = be = b(ac) = (ba)c = ec = c\]
	
	Первая часть утверждения является прямым следствием доказанного.
\end{proof}

\begin{proposition}
	В группе можно сокращать: 
	\[\forall a, b, c, \in G: ab = ac \rightarrow b = c\]
\end{proposition}

\begin{proof}
	Домножив обе части слева на $a^{-1}$, получим требуемое.
\end{proof}

\begin{definition}
	Группа называется $(G, \cdot)$ \textit{абелевой}, если умножение в ней коммутативно:
	\[\forall a, b \in G: ab = ba\]
\end{definition}

\begin{example}
	Абелевыми группами являются:
	\begin{itemize}
		\item $(\mathbb{Z}, +)$, $(\mathbb{Q}, +)$, $(\mathbb{R}, +)$, $(\mathbb{C}, +)$ (в отличие от $(\mathbb{N}, +)$, в котором нет обратных элементов)
		\item $(M_{n \times k}, +)$, $(V_i, +)$
		\item $(\mathbb{Q} \backslash \{0\}, \cdot) \equiv (\mathbb{Q}^*, \cdot)$, $(\mathbb{R} \backslash \{0\}, \cdot) \equiv (\mathbb{R}^*, \cdot)$, $(\mathbb{C} \backslash \{0\}, \cdot) \equiv (\mathbb{C}^*, \cdot)$
	\end{itemize}
\end{example}

\begin{example}
	Неабелевой группой является \textit{группа перестановок} $(S_n, \circ)$ (при условии, что $n \ge 3$):
	\[S_n = \{\sigma: \{1,\dots, n\} \rightarrow \{1,\dots, n\}~|~\sigma \text{ "--- биекция}\}\]
	
	Перестановка $\sigma$ часто обозначается как:
	\[\sigma = \begin{pmatrix}
	1&2&\dots&n\\
	\sigma(1)&\sigma(2)&\dots&\sigma(n)
	\end{pmatrix}\]
	
	Ассоциативность в $(S_n, \circ)$ выполнена: $(\sigma \circ \tau \circ \kappa) (k) = \sigma(\tau(\kappa(k)))$ независимо от расстановки скобок, т.\:к. $(\sigma \circ \tau)(k) = \sigma(\tau(k))$. Нейтральным элементом является \textit{тождественная перестановка} ($\id$). Существование обратного элемента также верно: $\forall \sigma \hm{\in} S_n: \exists \sigma^{-1} \hm{\in} S_n$, поскольку $\sigma$ "--- биекция.
\end{example}

\begin{definition}
	\textit{Суммой} множеств $A$ и $B$ называется следующее множество:
	\[A + B = \{a + b~|~a \in A, b \in B\}\]
	
	Данное определение позволяет ввести еще одну группу.
\end{definition}

\begin{definition}
	Числа $a, b \in \mathbb{Z}$ называются \textit{сравнимыми по модулю $n$}, если $n~|~(a - b)$. Обозначение "--- $a \equiv_n b$. Сравнимость является отношением эквивалентности. $\mathbb{Z}_n$ "--- это множество соответствующих классов эквивалентности, $|\mathbb{Z}_n| = n$. Класс, которому принадлежит $a \in \mathbb{Z}$ обозначается как $\overline{a}$ или $a + n\mathbb{Z}$.
\end{definition}

\begin{example}
	Абелевой группой является $(\mathbb{Z}_n, +)$.
	
	Сложение в $\mathbb{Z}_n$ определено как сложение множеств:
	\[(a + n\mathbb{Z}) + (b + n\mathbb{Z}) = (a + b) + n\mathbb{Z} \Leftrightarrow \overline{a} + \overline{b} = \overline{a + b}\]
	
	Ассоциативность, коммутативность, существование нейтрального и обратного элементов в $\mathbb{Z}_n$ выполнены как следствия соответствующих свойств сложения в $\mathbb{Z}$.
\end{example}

\begin{definition}
	\textit{Подгруппой} группы $G$ называется такое непустое ее подмножество $H$, что:
	\begin{itemize}
		\item $\forall a, b \in H: ab \in H$
		\item $\forall a \in H : a^{-1} \in H$
	\end{itemize}
	
	$H$ является группой с той же операцией, что и в $G$.
\end{definition}

\subsection{Кольца}

\begin{definition}
	\textit{Кольцом} называется множество $R$ с введенными на нем двумя бинарными операциями $+$ и $\cdot$, удовлетворяющими следующим условиям:
	\begin{itemize}
		\item $(R, +)$ "--- абелева группа, нейтральный элемент в которой обозначается как $0$
		\item $\forall a, b, c \in R: (ab)c = a(bc)$ (ассоциативность умножения)
		\item $\forall a, b, c \in R: a(b + c) = ab + ac, (a + b)c = ac + bc$ (дистрибутивность умножения относительно сложения)
		\item $\exists 1 \in R: \forall a \in R: a1 = 1a = a$ (существование нейтрального элемента относительно умножения)
	\end{itemize}
\end{definition}

\begin{definition}
	Кольцо называется $(R, +, \cdot)$ \textit{коммутативным}, если умножение в нем коммутативно:
	\[\forall a, b \in R: ab = ba\]
\end{definition}

\begin{proposition}
	$\forall a \in R: a0 = 0a = 0$.
\end{proposition}

\begin{proof}
	\begin{gather*}
	a0 + a0 = a(0 + 0) = a0\\
	a0 + a0 + (-a0) = a0 + (-a0)\\
	a0 = 0
	\end{gather*}
\end{proof}

\begin{example}
	Коммутативными кольцами являются:
	\begin{itemize}
		\item $(\mathbb{Z}, +, \cdot)$, $(\mathbb{Q}, +, \cdot)$, $(\mathbb{R}, +)$, $(\mathbb{C}, +, \cdot)$ (в отличие от $(\mathbb{N}, +, \cdot)$, поскольку $(\mathbb{N}, +)$ "--- не группа)
		\item $\mathbb{Z}[\sqrt{2}] = \{a + b\sqrt{2}~|~a, b \in \mathbb{Z}\}$
		\item $\mathbb{R}[x, y, \dots]$ "--- многочлены от переменных $x, , \dots$ (с коэффициентами из $\mathbb{R}$)
	\end{itemize}
\end{example}

\begin{example}
	Некоммутативным кольцом является $M_{n \times n}$ (при условии, что $n \ge 2$). Более того, некоммутативным кольцом также является $M_{n \times n}(R)$ "--- множество матриц с элементами из произвольного кольца, т.\:е. необязательно из $\mathbb{R}$.
\end{example}

\begin{proposition}
	$(\mathbb{Z}_n, +, \cdot)$ является кольцом, если определить умножение следующим образом: $\forall \overline{a}, \overline{b} \in \mathbb{Z}_n: \overline{a}\cdot\overline{b} = \overline{ab}$.
\end{proposition}

\begin{proof}
	Сначала проверим, что такое определение умножения корректно. Пусть $a' \in \overline{a}, b' \in \overline{b}$, тогда $a' = a + kn$, $b' = b + ln$ $(k, l \in \mathbb{Z})$. В таком случае $a'b' = ab + n(la' + kb' + kln)$, т.\:е. $a'b' \in \overline{ab}$.
	
	Ассоциативность умножения, дистрибутивность и существование нейтрального элемента относительно умножения в $\mathbb{Z}_n$ выполнены как следствия соответствующих свойств в $\mathbb{Z}$.
\end{proof}

\begin{definition}
	\textit{Подкольцом} кольца $R$ называется такое непустое его подмножество $S$, что:
	\begin{itemize}
		\item $(S, +)$ "--- подгруппа в $(R, +)$
		\item $\forall a, b \in S: ab \in S$
		\item $1 \in S$
	\end{itemize}

$S$ является кольцом с теми же операциями, что и в $R$.
\end{definition}

\subsection{Поля}

\begin{definition}
	Пусть $R$ "--- кольцо. Элемент $a \in R$ называется \textit{обратимым} (относительно умножения), если $\exists a^{-1} \in R: aa^{-1} \hm{=} a^{-1}a = 1$.
\end{definition}

\begin{definition}
	\textit{Группой обратимых элементов} кольца $R$ называется множество $R^*$ его обратимых элементов.
\end{definition}

\begin{proposition}
	$(R^*, \cdot)$ "--- группа.
\end{proposition}

\begin{proof} Множество $R^*$ непусто, поскольку $1 \in R^*$ (элемент $1$ является обратным сам себе). Умножение в $R^*$ определено корректно, поскольку если $a, b \in R^*$, то $\exists a^{-1}, b^{-1} \in R$, тогда $\exists (ab)^{-1} = b^{-1}a^{-1} \in R$, т.\:е. $ab \in R^*$. Проверим теперь свойства группы:
	\begin{itemize}
		\item $\forall a, b, c \in R^*: (ab)c = a(bc)$ (как следствие ассоциативности в $R$)
		\item $\exists 1 \in R^*: \forall a \in R^*: a1 = 1a = a$
		\item $\forall a \in R^*: \exists a^{-1} \in R^*: aa^{-1} = a^{-1}a = 1$ (оба элемента $a$ и $a^{-1}$ обратимы, поскольку являются обратными друг другу)
	\end{itemize}
\end{proof}

\begin{definition}
	\textit{Полем} называется множество $F$ такое, что $(F, +, \cdot)$ "--- коммутативное кольцо и $F^* = F\backslash\{0\}$.
\end{definition}

\begin{example}
	Полями являются:
	\begin{itemize}
		\item $\mathbb{Q}$, $\mathbb{R}$, $\mathbb{C}$
		\item $\mathbb{Q}[\sqrt{2}] = \{a + b\sqrt{2}~|~a, b \in \mathbb{Q}\}$
	\end{itemize}
\end{example}

\begin{proposition}
	$\mathbb{Z}_n$ "--- поле $\Leftrightarrow$ $n$ "--- простое число.
\end{proposition}

\begin{proof}
	Если $n = 1$, то $0$ и $1$ в $\mathbb{Z}_n$ совпадают, а $\mathbb{Z}_n^* = \mathbb{Z}_n$ "--- это не удовлетворяет определению поля.
	
	Пусть $n$ "--- составное число, $n = ab$ ($a, b \in \mathbb{N}$, $a, b > 1$). Тогда $\overline{a}, \overline{b} \ne \overline{0}$, но при этом $\overline{a}\cdot\overline{b} = \overline{0}$. Из этого следует, что $\overline{a} \not\in \mathbb{Z}_n^*$: если $\overline{a} \in \mathbb{Z}_n^*$, то, домножив слева на $\overline{a}^{-1}$, можно получить, что $\overline{b} = \overline{0}$, а это не так. Значит, в случае, когда $n$ составное, $\mathbb{Z}_n^*$ не может являться полем.
	
	Пусть $n$ "--- простое число. Докажем, что $\forall \overline{a} \in \mathbb{Z}_n \backslash \{0\}: \overline{a} \in \mathbb{Z}_n^*$. Рассмотрим числа $a, 2a, \dots, na$. Все они дают разные остатки при делении на $n$: если это было не так, то для некоторых $k, l \hm{\in} \{1, \dots, n\}$ ($k \ne l$) выполнено, что $n|(k-l)a$, а это невозможно. Следовательно, $\exists m \in \{1, \dots, n\}: am \equiv_n 1$, т.\:е. $\exists \overline{a}^{-1} = \overline{m}$.
\end{proof}

\begin{definition}
	\textit{Подполем} поля $F$ называется такое непустое его подмножество $S$, что:
	\begin{itemize}
		\item $(S, +, \cdot)$ "--- подкольцо в $(R, +, \cdot)$
		\item $\forall a \in S\backslash\{0\}: a^{-1} \in S$
	\end{itemize}
	
	$S$ является полем с теми же операциями, что и в $F$.
\end{definition}

\begin{definition}
	Пусть $F_1$ и $F_2$ "--- поля. \textit{Изоморфизмом} полей $F_1$ и $F_2$ называется такое отображение $\phi : F_1 \rightarrow F_2$, что:
	\begin{itemize}
		\item $\phi(a + b) = \phi(a) + \phi(b)$, $\phi(a \cdot b) = \phi(a) \cdot \phi(b)$ ($\phi$ сохраняет операции)
		\item $\phi$ "--- биекция
	\end{itemize}
	
	Если такой изоморфизм $\phi$ существует, то поля $F_1$ и $F_2$ называются \textit{изоморфными}. Обозначение "--- $F_1 \cong F_2$.
\end{definition}

\begin{proposition}
	Изоморфизм полей обладает следующими свойствами:
	\begin{itemize}
		\item $\phi(0) = 0$
		\item $\phi(1) = 1$
		\item $\phi(-a) = -\phi(a)$
		\item $\phi(a^{-1}) = (\phi(a))^{-1}$ $(a \ne 0)$
	\end{itemize}
\end{proposition}

\begin{proof}~
	\begin{itemize}
		\item $\phi(0) = \phi(0 + 0) = \phi(0) + \phi(0) \Rightarrow \phi(0) = 0$
		\item $\phi(1) = \phi(1\cdot1) = \phi(1)\cdot\phi(1) \Rightarrow \phi(1) = 1$ (на $\phi(1)$ можно сократить, потому что условие биективности гарантирует, что $\phi(1) \ne \phi(0) = 0$)
		\item $\phi(0) = \phi(a + (-a)) = \phi(a) + \phi(-a) \Rightarrow \phi(-a) = -\phi(a)$
		\item $\phi(1) = \phi(aa^{-1}) = \phi(a)\phi(a^{-1}) \Rightarrow \phi(a^{-1}) = (\phi(a))^{-1}$ $(a \ne 0)$
	\end{itemize}
\end{proof}

\begin{note}
	В любом поле $F$ можно определить целое число $n$:
	\begin{itemize}
		\item $n > 0$: $n_F = \underbrace{1 + \dots + 1}_{n}$
		\item $n = 0$: $n_F$ уже определено как нейтральный элемент относительно сложения
		\item $n < 0$: $n_F = -(-n)_F = -(\underbrace{1 + \dots + 1}_{-n})$, где $(-n) > 0$
	\end{itemize}

	Как следствие, $n_F + k_F = (n + k)_F$, $n_F\cdot k_F = (n\cdot k)_F$.
\end{note}

\begin{definition}
	Пусть $F$ "--- поле. Его \textit{характеристикой} называется наименьшее $k \in \mathbb{N}$ такое, что $k_F = 0$. Если такого $k$ нет, принято считать характеристику равной нулю. Обозначение "--- $\cha{F}$.
\end{definition}

\begin{proposition}
	Если $\cha{F} > 0$, то $\cha{F}$ "--- простое число.
\end{proposition}

\begin{proof}
	Пусть $\cha{F} = n$. Если $n = 1$, то $0$ и $1$ в $F$ совпалают, а $F^* = F$ "--- это не удовлетворяет определению поля.
	
	Пусть $n$ "--- составное число, $n = ab$ ($a, b \in \mathbb{N}$, $a, b > 1$). Тогда $a_F, b_F \ne 0$, но при этом $a_F\cdot b_F = 0$. Из этого следует, что $a_F \not\in F^*$: если $a_F \in F^*$, то, домножив слева на $a_F^{-1}$, можно получить, что $b_F = 0$, а это не так. Значит, случай составного $n$ невозможен.
\end{proof}

\begin{definition}
	Поле называется \textit{простым}, если оно не имеет подполей, отличных от него самого.
\end{definition}

\begin{theorem}
	Пусть $F$ "--- поле. Тогда:
	\begin{enumerate}
		\item Если $\cha{F} = p > 0$, то в $F$ существует простое подполе, изоморфное $\mathbb{Z}_p$.
		\item Если $\cha{F} = 0$, то в $F$ существует простое подполе, изоморфное $\mathbb{Q}$.
	\end{enumerate}
\end{theorem}

\begin{proof}~
	\begin{enumerate}
		\item Пусть $\cha{F} = p$. Рассмотрим $K = \{n_F \in F~|~n \in \mathbb{Z}\}$. Определим отображение $\phi: \mathbb{Z}_p \rightarrow K$ : $\phi(\overline{a}) = a_F$. Покажем, что определенное отображение корректно: если $a' \in \overline{a}$, то $a' = a + kp$ ($k \in \mathbb{Z}$) и $a'_F = a_F + (kp)_F = a_F$.
		
		Отображение $\phi$ сохраняет операции:
		\begin{gather*}
			\phi(\overline{a} + \overline{b}) = \phi(\overline{a + b}) \hm{=} (a+b)_F = a_F + b_F = \phi(\overline{a}) + \phi(\overline{b})\\
			\phi(\overline{a} \cdot \overline{b}) = \phi(\overline{a \cdot b}) = (a \cdot b)_F \hm{=} a_F \cdot b_F = \phi(\overline{a}) \cdot \phi(\overline{b})
		\end{gather*}
		
		Наконец, $\phi$ "--- биекция. Сюръективность очевидна: $\forall a_F \in K: \exists \overline{a} \in \mathbb{Z}_p: \phi(\overline{a}) = a_F$. Покажем инъективность: если $\phi(\overline{a}) \hm{=} \phi(\overline{b})$ (без ограничения общности можно считать, что $a \ge b$ и $a, b \in \{0,\dots, p-1\}$), то $(a - b)_F = \phi(\overline{a - b}) = \phi(\overline{a}) - \phi(\overline{b}) = 0$. Это возможно, только если $\overline{a} = \overline{b}$.
		
		Из доказанного также следует, что $K$ "--- подполе в $F$. Например, замкнутость относительно взятия обратного элемента по умножению можно показать, используя свойства отображения $\phi$:
		\begin{multline*}
		\forall a_F \in K, a_F \ne 0 : \exists \phi(\overline{a}^{-1}) \in K:\\
		\phi(\overline{a}^{-1}) a_F = \phi(\overline{a}^{-1})\phi(\overline{a}) = \phi(\overline{1}) = 1_F
		\end{multline*}
		
		Проверка остальных свойств подполя позволяет убедиться, что $K$ "--- поле, тогда $\phi$ "--- изоморфизм полей.
	
		\item Пусть $\cha{F} = 0$. Рассмотрим $K = \{\frac{a_F}{b_F}=a_F(b_F)^{-1}~|~a, b \hm{\in} \mathbb{Z}, b \ne 0\}$ ($(b_F)^{-1}$ существует, поскольку $b \ne 0$ и $\cha{F} = 0$). Определим отображение $\phi: \mathbb{Q} \rightarrow K$ : $\phi\left(\frac{a}{b}\right) = \frac{a_F}{b_F}$. Покажем, что определенное отображение корректно. Пусть $\frac{a}{b} = \frac{a'}{b'}$, тогда $\frac{a}{b} \hm{=} \frac{aa'}{a'b} \hm{=} \frac{aa'}{ab'} = \frac{a'}{b'}$. В то же время, $a_F(b_F)^{-1} = (ak)_F((bk)_F)^{-1}$ ($k \ne 0$), поскольку $(ak)_F((bk)_F)^{-1} = a_Fk_F(k_F)^{-1}(b_F)^{-1}$. Следовательно, $\frac{a_F}{b_F} = \frac{(aa')_F}{(a'b)_F} = \frac{(aa')_F}{(ab')_F} = \frac{a'_F}{b'_F}$.
		
		Отображение $\phi$ сохраняет операции:
		\begin{multline*}
		\phi\left(\frac{a}{b} + \frac{c}{d}\right) = \phi\left(\frac{ad + bc}{bd}\right) = (ad + bc)_F((bd)_F)^{-1} =\\
		= (ad)_F)((bd)_F)^{-1} + ((bc)_F)((bd)_F)^{-1} =\\ = \phi\left(\frac{ad}{bd}\right) + \phi\left(\frac{bc}{bd}\right) = \phi\left(\frac{a}{b}\right) + \phi\left(\frac{c}{d}\right)
		\end{multline*}
		
		\begin{multline*}
		\phi\left(\frac{a}{b}\cdot\frac{c}{d}\right) = \phi\left(\frac{ac}{bd}\right) = (ac)_F((bd)_F)^{-1} =\\
		= a_F(b_F)^{-1}c_F(d_F)^{-1} = \phi\left(\frac{a}{b}\right)\phi\left(\frac{c}{d}\right)
		\end{multline*}
			
		Наконец, $\phi$ "--- биекция. Сюръективность очевидна: $\forall a_F(b_F)^{-1} \hm{\in} K: \exists \frac{a}{b} \in \mathbb{Q}: \phi(\frac{a}{b}) = a_F(b_F)^{-1}$. Покажем инъективность: если $\phi\left(\frac{a}{b}\right) \hm{=} \phi\left(\frac{c}{d}\right)$, то $\phi\left(\frac{ad - bc}{bd}\right) = (ad - bc)_F((bd)_F)^{-1} = 0$. Это возможно, только если $ad - bc = 0$, т.\:е. $\frac{a}{b} = \frac{c}{d}$.
		
		Из доказанного также следует, что $K$ "--- подполе в $F$. Например, замкнутость относительно взятия обратного элемента по умножению можно показать, используя свойства отображения $\phi$:
			\begin{multline*}
			\forall a_F(b_F)^{-1} \in K, a_F \ne 0 : \exists \phi\left(\frac{b}{a}\right) \in K:\\
			a_F(b_F)^{-1}\phi\left(\frac{b}{a}\right) = \phi\left(\frac{a}{b}\right)\phi\left(\frac{b}{a}\right) = \phi\left(\frac{1}{1}\right) = 1_F(1_F)^{-1}
			\end{multline*}
			
		Проверка остальных свойств подполя позволяет убедиться, что $K$ "--- поле, тогда $\phi$ "--- изоморфизм полей.
	\end{enumerate}
\end{proof}