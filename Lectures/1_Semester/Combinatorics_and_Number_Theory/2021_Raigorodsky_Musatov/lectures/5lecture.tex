\subsection{Бинарные отношения}

\begin{definition}
	Подмножество $R \subset A \times A$ называется \textit{бинарным отношением}.
	
	$(x, y) \in R$ принято обозначать как $xRy$
\end{definition}

\subsubsection*{Свойства бинарных отношений}

\begin{itemize}
	\item Рефлексивность $\forall a \in A\ aRa$
	\item Антирефлексивность $\forall a \in A\ \neg aRa$
	\item Симметричность $\forall a, b \in A\ aRb \Ra bRa$
	\item Антисимметричность $\forall a, b \in A\ (aRb \wedge bRa) \Ra (a = b)$
	\item Транзитивность $\forall a, b, c \in A\ (aRb \wedge bRc) \Ra aRc$
	\item Антитранзитивность $\forall a, b, c \in A\ (aRb \wedge bRc) \Ra \neg aRc$
	\item Евклидовость $\forall a, b, c \in A (aRc \wedge bRc) \Ra aRb$
	\item Полнота $\forall a, b \in A (aRb \vee bRa)$
\end{itemize}

\subsection{Отношение эквивалентности}

\begin{definition}
	\textit{Отношением эквивалентности} называется бинарное отношение, которое обладает рефлексивностью, симметричностью и транзитивностью.
\end{definition}

\begin{theorem}
	Если на множестве $A$ задано отношение эквивалентности, то $A$ разбивается на классы эквивалентности - множества $A_{\alpha}$ такие, что
	\begin{itemize}
		\item $\forall x, y \in A_{\alpha} \Ra x \sim y$
		\item $\forall x \in A_{\alpha}, y \in A_{\beta}, \alpha \neq \beta \Ra x \not\sim y$
	\end{itemize}
\end{theorem}

\begin{proof}
	Пусть $K_x = \{y\ |\ y \sim x\}$
	
	\begin{enumerate}
		\item $x \in K_x$ (из рефлексивности)
		\item $K_x \cap K_y \neq \emptyset \Ra K_x = K_y$. Пусть $z \in K_x \cap K_y$. Тогда $z \sim x \Ra x \sim z$ и $z \sim y \Ra x \sim y$. Пусть $t \in K_x$, тогда $t \sim x \Ra t \sim y \Ra t \in K_y$. Отсюда $K_x \subset K_y$. Аналогично $K_y \subset K_x \Ra K_x = K_y$.
		\item $y \in K_x, z \in K_x \Ra y \sim z$
		\item $z \in K_x, t \in K_y, K_x \neq K_y \Ra z \not\sim t$. Если же $z \sim t$, то по симметричности и транзитивности $x \sim y$.
	\end{enumerate}
\end{proof}

\subsection{Фактормножество}

\begin{definition}
	\textit{Фактормножеством} $A/\sim$ называется множество классов эквивалентности по отношению $\sim$. То есть
	\begin{align*}
		&f : A \ra A/\sim, f(x) = K_x
		\\
		&x \sim y \lra f(x) = f(y)
	\end{align*}
\end{definition}

\subsection{Отношения (частичного) порядка}

\begin{definition}
	\textit{Отношением (частичного) порядка} называется бинарное отношение, которое обладает (анти)рефлексивностью, антисимметричностью и транзитивностью.
\end{definition}

\begin{definition}
	Отношением (частичного) \textit{линейного} порядка называется отношение (частичного) порядка, к которому добавили свойство полноты.
\end{definition}

\begin{definition}
	\textit{Упорядоченным множеством} называется пара из множества и отношения порядка на нём.
\end{definition}

\begin{example}
	$(A, \le_A)$
\end{example}

%%%%%%%%%%% Сюда бы диаграммы Хассе

\subsection{Изоморфизм}

\begin{definition}
	$(A, \le_A)$ изоморфно $(B, \le_B)$, если существует биекция $f : A \ra B$ такая, что $x \le_A y \lra f(x) \le_B f(y)$. Обозначается как $A \simeq B$.
\end{definition}

\subsection{Операции над упорядоченными множествами}

\begin{itemize}
	\item Сумма \\
	$(A, \le_A) + (B, \le_B) = (C, \le_C)$. $C = A \sqcup B$, $x \le_C y \lra \Unity{&{x, y \in A, x \le_A y} \\ &{x, y \in B, x \le_B y} \\ &{x \in A, y \in B}}$ %% пояснить $\sqcup$
	
	\item Произведение (используется обратный лексикографический порядок) \\
	$(A, \le_A) \cdot (B, \le_B) = (C, \le_C)$, где $C = A \times B$. $(a_1, b_1) \le_C (a_2, b_2)$, если $\Unity{&{b_1 <_B b_2} \\ &{b_1 = b_2, a_1 \le_A a_2}}$
\end{itemize}