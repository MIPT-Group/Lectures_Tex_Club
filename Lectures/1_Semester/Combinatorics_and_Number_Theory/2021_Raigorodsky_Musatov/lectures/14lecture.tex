\subsection{Степенные ряды и производящие функции}

\subsubsection*{Формальный степенной ряд}

\begin{definition}
	Мы уже рассматривали бесконечный многочлен вида:
	\[
		a_0 + a_1 x + a_2 x^2 + \ldots + a_n x^n + \ldots
	\]
	И нам не очень понравилось то, что здесь есть $x$, вместо которого возможно что-то подставить. Вместо этого, мы можем рассмотреть просто бесконечную последовательность чисел:
	\[
		(a_0, a_1, \ldots, a_n, \ldots),\ \ \forall n \in \N\ a_n \in \Cm
	\]
	Такую последовательность мы и будем называть \textit{формальным степенным рядом} (ФРС).
	
	Дополнительно введём обозначение $A_i$ для формального степенного ряда $A$, обозначающее $i$-ое число в последовательности.
\end{definition}

\subsubsection*{Арифметические операции над формальным степенным рядом}

Теперь можно определить операции над формальными степенными рядами:
\begin{itemize}
	\item Пусть есть 2 формальных степенных ряда $A = (a_0, \ldots, a_n, \ldots)$ и $B = (b_0, \ldots, b_n, \ldots)$. Тогда назовём их \textit{суммой} формальный степенной ряд $A + B$:
	\[
		A + B = (a_0 + b_0, \ldots, a_n + b_n, \ldots)
	\]
	
	\item Пусть есть 2 формальных степенных ряда $A$ и $B$. Тогда назовём их \textit{произведением} формальный степенной ряд $A \cdot B$ такой, что
	\[
		(A \cdot B)_n = A_nB_0 + A_{n - 1}B_{1} + \ldots + A_0B_n = \suml_{i = 0}^n A_{n - i}B_i = \suml_{i = 0}^n A_iB_{n - i}
	\]
	
	\item Пусть есть 2 формальных степенных ряда $A$ и $B$. Тогда возможно осуществить \textit{деление} $A / B = C$, если существует формальный степенной ряд $C$ такой, что $C \cdot B = A$. Для нахождения чисел $C$ необходимо последовательно решать уравнения из системы:
	\[
	\System{
		&{c_0 b_0 = a_0}
		\\
		&{c_0 b_1 + c_1 b_0 = a_1}
		\\
		&\vdots
		\\
		&{c_0 b_n + \ldots + c_n b_0 = a_n}
		\\
		&\vdots
	}
	\]
	Из записанных уравнений понятно, что необходимым и достаточным условием для деления является $b_0 \neq 0$.
\end{itemize}

\begin{note}
	Определив умножение, мы сразу получили возведение в натуральную степень, а также взятие натурального корня:
	\begin{itemize}
		\item \[
			A^n = \underbrace{A \cdot \ldots \cdot A}_n
		\]
		
		\item \[
			B = \sqrt[n]{A} \lra B^n = A
		\]
	\end{itemize}
\end{note}

\begin{note}
	Из определения операций видно, что если положить за единицу ряд
	\[
		1 = (1, 0, \ldots, 0, \ldots)
	\]
	А за некоторое $t$ ряд
	\[
		t = (0, 1, \ldots, 0, \ldots)
	\]
	То тогда $t^n$ будет иметь вид
	\[
		t^n = (\underbrace{0, \ldots, 0}_{n}, 1, 0, \ldots)
	\]
	Из этого следует, что любой ряд $A = (a_1, \ldots, a_n, \ldots)$ можно записать в следующем виде:
	\[
		A = (a_1, \ldots, a_n, \ldots) = a_1 \cdot 1 + a_2 \cdot t + \ldots + a_n \cdot t^n + \ldots
	\]
	То есть формальный степенной ряд - это бесконечный многочлен. Довольно легко проверить, что для формальных степенных рядов, как и для многочленов, верна ассоциативность и дистрибутивность, чем мы и воспользуемся далее.
\end{note}

\begin{example}
	Посчитаем ряд, который можно записать как $\frac{1}{(1 - x^2)^2}$:
	\[
		\frac{1}{(1 - x^2)^2} = \left(\frac{1}{1 - x}\right)^2\left(\frac{1}{1 + x}\right)^2
	\]
	где ряды под скобками запишутся как
	\begin{align*}
		&{\frac{1}{1 - x} = 1 + x + x^2 + \ldots + x^n + \ldots}
		\\
		&{\frac{1}{1 + x} = 1 - x + x^2 - \ldots + (-1)^n x^n + \ldots}
	\end{align*}
	Возведение в квадрат - это умножение ряда самого на себя. Отсюда имеем
	\begin{align*}
		&{\left(\frac{1}{1 - x}\right)^2 = 1 + 2x + 3x^2 + \ldots + (n + 1)x^n + \ldots}
		\\
		&{\left(\frac{1}{1 + x}\right)^2 = 1 - 2x + 3x^2 - \ldots + (-1)^n (n + 1)x^n + \ldots}
	\end{align*}
	Записать произведение квадратов в общем виде - можно, но не нужно. Посмотрим на коэффициент при $n$-м слагаемом:
	\[
		\left(\left(\frac{1}{1 - x}\right)^2\left(\frac{1}{1 + x}\right)^2\right)_n = \suml_{k = 0}^n (k + 1) \cdot (-1)^{n - k} (n + 1 - k)
	\]
	С другой стороны мы можем заметить, исходный ряд можно получить, если сделать подставить $x^2$ вместо $x$ для $\left(\frac{1}{1 - x}\right)^2$:
	\[
		\frac{1}{(1 - x^2)^2} = 1 + 2x^2 + 3x^4 + \ldots + (n + 1)x^{2n} + \ldots
	\]
	На нечётных местах коэффициенты обнулились, на чётных остались преждними. Коль скоро ряды равны, то коэффициенты в одинаковых местах также совпадают. Получили тождество:
	\[
		\suml_{k = 0}^n (k + 1) \cdot (-1)^{n - k} (n + 1 - k) = \System{
			&{0, \text{ если } n = 2l + 1}
			\\
			&{l + 1, \text{ если } n = 2l}
		}
	\]
\end{example}

Если рассматривать формальный степенные ряды как многочлены, то появляются ещё 2 операции, которые мы можем совершать над рядами:
\begin{itemize}
	\item Взятие производной от ряда. Если ряд $A$ имеет вид:
	\[
		f(t) = a_0 + a_1 t + a_2 t^2 + \ldots + a_n t^n + \ldots
	\]
	то его \textit{производной} назовётся ряд
	\[
		f'(t) = a_1 + 2a_2 t + \ldots + na_n t^{n - 1} + (n + 1) a_{n + 1} t^n + \ldots
	\]
	
	\item Если есть 2 формальных степенных ряда $f(t) = a_0 + a_1 t + \ldots + a_n t^n + \ldots$ и $g(t) = b_0 + b_1 t + \ldots + b_n t^n + \ldots$, то можно определить \textit{композицию} рядов $f(g(t))$:
	\[
		f(g(t)) = a_0 + a_1 g(t) + a_2 g^2(t) + \ldots + a_n g^n(t) + \ldots
	\]
	При этом, если $b_0 \neq 0$, то ряд не имеет смысла, так как свободный член будет бесконечен:
	\[
		f(g(t)) = a_0 + a_1 b_0 + a_2 b_0^2 + a_3 b_0^3 + \ldots + a_n b_0^n + \ldots
	\]
	Поэтому необходимым и достаточным условием композиции является равенство:
	\[
		b_0 = 0
	\]
\end{itemize}

\subsubsection*{Сходимость ряда}

\begin{definition}
	И всё-таки, мы можем рассмотреть формальный степенной ряд как функцию $f(x)$:
	\[
		f(x) = \suml_{k = 0}^\infty a_k x^k
	\]
	Будем говорить, что $f(x)$ \textit{сходится в точке} $x_0$, если последовательность частичных сумм этого ряда в этой точке имеет конечный предел. То есть
	\begin{align*}
		&{S_n(x_0) = \suml_{k = 0}^n a_k x_0^k}
		\\
		&{\exists \liml_{n \to \infty} S_n = S = f(x_0) \in \Cm}
	\end{align*}
\end{definition}

\begin{theorem} (Признак Коши-Адамара)
	Пусть задан формальный степенной ряд $A = (a_0, \ldots, a_n, \ldots)$, причём $\forall n \in \N\ a_n \in \R$. Обозначим за $\rho = \frac{1}{\varlimsup\limits_{k \to \infty} \sqrt[k]{|a_k|}}$. Тогда:
	\begin{itemize}
		\item Если $|x_0| < \rho$, то ряд сходится в $x_0$.
		
		\item Если $|x_0| > \rho$, то ряд расходится в $x_0$.
		
		\item Если же $|x_0| = \rho$, то может быть всё, что угодно.
	\end{itemize}
	При этом $\rho$ называется радиусом сходимости.
\end{theorem}

\begin{note}
	Взятие производной сохраняет радиус сходимости.
\end{note}

\begin{example}
	Со школьных лет известно, что
	\[
		1 + x + \ldots + x^n + \ldots = \frac{1}{1 - x},\ |x| < 1
	\]
	При этом про левую часть мы знаем, что это - формальный ряд, где $a_n = 1$. Значит, $\rho = \frac{1}{1} = 1$ - сходится с утверждением, которое заявляет нам признак Коши-Адамара.
\end{example}

\begin{definition}
	Если нам задана последовательность (ряд) $\{a_k\}_{k = 0}^\infty$, то её \textit{производящей функцией} называется
	\[
		f(x) = \suml_{k = 0}^\infty a_k x^k
	\]
\end{definition}

\begin{example} (Производящая функция чисел Фибоначчи)
	Как уже выяснено, числа Фибоначчи можно выписать в явном виде. Если положить $F_0 = 0,\ F_1 = 1$, то формула примет вид:
	\[
		F_n = \frac{1}{\sqrt{5}}\left(\frac{1 + \sqrt{5}}{2}\right)^n - \frac{1}{\sqrt{5}}\left(\frac{1 - \sqrt{5}}{2}\right)^n
	\]
	Посчитаем производяющую функцию для данной последовательности:
	\[
		f(x) = \suml_{k = 0}^\infty F_k x^k
	\]
	Мы хотим <<свернуть>> бесконечную сумму в какое-то деление конечных чисел, как для рядов, показанных выше. Посмотрим на следующие ряды:
	\begin{align*}
		&{xf(x) = F_0 x + F_1 x^2 + \ldots + F_n x^{n + 1} + \ldots}
		\\
		&{x^2f(x) = F_0 x^2 + F_1 x^3 + \ldots + F_n x^{n + 2} + \ldots}
		\\
		&{xf(x) + x^2f(x) = \underbrace{F_0 x}_{0} + \underbrace{(F_0 + F_1)}_{F_2}x^2 + \underbrace{(F_1 + F_2)}_{F_3}x^3 + \ldots = f(x) - x}
	\end{align*}
	Получили линейное уравнение относительно $f(x)$. Решив его, получим ответ:
	\[
		f(x) = \frac{x}{1 - x - x^2}
	\]
\end{example}