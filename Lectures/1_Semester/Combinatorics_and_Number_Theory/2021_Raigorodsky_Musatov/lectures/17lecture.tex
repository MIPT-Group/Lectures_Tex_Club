\begin{theorem} (Теорема Варнинга)
	Пусть $F \in \Z_p[x_1, \ldots, x_n]$, $\Deg F < n$ и при этом $N_p$ - число решений сравнения $F(x_1, \ldots, x_n) \equiv 0 \pmod p$ и $(0, \ldots, 0)$ входит в $N_p$. Тогда существует такое решение $(x_1, \ldots, x_n)$ сравнения, что
	\[
		\exists i \in [1; n] \such x_i \neq 0
	\]
	То есть существует нетривиальное решение.
\end{theorem}

\begin{proof}
	Прямое следствие теоремы Шевалле.
\end{proof}

\begin{theorem} (Обобщённая теорема Варнинга)
	Пусть $F_1, \ldots, F_k \in \Z_p[x_1, \ldots, x_n]$, $\Deg F_1 + \ldots + \Deg F_k < n$ и $(0, \ldots, 0)$ - решение системы сравнений:
	\[
	\System{
		&{F_1(x_1, \ldots, x_n) \equiv 0 \pmod p}
		\\
		&\vdots
		\\
		&{F_1(x_1, \ldots, x_n) \equiv 0 \pmod p}
	}
	\]
	Тогда утверждается, что существует и нетривиальное решение данной системы.
\end{theorem}

\begin{proof}
	\textcolor{red}{Оставляется в качестве домашнего задания читателю (не входит в курс). P.S. Возможно придумаю и допишу, не знаю.}
\end{proof}

\begin{lemma}
	Пусть есть $(a_1, b_1), \ldots, (a_{3p}, b_{3p}) \in \Z^2$ пар чисел и при этом
	\[
		\suml_{i = 1}^{3p} a_i \equiv \suml_{i = 1}^{3p} b_i \equiv 0 \pmod p
	\]
	Тогда $\exists I \subset \{1, \ldots, 3p\},\ |I| = p$, что выполнено утверждение:
	\[
		\suml_{i \in I} a_i \equiv \suml_{i \in I} b_i \equiv 0 \pmod p
	\]
\end{lemma}

\begin{proof}
	Для доказательства будем пользоваться уже сформулированными теоремами Варнинга и Шевалле. Положим за многочлен $F_1$ следующее:
	\[
		F_1(x_1, \ldots, x_n) = \suml_{i = 1}^{3p - 1} a_i \cdot x_i^{p - 1}
	\]
	Аналогично определим $F_2$ и $F_3$:
	\begin{align*}
		&{F_2(x_1, \ldots, x_{3p - 1}) := \suml_{i = 1}^{3p - 1} b_i x_i^{p - 1}}
		\\
		&{F_3(x_1, \ldots, x_{3p - 1}) := \suml_{i = 1}^{3p - 1} x_i^{p - 1}}
	\end{align*}
	В силу определения очевидно, что $\forall i \in \{1, 2, 3\}\ F_i(0, \ldots, 0) = 0$. При этом $\Deg F_1 + \Deg F_2 + \Deg F_3 = 3(p - 1) = 3p - 3 < 3p - 1$. Значит, можно применить обобщённую теорему Варнинга и заявить следующее:
	\[
		\exists (x_1, \ldots, x_{3p - 1}) \such \forall i \in \{1, 2, 3\}\ F_i(x_1, \ldots, x_{3p - 1}) \equiv 0 \pmod p
	\]
	Обозначим за $J$ множество номеров ненулевых координат. Оно непустое - в этом и суть теоремы Варнинга. Тогда заметим, что все нулевые координаты никак не влияют на значение многочлена. А отсюда, если прибавить к этому малую теорему Ферма, получается утверждение:
	\[
		\suml_{i = 1}^{3p - 1} a_i x_i^{p - 1} = \suml_{i \in J} a_i x_i^{p - 1} \equiv \suml_{i \in J} a_i \pmod p
	\]
	Аналогично для оставшихся многочленов:
	\begin{align*}
		&{\suml_{i \in J} b_i \equiv 0 \pmod p}
		\\
		&{\suml_{i \in J} x_i^{p - 1} \equiv \suml_{i \in J} 1 \equiv |J| \equiv 0 \pmod p}
	\end{align*}
	Мы почти доказали лемму. Осталось заметить, что $|J| \in \{p, 2p\}$, а от $3p$ и больше мы избавились из-за рассмотрения многочленов с $(3p - 1)$-й переменной.
	\begin{itemize}
		\item Если $|J| = p$, то теорема доказана.
		
		\item Если $|J| = 2p$, то возьмём за $I := \{1, \ldots, 3p\} \bs J$. Тогда $|I| = p$ и при этом
		\[
			\suml_{i \in I} a_i = \suml_{i = 1}^{3p} a_i - \suml_{i \in J} a_i \equiv 0 \pmod p
		\]
		Аналогично с $\suml_{i \in I} b_i \equiv 0 \pmod p$.
	\end{itemize}
\end{proof}

\begin{definition}
	\textit{Симметрическим многочленом} степени $k$ от $n$ переменных называется следующий многочлен:
	\[
		\sigma_k(x_1, \ldots, x_n) = \suml_{I \subset \{1, \ldots, n\} \over |I| = k} \suml_{i \in I} x_i
	\]
\end{definition}

\begin{example}
	\begin{align*}
		&{\sigma_1(x_1, \ldots, x_n) = x_1 + x_2 + \ldots + x_n}
		\\
		&{\sigma_2(x_1, \ldots, x_n) = x_1x_2 + x_1x_3 + \ldots + x_1x_n + x_2x_3 + \ldots + x_{n - 1}x_n}
		\\
		&{\sigma_n(x_1, \ldots, x_n) = x_1 \cdot \ldots \cdot x_n}
	\end{align*}
\end{example}

\begin{proof} (теоремы Роньяи)
	Докажем случай, когда $n = p$ - простое число. Дополнительно обозначим за $m = 4p - 2$. Как и в теореме ЭГЗ, предположим противное:
	\[
	\forall I \subset \{1, \ldots, m\},\ |I| = p\ \ \left(\suml_{i \in I} a_i \not\equiv 0 \pmod p\right) \vee \left(\suml_{i \in I} b_i \not\equiv 0 \pmod p\right)
	\]
	В силу доказанной леммы, мы можем усилить отрицание и сказать, что $|I| = 3p$, ведь если бы такое $I$ подходило, то из него можно было бы извлечь подходящее подмножество $|I'| = p$.
	
	Теперь рассмотрим многочлен $F(x_1, \ldots, x_m)$ вида:
	\begin{multline*}
		F(x_1, \ldots, x_m) = \left(\left(\suml_{i = 1}^m a_ix_i\right)^{p - 1} - 1\right) \cdot \left(\left(\suml_{i = 1}^m b_ix_i\right)^{p - 1} - 1\right) \cdot
		\\
		\left(\left(\suml_{i = 1}^m x_i\right)^{p - 1} - 1\right) \cdot \left(\sigma_p(x_1, \ldots, x_m) - 2\right)
	\end{multline*}
	Посмотрим, какие значения по модулю $p$ принимает $F$ на $(x_1, \ldots, x_m) \in \{0, 1\}^m$:
	\begin{enumerate}
		\item Пусть $(x_1, \ldots, x_m)$ таков, что в нём $p$ единиц и, соответственно, $m - p$ нулей. Обозначим за $I$ - множество индексов, где стоят единицы. Тогда понятно $|I| = p$. Более того, теперь суммы внутри первых двух скобок стали иметь вид:
		\begin{align*}
			&{\left(\suml_{i = 1}^m a_ix_i\right)^{p - 1} = \left(\suml_{i \in I} a_i\right)^{p - 1}}
			\\
			&{\left(\suml_{i = 1}^m b_ix_i\right)^{p - 1} = \left(\suml_{i \in I} b_i\right)^{p - 1}}
		\end{align*}
		По предположению хотя бы одна из них не обнуляется. Значит, возведение в степень $p - 1$ даст единицу по модулю $p$ и в итоге получим, что
		\[
			F(x_1, \ldots, x_m) \equiv 0 \pmod p
		\]
		
		\item Аналогично предыдущему случаю, но теперь $|I| = 3p$. Так как предположение усиляется, то и в данном случае
		\[
			F(x_1, \ldots, x_m) \equiv 0 \pmod p
		\]
		
		\item $|I| \not\equiv 0 \pmod p$. В таком случае, посмотрим на третью скобку многочлена:
		\[
			\left(\suml_{i = 1}^m x_i\right)^{p - 1} = \left(\suml_{i \in I} 1\right)^{p - 1} = |I|^{p - 1} \equiv 0 \pmod p
		\]
		Снова получили, что
		\[
			F(x_1, \ldots, x_m) \equiv 0 \pmod p
		\]
		
		\item $|I| = 2p$. Это последний случай, и он уже связан с симметрическим многочленом. Заметим, что если слагаемое содержит $x_j$, где $j \notin I$, то оно сразу обнуляется и не вносит вклада в значение $\sigma_p(x_1, \ldots, x_m)$. Отсюда следует 2 вещи: во-первых, каждое слагаемое - это просто единица, а во-вторых, этих слагаемых всего $C_{2p}^p$. Следовательно
		\[
			\sigma(x_1, \ldots, x_m) = C_{2p}^p \equiv 2 \pmod p
		\]
		Величина $C_{2p}^p$ по модулю $p$ уже доказывалась выше. В последний раз получили, что
		\[
			F(x_1, \ldots, x_m) \equiv 0 \pmod p
		\]
		
		\item $|I| = 0$. В таком случае
		\[
			F(x_1, \ldots, x_m) = F(0, \ldots, 0) = 2
		\]
		И это единственный набор $(x_1, \ldots, x_m)$, на котором $F$ отличен от нуля.
	\end{enumerate}
	Нам снова нужно совершить ужасное деяние: раскрыть скобки у данного многочлена. В общем случае слагаемое будет иметь вид:
	\[
		C \cdot x_{i_1}^{\alpha_{i_1}} \cdot \ldots \cdot x_{i_q}^{\alpha_{i_q}}
	\]
	Где $\forall l \in [1; q]\ \alpha_{i_l} \ge 1$. Сделаем все эти степени равными единицами, а полученный многочлен обозначим за $F'(x_1, \ldots, x_m)$. Коль скоро мы разобрали все случаи, когда $(x_1, \ldots, x_m) \in \{0, 1\}^m$, то для этих же наборов будет верно следующее:
	\[
		\forall (x_1, \ldots, x_m) \in \{0, 1\}^m\ \ F(x_1, \ldots, x_m) = F'(x_1, \ldots, x_m)
	\]
	Из всего вышесказанного несложно заметить, что тогда $F'$ можно указать явно(как минимум на наборах из $\{0, 1\}^m$, но как будет доказано ниже, это вообще единственный вид данного многочлена):
	\[
		F'(x_1, \ldots, x_m) = 2(1 - x_1) \cdot (1 - x_2) \cdot \ldots \cdot (1 - x_m)
	\]
	Чтобы доказать, что вид $F'(x_1, \ldots, x_m)$ единственнен, мы должны доказать, что все возможные мономы вида:
	\[
		x_{i_1} \cdot \ldots \cdot x_{i_q}
	\]
	образуют базис всех функций $f \colon \{0, 1\}^m \to \Z_p$. Размер множества из всех рассматриваемых мономов -- $2^m$ и при этом очевидно, что это множество образует линейно независимую систему (зафиксируем любой моном и рассмотрим набор $(x_1, \ldots, x_m)$, где ненулевыми будут только те $x_j$, что входят в рассматриваемый моном. Тогда все остальные обнулятся). Зафиксируем произвольную $f \colon \{0, 1\}^m \to \Z_p$. Тогда, $f$ точно можно выразить в базисе характеристических функций:
	\[
		\mu_u(v) = \System{
			&{1, \text{ если } u = v}
			\\
			&{0, \text{ иначе}}
		}
	\]
	где $u, v \in \{0, 1\}^m$. А любую такую характеристическую функцию можно явно записать через мономы:
	\[
		\mu_{(u_1, \ldots, u_m)}(v_1, \ldots, v_m) = \prodl_{i \colon u_i = 1} v_i \cdot \prodl_{j \colon u_j = 0} (1 - v_j)
	\]
	Тем самым мы доказали, что множество мономов образует базис в пространстве функций $f \colon \{0, 1\}^m \to \Z_p$. Значит, представление $F'$ единственно и мы его нашли.
	
	Если вид $F'$ действительно такой, то $\Deg F' = m = 4p - 2$. При этом $\Deg F' \le \Deg F = 3(p - 1) + p = 4p - 3$ - получили противоречие.
\end{proof}

\subsubsection*{Проблема Эрдеша-Гинзбурга-Зива в многомерном случае}

Обозначим нижнюю оценку для $d$-мерного случая за функцию от двух переменных $f(n, d)$. Тогда, известно следующее:
\begin{itemize}
	\item \(f(n, 1) = 2n - 1\)
	
	\item \(f(n, 2) = 4n - 3\)
	
	\item \(f(n, 3) \ge 8n - 7\), но уже доказано, что $f(n, 3) \ge 9n - 9$
	
	\item \(f(n, d) \ge 2^d \cdot (n - 1) + 1\) - обобщение рассказанных контрпримеров.
\end{itemize}
Полностью решены случаи лишь для $d = 1, 2$. Для всех остальных точные ответы остаются неизвестными.