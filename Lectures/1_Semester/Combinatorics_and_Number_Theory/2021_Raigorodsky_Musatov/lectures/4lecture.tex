\begin{proof}
	Пусть выполнено $A \leqsim B$ и $B \leqsim A$. Это означает, что есть биекции $f \colon A \to B_1$ и $g \colon B \to A_1$, где $A_1 \subset A$ и $B_1 \subset B$.
	
	Так как $f$ и $g$ - отображения, то можно посмотреть на образы $f(A_1) = B_2$ и $g(B_1) = A_2$. При этом верны утверждения:
	\begin{align*}
		&{B \supset B_1 \Ra g(B) \supset g(B_1)}
		\\
		&{A \supset A_1 \Ra f(A) \supset f(A_1)}
	\end{align*}
	То есть $B_1 \supset B_2$ и $A_1 \supset A_2$. Так можно продолжать итеративно и, положив за $A_0 = A$, $B_0 = B$, получить последовательности вложенных множеств:
	\begin{align*}
		&{A_0 \supset A_1 \supset A_2 \supset \ldots}
		\\
		&{B_0 \supset B_1 \supset B_2 \supset \ldots}
	\end{align*}
	При этом имеются равенства:
	\begin{align*}
		&{f(A_k) = B_{k + 1}}
		\\
		&{g(B_k) = A_{k + 1}}
	\end{align*}
	Из всего вышесказанного возникают 2 утверждения, которые мы положим в основу биекции $h \colon A \to B$:
	\begin{align*}
	&{f(A_k \bs A_{k + 1}) = B_{k + 1} \bs B_{k + 2}}
	\\
	&{g(B_k \bs B_{k + 1}) = A_{k + 1} \bs A_{k + 2}}
	\end{align*}
	Но этого недостаточно. Ещё могут быть такие элементы, которые не попадут ни в одно кольцо: они принадлежат множествам $C = \bigcap\limits_{i = 0}^\infty A_i$ и $D = \bigcap\limits_{i = 0}^\infty B_i$. Однако при этом заметим, что $f(C) = D$, $g(D) = C$. Если $c \in C$, то $\forall i\ \{c\} \subset A_i$, а из этого следует, что $\forall i\ \{f(c)\} \subset f(A_i)$. Аналогично для $d \in D$. Если же элемент $a \notin C$, то он лежит в каком-то кольце и про этот случай уже всё известно. В итоге получаем следующую биекцию:
	\[
		h(x) = \System{
			&{f(x),\ x \in A_{2k} \bs A_{2k + 1}}
			\\
			&{g^{-1}(x),\ x \in A_{2k + 1} \bs A_{2k + 2}}
			\\
			&{f(x),\ x \in C}
		}
	\]
\end{proof}