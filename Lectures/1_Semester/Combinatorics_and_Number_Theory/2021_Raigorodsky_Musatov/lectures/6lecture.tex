\subsection{Плотный порядок}

\begin{definition}
	Порядок \textit{плотен}, если $\forall x, y\ |\ x < y \Ra \exists z\ |\ x < z < y$.
\end{definition}

\begin{example}
	$\Q, \R$ обладают плотным порядком, а $\N, \Z$ не обладают плотным порядком.
\end{example}

\begin{theorem}
	Любые 2 счётных плотно линейно упорядоченных множества без наименьшего и наибольшего элементов изоморфны.
\end{theorem}

\textcolor{red}{Здесь наверное тоже нужно доказательство. Наверное.}

\begin{example}
	$\Q$, $\Q \cap (0;1)$, $\Q_2 = \{\frac{k}{2^n}\ |\ k \in \Z,\ n \in \N\}$, $\mathbb{A}$ - алгебраические числа.
\end{example}

\begin{proof}
\textcolor{red}{Здесь будет что-то дописано. Когда-нибудь.}
\end{proof}

\subsection{Предпорядки}

\begin{definition}
	Отношение предпорядка $\precsim$~---~это рефлексивное и транзитивное отношение.
\end{definition}

\begin{definition}
	Отношение \textit{полного} предпорядка~---~это такой предпорядок, что
	$$
		\forall a, b \Ra (a \precsim b) \vee (b \precsim a).
	$$
	Из полноты следует рефлексиность. В экономике отношение полного предпорядка называется \textit{рациональным предпочтением}.
\end{definition}

\begin{theorem} (Структурная теорема)
	Назовём \textit{отношением безразличия} следующее отношение: 
	$a \sim b := (a \precsim b) \wedge (b \precsim a)$, тогда:
	
	Для любого отношения предпорядка отношение безразличия $\sim$~---~это отношение эквивалентности. При этом $\precsim$ задаёт отношение порядка на фактормножестве.
\end{theorem}

\begin{proof}
	Проверим $\sim$ на отношение эквивалентности:
	\begin{enumerate}
		\item $a \sim a$, так как $a \precsim a$ (рефлексивность).
		\item $a \sim b = b \sim a$, так как конъюнкция симметрична.
		\item $(a \sim b) \wedge (b \sim c) \Ra \left\{\begin{aligned}
			a \precsim b, & &b \precsim c \\ b \precsim a, & &c \precsim b
		\end{aligned}\right\} \Ra \System{a \precsim c, \\ c \precsim a,} \Ra a \sim c.$
	\end{enumerate}
\end{proof}

\subsubsection*{Агрегирование}

\begin{definition}
	Пусть $\precsim_1, \dots, \precsim_n$ - предпорядки на одном и том же множестве.
	
	Агрегирование по большинству: $x \precsim y$, если $\#\{i\ |\ x \precsim_i y\} \ge \#\{i\ |\ x \succsim_i y\}$, где $\#$ означает количество.
\end{definition}

\begin{note}
	Может получиться нетранзитивное отношение. Таким примером служит цикл Кондорсе:
	
	\begin{align*}
	a &\prec_1 b \prec_1 c,
	\\
	b &\prec_2 c \prec_2 a,
	\\
	c &\prec_3 a \prec_3 b.
	\end{align*}
	Отсюда получим $a \prec b \prec c \prec a$.
\end{note}

\begin{theorem}[об агрегировании по большинству]
	Агрегированием по большинству \textbf{на конечном множестве} можно получить любое рефлексивное полное отношение.
\end{theorem}

\begin{proof}
	Пусть мы хотим $x \prec y$. Добавим 2 порядка: $x <_1 y <_1 a_1 <_1 \dots <_1 a_{n - 2}$, а другое $a_{n - 2} <_2 a_{n - 3} <_2 \dots <_2 a_1 <_2 x <_2 y$.
\end{proof}

\begin{definition}
	Пусть $\preceq_1, \ldots, \preceq_n$ - предпорядки на одном и том же множестве.
	
	Тогда их \textit{агрегированием по большинству} назовём отношение, в котором
	\[
		x \preceq y \lra (\forall i \in [1; n]\ x \preceq_i y)
	\]
\end{definition}

\begin{theorem}[об агрегировании консенсусом]
	Агрегирование порядков консенсусом~---~порядок. 
	Агрегирование предпорядков консенсусом~---~тоже порядок.
\end{theorem}

\begin{theorem}
	Любой предпорядок может быть получен агрегированием консенсусом полных предпорядков.
\end{theorem}

\subsection{Решётки (как упорядоченное множество)}

\begin{definition}
	Пусть задан некоторое частично упорядоченное множество $(A, \le)$. Тогда, \textit{верхняя грань} элементов $x$ и $y$~---~любой $z$ такой, что $z \ge x$ и $z \ge y$.
\end{definition}

\begin{definition}
	\textit{Точная верхняя грань (супремум)}~---~такая верхняя грань, что она $\le$ любой другой верхней грани. 
\end{definition}

\begin{definition}
	\textit{Точная нижняя грань (инфинум)}~---~такая нижняя грань, что она $\ge$ любой другой нижней грани.
\end{definition}

\begin{definition}
	\textit{Решётка}~---~это частично упорядоченное множество, в котором у любых $x$ и $y$, лежащих в нём, есть $\sup$ и $\inf$.
\end{definition}

\begin{note}
	Необходимо и достаточно существования такой грани, что она сравнима со всеми остальными из того же типа (то есть верхними или нижними).
\end{note}