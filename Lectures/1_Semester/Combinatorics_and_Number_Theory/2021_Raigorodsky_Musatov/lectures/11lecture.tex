\begin{proof}
	Пусть $\{d_1, \ldots, d_s\}$ --- это все делители числа $n \in \N$. Тогда множество $V$ можно представить как объединение множеств слов $V_i$ длины $n$ с одинаковым периодом $d_i$:
	\begin{align*}
		&V = V_1 \sqcup \ldots \sqcup V_s,
		\\
		&|V| = |V_1| + \ldots + |V_s|.
	\end{align*}
	
	Обозначим за $W_i$ --- множество слов длины $d_i$ с периодом $d_i$. Понятно, что
	\[
		|W_i| = |V_i| \Ra |V| = |W_1| + \ldots + |W_s|
	\]
	
	Ещё введём понятие $U_i$ --- это множество циклических слов длины $d_i$ и периодом $d_i$. Тогда
	\[
		|W_i| = d_i \cdot |U_i|.
	\]
	И обозначим $|U_i| =: M(d_i)$. Теперь $|V|$ можно записать как
	\[
		|V| = r^n = \suml_{i = 1}^s d_i |U_i| = \suml_{d \mid n} d \cdot M(d).
	\]
	Заметим, что если ввести функции
	\begin{align*}
		&f(n) = r^n,
		\\
		&g(n) = n \cdot M(n),
	\end{align*}
	то $M(n)$ можно посчитать через обращение Мёбиуса:
	\begin{align*}
		&g(n) = \suml_{d \mid n} \mu(d) \cdot f(n / d),
		\\
		&M(n) = \frac{1}{n}\suml_{d \mid n} \mu(d) \cdot r^{n / d}.
	\end{align*}
	Отсюда получаем
	\[
		T_r(n) = \suml_{d \mid n} M(d) = \suml_{d \mid n} \frac{1}{d} \left(\suml_{d' \mid d} \mu(d') r^{d / d'}\right)
	\]
\end{proof}

\subsection{Общая функция Мёбиуса}

\begin{definition}
	 \textit{Функцией Мёбиуса на частично упорядоченном множестве} (ЧУМе) $\trbr{\mcP, \preceq}$ называется функция $\mu$, определяемая как
	 \[
	 	\mu(x, y) = \System{
	 		&{1,\ x = y,}
	 		\\
	 		&{-\suml_{x \preceq z \prec y} \mu(x, z),\ x \prec y.}
 		}
	 \]
	 При этом считается, что $\forall y \in \mcP$ существует лишь конечное число $x \in \mcP$ таких, что $x \preceq y$.
\end{definition}

\begin{theorem} (Связь между общей и стандартной функцией Мёбиуса)
	Переобозначим стандартную функцию Мёбиуса за $\hat{\mu}$. Тогда, если $\trbr{\mcP, \preceq} = \trbr{\N, |}$, то
	\[
		\mu(y, x) = \hat{\mu}\left(\frac{x}{y}\right).
	\]
\end{theorem}

\begin{proof}
	Докажем теорему при помощи индукции по $\frac{x}{y}$:
	\begin{itemize}
		\item База: $\frac{x}{y} = 1 \Ra x = y$. Тогда
		\[
			\mu(x, x) = 1 = \hat{\mu}(1) \text{ --- верно}.
		\]
		
		\item Переход: $y \prec x \lra y | x$ и $y \neq x$. Значит
		\[
			x = y \cdot p_1^{\alpha_1} p_2^{\alpha_2} \ldots p_s^{\alpha_s},\ \alpha_i \ge 1.
		\]
		Тогда
		\[
			\mu(y, x) = -\suml_{y \preceq z \prec x} \mu(y, z).
		\]
		Из определения $z$ следует, что $\frac{z}{y} < \frac{x}{y}$. То есть мы можем применить предположение индукции:
		\[
			\mu(y, x) = -\suml_{y \preceq z \prec x} \hat{\mu}\left(\frac{z}{y}\right).
		\]
		Так как $y \preceq z$, то $z$ содержит в себе правую часть из выражения $x$ и можно записать следующее:
		\[
			\mu(y, x) = -\suml_{{0 \le \beta_i \le \alpha_i} \over {\exists j : \beta_j < \alpha_j}} \hat{\mu}(p_1^{\beta_1} \ldots p_s^{\beta_s}).
		\]
		Рассмотрим частный случай, когда $(a_1, \ldots, a_s) = (1, \ldots, 1)$. Тогда значения кортежа $(\beta_1, \ldots, \beta_s)$ являются битовым представлением числа от $0$ до $2^s - 2$. Отсюда имеем:
		\[
			\mu(y, x) = -\suml_{{0 \le \beta_i \le 1} \over {\exists \beta_j = 0}} \hat{\mu}(p_1^{\beta_1} \ldots p_s^{\beta_s}) = -\suml_{k = 0}^{s - 1} (-1)^k \cdot C_s^k = - \left(0 - (-1)^s \cdot C_s^s\right) = (-1)^s = \hat{\mu}\left(\frac{x}{y}\right).
		\]
		Теперь докажем другой случай, когда $\exists \alpha_j \ge 2$: если рассматривать $z$ такое, что в нём содержится $\beta_j \ge 2$, то всё слагаемое сразу будет ноль. Значит снова $0 \le \beta_i \le 1$, при этом кортеж уже может быть представлением числа $2^s - 1$:
		\[
			\mu(y, x) = -\suml_{{0 \le \beta_i \le 1}} \hat{\mu}(p_1^{\beta_1} \ldots p_s^{\beta_s}) = -\suml_{k = 0}^s (-1)^k C_s^k = 0 = \hat{\mu}\left(\frac{x}{y}\right).
		\]
	\end{itemize}
\end{proof}

\begin{theorem} (Общее обращение Мёбиуса) \label{globMebius}
	Пусть $\trbr{\mcP, \preceq}$ --- некоторый ЧУМ. Пусть $f\colon \mcP \ra \Cm$, $g(x) := \suml_{y \preceq x} f(y)$. Тогда
	\[
		f(x) = \suml_{y \preceq x} \mu(y, x) g(y).
	\]
\end{theorem}

\begin{lemma}
	\[
		\suml_{x \preceq y \preceq z} \mu(y, z) = \mathbb{I}_{x = z} = \System{
			&{1,\ x = z,}
			\\
			&{0,\ x \neq z.}
		}
	\]
\end{lemma}

\begin{proof}~
	\begin{enumerate}
		\item Если $x = z$, то
		\[
			\suml_{x \preceq y \preceq z} \mu(y, z) = \mu(x, z) = 1.
		\]
		
		\item Иначе $x \prec z$. Проведём доказательство индукция по длине максимальной цепи вида
		\[
			x \prec \ldots \prec \ldots \prec z.
		\]
		\begin{enumerate}
			\item База: такая цепочка имеет длину 2, то есть вида $x \prec z$,
			\[
				\suml_{x \preceq y \preceq z} \mu(y, z) = \mu(x, z) + \mu(z, z) = \left(-\suml_{x \preceq w \prec z} \mu(x, w)\right) + 1 = 0.
			\]
			
			\item Переход: цепочка имеет длину $\ge 3$, а для меньших уже доказано:
			\begin{multline*}
				\suml_{x \preceq y \preceq z} \mu(y, z) = \suml_{x \preceq y \prec z} \mu(y, z) + \mu(z, z) = 1 - \suml_{x \preceq y \prec z} \suml_{y \preceq u \prec z} \mu(y, u) = 1 - \suml_{x \preceq y \preceq u \prec z} \mu(y, u) =
				\\ =
				1 - \suml_{x \preceq u \prec z} \suml_{x \preceq y \preceq u} \mu(y, u) = 1 - \suml_{x \preceq y \preceq x} \mu(y, x) - \suml_{x \prec u \prec z} \suml_{x \preceq y \preceq u} \mu(y, u).
			\end{multline*}
			Так как в последнем слагаемом $x \prec u \prec z$, то максимальная цепочка, соединяющая $x$ и $u$, короче, чем $x$ и $z$. Значит, можно воспользоваться предположением индукции:
			\[
				\suml_{x \preceq y \preceq u} \mu(y, u) = \mathbb{I}_{x = u} = 0,
			\]
			то есть
			\[
				\suml_{x \preceq y \preceq z} \mu(y, z) = 1 - 1 - 0 = 0.
			\]
		\end{enumerate}
	\end{enumerate}
\end{proof}

\begin{proof} (Теоремы \ref{globMebius})
	Поступаем аналогично стандартной функции Мёбиуса:
	\begin{multline*}
		\suml_{y \preceq x} \mu(y, x) g(y) = \suml_{y \preceq x} \mu(y, x) \suml_{z \preceq y} f(z) = \suml_{z \preceq y \preceq x} \mu(y, x)f(z) = \suml_{z \preceq x} f(z) \cdot \suml_{z \preceq y \preceq x} \mu(y, x) =
		\\ =
		f(x) + \suml_{z \prec x}f(z) \cdot  \underbrace{\suml_{z \preceq y \preceq x} \mu(y, x)}_0 = f(x).
	\end{multline*}
\end{proof}