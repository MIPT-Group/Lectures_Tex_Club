\begin{definition}
	Если словами, то \textit{кортеж} - это упорядоченное мультимножество $(x_1, \dots, x_n)$.
	
	Если формально, то определение рекурсивно:
	\begin{enumerate}
		\item Кортеж длины 0 - это $\emptyset$
		\item Кортеж длины $n + 1$ - это упорядоченная пара $(h, T)$, где $T$ - кортеж длины $n$, а $h$ - элемент, который мы ставим на первое место в кортеже.
	\end{enumerate}
\end{definition}

\begin{example}
	$(x_1, x_2, \dots, x_n) = (h, T)$, где $h = x_1$, $T = (x_2, \dots, x_n)$.
\end{example}

\subsection{Декартово произведение}

\begin{definition}
	\textit{Декартовым произведением} множеств $A$ и $B$ называется множество
	$$
	A \times B = \{(a, b)\ |\ a \in A, b \in B\}
	$$
\end{definition}

\begin{note}
	Такое множество называется декартовым из-за близости с декартовой системой координат. Первый элемент даст координату по оси $x$, а второй по оси $y$.
\end{note}

\subsection{Декартова степень}

\begin{definition}
	\textit{Декартовой степенью} $n$ множества $A$ называется множество
	$$
	A^n = \{(a_1, \dots, a_n)\ |\ a_i \in A\}
	$$
\end{definition}

\subsubsection*{Ассоциативность декартового произведения}

\subsection{Отображения и соответствия}

\begin{definition}
	\textit{Соответствием} между $A$ и $B$ называется произвольное подмножество $A \times B$.
	$$
		F \subset A \times B
	$$
	Принадлежность пары $(a, b)$ данному соответствию $F$ принято обозначать как
	$$
		\big((a, b) \in F\big) := \big(b \in F(a)\big)
	$$
	Само соответствие порой обозначают как
	$$
		F : A \rra B
	$$
\end{definition}

\begin{definition}
	\textit{Отображением} или же \textit{функцией} между множествами $A$ и $B$ называется однозначное соответствие на этих же множествах, то есть
	$$
		\forall a \in A \Ra \exists! b \in B\ |\ b = F(a)
	$$
	Или же говорят, что
	$$
		F : A \ra B
	$$
\end{definition}

\begin{definition}
	\textit{Аргументом} функции называется $a \in A$.
\end{definition}

\begin{definition}
	\textit{Значением функции} называется $b \in B$.
\end{definition}

\begin{definition}
	Функция называется \textit{частично определенной}, если допускается отсутствие значения у какого-либо аргумента. То есть
	$$
		\forall a \in A \Ra |F(a)| \le 1
	$$
\end{definition}

\subsubsection*{Свойства соответствий}

Соответствие называется
\begin{itemize}
	\item \textit{Инъективным}, если $\forall a_1, a_2 \in A\ |\ a_1 \neq a_2 \Ra F(a_1) \cap F(a_2) = \emptyset$
	\item \textit{Сюръективным}, если $\forall b \in B\ \ \exists a \in A\ |\ b \in F(a)$
	\item \textit{Биективным}, если оно является одновременно и инъективным, и сюръективным.
\end{itemize}

\subsubsection*{Свойства отображений}

Отображение называется
\begin{itemize}
	\item \textit{Инъекцией}, если $\forall a_1, a_2 \in A\ |\ a_1 \neq a_2 \Ra F(a_1) \neq F(a_2)$
	\item \textit{Сюръекцией}, если $\forall b \in B\ \ \exists a \in A\ |\ b =k F(a)$
	\item \textit{Биекцией}, если оно является одновременно и инъекцией, и сюръекцией.
\end{itemize}

\begin{definition}
	$F^{-1}$ называется \textit{обратным соответствием} к $F$, если
	$$
		(a, b) \in F \lra (b, a) \in F^{-1}
	$$
\end{definition}

\begin{definition}
		$F^{-1}$ называется \textit{обратным отображением} к $F$, если
	$$
		b = F(a) \lra a = F^{-1}(b)
	$$
\end{definition}

\subsection{Образ и прообраз}

\begin{definition}
	Пусть $S \subset A$. Тогда, $F(S) := \bigcup\limits_{a \in S} F(a)$ называется \textit{образом} множества $S$.
\end{definition}

\begin{definition}
	Пусть $T \subset B$. Тогда, $F^{-1}(T) := \bigcup\limits_{b \in T} F^{-1}(b)$ называется \textit{прообразом} множества $T$.
\end{definition}

\subsection{Композиция}

\begin{definition}
	Пусть $F : A \rra B$, $G : B \rra C$. Тогда, соответствие $G \circ F$ называется \textit{композицией} соответствий $F$ и $G$.
	\begin{align*}
		&G \circ F : A \rra C
		\\
		&(x, z) \in G \circ F \lra \exists y\ |\ \big((x, y) \in F\big) \wedge \big((y, z) \in G\big)
	\end{align*}
\end{definition}

\begin{definition}
	Пусть $F : A \ra B$, $G : B \ra C$. Тогда, отображение $G \circ F$ называется \textit{композицией} отображений $F$ и $G$.
	\begin{align*}
	&G \circ F : A \ra C
	\\
	&(x, z) \in G \circ F \lra \exists y\ |\ \big(y = F(x)) \wedge \big(z =  G(y)\big)
	\end{align*}
	Принято обозначать $G \circ F := G(F(x))$
\end{definition}

\subsubsection*{Свойства композиции}

Рассматриваются такие соответствия/отображения, что композиция существует.
\begin{enumerate}
	\item $H \circ (G \circ F) = (H \circ G) \circ F$ (ассоциативность)
	\item Коммутативность \textbf{необязательно} выполнена.
	\begin{example}
		Пусть $F : A \rra B$, $G : B \rra A$. Тогда $G \circ F : A \rra A$, но при этом $F \circ G : B \rra B$.
	\end{example}
	\item $\exists id_A : A \ra A,\ id_B : B \ra B\ |\ F \circ id_A = id_B \circ F = F$ (существование отображения множества в себя, т.е. $a = id_A(a)\ \forall a \in A$)
	\item $F^{-1} \circ F = id_A$, если $F : A \ra B$
	\item $F \circ F^{-1} = id_B$, если $F : A \ra B$
\end{enumerate}