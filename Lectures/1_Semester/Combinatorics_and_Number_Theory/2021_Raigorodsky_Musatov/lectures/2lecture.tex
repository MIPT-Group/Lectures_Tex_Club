\begin{definition}
	Если словами, то \textit{кортеж} - это упорядоченное мультимножество $(x_1, \dots, x_n)$.
	
	Если формально, то определение рекурсивно:
	\begin{enumerate}
		\item Кортеж длины 0 - это $\emptyset$
		\item Кортеж длины $n + 1$ - это упорядоченная пара $(h, T)$, где $T$ - кортеж длины $n$, а $h$ - элемент, который мы ставим на первое место в кортеже.
	\end{enumerate}
\end{definition}

\begin{example}
	$(x_1, x_2, \dots, x_n) = (h, T)$, где $h = x_1$, $T = (x_2, \dots, x_n)$.
\end{example}

\subsection{Декартово произведение}

\begin{definition}
	\textit{Декартовым произведением} множеств $A$ и $B$ называется множество
	$$
	A \times B = \{(a, b)\ |\ a \in A, b \in B\}
	$$
\end{definition}

\begin{note}
	Такое множество называется декартовым из-за близости с декартовой системой координат. Первый элемент даст координату по оси $x$, а второй по оси $y$.
\end{note}

\subsection{Декартова степень}

\begin{definition}
	\textit{Декартовой степенью} $n$ множества $A$ называется множество
	$$
	A^n = \{(a_1, \dots, a_n)\ |\ a_i \in A\}
	$$
\end{definition}

\subsubsection{Ассоциативность декартового произведения}

\subsection{Отображения и соответствия}

\begin{definition}
	\textit{Соответствием} между $A$ и $B$ называется произвольное подмножество $A \times B$.
	$$
		F \subset A \times B
	$$
	Принадлежность пары $(a, b)$ данному соответствию $F$ принято обозначать как
	$$
		(a, b) \in F := b \in F(a)
	$$
\end{definition}