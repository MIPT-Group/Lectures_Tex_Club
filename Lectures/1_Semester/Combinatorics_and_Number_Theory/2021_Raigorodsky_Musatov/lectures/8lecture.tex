\begin{note}
	Если расписать 5е свойство для $n = 1, 2, 3, \ldots$, то можно получить сумму $1, 2, 3, \ldots$ степеней первых $m + 1$ натуральных чисел соответственно.
\end{note}

\subsection{Полиномиальный коэффициент}

У нас есть $n_i$ объектов $a_i$ для всех $i \in [1, \ldots, t]$. Сколько различных последовательностей можно составить, используя абсолютно все объекты?

Заметим, что у нас всегда одинаковая длина слова, равная $n$:
\[
	n = \suml_{i = 1}^t n_i
\]

Чтобы собрать слово, будем последовательно расставлять все $t$ групп объектов.
\begin{itemize}
	\item Сколько мест есть для первой группы? - $n$
	\item Сколько мест есть для второй группы? - $n - n_1$ (первую поставили первой)
	
	\item~\vdots
	\item Сколько мест есть для последней группы? - $n - n_1 - \ldots - n_{t - 1} = n_t$
\end{itemize}

Очевидно, что для $i$-й группы происходит выбор $n_i$ мест из тех, что остались не занятыми. В итоге имеем
\begin{multline*}
	P(n_1, n_2, \ldots, n_t) = C_n^{n_1} \cdot C_{n - n_1}^{n_2} \cdot C_{n - n_1 - n_2}^{n_3} \cdot \ldots \cdot C_{n - n_1 - \ldots - n_{t - 1}}^{n_t} = \\
	\frac{n!}{n_1! (n - n_1)!} \cdot \frac{(n - n_1)!}{n_2! (n - n_1 - n_2)!} \cdot \frac{(n - n_1 - n_2)!}{n_3! (n - n_1 - n_2 - n_3)!} \cdot \ldots \cdot \frac{(n - n_1 - \ldots - n_{t - 1})!}{n_t! (n - n_1 - \ldots - n_t)!} = \\
	\frac{n!}{n_1! \cdot n_2! \cdot n_3! \cdot \ldots \cdot n_t! \cdot 0!} = \frac{n!}{n_1! \cdot n_2! \cdot n_3! \cdot \ldots \cdot n_t!}
\end{multline*}

\begin{definition}
	Число $P(n_1, \ldots, n_t)$ называют \textit{полиномиальным коэффициентом}
\end{definition}

\subsection{Полиномиальная формула}

\begin{definition}
	Бином Ньютона позволяет выяснить разложение любой степени для суммы двух элементов. Его обобщением на случай $t$ переменных служит \textit{полиномиальная формула}.
\end{definition}

\begin{theorem}
	\textit{Полиномиальной формулой} называется выражение
	\[
		(x_1 + \ldots + x_t)^n = \suml_{\forall n_1 + \ldots + n_t = n} P(n_1, \ldots, n_t) \cdot x_1^{n_1} x_2^{n_2} \cdot x_t^{n_t}
	\]
\end{theorem}

\begin{proof}
	Аналогично биному Ньютона распишем скобки:
	\[
		(x_1 + \ldots + x_t)^n = \underbrace{(x_1 + \ldots + x_t) \cdot \ldots \cdot (x_1 + \ldots + x_t)}_{n \text{ раз}}
	\]
	
	Чтобы получить полное слагаемое, нам нужно выбрать из каждой скобки по одному объекту. Если его привести, то мы получим выражение вида
	\[
		x_1^{n_1} \cdot \ldots \cdot x_t^{n_t}
	\]
	При этом верно равенство:
	\[
		n_1 + \ldots + n_t = n
	\]
	где $n_i$ значит то же, что и до этого: количество раз, сколько мы выбрали (имеем) $x_i$.
	
	Несложно понять, что такое слагаемое мы могли получить разными способами - в зависимости от того, из каких скобок какие объекты мы брали. Количество различных способов набрать $n_1, n_2, \ldots, n_t$ из скобок определяется полиномиальным коэффициентом. Отсюда 
	\[
		(x_1 + \ldots + x_t)^n = \suml_{\forall n_1 + \ldots + n_t = n} P(n_1, \ldots, n_t) \cdot x_1^{n_1} x_2^{n_2} \cdot x_t^{n_t}
	\]
\end{proof}

\subsubsection*{Сумма полиномиальных коэффициентов}

\begin{proposition}
	\[
		\suml_{\forall n_1 + \ldots + n_t = n} P(n_1, \ldots, n_t) = k^n
	\]
\end{proposition}

\begin{proof}
	Просто положим в полиномиальной формуле все $x_i = 1$
\end{proof}

\subsection{Формула включений и исключений}

Пусть есть $N$ элементов. Обозначим $N(\alpha_i)$ - количество элементов, обладающих свойством $\alpha_i$. $N(\alpha'_i)$ - количество элементов, не обладающих свойством $\alpha_i$. Ну и понятно, что $N(\alpha_i, \alpha'_j)$ - количество элементов, обладающих свойством $\alpha_i$ \textbf{и} не обладающих свойством $\alpha_j$.

\begin{theorem}
	Если мы рассмотрим $n$ свойств, которые мы можем приписать $N$ объектам, то имеет место \textit{формула включений и исключений}:
	\begin{multline*}
		N(\alpha'_1, \ldots, \alpha'_n) = N - N(\alpha_1) - \ldots - N(\alpha_n) + \\
		N(\alpha_1, \alpha_2) + \ldots + N(\alpha_{n - 1}, \alpha_n) - \ldots + (-1)^n N(\alpha_1, \ldots, \alpha_n)
	\end{multline*}
\end{theorem}

\begin{proof}
	Воспользуемся математической индукцией
	\begin{itemize}
		\item База $n = 1$:
		\[
			N(\alpha'_1) = N - N(\alpha_1)
		\]
		верность очевидна
		
		\item Предположение индукции: формула включений и исключений верна \textbf{для любых $N$ объектов и для любых $n$ свойств}. Докажем, что она также верна и в случае $n + 1$го свойства для данного $N$.
		
		Применим предположение индукции для $N$ объектов и свойствам $\alpha_1, \ldots, \alpha_n$:
		\begin{multline*}
			N(\alpha'_1, \ldots, \alpha'_n) = N - N(\alpha_1) - \ldots - N(\alpha_n) + \\
			N(\alpha_1, \alpha_2) + \ldots + N(\alpha_{n - 1}, \alpha_n) - \ldots + (-1)^n N(\alpha_1, \ldots, \alpha_n)
		\end{multline*}
		Теперь сделаем то же самое для $M \le N$ объектов, которые точно обладают свойством $\alpha_{n + 1}$, и свойств $\alpha_1, \ldots, \alpha_n$.
		\begin{multline*}
			M(\alpha'_1, \ldots, \alpha'_n) = M - M(\alpha_1) - \ldots - M(\alpha_n) + \\
			M(\alpha_1, \alpha_2) + \ldots + M(\alpha_{n - 1}, \alpha_n) - \ldots + (-1)^n M(\alpha_1, \ldots, \alpha_n)
		\end{multline*}
		В силу определения $M$ также верно, что $M := N(\alpha_{n + 1})$. То есть можно переписать последнее выражение в виде
		\begin{multline*}
			N(\alpha'_1, \ldots, \alpha'_n, \alpha_{n + 1}) = N(\alpha_{n + 1}) - N(\alpha_1, \alpha_{n + 1}) - \ldots - N(\alpha_n, \alpha_{n + 1}) + \\
			N(\alpha_1, \alpha_2, \alpha_{n + 1}) + \ldots + N(\alpha_{n - 1}, \alpha_n, \alpha_{n + 1}) - \ldots + (-1)^n N(\alpha_1, \ldots, \alpha_n, \alpha_{n + 1})
		\end{multline*}
		Теперь вычтем полученное выражение из того, что было для всех $N$ объектов:
		\begin{multline*}
			N(\alpha'_1, \ldots, \alpha'_n, \alpha'_{n + 1}) = N(\alpha'_1, \ldots, \alpha'_n) - N(\alpha'_1, \ldots, \alpha'_n, \alpha'_{n + 1}) = \\
			N - N(\alpha_1) - \ldots - N(\alpha_n) - N(\alpha_{n + 1}) + \\
			N(\alpha_1, \alpha_2) + \ldots + N(\alpha_n, \alpha_{n + 1}) - \ldots + (-1)^n N(\alpha_1, \ldots, \alpha_n, \alpha_{n + 1})
		\end{multline*}
	\end{itemize}
\end{proof}

\begin{note}
	По понятным причинам слагаемых, содержащих ровно $k$ свойств, будет $C_n^k$.
\end{note}

\begin{corollary}
	Пусть у нас есть множество $A = \{a_1, \ldots, a_n\}$. Рассмотрим все возможные $m$-размещения с повторениями из этого множества, при этом $m < n$. их $N := n^m$ штук. Положим их объектами для формулы включений и исключений и скажем, что $N(\alpha_i)$ - это все размещения, в которые \textbf{не} входит элемент $a_i$. Тогда, верны следующие утверждения:
	\begin{align*}
		&N(\alpha_i) = (n - 1)^m
		\\
		&N(\alpha_i, \alpha_j) = (n - 2)^m
		\\
		&N(\alpha_1, \ldots, \alpha_n) = (n - n)^m = 0
		\\
		&N(\alpha'_1, \ldots, \alpha'_n) = 0, \text{ так как } m < n
	\end{align*}
	По формуле включений и исключений имеем:
	\[
		\suml_{k = 0}^n (-1)^k \cdot C_n^k (n - k)^m = 0
	\]
\end{corollary}