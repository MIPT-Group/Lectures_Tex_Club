\subsection{Степень соответствия}

\begin{definition}
	Пусть задано соответствие $F : A \rra A$. Тогда,
	$$
		F^n = \underbrace{F \circ \dots \circ F}_{n \text{ раз}}
	$$
	Ну и понятно, что $n \in \N$
\end{definition}

\subsubsection*{Свойства степени}

\begin{itemize}
	\item $F^{n + m} = F^n \circ F^m$
	\item $(F^n)^m = F^{n \cdot m}$
\end{itemize}

\subsection{Возведение множества в степень другого множества}

Пусть есть $A$ и $B$ такие, что $|A| = n$, $|B| = k$.

Вопрос: сколько существует \textit{различных} отображений из $A$ в $B$?

Ответ: $k^n$.

\begin{definition}
	Степенью $A$ множества $B$ называется множество всевозможных отображений из множества $A$ в множество $B$:
	$$
		|B^A| = |B|^{|A|}
	$$
\end{definition}

\subsubsection*{Случай с $k = 2$}

Заметим, что для полного определения $F : A \ra B$ нам необходимо и достаточно задать $F^{-1}(b_0)$ или $F^{-1}(b_1)$. Тогда $F(x)$ можно определить очень просто:
$$
	F(x) = \System{&{b_0, \text{ если } x \in F^{-1}(b_0)} \\ &{b_1 \text{ иначе}}}
$$

Таким образом, получается вывод: любое подмножество \textit{однозначно} сопоставляется функции с двумя значениями.

\begin{definition}
	\textit{Булеаном} называют множество всех подмножеств множества $A$. Обозначается как $2^A$.
\end{definition}

\subsubsection*{Случай с $n = 0$}

Пусть $A = \emptyset$. Тогда посмотрим произвольное отображение $F : A \ra B$. Как известно, $F$ представляет собой подмножество $A \times B = \emptyset \times B = \emptyset$. То есть $F = \emptyset$, при этом данное соответствие является отображением, так как любому элементу множества $A$ соответствует ровно один элемент множества $B$. Отсюда следствие:
$$
	B^{\emptyset} = \{\emptyset\}
$$
Причём это верно для любого $B$, даже для $B = \emptyset$.

\subsubsection*{Случай с $n \neq 0, k = 0$}

$F \subset A \times \emptyset = \emptyset$, но если $A \neq \emptyset$, то такое соответствие не является отображением, так как любому элементу из $A$ ничего не соответствует (частично определенная функция).

\subsubsection*{Свойства множества в степени множества}

\begin{enumerate}
	\item $(A \times B)^C = A^C \times B^C$
	\item $B \cap C = \emptyset \Ra A^{B \cup C} = A^B \times A^C$
	\item $A^{B \times C} = (A^B)^C$
\end{enumerate}