\subsection{Проблема Эрдеша-Гинзбурга-Зива}

\begin{theorem} (Теорема Эрдеша-Гинзбурга-Зива для одномерного случая)
	Для любых целых чисел $a_1, \ldots, a_{2n - 1}$ существует подмножество $I \subset \{1, 2, \ldots, 2n - 1\}$ такое, что $|I| = n$, а также
	\[
		\suml_{i \in I} a_i \equiv 0 \pmod n
	\]
\end{theorem}

\begin{note}
	Числа в теореме могут быть любыми, то есть и совпадать тоже. Более того, нам не важно само число, а только его вычет в кольце $n$, поэтому можно говорить, что мы имеем дело не с числами, а с вычетами и тем самым позволяем себе брать любого представителя класса вычета.
	
	Теорема важна тем, что даёт точную нижнюю грань для количества $a_i$: если взять $2n - 2$ числа, то она верна не всегда. Контрпримером будет
	\[
		\underbrace{0, \ldots, 0}_{n - 1}, \underbrace{1, \ldots, 1}_{n - 1}
	\]
	Всего чисел здесь $2n - 2$, но какие ни возьми, делиться на $n$ без остатка они не будут.
\end{note}

\begin{proof}
	Доказательство состоит из двух частей:
	\begin{enumerate}
		\item $n = p$ - простое.
		
		Обозначим за $S = \suml_{I \subset \{1, \ldots, 2p - 1\} \over |I| = p} \left(\suml_{i \in I} a_i\right)^{p - 1}$. Предположим, что теорема не выполнена:
		\[
			\forall I \subset \{1, \ldots, 2p - 1\},\ |I| = p \Ra \suml_{i \in I} a_i \not\equiv 0 \pmod p
		\]
		Но при этом, согласно малой теореме Ферма мы знаем, что
		\[
			\left(\suml_{i \in I} a_i\right)^{p - 1} \equiv 1 \pmod p
		\]
		Отсюда имеем следующее:
		\[
			S \equiv C_{2p - 1}^p \pmod p
		\]
		Если вычислять значения числа сочетаний в кольце, то можно заметить факт:
		\[
			C_{2p - 1}^p \equiv 1 \pmod p
		\]
		Остаётся его только доказать. Для этого обратим внимание на соотношение:
		\[
			C_{2p}^p = 2C_{2p - 1}^p
		\]
		То есть достаточно доказать, что $C_{2p}^p \equiv 2 \pmod p$. Рассмотрим комбинаторное тождество:
		\[
			C_{2p}^0 + \ldots + C_{2p}^p + \ldots + C_{2p}^{2p} = 4^p
		\]
		
		Будем рассматривать $p > 5$, а для меньших просто проверим руками. Тогда $4^p = 4^{p - 1} \cdot 4 \equiv 4 \pmod p$. При этом $C_{2p}^0 = C_{2p}^{2p} = 1 \equiv 1 \pmod p$. Что можно сказать про $C_{2p}^k,\ k \in [1; p - 1]$ (для $k > p$ всё зеркально)?
		\[
			C_{2p}^k = \frac{(2p)!}{k!(2p - k)!} = \frac{2p \cdot (2p - 1) \cdot \ldots \cdot (2p - k + 1)}{k!}
		\]
		Коль скоро $k!$ не делится на $p$, то делимость на $p$ зависит только от числителя, который, очевидно, на $p$ делится. Отсюда
		\[
			C_{2p}^k \equiv 0 \pmod p
		\]
		В итоге мы получили, что
		\[
			S \equiv 1 \pmod p
		\]
		Осталось в рамках предположения опровергнуть его при помощи доказательства делимости $S$ на $n$. Для этого придётся совершить ужасное дело: раскрыть $(p - 1)$-ю степень внутренней суммы и перегруппировать слагаемые. Выберем произвольное $I \subset \{1, \ldots, 2p - 1\}$. Из него выберем подмножество индексов $\{i_1, \ldots, i_q\} \subset I$ - это элементы, чьи степени в слагаемом положительны. То есть
		\[
			\left(\suml_{i \in I} a_i\right)^{p - 1} = \suml_{\{i_1, \ldots, i_q\} \subset I \over 1 \le q \le p - 1} P(\alpha_{i_1}, \ldots, \alpha_{i_q}) \cdot a_{i_1}^{\alpha_{i_1}} \cdot \ldots \cdot a_{i_q}^{\alpha_{i_q}}
		\]
		При этом выполнены условия:
		\begin{align*}
			&{\forall l \in [1; q]\ \alpha_{i_l} \ge 1}
			\\
			&{\alpha_{i_1} + \ldots + \alpha_{i_q} = p - 1}
		\end{align*}
		Теперь зададимся вопросом: а сколько же таких слагаемых, которые записаны под суммой, получится, если расписать внешнюю сумму? Ровно столько, сколько есть множеств $I \supset \{i_1, \ldots, i_q\}$ для некоторого конкретного набора $\{i_1, \ldots, i_q\}$. Это число в точности равно $C_{2p - 1 - q}^{p - q}$. Осталось заметить, что оно всегда делится на $p$ для $q \in [1; n - 1]$:
		\[
			C_{2p - 1 - q}^{p - q} = \frac{(2p - 1 - q)!}{(p - q)! \cdot (p - 1)!}
		\]
		Ни одно из нижних слагаемых не делится на $p$, а числитель точно содержит в себе $p!$. Значит, мы можем перегруппировать слагаемые в $S$ так, что каждое делится на $p$. Следовательно
		\[
			S \equiv 0 \pmod p
		\]
		Противоречие.
		
		\item Доказать, что если $n, m$ удовлетворяют теореме ЭГЗ, то и $n \cdot m$ удовлетворяет ей.
	\end{enumerate}
	По основной теореме арифметики любое натуральное число больше единицы представляется единственным образом как произведение простых в некоторых степенях, для которых мы знаем, что теорема верна. $\Ra$ доказали для всех натуральных чисел (единица тривиальна).
\end{proof}

Изначальная проблема была сформулирована для целых чисел. Но что мешает обобщить её, скажем, на $\Z^d,\ d \ge 1$? Так и поступили учёные. История случая $d = 2$ такова:
\begin{enumerate}
	\item В 70е годы Кемниц высказал гипотезу, что в $\Z^2$ теорема Эрдеша-Гинзбурга-Зива будет верна для $4n - 3$ пар чисел. Для $4n - 4$ существует простой контрпример: нужно взять $n - 1$ пару $(0, 0)$, $n - 1$ пару $(0, 1)$, $n - 1$ пару $(1, 0)$ и $n - 1$ пару $(1, 1)$. Несложно увидеть, что сумма $n$ пар никогда не будет делиться на $n$.
	
	\item В 90е математиками Алоном и Дубинером было доказано, что начиная с некоторого $n_0$ минимальное количество пар чисел точно $\le 6n - 5$.
	
	\item В 2004м году математик Р\'{о}ньяи доказал, что минимальное количество пар не может быть больше $4n - 2$.
	
	\item В 2006м году математик Райер (нем. \textit{Reiher}) подтвердил гипотезу Кемница и доказал теорему Эрдеша-Гинзбурга-Зива в случае $d = 2$.
\end{enumerate}

\begin{theorem} (Теорема Роньяи)
	Для любого множества пар $(a_1, b_1), \ldots, (a_{4n - 2}, b_{4n - 2})$ найдётся подмножество $I \in \{1, \ldots, 4n - 2\},\ |I| = n$ такое, что
	\[
		\suml_{i \in I} a_i \equiv \suml_{i \in I} b_i \equiv 0 \pmod n
	\]
\end{theorem}

Перед тем, как доказывать теорему, нам необходимо ввести обозначение для многочлена многих переменных - $F(x_1, \ldots, x_n)$.
\begin{example}
	\(F(x_1, x_2) = x_1^3 + x_1^2 x_2 + 3x_2^{15}\)
\end{example}

При этом обычно рассматривают многочлены над некоторым полем (то, откуда берутся коэффициенты). В нашем случае мы будет работать с многочленами многих переменных над полем $\Z_p$:
\[
	F \in \Z_p[x_1, \ldots, x_n]
\]

\begin{theorem} (Теорема Шевалле)
	Пусть $F \in \Z_p[x_1, \ldots, x_n]$, $\Deg F < n$ и при этом $N_p$ - число решений сравнения $F(x_1, \ldots, x_n) \equiv 0 \pmod p$ (естественно решения $(x_1, \ldots, x_n) \in \Z_p^n$. То есть $N_p \le p^n$). Тогда утверждается, что
	\[
		N_p \equiv 0 \pmod p
	\]
\end{theorem}

\begin{proof}
	Для начала заметим способ, которым можно сосчитать вычет $N_p$:
	\[
		N_p \equiv \suml_{x_1 = 1}^p \suml_{x_2 = 1}^p \ldots \suml_{x_n = 1}^p \left(1 - F^{p - 1}(x_1, \ldots, x_n)\right) \pmod p
	\]
	Суммы в конечном итоге фиксируют набор $(x_1, \ldots, x_n)$.
	
	Если он является решением, то $F(x_1, \ldots, x_n) \equiv 0 \pmod n$ и следовательно $p - 1$ степень тоже имеет вычет 0, то есть в сумму добавится единица.
	
	Если же верно $F(x_1, \ldots, x_n) \not\equiv 0 \pmod n$, то по малой теореме Ферма $p - 1$ степень значения многочлена будет сравнима с единицей по модулю $p$. Стало быть, такой многочлен обратит слагаемое в 0 и не будет никак учтён.
	
	При этом заметим, что если раскрыть всё суммирование в слагаемые, то все единицы в сумме дадут $p^n \Ra$ не имеют вклада в вычет по модулю $p$. Это означает, что нам надо проверить лишь следующее утверждение:
	\[
		\suml_{x_1 = 1}^p \ldots \suml_{x_n = 1}^p F^{p - 1}(x_1, \ldots, x_n) \equiv 0 \pmod p
	\]
	Отметим, что оценка на степень многочлена в сумме - это $\Deg F^{p - 1} \le (n - 1) \cdot (p - 1)$. Рассмотрим произвольный моном в этом многочлене. Он имеет вид $C \cdot x_1^{\alpha_1} \cdot \ldots x_n^{\alpha_n}$, где $0 \le \alpha_i \le \Deg F^{p - 1}$ и $\suml_{i = 1}^n \alpha_i = \Deg F^{p - 1}$. Если мы докажем утверждение ниже, то докажем и предыдущее тоже:
	\[
		\suml_{x_1 = 1}^p \ldots \suml_{x_n = 1}^p x_1^{\alpha_1} \cdot \ldots x_n^{\alpha_n} \equiv 0 \pmod p
	\]
	\begin{enumerate}
		\item $p = 2$. Тогда $\alpha_1 + \ldots + \alpha_n \le (n - 1)(p - 1) = n - 1$. По принципу Дирихле это значит, что найдётся хотя бы один $\alpha_i = 0$, а так как мы можем произвольно менять знаки суммирования, то <<подвинув>> суммирование $x_i$ вправо, можно вынести все остальные сомножители монома за знак суммирования по $x_i$ и получить внутри следующее:
		\[
			\suml_{x_i = 1}^p x_i^{\alpha_i} = p \equiv 0 \pmod p
		\]
		
		\item $p \ge 3$. Эта ситуация тоже разбивается на 2:
		\begin{enumerate}
			\item Нашлось $\alpha_i = 0$. Тогда действуем как в первом случае
			
			\item $\forall i \in [1; n]\ \alpha_i \ge 1$, но при этом всё ещё верно $\alpha_1 + \ldots + \alpha_n \le (n - 1)(p - 1)$. Тогда, согласно принципу Дирихле:
			\[
				\exists i \in [1; n] \such 1 \le \alpha_i \le p - 2
			\]
		\end{enumerate}
		\begin{proposition}
			Если $1 \le \alpha_i \le p - 2$, то существует $a > 1$ такое, что
			\[
				a^{\alpha_i} \not\equiv 1 \pmod p
			\]
		\end{proposition}
		\begin{proof}
			Если перенести единицу влево, то получим
			\[
				a^{\alpha_i} - 1 \not\equiv 0 \pmod p
			\]
			Теперь слева записан многочлен степени $\alpha_i$ над кольцом $\Z_p$. В курсе алгебры доказывается, что у этого многочлена $\alpha_i$ корней, а так как $a = 1$ является корнем, то отличных от 1 и 0 у него $\alpha_i - 1 < p - 2$ корней, что меньше числа вычетов, удовлетворяющих тем же условиям $\Ra$ найдётся нужное $a > 1$.
		\end{proof}
		
		Зная этот факт, обозначим за $S = \suml_{x_i = 1}^p x_i^{\alpha_i}$ и рассмотрим выражение $a^{\alpha_i} \cdot S$:
		\[
			a^{\alpha_i} \cdot S = \suml_{x_i = 1}^p (ax_i)^{\alpha_i} \equiv \suml_{x_i = 1}^p x_i^{\alpha_i} = S \pmod p
		\]
		То есть
		\[
			a^{\alpha_i} \cdot S \equiv S \pmod p \Ra S(a^{\alpha_i} - 1) \pmod p
		\]
		Так как $a^{\alpha_i} \not\equiv 1 \pmod p$, то $(a^{\alpha_i} - 1) \not\equiv 0 \pmod p$. Значит
		\[
			S \equiv 0 \pmod p
		\]
		Что и требовалось доказать.
	\end{enumerate}
\end{proof}