\section{Основы комбинаторики}

\subsection{Правило сложения}

\begin{proposition}
	Пусть у нас есть множество $A$, содержащее $n$ объектов, и $B$, содержащее $m$ объектов. Тогда, число способов выбрать 1 объект из $A$ \textbf{или} один объект из $B$ равно $n + m$.
\end{proposition}

\subsection{Правило умножения}

\begin{proposition}
	Пусть у нас есть множество $A$, содержащее $n$ объектов, и $B$, содержащее $m$ объектов. Тогда, число способов выбрать 1 объект из $A$ \textbf{и} один объект из $B$ равно $n \cdot m$.
\end{proposition}

\subsection{Способы выбора объектов из множества}

\subsubsection*{Размещения с повторениями}

\begin{definition}
	Числом $\bar{A}_n^k$ называется количество способов выбрать $k$ элементов из множества $n$ элементов так, что при этом нам \textbf{важен} порядок выбора и мы \textbf{допускаем} повторения элементов.
\end{definition}

\begin{note}
	Читается как ``$A$ из $n$ по $k$ с чертой``
\end{note}

\begin{note}
	Размещение из $k$ элементов с повторениями также называют \textit{$k$-размещением с повторениями}
\end{note}

\begin{theorem}
	\[
		\bar{A}_n^k = n^k
	\]
\end{theorem}

\begin{proof}
	Сколько способов выбрать $i$-й элемент ($1 \le i \le k$)? Ровно $n$. Мы последовательно выбираем 1-й \textbf{и} 2-й \textbf{и} 3-й \textbf{и} ... \textbf{и} $k$-й. То есть
	\[
		\bar{A}_n^k = \underbrace{n \cdot \ldots \cdot n}_{\text{k раз}} = n^k
	\]
\end{proof}

\subsubsection*{Размещения без повторений}

\begin{definition}
	\textit{Факториалом} числа $n > 0$ называют число
	\[
		n! = n \cdot (n - 1) \cdot \ldots \cdot 2 \cdot 1
	\]
	При этом считают, что
	\[
		0! = 1! = 1
	\]
\end{definition}

\begin{definition}
	Числом $A_n^k$ называется количество способов выбрать $k$ элементов из множества $n$ элементов так, что при этом нам \textbf{важен} порядок выбора и мы \textbf{не допускаем} повторения элементов.
\end{definition}

\begin{note}
	Читается как ``$A$ из $n$ по $k$``
\end{note}

\begin{note}
	Размещение из $k$ элементов без повторений также называют \textit{$k$-размещением без повторений}.
	
	При этом $n$-размещение без повторений называется \textit{перестановкой}.
\end{note}

\begin{theorem}
	\[
		A_n^k = n \cdot (n - 1) \cdot \ldots \cdot (n - k + 1) = \frac{n!}{(n - k)!}
	\]
\end{theorem}

\begin{proof}
	Сколько способов выбрать $i$-й элемент ($1 \le i \le k$)? Мы можем взять лишь те элементы, которые ещё не брали. Таких $n - i + 1$. Мы выбираем 1-й \textbf{и} 2-й \textbf{и} 3-й \textbf{и} ... \textbf{и} $k$-й. То есть
	\[
		A_n^k = n \cdot (n - 1) \cdot \ldots \cdot (n - k + 1) = \frac{n!}{(n - k)!}
	\]
\end{proof}

\subsubsection*{Сочетания без повторений}

\begin{definition}
	Множество элементов, из которых составлен объект (то же множество, перестановка или размещение), называется \textit{набором} или же \textit{сочетанием}.
\end{definition}

\begin{definition}
	Числом $C_n^k$ называется количество способов выбрать $k$ элементов из множества $n$ элементов так, что при этом нам \textbf{не важен} порядок выбора и мы \textbf{не допускаем} повторения элементов. (То есть количество различных наборов размера $k$, которые можно получить из $n$ элементного множества)
\end{definition}

\begin{note}
	Читается как ``$C$ из $n$ по $k$``
\end{note}

\begin{note}
	Сочетание из $k$ элементов без повторений также называют \textit{$k$-сочетанием без повторений}.
\end{note}

\begin{theorem}
	\[
		C_n^k = \frac{A_n^k}{k!} = \frac{n!}{k! (n - k)!}
	\]
\end{theorem}

\begin{proof}
	Для некоторого множества из $n$ элементов мы знаем число $k$-размещений без повторений. При этом, все размещения можно распределить по группам, где все имеют одинаковый набор. Сколько существует $k$-размещений из множества с $k$ элементами?
	\[
		A_k^k = \frac{k!}{0!} = k!
	\]
	Это значит, что всего различных наборов из множества $n$ элементов будет
	\[
		C_n^k = \frac{A_n^k}{A_k^k} = \frac{A_n^k}{k!} = \frac{n!}{k! (n - k)!}
	\]
\end{proof}

\subsubsection*{Сочетания с повторениями}

\begin{definition}
	Числом $\bar{C}_n^k$ называется количество способов выбрать $k$ элементов из множества $n$ элементов так, что при этом нам \textbf{не важен} порядок выбора и мы \textbf{допускаем} повторения элементов. (Количество различных $k$-сочетаний с повторениями)
\end{definition}

\begin{theorem}
	\[
		\bar{C_n^k} = C_{n + k - 1}^k
	\]
\end{theorem}

\begin{proof}
	Поймём, что любой набор с повторениями определяется числами вхождений каждого элемента в данный набор. Для определённости будем считать, что мы работаем с множеством $A$:
	\[
		A = \{a_1, \dots, a_n\}
	\]
	Обозначим как $u_i$ - число вхождений $a_i$ в данный набор. Тогда сразу следует равенство
	\[
		u_1 + \ldots + u_n = k
	\]
	Теперь, построим последовательность нулей и единиц, которая однозначно задаст нам набор:
	\[
		\underbrace{1 \dots 1}_{u_1 \text{ раз}}
		0
		\underbrace{1 \dots 1}_{u_2 \text{ раз}}
		0 \dots 0
		\underbrace{1 \dots 1}_{u_n \text{ раз}}
	\]
	То есть мы записываем между нулями-разделителями столько единиц, сколько у нас имеется $a_i$ в наборе.
	\begin{itemize}
		\item Сколько всего нулей? - $n - 1$
		\item Сколько всего единиц? - $k$
		\item Какова длина всей последовательности? - $n + k - 1$
	\end{itemize}
	При этом длина последовательности всегда одинакова. Давайте просто выберем $k$ позиций среди $n + k - 1$ в ней, куда мы поставим единицы, а в остальных местах будут нули. Тогда мы получим какую-то последовательность, которая точно описывает какой-то из наборов. Отсюда
	\[
		\bar{C}_n^k = C_{n + k - 1}^k
	\]
\end{proof}

\subsection{Принцип Дирихле}

\begin{definition}
	Пусть у нас есть $n + 1$ кролик и $n$ клеток для них. Тогда абсолютно очевидно, что если мы заполним все клетки, то в одной из них будет 2 кролика. Это и называется \textit{принципом Дирихле}. Более формально ещё можно сказать так:
	
	\textit{
	Если у нас есть $nk + 1$ объект и $n$ ящиков, то в каком-нибудь ящике окажется не менее $k + 1$ объектов.
	}
\end{definition}

\begin{example}
	У нас есть квадрат со стороной 2. Если мы выберем 5 произвольных точек на его границах или внутри него, то хотя бы 2 из них будут на расстоянии не более $\sqrt{2}$.
\end{example}

\begin{proof}
	Разделим квадрат на 4 меньших со стороной 1. Тогда, по принципу Дирихле хотя бы в одном из таких квадратов будет 2 точки, а наибольшее расстояние между точками в квадрате - это его диагональ, то есть $\sqrt{1^2 + 1^2} = \sqrt{2}$
\end{proof}

\subsection{Бином Ньютона}

\begin{definition}
	\textit{Биномом Ньютона} называется выражение
	\[
		(a + b)^n = \suml_{k = 0}^n C_n^k a^k b^{n - k}
	\]
\end{definition}

\begin{proof}
	$n$-ю степень суммы можно записать в виде
	\[
		(a + b)^n = \underbrace{(a + b) \cdot (a + b) \cdot \ldots \cdot (a + b)}_{n \text{ раз}}
	\]
	Чтобы получить слагаемое в сумме, мы должны последовательно выбрать из каждой скобки $a$ или $b$. Любое слагаемое точно будет иметь вид
	\[
		a^k \cdot b^{n - k}
	\]
	Более того, чтобы определить слагаемое, нам необходимо и достаточно знать, сколько $a$ мы выбрали. При этом выбирать его можно в любых скобках, а это можно сделать $C_n^k$ способами. Отсюда и формула
	\[
		(a + b)^n = \suml_{k = 0}^n C_n^k a^k b^{n - k}
	\]
\end{proof}

\subsection{Свойства биномиальных коэффициентов}

\begin{theorem}~
	\begin{enumerate}
		\item $C_n^k = C_n^{n - k}$
		\item $C_n^k = C_{n - 1}^k + C_{n - 1}^{k - 1}$
		\item $C_n^0 + C_n^1 + C_n^2 + \ldots + C_n^n = 2^n$
		\item $\left(C_n^0\right)^2 + \ldots + \left(C_n^n\right)^2 = C_{2n}^n$
		\item $C_{n + m}^{n - 1} + C_{n + m - 1}^{n - 1} + \ldots + C_{n - 1}^{n - 1} = C_{n + m}^n = C_{n + m}^m,\ m \ge 0$
	\end{enumerate}
\end{theorem}

\begin{proof}~
	\begin{enumerate}
		\item Выбрать $k$ объектов из $n$ - это то же самое, что оставить $n - k$ объектов из $n$.
		
		\item Количество способов выбрать $k$-набор из $n$ элементного множества уже известно: $C_n^k$. Заметим, что каждый набор либо содержит $n$-й элемент, либо нет. То есть все наборы можно разбить на 2 группы:
		\begin{enumerate}
			\item Все наборы, которые содержат $n$-й элемент. Помимо него в них ещё надо выбрать $k - 1$ элемент, а стало быть, их всего $C_{n - 1}^{k - 1}$ штук.
			
			\item Все наборы, которые не содержат $n$-й элемент. То есть набор выбирается только из первых $n - 1$ элементов. Отсюда их $C_{n - 1}^k$ штук.
		\end{enumerate}
		Так как эти две группы в сумме составляют все возможные наборы, то и очевиден ответ
		\[
			C_n^k = C_{n - 1}^k + C_{n - 1}^{k - 1}
		\]
		
		\item Сколько существует подмножеств у множества $n$ элементов? - $2^n$. С другой стороны, каждое из этих подмножеств характеризуется своей мощностью, а число подмножеств, чья мощность $i$, равно $C_n^i$. Отсюда
		\[
			\suml_{i = 0}^n C_n^i = 2^n
		\]
		
		\item Рассмотрим всевозможные $n$-наборы из $2n$ элементного множества. Их $C_{2n}^n$ штук. С другой стороны, пусть $i \in [0, \ldots, n]$ - количество элементов для набора, которые мы возьмём из первой части исходного множества. Тогда из второй части мы выберем $n - i$ элементов. В итоге, получим сумму
		\[
			\suml_{i = 0}^n C_n^i \cdot C_n^{n - i} = \suml_{i = 0}^n \left(C_n^i\right)^2 = C_{2n}^n
		\]
		
		\item Давайте рассмотрим всевозможные $m$-сочетания с повторениями в множестве $A = \{a_1, \ldots, a_{n + 1}\}$. Их $\bar{C}_{n + 1}^m = C_{n + 1 + m - 1}^m = C_{n + m}^m = C_{n + m}^n$ штук. С другой стороны, каждое сочетание принадлежит группе наборов, содержащей $i \in [0, \ldots, m]$ элементов $a_1$. Отсюда
		\[
			C_{n + m}^n = \suml_{i = 0}^m \bar{C}_n^{m - i} = \suml_{i = 0}^m C_{n + m - i - 1}^{m - i} = \suml_{i = 0}^m C_{n + m - i - 1}^{(n + m - i - 1) - (m - i)} = \suml_{i = 0}^m C_{n + m - i}^{n - 1}
		\]
	\end{enumerate}
\end{proof}