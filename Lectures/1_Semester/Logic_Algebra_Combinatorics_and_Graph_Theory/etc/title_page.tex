%%% Если будут вопросы по преамбуле, не стесняйтесь спрашивать

%%% Всю шаблонную информацию можно менять тут
\newcommand{\FullCourseNameFirstPart}{\so{АЛГЕБРА~ЛОГИКИ,~КОМБИНАТОРИКА}}
\newcommand{\FullCourseNameSecondPart}{\so{и~ТЕОРИЯ~ГРАФОВ}}
\newcommand{\SchoolName}{ФПМИ}
\newcommand{\TrackName}{МФ}
\newcommand{\SemesterNumber}{I}
\newcommand{\LecturerInitials}{Рубцов Александр Алексаднрович}
\newcommand{\AutherInitials}{Халин Алексей}
\newcommand{\VKLink}{https://vk.com/zythorn}
\newcommand{\OverleafLink}{https://www.overleaf.com/}
\newcommand{\GithubLink}{https://github.com/MIPT-Group/Lectures_Tex_Club}

\begin{titlepage}
	\clearpage\thispagestyle{empty}
	\centering
	
	\textit{Федеральное государственное автономное учреждение \\
		высшего профессионального образования}
	\vspace{0.5ex}
	
	\textbf{Московский Физико-Технический Институт \\ КЛУБ ТЕХА ЛЕКЦИЙ}
	\vspace{20ex}
	
	\rule{\linewidth}{0.5mm}
	{\textbf{\FullCourseNameFirstPart}}
	\\
	{\textbf{\FullCourseNameSecondPart}}
	\rule{\linewidth}{0.5mm}
	
	\SemesterNumber\ СЕМЕСТР
	\\
	Физтех Школа: \textit{\SchoolName}
	\\
	Направление: \textit{\TrackName}
	\\
	Лектор: \textit{\LecturerInitials}
	\vspace{1ex}
	
	\begin{figure}[!ht]
		\centering
		\includegraphics[width=0.4\textwidth]{logo_LTC.png}
		\label{fig:my_label}
	\end{figure}
\begin{flushright}
	\noindent
	Автор: \href{\VKLink}{\textit{\AutherInitials}}
	\\
	\href{\GithubLink}{\textit{Проект на github}} % Опционально, если хотите учавствовать в рейтинге
\end{flushright}
	
	\vfill
	\CourseDate\ года
	\pagebreak
	
\end{titlepage}
