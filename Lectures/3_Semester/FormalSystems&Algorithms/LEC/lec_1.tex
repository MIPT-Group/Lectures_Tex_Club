\section{Лекция 1 : Введение в математическую логику и теорию алгоритмов}
\noindent Основная идея математической логики: оперировать математические рассуждениями с точки зрения математики -- <<метематематика>>, для этого необходимо формализовать эти рассуждения: построить некие математические объекты, отражающие приемы выводов, доказательств и т.п. , в частности это даст возможность доказывать разрешимость/неразрешимость задач в заданных формальных системах. \\
При формализации рассуждений необходимо различать:
\begin{itemize}
\item \tit{семантика} -- о чем рассуждения
\item \tit{синтаксис} -- формальное представление математических рассуждений
\end{itemize}
В данной лекции будут рассматриваться булевы функции : \\ 
$f: \figbr{0, 1}^n \rar \figbr{0, 1}$ ($0$ отождествляется с <<ложью>>, ложными высказываниями, $1$ - с <<истинной>>)
\begin{definition}
\tit{семантическое следствие} \\
булева функция $g$ -- семантическое следствие $f_1, f_2, \dots , f_n$ ($f_1, f_2, \dots , f_n \hm{\vDash} g $) если из $f_1(x) = f_2(x) =\dots = f_n(x) = 1$ следует, что $g(x) = 1$
\end{definition}
\begin{itemize}
\item $f = x_1 \XOR x_2 \XOR \dots \XOR x_n, g = x_1 \lor x_2 \lor \dots \lor x_n, ~ f \vDash g$ -- действительно $f=1$ когда нечетное количество переменных $x_i =1$, как следствие хотя бы одна переменная равна 1, единственный набор на котором $g = 0$ : $x_1 = x_2 = \dots x_n = 0$ 
\item $x, x \rar y \vDash y$  -- $x = 1$, импликация $1 \to y = 1$ только при $y=1$
\end{itemize}
\begin{definition} \tit{булева формула} \\
задан алфавит : $\figbr{x_i, \lor, \land, \neg ,\rar, (, )}$ (неограниченное число переменных, скобки ,логические операции : конъюнкция, дизъюнкция, импликация, отрицание). Множество всех булевых формул задает следующим образом:
\begin{itemize}
\item $x_i$ -- формула 
\item если $A$, $B$ -- формулы, то $\neg A, A \land B, A \lor B, A \rar B$  -- формулы (рекурсивная зависимость)
\item других формул нет
\end{itemize}
\end{definition}
\noindent Каждая булева формула задает булеву функцию, каждая булева функция может быть задана булевой формулой (в общем говоря не единственной )
\begin{definition} \tit{тавтология} \\
формула, задающая функцию, тождественно равной 1, называется тавтологией
\end{definition}
\begin{example}
Функции $ x \rar x, ~ x \lor \neg x$ являются тавтологичны    
\end{example}
Для функцией от небольшого числа переменных тавтологичность легко проверить построением таблицы значений, но этот способ плох для функций от большого числа переменных, т.к. количество возможных наборов значений переменных растет экспоненциально.\\
Другим способом проверки может служить перебор случаев. На пример, докажем, что $A \rar(B \rar A)$ -- тавтология. Импликация ложна только в одно случае: $A = 1 \land B \rar A = 0$, но $B \rar A = 0 \Rar B = 1 \land A = 0$ -- противоречие, других возможных случаев обращение в 0 нет, значит рассматриваемая формула тавтология.\\
рассмотрим некую систему доказательств: $F$ -- утверждение, $P$ --некое доказательство, $A$ -- алгоритм проверки доказательства: $A(F, P) = 1$ в случае верного доказательства, в противном случае $A(F, P) = 0$.
\begin{itemize}
\item если $F$ -- тавтология, то $\exists P : A(F, P) = 1$
\item если $F$ -- не тавтология, то $\forall P ~A(F, P) = 0$
\end{itemize}
иными словами: возможно доказать любую тавтологию и любая не тавтология недоказуема.
\subsection{Исчисление высказывание (ИВ)}
Для простоты достаточно рассматривать только булевы функции $\rar, \neg$, т.к. они образуют базис (в частности : $x \lor y =  \neg x \rar y, ~ x \land y = \neg(x \rar \neg y)$)\\
Рассмотрим три любые формулы: $A, B, C$. Рассмотрим схемы аксиом: \begin{enumerate}
\item  $A \rar (B \rar A)$
\item $A \rar (B \rar C) \rar \br{\br{A \rar B} \rar \br{A \rar C}}$
\item $\br{\neg B \rar \neg A} \rar \br{\br{\neg B \rar A} \to B}$
\end{enumerate}
\begin{theorem}
все 3 схемы аксиом являются тавтологиями.
\end{theorem}
\beginproof тавтологичность схемы 1 была доказана выше.

Докажем тавтологичность второй схемы. Предположим, что существует такой набор значений $A, B, C$, на котором  формула 2 $=0$, тогда т.к. импликация $= 0$ только как $0 \to 1$, получаем:
\begin{equation} \label{eq:main_system}
\left\{ \begin{aligned} 
\br{A \to B} \to \br{A \to C} = 0 \\
A \to \br{B \to C} = 1 
\end{aligned} \right.
\end{equation} 
Из первой строчки системы \eqref{eq:main_system}
\begin{equation} \label{eq:folow_system}
\left\{ \begin{aligned} 
A \to B = 1 \\
A \to C = 0
\end{aligned} \right.
\end{equation}
Единственное решение системы \eqref{eq:folow_system} : $A = B = 1, ~ C = 0$. Подставляя этот набор во второе уравнение системы \eqref{eq:main_system}: $1 \to \br{1 \to 0} = 0$ -- противоречие.

Для доказательства тавтологичности третьей схемы построим таблицу значений:
\begin{table}[!h]
\begin{tabular}{|c|c|c|c|c|c|}
\hline 
A & B & $\neg B \to \neg A$ & $\neg B \to A$ & $\br{\neg B \to A} \to B$ & $\br{\neg B \to \neg A} \to \br{\br{\neg B \to A} \to B}$ \\ 
\hline 
0 & 0 & 1 & 0 & 1 & 1 \\ 
\hline 
0 & 1 & 1 & 1 & 1 & 1 \\ 
\hline 
1 & 0 & 0 & 1 & 0 & 1 \\ 
\hline 
1 & 1 & 1 & 1 & 1 & 1 \\ 
\hline 
\end{tabular} 
\end{table}

Видим, что на всех возможных значений $A, B$ схема 3 равна 1, тавтологичность доказана \qed
\begin{definition} \label{def:MP} \tit{правило вывода Modus ponens}\\
$\frac{A, A \rar B}{B}$, то есть: если $A$ и $A\to B$ — истинные высказывания, то $B$ также истинно.
\end{definition}
\begin{definition} \tit{Вывод}\\
Зафиксируем множество формул $G$. Вывод - цепочка формул $F_1, F_2, \dots , F_n$, где для каждой $F_i$ выполняется хотя бы одно из трех условий: 
\begin{itemize}
\item $F_i$ -- аксиома
\item $F_i \in G $
\item $\exists k, j < i : F_k = F_j \to F_i$ (применение Modus ponens )
\end{itemize}
\end{definition}
\begin{definition} \tit{синтаксическое следствие} \\
$F$ -- синтаксическое следствие из $G$ ($G \vdash F$) если существует вывод $F$ из $Г$
\end{definition}
\begin{definition} \tit{выводимые в ИВ формулы}\\
формулы, выводимые из набора аксиом -- выводимые в ИВ формулы
\end{definition}
Для того чтобы доказать, что исчисление высказываний есть система доказательств тавтологии нужно доказать равносильность семантического и синтаксического следствия\\
В этой лекции будет доказано только утверждения в одну сторону: теорема корректности.
\subsection{Теорема о корректности} 
\begin{theorem} \label{th:correct} \tit{теорема корректности}\\
\begin{equation}
G \vdash F \Rar G \vDash F 
\end{equation}
Другими словами, в ИВ выводими только тавтологии
\end{theorem}
\beginproof
Доказывать будем индукцией по длине вывода\\
Рассмотрим произвольный вектор значений переменных $\al: F(\al) \hm{=} 1, \forall F \in G$

\tit{База индукции:} длина вывода = 1, пусть вывод состоит из одной формулы $A$, тогда $A$ или аксиома, или $A \in G$ (см. определение вывода). по теорема 1.1. получаем, что $A$ или тавтология, тогда утверждение очевидно выполняется, если же $A \in G$, получаем, что $A \to A$, что так же является тавтологией.

\tit{Шаг индукции:} Предположим, что теорема выполняется для всех выводов, длина которых не больше $n$. Докажем, что теорема справедлива и для любого вывода, длины $n+1$. Рассмотри последнюю $A_{n+1}$ строчку вывода. Случаи, когда $A_{n+1}$ аксиома, или $A_{n+1} \in G$ разобраны выше, в доказательстве базы. Остается единственный случай: $A_{n+1}$ выведено через modus ponens : $\frac{A_k, A_k \to A_{n+1}}{A_{n+1}}$. Тогда по предположению индукции $A_{k}(\al)=1$ и как было показано выше: $A_k, A_k \to A_{n+1} \vDash A_{n+1}$ \qed
