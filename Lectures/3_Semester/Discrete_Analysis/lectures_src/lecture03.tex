\begin{theorem}[Эрдеш, Хватал]
	Пусть $G(V, E)$ - граф, $ |V| \geqslant 3 $ и $\alpha(G) \leqslant \varkappa(G)$. Тогда $G$ -- гамильтонов.
\end{theorem}

\begin{proof}
	\:
	\begin{enumerate} \renewcommand{\theenumi}{\bfseries\arabic{enumi}}
		\item Предположим, что в $ G $ нет циклов. Поскольку $1 \leqslant \alpha(G)$ (множество из одной вершины всегда независимое), то $ \alpha(G) \leqslant \varkappa(G) \Rightarrow 1 \leqslant \alpha(G)$, следовательно, мы должны удалить хотя бы одну вершину, чтобы $ G $ стал несвязен. Отсюда $ G $ связен и без циклов, т.е. дерево. Для $ |V| < 3$ утверждение очевидно. Если же $ |V| \geqslant 3$, то в дереве найдется две висячих вершины. Две вершины являются независимым множеством мощности 2, откуда $ \alpha(G) \geqslant 2$. Но в дереве на $ \geqslant 3$ вершинах, очевидно, есть и вершина степени $ \geqslant 2$. Её удаление разбивает граф на компоненты связности, откуда $ \varkappa = 1$. Получаем противоречие.
		
		\item Пусть в $ G $ есть цикл. Рассмотрим самый длинный простой цикл и предположим, что он не покрывает все вершины: $ C = {x_1, \ldots, x_k}, k < |V| = n $. Удалим вершины цикла и обозначим новый граф за $ G^\prime $. Пусть $ W $ -- множество вершин одной из компонент в $ G^\prime $. Обозначим $ N_{w}\left(G \right) = \left\lbrace y \in V\setminus W : \exists z \in W : (y, z) \in E(G)\right\rbrace$. т.\:е. множество соседей вершин из $ W $ в исходном графе $ G $.
		
		%\begin{wrapfigure}{r}{0.2\textwidth}
		%	\centering
		%	\begin{tikzpicture}
		%		
		%		\def \n {8}
		%		\def \k {6}
		%		\def \radius {2.5cm}
		%		\def \margin {10} % margin in angles, depends on the radius
				
		%		\foreach \s in {1,...,\n}
		%		{
		%			\node[draw, circle] at ({360/\n * (\s - 1)}:\radius) %{$x_{\s}$};
		%			\draw[->, >=latex] ({360/\n * (\s - 1)+\margin}:\radius) 
		%			arc ({360/\n * (\s - 1)+\margin}:{360/\n * %(\s)-\margin}:\radius);
		%		}
		%	\end{tikzpicture}
		%\end{wrapfigure}
	
		\begin{prop}
			\:
			\begin{enumerate} \renewcommand{\theenumi}{\arabic{enumi}}
				\item $ N_{w}\left(G \right) \subseteq C$
				\item $ N_{w}\left(G \right) \neq C$
				\item $ \varkappa(G) \leqslant |N_w(G)| $
			\end{enumerate} \label{prop:N_w}
			
		\end{prop}
		
		\begin{proof}
			\:
			\begin{enumerate} \renewcommand{\theenumi}{\arabic{enumi}}
				\item Если предположить, что есть вершина не из $ C $, она после удаления $ С $ осталась в $ W $, противоречие.
				\item Если нашлись две вершины $ x_i, x_{i + 1} \in N_w(G) $, то получим более длинный цикл, заменив в $ C $ кусок $\dots \rightarrow x_i \rightarrow x_{i + 1} \rightarrow \dots$ на $ \dots \rightarrow x_i \rightarrow c \rightarrow x_{i + 1} \rightarrow \dots$.
				\item Когда мы удаляем $ N_w(G), $ по пункту (b) остается хотя бы одна вершина цикла $ C $, не соединенная с множеством $ W $, а также остается сам $ W $, т.е. граф распадается на компоненты связности.
			\end{enumerate}
		\end{proof}
		Обозначим $ M = \left\lbrace x_{i + 1} : x_i \in N_w(G)\right\rbrace $, другими словами, это - множество вершин цикла $ C $, родители которых соединены с $ W $. Очевидно, $ |M| = |N_w(G)| $
		\begin{prop}
			$ M $ - независимое множество.
		\end{prop}
		\begin{proof}
			Предположим, что множество $ M $ не независимое, т.е. $ \exists\: x_{i + 1}, x_{j + 1} \in M : (x_{i + 1}, x_{j + 1}) \in E $ и $ x_i, x_j \in N_w(G) $. Пусть вершины $ x_{i}, x_{j} $ соединены с какими-то вершинами $ a, b $ из $ M $ соответственно ($ a $ и $ b $ могут совпадать).
			
			
			Имеем исходный цикл $ C $ : $x_1 \rightarrow \dots \rightarrow x_i \rightarrow x_{i + 1} \rightarrow \dots \rightarrow x_j \rightarrow x_{j + 1} \rightarrow \dots \rightarrow x_n$. Можем рассмотреть новый цикл $ C^\prime $ :
			$x_1 \rightarrow \dots \rightarrow x_i \rightarrow a \rightarrow \dots \rightarrow b \rightarrow x_j \rightarrow \ldots \rightarrow x_{i + 1} \rightarrow x_{j + 1} \dots \rightarrow x_n \rightarrow x_1$. Он более длинный, поскольку добавились как минимум три ребра, а из исходного цикла были взяты все ребра, кроме $ x_ix_{i + 1}, x_jx_{j + 1} $. Это противоречит выбору $ C $ как наибольшего.
			
		\end{proof}
		Получаем, что $ \alpha(G) \geqslant |M|$. Но из пункта (c) утверждения \ref{prop:N_w} имеем $ \varkappa(G) \leqslant |N_w(G)| $, откуда $ \alpha(G) \geqslant  |N_w(G)| \geqslant \varkappa(G) $.
		
		\begin{note}
			[Вставить замечание] На самом деле можно улучшить оценку до ... При этом $ M \subset C $.
		\end{note}
			
	\end{enumerate}
\end{proof}

\begin{Def}
	Граф называется \emph{регулярным}, если степени всех его вершин равны.
\end{Def}

Рассмотрим следующий интересный граф и докажем, что он Гамильтонов.

$ V = \left\lbrace A \subset {1, 2, \ldots n} : |A| = 3 \right\rbrace $. Ребрами соединены два множества, если они пересекаются по одному элементу. Формально, $ E = \left\lbrace (A, B) : |A\cap B| = 1\right\rbrace  $.

Очевидно, $ |V| = C_n^3 $. Найдем $  |E| $. Степень каждой вершины $  G -- 3 C_{n - 3}^2$, поскольку для каждого множества $ {a, b, c} $ мы можем выбрать двухэлементное подмножество и дополнить его до трехэлементного одним из элементов $ {a, b, c} $. Тогда, поскольку $  G $ регулярен,  $ |E| = \frac{1}{2} \cdot C_n^3 \cdot 3 \cdot  C_{n - 3}^2 $.

Асимптотически $ |E| \sim \dfrac{3}{2} \cdot \dfrac{n^3}{6} \cdot \dfrac{n^2}{2} =\dfrac{n^5}{8} $.

Вспомним, что признаку Дирака удволетворяют графы, у которых $ m $ вершин и не меньше, чем $ \frac{m^2}{4} $ ребер.
В нашем случае, поскольку $|V| = O(n^3)$, а $|E| = O((n^3)^{\frac{5}{3}}) $. Следовательно, при больших $ n $ наш граф не будет удовлетворять признаку Дирака.

Посмотрим на $ \alpha(G) $. Укажем $ W \subset V : W = {A_1, \ldots A_k} : |A_i \cap A_j| = 1 $.
Каждому множеству $ W $ сопоставим вектор из 0 и 1 в $ \R^{|V|} $, где на позиции $ i $ стоит 1, когда $ i \in W $, и 0 иначе.

\begin{lemma}
	$ \vec{x_1}, \ldots \vec{x_k} $ линейно независимы над $ \Z_2 $
\end{lemma}  

\begin{proof}
	Предположим, что $ c_1 \vec{x_1} + \ldots + c_k \vec{x_k} = 0 $, где $ c_i = 0 $ или $ c_i = 1 $.
	Скалярно умножим выражение на $ x_1 $.
	
	Получим $ c_1 (\vec{x_1}, \vec{x_1}) + \ldots + c_k (\vec{x_k}, \vec{x_1}) = 0$
\end{proof}

Отсюда будет следовать, что $ \alpha(G) \leqslant |V| $.


*Консрукции для n-1 и т-2*

\begin{lemma}
	Пусть дан $ G = (V, E) $. Пусть $ u, v \in V $. Обозначим за $ f(u, v) $ количество общих соседей для $ u $ и $ v $. Тогда $ \varkappa(G) \geqslant \min_{v, u \in V} f(u, v) $
\end{lemma}
 \begin{proof}
 	Для удобства обозначим $ k = \min_{v, u \in V} f(u, v) $.
 	Предположим, мы удалили из графа не более $ k - 1 $ вершин и он рапсался на компоненты связности. Но тогда для любых двух вершин $ u, v $ из разных компонент, поскольку количество их соседей не меньше $ k $, одна из общих вершин не будет удалена, а следовательно между ними есть путь. Противорчечие, откуда получаем требуемое соотношение.
 \end{proof}

Воспользуемся теоремой для оценки $ \alpha(G) $ 


$ G(n, r, s) $ 

\begin{brainer}
	Рассмотрим $ G(n, \frac{n}{2}, \frac{n}{4}), n = 4k $.
	\begin{enumerate} \renewcommand{\theenumi}{\arabic{enumi}}
		\item Найдите число треугольников в этом графе
		\item Какой объект соответствует максимальной клике в этом графе?
	\end{enumerate}
\end{brainer}









