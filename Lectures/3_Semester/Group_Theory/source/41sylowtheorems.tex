\section{Теоремы Силова}

\begin{definition}
	Пусть $G$ "--- конечная группа, $|G| = n$, $p$ "--- простой делитель числа $n$, $n = p^ks$, $(p, s) = 1$. Тогда подгруппа $H \le G$ такая, что $|H| = p^k$, называется \textit{силовской $p$-подгруппой} группы $G$.
\end{definition}

\begin{note}
	Если $G$ "--- конечная группа, $|G| = n$, $t \mid n$, то необязательно в $G$ есть подгруппа порядка $t$. Например, в группе $A_5$, $|A_5| = 60$ нет подгруппы порядка 30, поскольку такая подгруппа была бы нормальной, а $A_5$ "--- простая.
\end{note}

\begin{theorem}[Первая теорема Силова]
	Пусть $G$ "--- конечная группа, $p$ "--- простой делитель числа $n$. Тогда в $G$ существует силовская $p$-подгруппа. Более того, любая $p$-подгруппа в $G$ содержится в некоторой силовской.
\end{theorem}

\begin{theorem}[Вторая теорема Силова]
	Пусть $G$ "--- конечная группа, $p$ "--- простой делитель числа $n$. Тогда все силовские $p$-подгруппы в $G$ сопряжены.
\end{theorem}

\begin{theorem}[Третья теорема Силова]
	Пусть $G$ "--- конечная группа, $p$ "--- простой делитель числа $n$, $n = p^ks$, $(p, s) = 1$. $N_p$ "--- количество силовских $p$-подгрупп в $G$. Тогда $N_p \equiv_p 1$ и $N_p \mid s$.
\end{theorem}