\begin{proposition} Пусть $G$ "--- группа. Тогда:
	\begin{enumerate}
		\item $\forall g, x, y \in G: g^{xy} = (g^x)^y$
		\item $\forall g_1, g_2, x \in G: (g_1g_2)^x = g_1^xg_2^x$
	\end{enumerate}
\end{proposition}

\begin{proof} Произведем непосредственную проверку:
	\begin{enumerate}
		\item $(g^x)^y = y^{-1}(x^{-1}gx)y = (xy)^{-1}g(xy) = g^{xy}$
		\item $g_1^xg_2^x = (x^{-1}g_1x)(x^{-1}g_2x) = x^{-1}(g_1g_2)x = (g_1g_2)^x$
	\end{enumerate}
\end{proof}

\begin{proposition}
	Сопряженность является отношением эквивалентности в группе $G$.
\end{proposition}

\begin{proof} Произведем непосредственную проверку:
	\begin{itemize}
		\item (Рефлексивность) $\forall g \in G: g = g^e$
		\item (Симметричность) Если $g_2 = g_1^x$, то $g_2^{x^{-1}} = (g_1^x)^{x^{-1}} = g_1^e = g_1$
		\item (Транзитивность) Если $g_2 = g_1^x$, $g_3 = g_2^y$, то $g_3 = (g_1^x)^y = g_1^{xy}$
	\end{itemize}
\end{proof}

\begin{definition}
	Пусть $G$ "--- группа, $g \in G$. \textit{Классом сопряженности} элемента $g$ называется множество $g^G$, то есть класс эквивалентности по отношению сопряженности, содержащий $g$.
\end{definition}

\begin{proposition}
	Пусть $G$ "--- группа, $H \le G$. Тогда $H \normal G \hm\Leftrightarrow H$ "--- объединение некоторого количества классов сопряженности в $G$.
\end{proposition}

\begin{proof}
	$H \normal G \Leftrightarrow \forall g \in G: g^{-1}Hg \subset H \Leftrightarrow \forall g \in G: \forall h \hm\in H: h^g \in H \Leftrightarrow \forall h \in H: h^G \subset H \Leftrightarrow H = \bigcup\limits_{h \in H}h^G$.
\end{proof}

\begin{exercise}
	Пусть $G$ "--- группа, $g_1, g_2 \in G$. Тогда $g_1^Gg_2^G$ "--- объединение нескольких классов сопряженности, причем необязательно одного.
\end{exercise}

\begin{solution}
	Пусть $x, y \in G, x^{-1}g_1xy^{-1}g_2y \in g_1^Gg_2^G$. Нам достаточно показать, что если $a \in G$ сопряжен с $x^{-1}g_1xy^{-1}g_2y$, то $a \in g_1^Gg_2^G$. Действительно, в силу сопряженности $a = z^{-1}x^{-1}g_1xy^{-1}g_2yz$ для некоторого $z \in G$, то есть $a = (zx)^{-1}g_1(zx)(yz)^{-1}g_2(yz) \in g_1^Gg_2^G$. Пример, когда классов сопряженности в $g_1^Gg_2^G$ действительно несколько, легко строится на основе следующего примера.
\end{solution}

\begin{example}
	Пусть $\sigma \in S_n$. Представим $\sigma$ в виде произведения независимых циклов, $\sigma \hm= (a_1\dotsc a_k)(b_1\dotsc b_l)\dotsc$, и рассмотрим $\sigma^\tau$ для произвольного $\tau \in S_n$. Если $a_i' := \tau^{-1}(a_i), i \hm\in \{1, \dots, k\}$, то $\sigma^\tau(a_i') \hm= (\tau^{-1}\sigma\tau)(a_i') = a_{i + 1}'$. Значит, $\sigma^\tau = (a_1'\dotsc a_k')(b_1'\dotsc b_l')\dotsc$, и, следовательно, $\sigma^{S_n}$ состоит из перестановок того же циклического типа, что и $\sigma$, причем из всех, потому что по каждой такой перестановке легко восстанавливается соответствующая $\tau \in S_n$.
\end{example}

\subsection{Гомоморфизмы групп}

\begin{definition}
	Пусть $G, H$ "--- группы. Отображение $\phi : G \hm\to H$ называется \textit{гомоморфизмом групп} $G$ и $H$, если $\forall g_1, g_2 \in G: \phi(g_1g_2) = \phi(g_1)\phi(g_2)$. Сюръективный гомоморфизм называется \textit{эпиморфизмом}, инъективный гомоморфизм --- \textit{мономорфизмом}.
\end{definition}

\begin{note}
	Изоморфизм групп "--- это гомоморфизм, являющийся биекцией, то есть эпиморфизмом и мономорфизмом одновременно.
\end{note}

\begin{definition}
	Пусть $G, H$ "--- группы, $\phi : G \hm\to H$ "--- гомоморфизм $G$ и $H$. Тогда:
	\begin{itemize}
		\item \textit{Образом} $\phi$ называется $\im\phi := \{\phi(g): g \in G\} = \phi(G)$
		\item \textit{Ядром} $\phi$ называется $\ke\phi := \{g \in G: \phi(g) = e\} = \phi^{-1}(e)$
	\end{itemize}
\end{definition}

\begin{note}
	Далее мы часто будем обозначать $\phi(g)$ как $\overline{g}$.
\end{note}

\begin{example}
	Гомоморфизмами соответствующих групп являются:
	\begin{enumerate}
		\item Любой изоморфизм групп, в частности, изоморфизм вида $\phi: G \to G$ --- \textit{автоморфизм} группы $G$
		\item $\phi: G \to H$, $\forall g \in G: \phi(g) = e$, где $G, H$ "--- произвольные группы
		\item Сопряжение при помощи $x \in G$, где $G$ "--- произвольная группа, поскольку $\forall g_1, g_2 \hm\in G: (g_1g_2)^x = g_1^xg_2^x$ (более того, сопряжение "--- это автоморфизм, поскольку существует обратное отображение: $\forall g \in G: \phi^{-1}(g) = g^{x^{-1}}$)
		\item $\det: GL_n(F) \to F^*$, поскольку $\forall A, B \in GL_n(F): \det(AB) \hm= \det A \det B$
		\item $\sgn: S_n \to \Q^*$, поскольку $\forall \sigma, \tau \in S_n: \sgn(\sigma\tau) \hm= \sgn\sigma\sgn\tau$
		\item Отображение $\phi: \Z \to \Z_n$ такое, что $\forall a \in \Z: \phi(a) = a + n\Z$, поскольку $\forall a, b \in \Z: \phi(a + b) \hm= (a + b) + n\Z = \phi(a) + \phi(b)$
	\end{enumerate}
\end{example}

\begin{proposition}
	Пусть $G, H$ "--- группы, $\phi: G \to H$ "--- гомоморфизм. Тогда $\phi(e) = e$ и $\forall g \in G: \phi(g^{-1}) = (\phi(g))^{-1}$.
\end{proposition}

\begin{proof}~
	\begin{itemize}
		\item $\overline{e}^2 = \overline{e^2} = \overline{e} \Rightarrow \overline{e} = e$
		\item $\overline{g^{-1}}\overline{g} = \overline{g^{-1}g} = \overline{e} = e \Rightarrow \overline{g}^{-1} = \overline{g^{-1}}$
	\end{itemize}
\end{proof}

\begin{proposition}
	Пусть $G, H$ "--- группы, $\phi: G \to H$ "--- гомоморфизм, $K := \ke\phi$. Тогда $\forall g \in G: \phi^{-1}(\overline{g}) = gK = Kg$.
\end{proposition}

\begin{proof}
	Пусть $a \in G$. Тогда $a \in \phi^{-1}(\overline{g}) \Leftrightarrow \overline{a} = \overline{g} \Leftrightarrow e \hm= \overline{g}^{-1}\overline{a} = \overline{g^{-1}a} \Leftrightarrow g^{-1}a \hm\in \ke\phi = K \Leftrightarrow a \in gK$. Аналогично доказывается, что $a \in \phi^{-1}(\overline{g}) \Leftrightarrow a \in Kg$.
\end{proof}

\begin{corollary}
	$\phi$ "--- мономорфизм $\Leftrightarrow \ke\phi = \{e\}$.
\end{corollary}

\begin{proposition}
	Пусть $G, H$ "--- группы, $\phi: G \to H$ "--- гомоморфизм. Тогда:
	\begin{enumerate}
		\item $\im\phi \le H$
		\item $\ke\phi \normal G$
	\end{enumerate}
\end{proposition}

\begin{proof}~
	\begin{enumerate}
		\item Если $h_1, h_2 \in \im\phi$, то $h_1 = \overline{g_1}$, $h_2 = \overline{g_2}$, откуда $h_1h_2 = \overline{g_1g_2} \hm\in \im\phi$ и $h_1^{-1} = \overline{g^{-1}} \in \im\phi$
		\item Если $g_1, g_2 \in \ke\phi$, то $\overline{g_1} = \overline{g_2} = e$, откуда $\overline{g_1g_2} = e$ и $\overline{g_1^{-1}} = e$, и, более того, $\forall g \in G: gK = Kg = \phi^{-1}(g)$
	\end{enumerate}
\end{proof}

\begin{note}
	Если $G, H$ "--- группы, $\phi: G \to H$ "--- гомоморфизм и $G' \le G$, то $\phi|_{G'}: G' \hm\to H$ "--- тоже гомоморфизм, поэтому $\phi(G') \hm= \im\phi|_{G'} \le H$. С другой стороны, если $H' \le H$, то существует гомоморфизм $\psi = \id|_{H'}: H' \to H$ такой, что $H' = \im\psi$.
\end{note}

\begin{definition}
	Пусть $G$ "--- группа, $K \normal G$. Определим операцию на $G / K$ следующим образом: $\forall g_1, g_2 \in G: g_1K\cdot g_2K \hm= g_1(Kg_2)K = g_1(g_2K)K = g_1g_2K$.
\end{definition}

\begin{proposition}
	 Пусть $G$ "--- группа, $K \normal G$. Тогда $(G / K, \cdot)$ "--- группа. Более того, отображение $\pi: G \to G / K$, $\forall g \in G: \pi(g) \hm= gK$, является эпиморфизмом.
\end{proposition}

\begin{proof}
	Мы уже показали, что операция умножения на $G / K$ определена корректно. Проверим непосредственно, что множество $G / K$ является группой:
	\begin{itemize}
		\item (Ассоциативность) $\forall g_1K, g_2K, g_3K \in G / K: (g_1Kg_2K)g_3K \hm= (g_1g_2g_3)K \hm= g_1K(g_2Kg_3K)$
		\item (Нейтральный элемент) $\exists K \in G / K: \forall gK \in G / K: (gK)K \hm= K(gK) = gK$
		\item (Обратный элемент) $\forall gK \in G / K: \exists (gK)^{-1} = g^{-1}K \in G / K: (gK)(gK)^{-1} \hm= (gK)^{-1}(gK) = K$
	\end{itemize}

	Отображение $\pi$ "--- это гомоморфизм по определению, и его сюръективность очевидна: $\forall gK \in K: \exists g \in G: \pi(g) = gK$.
\end{proof}

\begin{note}
	$g \in \ke\pi \Leftrightarrow \pi(g) = gK = K \Leftrightarrow g \in K$, поэтому $\ke\pi = K$. Значит, любая нормальная подгруппа является ядром некоторого гомоморфизма, точно так же, как любая подгруппа является образом некоторого гомоморфизма.
\end{note}

\begin{definition}
	Пусть $G$ "--- группа, $K \normal G$. Группа $G / K$ называется \textit{факторгруппой} $G$ по $K$.
\end{definition}

\begin{theorem}[Основная теорема о гомоморфизме]~
	\begin{enumerate}
		\item Пусть $G$ "--- группа, $K \normal G$. Тогда $\exists \pi: G \to K$ "--- эпиморфизм такой, что $\ke\phi \hm= K$.
		\item Пусть $G, H$ "--- группы, $\phi: G \to H$ "--- гомоморфизм. Тогда $K \hm{:=} \ke\phi \normal G$, и, более того, $\im\phi \cong G / K$.
	\end{enumerate}
\end{theorem}

\begin{proof}
	Большая часть теоремы уже была доказана выше, остается доказать лишь последнее утверждение. Изобразим его на коммутативной диаграмме:
	\[
		\begin{tikzcd}[row sep = huge]
			G \arrow{rr}{\phi} \arrow[swap]{dr}{\pi} && \im\phi \le H \arrow[dashrightarrow, swap, xshift = -4pt]{dl}{\psi}\\
			& G / K \arrow[dashrightarrow, swap, xshift = 4pt]{ur}{\Theta} &
		\end{tikzcd}
	\]
	
	Нам требуется предъявить такой изоморфизм $\Theta: G / K \to \im\phi$, что $\Theta \circ \pi = \phi$. Построим $\psi := \Theta^{-1}$: $\forall \overline{g} \in \im\phi: \psi(\overline{g}) = \phi^{-1}(g) \hm= gK \in G / K$. Проверим, что это изоморфизм:
	\begin{itemize}
		\item (Гомоморфизм) $\forall \overline{g_1}, \overline{g_2} \in \im\phi: \psi(\overline{g_1}\overline{g_2}) = \psi(\overline{g_1g_2}) = g_1g_2K \hm= (g_1K)(g_2K) = \psi(\overline{g_1})\psi(\overline{g_2})$
		\item (Сюръективность) $\forall gK \in G / K: gK = \psi(\overline{g})$
		\item (Инъективность) Достаточно показать, что $\ke\psi = \{e\}$: $\forall \overline{g} \hm\in \im\phi: \psi(\overline{g}) = K \hm\Rightarrow gK = K \Rightarrow g \in K \Rightarrow \overline{g} = \overline{e}$
	\end{itemize}

	Теорема доказана, но убедимся непосредственно в том, что диаграмма коммутативна, то есть $\Theta \circ \pi = \phi$, или $\Theta^{-1} \circ \phi = \psi \circ \phi = \pi$: $\forall g \hm\in G: \psi(\phi(g)) = \phi(\overline{g}) = gK = \pi(g)$.
\end{proof}

\begin{note}
	Гомоморфный образ группы, будь во имя коммунизма изоморфен факторгруппе по ядру гомоморфизма!
\end{note}