\section{лекция 7}
\subsection{повтороние, что есть элиминация кванторов}
лекция началась с повторения материала лекции 6:
элиминация кванторов (переход к эквивалентной формуле, но уже не содержащей кванторов) и ее свойства.\\
Безкванторная формула это в сущности булева формула от атомарной, а булева формула представляется как ДНФ, но тогда $ \exists x (A \lor B) = \exists x A \lor \exists x B $, кроме этого  $\forall x A \sim \neg \exists x \neg A$.
Этими рассуждениями мы упростили задачу: достаточно элиминировать квантор существования, стоящий перед дизъюнктом (заменяя на каждом шаге по одному квантору, если квантор всеобщности, то заменить его на квантор сущетвования, двигаясь вверх к вершине разбора формулы).\\
\subsection{вернеся к алгебре Тарского}.
Сначала разберемся, что есть терм в алгебре Тарского $\av{\RR, <, = ,  +, \times, 0, 1}$, легко видеть, что это будет многочлен с целыми, неотрицательными числами (отрицательные числа мы можем ввести только с помощью кванторов).\\
Тогда любая атомарная формула, имет вид $h(x_1, ... , x_n) > 0$ (это можно добиться переносом всем многочленов в одну сторону) -- важный момент: коэффициенты в $h$ уже могут быть отрицательными, аналогично для предиката равенства, так же от этих двух видов формул легко брать отрицание.\\
Значит надо научиться элиминировать кванторы от формул вида:
\begin{equation}
    \exists x : \left\{  \begin{aligned} 
                            & f_1(x, y1, ... ,y_{k}) > 0 \\
                            & f_2(x, y1, ... ,y_{k}) > 0 \\
                            & \cdots \\ 
                            & f_m(x , y_1, ... , y_k) > 0 \\
                            & f_{m+1}(x, y_1, ... , y_k) = 0 \\
                            & \cdots \\
                            & f_{N} (x, y_1, ... , y_k) = 0
                        \end{aligned}  
    \right.
\end{equation}
