\section{Лекция 2}
\subsection{Вспоминаем прошлое занятие}
\noindent В прошлой лекции было дано определение формальной системы. Напомним, как она устроена:
\begin{itemize}
    \item В формальной системе рассматриваются манипуляциии с формулами. Формула задается индуктивно. Что же такое формула? 
    \begin{equation*}
    \text{Формула} =
    \begin{cases}
    x_i\\
    (A \rar B)\\
    (\neg A)
    \end{cases}
    \end{equation*}
    \item Исчисление высказываний содержит бесчисленное множество аксиом. Они задаются тремя схемами:
\begin{enumerate}
\item  $A \rar (B \rar A)$
\item $A \rar (B \rar C) \rar \br{\br{A \rar B} \rar \br{A \rar C}}$
\item $\br{\neg B \rar \neg A} \rar \br{\br{\neg B \rar A} \to B}$
\end{enumerate}
\item Правило вывода - \textbf{Modus ponens}
\begin{center}
$\frac{A, A \rar B}{B}$
\end{center}
Tо есть: если $A$ и $A\to B$ — истинные высказывания, то $B$ также истинно.\\
\textbf{Вывод} - это последовательность формул, в которой каждая формула обоснована.\\
\textbf{Обоснование} - либо формула является аксиомой, либо получается из предыдущей по Modus ponens.
\end{itemize}

\newpage

\subsection{Каково множество выводимых формул?}
\noindent Мы получаем множество выводимых формул. Косвенным образом задаём множество формул, которые выводимы. Все \textit{аксиомы} выводимы, тк можем написать вывод длины $1$, содержащий данную аксиому. Но также выводимы и другие формулы, тк применение \textit{Modus ponens} даёт новые формулы.\\
Перейдем к следующему вопросу:
\begin{center}
    Почему не все формулы выводимы?
\end{center}
Всопмним \textit{теорему корректности}:
\begin{equation}
 \vdash F \Rar  F - \textit{тавтология}
\end{equation}
Теперь мы можем рассмотреть $A$ и $\neg A$. Не более одной из них является тавтологией. Значит, хотя бы одна из этих двух формул невыводима.

\subsection{Поговорим о противоречии.}
\noindent 
\begin{definition} 
$A$ и $\neg A$ - \textit{противоречие}\\
\end{definition}
\begin{lemma} 
$A,\neg A \vdash B$
\end{lemma}
\begin{lemma}
$B\vdash A \rar B $
\end{lemma}
\beginproof\\
Произведем построение вывода:
    \hspace*{18mm} 1. $B$ (гипотеза)\\
    \hspace*{80mm} 2. $B\rar (A \rar B) $ (аксиома)\\
    \hspace*{80mm} 3. $A \rar B$ (из $1$ и $2$ с помощью MP)\\
\textbf{Док-во леммы 2.1.}\\
Произведем набросок вывода:
\hspace*{18mm} 1. $A$ (гипотеза)\\
\hspace*{80mm} 2. $\neg A$ (гипотеза)\\
\hspace*{80mm} 3. $\neg B \rar \neg A$ (Лемма $2.2$)\\
\hspace*{80mm} 4. $\neg B \rar A$ (Лемма $2.2$)\\
\hspace*{80mm} 5. $(\neg B \rar \neg A) \rar ((\neg B \rar A) \rar B)$ (аксиома $3$)\\
\hspace*{80mm} 6. $(\neg B \rar A) \rar B$ (из $3$ и $5$ с помощью MP)\\
\hspace*{80mm} 7. $B$ (из $4$ и $6$ с помощью MP)

\newpage

\subsection{Теорема полноты исчислений высказываний.}
\noindent 
\begin{theorem}
$F -$ \textit{тавтология} $\Rar$ \hspace{2mm} $\vdash F$
\end{theorem}
\textbf{Для доказательства нужны две вещи:}
\begin{enumerate}
    \item \textit{факты о выводимости}
    \item \textit{связи семантики (тавтологии) и синтаксиса (выводимость)}
\end{enumerate}
\textbf{Но к этой теореме мы вернемся на следующей лекции.}

\subsection{Теорема дедукции.}
\noindent 
\begin{theorem} \label{th:deduction}
\textit{Г,} $A \vdash B \Leftrightarrow$ \textit{Г} $ \vdash A \rar B$
\end{theorem}
\beginproof\\

($\leftarrow$)
\begin{equation*}
\textit{Г}  \vdash A \rar B \Rar 
    \begin{cases}
    F_1\\
    \cdot\\
    \cdot\\
    \cdot\\
    F_n = A \rar B
    \end{cases}
    \end{equation*}
Будем считать, что последняя формула в выводе (без ограничения общности) является $A \rar B$.\\ 


Теперь надо доказать, что \textit{Г,} $A \vdash B$: 
\begin{equation*}
\textit{Г,} A \vdash B \Rar 
    \begin{cases}
    F_1\\
    \cdot\\
    \cdot\\
    \cdot\\
    F_n = A \rar B\\
    A\\
    B\\
    \end{cases}
    \end{equation*}
Причем обоснование первых $n$ формул не изменяется. После этого добавляется новая гипотеза $A$, и, применив $MP$, получаю $B$.\\

($\rightarrow$) Воспользуемся индукцией по длине вывода:\\
Предположим, что у нас есть вывод формулы $B$ из гипатез, принадлежащих Г, и дополнительной гипотезы $A$.
Заметим, что последняя формула в данном выводе будет $B$ (из вышеупомянутого). \\
Совершим следующую манипуляцию - перепишем всю последовательность формул, ставя каждую формулу в заключение импликации, а в послыку, ставя формулу $A$. Так мы получим последовательность формул, не являющуюся выводом:
\begin{equation*}
    \begin{cases}
    A \rar F_1\\
    \cdot\\
    \cdot\\
    \cdot\\
    A \rar F_n = A \rar B\\
    \end{cases}
    \end{equation*}
Но мы будем доказывать, что эту последовательность можно расширить, добавляя в нее еще какие то формулы так, чтобы у нас получился вывод из мн-ва формул Г.\\



Далее рассуждение будет индуктивным. Если до какого-то места у нас все выполняется, то следующая за ней формула пока что не обоснована:
\begin{center}
   $ \textit{Г}  \vdash A \rar B$\\
   $ \cdot \cdot \cdot$\\
    $A \rar F_k$\\
    \hspace{70mm}$-------\rightarrow$ \textit{Вставка для обоснования}\\
   $ A \rar F_{k+1}$\\
   $ \cdot \cdot  \cdot$
\end{center}
Варианты вставки (зависит от обоснования в первом выводе):\\
 \textbf{1. Если $F_{k+1}$ - аксиома или гипотеза $\Rightarrow$}
    $A \rar F_k$ (обосновано по инд. предположению)\\
    \hspace*{80mm} Добавляем $ F_{k+1}$ (аксиома или гипотеза)\\
    \hspace*{80mm} Добавляем $ F_{k+1} \rar (A\rar F_{k+1})$ (аксиома 1)\\
    \hspace*{80mm} $A\rar F_{k+1}$ (MP из добавленных)\\
\newpage
   \textbf{2. Если $F_{k+1}$ получена с помщью MP} $\Rightarrow$
    Рассмотрим исходный вывод:\\
    \hspace*{100mm} $F_i$\\
    \hspace*{100mm}$\cdot \cdot \cdot$\\
    \hspace*{90mm} $F_j =F_i \rar F_{k+1}$\\
    \hspace*{100mm}$\cdot \cdot \cdot$\\
    \hspace*{100mm} $F_{k+1}$ (MP)\\
    Уже построенная (не понял слово) нового вывода:
    \begin{center}
        $A \rar F_i$\\
        $\cdot \cdot \cdot$\\
        $A \rar (F_i \rar F_{k+1})$\\
        $\cdot \cdot \cdot$\\
        $A \rar F_k$\\
        Добавляем $(A \rar (F_i \rar F_{k+1}))\rar ((A \rar F_i) \rar (A \rar F_{k+1}))$ (аксиома 2)\\
        Добавляем $(A \rar F_i) \rar (A \rar F_{k+1})$ (MP из добавленных)\\
        $(A \rar F_{k+1})$ (MP)
    \end{center}
Добавление строчек можно считать шагом индукции.



Но при рассуждениях мы не учли еще один случай.\\
 \textbf{3. Особый случай $F_{k+1} = A$}\\
 \begin{lemma}
 $\vdash A \rar A$
 \end{lemma}
 \conclude
 \hspace*{18mm} 1. $A \rar (A\rar A)$ (аксиома 1) \\
\hspace*{35mm} 2. $A \rar ((A\rar A) \rar A)$ (аксиома 1)\\
\hspace*{35mm} 3. $(A \rar ((A\rar A) \rar A))\rar ((A \rar (A\rar A))\rar (A\rar A))$ (аксиома 2) \\
\hspace*{35mm} 4. $(A\rar (A\rar A))\rar (A\rar A)$ (из $2$ и $3$ с помощью MP)\\
\hspace*{35mm} 5. $A\rar A$ (из $1$ и $4$ с помощью MP)\\
Вернемся к доказательству:
\begin{center}
    $A\rar F_k$ (уже обосновано)\\
    \hspace{40mm} $-----------\rar$ пользуемся леммой 2.3.\\
    $A\rar F_{k+1} = A\rar A$
\end{center}
Следовательно, теорема дедукции доказана.
\qed
\subsection{Связь семантики и синтаксиса.}
\noindent Для того, чтобы сформулировать данную связь введем несколько новых определений:\\
\begin{definition}
$\alpha$-версия формулы ($\alpha$ - набор значений переменных):
\begin{center}
\begin{equation}
    F^{\alpha} = 
    \begin{cases}
    F, \hspace{5mm}  \text{если} \hspace{5mm} F(\alpha) = 1\\
    \neg F, \hspace{5mm} \text{если} \hspace{5mm} F(\alpha) = 0
    \end{cases}
    \end{equation}
\end{center}
Грубо говоря, $\alpha$-версия формулы - это либо $F$, либо $\neg F$\\
\textbf{Важный частный случай:} пусть есть переменная $x$, тогда $x^{\alpha}$ - \textit{литерал}:
\begin{center}
\begin{equation}
    x^{\alpha} = 
    \begin{cases}
    x, \hspace{5mm}  \text{если} \hspace{5mm} x(\alpha) = 1\\
    \neg x, \hspace{5mm} \text{если} \hspace{5mm} x(\alpha) = 0
    \end{cases}
    \end{equation}
\end{center}
\end{definition}

\subsection{Лемма Кальмара.}
\begin{theorem} \label{th:Kalmar} \tit{Лемма Кальмара}
\noindent Пусть в $F$ входят переменные $x_1,x_2,...,x_n$.\\ Тогда для любого $\alpha :  x_1^{\alpha},x_2^{\alpha},...,x_n^{\alpha} \vdash F^{\alpha} $ (контрабандой протащили семантику в нашу синтаксическую теорию)
\end{theorem}
\beginproof Проведем индукцию по разбору формулы (индукция по длине формулы).\\
\textbf{База:} $F = x_i \Rightarrow x_1^{\alpha},x_2^{\alpha},...,x_n^{\alpha} \vdash x_i^{\alpha} $\\
\textbf{Переход:} пусть для формул длины $k$ лемма доказана.\\
\vspace{3mm}
Теперь докажем для формул большей длины:\\
\textbf{1.} $F = \neg A \hspace{3mm} A(\alpha) = 0 \hspace{3mm} A^{\alpha} = \neg A \Rightarrow F(\alpha) = 1 \hspace{3mm} F^{\alpha} = F = \neg A$\\
Индуктивное предположение: $\{x_i^{\alpha}\}\vdash A^{\alpha} = \neg A$\\
\vspace{3mm}
Нужно доказать: $\{x_i^{\alpha}\}\vdash F^{\alpha} = \neg A$.
Но мы это доказали ранее.\\
\textbf{2.} $F = \neg A \hspace{3mm} A(\alpha) = 1 \hspace{3mm} A^{\alpha} = A \Rightarrow F(\alpha) = 0 \hspace{3mm} F^{\alpha} = \neg \neg A$\\
Индуктивное предположение: $\{x_i^{\alpha}\}\vdash A^{\alpha} =  A$\\
Нужно доказать: $\{x_i^{\alpha}\}\vdash F^{\alpha} =\neg \neg A$\\
При этом, тк вывод синтаксический, то ссылаться на то, что двойное отрицание $A$ это не то же самое, что формула $A$ (тк это семантический факт).\\
\newpage
Для продолжения поспользуемся леммой:
\begin{lemma}
$A \vdash \neg \neg A$ (доказательство будет приведено позже)
\end{lemma}
Получается, после индуктивного предположения мы пользуемся леммой 2.4. и приходим к искомому результату.\\
\newline
\textbf{3.} $F = (A \rar B) \hspace{3mm} A(\alpha) = 0 \hspace{3mm} A^{\alpha} = \neg A \Rightarrow F(\alpha) = 1 \hspace{3mm} F^{\alpha} = A \rar B$\\
Индуктивное предположение: $\{x_i^{\alpha}\}\vdash A^{\alpha} =  \neg A$\\
Нужно доказать: $\{x_i^{\alpha}\}\vdash F^{\alpha} = A \rar B$\\
Схема рассуждения такая же, как и в предыдущем случае. \\
Нам достаточно доказать: $\neg A \vdash A \rar B$\\
Воспользовавшись теоремой дедукции поймем, что 
\begin{center}
    $\neg A \vdash A \rar B \Leftrightarrow \neg A , A \vdash B$ (Лемма 2.1, а ее мы уже доказывали)
\end{center}
\textbf{4.} $F = (A \rar B) \hspace{3mm} B(\alpha) = 1 \hspace{3mm} B^{\alpha} = B \Rightarrow F(\alpha) = 1 \hspace{3mm} F^{\alpha} = A \rar B$\\
Индуктивное предположение: $\{x_i^{\alpha}\}\vdash B^{\alpha} = B$\\
Нужно доказать: $\{x_i^{\alpha}\}\vdash F^{\alpha} = A \rar B$\\
Нам достаточно доказать: $B \vdash A \rar B$ (Доказали в лемме 2.2.)\\

\textbf{5.} $F = (A \rar B) \hspace{3mm} A(\alpha) = 1 \hspace{3mm} B(\alpha) = 0 \hspace{3mm} A^{\alpha} = A \hspace{3mm} B^{\alpha} = \neg B \Rightarrow F(\alpha) = 0 \hspace{3mm} F^{\alpha} = \neg (A \rar B)$\\
Индуктивное предположение: $\{x_i^{\alpha}\}\vdash A \hspace{3mm} \{x_i^{\alpha}\}\vdash \neg B  $\\
Нужно доказать: $\{x_i^{\alpha}\}\vdash \neg (A\rar B)$\\
Для продолжения поспользуемся леммой:
\begin{lemma}
$A, \neg B \vdash \neg (A\rar B)$ (доказательство будет приведено позже)
\end{lemma}
Получается, после индуктивного предположения мы пользуемся леммой 2.5. и приходим к искомому результату.\\
\textbf{Все случаи разобраны, а значит - лемма Кальмара доказана}
\qed

Доказательство лемм 2.4. и 2.5. требует дополнительные леммы:
\begin{lemma}
$\neg \neg A\vdash A$
\end{lemma}
\begin{lemma}
$A\rar B \vdash \neg B \rar \neg A$ 
\end{lemma}
\begin{lemma}
$A\rar B, B\rar C \vdash A \rar C$ \\
\end{lemma}
\textbf{Доказательство леммы 2.4.} $(A\vdash \neg \neg A)$\\

\begin{enumerate}
\item Воспользуемся третьей схемой аксиом
  \hspace{5mm} $(\neg \neg \neg A\rar \neg A) \rar ((\neg \neg \neg A\rar  A) \rar \neg \neg A)$
\item Причем $(\neg \neg \neg A\rar \neg A)$ получили по лемме 2.6.
\item Причем $(\neg \neg \neg A\rar  A)$ получили по лемме 2.2.
\item А затем 2 раза применить $MP$
\item Доказательство закончено, тк получим исходную формулу.
\end{enumerate}
\qed

\textbf{Доказательство леммы 2.5.} $A, \neg B \vdash \neg (A\rar B)$\\
\begin{center}
По теореме дедукции $A, \neg A \rar B \vdash B$ (формула выводится из MP и двух гипотез) \\
$\Updownarrow$\\
$ A \vdash (A\rar B)\rar B$ (перенесли гипотезу в формулу в качестве посылки)\\
$\Downarrow$ \\
По частному случаю леммы 2.7. \\
$\Downarrow$\\
$A \vdash \neg B \rar \neg (A\rar B)$\\
$\Updownarrow$\\
По теореме дедукции\\
$\Updownarrow$\\
$A, \neg B \vdash \neg (A\rar B)$\\ (посылку импликации в той формуле, которую мы выводим, переместить в гипотезы)
\end{center}
\qed\\
\newpage
\textbf{Доказательство леммы 2.8.:} 
\begin{center}
$A\rar B, B\rar C \vdash A \rar C$\\
$\Updownarrow$\\
По теореме дедукции\\
$\Updownarrow$\\
$A, A\rar B, B\rar C \vdash C$\\
$\Downarrow$\\
\hspace{32mm} 1. $A$ (гипотеза)\\
\hspace{42mm} 2. $A\rar B$ (гипотеза) \\
\hspace{20mm} вывод: \hspace{26mm} 3. $B$ (из $1$ и $2$ с помощью MP)\\
\hspace{43mm} 4. $B\rar C$ (гипотеза) \\
\hspace{62mm} 5. $C$ (из $3$ и $4$ с помощью MP) \\
\end{center}
\qed

\textbf{Доказательство леммы 2.6.} ($\neg \neg A\vdash A$)\\
Док-во:\\
\begin{enumerate}
\item Воспользуемся третьей схемой аксиом
  \hspace{5mm} $(\neg A\rar \neg \neg A) \rar ((\neg A\rar \neg  A) \rar A)$
\item Причем $(\neg A\rar \neg \neg A)$ получили по лемме 2.2.
\item Причем $(\neg A\rar \neg  A)$ получили по лемме 2.3.
\item А затем 2 раза применить $MP$
\item Доказательство закончено, тк получим исходную формулу.
\end{enumerate}
\qed\\
\newpage
\textbf{Доказательство леммы 2.7.:} 
\begin{center}
$A\rar B \vdash \neg B \rar \neg A$\\
$\Updownarrow$\\
По теореме дедукции\\
$\Updownarrow$\\
$\neg B, A\rar B \vdash \neg A$\\
(посылки сделать гипотезами)\\
$\Downarrow$\\
Эскиз вывода:\\
1. $(\neg \neg A \rar \neg B)\rar ((\neg \neg A \rar B)\rar \neg A)$  (3-я схема аксиом)\\
\hspace{-47mm} 2. $\neg \neg A \rar \neg B$ (лемма 2.2. из $\neg B$) \\
\hspace{-87mm} 3. $\neg \neg A \rar A$\\ 
(применить транзитивность и  заметить,что равносильно по теореме дедукции лемме 2.6.)\\
\hspace{-71mm} 4. $A\rar B$ (гипотеза) \\
\hspace{-62mm} 5. $\neg \neg A \rar  B$ (лемма 2.8.) \\
\hspace{-21mm} 6. $(\neg \neg A \rar  B)\rar \neg A$ (из $1$ и $2$ с помощью MP)\\
\hspace{-49mm} 7. $\neg A$ (из $5$ и $6$ с помощью MP)
\end{center}
\qed
