\subsection{Основные свойства интеграла Лебега.}
\begin{lemma}[Признак суммируемости]
Если $f(x)$ суммируема на измеримом по Лебегу множестве $E \subset \R^n$ конечной меры, а $F$ измерима на $E$ и $|F(x)| \leqslant f(x)\ \forall x \in E$, то $F$ суммируема на $E$.
\end{lemma}
\begin{proof}
Заметим $\forall P: E = \bigsqcup\limits_{k=0}^{\infty}E_k\ \ \ \ \ \mathcal{U}(P, |F|) \leqslant \mathcal{U}(P, f)$

$(\forall \varepsilon > 0) \exists P\ \ \ \ \  \mathcal{U}(P,f) \leqslant \int\limits_E f(x)d\mu(x)+\varepsilon$

$\exists P: \mathcal{U}(P, |F|) < +\infty \Rightarrow \inf\limits_P \mathcal{U}(P, |F|) = \int\limits_P |F(x)|d\mu(x)<+\infty$

По определению $F$ - суммируема $\Leftrightarrow F^+ \text{ и } F^-$ суммируемы. 

Видно, что $0\leqslant F^\pm(x) \leqslant |F(x)| \Rightarrow \mathcal{U} (P,F^\pm) \leqslant \mathcal{U}(P,|F|)$.
\end{proof}


\begin{theorem}[Линейность и монотонность интеграла Лебега]

\begin{enumerate}
\item(Линейность) Если
\end{enumerate}
\end{theorem}