\section{логика первого порядка}
Общая структура записи в логике первого порядка: $R(x_1 , x_2, x_3 ... , x_n)$, где $R$ -- некоторое отношение (делимость, если числа целые, больше/меньше и. тд.). То есть \tit{отношение} в нашей терминологии это некоторое подмножество. \\
Таким образом можно исследовать отношения как подмножества, применяя к ним теоретикомножественные операции, соответствующие логическим операциям ($\cup, \lor$) , ($\cap , \land$) и другие. \\
Рассмотрим один пример записи:
\begin{example}
$x^2 + 1 =  0$ -- правая часть понятна: стоит только одна константа и отношение равенства. Слева стоит уже функциия, зависящая, в данном случае, от ДВУХ переменных : константа <<1>> и переменная <<x>>. Стоит заметить, что с точки зрения формальной конструкции это уравнение имеет вид: $Eq (a( m(x, x), 1), 0)$ -- функция $m$ -- возведение в квадрат, $a$ -- сложение, легко видеть, что формальная запись сложнее и более длинная, поэтому иногда мы не будем переходить к формальной записи, а использовать символ <<>>
\end{example}
Кроме этого, для записи высказывание нам понадобятся \tit{кванторы}. Дадим формальное опредление для квантора всеобщности ($\forall$) и существования ($\exists$):
\begin{equation}
    \forall x R(x, y_1, y_2, ... , y_n) = R'(y_1 , y_2, ... , y_n) =  \left\{ \begin{aligned} 
                       & True , \text{ при любом } x : (x, y_1,y_2,... , y_n) \in R \\
                        & False \text { иначе}
             \end{aligned} \right.
\end{equation}
\begin{equation}
    \exists x R(x, y_1, y_2, ... , y_n) = R'(y_1 , y_2, ... , y_n) =  \left\{ \begin{aligned} 
                       & True , \text{ существует } x : (x, y_1,y_2,... , y_n) \in R \\
                        & False \text { иначе}
             \end{aligned} \right.
\end{equation}

Мы часто будем сталкиваться с \tit{задачей выразимости} : пусть у нас заданы некоторые отнощения, какие еще отношения мы можем вывести (выразить) через данные

\subsection{ пример: алгебра Тарского}
в качестве примера рассмотрим алгебру Тарского: множество вещественых чисел, операции +, -, отношения равенства, больше/ меньше, больше или равено / меньше или равно и так же две константы: $0, 1$\\
вопрос: можно ли выразить предикат <<$x$ -- целлое>> в алгебре Тарского ?\\
Давайте сначала <<определим>> натуральные числа в алгебре Тарского: <<$x$ -- натуральное>> $=$ <<$x=1 \lor \exists y \in N : x = y+1$ >>\\
Это опредление содержит рекуррентную зависоимость, рекусивные определения могут вызвать некоторые затрудения (на пример, мы можем записать <<$R(x) =(def) R(x)$>> -- так же руккурентное  определение, но как видно, оно бесполезно), значит нам необходимо точное определение формул первого порядка\\
\subsection{формулы первого порядка}
Сначада зададим алфавит: неограниченное кол-во переменных $x_i$, функциональные символы $f_{i}^{k}$, константы $c_i$ , предикатные символы $A_{i}^{k}$ и специальные символы: $\to, \neg , \lor \, \land , \sim ,( , ) , \forall, \exists$ \\
Теперь дадим вспомогательное определение: \tit{терм}.
\begin{definition} \tit{терм} \\
$x_i , c_i$ -- термы ($t$)\\
$f^{k}_{j} (t_1, t_2 , ... ,t_k)$ , где $t_i$ -- терм, -- терм  \\
других термов нет
\end{definition}
теперь определим атомарную формулу;
\begin{definition} \tit{атомарная формула} \\
атомарная формула имеет вид: $A^K(t_1, ... , t_k)$ -- $A^k$ -- предикатный символ, $t_i$ -- терм
\end{definition}
\begin{definition} \tit{формула} \\
1) элементарная формула -- формула \\
2) если $A, B $ -- формулы , то
\begin{equation}
    \begin{aligned}
        & \neg A \\
        & A \to B \\
        & A \land B \\
        & A \lor B \\
        & A \sim B \\
        & \forall x A, ~ x  \text{ -- переменная} \\
        & \exists x A, ~ x  \text{ -- переменная} \\
    \end{aligned}
\end{equation}
так же формулы \\
3) других формул нет
\end{definition}
Аналогично, как и для ИВ, можно построить дерево разбора формул (сложность возникатся в том, что теперь могут быть поддервья атомарных формул)\\

\subsection{оценка формулы}
\begin{definition} \tit{модель} \\
модель -- некоторое множество + сигнатура: набор предикатов, функциональных символов и констант: $<M,  A_i^k , f_i^k , c_i>$, причем $M$ непусто, а так же задана \tit{модель} : $[A_i] \subseteq M^k , [f^k_i]: M^k \to M , [c_i] \in M$
\end{definition}
Сначала задается оценка переменных: $[x]_{\pi} \in M$, затем оценка каждого терма: $[f^k(t_1, t_2 , ... t_k)]_{\pi} = [f^k]([t_1], [t_2] , ..., [t_k])$ и соответственно оценка атомарной формулы:
\begin{equation}
    [A^k(t_1, ... , t_k)]_{\pi} = \left\{ \begin{aligned}
            &1 , ([t_1], ... , [t_k]) \in [A^k]_{\pi} \\
            &0 , \text { иначе}
    \end{aligned} \right.
\end{equation}

Теперь рассмотрим, если формула стоит под квантором всеобщности (квантор существования аналогично): $[\forall x A]_{\pi}$ истинна тогда и только тогда, когда истинны всевозможные оценки $[A]_{\pi '}$, для всех $\pi '$, отличающихся только значениями $x$\\

\begin{definition} \tit{связанное вхождение} \\
вхождение переменной называется \tit{связянным}, если она присутствует в области действия квантора, то есть $\forall x A(..., y , ...)$ -- аналогично длвя квантора существования
\end{definition}
Другие же вхождения переменных называются \tit{свободными}
 
\begin{definition}\tit{параметр} \\
\tit{параметр} -- переменная, имеющая свободное вхождение
\end{definition} 
\begin{lemma}
 значение формулы зависит только от  оценок параметра
\end{lemma}
\beginproof
индукция по длине формул. \\
\tit{база} случай атомарных формул -- тривиален, все вхождения свободны, т.к. нет квантора \\
при применении булевой связки, опять таки все вхождения свободны, значит лемма выполнятеся \\
теперь рассмотрим  навешенный квантор, по определению значения $\forall xA , \exists xA$ не зависит от $x$. если $x$ был параметром, то после навешания квантора $x$ перестанет быть таковым, но новые параметры в формуле не образуются, очевидно, значит опять таки, значение формулы зависит только от параметров. \\
После четкого опредления ввернемся к примеру алгебры Тарского: видно, что наша <<формула>> ошибочная, т.к. она содержит рекурсию, которая по определению запрещена (в дальнейшем мы докажем невыразимость целостности числа в алгебре Тарского)

\begin{definition} \tit{общезначимые формулы}
формулы, которые истинны в любой модели и при любой оценки переменных
\end{definition}

\begin{definition} \tit{равносильные формулы}
$A$ и $B$ равносильны тогда и только тогда, когда $A \sim B$ -- общезначимая
\end{definition}

\begin{example}
пусть $F$ -- тавтология, в терминах ИВ, тогда $F(A_1, ... , A_n)$  ($A_i$ -- формула первого порядка) -- общезначимая
\end{example}

\begin{example}
$\forall xA \sim \neg \exists x \neg A$
\end{example}

\begin{lemma} \label{lemma replacement} \tit{лемма о замене}
 пусть $A \sim B$. Рассмотрим произвольную формула $F$, содержащую $A$ как подформулу. Заменим все вхождения $A$ на $B$: $F' = F_{A \leftarrow B}$
\end{lemma}
\beginproof следует из правил оценки и дерева разбора

\begin{remark}
$A(x)$ и $A(y)$ не равносильны !
\end{remark}
Действительно, пусть $A$ -- унарный придикат, носитель модели $\figbr{0, 1}$ : $A(0) = false, A(1) \hm{=} true$. рассмотрим  значение переменных $ (x, y) = (0, 1)$. $A(x) = 0, A(y) = 1$ --на одной и тоже оценки переменных, разные значения формул -- противоречие 
