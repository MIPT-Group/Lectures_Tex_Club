\section{Полная приводимость представлений}

\begin{theorem}
	Пусть $G$ "--- конечная группа, и $\cha{K} \nmid |G|$. Тогда любое представление группы $G$ в пространстве $V$ вполне приводимо.
\end{theorem}

\begin{proof}
	Пусть $U \le V$ "--- $G$-инвариантное подпространство, $\pi : V \to U$ "--- проектор на $U$. Определим оператор $\pi_0 \in \mc L(V)$ следующим образом:
	\[\pi_0 := \frac1{|G|}\sum_{g \in G}(g\circ\pi\circ g^{-1})\]
	
	Сначала проверим, что $\pi_0$ "--- тоже проектор на $U$. С одной стороны, очевидно, что $\im\pi_0 \subset U$. С другой стороны, для произвольного вектора $u \in U$ выполнены следующие равенства:
	\[\pi_0(u) = \frac1{|G|}\sum_{g \in G}g(\pi(g^{-1}u)) = \pi_0(u) = \frac1{|G|}\sum_{g \in G}g(g^{-1}u) = \pi_0(u) = \frac1{|G|}\sum_{g \in G}u = u\]
	
	Теперь проверим, что $\pi_0$ коммутирует с каждым элементом $h \in G$:
	\[h\circ\pi_0\circ h^{-1} = \frac1{|G|}\sum_{g \in G}((hg)\circ\pi\circ (hg)^{-1}) = \frac1{|G|}\sum_{\widetilde g \in G}(\widetilde g\circ\pi\circ \widetilde g^{-1}) = \pi_0\]
	
	Таким образом, по уже доказанному критерию, любое $G$-инвариантное пространство в $V$ имеет инвариантное прямое дополнение, поэтому $V$ вполне приводимо.
\end{proof}

\begin{note}
	Если $\cha{K} \mid |G|$, то теорема уже необязательно верна. Пусть $p := \cha{K}$. Рассмотрим тавтологическое представление следующей группы в пространстве $K^2$:
	\[G = \left\{\begin{pmatrix}
		1&a\\
		0&1
	\end{pmatrix} : a \in \F_p\right\} \le \GL(K^2)\]

	Тогда $|G| = p$, и легко убедиться, что представление имеет единственное инвариантное подпространство $\gl(1, 0)^T\gr$.
\end{note}

\begin{theorem}
	Пусть $G$ "--- конечная группа с представлением в $V$, и $\cha{K} \nmid |G|$. Определим оператор $S \in \mc L(V)$ следующим образом:
	\[S := \frac1{|G|}\sum_{g \in G}g\]
	
	Тогда $S$ "--- $G$-эквивариантный проектор на $V^G := \{v \in V: \forall g \in G: gv=v\}$.
\end{theorem}

\begin{proof}
	Достаточно заметить, что для любого $h \in G$ выполнено $S = h \circ S = S \circ h$, откуда $\im{S} \subset V^G$ и $S$ "--- $G$-эквивариантный. Кроме того, очевидно, что $S|_{V^G} = \id$.
\end{proof}

\begin{corollary}
	Пусть $G$ "--- конечная группа с представлением в $V$, и $\cha{K} \nmid |G|$. Тогда:
	\[\frac1{|G|}\sum_{g \in G}\tr{g} = \dim{V^G}\]
\end{corollary}

\begin{proof}
	Возьмем след обеих частей равенства, определяющего оператор $S$, и заметим, что $\tr{S} = \dim{V^G}$, поскольку $S$ "--- проектор на $\dim{V^G}$.
\end{proof}

\begin{theorem}
	Пусть $\phi : V_1 \to V_2$ "--- нетривиальный гомоморфизм неприводимых представлений группы $G$ в пространствах $V_1$ и $V_2$. Тогда $\phi$ является изоморфизмом представлений.
\end{theorem}

\begin{proof}
	$\ke\phi \le V_1$, $\im\phi \le V_2$ "--- инвариантные относительно соответствующих представлений подпространства. Тогда, в силу неприводимости и нетривиальности, $\ke\phi = \{0\}$ и $\im\phi = V_2$. Значит, $\phi$ инъективен и сюръективен.
\end{proof}

\begin{theorem}[лемма Шура]
	Пусть поле $K$ алгебраически замкнуто, представление группы $G$ в пространстве $V$ неприводимо, и $\phi : V \to V$ "--- эндоморфизм представления. Тогда $\phi$ "--- скалярный оператор.
\end{theorem}

\begin{proof}
	В силу алгебраической замкнутости, у $\phi$ существует собственный вектор $v \in V$ с собственным значением $\lambda \in K$. Тогда оператор $\phi - \lambda E \in \mc L(V)$ "--- тоже эндоморфизм, причем вырожденный, тогда, в силу неприводимости, $\phi - \lambda E = 0 \ra \phi = \lambda E$.
\end{proof}

\begin{corollary}
	Пусть поле $K$ алгебраически замкнуто, $\mc L^G(V_1; V_2)$ "--- пространство гомоморфизмов неприводимых представлений группы $G$ в пространствах $V_1$ и $V_2$. Тогда:
	\[\dim{\mc L^G(V_1; V_2)} = \left\{\begin{aligned}
		0,\text{ если }V_1\not\cong_G V_2\\
		1,\text{ если }V_1\cong_G V_2
	\end{aligned}\right.\]
\end{corollary}

\begin{proof}
	Поскольку любой нетривиальный гомоморфизм неприводимых представлений является изоморфизмом, то $V_1 \cong_G V_2 \lra$ существует нетривиальный гомоморфизм $\phi : V_1 \to V_2$.
	\begin{itemize}
		\item Если нетривиального гомоморфизма не существует, то и $\dim{\mc L^G(V_1; V_2)} = 0$.
		\item Если существуют нетривиальные гомоморфизмы $\phi, \psi \in L^G(V_1; V_2)$, то оба они осуществляют изоморфизм $V_1 \gcong V_2$. Тогда $\phi^{-1}\circ\psi$ "--- автоморфизм представления группы $G$ в $V_1$, откуда по лемме Шура $\phi^{-1}\circ\psi = \lambda E \ra \psi = \lambda \phi$, где $\lambda \in K^*$. Значит, $\dim{\mc L^G(V_1; V_2)} = 1$.\qedhere
	\end{itemize}
\end{proof}

\begin{theorem}
	Пусть представление группы $G$ в пространстве $V$ вполне приводимо, и заданы два разложения пространства $V = \bigoplus_{i = 1}^nV_i = \bigoplus_{j = 1}^mW_j$. Тогда для произвольного класса изоморфности $C$ неприводимых подпространств в $V$ выполнено равенство:
	\[\bigoplus_{i \in \{1, \dotsc, n\} : V_i \in C}V_i = \bigoplus_{j \in \{1, \dotsc, m\}: W_j \in C}W_j\]
	
	Кроме того, $n = m$, и существует перестановка $\sigma \in S_n$ такая, что для каждого $i \in \{1, \dotsc, n\}$ выполнено $V_i \gcong W_{\sigma(i)}$.
\end{theorem}

\begin{proof}
	Зафиксируем индексы $i\in \{1, \dotsc, n\}$ и $j \in \{1, \dotsc, m\}$. Обозначим через $\pi_j$ проектор на подпространство $W_j$ и рассмотрим цепочку гомоморфизмов:
	\[V_i \xhookrightarrow[]{\id} V \xrightarrow{\pi_j} W_j\]
	
	Тогда либо $V_i \gcong W_j$, либо $\pi_j(V_i) = \{0\}$, что равносильно условию $V_i \cap W_j = \{0\}$. \pagebreak В силу произвольности выбора индексов $i$ и $j$, для каждого класса изоморфности $C$ получаем  следующее включение:
	
	\[\bigoplus_{i \in \{1, \dotsc, n\} : V_i \in C}V_i \subset \bigoplus_{j \in \{1, \dotsc, m\}: W_j \in C}W_j\]
	
	Обратное включение доказывается аналогично. Последняя часть теоремы следует из уже доказанного: суммы размерностей подпространств из $\{V_i\}_{i = 1}^n$ и $\{W_j\}_{j = 1}^m$, принадлежащих фиксированному классу изоморфности, совпадают, что и дает требуемое.
\end{proof}

\begin{definition}
	Пусть представление группы $G$ в пространстве $V$ вполне приводимо, $C$ "--- произвольный класс изоморфности неприводимых подпространств в $V$. \textit{Изотипной компонентой} класса $C$ называется подпространство $\oplus_{i \in \{1, \dotsc, n\} : V_i \in C}V_i$ для произвольного разложения $V = \bigoplus_{i = 1}^nV_i$.
\end{definition}

\begin{proposition}
	Пусть поле $K = \R$, $G$ "--- абелева, и представление группы $G$ в пространстве $V$ неприводимо. Тогда $\dim {V} \le 2$.
\end{proposition}

\begin{proof}
	Рассмотрим пространство $\widetilde V := \gl V\gr_\Cm$ над полем $\Cm \supset \R$, тогда, по уже доказанному, все элементы $g \in G$ имеют общий собственный вектор $v \in \widetilde V$ с собственным значением $\lambda \in \Cm$. Но тогда $\re{v}, \im{v} \in V$, и эти векторы порождают инвариантное подпространство $\gl \re{v}, \im{v}\gr_\R \le V$ размерности не большей, чем $2$.
\end{proof}

\begin{proposition}
	Пусть $H \normal G$, $U \le V$ "--- $H$-инвариантное подпространство. Тогда для любого $g \in G$ подпространство $gU \le V$ тоже $H$-инвариантно.
\end{proposition}

\begin{proof}
	Заметим, что для любого $g \in G$ выполнены следующие равенства:
	\[H(gU) = (Hg)U = (gH)U = gU\qedhere\]
\end{proof}

\begin{note}
	В терминах утверждения выше, следующая диаграмма коммутативна:
	\[
	\begin{tikzcd}[row sep = large]
		U \arrow[swap]{d}{g^{-1}hg} \arrow{r}{g} & gU \arrow{d}{h}\\
		g^{-1}hgU = U \arrow{r}{g} & hgU = gU
	\end{tikzcd}
	\]
\end{note}