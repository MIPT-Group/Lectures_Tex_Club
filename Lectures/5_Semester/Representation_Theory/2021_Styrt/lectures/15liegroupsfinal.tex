\section{Представления групп Ли}

\begin{theorem}
	Пусть $A$ "--- ассоциативная алгебра $A$, и $G = A^*$ "--- группа Ли. Тогда коммутирование в касательной алгебре $TG$ задается как коммутатор в алгебре $A$, то есть для любых $x, y \in TG$ выполнено следующее:
	\[[x, y] = xy - yx,~x, y \in TG\]
\end{theorem}

\begin{proof}
	Построение касательной алгебры группы Ли требовало сдвига единичного элемента в начало координат, поэтому произвольные элементы $a, b \in A^*$ представим в виде $a = 1 + x$, $b = 1+y$, где $x, y \in A$. Тогда:
	\begin{multline*}
		\cm(a, b) - 1 = (ab - ba)(ba)^{-1} = (xy - yx)(ba)^{-1} =
		\\
		= (xy - yx)(1 + o(1)) = xy - yx + o(\|z\|^2),~z \to 0
	\end{multline*}
	
	Значит, $xy - yx$ "--- это кососимметрическая форма, соответствующая коммутатору в группе $G$, поэтому она и задает коммутирование в касательной алгебре $TG$.
\end{proof}

\begin{definition}
	Пусть $X, Y$ "--- гладкие многообразия. \textit{Рангом} гладкого отображения $g : X \to Y$ в точке $x \in X$ называется величина $\rk_xg := \rk (d_xg)$.
\end{definition}

\begin{theorem}[\textit{без доказательства}]
	Пусть $g : X \to Y$ "--- гладкое отображение постоянного ранга $k$, и $x \in X$, $y = g(x) \in Y$. Тогда:
	\begin{enumerate}
		\item Множество $g(X) \subset Y$ является гладким многообразием размерности $k$, причем выполнено равенство $T_{y}g(X) = \im(d_{x}g) \subset T_{y}Y$
		
		\item Множество $g^{-1}(y) \subset X$ является гладким многообразием размерности $\dim{X} - k$, причем выполнено равенство $T_x g^{-1}(y) = \ke(d_xg) \subset T_xX$
	\end{enumerate}
\end{theorem}

\begin{definition}
	\textit{Действием группы Ли $G$ на гладком многообразии $X$} называется действие группы $G$ на $X$ как на множестве такое, что отображение $(g, x) \mapsto gx$ является гладким.
\end{definition}

\begin{example}
	Рассмотрим несколько примеров действий групп Ли:
	\begin{enumerate}
		\item Группа Ли $G$ действует на себе левыми и правыми сдвигами, а также сопряжениями.
		\item Пусть $A$ "--- конечномерная ассоциативная алгебра. Тогда группа Ли $G = A^*$ действует на $A$ левыми и правыми сдвигами, а также сопряжениями.
	\end{enumerate}
\end{example}

\begin{definition}
	\textit{Представлением группы Ли $G$ в пространстве $V$} называется действие группы Ли $G$ на $V$ как на многообразии такое, что каждый ее элемент действует на $V$ как линейный оператор.
\end{definition}

\begin{note}
	Имеет место эквивалентное определение, согласно которому представлением группы Ли называется гомоморфизм групп Ли $R : G \to \GL(V)$. Чтобы показать это, достаточно проверить, что отображение $(g, x) \mapsto gx$ гладкое $\lra$ отображение $R$ гладкое.
	\begin{itemize}
		\item[$\la$] Отображение $(g, x) \mapsto gx$ представимо в виде $(g, x) \mapsto (R(g), x) \mapsto (R(g))(x) = gx$ и потому является гладким
		\item[$\ra$] Пусть $(e_i)$ "--- базис в $V$, тогда каждое отображение вида $g \mapsto (R(g))(e_i) = R(g)_i$, где $R(g)_i$ "--- соответствующий столбец матрицы оператора $R(g)$, является гладким, поэтому гладким является и отображение $R$
	\end{itemize}
\end{note}

\begin{definition}
	Пусть $G$ "--- группа Ли. Подгруппа $H \le G$ называется \textit{подгруппой Ли}, если она сама является группой Ли.
\end{definition}

\begin{theorem}
	Пусть группа Ли $G$ действует на многообразии $X$, $x \in X$, и $\alpha_x$ "--- отображение вида $g \mapsto gx$, $d\alpha_x$ "--- дифференциал этого отображения в точке $e$. Тогда:
	\begin{enumerate}
		\item Стабилизатор $G_x := \{g \in G: gx = x\} \subset G$ является подгруппой Ли в $G$, причем $TG_x = \ke (d\alpha_x)$
		\item Орбита $G(x) := \{gx : g \in G\} \subset X$ является гладким многообразием размерности $\dim{G} - \dim{G_x}$, причем $T_xG(x) = \im(d\alpha_x)$
	\end{enumerate}
\end{theorem}

\begin{proof}
	Достаточно показать, что отображение $\alpha_x$ имеет постоянный ранг, чтобы воспользоваться теоремой выше. Для этого заметим, что следующая диаграмма коммутативна для каждого элемента $h \in G$:
	\[
	\begin{tikzcd}[row sep = large]
		G \arrow[swap]{d}{l_{h}} \arrow{r}{\alpha_x} & X \arrow{d}{h}\\
		G \arrow{r}{\alpha_x} & X
	\end{tikzcd}
	\]
	
	Значит, $d_{gx}h \circ d_g{\alpha_x} = d_{hg}{\alpha_x} \circ d_{l_{h}}g$ для любого $g \in G$. Тогда, поскольку $d_{gx}h$ и $d_{l_{h}}g$ невырожденны, имеем $\rk (d_g{\alpha_x}) = \rk (d_{hg}{\alpha_x})$. В силу произвольности выбора элементов $g, h \in G$, ранг отображения $\alpha_x$ постоянен и равен $\rk d{\alpha_x}$.
\end{proof}

\begin{corollary}
	Пусть $\phi: G \to H$ "--- гомоморфизм групп Ли, $d\phi: TG \to TH$ "--- соответствующий гомоморфизм касательных алгебр. Тогда:
	\begin{enumerate}
		\item Образ $\im\phi \subset H$ является подгруппой Ли в $H$, причем $T(\im\phi) = \im{(d\phi)} \subset TH$
		\item Ядро $\ke\phi \subset G$ является подгруппой Ли в $G$, причем $T(\ke\phi) = \ke(d\phi) \subset TG$
	\end{enumerate}
\end{corollary}

\begin{proof}
	Отображение $(g, h) \mapsto (\phi(g), h) \mapsto \phi(g)h$ является гладким, поэтому группа Ли $G$ действует на $H$ как на многообразии. Применяя предыдущую теорему к этому действию, получаем требуемое.
\end{proof}

\begin{definition}
	Пусть группа Ли $G$ имеет представление $R$ в пространстве $V$. \textit{Касательным представлением} называется гомоморфизм алгебр $dR : TG \to \mathfrak{gl}(V)$, обозначаемый через $\rho$.
\end{definition}

\begin{corollary}
	Пусть группа Ли $G$ имеет представление в пространстве $V$, и $v \in V$. Тогда стабилизатор $G_v := \{g \in G: gv = v\} \subset G$ является подгруппой Ли в $G$, причем выполнено равенство $TG_v = \{x \in TG: xv = 0\}$.
\end{corollary}

\begin{proof}
	Будем применять теорему выше к действию группы Ли $G$ на $V$. Рассмотрим отображение $\alpha_v: G \to V$ вида $g \mapsto gv = (R(g))(v)$, тогда $d\alpha_v$ имеет вид $x \mapsto (\rho(x))(v)$. Следовательно, $TG_v = \ke{(d\alpha_v)} = \{x : (\rho(x))(v) = xv = 0\}$.
\end{proof}

\begin{note}
	Нашей глобальной целью является показать, что касательная алгебра группы Ли является алгеброй Ли. Чтобы подчеркнуть это, далее мы будем обозначать касательную алгебру через $\mf g$ в случае группы $G$, или через $\mf h$ в случае группы $H$.
\end{note}

\begin{theorem}
	Пусть $G \subset \GL(V)$ "--- подгруппа Ли в $\GL(V)$, $\eta \in (V^*)^{\otimes k}$ "--- $k$-линейная форма. Тогда множество $H := \{h \in G: \forall v_1, \dotsc, v_k \in V: \eta(hv_1, \dotsc, hv_k) = \eta(v_1, \dotsc, v_k)\}$ образует подгруппу Ли в $G$, причем выполнено следующее равенство:
	\[\mf h = \left\{x \in \mf g: \forall v_1, \dotsc, v_k \in V: \sum_{i = 1}^k\eta(v_1, \dotsc, v_{i - 1}, x v_i, v_{i+1}, \dotsc, v_k) = 0\right\}\]
\end{theorem}

\begin{proof}
	Рассмотрим следующее действие группы Ли $G$ на пространстве $(V^*)^{\otimes k}$:
	\[(g\widetilde\eta)(v_1, \dotsc, v_k) := \widetilde\eta(g^{-1}v_1, \dotsc, g^{-1}v_k),~g \in G, \widetilde\eta \in (V^*)^{\otimes k}, v_1, \dotsc, v_k \in V\]
	
	Тогда множество $H$ является стабилизатором формы $\eta \in (V^*)^{\otimes k}$ и потому является подгруппой Ли. Остается доказать равенство для $\mf h$. Для этого продифференцируем равенство $\eta(h^{-1}v_1, \dotsc, h^{-1}v_k) = \eta(v_1, \dotsc, v_k)$ по $h$ в точке $e$ при фиксированных $v_1, \dotsc, v_k \in V$:
	\[-\sum_{i = 1}^k\eta(v_1, \dotsc, v_{i - 1}, x v_i, v_{i+1}, \dotsc, v_k) = 0\]
	
	Здесь $h \in G$ под действием дифференциала переходит в $x \in \mf g$.
\end{proof}

\begin{definition}
	Пусть $A$ "--- алгебра. \textit{Дифференцированием} алгебры $A$ называется линейный оператор $D \in \mathcal{L}(A)$, удовлетворяющий \textit{тождеству Лейбница}:
	\[D(ab) = (Da)b + a(Db),~a, b \in A\]
	
	Дифференцирования алгебры $A$ образуют алгебру с операциями сложения и композиции, обозначаемую через $\Der{A}$.
\end{definition}

\begin{corollary}
	Пусть $A$ "--- алгебра, $G \subset \GL(A)$ "--- подгруппа Ли в $\GL(A)$. Тогда множество $H := G \cap \Aut{A}$ образует подгруппу Ли в $G$, причем $\mf h = \mf g \cap \Der{A}$.
\end{corollary}

\begin{proof}
	Аналогично предыдущей теореме, но с рассмотрением действия пространстве $(V^*)^{\otimes 2} \otimes V$.
\end{proof}

\begin{corollary}
	Пусть $A$ "--- алгебра. Тогда группа автоморфизмов $\Aut{A}$ алгебры $A$ является группой Ли, причем $T(\Aut{A}) = \Der{A}$.
\end{corollary}

\begin{definition}
	Пусть $G$ "--- группа Ли. Для произвольного элемента $g \in G$ зададим автоморфизм $a_g \in \Aut{G}$ как $h \mapsto ghg^{-1}$. \textit{Присоединенным представлением} называется гомоморфизм групп Ли $\Ad: G \to \GL(\mf g)$ вида $g \mapsto da_g$.
\end{definition}

\begin{note}
	Проверим, что линейный оператор $da_g$ действительно является невырожденным. Для этого заметим, что сопоставление $a_e = da_e = E$, и для любых $g, h \in G$ выполнено $da_{gh} = d(a_g \circ a_h) = da_g \circ da_h$, то есть любой оператор вида $da_g$ обратим.
\end{note}

\begin{theorem}
	Пусть $G$ "--- группа Ли, $\ad: \mf g \to \mf{gl}(\mf g)$ "--- дифференциал присоединенного представления. Тогда для любых $x, y \in \mf{g}$ выполнено следующее равенство:
	\[(\ad (x))(y) = [x, y]\]
\end{theorem}

\begin{proof}
	Будем снова считать, что единичный элемент является началом координат, и зафиксируем элемент $x \in \mf{g}$, тогда:
	\[\cm(x, y) = xyx^{-1}y^{-1} = \fr{(a_x(y), y)} = a_x(y) - y + o(\|y\|),~y \to 0\]
	
	Следовательно, $d_yc(x, 0) = \Ad(x) - E$. Получено отображение $d_yc: \mf g \to \GL(\mf g)$. Дифференцируя его по $x$, получим, что $d_{xy}c(0, 0) = \ad$. Следовательно:
	\[c(x, y) = (\ad (x))(y) + o(\|z\|^2),~z \to 0\]
	
	Таким образом, $(\ad (x))(y) = [x, y]$ для любых $x, y \in \mf g$, что и требовалось.
\end{proof}

\begin{theorem}
	Пусть $\mf g$ "--- антикоммутативная алгебра с операцией $[\cdot, \cdot]$. Тогда следующие условия эквивалентны:
	\begin{enumerate}
		\item $\mf g$ "--- алгебра Ли
		\item Для любого $x \in \mf g$ оператор $\ad (x) := [x, \cdot]$ является дифференцированием
	\end{enumerate}
\end{theorem}

\begin{proof}
	Зафиксируем произвольные $x, y, z \in \mf g$, тогда условие $(1)$ эквивалентно следующему равенству:
	\[[x, [y, z]] + [y, [z, x]] + [z, [x, y]] = 0\]
	
	В силу антикоммутативности и билинейности, это эквивалентно следующему:
	\[[x, [y, z]] = [y, [x, z]] + [[x, y], z]\]
	
	Полученное равенство, в свою очередь, эквивалентно условию $(2)$.
\end{proof}

\begin{corollary}
	Пусть $G$ "--- группа Ли. Тогда ее касательная алгебра $\mf g$ является алгеброй Ли.
\end{corollary}

\begin{proof}
	Алгебра $\mf g$ антикоммутативна, поэтому достаточно проверить, что для любого $x \in \mf g$ оператор $\ad (x) = [x, \cdot]$ является дифференцированием. Но для любого $g \in G$ выполнено $a_g \in \Aut G$, тогда $\im (da_g) = T(\im a_g) = \mf g$, поэтому $da_g \in \Aut\mf g$. Следовательно, $\Ad$ является гомоморфизмом групп Ли $G$ и $\Aut{\mf g}$, тогда $\ad$ является гомоморфизмом касательных алгебр $\mf g$ и $T(\Aut{\mf g}) = \Der{\mf g}$. Значит, оператор $\ad (x) = [x, \cdot]$ действительно является дифференцированием для любого $x \in \mf g$.
\end{proof}

\begin{note}
	На самом деле, есть еще одно условие, эквивалентное тому, что антикоммутативная алгебра $\mf g$ является алгеброй Ли, согласно которому сопоставление $\ad$ вида $x \mapsto [x, \cdot]$ является гомоморфизмом алгебр $\mf g$ и $\mf{gl}(\mf g)$. Из него еще проще вывести, что касательная алгебра группы Ли является алгеброй Ли.
\end{note}