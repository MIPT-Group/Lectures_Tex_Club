\section{Представления групп перестановок}

\begin{problem}
	Постройте таблицу характеров неприводимых представлений группы $S_n$ над алгебраически замкнутым полем $K$.
\end{problem}

\begin{solution}
	Группа $S_3$ имеет три неприводимых представления:
	\begin{itemize}
		\item Тождественное представление $R_\id$ в одномерном пространстве $V$ следующего вида:
		\[\sigma x = x,~\sigma \in S_3, x \in V\]
		
		\item Знаковое представление $R_{\mathrm{sgn}}$ в одномерном пространстве $V$ следующего вида:
		\[\sigma x = (\sgn\sigma)x,~\sigma \in S_3, x \in V\]
		
		\item Представление $R_0$ в пространстве $V = \{(x_1, x_2, x_3) \in K^3 : x_1 + x_2 + x_3 = 0\}$ размер-ности $2$, заданное на базисных векторах $e_1, e_2, e_3$ в $K^3$ следующим образом:
		\[\sigma e_i = e_{\sigma(i)},~\sigma \in S_3, i \in \{1, 2, 3\}\]
	\end{itemize}
	
	Таблица характеров представлений имеет следующий вид:
	\begin{center}
		\begin{tabular}{r|c|c|c}
			& $\{\id\}$ & $(12)^{S_3}$ & $(123)^{S_3}$ \\ \hline
			$R_\id$ &     1     &     $1$      &       1       \\
			$R_{\mathrm{sgn}}$ &     1     &     $-1$     &       1       \\ 
			$R_0$ &     2     &      0       &     $-1$
		\end{tabular}
	\end{center}
	
	Видно также, что характеры представлений выше образуют ортонормированную систему.
\end{solution}

\begin{problem}
	Опишите все неприводимые представления группы $A_4$ в пространстве $V$ над алгебраически замкнутым полем $K$ и постройте таблицу характеров представлений.
\end{problem}

\begin{solution}
	Сначала найдем все одномерные неприводимые представления. Они естественным образом соответствуют гомоморфизмам из $A_4$ в $K^*$. Но группа $K^*$ "--- абелева, поэтому ядро любого такого гомоморфизма содержит $A_4' = V_4$. Значит, $\phi$ можно представить в следующем виде:
	\[
	\begin{tikzcd}[row sep = small, column sep = small]
		A_4 \arrow[swap]{dr}{\pi} \arrow[]{rr}{\phi} && K^*\\
		& A_4 / V_4 \cong \Z_3 \arrow[swap]{ur}{\psi}&
	\end{tikzcd}
	\]
	
	Следовательно, достаточно описать гомоморфизмы из $\Z_3 = \gl a \gr$ в $K^*$. Их ровно $3$, и они имеют вид $a \mapsto \lambda$, где $\lambda \in K^*$ "--- корень из единицы степени $3$. Заметим теперь, что изоморфизм $A_4 / V_4 \cong \Z_3$ осуществляется сопоставлением $(123)V_4 \mapsto a$, поэтому полученные представления достаточно задать на $(123)$ как $(123)x := \lambda x$ при всех $x \in V$. Легко видеть, что три полученных представления попарно неизоморфны.
	
	Наконец, сумма квадратов размерностей неприводимых представлений равна $|A_4| = 12$. Значит, осталось найти одно неприводимое представление размерности $3$. Стандартное представление группы $A_4$ в пространстве $\{(x_1, \dotsc, x_4)^T \in K^4 : x_1 + \dotsb + x_4 = 0\}$ не имеет инвариантных одномерных представлений, поэтому это именно оно. Таким образом, получены следующие представления:
	\begin{itemize}
		\item Представление $R_\lambda$, где $\lambda \in K^*$ "--- корень из единицы степени $3$, в одномерном пространстве $V$, имеющее следующий вид:
		\begin{gather*}
			(123) x = \lambda x,~x \in V\\
			\sigma x = x,~\sigma \in V_4, x \in V
		\end{gather*}
		
		\item Представление $R_0$ в пространстве $V = \{(x_1, \dotsc, x_4) \in K^4 : x_1 + \dotsb + x_4 = 0\}$ размерности $3$, заданное на базисных векторах $e_1, \dotsc, e_4$ в $K^4$ следующим образом:
		\[\sigma e_i = e_{\sigma(i)},~\sigma \in A_4, i \in \{1, \dotsc, 4\}\]
	\end{itemize}
	
	Таблица характеров представлений имеет следующий вид:
	\begin{center}
		\begin{tabular}{r|c|c|c|c}
			& $\{\id\}$ & $(123)^{A_4}$ & $(132)^{A_4}$ & $(12)(34)^{A_4}$ \\ \hline
			$R_\lambda$ &    $1$    &   $\lambda$   &  $\lambda^2$  &      $1$      \\ \hline
			$R_0$ &    $3$    &      $0$      &      $0$      &       $-1$
		\end{tabular}
	\end{center}
	
	Как и в прошлых примерах, видно, что характеры представлений выше образуют ортонормированную систему.
\end{solution}

\begin{problem}
	Опишите все неприводимые представления группы $S_4$ в пространстве $V$ над алгебраически замкнутым полем $K$ таким, что $\cha{K} \nmid |S_4|$, и постройте таблицу характеров.
\end{problem}

\begin{example}
	 Аналогично прошлому примеру, заметим сначала, что $S_4' = A_4$, и найдем все одномерные неприводимые представления. Достаточно описать гомоморфизмы из $S_4 / A_4 \cong \Z_2 = \gl a \gr$ в $K^*$. Их ровно $2$, и они имеют вид $a \mapsto \lambda$, где $\lambda = \pm 1$. Изоморфизм $S_4 / A_4 \cong \Z_2$ осуществляется сопоставлением $(12)A_4 \mapsto a$, поэтому полученные представления достаточно задать на $(12)$ как $(12) x = \lambda x$ при всех $x \in V$. Очевидно, полученные представления неизоморфны.
	
	Сумма квадратов размерностей неприводимых представлений равна $|S_4| = 24$, поэтому осталось найти одно неприводимое представление размерности $2$ и два неприводимых представления размерности $3$. Первое представление размерности $3$ "--- это стандартное представление группы $S_4$ в пространстве $\{(x_1, \dotsc, x_4)^T \in K^4 : x_1 + \dotsb + x_4 = 0\}$. Второе же соответствует действию группы $S_4$ на $K^3$ как группы вращений куба. Оно неприводимо, поскольку не содержит одномерных инавриантных подпространств, а пространство $K^3$ вполне приводимо в силу ограничения $\cha{K} \nmid |S_4|$.
	
	Найдем неприводимое представление размерности $2$. Поскольку представление "--- это гомоморфизм из группы $G$ в пространство линейных операторов, воспользуемся тем, что $S_4 / V_4 \cong S_3$, и рассмотрим следующую последовательность гомоморфизмов:
	\[S_4 \xrightarrow{\pi} S_4 / V_4 \cong S_3 \xrightarrow{R'_0} \GL_2(K)\]
	
	Здесь $\pi$ "--- факторизация по подгруппе $V_4$, $R'_0$ "--- неприводимое представление группы $S_3$ размерности $2$. Очевидно, тогда представление $R_{S_3} := R_0' \circ \pi$ тоже неприводимо. Таким образом, получены следующие представления:
	\begin{itemize}
		\item Тождественное представление $R_\id$ в одномерном пространстве $V$ следующего вида:
		\[\sigma x = x,~\sigma \in S_4, x \in V\]
		
		\item Знаковое представление $R_{\mathrm{sgn}}$ в одномерном пространстве $V$ следующего вида:
		\[\sigma x = (\sgn\sigma)x,~\sigma \in S_4, x \in V\]
		
		\item Представление $R_0$ в пространстве $V = \{(x_1, \dotsc, x_4) \in K^4 : x_1 + \dotsb + x_4 = 0\}$ размерности $3$, заданное на базисных векторах $e_1, \dotsc, e_4$ в $K^4$ следующим образом:
		\[\sigma e_i = e_{\sigma(i)},~\sigma \in S_4, i \in \{1, \dotsc, 4\}\]
		
		\item Представление $R_c$ в $K^3$, заданное на базисных векторах $e_1, e_2, e_3$ в $K^3$, параллельных ребрам куба, следующим образом:
		\[\sigma e_i = \sigma_c(e_{i}),~\sigma \in S_4, i \in \{1, 2, 3\}\]
		
		Здесь для каждого $\sigma \in S_4$ через $\sigma_c$ обозначен соответствующий оператор из группы вращений куба в $K^3$.
		
		\item Представление $R_{S_3}$ в пространстве $V = \{(x_1, x_2, x_3) \in K^3 : x_1 + x_2 + x_3 = 0\}$ размерности $2$, заданное на базисных векторах $e_1, e_2, e_3$ в $K^3$ следующим образом:
		\begin{gather*}
			\sigma e_i = e_{\sigma(i)},~\sigma \in S_3, i \in \{1, 2, 3\}\\
			\sigma x = x,~\sigma \in V_4, x \in V
		\end{gather*}
	\end{itemize}

	Вычислим характеры данных представлений. Это нетривиально только для $R_c$ и $R_{S_3}$.  Для представления $R_c$ заметим, что изоморфизм группы $S_4$ и группы вращений куба в $K^3$ осуществляется сопоставлением $\sigma \mapsto \sigma_c$, и проследим, как при действии оператора $\sigma_c$ преобразуются базисные векторы $e_1, e_2, e_3 \in K^3$, параллельные ребрам куба. Для представления $R_{S_3}$ воспользуемся тем, что характер каждой перестановки в смежном классе $\sigma V_4$ при $\sigma \in S_3$ равен характеру перестановки $\sigma$ в представлении $R_0'$. Итоговая таблица характеров представлений имеет следующий вид:
	\begin{center}
		\begin{tabular}{r|c|c|c|c|c}
			& $\{\id\}$ & $(12)^{S_4}$ &  $(123)^{S_4}$ & $(1234)^{S_4}$ & $(12)(34)^{S_4}$ \\ \hline
			$R_\id$ &    $1$    &   $1$   &  $1$  &      $1$ & $1$     \\
			$R_{\mathrm{sgn}}$ &    $1$    &   $-1$   &  $1$  &      $-1$ & $1$     \\
			$R_0$ &    $3$    &      $1$      &      $0$      &       $-1$ &  $-1$ \\
			$R_c$ &    $3$    &      $-1$      &      $0$      &       $1$ & $-1$\\ 
			$R_{S_3}$ &    $2$    &      $0$      &      $-1$      &       $0$ & $2$\\
		\end{tabular}
	\end{center}
	
	Как и в прошлых примерах, видно, что характеры представлений выше образуют ортонормированную систему.
\end{example}

\begin{note}
	Если $\cha{K} \nmid n$, то значения характера представления $R_0$ группы $S_n$ в пространстве $W := \{(x_1, \dotsc, x_n)^T \in K^n : x_1 + \dotsb + x_n = 0\}$ можно вычислять проще, используя следующее соображение. При заданных ограничениях на поле $K$, $K^n = W \oplus U$, где $U = \{(x_1, \dotsc, x_n)^T \in K^n : x_1 = \dotsb = x_n\}$. Тогда для любой перестановки $\sigma \in S_n$ выполнено равенство:
	\[\Chi_{V}(\sigma) = \Chi_W(\sigma) + \Chi_U(\sigma)\]
	
	Поскольку все перестановки действуют на $U$ тождественно, $\Chi_U(\sigma) = 1$. По определению следа оператора, $\Chi_V(\sigma) = |\{i \in \{1, \dotsc, n\} : \sigma(i) = i\}|$. Значит, получена следующая формула:
	\[\Chi_W(\sigma) = |\{i \in \{1, \dotsc, n\} : \sigma(i) = i\}| - 1\]
\end{note}

\begin{note}
	Если $\cha{K} = 0$ и группа $G$ конечна, то значения характеров неприводимых представлений группы $G$ в пространстве $V$ над $K$ иногда вычисляются проще. Пусть $\Chi_1, \Chi_2$ "--- характеры неприводимых представлений, причем значения первого известны на всех классах сопряженности известны, а значения второго --- на всех, кроме одного класса $h^G$, причем $\Chi_1(h^{-1}) \ne 0$. Тогда, пользуясь ортогональностью характеров, получим:
	\[0 = |G|(\Chi_1, \Chi_2) = \sum_{g \in G}\Chi_1(g^{-1})\Chi_2(g)\]
	
	Решая уравнение выше относительно $\Chi_2(h)$, получим следующее равенство:
	\[\Chi_2(h) = -\frac{\sum_{g \in G \bs h^G}\Chi_1(g^{-1})\Chi_2(g)}{|h^G|\Chi_1(h^{-1})}\]
\end{note}