\section{Групповая алгебра}

\begin{definition}
	Пусть $G$ "--- конечная группа. \textit{Групповой алгеброй} группы $G$ называется пространство $K[G]$ с базисом $(e_g)_{g \in G}$, в котором $G$ имеет регулярное представление, со структурой алгебры, задаваемой произведением, определенным на базисных векторах следующим образом:
	\[e_g \cdot e_h := e_{gh},~g, h \in G\]
\end{definition}

\begin{proposition}
	Пусть $G$ "--- конечная группа с представлением в пространстве $V$. Тогда отображение $\phi: \mc L^G(K[G], V) \to V$ такое, что для любого $A \in L^G(K[G], V)$ выполнено $\phi(A) = A(e_e)$, является изоморфизмом линейных пространств.
\end{proposition}

\begin{proof}
	Линейность отображения очевидна. Проверим его биективность. С одной стороны, оператор $A \in L^G(K[G], V)$ не \pagebreak более чем однозначно задается значением $A(e_e) =: v$, поскольку для любого $g \in G$ выполнено $A(e_g) = A(ge_e) = gv$. С другой стороны, заданный таким образом линейный оператор получается $G$-эквивариантным по построению, поскольку имеет место следующая коммутативная диаграмма:
	\[
	\begin{tikzcd}[row sep = large]
		e_h \arrow[swap]{d}{g} \arrow{r}{A} & hv \arrow{d}{g}\\
		e_{gh} \arrow{r}{g} & ghv
	\end{tikzcd}
	\]
	
	Значит, отображение биективно, и получено требуемое.
\end{proof}

\begin{corollary}
	Пусть $G$ "--- конечная группа, поле $K$ алгебраически замкнуто, и $\cha{K} \nmid |G|$. Тогда любое неприводимое представление группы $G$ входит в разложение регулярного представления в $K[G]$ с кратностью, равной $\dim{V}$.
\end{corollary}

\begin{proof}
	Пусть $K[G] = \bigoplus_{i = 1}^m V_i$ "--- разложение $K[G]$ на неприводимые подпространства. Рассмотрим произвольное неприводимое представление группы $G$ в пространстве $V$. Тогда, из существования изоморфизма между $V$ и $L^G(K[G], V)$, имеем:
	\[\dim{V} = \dim\mc L^G(K[G], V) = \dim \left(\bigoplus_{i = 1}^m\mc L^G(V_i, V)\right) = \sum_{i = 1}^m\dim\mc L^G(V_i, V)\]
	
	Но $\dim\mc L^G(V_i, V) = \mathbb I(V_i \cong V)$ для любого $i \in \{1, \dotsc, m\}$ в силу неприводимости пространств $V_i$ и $V$, поэтому $|\{i \in \{1, \dotsc, m\}: V_i \cong V\}| = \dim{V}$.
\end{proof}

\begin{corollary}
	Пусть $G$ "--- конечная группа, поле $K$ алгебраически замкнуто, и $\cha{K} \nmid |G|$. Тогда число попарно неизоморфных неприводимых представлений группы $G$ конечно. Более того, если $R_1, \dotsc, R_k$ "--- все попарно неизоморфные неприводимые представления группы $G$ в пространствах $W_1, \dotsc, W_k$ соответственно, то:
	\[\sum_{i = 1}^k\dim^2{W_i} = |G|\]
\end{corollary}

Теперь мы докажем более сильную теорему, обобщающую результат выше. Для этого потребуется одно вспомогательное утверждение.

\begin{definition}
	Пусть $A$ "--- ассоциативная алгебра с единицей, $a \in A$. \textit{Оператором левого сдвига} называется оператор $l_a \in \mc L(A)$ такой, что для всех $b \in A$ выполнено $l_a(b) = ab$. \textit{Оператором правого сдвига} называется оператор $r_a \in \mc L(A)$ такой, что для всех $b \in A$ выполнено $r_a(b) = ba$.
\end{definition}

\begin{proposition}
	Пусть $A$ "--- ассоциативная алгебра с единицей. Рассмотрим сопоставление $\phi: a \mapsto r_a$, $a \in A$. Тогда $\phi$ инъективно, причем его образом является множество линейных операторов на $A$, коммутирующих со всеми операторами левого сдвига.
\end{proposition}

\begin{proof}
	Сопоставление $\phi$ линейно, и поскольку $\ke\phi = \{0\}$, то оно инъективно. Кроме того, очевидно, что $\im\phi \subset \{C\in \mc L(A): \forall a \in A: Cl_a = l_aC\}$. Докажем обратное включение. Пусть оператор $C \in \mc L(A)$ таков, что для любого $a \in A$ выполнено $Cl_a = l_aC$, тогда $C(a) = Cl_a(1) = l_aC(1) = aC(1)$. Значит, $C = r_{C(1)}$.
\end{proof}

\begin{note}
	Из доказательства выше следует, что $\phi$ является <<антиизоморфизмом>> алгебр $A$ и $\{C\in \mc L(A): \forall a \in A: Cl_a = l_aC\} \le \mc L(A)$. Это утверждение можно переписать в виде $\{C\in \mc L(A): \forall a \in A: Cl_a = l_aC\} \cong A^{op}$, где $A^{op}$ "--- пространство $A$ с умножением, меняющим множители местами относительно обычного умножения в $A$.
\end{note}

\begin{theorem}
	Пусть $G$ "--- конечная группа, поле $K$ алгебраически замкнуто, $\cha{K} \nmid |G|$. Тогда если $R_1, \dotsc, R_k$ "--- все попарно неизоморфные неприводимые представления группы $G$ в пространствах $W_1, \dotsc, W_k$ размерностей $n_1, \dotsc, n_k$ соответственно, то:
	\[K[G] \cong \bigoplus_{i=1}^kM_{n_i}(K)\]
\end{theorem}

\begin{proof}
	По уже доказанному, операторы на $K[G]$, коммутирующие с левыми сдвигами, образуют алгебру, изоморфную $K[G]^{op}$. Но такие операторы "--- это в точности $G$-эквивариантные операторы на $K[G]$, образующие алгебру $\mc L^G(K[G])$. Тогда:
	\[\mc L^G(K[G]) = \mc L^G(K[G]; K[G]) = \mc L^G\left(\bigoplus_{i=1}^mV_i; \bigoplus_{i=1}^mV_i\right) = \bigoplus_{i = 1}^m\bigoplus_{j=1}^m \mc L^G(V_i; V_j)\]
	
	В силу неприводимости пространств, $L^G(V_i; V_j) = \{0\}$ при $V_i \centernot\cong V_j$, тогда:
	\[\mc L^G(K[G]) = \bigoplus_{i, j \in \{1, \dotsc, m\}: V_i \cong V_j}\mc L^G(V_i; V_j) = \bigoplus_{i = 1}^k\left(\mc L^G(W_i)\right)^{\oplus n_i^2}\]
	
	Значит, матрица любого оператора из $\mc L^G(K[G])$ имеет блочно-диагональный вид, причем	для любого $i \in \{1, \dotsc, k\}$ блок размера $n_i^2 \times n_i^2$ состоит из скалярных блоков размера $n_i \times n_i$, каждый из которых отвечает оператору из $\mc L^G(W_i)$. Отсюда получаем следующее:
	\[\mc L^G(K[G]) = \bigoplus_{i = 1}^kM_{n_i}(K)\]
	
	Таким образом, выполнена следующая цепочка:
	\[K[G] \cong L^G(K[G])^{op} = \left(\bigoplus_{i = 1}^kM_{n_i}(K)\right)^{op} = \bigoplus_{i=1}^kM_{n_i}(K)^{op} \cong \bigoplus_{i=1}^kM_{n_i}(K)\]
	
	Последний изоморфизм осуществляется оператором транспонирования.
\end{proof}

\begin{proposition}
	Пусть $G$ "--- конечная группа с представлением в $V$. Тогда на $V$ можно определить скалярное произведение $(\cdot, \cdot)$ таким образом, что для любого элемента $g \in G$ и векторов $u, v \in V$ выполнено $(gu, gv) = (u, v)$.
\end{proposition}

\begin{proof}
	Зафиксируем некоторое скалярное произведение $\gl \cdot, \cdot \gr$ на $V$. Определим искомое скалярное произведение следующим образом:
	\[(u, v) := \sum_{g \in G}\gl gu, gv \gr,~u, v \in V\]
	
	Билинейность и положительная определенность очевидны. Проверим, что полученное скалярное произведение $G$-эквивариантно. Зафиксируем произвольный элемент $h \in G$ и векторы $u, v \in V$, тогда:
	\[(hu, hv) = \sum_{g \in G}\gl ghu, ghv \gr =  \sum_{\widetilde g \in G}\left\gl \widetilde gu, \widetilde gv \right\gr = (u, v)\]
	
	Таким образом, получено требумое.
\end{proof}

\begin{note}
	Из утверждения выше легко вывести полную приводимость вещественных представлений, не опираясь на общую теорему. Действительно, если $G$ имеет представление в пространстве над $\R$, то любое $G$-инвариантное подпространство имеет ортогональное дополнение относительно $G$-эквивариантного скалярного произведения, которое тоже будет $G$-инвариантным.
\end{note}