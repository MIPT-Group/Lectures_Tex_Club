\section{Характеры представлений}

\begin{definition}
	\textit{Характером} группы $G$ над $K$ называется функция $\Chi : G \to K$, постоянная на классах сопряженности. Все характеры группы $G$ над полем $K$ образуют линейное пространство над $K$ с естественными операциями сложения и умножения на скаляры из $K$.
\end{definition}

\begin{note}
	Условие из определения характера можно переписать следующими эквивалентными способами:
	\begin{align*}
		\Chi(g^{-1}hg) = \Chi(h),~g, h \in G\\
		\Chi(gh) = \Chi(hg),~g, h \in G
	\end{align*}
\end{note}

\begin{note}
	Пусть $G$ "--- конечная группа, $\cha K \nmid |G|$. Определим на пространстве $X_K(G)$ характеров группы над $K$ следующую билинейную форму:
	\[(\Chi_1, \Chi_2) := \frac1{|G|}\sum_{g \in G}\Chi_1(g)\Chi_2(g^{-1}),~\Chi_1, \Chi_2 \in X_K(G)\]
	
	Легко заметить, что полученная форма симметрична. Проверим, что она невырожденна, то есть имеет нулевое ядро. Действительно, пусть для некоторого $\Chi_0 \in X_G(G)$ и любого $\Chi \in X_K(G)$ выполнено $(\Chi_0, \Chi) = 0$. Тогда, определяя $\Chi$ как функцию, отличную от нуля и равную единице только на одном классе сопряженности $h^G := \{h^g : g \in G\}$, получим следующее:
	\[\sum_{h^g \in h^G}\Chi_0\left((h^{-1})^g\right) = \sum_{(h^{-1})^g \in (h^{-1})^G}\Chi_0\left((h^{-1})^g\right) = \left|(h^{-1})^G\right| \Chi_0(h^{-1}) = 0\]
	
	Поскольку $\left|(h^{-1})^G\right| = \left|h^G\right| \mid |G|$, то $\cha K \nmid \left|(h^{-1})^G\right|$, откуда $\Chi_0(h^{-1}) = 0$. В силу произвольности выбора класса сопряженности, $\Chi_0 = 0$.
\end{note}

\begin{definition}
	\textit{Характером представления} группы $G$ в пространстве $V$ называется функция $\Chi_V: G \to K$ следующего вида:
	\[\Chi_V(g) = \tr(g),~g \in G\]
\end{definition}

\begin{proposition}
	Пусть $V, W$ "--- пространства над полем $K$, группа $G$ имеет представление в $V$ и $W$. Тогда для произвольных характеров соответствующих представлений группы $G$ и произвольных элементов $g \in G$ выполнены следующие равенства:
	\begin{enumerate}
		\item $\Chi_{V \oplus W}(g) = \Chi_V(g) + \Chi_W(g)$
		
		\item Если $V$ "--- вполне приводимое пространство, а $U \le V$ "--- его $G$-инвариантное подпространство, то $\Chi_{V \bs U}(g) = \Chi_V(g) - \Chi_U(g)$
		
		\item $\Chi_{V \otimes W}(g) = \Chi_V(g)\Chi_W(g)$
		
		\item $\Chi_{V^*}(g) = \Chi_V(g^{-1})$
		
	\end{enumerate}
	
	\begin{proof}~
		\begin{enumerate}
			\item Матрица оператора $g$ в $V \oplus W$ имеет блочно-диагональный вид с блоками, являющимися матрицами оператора $g$ в $V$ и $W$.
			
			\item Выберем $\mc A$-инвариантное подпространство $W \le V$ такое, что $U \oplus V$. Зафиксируем базисы $e'$, $e''$ в пространствах $U$ и $W$, тогда их объединение $e$ является базисом в $V$. Пусть $g \leftrightarrow_{e'} A$, $g \leftrightarrow_{e''} B$, тогда:
			\[g \leftrightarrow_e \left(\begin{array}{@{}c|c@{}}
					A & *\\
					\hline
					0 & B
				\end{array}\right)\]
			
			Следовательно, $\Chi_V(g) = \Chi_{V \bs U}(g) + \Chi_U(g)$.
			
			\item Пусть $e := (e_i)$, $f := (f_j)$ "--- базисы в пространствах $V$, $W$. Если $g \leftrightarrow_{e} A$, $g \leftrightarrow_{f} B$, то тогда $g \leftrightarrow_{e \otimes f} A \otimes B$. Значит, выполнены следующие равенства:
			\[\Chi_{V \otimes W}(g) = \tr(A \otimes B) = \tr{A}\tr{B}=\Chi_V(g)\Chi_W(g)\]
			
			\item Зафиксируем двойственные базисы $(e_i)$ и $(\xi_j)$ в пространствах $V$ и $V^*$. Тогда:
			\[(g\xi_j)(e_i) = \xi_j(g^{-1} e_i)\]
			
			Из равенства выше следует требуемое.\qedhere
		\end{enumerate}
	\end{proof}
\end{proposition}

\begin{corollary}
	Пусть $V, W$ "--- пространства над полем $K$, группа $G$ имеет представление в $V$ и $W$. Тогда для произвольных характеров соответствующих представлений группы $G$ и произвольных элементов $g \in G$ выполнено равенство:
	\[\Chi_{\mc L(V ; W)}(g) = \Chi_V(g^{-1})\Chi_W(g)\]
\end{corollary}

\begin{proof}
	Воспользуемся изоморфизмом представлений $\mc L(V; W) \cong_G V^* \otimes W$ и применим утверждение выше к представлению группы $G$ в пространстве $V^* \otimes W$.
\end{proof}

\begin{note}
	Равенство выше можно получить и непосредственно. Вспомним, что представление группы $G$ в пространстве $\mc L(V; W)$ имеет следующий вид:
	\[gA := gAg^{-1},~g \in G, A \in \mc L(V; W)\]
	
	Зафиксируем базисы $(v_1, \dotsc, v_n)$, $(w_1, \dotsc, w_m)$ в пространствах $V$ и $W$, а также базис $(E_{ij})_{i = 1, j = 1}^{m, n}$ в пространстве $M_{m \times n}(K) \cong L(V; W)$, и заметим, что тогда для произвольных $i \in \{1, \dotsc, m\}$, $j \in \{1, \dotsc, n\}$ выполнено $(gE_{ij}g^{-1})_{ij} = g_{ii}(g^{-1})_{jj}$. Следовательно:
	\[\Chi_{\mc L(V; W)}(g) = \sum_{i = 1}^{m}\sum_{j = 1}^{n}g_{ii}(g^{-1})_{jj} = \left(\sum_{i = 1}^{m}g_{ii}\right)\left(\sum_{j = 1}^{n}(g^{-1})_{jj}\right) = \Chi_W(g)\Chi_V(g^{-1})\]
\end{note}

\begin{theorem}
	Пусть $G$ "--- конечная группа, $\cha{K} \nmid |G|$. Тогда для любых представлений группы $G$ в пространствах $V, W$ выполнено следующее равенство:
	\[(\Chi_{V}, \Chi_{W}) = \dim \mc L^G(V; W)\]
\end{theorem}

\begin{proof}
	По теореме о сумме следов элементов группы, в условиях теоремы выполнены следующие равенства:
	\[(\Chi_{V}, \Chi_{W}) = \frac1{|G|}\sum_{g \in G}\Chi_{V}(g^{-1})\Chi_{W}(g) = \frac1{|G|}\sum_{g \in G}\Chi_{\mc L(V; W)}(g) = \dim{\mc L(V; W)^G}\]
	
	Но $A \in \mc L(V; W)$ "--- неподвижная точка представления группы $G$ $\lra$ для любого $g \in G$ выполнено $A = gAg^{-1}$ $\lra$ для любого $g \in G$ выполнено $Ag = gA$ $\lra$ $A \in \mc L^G(V; W)$. Значит, получено требуемое.
\end{proof}

\begin{corollary}
	Пусть $G$ "--- конечная группа, поле $K$ алгебраически замкнуто, и $\cha{K} \nmid |G|$. Тогда характеры всех классов изоморфности неприводимых представлений группы $G$ в пространстве $V$ образуют ортонормированную систему в $X_K(G)$.
\end{corollary}

\begin{proof}
	Применим теорему выше к неприводимым представлениям группы $G$ в пространствах $V_i$ и $V_j$, тогда:
	\[(\Chi_{V_i}, \Chi_{V_j}) = \dim \mc L^G(V_i; V_j) = \mathbb I(V_i \cong V_j)\qedhere\]
\end{proof}

\begin{theorem}
	Пусть $G$ "--- конечная группа, поле $K$ алгебраически замкнуто, и $\cha{K} = 0$. Тогда характеры всех классов изоморфности неприводимых представлений группы $G$ в пространстве $V$ образуют ортонормированный базис в $X_K(G)$.
\end{theorem}

\begin{proof}
	Очевидно, ортонормированная система линейно независима, поэтому достаточно проверить, что линейная оболочка характеров из условия совпадает с $X_K(G)$. Предположим, что это не так, тогда существует характер $\Chi \in X_K(G)\bs\{0\}$, ортогональный всем характерам неприводимых представлений. Зафиксируем неприводимое представление $R$ группы $G$ в пространстве $V$ с характером $\Chi_R$. Тогда:
	\[\sum_{g \in G}\Chi{(g^{-1})}\Chi_R(g) = 0\]
	
	Положим $C_g := \Chi(g^{-1})$ для каждого $g \in G$, тогда:
	\[\tr\left(\,\sum_{g \in G}C_g R(g)\right) = 0\]
	
	Заметим теперь, что для любых $g, h \in G$ выполнено $C_{g} = C_{g^h}$ по определению характера. Докажем, что тогда элемент $a := \sum_{g \in G}C_ge_g \in K[G]$ является центральным в групповой алгебре $K[G]$, то есть коммутирует с остальными элементами алгебры. Достаточно проверить это на базисных элементах вида $e_h$, где $h \in G$:
	\[(e_{h})^{-1}ae_h = e_{h^{-1}}ae_h = \sum_{g \in G}C_ge_{g^h} = \sum_{\widetilde g \in G}C_{\widetilde g}e_{\widetilde g} = a \ra ae_h = e_ha\]
	
	Аналогичным образом легко проверить, что оператор $A := \sum_{g \in G}C_gR(g) \in \mc L(V)$ является $G$-эквивариантным. Но тогда, по лемме Шура, $A = \lambda E$ для некоторого $\lambda \in K$, откуда:
	\[\lambda\dim{V} = \tr{A} = \tr\left(\,\sum_{g \in G}C_gR(g)\right) = 0\]
	
	Поскольку $\cha{K} = 0$, то $\lambda = 0$. В силу произвольности выбора неприводимого представления $R$ и полной приводимости любого представления группы $G$, для любого представления $\widetilde R$ оператор $\sum_{g \in G}C_g\widetilde R(g)$ "--- нулевой. В частности, это верно для регулярного представления группы $G$ в пространстве $\gl e_g \gr_{g \in G}$, тогда для любого $g \in G$ имеем:
	\[\left(\sum_{g \in G}C_gg\right)(e_e) = \sum_{g \in G}C_ge_g = 0\]
	
	Поскольку $(e_g)_{g \in G}$ "--- базис, то для всех $g \in G$ выполнено $C_g = 0$. Значит, $\Chi$ "--- нулевой характер. Получено противоречие.	
\end{proof}

\begin{corollary}
	Пусть $G$ "--- конечная группа, поле $K$ алгебраически замкнуто, и $\cha{K} = 0$. Тогда количество классов изоморфности неприводимых представлений группы $G$ в пространстве $V$ равно количеству классов сопряженности в $G$.
\end{corollary}

\begin{problem}
	Постройте таблицу характеров неприводимых представлений группы $\Z_n$ над алгебраически замкнутым полем $K$.
\end{problem}

\begin{solution}
	Пусть $\Z_n  = \gl a \gr$. Зафиксируем $\xi \in K$ "--- примитивный корень из единицы степени $n$. Тогда все неприводимые представления имеют место в одномерном пространстве $V$ и имеют следующий вид:
	\[ax = \xi^k x,~x \in V\]
	
	Здесь число $k \in \{0, \dotsc, n - 1\}$ "--- фиксированное для каждого из представлений. Обозначим соответствующие представления через $R_{\xi^k}$. Заметим также, что классы сопряженности в $\Z_n$ имеют вид $\{a^m\}$, где $m \in \{0, \dotsc, n - 1\}$, поскольку $\Z_n$ "--- абелева. Построим таблицу характеров неприводимых представлений:
	\begin{center}
		\begin{tabular}{r|c|c|c|c|c|c}
			            & $\{e\}$  & $\{a\}$  &  \dots   & $\{a^m\}$  &  \dots  & $\{a^{n-1}\}$  \\ 
			            \hline
			   $R_1$    &    1     &    1     &  \dots   &     1      &  \dots  & 1 \\
			  $R_\xi$   &    1     &  $\xi$   &  \dots   &  $\xi^m$   &  \dots  & $\xi^{n-1}$ \\
			 $\vdots$   & $\vdots$ & $\vdots$ & $\ddots$ & $\vdots $  & $\vdots$ & $\vdots$ \\
			$R_{\xi^k}$ &    1     & $\xi^k$  &  \dots   & $\xi^{km}$ &  \dots  &  $\xi^{k(n-1)}$ \\
			 $\vdots$   & $\vdots$ & $\vdots$ & $\ddots$ &  $\vdots$  & $\ddots$ & $\vdots$\\
		 $R_{\xi^{n-1}}$ &    $1$     & $\xi^{n-1}$  &  \dots   & $\xi^{(n-1)m}$ &  \dots  &  $\xi^{(n-1)^2}$ \\
		\end{tabular}
	\end{center}

	Можно проверить непосредственно, пользуясь не общей теоремой, доказанной в предположении, что $\cha{K} = 0$, а свойствами корней из единицы в поле $K$, что характеры таких представлений образуют ортонормированную систему.
\end{solution}