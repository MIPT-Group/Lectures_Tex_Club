\section{Характеры представлений}

\begin{definition}
	\textit{Характером} группы $G$ над $K$ называется функция $\Chi : G \to K$, постоянная на классах сопряженности. Все характеры группы $G$ над полем $K$ образуют линейное пространство над $K$ с естественными операциями сложения и умножения на скаляры из $K$.
\end{definition}

\begin{note}
	Условие из определения характера можно переписать следующими эквивалентными способами:
	\begin{align*}
		\Chi(g^{-1}hg) = \Chi(h),~g, h \in G\\
		\Chi(gh) = \Chi(hg),~g, h \in G
	\end{align*}
\end{note}

\begin{note}
	Пусть $G$ "--- конечная группа, $\cha K \nmid |G|$. Определим на пространстве $X_K(G)$ характеров группы над $K$ следующую билинейную форму:
	\[(\Chi_1, \Chi_2) := \frac1{|G|}\sum_{g \in G}\Chi_1(g)\Chi_2(g^{-1}),~\Chi_1, \Chi_2 \in X_K(G)\]
	
	Легко заметить, что полученная форма симметрична. Проверим, что она невырожденна, то есть имеет нулевое ядро. Действительно, пусть для некоторого $\Chi_0 \in X_G(G)$ и любого $\Chi \in X_K(G)$ выполнено $(\Chi_0, \Chi) = 0$. Тогда, определяя $\Chi$ как функцию, отличную от нуля и равную единице только на одном классе сопряженности $C_h := \{h^g : g \in G\}$, получим следующее:
	\[\sum_{h^g \in C_h}\Chi_0\left((h^{-1})^g\right) = \sum_{(h^{-1})^g \in C_{h^{-1}}}\Chi_0\left((h^{-1})^g\right) = |C_{h^{-1}}| \Chi_0(h^{-1}) = 0\]
	
	Поскольку $|C_{h^{-1}}| = |C_h| \mid |G|$, то $\cha K \nmid |C_{h^{-1}}|$, откуда $\Chi_0(h^{-1}) = 0$. В силу произвольности выбора класса сопряженности, $\Chi_0 = 0$.
\end{note}

\begin{definition}
	\textit{Характером представления} группы $G$ в пространстве $V$ называется функция $\Chi_V: G \to K$ следующего вида:
	\[\Chi_V(g) = \tr(g),~g \in G\]
\end{definition}

\begin{proposition}
	Пусть $V, W$ "--- пространства над полем $K$, группа $G$ имеет представление в $V$ и $W$. Тогда для произвольных характеров соответствующих представлений группы $G$ и произвольных элементов $g \in G$ выполнены следующие равенства:
	\begin{enumerate}
		\item $\Chi_{V \oplus W}(g) = \Chi_V(g) + \Chi_W(g)$
		
		\item Если $V$ "--- вполне приводимое пространство, а $U \le V$ "--- его $G$-инвариантное подпространство, то $\Chi_{V \bs U}(g) = \Chi_V(g) - \Chi_U(g)$
		
		\item $\Chi_{V \otimes W}(g) = \Chi_V(g)\Chi_W(g)$
		
		\item $\Chi_{V^*}(g) = \Chi_V(g^{-1})$
		
	\end{enumerate}
	
	\begin{proof}~
		\begin{enumerate}
			\item Матрица оператора $g$ в $V \oplus W$ имеет блочно-диагональный вид с блоками, являющимися матрицами оператора $g$ в $V$ и $W$.
			
			\item Выберем $\mc A$-инвариантное подпространство $W \le V$ такое, что $U \oplus V$. Зафиксируем базисы $e'$, $e''$ в пространствах $U$ и $W$, тогда их объединение $e$ является базисом в $V$. Пусть $g \leftrightarrow_{e'} A$, $g \leftrightarrow_{e''} B$, тогда:
			\[g \leftrightarrow_e \left(\begin{array}{@{}c|c@{}}
					A & *\\
					\hline
					0 & B
				\end{array}\right)\]
			
			Следовательно, $\Chi_V(g) = \Chi_{V \bs U}(g) + \Chi_U(g)$.
			
			\item Пусть $e := (e_i)$, $f := (f_j)$ "--- базисы в пространствах $V$, $W$. Если $g \leftrightarrow_{e} A$, $g \leftrightarrow_{f} B$, то тогда $g \leftrightarrow_{e \otimes f} A \otimes B$. Значит, выполнены следующие равенства:
			\[\Chi_{V \otimes W}(g) = \tr(A \otimes B) = \tr{A}\tr{B}=\Chi_V(g)\Chi_W(g)\]
			
			\item Зафиксируем двойственные базисы $(e_i)$ и $(\xi_j)$ в пространствах $V$ и $V^*$. Тогда:
			\[(g\xi_j)(e_i) = \xi_j(g^{-1} e_i)\]
			
			Из равенства выше следует требуемое.
		\end{enumerate}
	\end{proof}
\end{proposition}