\section{Гомоморфизмы представлений}

\begin{definition}
	\textit{Гомоморфизмом представлений} группы $G$ в пространствах $V$ и $W$ называется линейное отображение $\phi : V \to W$ такое, что для любого $g \in G$ выполнено $g\phi \equiv \phi g$, то есть для любого $x \in V$ выполнено $g(\phi(v)) = \phi(gv)$.
\end{definition}

\begin{proposition}
	Пусть $\phi : V \to W$, $\psi : W \to U$ "--- гомоморфизмы представлений группы $G$. Тогда $\psi\phi: V \to U$ "--- тоже гомоморфизм представлений.
\end{proposition}

\begin{proof}
	Достаточно заметить, что для любых $g \in G$ и $x \in V$ выполнены равенства $g(\psi(\phi(v))) = \psi(g(\phi(v))) = \psi(\phi(gv))$.
\end{proof}

\begin{note}
	Рассуждение выше может быть изображено на следующей коммутативной диаграмме:
	\[
	\begin{tikzcd}[row sep = large]
		V \arrow{r}{\phi} \arrow[swap]{d}{g} & W \arrow{r}{\phi} \arrow[swap]{d}{g} & U \arrow[swap]{d}{g}\\
		V \arrow{r}{\phi} & W \arrow{r}{\phi} & U
	\end{tikzcd}
	\]
\end{note}

\begin{definition}
	\textit{Изоморфизмом представлений} группы $G$ в пространствах $V$ и $W$ называется биективный гомоморфизм представлений $\phi : V \to W$. Если $W = V$ и рассматриваемые представления одинаковы, то такой изоморфизм называется \textit{автоморфизмом представлений}.
\end{definition}

\begin{example}
	Отображение $\id : V \to V$ является автоморфизмом любого представления $G$ в пространстве $V$.
\end{example}

\begin{note}
	Пусть $\phi : V \to W$ "--- изоморфизм представлений группы $G$ в пространствах $V$ и $W$. Легко видеть, что тогда $\phi^{-1} : W \to V$ "--- тоже является изоморфизмом представлений, поскольку для любого $g \in G$ выполнено $g\phi \equiv \phi g \lra g^{-1}\phi^{-1} \equiv \phi^{-1}g^{-1}$.
\end{note}

\begin{definition}
	\textit{Алгеброй} над полем $K$ называется линейное пространство $A$, на котором определена билинейная операция умножения $\cdot : A^2 \to A$. Алгебра называется:
	\begin{itemize}
		\item \textit{Ассоциативной}, если для любых $a, b, c \in A$ выполнено $(ab)c = a(bc)$
		\item \textit{Алгеброй с единицей}, если существует элемент $1 \in A$ такой, что для любого $a \in A$ выполнено $a1 = 1a = a$
		\item \textit{Коммутативной}, если для любых $a, b \in A$ выполнено $ab = ba$
	\end{itemize}
\end{definition}

\begin{definition}
	\textit{Алгеброй Ли} называется линейное пространство $A$, на котором определена билинейная операция \textit{коммутирования} $[\cdot,\cdot] : A^2 \to A$, удовлетворяющая следующим условиям:
	\begin{enumerate}
		\item Для любых $x, y \in A$ выполнено $[x, y] = -[y, x]$
		\item Для любых $x, y, z \in A$ выполнено $[x, [y, z]] + [y, [z, x]] + [z, [x, y]] = 0$
	\end{enumerate}
\end{definition}

\begin{note}
	Зафиксируем базис $(e_1, \dotsc, e_n)$ в алгебре Ли $A$. Если $x, y \in A$ имеют координаты $(x_1, \dotsc, x_n)^T$ и $(y_1, \dotsc, y_n)^T$, то их коммутатор $[x, y]$ можно вычислить по следующей формуле:
	\[[x, y] = \left[\sum_{i=1}^nx_ie_i, \sum_{j=1}^ny_je_j\right] = \sum_{i=1}^n\sum_{j=1}^nx_iy_j[e_i, e_j]\]
\end{note}

\begin{proposition}
	Пусть $A$ "--- ассоциативная алгебра. Определим операцию коммутирования следующим образом:
	\[[x, y] := xy - yx,~x, y \in A\]
	
	Тогда с данной операцией алгебра $A$ образует алгебру Ли.
\end{proposition}

\begin{proof}
	Первое свойство коммутирования тривиально, докажем второе. Зафиксируем $x, y, z \in A$, тогда:
	\[[x, [y, z]] + [y, [z, x]] + [z, [x, y]] = [x, yz - zy] + [y, zx - xz] + [z, xy - yx] \stackrel{*}{=} 0\]
	
	Равенство $(*)$ выше получается раскрытием всех коммутаторов по определению.
\end{proof}

\begin{definition}
	\textit{Представлением ассоциативной алгебры $A$ в пространстве $V$} называется гомоморфизм алгебр $\phi : A \to \mc L(V)$. Представлением алгебры Ли $\mathfrak G$ в пространстве $V$ называется гомоморфизм алгебр Ли $\phi : \mathfrak{G} \to \mathfrak{L}(V)$, где под $V$ понимается алгебра Ли, построенная по алгебре $V$.
\end{definition}