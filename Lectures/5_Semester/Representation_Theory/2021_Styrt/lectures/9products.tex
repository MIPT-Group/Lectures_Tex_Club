\section{Представления произведений групп}

\begin{proposition}
	Пусть представление группы $G$ в пространстве $V$ вполне приводимо, $W \le V$ "--- изотипная компонента представления. Тогда $W$ инвариантно относительно любого оператора из $\mc L^G(V)$.
\end{proposition}

\begin{proof}
	Зафиксируем разложение $V = \bigoplus_{i=1}^n V_i$, индексы $i, j \in \{1, \dotsc, n\}$ и оператор $A \in \mc L^G(V)$. Обозначим через $\pi_j$ проектор на подпространство $V_j$ и рассмотрим следующую цепочку гомоморфизмов:
	\[V_i \xhookrightarrow[]{\id} V \xrightarrow{A} V \xrightarrow{\pi_j} V_j\]
	
	Если $V_i \not\cong_G V_j$, то $\pi_j(A(V_i)) = \{0\}$ из неприводимости. Тогда, в силу произвольности выбора индексов $i$ и $j$, для каждой изотипной компоненты $W$ выполнено включение $A(W) \le W$.
\end{proof}

\begin{theorem}
	Пусть $G, H$ "--- конечные группы, поле $K$ алгебраически замкнуто, причем $\cha{K} \nmid |G|, |H|$. Тогда неприводимые представления группы $G \times H$ "--- это тензорные произведения всевозможных неприводимых представлений групп $G$ и $H$.
\end{theorem}

\begin{proof}
	Рассмотрим произвольное неприводимое представление группы $G \times H$ в пространстве $L$. Представление группы $G \cong G \times \{e_H\} \le G \times H$ в пространстве $L$ вполне приводимо. Поскольку подгруппа $H \cong \{e_G\} \times H \le G \times H$ коммутирует с $G$, то операторы из $H$ являются эндоморфизмами представления группы $G$ в $L$. Значит, они сохраняют изотипные компоненты представления группы $G$ в пространстве $L$.
	
	Итак, любая изотипная компонента представления группы $G$ в $L$ инвариантна относительно $G \times H$. Но пространство $L$ неприводимо, поэтому оно совпадает с единственной изотипной компонентой. Тогда для некоторого неприводимого относительно $G$ подпространства $V \le L$ и некоторого $n \in \N$ выполнено равенство:
	\[L = V^{\oplus n} =: V_1 \oplus \dotsb \oplus V_n\]
	
	Зафиксируем произвольный элемент $h \in H$ и индексы $i, j \in \{1, \dotsc, n\}$. Обозначим через $\pi_j$ проектор на подпространство $V_j$ и рассмотрим следующую цепочку гомоморфизмов:
	\[V_i \xhookrightarrow[]{\id} V \xrightarrow{h} V \xrightarrow{\pi_j} V_j\]
	
	Поскольку $V_i \cong_G V \cong_G V_j$, то по лемме Шура для некоторого скаляра $a_{ij} \in K$ выполнено равенство $\pi_j \circ h \circ \id_i = a_{ij}E$. Положим $A := (a_{ij}) \in M_n(K)$, тогда для любого набора $(v_1, \dotsc, v_n) \in L$ выполнено следующее:
	\[h(v_1, \dotsc, v_n) = \left(\sum_{i = 1}^na_{i1}v_i, \dotsc, \sum_{i = 1}^na_{in}v_i\right) = (v_1, \dotsc, v_n)A\]
	
	Рассмотрим пространство $W := K^n$ с базисом $(e_1, \dotsc, e_n)$, тогда, отождествляя выражения вида $v_1 \otimes e_1 + \dotsb + v_n \otimes e_n$ с наборами $(v_1, \dotsc, v_n)$ при $v_1, \dotsc, v_n \in V$, получим:
	\[V \otimes W = \{v_1 \otimes e_1 + \dotsb + v_n \otimes e_n : v_1, \dotsc, v_n \in V\} = L\]
	
	Оператор $h$ действует на произвольный элемент $v \otimes e_i \in V \otimes W$ следующим образом:
	\[h(v \otimes e_i) = h(0, \dotsc, 0, v_i, 0, \dotsc, 0) = \left(a_{i1}v, \dotsc, a_{in}v\right) = \sum_{j=1}^na_{ij}v\otimes e_{j} = v \otimes \left(\sum_{j=1}^na_{ij} e_{j}\right)\]
	
	Если обозначить через $B \in \mc L(W)$ оператор с матрицей $A^T$, то $h(v \otimes e_i) = v \otimes B(e_i)$. Кроме того, $g(v \otimes e_i) = (gv) \otimes e_i$. Тогда, по линейности, для любого элемента $v \otimes w \in V \otimes W$ выполнены следующие равенства:
	\begin{gather*}
		g(v \otimes w) = (gv) \otimes w\\
		h(v \otimes w) = v \otimes (B(w))
	\end{gather*}

	Легко заметить, что сопоставление $h \mapsto B$ задает представление группы $H$ в пространстве $W$, поскольку $e \mapsto E$ и если $h_1 \mapsto B_1$, $h_2 \mapsto B_2$, то $h_1h_2 \mapsto B_1B_2$. Значит, представление группы $G \times H$ в пространстве $L = V \otimes W$ является тензорным произведением представлений группы $G$ в $V$ и группы $H$ в $W$. Кроме того, представление группы $H$ неприводимо. Действительно, если $U \le W$ "--- $H$-инвариантное, то $V \otimes U < V \otimes W$ "--- $G \times H$-инвариантное, откуда либо $U = \{0\}$, либо $U = W$.
	
	Пусть теперь представления групп $G$ и $H$ в пространствах $V$ и $W$ неприводимы. Покажем, что тогда представление группы $G \times H$ в пространстве $V \otimes W$ неприводимо. Положим $m := \dim{W}$, тогда разложение пространства $V \otimes W$ на неприводимые относительно $G \le G \times H$ подпространства имеет вид:
	\[V \otimes W = V^{\oplus m}\]
	
	Пусть $\{0\} < U \le V \otimes W$ "--- неприводимое относительно $G \times H$ подпространство. Тогда, в силу уже доказанного, оно имеет вид $U = V' \otimes W'$, где $V'$, $W'$ "--- пространства, неприводимые относительно групп $G$ и $H$ соответственно. Положим $m' := \dim{W'}$, тогда разложение пространства $U$ на неприводимые относительно $G \le G \times H$ подпространства имеет вид:
	\[U = (V')^{\oplus m'}\]
	
	Тогда, в силу неприводимости пространства $V'$, $V' \cong_G V$, откуда $\dim{V} = \dim{V'}$. Аналогично, $\dim{W} = \dim{W'}$, поэтому $U = V \otimes W$. Значит, в силу произвольности выбора подпространства $U$, пространство $V \otimes W$ неприводимо.
\end{proof}

\begin{theorem}
	Пусть $G, H$ "--- конечные группы, поле $K$ алгебраически замкнуто, причем $\cha{K} \nmid |G|, |H|$. Тогда для любых неприводимых представлений группы $G$ в пространствах $V_1, V_2$ и группы $H$ в пространствах $W_1, W_2$ выполнено:
	\[V_1 \otimes W_1 \cong_{G \otimes H} V_2 \otimes W_2 \lra V_1 \cong_G V_2,~ W_1\cong_H W_2\]
\end{theorem}

\begin{proof}
	Нетривиально только доказательство $\ra$. Положим $m_i := \dim{W_i}$ при $i \in \{1, 2\}$, тогда разложение пространств $V_1 \otimes W_1 \cong_G V_2 \otimes W_2$ на неприводимые относительно $G \le G \times H$ подпространства имеют вид:
	\[V_1^{\oplus m_1} = V_1 \otimes W_1 \cong_G V_2 \otimes W_2 = V_2^{\oplus m_2}\]
	
	Но разложение единственно, поэтому $V_1 \cong_G V_2$. Аналогично, $W_1 \cong_G W_2$
\end{proof}