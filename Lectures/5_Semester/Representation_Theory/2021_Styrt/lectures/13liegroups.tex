\section{Группы Ли}

\begin{definition}
	\textit{Группой Ли} называется группа $G$ со структурой гладкого многообразия, в соответствии с которой гладкими являются следующие отображения:
	\begin{itemize}
		\item $\mu: G^2 \to G$ "--- умножение
		\item $\tau: G \to G$ "--- взятие обратного элемента
	\end{itemize}
\end{definition}

\begin{note}
	Если $X, Y$ "--- гладкие многообразия, то $X \times Y$ тоже является гладким многообразием. Если $\{(U_\alpha, \phi_\alpha)\}_{\alpha \in \mf A}$ и $\{(V_\beta, \psi_\beta)\}_{\beta \in \mf B}$ "--- системы локальных карт на $X$ и $Y$, тогда топология пространства $X \times Y$ порождается декартовыми произведениями открытых множеств, а система локальных карт имеет вид $\{(U_\alpha \times V_\beta, \phi_\alpha \times \psi_\beta)\}_{\alpha \in \mf A, \beta \in \mf B}$.
\end{note}

\begin{example}
	Пусть $A$ "--- конечномерная ассоциативная алгебра с единицей над $\R$. Можно считать, что $A = \R^n$, и топология на $A$ задается как на $\R^n$ с произвольной нормой, поскольку все нормы на $\R^n$ эквивалентны. Покажем, что группа $A^*$ является группой Ли. 
	
	Сначала заметим, что элемент $a \in A$ обратим слева $\ra$ $a$ не является левым делителем нуля $\lra$ оператор левого сдвига $l_a \in \mc L(A)$ инъективен $\lra$ оператор $l_a$ является невырожденным $\lra aA = A \hm\lra a \in A$ обратим справа. Аналогично доказывается, элемент $a \in A$ обратим справа $\ra$ $a$ обратим слева. Тогда, в частности, $A^* = \{a \in A: \det{l_a} \ne 0\}$. Но  функция $\det{l_a}$ непрерывна на $A$, поэтому множество $A^* \subset A$ является открытым в $\R^n$, тогда системой карт на $A^*$ можно объявить $\{U(a), \id_{U(a)}\}_{a \in A^*}$.
	
	Гладкость отображения $\mu$ очевидна, поэтому остается проверить гладкость отображения $\tau$. Сначала докажем его гладкость в точке $1 \in A$. Для этого нам потребуется норма $\| \cdot \|$ на $A$ такая, что для любых $x, y \in A$ выполнено $\|xy\| \le \|x\| \|y\|$. Зафиксируем евклидову норму $\| \cdot \|_2$ на $A$. Поскольку сфера $S := \{a \in A: \|a\|_2 = 1\}$ "--- компакт, то функция $f(xy) := \|xy\|_2$ ограничена сверху на $S \times S$ константой $C$, тогда для любых $x, y \in A$ выполнено $\|xy\|_2 \le C\|x\|_2\|y\|_2$. Тогда норма $\|\cdot\| := \frac{1}{C}\|\cdot\|_2$ является искомой. Зафиксируем теперь $c \in A^*$ такой, что $\|c\| < 1$. Тогда, в силу выбора нормы, ряд $\sum_{n=0}^\infty c^n$ сходится, причем выполнены равенства:
	\[(1-c)\left(\sum_{n=0}^\infty c^n\right) = (1-c)\lim_{N \to \infty}\left(\sum_{n=0}^Nc^n\right) = \lim_{N \to \infty}\left(1 - c^N\right) = 1\]
	
	Значит, элемент $1 - c$ обратим и имеет обратный элемент $\sum_{n=0}^\infty c^n$. Следовательно, $\tau$ в окрестности единицы представима следующим рядом:
	\[\tau(a) = \sum_{n=0}^\infty (-1)^n(a - 1)^n,~\|a - 1\| < 1\]
	
	Из этого следует гладкость отображения $\tau$ в точке $1$. Теперь зафиксируем произвольный элемент $a \in A^*$, тогда отображение $\tau$ представимо в виде $\tau = r_{a^{-1}} \circ \tau \circ l_{a^{-1}}$. Отображения $r_{a^{-1}}$ и $l_{a^{-1}}$ "--- это невырожденные линейные операторы, поэтому они являются гладкими на $A$, а отображение $\tau$ является гладким в единице, тогда, поскольку $l_{a^{-1}}(a) = 1$, отображение $\tau$ также является гладким в точке $a$.
\end{example}

\begin{note}
	Частный случай примера выше "--- это группа $\GL(V)$ для конечномерного линейного пространства $V$. Кроме того, более простым примером группы Ли является само пространство $V$ над $\R$ или $\Cm$ как абелева группа по сложению.
\end{note}