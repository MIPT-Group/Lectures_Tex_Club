\section{Модули над кольцами}

\begin{definition}
	Пусть $A$ "--- ассоциативное кольцо с единицей. \textit{Модулем} над кольцом $A$ называется абелева группа $M$, на которой задано действие кольца $A$ такое, что каждый элемент $a \in A$ действует на $M$ как гомоморфизм, то есть для любых $a \in A$ и $x, y \in M$ выполнено $a(x + y) = ax + ay$.
\end{definition}

\begin{note}
	Напомним, какими свойствами по определению обладает действие кольца на множестве $M$:
	\begin{enumerate}
		\item Для любых $a, b \in A$ и $m \in M$ выполнено $(a + b)m = am + bm$
		\item Для любых $a, b \in A$ и $m \in M$ выполнено $(ab)m = a(bm)$
		\item Для любого $m \in M$ выполнено $1m = m$
	\end{enumerate}
	
	Кроме того, имеет место эквивалентное определение, согласно которому действием кольца на модуле $M$ называется гомоморфизм $A \to \End{M}$.
\end{note}

\begin{example}
	Рассмотрим несколько примеров модулей:
	\begin{enumerate}
		\item Пусть $A$ "--- поле, $M$ "--- линейное пространство над $A$, тогда действие на модуле $M$ можно задать следующим образом:
		\[am := a\cdot m,~a \in A, m \in M\]
		\item Пусть $A$ "--- кольцо, $M \supset A$ "--- тоже кольцо, тогда можно рассмотреть действие $A$ на $M$ себе левыми сдвигами:
		\[am := a\cdot m,~a\in A, m \in M\]
	\end{enumerate}
\end{example}

\begin{definition}
	Пусть $M$ "--- модуль над кольцом $A$. Подгруппа $N \le M$ называется \textit{подмодулем} модуля $M$, если для любого $a \in A$ выполнено $aN \subset N$.
\end{definition}

\begin{definition}
	Пусть $M$ "--- модуль над кольцом $A$. \textit{Фактормодулем} модуля $M$ по подмодулю $N \le M$ называется факторгруппа $M / N$, на которой действие кольца $A$ задано следующим образом:
	\[a(m + N) := am + N\]
\end{definition}

\begin{note}
	Как и в случае факторпредставления, следует проверить корректность определения фактормодуля, и проверка производится аналогично.
\end{note}

\begin{definition}
	Пусть $M_1, M_2$ "--- модули над кольцом $A$. \textit{Прямой суммой модулей} $M_1$, $M_2$ называется группа $M_1 \oplus M_2$, на которой действие кольца $A$ задано следующим образом:
	\[a(m_1, m_2) := (am_1, am_2),~a \in A, (m_1, m_2) \in M_1 \oplus M_2\]
\end{definition}

\begin{note}
	Нетрудно проверить, что сумма $N_1 + N_2$ подмодулей $N_1, N_2 \le M$ как подгрупп образует подмодуль в $M$.
\end{note}

\begin{definition}
	Модуль $M$ над кольцом $A$ называется \textit{простым}, если он нетривиален и не существует модуля $N$ такого, что $\{0\} < N < M$.
\end{definition}

\begin{definition}
	\textit{Гомоморфизмом модулей} $M$ и $N$ над кольцом $A$ называется гомоморфизм групп $\phi : M \to N$ такой, что для любого $a \in A$ выполнено $a\circ\phi = \phi\circ a$, то есть для любого $m \in V$ выполнено $a(\phi(m)) = \phi(am)$.
\end{definition}

\begin{note}
	Аналогично случаю представлений, имеют место следующие свойства:
	\begin{itemize}
		\item Если $\phi, \psi: M \to N$ "--- гомоморфизмы модулей над кольцом $A$, то $\phi + \psi$ "--- тоже
		\item Если $\phi: M \to N$, $\psi: N \to K$ "--- гомоморфизмы модулей над кольцом $A$, то отображение $\psi \circ \phi$ "--- тоже
	\end{itemize}
\end{note}

\begin{proposition}
	Пусть $\phi : M_1 \to M_2$ "--- гомоморфизм модулей над кольцом $A$. Тогда:
	\begin{enumerate}
		\item Если $N_1 \le M_1$ "--- подмодуль, то $\phi(N_1)$ "--- тоже
		\item Если $N_2 \le M_2$ "--- подмодуль, то $\phi^{-1}(N_2)$ "--- тоже
	\end{enumerate}
\end{proposition}

\begin{proof}~
	\begin{enumerate}
		\item Заметим, что для любого элемента $a \in A$ выполнены следующие равенства:
		\[a(\phi(N_1)) = \phi(aN_1) \le \phi(N_1)\]
		\item Рассмотрим произвольный элемент $a \in A$. Для каждого элемента $m_1 \in \phi^{-1}(N_2)$ положим $m_2 := \phi(m_1) \in N_2$ и заметим следующее:
		\[\phi(am_1) = a(\phi(m_1)) = am_2 \in N_2\]
		
		Значит, $am_1 \in \phi^{-1}(N_2)$, откуда $a(\phi^{-1}(N_2)) \le \phi^{-1}(N_2)$.\qedhere
	\end{enumerate}
\end{proof}

\begin{definition}
	Гомоморфизм $\phi$ модулей $M, N$ над кольцом $A$ называется:
	\begin{itemize}
		\item \textit{Эндоморфизмом}, если $M = N$ и действия $A$ на этих модулях совпадают
		\item \textit{Изоморфизмом}, если $\phi$ биективен
		\item \textit{Автоморфизмом}, если $\phi$ "--- эндоморфизм и изоморфизм
	\end{itemize}
\end{definition}

\begin{definition}
	Модули $M$ и $N$ над кольцом $A$ называются \textit{изоморфными}, если существует изоморфизм $\phi : M \to N$. Обозначение "--- $M \cong N$.
\end{definition}

\begin{theorem}[основная теорема о гомоморфизме]
	Пусть $\phi: M \to N$ "--- гомоморфизм модулей над кольцом $A$. Тогда $\im\phi \cong M / \ke\phi$.
\end{theorem}

\begin{proof}
	Как известно, сопоставление $\psi$ вида $m + \ke\phi \mapsto \phi(m)$ осуществляет изоморфизм групп пространств $M / \ke\phi \cong \im\phi$. Остается проверить, что $\psi \circ a = a \circ \psi$ для любого $g \in G$. Зафиксируем $m \in M$ и рассмотрим следующую диаграмму:
	\[
	\begin{tikzcd}[row sep = large]
		m + \ke\phi \arrow[swap]{d}{a} \arrow{r}{\phi} & \phi(m) \arrow{d}{a}\\
		am + \ke\phi \arrow{r}{\phi} & a(\phi(m)) = \phi(am)
	\end{tikzcd}
	\]
	
	Как видно из диаграммы, получено требуемое.
\end{proof}

\begin{definition}
	Пусть $M$ "--- модуль над кольцом $A$, $S \subset M$. \textit{Подмодулем, порожденным множеством $S$,} называется наименьший по включению подмодуль $N$ такой, что $N \supset S$. Обозначение "--- $\gl S \gr$
\end{definition}

\begin{note}
	Подмодуль, порожденный множеством $S$, существует, поскольку его можно задать следующим образом:
	\[\gl S \gr = \bigcap_{N \le M : N \supset S}N\]
	
	Кроме того, имеет место следующее равенство:
	\[\gl S \gr = \{a_1s_1 + \dotsb + a_ks_k : a_1, \dotsc, a_k \in A, s_1, \dotsc, s_k \in S\}\]
	
	Действительно, если обозначить правую часть равенства выше через $K$, то множество $K$ содержится в любом модуле, содержащем $S$, и при этом само образует подмодуль.
\end{note}

\begin{definition}
	Модуль $M$ над кольцом $A$ называется \textit{конечнопорожденным}, если существует конечное множество $S \subset M$ такое, что $\gl S \gr = M$.
\end{definition}

\begin{definition}
	\textit{Свободным модулем} над кольцом $A$ называется модуль $A^{\oplus n}$, на котором $A$ действует левыми сдвигами.
\end{definition}

\begin{proposition}
	Всевозможные конечнопорожденные модули над кольцом $A$ "--- это фактормодули свободных модулей над кольцом $A$.
\end{proposition}

\begin{proof}
	Пусть $S = \{s_1, \dotsc, s_n\} \subset M$. Рассмотрим сопоставление следующего вида:
	\[(a_1, \dotsc, a_n) \mapsto a_1s_1 + \dotsb + a_ns_n\]
	
	Это сопоставление осуществляет гомоморфизм модулей $A^{\oplus n}$ и $\gl S \gr$, поэтому $\gl S \gr$ изоморфен фактормодулю группы $A^{\oplus n}$ по ядру данного гомоморфизма. С другой стороны, очевидно, что любой фактормодуль свободной группы $A^{\oplus n}$ является ее гомоморфным образом и порождается образами элементов $(1, 0, \dotsc, 0), \dotsc, (0, \dotsc, 0, 1) \in A^{\oplus n}$.
\end{proof}

\begin{definition}
	Пусть $A$ и $B \supset A$ "--- целостные кольца. Элемент $x \in B$ называется \textit{целым} над $A$, если существует приведенный многочлен $f \in A[t]$, то есть многочлен со старшим коэффициентом $1$, такой, что $f(x) = 0$.
\end{definition}

\begin{theorem}
	Пусть $A$ и $B \supset A$ "--- целостные кольца. Элемент $x \in B$ является целым над $A$ $\lra$ расширение $A[x]$ образует конечнопорожденный модуль над $A$.
\end{theorem}

\begin{proof}~
	\begin{enumerate}
		\item[$\ra$] По условию, для некоторого $n \in \N$ и элементов $a_0, \dotsc, a_{n - 1} \in A$ выполнено равенство $x^n + a_{n-1}x^{n-1} + \dotsb + a_0 = 0$. Тогда, по индукции, любой элемент вида $x^{n + m}$ при $m \in \N \cup \{0\}$ линейно выражается через $1, x, \dotsc, x^{n- 1}$, поэтому $A[x] = \gl1, x, \dotsc, x^{n- 1}\gr$.
		\item[$\la$]To be continued\dots\qedhere
	\end{enumerate}
\end{proof}

\begin{corollary}
	Пусть $A$ и $B \supset A$ "--- целостные кольца. Тогда целые над $A$ элементы из $B$ образуют подкольцо в $B$.
\end{corollary}