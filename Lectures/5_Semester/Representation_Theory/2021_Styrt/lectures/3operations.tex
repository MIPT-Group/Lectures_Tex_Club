\section{Операции над представлениями}

\textbf{Здесь и далее} будем обозначать через $V_1, V_2$ линейные пространства над полем $K$. Будем считать, что пространства $V_1, V_2$ конечномерны.

\begin{definition}
	Пусть $R_1, R_2$ "--- представления группы $G$ в $V_1, V_2$. \textit{Прямой суммой представлений} $R_1, R_2$ называется представление $R_1 \oplus R_2$ группы $G$ в пространстве $V_1 \oplus V_2$ следующего вида:
	\[g(v_1, v_2) := (gv_1, gv_2),~g \in G, v_1 \in V_1, v_2 \in V_2\]
\end{definition}

\begin{note}
	Напомним, что \textit{тензорным произведением пространств $V_1, V_2$} называется пространство $T$ над $K$ вместе с билинейным отображением $f: V_1 \times V_2 \to T$ таким, что для любых базисов $(e_i)$ в $U$ и $(f_j)$ в $V$ векторы $t_{ij} := f({e_i}, {f_j})$ образуют базис в $T$.
	
	В предыдущих курсах было доказано, что тензорное произведение существует и единственно с точностью до изоморфизма. Этот объект обозначается через $V_1 \otimes V_2$, а векторы вида $f(v_1, v_2)$, где $v_1 \in V_1, v_2 \in V_2$, --- через $v_1 \otimes v_2$. Такие векторы называются \textit{разложимыми тензорами}.
	
	Известно также, что для любого билинейного отображения $b: V_1 \times V_2 \rightarrow V$ существует единственное линейное отображение $\phi_b: V_1 \otimes V_2 \rightarrow V$ такое, что $\phi_b \circ f = b$, то есть имеет место следующая коммутативная диаграмма:
	\[
	\begin{tikzcd}[row sep = large]
		V_1 \arrow[swap]{dr}{f} \times V_2 \arrow[]{rr}{b} && V\\
		& V_1 \otimes V_2 \arrow[swap]{ur}{\phi_b}&
	\end{tikzcd}
	\]
	
	Более того, сопоставление $b \mapsto \phi_b$ осуществляет изоморфизм линейных пространств $\mathcal{B}(V_1, V_2; V)$ и $\mathcal{L}(V_1 \otimes V_2; V)$.
\end{note}

\begin{definition}
	Пусть $R_1, R_2$ "--- представления группы $G$ в $V_1, V_2$. \textit{Тензорным произведением представлений} $R_1, R_2$ называется представление $R_1 \otimes R_2$ группы $G$ в пространстве $V_1 \otimes V_2$, заданное на разложимых тензорах пространства $V_1 \otimes V_2$ следующим образом:
	\[g(v_1 \otimes v_2) := gv_1 \otimes gv_2,~g \in G, v_1 \in V_1, v_2 \in V_2\]
\end{definition}

\begin{definition}
	Пусть $R$ "--- представление группы $G$ в $V$. \textit{Представлением, двойственным к $R$,} называется представление $R^*$ группы $G$ в пространстве $V^*$ следующего вида:
	\[g^*\xi := \xi\circ g^{-1},~g \in G, \xi \in V^*\]
\end{definition}

\begin{note}
	Двойственное представление $R^*$ действительно является представлением, поскольку для любых элементов $g, h \in G$ и функционала $\xi \in V^*$ выполнено следующее:
	\[(gh)^*\xi = \xi \circ (gh)^{-1} = \xi \circ h^{-1} \circ g^{-1} = g^*(\xi \circ h^{-1})= g^*(h^*\xi)\]
\end{note}

\begin{definition}
	Пусть $R$ "--- представление группы $G$ в $V_1$. \textit{Представлением $R(V_1; V_2)$} называется представление группы $G$ в пространстве $\mc L(V_1; V_2)$ следующего вида:
	\[gA := gAg^{-1},~g \in G, A \in \mc L(V_1; V_2)\]
\end{definition}

\begin{proposition}
	Пусть $R_1, R_2$ "--- представления группы $G$ в $V_1, V_2$. Тогда имеет место изоморфизм представлений $V_1^* \otimes V_2 \gcong \mc L(V_1; V_2)$, осуществляемый сопоставлением вида $\xi_1 \otimes x_2 \mapsto \phi(\xi_1 \otimes x_2)$, где:
	\[\phi(\xi_1 \otimes x_2)(x_1) := \xi_1(x_1)x_2,~x_1 \in V_1\]
\end{proposition}

\begin{proof}
	Известно, что описанное выше сопоставление осуществляет изоморфизм $V_1^* \otimes V_2 \cong \mc L(V_1; V_2)$. Проверим, что для каждого $g \in G$ выполнено $g \circ \phi = \phi \circ g$. В силу линейности, достаточно проверить это равенство на разложимых тензорах. Зафиксируем произвольные $g \in G$, $\xi_1 \in V_1^*$, $x_2 \in V_2$ и $x_1 \in V_1$, тогда, с одной стороны:
	\[\phi(g(\xi_1 \otimes x_2))(x_1) = \phi(g^*\xi_1 \otimes gx_2)(x_1) = \phi((\xi_1 \circ g^{-1}) \otimes gx_2)(x_1) = (\xi_1(g^{-1}x_1))gx_2\]
	
	С другой стороны, выполнено следующее:
	\[g(\phi(\xi_1 \otimes x_2))(x_1) = (g\phi(\xi_1 \otimes x_2)g^{-1})(x_1) = (\xi_1(g^{-1}x_1))gx_2\]
	
	Таким образом, $g \circ \phi = \phi \circ g$, из чего следует требуемое в силу произвольности выбора элемента $g \in G$.
\end{proof}

\begin{proposition}
	Пусть $\phi: V_1 \to V_2$ "--- гомоморфизм представлений группы $G$ в пространствах $V_1$ и $V_2$. Тогда:
	\begin{enumerate}
		\item Если $U_1 \le V_1$ "--- инвариантное, то $\phi(U_1) \le V_2$ "--- тоже инвариантное
		\item Если $U_2 \le V_2$ "--- инвариантное, то $\phi^{-1}(U_2) \le V_1$ "--- тоже инвариантное
	\end{enumerate}
\end{proposition}

\begin{proof}~
	\begin{enumerate}
		\item Заметим, что для любого элемента $g \in G$ выполнены следующие равенства:
		\[g(\phi(U_1)) = \phi(gU_1) = \phi(U_1)\]
		\item Рассмотрим произвольный элемент $g \in G$. Для любого вектора $u_1 \in \phi^{-1}(U_2)$ положим $u_2 := \phi(u_1) \in U_2$ и заметим следующее:
		\[\phi(gu_1) = g(\phi(u_1)) = gu_2 \in U_2\]
		
		Значит, $gu_1 \in U_2$, откуда $g(\phi^{-1}(U_2)) \subset \phi^{-1}(U_2)$.\qedhere
	\end{enumerate}
\end{proof}

\begin{corollary}
	Пусть $\phi: V_1 \to V_2$ "--- гомоморфизм представлений группы $G$ в пространствах $V_1$ и $V_2$. Тогда $\im\phi \le V_2$ и $\ke\phi \le V_1$ инвариантны относительно соответствующих представлений.
\end{corollary}

\begin{proof}
	Достаточно заметить, что $\im\phi = \phi(V_1)$, $\ke\phi = \phi^{-1}(\{0\})$, а подпространства $V_1 \le V_1$ и $\{0\} \le V_2$ всегда инвариантны.
\end{proof}

\begin{theorem}[основная теорема о гомоморфизме]
	Пусть $\phi: V_1 \to V_2$ "--- гомоморфизм представлений группы $G$ в пространствах $V_1$ и $V_2$. Тогда $\im\phi \gcong V_1 / \ke\phi$.
\end{theorem}

\begin{proof}
	Как известно, сопоставление $\psi$ вида $x_1 + \ke\phi \mapsto \phi(x_1)$ осуществляет изоморфизм линейных пространств $V_1 / \ke\phi \cong \im\phi$. Остается проверить, что $\psi \circ g = g \circ \psi$ для любого $g \in G$. Зафиксируем $v_1 \in V_1$ и рассмотрим следующую диаграмму:
	\[
	\begin{tikzcd}[row sep = large]
		v_1 + \ke\phi \arrow[swap]{d}{g} \arrow{r}{\phi} & \phi(v_1) \arrow{d}{g}\\
		gv_1 + \ke\phi \arrow{r}{\phi} & g(\phi(v_1)) = \phi(gv_1)
	\end{tikzcd}
	\]
	
	Как видно из диаграммы, получено требуемое.
\end{proof}

\begin{problem}
	Опишите все неприводимые представления группы $\Z_n$ в пространстве $V$ над алгебраически замкнутым полем $K$.
\end{problem}

\begin{solution}
	 Группа $\Z_n$ "--- абелева, поэтому $\dim{V} = 1$ в силу неприводимости. Пусть $\Z_n = \gl a\gr$, $V = \gl v \gr$, и $av = \lambda v$ для некоторого $\lambda \in K$, тогда $v = a^nv = \lambda^n v$, откуда $\lambda = \sqrt[n]{1}$. Проверим, что все $n$ представлений такого вида попарно неизоморфны. Пусть представления вида $av = \lambda_1 v$ и $av = \lambda_2 v$ изоморфны для некоторых $\lambda_1, \lambda_2 \in K$. Обозначим соответствующий изоморфизм через $\phi$, тогда:
	\[\lambda_1\phi(v) = \phi(av) = a(\phi(v)) = \lambda_2\phi(v)\]
	
	Поскольку $\phi(v) \ne 0$, получим, что $\lambda_1 = \lambda_2$.
\end{solution}

\begin{problem}
	Опишите все неприводимые представления группы $S_3$ в пространстве $V$ над алгебраически замкнутым полем $K$.
\end{problem}

\begin{solution}
	Заметим, что $S_3 = \gl a, b \mid a^3 = b^2 = e, ab = ba^2\gr$. В силу алгебраической замкнутости, у оператора $a$ есть собственный вектор $v \in V$ с собственным значением $\lambda \in K$. Положим $u := bv$, тогда выполнены следующие равенства:
	\begin{align*}
		av &= \lambda v
		\\
		bu &= v
		\\
		au &= (ba^2b)u = \lambda^2u
	\end{align*}
	
	 Значит, пространство $U := \gl v, u\gr$ инвариантно относительно $S_3$, и $v, u$ "--- собственные векторы для $a$ с собственными значениями $\lambda, \lambda^2$, причем $\lambda = \sqrt[3]{1}$. Возможны несколько случаев:
	 \begin{itemize}
	 	\item Если $\dim{U} = 1$, то $u = \alpha v$ для некоторого $\alpha \in K$, тогда $\lambda = \lambda^2 \ra \lambda = 1$, откуда $a = \id$. Кроме того, выполнено следующее:
	 	\[v = b^2v = \alpha b^2u = \alpha u = \alpha^2 v \ra \alpha = \pm1\]
	 	
	 	Значит, либо $b = \id$, либо $b = -\id$. Таким образом, либо представление тождественно, либо оно имеет следующий вид:
	 	\[\sigma x := (\sgn\sigma)x,~\sigma \in S_3, x \in U\]
	 	
	 	\item Если $\dim{U} = 2$, то $\lambda \ne 1$ в силу неприводимости, иначе, выбрав собственный вектор оператора $b$, получим одномерное инвариантное подпространство. Легко проверить, что все представления с $\lambda = \omega$ или $\lambda = \omega^2$ попарно изоморфны. Предъявим любое из них, например, представление, переставляющее координаты векторов в $U = \{(x_1, x_2, x_3)^T \in K^3: x_1 + x_2 + x_3 = 0\}$.
	 \end{itemize}
\end{solution}

\begin{proposition}
	Пусть $R_1, R_2$ "--- представления группы $G$ в пространствах $V_1, V_2$. Тогда $(V_1 \oplus V_2) / V_1 \gcong V_2$.
\end{proposition}

\begin{proof}
	Применим основную теорему о гомоморфизме к гомоморфизму представлений $\phi: V_1 \oplus V_2 \to V_2$, осуществляющему проекцию на подпространство $V_2$.
\end{proof}

\begin{definition}
	Пусть $U \le V$. Оператор $A \in \mc L(V)$ называется \textit{проектором на подпространство} $U$, если выполнены следующие условия:
	\begin{enumerate}
		\item $\im A \subset U$
		\item $A|_{U} = \id$
	\end{enumerate}

	Заметим, что если $A$ "--- проектор на некоторое подпространство, то это подпространство однозначно восстанавливается как $\im A$, поэтому можно называть $A$ \textit{проектором}, не указывая подпространство.
\end{definition}

\begin{note}
	Легко видеть, что в силу условия $(2)$ включение $\im A \subset U$ в условии $(1)$ можно заменить на равенство $\im A = U$. Кроме того, заметим, что $A \in \mc L(V)$ является проектором $\lra$ $A|_{\im\phi} = \id$.
\end{note}

\begin{proposition}
	Оператор $A \in \mc L(V)$ является проектором $\lra$ $A^2 = A$.
\end{proposition}

\begin{proof}
	Нетривиально только доказательство $\la$. Положим $U := \im{A}$. Для каждого $u \in U$ можно выбрать $v \in V$ такой, что $u = A(v)$, тогда $A(u) = A^2(v) = A(v) = u$. Значит, $A|_U = \id$.
\end{proof}

\begin{proposition}
	Пусть $A \in \mc L(V)$ "--- проектор. Тогда $V = \im{A} \oplus \ke{A}$.
\end{proposition}

\begin{proof}
	С одной стороны, $\im{A} \cap \ke{A} = \{0\}$, поскольку если $u \in \im{A} \cap \ke{A}$, то $u = A(u) = 0$. Значит, сумма $\im{A} + \ke{A}$ "--- прямая. С другой стороны, любой вектор $v \in V$ можно представить в виде $v = A(v) + (v - A(v))$, где $A(v) \in \im{A}$, а $(v - A(v)) \in \ke{A}$, поскольку $A(v - A(v)) = A(v) - A^2(v) = 0$.
\end{proof}

\begin{corollary}
	Пусть $U \le V$, $\pi \in \mc L(V)$ "--- проектор на $U$. Тогда $\tr\pi = \dim{U}$.
\end{corollary}

\begin{proof}
	В базисе, являющемся объединением базисов в $U$ и $\ke\pi$, оператор $\pi$ имеет матрицу $\diag{(1, \dotsc, 1, 0, \dotsc, 0)}$, в которой ровно $\dim{U}$ единиц.
\end{proof}

\begin{definition}
	Пусть $\mc A\subset \mc L(V)$. Оператор $\phi \in \mc L(V)$ называется \textit{$\mc A$-эквивариантным}, если для любого $A \in \mc A$ выполнено $\phi \circ A = A \circ \phi$.
\end{definition}

\begin{proposition}
	Пусть подпространство $U \le V$ инвариантно относительно представления группы $G$ в $V$. Тогда существует инвариантное подпространство $W \le V$ такое, что $V = U \oplus W$ $\lra$ существует $G$-эквивариантный проектор $A$ на $U$.
\end{proposition}

\begin{proof}~
	\begin{enumerate}
		\item[$\ra$] Построим $A$ следующим образом: представим каждый вектор $v \in V$ в виде $v = u + w$, где $u \in U, w \in W$, и положим $A(v) := u$.
		\item[$\la$] Представим $V$ в виде $V = U \oplus \ke{A}$, тогда $\ke{A}$ инвариантно относительно представления, поскольку для любого $v \in \ke{A}$ выполнено $0 = g(A(v)) = A(gv)$.\qedhere
	\end{enumerate}
\end{proof}