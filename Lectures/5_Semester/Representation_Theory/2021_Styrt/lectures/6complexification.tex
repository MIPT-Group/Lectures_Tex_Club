\section{Комплексификация и овеществление}

\begin{definition}
	Пусть $V$ "--- пространство над $\R$. Его \textit{комплексификацией} называется пространство $V(\Cm) := \{u + iv : u, v \in V\}$ над $\Cm$, на котором естественным образом заданы операции сложения и умножения на числа из $\Cm$.
\end{definition}

\begin{definition}
	Пусть $W$ "--- пространство над $\Cm$. Его \textit{овеществлением} называется пространство $W_\R$ над $\R$, совпадающее с $W$ как множество и как аддитивная группа, на котором определено умножение только на числа из $\R$.
\end{definition}

\begin{note}
	Легко видеть, что $\dim{V(\Cm)} = \dim{V}$, при этом $\dim{W_\R} = 2\dim{W}$. Имеют место следующие диаграммы:
	\[
	\begin{tikzcd}[
		row sep = small,
		column sep = large
		]
		V \arrow{dr}{\Cm} &  & & \arrow[swap]{dl}{\R} W\\
		& \arrow{dl}{\R} V(\Cm) & W_\R \arrow[swap]{dr}{\Cm} & \\
		V(\Cm)_\R & & & W_\R(\Cm)
	\end{tikzcd}
	\]
	
	Заметим, что $V(\Cm)_\R = V \oplus iV$. Далее мы исследуем строение пространства $W_\R(\Cm)$.
\end{note}

\begin{proposition}
	Эквивалентны следующие условия:
	\begin{enumerate}
		\item Множество $W$ образует линейное пространство над $\Cm$
		\item Множество $W$ образует линейное пространство над $\R$, причем существует оператор $I \in \mc L(W)$ такой, что $I^2 = -E$
	\end{enumerate}
\end{proposition}

\begin{proof}~
	\begin{itemize}
		\item\imp{1}{2}Тривиально, поскольку $W_\R$ является пространством над $\R$, а искомым оператором будет скалярный оператор $iE$.
		\item\imp{2}{1}Для кажого $w \in W$ и $a + bi \in \Cm$ положим $(a + bi)w := aw + bI(w)$. Легко проверить, что с заданным таким образом умножением на числа из $\Cm$ будет получено пространство над $\Cm$.\qedhere
	\end{itemize}
\end{proof}

\begin{definition}
	Пусть $V$ "--- линейное пространство над $\R$. Оператор $I \in \mc L(V)$ такой, что $I^2 = -E$, называется \textit{оператором комлпексной структуры}.
\end{definition}

\begin{proposition}
	Пусть $W$ "--- пространство над $\Cm$. Тогда $W_\R(\Cm) = W_1 \oplus W_{-1}$, где пространство $W_1$ изоморфно $W$, а $W_{-1}$ --- антиизоморфно $W$.
\end{proposition}

\begin{proof}
	По предыдущему утверждению, на $W_\R$ определен оператор комплексной структуры $I \in \mc L(W_\R)$. Его можно естественным образом продолжить на $W_\R(\Cm)$, положив $I(x + iy) := I(x) + iI(y)$ для любых $x, y \in W_\R$. Заметим, что $(iI)^2 = E$, поэтому собственными значениями оператора $iI \in \mc L(W_\R(\Cm))$ могут быть только числа $\pm1$. Значит, $W_\R(\Cm) = W_1 \oplus W_{-1}$, где $W_1, W_{-1}$ "--- собственные подпространства оператора $iI$. Зафиксируем произвольный $x + iy \in W_\R(\Cm)$, тогда:
	\[iI(x + iy) = i(I(x) + iI(y)) = -I(y) + iI(x)\]
	
	Значит, если $x + iy$ "--- собственный вектор оператора $iI$ с собственным значением $\pm1$, то $x = \mp I(y) \lra y = \pm I(x)$. Следовательно, $W_{\pm1} = \{x \pm iI(x) : x \in W\}$, и имеет место естественная вещественно-линейная биекция $\phi_{\pm1}$ между $W_{\pm1}$ и $W$ вида $x \mapsto x \pm iI(x)$, при этом верны следующие равенства:
	\begin{align*}
		\phi_1((\alpha + \beta I)x) &= (\alpha + i\beta)(x + iI(x))\\
		\phi_{-1}((\alpha + \beta I)x) &= (\alpha - i\beta)(x + iI(x))\qedhere
	\end{align*}
\end{proof}

\begin{note}
	Операторы на $W$ можно считать операторами на $W_\R$, коммутирующими с оператором комплексной структуры $I$, причем каждый оператор $A \in \mc L(W)$ можно естественным образом продолжить на $W_\R(\Cm)$ как $A(x + iy) := A(x) + iA(y)$ для любых $x, y \in W$. Поскольку любой оператор $A \in \mc L(W)$ коммутирует с $iI$, то пространства $W_1$ и $W_{-1}$ инвариантны относительно $A$. Значит, если $\mc A \subset L(W)$ и пространство $W$ неприводимо относительно $\mc A$, то пространство $W_\R(\Cm)$ уже приводимо относительно $\mc A$. Аналогичное верно и для обратной ситуации $V \mapsto V(\Cm) \mapsto V(\Cm)_\R$.
\end{note}

\begin{note}
	Пусть $V$ "--- пространство над $V$. Заметим, что если $U \le V$ "--- инвариантное относительно оператора $A \in \mc L(V)$ подпространство, то подпространство $U(\Cm) \le V(\Cm)$ инвариантно относительно естественного продолжения оператора $A$.
\end{note}

\begin{theorem}
	Пусть пространство $V$ над $\R$ неприводимо относительно семейства операторов $\mc A \subset \mc L(V)$, а пространство $V(\Cm)$ приводимо относительно $\mc A$. Тогда на $V$ определен $\mc A$-эквивариантный оператор комплексной структуры.
\end{theorem}

\begin{proof}
	Пусть $\{0\} < U < V(\Cm)$ "--- $\mc A$-инвариантное подпространство. Рассмотрим $\R$-линейный оператор $\tau : V(\Cm) \to V(\Cm)$ такой, что $\tau (x + iy) = x - iy$ для любых $x, y \in V$. Тогда он является автоморфизмом представления семейства $\mc A$ в $V(\Cm)_\R$, поэтому подпространство $\tau U \le V(\Cm)$ тоже $\mc A$-инвариантно, откуда $U \cap \tau U$, $U + \tau U$ "--- тоже $\mc A$-инвариантные. Заметим теперь, что любое $\tau$-инвариантное подпространство в $V(\Cm)$ имеет вид $W(\Cm)$ для некоторого $W \le V$. Тогда:
	\begin{align*}
		U \cap \tau U &= W_1(\Cm),~W_1 \le V\\
		U + \tau U &= W_2(\Cm),~W_2 \le V
	\end{align*}
	
	В силу $\mc A$-инвариантности подпространств $U \cap \tau U, U + \tau U \le V(\Cm)$, подпространства $W_1, W_2 \le V$ тоже $\mc A$-инвариантны, тогда, поскольку подпространство $U$ нетривиально, $W_1 = \{0\}$ и $W_2 = V$. Следовательно, $V(\Cm) = U \oplus \tau U$. Тогда:
	\[\dim_\R V = \dim_\Cm V(\Cm) = \dim_\Cm U + \dim_\Cm \tau U = 2\dim_\Cm U = \dim_\R U\]
	
	Докажем, что оператор взятия действительной части $\re : U \to V$ "--- это изоморфизм пространств над $\R$. В силу равенства размерностей, достаточно проверить сюоръективность. Поскольку $U$ "--- $\mc A$-инвариантное над $\R$ подпространство, то $\re U \le V$ "--- тоже. Тогда, в силу неприводимости, возможны два варианта:
	\begin{enumerate}
		\item $\re U= \{0\}$, тогда $U \le iV \ra U = iU \le V \ra U \le V \cap iV = \{0\}$ --- противоречие
		\item $\re U = V$, тогда $U$ и $V$ изоморфны как пространства над $\R$, что и требовалось
	\end{enumerate} 
		
	Наконец, на $U$ определен $\mc A$-эквивариантный оператор комплексной структуры $J$, унаследованный из $V(\Cm)$ как умножение на $i$. Но тогда и на $V$ определен соответствующй оператор комплексной структуры $I$, что и дает требуемое.
\end{proof}

%%% прив -> при ввсегда
%%% неприв -> прив тогда и только тогда, когда исходное пр-во - эо чье-то овеществленме (в одну сторону легко, в другую - по теореме)

%%% эквивариантноснть комплексной структуры и вещественной формы

\begin{theorem}
	Пусть пространство $W$ над $\Cm$ неприводимо относительно семейства операторов $\mc A \subset \mc L(W)$, а пространство $W_\R$ приводимо относительно $\mc A$. Тогда $W = U \oplus iU$ для некоторого $\mc A$-инвариантного пространства $U$ над $\R$.
\end{theorem}

\begin{proof}
	Пусть $\{0\} < U < W_\R$ "--- $\mc A$-инвариантное подпространство. Тогда, в силу неприводимости пространства $W$, выполнено $U + iU = \gl U\gr_\Cm = W$. Заметим, что подпространство $U \cap iU$  над $\Cm$ "--- $\mc A$-инвариантное, поэтому $U \cap iU = \{0\}$, откуда $W = U \oplus iU$, что и дает требуемое.
\end{proof}

\begin{note}
	Зафиксируем семейство операторов $\mc A$. В силу теорем и замечаний выше, можно сделать следующие выводы:
	\begin{itemize}
		\item Приводимое относительно $\mc A$ пространство всегда переходит в приводимое при комплексификации или овеществлении
		
		\item Неприводимое относительно $\mc A$ пространство над $\R$ при комплексификации переходит в приводимое тогда и только тогда, когда оно само является овеществлением некоторого пространства, согласованного с семейством $\mc A$
		
		\item Неприводимое относительно $\mc A$ пространство над $\Cm$ при овеществлении переходит в приводимое тогда и только тогда, когда оно само является комплексификацией некоторого пространства, согласованного с семейством $\mc A$
	\end{itemize}
\end{note}

\begin{definition}
	Пространство $V$ над $\R$ называется \textit{абсолютно неприводимым} относительно $\mc A \subset \mc L(V)$, если пространство $V(\Cm)$ неприводимо относительно $\mc A$.
\end{definition}