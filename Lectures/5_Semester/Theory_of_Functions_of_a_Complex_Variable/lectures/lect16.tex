\begin{flushright}
    \textit{Лекция 16 (от 27.10)}
\end{flushright}
\theorem (Руше)
Пусть $G$~--- односвязная область, $\ogamma$~--- замкнутая положительно
ориентированная кусочно гладкая кривая в этой области. Пусть $f,g: G \mapsto
\CC$ регулярны, и
\begin{equation}\label{(20.3)}
    \forall z \in \ogamma \ \left| f(z) \right| > \left| g(z) \right|
\end{equation}
Тогда $f(z)$ и $h(z) = f(z)+g(z)$ имеют внутри $\ogamma$ одинаковое число нулей
с учетом их порядка.
\pr
\begin{align*}
  & \forall z \in \ogamma \ \left| f(z) \right| > \left| g(z) \right| \Rightarrow \forall z \in \ogamma \ f(z) \neq 0
\end{align*}
\begin{align*}
  & \forall z \in \ogamma \ \left| h(z) \right|\geq \left| f(z) \right| - \left| g(z) \right| > 0 \Rightarrow \forall z \in \ogamma h(z) \neq 0
\end{align*}
\begin{align*}
  & \Delta_{\ogamma}\argt h(z) = \Delta_{\ogamma} \argt \left( f(z)\left( 1+\frac{g(z)}{f(z)} \right) \right) = \Delta_{\ogamma}\argt f(z) +\Delta_{\ogamma}\argt \left( 1+\frac{g(z)}{f(z)} \right)
\end{align*}
\begin{align*}
  & \ogamma: w = 1 + \frac{g(z)}{f(z)} \Rightarrow \left| w-1 \right| = \left| \frac{g(z)}{f(z)} \right| < 1
\end{align*}
Пусть $\Gamma = w(\ogamma)$; $0 \not \in B_1(1)$ и эта область односвязна,
значит, по лемме $2$ $\S 14$
\begin{align*}
  & \Delta_{\ogamma}\argt \left( 1+\frac{g(z)}{f(z)} \right) = 0
\end{align*}
и тогда
\begin{align*}
  & N_h = \frac{1}{2\pi}\Delta_{\ogamma}\argt h(z) = \frac{1}{2\pi}\Delta_{\ogamma}\argt f(z) = N_f
\end{align*}
\theorem (Гаусса)
Многочлен
\begin{align*}
  & P_n(z) = c_0+zc_1+z^2c_2+\dots+z^nc_n
\end{align*}
имеет в $\CC$ ровно $n$ корней с учетом их порядка.
\pr
Рассмотрим $f(z) = z^n$, $g(z) = P_n(z) - f(z)$. Как известно,
\begin{align*}
  & \left| \frac{g(z)}{f(z)} \right| \us{z \to \infty}{\to} 0 \Rightarrow \exists R_0 > 0: \ \forall R \geq R_0 \ \forall z \in \gamma_R \left| f(z) \right| > \left| g(z) \right|
\end{align*}
Значит, по теореме Руше функция $P_n(z)$ имеет столько же нулей, сколько и $f(z)
= z^n$, на $B_R(0)$, с учетом порядка, т.~е. ровно $n$ штук.
\Example
Функция Жуковского
\begin{align*}
  & w = \frac{1}{2}\left( z+\frac{1}{z} \right)
\end{align*}
У функции $\pm i$~--- нули, $0$~--- полюс первого порядка.
\\
Рассматривая $R>1$, получим
\begin{align*}
  & \Delta_{\gamma_R}\argt w(z) = 2 \pi (N-P) = 2\pi
\end{align*}
\begin{align*}
  & \Delta_{\gamma_{\frac{1}{R}}}\argt w(z) = 2 \pi (N-P) = -2\pi
\end{align*}
\section{$\S 21.$ Геометрические принципы.}
