\LARGE{Лекция 5 (от 21.09)}
\normalsize
% \begin{center}
\section{$\S 7.$ Интегральная теорема Коши}
% \end{center}
\theorem (Коши)
Пусть $G$~--- односвязная область, пусть $f: G \mapsto \CC$
регулярна. Тогда для любой простой замкнутой кусочно гладкой кривой
$\overset{\circ}{\gamma} \subseteq G$ \hypertarget{(1)}{выполняется}
\begin{align}
  \int_{\overset{\circ}{\gamma}}f(z)dz = 0
\end{align}
\lemma (Гурса)
Пусть $G$~--- область, $f: G \mapsto \CC$ регулярна. Тогда для
любого треугольника из $G$ \hypertarget{(2)}{верно}
\begin{align}
  \int_{\partial \triangle} f(z) dz = 0
\end{align}
\pr
\begin{align*}
  \triangle ABC \subseteq G
\end{align*}
\begin{align*}
  I = \int_{\partial \triangle ABC}f(z)dz
\end{align*}
Разобьем треугольник средними линиями:
\begin{align*}
  \triangle ABC = \bigcup_{k=1}^{4}\triangle_k
\end{align*}
Тогда
\begin{align*}
  I = \sum_{k=1}^4\int_{\partial \triangle_k}f(z)dz
\end{align*}
Докажем, что
\begin{align*}
  \exists k_0: \left| \int_{\partial \triangle_{k_0}} f(z)dz\right|\geq\frac{\left| I \right|}{4}
\end{align*}
Очевидно от противного: т.~к. триангуляция с ориентацией, то если бы все были
меньше, то нельзя было бы набрать $I$.
\\
Назовем этот треугольник $\triangle^1$, а $\triangle ABC = \triangle^0$.
Аналогично построению $\triangle^1$ из $\triangle^0$ можем построить
бесконечную последовательность $\triangle^{N+1}$ из $\triangle^N$, и для них
\begin{align*}
  \left| \int_{\partial \triangle^N} f(z)dz\right|\geq\frac{\left| I \right|}{4^N}
\end{align*}
\begin{align*}
  P_N = \frac{P_0}{2^N}
\end{align*}
В силу компактности
\begin{align*}
  \exists z_0 \in \bigcap_{N=1}^{\infty}\triangle^N
\end{align*}
Т.~к. $f$ дифференцируема в $z_0$, то по определению дифференцируемости
\begin{align*}
  \exists B_{\delta_0}(z_0): \ \forall z \in B_{\delta_0}(z_0) \ f(z) = f(z_0)+f'(z_0)(z-z_0) + o(z-z_0)
\end{align*}
\begin{align*}
  o(z-z_0): \ \forall \varepsilon > 0 \exists \delta_1 \leq \delta_0: \forall z \in B_{\delta_1}(z_0) \ \left| o(z-z_0) \right| \leq \varepsilon\left| z-z_0 \right|
\end{align*}
\begin{align*}
  & \int_{\partial \triangle^N}f(z)dz = f(z_0)\int_{\partial \triangle^N}dz+f'(z_0)\int_{\partial \triangle^n}zdz - z_0 f'(z_0)\int_{\partial \triangle^N} dz + \int_{\partial \triangle^N}o(z-_0)dz = \\
  & = \int_{\partial \triangle^N}o(z-z_0)dz
\end{align*}
причем
\begin{align*}
  \forall z \in \triangle^N \left| z-z_0 \right| < \delta_1
\end{align*}
Тогда
\begin{align*}
  \left| \int_{\partial \triangle^N}f(z)dz \right| \leq \int_{\partial \triangle^N} \left| o(z-z_0) \right|\cdot \left| dz \right| \leq \varepsilon\int_{\partial\triangle^N}\left| z-z_0 \right|\cdot\left| dz \right| \leq \varepsilon P^2_N \leq \varepsilon \frac{P^2_0}{4^N}
\end{align*}
\begin{align*}
  \frac{\left| I \right|}{4^N} \leq \varepsilon \frac{P^2_0}{4^N}
\end{align*}
\begin{align*}
  I = 0
\end{align*}
ч.~т.~д.
\corollary (леммы)
Пусть $G$~--- односвязная область, а $\gamma_{br} \subseteq
G$~--- замкнутая жорданова ломаная; $f:G \mapsto \CC$ регулярна. \hypertarget{(3)}{Тогда}
\begin{align}
  \int_{\gamma_{br}} f(z)dz = 0
\end{align}
\pr
По теореме Жордана существует триангуляция с ориентацией для $\gamma_{br}$;
тогда 
\begin{align*}
  \int_{\gamma_{br}} f(z)dz = \sum_{k=1}^K\int_{\partial\triangle^K}f(z)dz = 0
\end{align*}
ч.~т.~д.
\pr (теоремы Коши)
Для любой ломаной $\gamma_{br}\subseteq G$ выполняется \hyperlink{(3)}{(3)}.
Тогда по теореме $3$ $\S 6$ (п. $2$) $f dz$~--- полный дифференциал,
соответственно, по теореме $3$ $\S 6$ (п. $1$) выполняется \hyperlink{(1)}{(1)}.
\note
Нельзя так просто убрать односвязность области:
\begin{align*}
  \int_{\left| z \right| = 1} \frac{dz}{z} = 2 \pi i \neq 0
\end{align*}
но $\dst \frac{1}{z}$ регулярна в $\CC \setminus \{0\}$.
