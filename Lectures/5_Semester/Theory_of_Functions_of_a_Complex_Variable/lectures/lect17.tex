\begin{flushright}
    \textit{Лекция 17 (от 02.11)}
\end{flushright}
\section{$\S 22.$ Конформные отображения в $\overline{\CC}$.}
\begin{center}
    \textbf{Геометрический смысл аргумента и модуля производной}
\end{center}
Рассмотрим $f: B_r(z_0) \mapsto \CC$, $f'(z_0) \neq 0$.
\\
Пусть $w = f'(z_0)$, $w-w_0 = f'(z_0)(z-z_0) + o(z-z_0) = \alpha (z-z_0) +
o(z-z_0)$.
\\
Покоординатно:
\begin{align*}
  & f'(z_0) = u_x+iv_x, \ f=u+v
\end{align*}
\begin{align*}
  & \left( \begin{matrix}
          \Delta u \\
          \Delta v
      \end{matrix} \right) = \left( \begin{matrix}
          u_x & -v_x \\
          v_x & u_x
      \end{matrix} \right)  \left( \begin{matrix}
          \Delta x \\
          \Delta y
      \end{matrix} \right) = K \left( \begin{matrix}
          \frac{u_x}{K} & \frac{-v_x}{K} \\
          \frac{v_x}{K} & \frac{u_x}{K}
      \end{matrix} \right) \left( \begin{matrix}
          \Delta x \\
          \Delta y
      \end{matrix} \right)
\end{align*}
\begin{align*}
  & K = \sqrt{u^2_x+v^2_x} = \left| f'(z_0) \right|
\end{align*}
Видим, что это ортогональное преобразование.
\\
\textbf{Свойство сохранения окружности в малом:}
\\
Рассмотрим $\gamma_r = \left\{ z: \left| z-z_0 \right| = r, \ 0 < r <
    r_0\right\}$. Пусть в области, ограниченной кривой, производная ненулевая.
\begin{align*}
  & \left| \Delta w \right| \approx \left| f'(z_0) \right|\cdot \left| \Delta z \right| \approx K r
\end{align*}
(получаем <<примерно окружность>>).
\\
\textbf{Свойство сохранения углов:}
\\
Рассмотрим теперь $\gamma_1, \gamma_2: \ z = z_k(t), \ t \in \left[ t_0 -\delta;
    t_0+\delta\right], \ z_k(t_0) = z_0$. Пусть при таких $t$ $z'_k\neq 0$, угол
между кривыми $\alpha$. Пусть $\gamma_1^* = f(\gamma_1)$, $\gamma_2^* =
f(\gamma_2)$. Тогда угол между $\gamma_1^*$ и $\gamma_2^*$ также равен $\alpha$.
\\
Действительно,
\begin{align*}
  & w'_k(t_0) = f'(z_0)z'_k(t_0)
\end{align*}
\begin{align*}
  & \Arg w'_k(t_0) = \argm f'(z_0) + \Arg z'_k(t_0)
\end{align*}
(изменение на одинаковый угол).
\begin{center}
    \textbf{Конформные отображения в $\CC$}
\end{center}
\Def
Функция $f: G \mapsto \CC$ называется \textbf{конформной в точке $z_0$}, если
$f= u+iv$, $u$ и $v$ дифференцируемы в $z_0$ и линейное отображение вида
\begin{equation}\label{(22.1)}
    \begin{cases}
        du = u_x(x_0,y_0)\Delta x + u_y(x_0,y_0) \Delta y \\
        dv = v_x(x_0,y_0)\Delta x + v_y(x_0,y_0) \Delta y
    \end{cases}
\end{equation}
является суперпозицией линейного растяжения и поворота относительно нуля.
\theorem
Функция $f$ конформна в $Z_0 \in \CC$ тогда и только тогда, когда она в этой
точке дифференцируема, а производная отлична от нуля.
\pr
~
\begin{itemize}
    \item $\Leftarrow$
    \\
    Следует из геометрического смысла и определения.
    \item $\Rightarrow$
    \\
    Пусть $f$ конформна в $z_0$. Тогда $\exists K > 0$, $\exists \theta \in
    [0;2\pi)$ такие, что из \eqref{(22.1)} получаем
    \begin{equation}\label{(22.2)}
        \left( \begin{matrix}
                \Delta u \\
                \Delta v
            \end{matrix} \right) = K \left( \begin{matrix}
                \cos \theta & \sin \theta \\
                -\sin \theta & \cos \theta
            \end{matrix} \right) \left( \begin{matrix}
                \Delta x \\
                \Delta y
            \end{matrix} \right)
    \end{equation}
    В силу дифференцируемости $u$ и $v$
    \begin{align*}
      & u_x = K \cos \theta, \ u_y = K \sin \theta, \ v_x = -K \sin \theta, \ u_y = K \cos \theta
    \end{align*}
    Выполняется УКР, значит, $\exists f'(z_0)$, $\left| f'(z_0) \right| = K >
    0$.
\end{itemize}
\Def
$f: G \mapsto \CC$ \textbf{конформна на области $G$}, если $f$ однолистна на $G$
и конформна в каждой ее точке.
\corollary
$f$ конформна в $G \subset \CC$ $\Leftrightarrow$ $f$ однолистна и регулярна в
$G$.
\begin{center}
    \textbf{Конформные отображения в $\CCC$}
\end{center}
\underline{\textbf{Свойства стереографической проекции}}
\begin{enumerate}
    \item Образы любых двух пересекающихся кривых на комплексной плоскости будут
    пересекаться на сфере Римана под тем же углом.
    \item $w = \frac{1}{z}: \CCC \mapsto \CCC$ соответствует при
    стереографической проекции отображению сферы Римана на себя путем поворота
    ее на $\pi$ относительно ее диаметра с концами в точках, являющихся образами
    $1$ и $-1$ на $\CC$.
\end{enumerate}
\Def
Пусть $f$ имеет УОТ в $\infty$. Тогда $f$ называется \textbf{конформной в
  $\infty$}, если $g(z) = f\left( \dst \frac{1}{z} \right)$, доопределенная по
непрерывности в нуле, конформна в нуле.
\Def
Пусть $a \in \CCC$~--- полюс или СОТ $f$. Тогда $f$ называется
\textbf{конформной в $a$}, если $\varphi(z) = \dst \frac{1}{f(z)}$,
доопределенная по непрерывности, конормна в этой точке.
\Exse
Доказать, что в определении $22.4$ допустим лишь полюс $1$ порядка.
\Def
Пусть $f: G \mapsto \CCC$. Тогда $f$ называется \textbf{конформной в области
  $G$}, если она однолистна на ней и конформна в каждой ее точке.
\prop
$f$ конформна в $G \subseteq \CCC$, если $f$ однолистна на ней и регулярна на
ней, за исключением, быть может, двух точек:
\begin{itemize}
    \item $\infty$, если $\infty \in G$ и является УОТ или полюсом $1$ порядка;
    \item $a \in G$, $a \neq \infty$~--- полюс $1$ порядка, если $\infty$~---
    УОТ или $\infty \not \in G$.
\end{itemize}
\section{$\S 23.$ Дробно-линейные функции.}
