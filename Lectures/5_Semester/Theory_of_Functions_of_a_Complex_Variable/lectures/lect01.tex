\begin{flushright}
    \textit{Лекция 1 (от 07.09)}
\end{flushright}
\section{$\S 1.$ Комплексные числа}
\Def
Пусть $z = \left( x;y \right) \in \RR^2$. Пусть определены операции:
\begin{enumerate}
    \item \textbf{Сложение:} $z_1 + z_2 = \left( x_1+x_2; y_1+y_2 \right)$
    \item \textbf{Умножение на действительное число:} $\lambda z = \left(\lambda
        x, \lambda y \right)$
    \item \textbf{Расстояние:} $\rho(z_1, z_2) = \left| \left| z_1 - z_2 \right|
    \right| = \sqrt{\left( x_1 - x_2 \right)^2 + (y_1 - y_2)^2}$
\end{enumerate}
Добавим операцию
умножения друг на друга:
\begin{align*}
  & z_1 \cdot z_2 = \left( x_1 \cdot x_2 - y_1 \cdot y_2; x_1 \cdot y_2 + x_2 \cdot y_1 \right)
\end{align*}
Будем называть это \textbf{комплексными числами} $\CC$.
\\
Пусть $1 \sim (1; 0)$~--- \textbf{единица}, $i \sim (0;1)$~--- \textbf{мнимая
  единица}.
\\
Тогда $z = 1 \cdot x + i \cdot y \Leftrightarrow z = x + iy$~---
\textbf{алгебраическая запись}.
\\
Назовем $x = \Real z$ \textbf{действительной частью}, $y =
\Img z$~--- \textbf{мнимой}.
\\
\textbf{Сопряженным} к $z = x+iy$ называем $\bar{z} = x-iy$,
\textbf{модулем}~--- выражение $\left| z \right| = \sqrt{x^2+y^2}$. Легко
видеть, что $z\bar{z} = \left| z \right|^2$, полагая $i^2 = -1$.
\\
Можно представлять комплексные числа в виде $x = r \cos \varphi$, $y = r \sin
\varphi$, называя $r$ \textbf{модулем}, $\varphi$ \textbf{аргументом}, а $\arg z
\in (-\pi; \pi]$~--- одно из таких $\varphi$~--- \textbf{главным аргументом}.
Также \textbf{аргументом} называется множество $\Arg z = \left\{ \arg z + 2\pi k
    \mid k \in \ZZ \right\}$.
\\
Видим, что
\begin{align*}
  & z = \left| z \right|\cos \varphi + i \left| z \right|\sin \varphi = \left| z \right|\left( \cos \varphi + i \sin \varphi \right)
\end{align*}
Введем \textbf{комплексную экспоненту}~--- $e^{i\varphi} = \cos \varphi + i \sin
\varphi$, тогда $z = \left| z \right|e^{i\varphi}$.
\\
Для произведения тогда выполняется
\begin{align*}
  & z_1z_2 = \left| z_1 \right|\left( \cos \varphi_1 + i \sin \varphi_2 \right)\left| z_2 \right|\left( \cos \varphi_2 + i \sin \varphi_2 \right) = \left| z_1 \right|\cdot \left| z_2 \right| \left( \cos \left( \varphi_1+\varphi_2 \right) + i \sin \left( \varphi_1+\varphi_2 \right)\right)
\end{align*}
\begin{align*}
  & \left|z_1z_2\right| = \left| z_1 \right|\cdot \left| z_2 \right|
\end{align*}
\begin{align*}
  & \Arg(z_1z_2) = \Arg z_1 + \Arg z_2
\end{align*}
\textbf{Суммой Минковского} называется множество $A+B = \sets{a+b \mid a\in A, \
  b \in B}$.
\\
На комплексных числах определено \textbf{деление} на $z\neq 0$: $z_1z = z_2
\Rightarrow z = \dst \frac{z_2}{z_1}$; если $z_2=1$, то $z = \dst \frac{1}{z_1}
= z_1^{-1}$.
\\
Нахождение частного эквивалентно решению системы
\begin{align*}
  & \left\{ \begin{matrix}
          x_1x-y_1y=x_2 \\
          y_1x+x_1y = y_2
      \end{matrix} \right.
\end{align*}
Это можно выразить как
\begin{align*}
  & z = \frac{\bar{z_1}z_2}{\left| z_1 \right|^2} = \frac{\left| z_2 \right|e^{i \varphi_2}}{\left| z_1 \right| e^{i \varphi_1}} = \frac{\left| z_2 \right|}{\left| z_1 \right|}e^{i\left( \varphi_2-\varphi_1 \right)}
\end{align*}
Зная, что $e^{i \varphi_1}e^{i \varphi_2} = e^{i\left( \varphi_1+\varphi_2
  \right)}$, можем записать \textbf{формулу Муавра}:
\begin{align*}
  & \forall n \in \NN \ z^n = \left| z \right|^ne^{i n \varphi} 
\end{align*}
Также определена операция \textbf{извлечения корня} из $z^n = a \neq 0$.
Полагаем $z=re^{i\varphi}$, $a = \left| a \right|e^{i\alpha}$. Тогда $r^n =
\left| \alpha \right|$, а $n \varphi = \alpha + 2 \pi k, \ k \in \ZZ$. Отсюда $r
= \sqrt[n]{\left| a \right|}$, $\varphi_k = \dst \frac{\alpha+2\pi k}{n}$. тогда
определим корень как
\begin{align*}
  & \sets{\sqrt[n]{a}} = \sets{\sqrt[n]{\left| a \right|}\left( \cos \frac{\alpha + 2 \pi k}{n} + i \sin \frac{\alpha + 2 \pi k}{n} \right)\mid k \in \{0, \dots, n-1\}}
\end{align*}
\Example
\begin{align*}
  & \sets{\sqrt[4]{i}} = \sets{\exp\left( i \frac{\dst \frac{\pi}{2} + 2 \pi k}{4} \right)\mid k \in \{0, \dots, 3\}}
\end{align*}
\section{$\S 2.$ Последовательность. Предел. Ряды. Функции. Расширенная
  комплексная плоскость}
