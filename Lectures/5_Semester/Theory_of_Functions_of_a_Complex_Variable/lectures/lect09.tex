\begin{flushright}
    \textit{Лекция 9 (от 05.10)}
\end{flushright}
\corollary
\begin{enumerate}
    \item $a \neq \infty$~--- полюс $f$ $\Leftrightarrow$ $\exists
    \overset{\circ}{B}_{\rho}(a), \exists m \in \NN, \exists p: B_{\rho}(a)
    \mapsto \CC$~--- регулярная, причем $p(a) \neq 0$, для которой верно:
    \begin{equation}\label{(12.3)}
        f(z) = \frac{p(z)}{(z-a)^m} \forall z \in \overset{\circ}{B}_\rho(a)
    \end{equation}
    \item $a = \infty$~--- полюс $f$ $\Leftrightarrow$ $\exists
    \overset{\circ}{B}_{\rho}(\infty), \exists m \in \NN, \exists h:
    B_{\rho}(\infty) \mapsto \CC$~--- регулярная, причем $\dst\lim_{z \to
      \infty} h(z) = h(\infty) \neq 0$, для которой верно:
    \begin{equation}\label{(12.4)}
        f(z) = z^mh(z) \forall z \in \overset{\circ}{B}_\rho(\infty)
    \end{equation}
\end{enumerate}
\Def \label{[12.3]}
Пусть $a$~--- полюс $f$, $m$ определено из \eqref{(12.3)} или
\eqref{(12.4)}. Тогда $m$ называется \textbf{порядком полюса $a$}.
\Def \label{[12.4]}
Пусть $g: B_R(a) \mapsto \CC$ регулярна, и $g(a) = g'(a) = \dots = g^{(n-1)}(a)
= 0$, $g^{(n)}(a)\neq 0$. Тогда $a$ называется \textbf{нулем $n$-го порядка
  функции $g$}.
\corollary
Пусть $g$, $h: B_\rho(a) \mapsto \CC$ регулярны; пусть $a$~--- нуль $k$-го
порядка функции $g$, причем $k \geq 0$, и $m$-го порядка функции $h$, причем $m
\geq 1$. Тогда для $f = \dst \frac{g}{h}$ точка $a$ будет:
\begin{enumerate}
    \item полюсом $m-k$-го порядка, если $m > k$;
    \item УОТ, если $m \leq k$.
\end{enumerate}
\pr
По определению \ref{[12.4]}
\begin{align*}
  g(z) = (z-a)^kg_1(z), \ g_1(a) \neq 0
\end{align*}
\begin{align*}
  g(z) = (z-a)^mh_1(z), \ h_1(a) \neq 0
\end{align*}
Тогда
\begin{align*}
  p(z) = \frac{g_1(z)}{h_1(z)}
\end{align*}
регулярна в некоторой окрестности $a$, $p(a) \neq 0$. Тогда из следствия $1$
получаем желаемое.
\note
Существуют неизолированные особые точки. Пусть
\begin{align*}
  f(z) = \frac{1}{\sin \frac{\pi}{z}}
\end{align*}
Тогда $\dst \frac{z_n} = \frac{1}{n}$~--- особые точки, но $0$~---
неизолированная особая точка (предельная точка полюсов).
\Exse
Пусть $a$~--- полюс $f$ и существенно особая точка $g$. Чем является $a$ для $h
= fg$?
\Exse
Пусть $a$~--- полюс $f$. Чем является $a$ для $g= f^2$?
\Exse
Пусть $a$~--- существенно особая точка $f$. Чем является $a$ для $g = \dst
\frac{1}{f}$?
\Exse
Пусть $a$~--- полюс $f$. Чем является $a$ для $g = e^f$?
\section{$\S 13.$ Теория вычетов}
