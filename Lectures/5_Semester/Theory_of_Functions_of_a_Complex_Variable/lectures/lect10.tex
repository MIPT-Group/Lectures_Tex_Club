\begin{flushright}
    \textit{Лекция 10 (от 06.10)}
\end{flushright}
\Example
Вычисление несобственных интегралов.
\begin{equation}\label{(13.9)}
  I = \int_{-\infty}^{\infty}F_{n,m}(x)dx
\end{equation}
\begin{align*}
  & F_{n,m}(x) = \frac{P_n(x)}{Q_m(x)}, \ Q_m \neq 0 \forall x \in \RR, \ m > n+1
\end{align*}
Пусть $R_0 \in (0, R)$,$\gamma_r = [-R;R]\cap C_R$ (см. рис. \ref{fig:13.1}).
Пусть $z_k^+$~--- нули $Q_m(z)$, причем $\Img z_k^+ > 0, \ k \in \{1, \dots,
n\}$, $R_0 = \max \sets{z_k^+}$. Пусть
\begin{align*}
  & I_R = \int_{\gamma_R}F_{n,m}(z)dz = 2 \pi i \sum_{k=1}^n\us{z_k}{\res}F_{n,m}, \ R > R_0
\end{align*}
Но
\begin{align*}
  & I_R = \int_{C_R}F_{n,m}(z)dz + \int_{-R}^RF_{n,m}(z)dz
\end{align*}
Легко заметить, что
\begin{align*}
  & \lim_{R \to \infty} \int_{C_R}F_{n,m}(z)dz = 0
\end{align*}
Значит,
\begin{equation}\label{(13.10)}
  I = \int_{-\infty}^{\infty}F_{n,m}(x)dx = 2 \pi i \sum_{k=1}^n \us{z_k}{\res} F_{n,m}
\end{equation}
\lemma
Пусть $\Phi(z)$ непрерывна на $\sets{z \mid \Img z \geq 0, \abs{z} \geq R_0 >
  0}$.
\\
Пусть $\varepsilon(R) = \max \left\{ \left| \Phi(z) \right| : z \in C_R\right\},
\ R > R_0$. Пусть $\dst \lim_{R \to \infty}R\varepsilon(R) = 0$.
\\
Тогда
\begin{align*}
  & \lim_{R \to \infty }\int_{C_R}\Phi(z)dz = 0
\end{align*}
\pr
\begin{align*}
  & \left| \int_{C_R}\Phi(z)dz \right|\leq \int_{C_R}\left| \Phi(z) \right| dz \leq \varepsilon(R) \int_{C_R}dz = \pi R \varepsilon(R) \us{R \to \infty}{\to} 0
\end{align*}
Применив лемму, докажем равенство \eqref{(13.10)}.
\begin{align*}
  & F_{n,m}(z) = \frac{z^n(1+o(1))}{z^m(1+o(1))}
\end{align*}
\begin{align*}
  & \left| F_{n,m} \right| \leq 2 \left| z \right|^{n-m} \leq 2 R^{n-m}
\end{align*}
\begin{align*}
  & R\varepsilon(R) \leq 2R^{n-m+1} \us{R \to \infty}{\to} 0
\end{align*}
Таким образом доказали формулу \eqref{(13.10)}.
\Example
Вычисление несобственных интегралов.
\begin{equation}\label{(13.11)}
  I = \int_{-\infty}^{\infty}e^{i\alpha x}F_{n,m}(x)dx
\end{equation}
\begin{align*}
  & \alpha > 0, \ F_{n,m}(x) = \frac{P_n(x)}{Q_m(x)}, \ Q_m \neq 0 \forall x \in \RR, \ m > n
\end{align*}
Так же, как в предыдущем примере, задаем контур и $R$. По теореме Коши о вычетах
\begin{align*}
  & I_R = \int_{\gamma_R}e^{i\alpha z}F_{n,m}(z) dz = 2 \pi i \sum_{k=1}^n\us{z_k^+}{\res}\left( e^{i\alpha z} F_{n,m}(z)\right)
\end{align*}
Покажем, что
\begin{align*}
  & const = I_R = \int_{C_R}e^{i\alpha z}F_{n,m}(z) dz + \int_{-R}^Re^{i\alpha z}F_{n,m}(z) dz \to \int_{-R}^Re^{i\alpha z}F_{n,m}(z) dz
\end{align*}
то есть
\begin{equation}\label{(13.12)}
    \lim_{R \to \infty} \int_{C_R}e^{i\alpha z}F_{n,m}(z)dx = 0 \Rightarrow I = 2 \pi i \sum_{k=1}^n \us{z_k^+}{\res}\left( e^{i \alpha z}F_{n,m}(z) \right)
\end{equation}
\lemma (Жордана)
Пусть $\Phi(z)$ непрерывна на $\left\{ z \mid \left| z \right| \geq R_0, \Img z
    \geq 0 \right\}, \ R > R_0$, $C_R$~--- семейство полуокружностей $C_R = \{z
\mid \left| z \right| = R, \Img z \geq 0\}$, $ \varepsilon(R) = \max \left\{
    \left| \Phi(z) \right| : z \in C_R \right\}$, $ \dst \lim_{R \to \infty}
\varepsilon(R) = 0$, $\alpha > 0$. Тогда
\begin{equation}\label{(13.13)}
    \lim_{R \to \infty}\int_{C_R}e^{i \alpha z}\Phi(z)dz = 0
\end{equation}
\pr
$z = x + iy \in C_R$. Значит, положим $z = R \cos \varphi$, $y = R \sin \varphi$,
$ \varphi \in [0,\pi]$. Тогда
\begin{align*}
  & e^{i \alpha z} = e^{i \alpha(x+iy)} = e^{-\alpha y + i \alpha x} = e^{-\alpha R \sin \varphi + i \alpha R \cos \varphi}
\end{align*}
\begin{align*}
  & \left| e^{i \alpha z} \right| = e^{-\alpha R \sin varphi}
\end{align*}
\begin{align*}
  & z = R e^{i \varphi} \Rightarrow dz = R i e^{i \varphi} d \varphi
\end{align*}
\begin{align*}
  & \left| \int_{C_R}e^{i \alpha z}\Phi(z) dz \right| \leq \int_{0}^{\pi}e^{-\alpha R \sin \varphi}\varepsilon(R) R d \varphi = \varepsilon(R) R \int_{0}^{\pi}e^{-\alpha R \sin \varphi}d \varphi = 2 \varepsilon(R) R \int_{0}^{\frac{\pi}{2}}e^{-\alpha R \sin \varphi}d \varphi \leq \\
  & \leq 2 \varepsilon(R) R \int_{0}^{\frac{\pi}{2}}\exp\left( -\alpha R \frac{2 \varphi}{\pi} \right) d \varphi = \frac{2 \varepsilon(R) R \pi}{2 \alpha R}\left( 1 - e^{-alpha R} \right) \leq \frac{\pi}{\alpha} \varepsilon(R) \us{R \to \infty}{\to} 0
\end{align*}
Здесь использовали соотношение: $\forall \varphi \in \left[ 0, \dst
    \frac{\pi}{2} \right] \ \sin \varphi \geq \dst \frac{2 \varphi}{\pi}$.
\begin{align*}
  & z = R e^{i \varphi} \Rightarrow dz = R i e^{i \varphi} d \varphi
\end{align*}
Применив лемму, докажем равенство \eqref{(13.12)}. При $\left| z \right| > R$ и
$m-n \geq 1$
\begin{align*}
  & \left| F_{n,m}(z) \right| \leq 2 \left| z \right|^{n-m} \leq 2 R^{n-m} \leq \frac{2}{R} \us{R \to \infty}{\to} 0
\end{align*}
\begin{align*}
  & \int_{-\infty}^\infty e^{i \alpha x}F_{n,m}(x) dx = \int_{-\infty}^\infty \cos\alpha xF_{n,m}(x) dx + i \int_{-\infty}^\infty \sin \alpha x F_{n,m}(x) dx
\end{align*}
Таким образом доказали формулу \eqref{(13.12)}.
\section{$\S 14.$ Приращение аргумента $z$ вдоль кривой}