\begin{flushright}
    \textit{Лекция 2 (от 08.09)}
\end{flushright}
\section{$\S 3.$ Дифференцирование}
Пусть $z_0 \in \CC$, $B_r(z_0)$ и $f: B_r(z_0) \mapsto \CC$
\Def
Пусть $f: B_r(z_0) \mapsto \CC$. Если
\begin{equation}\label{(3.1)}
    \exists \dst \lim_{z \mapsto z_0} \frac{f(z)-f(z_0)}{z-z_0}
\end{equation}
то он называется \textbf{производной функции $f$ в точке $z_0$} (записывается
$f'(z_0)$).
\\
Можем расписать предел:
\begin{equation}\label{(3.2)}
    \forall \varepsilon > 0 \exists \delta(\varepsilon)>0: \ z \in B_\delta(z_0), \ \left| \frac{f(z) - f(z_0)}{z-z_0} - f'(z_0) \right| < \varepsilon
\end{equation}
\begin{equation}\label{(3.3)}
    f(z) = f(z_0) + f'(z_0)(z-z_0) + \alpha(z-z_0)
\end{equation}
\begin{equation}\label{(3.4)}
    \lim_{z \to z_0}\frac{\alpha(z-z_0)}{z-z_0} = 0
\end{equation}
\begin{align*}
  & f(z_0) = u(x,ym) + iv(x,y)
\end{align*}
\Def
Говорят, что $f: B_r(z_) \mapsto \CC$ \textbf{дифференцируема в точке $z_0$},
если
\begin{equation}\label{(3.5)}
    \exists A \in \CC: \ f(z) = f(z_0)+A(z-z_0)+o(z-z_0)
\end{equation}
\lemma
$f: B_r(z_0) \mapsto \CC$ дифференцируема в $z_0$ $\Leftrightarrow$ $\exists
f'(z_0)$, $A = f'(z_0)$
\pr
Очевидно.
\\
Обозначим: $\Delta x = x-x_0$, $\Delta y = y - y_0$, $\Delta z = z-z_0 = \Delta
x + i \Delta y$, $\Delta u = u(x,y)-u(x_0,y_0)$, $\Delta v = v(x,y)-v(x_0,y_0)$,
$\Delta f = f(x,y)-f(x_0,y_0) = \Delta u + i \Delta v$.
\theorem
$f: B_r(z_0) \mapsto \CC$ дифференцируема в $z_0$ т огда и только тогда, когда
\begin{itemize}
    \item $u(x,y)$, $v(x,y)$ дифференцируемы в $(x_0,y_0)$ (в $\RR^2$)
    \item выполняется условие Коши-Римана (УКР):
    \begin{equation}\label{(3.6)}
        \left\{ \begin{matrix}
                \dst \frac{\partial u}{\partial x} = \dst \frac{\partial v}{\partial y} \\
                \dst \frac{\partial v}{\partial x} = - \dst \frac{\partial u}{\partial y}
            \end{matrix} \right.
    \end{equation}
\end{itemize}
При этом
\begin{equation}\label{(3.7)}
    f'(z_0) = \frac{\partial u}{\partial x}(x_0,y_0) + i\frac{\partial v}{\partial x}(x_0,y_0) = \frac{\partial v}{\partial y}(x_0,y_0) - i\frac{\partial u}{\partial y}(x_0,y_0)
\end{equation}
\pr
Докажем в обе стороны.
\begin{itemize}
    \item Необходимость.
    \begin{align*}
      & \exists f'(z_0) = a+ib = A \in \CC
    \end{align*}
    Значит, по определению дифференцируемости
    \begin{align*}
      & \Delta f = A \Delta z + \alpha(\Delta z); \ \alpha(\Delta z) = \alpha_1(\Delta x, \Delta y) + \alpha_2(\Delta x, \Delta y)
    \end{align*}
    \begin{align*}
      & \left\{ \begin{matrix}
              \Delta u = a \Delta x - b \Delta y + \alpha_1(\Delta x, \Delta y) \\
              \Delta v = b \Delta x + a \Delta y + \alpha_2(\Delta x, \Delta y)
          \end{matrix} \right.
    \end{align*}
    \begin{align*}
      & \left| \alpha_1 \right| \leq \left| \alpha \right|, \ \left| \alpha_2 \right| \leq \left| \alpha \right|
    \end{align*}
    \begin{align*}
      & \lim_{(\Delta x; \Delta y) \to (0;0)}\left( \frac{\left| \alpha_1(\Delta x, \Delta y) \right|}{\sqrt{\left( \Delta x \right)^2 + \left( \Delta y \right)^2}} \right) \leq \lim_{\Delta z \to 0}\left( \frac{\left| \alpha(\Delta z) \right|}{\left| \Delta z \right|} \right) = 0
    \end{align*}
    Значит, $u$ дифференцируема, причем $\dst \frac{\partial u}{\partial x} =
    a$, $\dst \frac{\partial u}{\partial y} = -b$.
    \\
    Аналогично для $v$, причем $\dst \frac{\partial v}{\partial x} = b$, $\dst
    \frac{\partial v}{\partial y} = a$.
    \\
    Видим выполнимость УКР.
    \\
    Выполнимость \eqref{(3.7)} очевидна.
    \item Достаточность.
    \\
    Пусть $u$, $v$ дифференцируемы в $(x_0,y_0)$ и выполняется УКР. Тогда
    \begin{align*}
      & \Delta f = \Delta u + i \Delta v = \frac{\partial u}{\partial x} \Delta x + \frac{\partial u}{\partial y} \Delta y + \alpha_1(\Delta x, \Delta y) + i \left( \frac{\partial v}{\partial x} \Delta x + \frac{\partial v}{\partial y} \Delta y + \alpha_2(\Delta x, \Delta y)\right) = \\
      & \frac{\partial u}{\partial x} \Delta x - \frac{\partial v}{\partial x} \Delta y + \alpha_1(\Delta x, \Delta y) + i \left( \frac{\partial v}{\partial x} \Delta x + \frac{\partial u}{\partial x} \Delta y + \alpha_2(\Delta x, \Delta y)\right) = \left( \frac{\partial u}{\partial x} + i \frac{\partial v}{\partial x}\right)\cdot \\
      & \cdot \left( \Delta x + i \Delta y \right)+ \alpha_1(\Delta x, \Delta y) + i \alpha_2(\Delta x, \Delta y)
    \end{align*}
    Значит,
    \begin{align*}
      & \exists f'(z_0) = \frac{\partial u}{\partial x} + i \frac{\partial v}{\partial x}
    \end{align*}
\end{itemize}
\Example
\begin{align*}
  & w = z^2
\end{align*}
\begin{itemize}
    \item[I] Найдем производную по определению.
    \begin{align*}
      & \frac{\Delta w}{\Delta z} = \frac{\left( z_0-\Delta z \right)^2 - z_0^2}{\Delta z} = \frac{2z_0\Delta z + \left( \Delta z \right)^2}{\Delta z} = 2z_0 \Delta z \us{\Delta z \to 0}{\to} 2z_0
    \end{align*}
    \item[II] Найдем производную с использованием теоремы.
    \begin{align*}
      & z^2 = x^2-y^2+2ixy \Rightarrow u = x^2-y^2, \ v = 2xy
    \end{align*}
    \begin{align*}
      & u_x = 2x = v_y; \ u_y = -2y = - v_x
    \end{align*}
    Видим выполнимость УКР, поэтому
    \begin{align*}
      & w' = u_x+iv_x = 2x + 2iy = 2z
    \end{align*}
\end{itemize}
\Example
\begin{align*}
  & w = \left| z \right|^2
\end{align*}
\begin{align*}
  & \left| z \right|^2 = x^2+y^2 \Rightarrow u = x^2+y^2, \ v = 0
\end{align*}
\begin{align*}
  & u_x = 2x \neq 0 = v_y; \ u_y = 2y \neq 0 = - v_x
\end{align*}
(за исключением точки $(0;0)$). Видим выполнение УКР только в точке $(0;0)$, а
значит, и функция дифференцируема только в этой точке (с производной, равной
$0$).
\Example
\begin{align*}
  & w = \ol{z}
\end{align*}
\begin{align*}
  & \ol{z} = x - iy \Rightarrow u = x, \ v = -y
\end{align*}
\begin{align*}
  & u_x = 1 \neq -1 = v_y; \ u_y = 0 = - v_x
\end{align*}
Видим, что УКР не выполняется ни в одной точке $\CC$, а значит, функция нигде не
дифференцируема.
\prop
Если $f(z)$, $g(z)$ дифференцируемы, то:
\begin{itemize}
    \item дифференцируема и сумма:
    \begin{align*}
      & \left( f+g \right)' = f'+g'
    \end{align*}
    \item дифференцируемо и произведение:
    \begin{align*}
      & \left( fg \right)' = f'g+fg'
    \end{align*}
    \item дифференцируемо и частное (когда $g(z) \neq 0$):
    \begin{align*}
      & \left( \frac{f}{g} \right)' = \frac{f'g+fg'}{g^2}
    \end{align*}
\end{itemize}
\pr
Доказательства легко провести через УКР с использованием действительного
анализа.
\section{$\S 4.$ Регулярные и гармонические функции}
