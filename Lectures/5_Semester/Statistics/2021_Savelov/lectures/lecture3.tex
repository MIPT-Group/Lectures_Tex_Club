\section{Параметрическая модель}
\begin{definition}
Вероятностно-статистическая модель называется параметрической, если семейство распределений $\displaystyle \mathcal{P}$ параметризовано, т.е. $\displaystyle \mathcal{P} =\{P_{\theta } :\theta \in \Theta \}$.
\end{definition}
\begin{example}
$\displaystyle \mathcal{P} =\{exp( \theta ) ,\theta  >0\}$ -- семейство экспоненциальных распределений, параметризованных $\displaystyle \theta $.
\end{example}
\section{Статистики и оценки}

Пусть $\displaystyle (\mathcal{X} ,\mathcal{B}_{\mathcal{X}} ,\mathcal{P})$ -- вероятностно-статистическая модель, $\displaystyle X$ -- наблюдение, $\displaystyle ( E,\mathcal{E})$ -- измеримое пространство.
\begin{definition}
Измеримое отображение $\displaystyle S:X\rightarrow E$ называется статистикой от наблюдения $\displaystyle X$.
\end{definition}
\begin{definition}
Если $\displaystyle E=\Theta $ ($\displaystyle E=\tau ( \theta )$), то $\displaystyle S$ -- оценка параметра $\displaystyle \theta $ ($\displaystyle \tau ( \theta )$).
\end{definition}
\begin{example}
Пусть $\displaystyle X=( X_{1} ,\dotsc ,X_{n})$ -- выборка из распределения в $\displaystyle \mathbb{R}^{m}$.
\begin{enumerate}
    \item Если $\displaystyle g( x)$ -- борелевская функция, то $\displaystyle \overline{g( x)} =\frac{1}{n}\sum _{i=1}^{n} g( X_{i})$ называется выборочной характеристикой функции $\displaystyle g$.
    
    \item $\displaystyle \overline{X} =\frac{1}{n}\sum _{i=1}^{n} X_{i}$ -- выборочное среднее.
    
    \item Для $\displaystyle m=1\ \overline{X^{k}} =\frac{1}{n}\sum _{i=1}^{n} X_{i}^{k}$ -- выборочный момент $\displaystyle k$-го порядка.
\end{enumerate}
\end{example}
Приведем примеры функций \ от выборочных характеристик $\displaystyle S( X) =h\left(\overline{g_{1}( X)} ,\dotsc \overline{g_{k}( X)}\right)$, где $\displaystyle h$ -- борелевская функция.
\begin{example}
Все для $\displaystyle m=1$
\begin{enumerate}
    \item $\displaystyle S^{2} =\overline{X^{2}} -(\overline{X})^{2}$ -- выборочная дисперсия
    
    \item $\displaystyle M_{k} =\frac{1}{n}\sum _{i=1}( X_{i} -\overline{X})^{k}$ -- выборочный центральный момент $\displaystyle k$-го порядка.
\end{enumerate}
\end{example}
\begin{exercise}
Показать, что $\displaystyle S^{2} =M_{2}$.
\end{exercise}
\begin{exercise}
Показать, что $\displaystyle M_{k}$ -- функция от выборочных характеристик.
\end{exercise}
\begin{exercise}
Что будет аналогами в случае $\displaystyle m >1$?
\end{exercise}
\begin{definition}
Для $\displaystyle m=1$ определим $\displaystyle X_{( 1)} =\min( X_{1} ,\dotsc ,X_{n}),\ X_{( 2)} =\min(\{X_{1},\dotsc, X_{n}\} \backslash \{X_{( 1)}\}),\ \dotsc,\ X_{( n)} =\max( X_{1}, \dotsc ,X_{n})$. Тогда $\displaystyle X_{( k)}$ называется $\displaystyle k$-ой порядковой статистикой выборки $\displaystyle X$, а $\displaystyle ( X_{( 1)} ,\dotsc X_{( n)})$ называется вариационным рядом.
\end{definition}
\section{Свойства оценок}

Пусть $\displaystyle X=( X_{1} ,\dotsc ,X_{n})$ -- выборка из неизвестного распределения $\displaystyle P\in \{P_{\theta } ,\theta \in \Theta \}$, где $\displaystyle \Theta \subset \mathbb{R}^{k}$.
\begin{definition}
Оценка $\displaystyle \theta ^{*}( X)$ называется несмещенной, оценкой параметра $\displaystyle \theta $, если $\displaystyle \forall \theta \in \Theta \ E_{\theta } \theta ^{*}( X) =\theta $, где $\displaystyle E_{\theta }$ -- математическое ожидание, в случае, когда элементы выборки имеют распределение $\displaystyle P_{\theta }$.
\end{definition}
\begin{example}
$\displaystyle \{N( \theta ,1) ,\theta \in \mathbb{R}\}$, тогда $\displaystyle \overline{X}$ и $\displaystyle X_{1}$ являются несмещенными оценками параметра $\displaystyle \theta $.
\end{example}
Пусть $\displaystyle X_{1} ,\dotsc ,X_{n} ,\dotsc $ -- выборка неограниченного размера из распределения $\displaystyle P\in \{P_{\theta } ,\theta \in \Theta \} ,\Theta \subset \mathbb{R}^{k}$.
\begin{definition}
Оценка $\displaystyle \theta _{n}^{*}( X_{1} ,\dotsc ,X_{n})$ (а точнее, последовательность оценок) называется состоятельной, если $\displaystyle \forall \theta \in \Theta \ \theta _{n}^{*}\xrightarrow{P_{\theta }} \theta $, т.е.


\begin{equation*}
\forall \varepsilon  >0\ \lim _{n\rightarrow \infty } P_{\theta }\left(\left\Vert \theta _{n}^{*}( X) -\theta \right\Vert _{2}  >\varepsilon \right) =0.
\end{equation*}
\end{definition}
\begin{definition}
$\displaystyle \theta _{n}^{*}( X_{1} ,\dotsc ,X_{n})$ называется сильно состоятельной оценкой параметра $\displaystyle \theta $, если $\displaystyle \theta _{n}^{*}\xrightarrow{P_{\theta } -п.н.} \theta $, т.е.


\begin{equation*}
P_{\theta }\left( \theta _{n}^{*}( X)\rightarrow \theta \right) =1\ \forall \theta \in \Theta .
\end{equation*}
\end{definition}
\begin{example}
$\displaystyle \{N( \theta ,1) ,\theta \in \mathbb{R}\}$, $\displaystyle \overline{X}$ -- сильно состоятельная оценка параметра $\displaystyle \theta $ по УЗБЧ.
\end{example}
Пусть $\displaystyle X_{i}$ -- случайные величины образуют выборку.
\begin{definition}
Оценка $\displaystyle \theta _{n}^{*}( X_{1} ,\dotsc ,X_{n})$ называется асимптотически нормальной оценкой параметра $\displaystyle \theta $, если


\begin{equation*}
\sqrt{n}\left( \theta _{n}^{*}( X) -\theta \right)\xrightarrow{d_{\theta }} N\left( 0,\sigma ^{2}( \theta )\right) ,
\end{equation*}


где $\displaystyle \sigma ^{2}( \theta )$ называется асимптотической дисперсией.
\end{definition}
\begin{note}
В многомерном случае вместо асимптотической дисперсии появляется асимптотическая матрица ковариаций $\displaystyle \Sigma ( \theta )$.
\end{note}
\begin{example}
$\displaystyle \{N( \theta ,1) ,\theta \in \mathbb{R}\} ,\overline{X}$ является асимптотически нормальной оценкой параметра $\displaystyle \theta $ по ЦПТ.
\end{example}
\begin{note}
Свойства состоятельности, сильной состоятельности и асимптотической нормальности называются асимптотическими свойствами и имеют смысл только при растущем объеме выборки.
\end{note}
\begin{note}
Также, можно оценивать $\displaystyle \tau ( \theta )$. Все свойства оценок определяются аналогично.
\end{note}
\begin{theorem} \
\begin{enumerate}
    \item $\displaystyle \theta _{n}^{*}$ -- сильно состоятельная оценка параметра $\displaystyle \tau ( \theta )$$\displaystyle \Rightarrow $$\displaystyle \theta _{n}^{*}$ -- состоятельная оценка параметра $\displaystyle \tau ( \theta )$.
    
    \item \ $\displaystyle \theta _{n}^{*}$ -- асимптотически нормальная оценка параметра $\displaystyle \tau ( \theta )$$\displaystyle \Rightarrow $$\displaystyle \theta _{n}^{*}$ -- состоятельная оценка параметра $\displaystyle \tau ( \theta )$.
    
    \item Никакие другие импликации неверны.
\end{enumerate}

\end{theorem}
\begin{proof} \
\begin{enumerate}
    \item $\displaystyle \theta _{n}^{*}\xrightarrow{P_{\theta } -п.н.} \theta \Rightarrow \theta _{n}^{*}\xrightarrow{P_{\theta }} \theta $.
    
    \item Так как $\displaystyle \frac{1}{\sqrt{n}}\xrightarrow{P_{\theta }} 0$, то по лемме Слуцкого


\begin{equation*}
\frac{1}{\sqrt{n}}\sqrt{n}\left( \theta _{n}^{*}( X) -\tau ( \theta )\right)\xrightarrow{d_{\theta }} \xi \cdotp 0,\forall \theta \in \Theta ,
\end{equation*}


где $\displaystyle \xi \sim N\left( 0,\sigma ^{2}( \theta )\right)$. Тогда $\displaystyle \theta _{n}^{*}( X) -\tau ( \theta )\xrightarrow{d_{\theta }} 0\Rightarrow \theta _{n}^{*}( X) -\tau ( \theta )\xrightarrow{P_{\theta }} 0\Rightarrow \theta _{n}^{*}( X)\xrightarrow{P_{\theta }} \tau ( \theta )$.
\end{enumerate}
\end{proof}
 
\begin{exercise}
Доказать п.3.
\end{exercise}
\begin{exercise}
Верно ли, что квадрат несмещенной оценки несмещен?
\end{exercise}

\begin{proposition}
Пусть $\displaystyle \theta _{n}^{*}( X)$ -- (сильно) состоятельная оценка $\displaystyle \theta $, $\displaystyle \tau ( \theta ) :\mathbb{R}^{k}\rightarrow \mathbb{R}^{s}$ -- непрерывная функция. Тогда $\displaystyle \tau \left( \theta ^{*}\right)$ -- (сильно) состоятельная оценка $\displaystyle \tau ( \theta )$.
\end{proposition}
\textit{Доказательство}

$\displaystyle \theta _{n}^{*}\xrightarrow{P_{\theta } -п.н.} \theta $. По теореме о наследовании сходимостей $\displaystyle \tau \left( \theta _{n}^{*}\right)\xrightarrow{P_{\theta } -п.н.} \tau ( \theta )$.

$\displaystyle \theta _{n}^{*}\xrightarrow{P_{\theta }} \theta $. По теореме о наследовании сходимостей $\displaystyle \tau \left( \theta _{n}^{*}\right)\xrightarrow{P_{\theta }} \tau ( \theta )$.\qed 
\begin{lemma}
(о наследовании асимптотической нормальности) Пусть $\displaystyle \theta _{n}^{*}$ -- асимптотически нормальная оценка $\displaystyle \theta \in \Theta \subset \mathbb{R}$ с асимптотической дисперсией $\displaystyle \sigma ^{2}( \theta )$, $\displaystyle \tau ( \theta ) :\mathbb{R}\rightarrow \mathbb{R}$ дифференцируема $\displaystyle \forall \theta \in \Theta $. Тогда $\displaystyle \tau \left( \theta _{n}^{*}( X)\right)$ -- асимптотически нормальная оценка для $\displaystyle \tau ( \theta )$ с асимптотической дисперсией $\displaystyle \sigma ^{2}( \theta ) \cdotp ( \tau '( \theta ))^{2}$.
\end{lemma}
\textit{Доказательство}

Из определения асимптотической нормальности $\displaystyle \sqrt{n}\left( \theta _{n}^{*}( X) -\theta \right)\xrightarrow{d_{\theta }} \xi \ \sim \ N\left( 0,\sigma ^{2}( \theta )\right)$. Обозначим $\displaystyle \xi _{n} =\sqrt{n}\left( \theta _{n}^{*}( X) -\theta \right) ,b_{n} =\frac{1}{\sqrt{n}}$. Тогда по следствию из леммы Слуцкого


\begin{equation*}
\frac{\tau ( \theta +\xi _{n} b_{n}) -\tau ( \theta )}{b_{n}} =\sqrt{n}\left( \tau \left( \theta _{n}^{*}\right) -\tau ( \theta )\right)\xrightarrow{d_{\theta }} \xi \cdotp \tau '( \theta ) .
\end{equation*}
\qed 
\begin{exercise}
Показать, что несмещенность оценки не наследуется.
\end{exercise}
\begin{proposition}
(о наследовании асимптотической нормальности в многомерном случае, б/д) Пусть $\displaystyle \Theta \subset \mathbb{R}^{k} ,\tau :\mathbb{R}^{k}\rightarrow \mathbb{R}^{s}$ -- дифференцируема в точках $\displaystyle \Theta $, $\displaystyle \theta _{n}^{*}$ -- асимптотически нормальная оценка для $\displaystyle \theta $ с асимптотической ковариационной матрицей $\displaystyle \Sigma ( \theta )$. Тогда $\displaystyle \tau \left( \theta _{n}^{*}\right)$ -- асимптотически нормальная оценка для $\displaystyle \tau ( \theta )$ с асимптотической ковариационной матрицей $\displaystyle \left(\frac{\partial \tau }{\partial \theta }\right) \cdotp \Sigma ( \theta ) \cdotp \left(\frac{\partial \tau }{\partial \theta }\right)^{T}$.
\end{proposition}