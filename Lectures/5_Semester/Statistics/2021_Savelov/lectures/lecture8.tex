\begin{definition}
    Статистика $S(X)$ называется \textit{полной} для $\theta$, если из условия $E_\theta f(S(X)) = 0$ следует, что $\forall \theta \hookrightarrow f(S(X)) = 0\ \forall \theta\ P_\theta-a.s.$
\end{definition}
\begin{theorem}
    (Лемана-Шеффе, об оптимальной оценке) Пусть $\displaystyle T( X)$ -- полная достаточная статистика для $\displaystyle \{P_{\theta } ,\ \theta \in \Theta \} ,\ \phi ( T( X))$ -- несмещенная оценка для $\displaystyle \tau ( \theta )$. Тогда $\displaystyle \phi ( T( X))$ не хуже любой несмещенной оценки $\displaystyle \tau ( \theta )$ в среднеквадратичном подходе, и если $\displaystyle \phi ( T) \in L_{2}$, то $\displaystyle \phi ( T)$ -- оптимальная оценка.
\end{theorem}
\begin{proof}
    Пусть $\displaystyle \tilde{d}$ -- несмещенная оценка $\displaystyle \tau ( \theta )$. Тогда $\displaystyle \tilde{\phi }( T) =E\left(\tilde{d} \ |\ T\right)$ не хуже $\displaystyle \tilde{d}$ и является несмещенной. Значит, $\displaystyle E_{\theta }\left( \phi ( T) -\tilde{\phi }( T)\right) =0$. Возьмем $\displaystyle h=\phi -\tilde{\phi }$. Тогда $\displaystyle \forall \theta \ E_{\theta } h( T) =0$. По определению полной статистики $\displaystyle h( T) =0\ P_{\theta } -a.s.\ \forall \theta $, \ а следовательно, $\displaystyle \phi ( T) =\tilde{\phi }( T) \ ( P_{\theta } -a.s.)$.
    
    Пусть теперь $\displaystyle \phi ( T) \in L_{2}$. В силу следствия достаточно доказать, что $\displaystyle \phi ( T)$ -- единственная несмещенная $\displaystyle T$-измеримая оценка.
    
    Пусть $\displaystyle \psi ( T)$ -- несмещенная $\displaystyle T$-измеримая оценка. Тогда $\displaystyle E_{\theta }( \psi ( T) -\phi ( T)) =0\Rightarrow \psi ( T) =\phi ( T)$.
\end{proof}
\begin{corollary}
    Пусть $\displaystyle T( X)$ -- полная достаточная статистика, $\displaystyle d( X)$ -- несмещенная оценка для $\displaystyle \tau ( \theta ) \ ( d\in L_{1})$. Тогда $\displaystyle \phi =E_{\theta }( d\ |\ T)$ не хуже любой другой оценки в среднеквадратичном подходе. Если $\displaystyle \phi \in L_{2}$, то $\displaystyle \phi $ -- оптимальна.
\end{corollary}
\begin{proof}
    $\displaystyle E_{\theta }( d\ |\ T( X))$ -- несмещенная оценка и является функцией от $\displaystyle T$.
\end{proof}
\begin{theorem}
    (об экспоненциальном семействе, б/д) Пусть $\displaystyle X_{1} ,\ \dotsc ,\ X_{n}$ -- выборка из экспоненциального семейства распределений. Если множество значений вектор-функции $\displaystyle \forall \theta \ \overline{a}( \theta ) =\begin{pmatrix}
    a_{0}( \theta ) & \dotsc  & a_{k}( \theta )
    \end{pmatrix}$ из определения экспоненциального семейства содержит $\displaystyle k$-мерный параллелепипед в $\displaystyle \mathbb{R}^{k}$, то $\displaystyle T( X) =\begin{pmatrix}
    T_{1}( X) & \dotsc  & T_{k}( X)
    \end{pmatrix}$ -- полная достаточная статистика.
\end{theorem}
\begin{note}
    Для того, чтобы вектор-функция $\overline{a}(\theta)$ содержала $k$-мерный параллелепипед, можно потребовать, чтобы $\Theta$ содержало открытое множество, и чтобы $a_1(\theta),\, \ldots,\, a_k(\theta)$ были гладкими.
\end{note}

\begin{proposition}
    (Алгоритм поиска оптимальной оценки)
    \begin{enumerate}
    \item Ищем достаточную статистику
    \item Проверяем полноту
    \item Если полна, то решаем уравнение несмещенности: $\displaystyle E_{\theta } g( T( X)) =\tau ( \theta ) \ \forall \theta $.
    \end{enumerate}
\end{proposition}

\section{Интервальное оценивание}

Пусть $\displaystyle X$ -- наблюдение с неизвестным распределением $\displaystyle \{P_{\theta } ,\ \theta \in \Theta \} ,\ \Theta \subset \mathbb{R}$. 

\begin{definition}
    Пара статистик $\displaystyle ( T_{1}( X) ,T_{2}( X))$ называется доверительным интервалом уровня доверия $\displaystyle \gamma $ для параметра $\displaystyle \theta $, если $\displaystyle \forall \theta \in \Theta $ выполняется
\begin{equation*}
    P( T_{1}( X) < \theta < T_{2}( X)) \geqslant \gamma .
\end{equation*}
Если равенство достигается при всех $\displaystyle \theta $, то доверительный интервал называется точным.
\end{definition}
\begin{note}
    На практике используют $\displaystyle \gamma =0,9;\ 0,95;\ 0,99$.
\end{note}
Иногда удобно использовать односторонние доверительные интервалы вида $\displaystyle ( -\infty ,\ T_{2}( X))$ или $\displaystyle ( T_{1}( X) ,\ +\infty )$.

В случае многомерного $\displaystyle \theta \in \Theta \subset \mathbb{R}^{k}$ можно определить понятие доверительного интервала для компонент вектора $\displaystyle \theta _{i}$ вектора $\displaystyle \theta =\begin{pmatrix}
\theta _{1} & \dotsc  & \theta _{k}
\end{pmatrix}$ и для скалярных функций $\displaystyle \tau ( \theta )$.
\begin{definition}
    Множество $\displaystyle S\subset \Theta $ называется доверительным множеством уровня доверия $\displaystyle \gamma $, если $\displaystyle \forall \theta \in \Theta \hookrightarrow P_{\theta }( \theta \in S( X)) \geqslant \gamma $.
\end{definition}
\subsection{Метод центральной статистики}


\begin{definition}
    Предположим, что существует известная одномерная функция $\displaystyle G( X,\theta )$, такая что ее распределение не зависит от параметра $\displaystyle \theta $. Тогда такая функция называется центральной статистикой.
\end{definition}
Пусть $\displaystyle \gamma _{1} ,\ \gamma _{2} \in ( 0,\ 1)$ таковы, что $\displaystyle \gamma _{2} -\gamma _{1} =\gamma $, и при $\displaystyle i=1,\, 2$ существует $\displaystyle g_{i}$ -- $\displaystyle \gamma _{i}$-квантиль $\displaystyle G( X,\ \Theta )$. Тогда $\displaystyle \forall \theta \in \Theta $
\begin{equation*}
    P_{\theta }( g_{1} \leqslant G( X,\ \theta ) \leqslant g_{2}) \geqslant \gamma _{2} -\gamma _{1} =\gamma .
\end{equation*}
Тогда $\displaystyle S( X) :=\{\theta \in \Theta :\ g_{1} \leqslant G( X,\ \theta ) \leqslant g_{2}\} \Rightarrow P_{\theta }( \theta \in S( X)) =P_{\theta }( g_{1} \leqslant G( X,\ \theta ) \leqslant g_{2}) \geqslant \gamma $, т.е. $\displaystyle S( X)$ -- доверительное множество уровня $\displaystyle \gamma $.
\begin{example}
$\displaystyle X_{i} \ \sim \ \mathcal{N}( \theta ,\ 1) ,\ \theta \in \mathbb{R}$. Строим доверительный интервал \ для $\displaystyle \theta $. Так как $\displaystyle \overline{X} \ \sim \ \mathcal{N}\left( \theta ,\ \frac{1}{n}\right)$, то $\displaystyle \sqrt{n}(\overline{X} -\theta ) \ \sim \ \mathcal{N}( 0,\ 1)$. Тогда $\displaystyle G( X,\ \theta ) =\sqrt{n}(\overline{X} -\theta )$ -- центральная статистика. Пусть $\displaystyle u_{p}$ -- $\displaystyle p$-квантиль стандартного нормального распределения. Тогда
\begin{equation*}
\forall \theta \hookrightarrow P_{\theta }\left( u_{\frac{1-\gamma }{2}} \leqslant \sqrt{n}(\overline{X} -\theta ) \leqslant u_{\frac{1-\gamma }{2}}\right) =P\left(\sqrt{n}| \overline{X} \ -\ \theta | \leqslant u_{\frac{1+\gamma }{2}}\right) .
\end{equation*}
Тогда $\displaystyle \left(\overline{X} -\frac{1}{\sqrt{n}} u_{\frac{1-\gamma }{2}} ,\ \overline{X} +\frac{1}{\sqrt{n}} u_{\frac{1+\gamma }{2}}\right)$ -- доверительный интервал для $\displaystyle \theta $.
\end{example}
\begin{lemma}
$\displaystyle X_{1} ,\ \dotsc ,\ X_{n}$ -- независимые одинаково распределенные случайные величины с непрерывной функцией распределения $\displaystyle F( x)$. Тогда $\displaystyle G( X_{1} ,\ \dotsc ,\ X_{n}) =-\sum _{i=1}^{n}\ln F( x_{i}) \ \sim \ \Gamma ( 1,\ n)$.
\end{lemma}
\begin{proof}
Покажем, что $\displaystyle F( x_{i}) \, \sim \, \mathcal{U}[ 0,\ 1]$.


\begin{equation*}
P( F( x_{i}) \leqslant y) =P\left( x_{i} \leqslant F^{-1}( y)\right) ,
\end{equation*}
где $\displaystyle y\in ( 0,\ 1) ,\ F^{-1}( y) =\min( x:\ F( x) =y)$. Следовательно, $\displaystyle F\left( F^{-1}( y)\right) =y$. Тогда


\begin{equation*}
-\ln F( x_{i}) \ \sim \ Exp( 1) \Rightarrow G( X_{1} ,\ \dotsc ,\ X_{n}) =-\sum _{i=1}^{n}\ln F( X_{i}) \ \sim \ \Gamma ( 1,\ n) .
\end{equation*}
\end{proof}
\begin{corollary}
Если $\displaystyle X_{1} ,\ \dotsc ,\ X_{n}$ -- выборка из распределения $\displaystyle P\in \{P_{\theta } ,\ \theta \in \Theta \}$, причем $\displaystyle \forall \theta $ функция распределения $\displaystyle F_{\theta }( x)$ непрерывна по $\displaystyle x$, тогда $\displaystyle G( X_{1} ,\ \dotsc ,\ X_{n} ,\ \theta ) =-\sum _{i=1}^{n}\ln F_{\theta }( X_{i})$ -- центральная статистика.
\end{corollary}