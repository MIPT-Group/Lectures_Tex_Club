\section{Комплексные числа}

\begin{definition}
	\textit{Множеством комплексных чисел} $\Cm$ называется множество $\R^2$ с арифметическими операциями, введенными следующим образом:
	\begin{itemize}
		\item $\forall (x_1, y_1), (x_2, y_2) \in \Cm: (x_1, y_1) + (x_2, y_2) := (x_1 + x_2, y_1 + y_2)$
		\item $\forall (x_1, y_1), (x_2, y_2) \in \Cm: (x_1, y_1)\cdot(x_2, y_2) := (x_1x_2 - y_1y_2, x_1y_2 + x_2y_1)$
	\end{itemize}
\end{definition}

\begin{note}
	В предыдущих курсах уже доказывалось, что комплексные числа с определенными выше арифметическими операциями образуют \textit{поле}. Отметим также, что имеет место естественное вложение $\R \emb \Cm$, осуществляемое сопоставлением $x \mapsto (x, 0)$, поэтому числа вида $(x, 0) \in \Cm$ мы далее будем обозначать через $x$.
\end{note}

\begin{definition}
	\textit{Модулем} числа $z = (x, y) \in \Cm$ называется величина $|z| := \sqrt{x^2 + y^2}$.
\end{definition}

\begin{note}
	Модуль комплексного числа "--- это евклидова норма на $\R^2$. Для евклидовой нормы в предыдущих курсах доказывалось \textit{неравенство треугольника}, согласно которому для любых $z_1, z_2 \in \Cm$ выполнено неравенство $|z_1 + z_2| \le |z_1| + |z_2|$.
\end{note}

\begin{definition}
	Элемент $i = (0, 1) \in \Cm$ называется \textit{мнимой единицей}.
\end{definition}

\begin{note}
	Из определения выше следует, что $i^2 = -1$. Другой способ построения поля комплексных чисел предполагает расширение поля $\R$ корнем многочлена $x^2 + 1$.
\end{note}

\begin{definition}
	\textit{Алгебраической формой записи} комплексного числа $z = (x, y) \in \Cm$ называется выражение $x + iy$. \textit{Вещественной частью} числа $z$ называется число $\re{z} := x$, \textit{мнимой частью} --- число $\im{z} := y$. Комплексные числа с нулевой вещественной частью называются \textit{чисто мнимыми}.
\end{definition}

\begin{definition}
	Числом, \textit{сопряженным} к числу $z = x + iy \in \Cm$, называется число $\overline{z} := x - iy$. Сопоставление $z \mapsto \overline{z}$ называется \textit{сопряжением}.
\end{definition}

\begin{note}
	Нетрудно проверить, что операция сопряжения является автоморфизмом поля $\Cm$. В частности, она коммутирует со всеми арифметическими операциями в $\Cm$. Отметим также, что для любого $z \in \Cm$ выполнено $|z|^2 = z\overline{z}$.
\end{note}

\begin{proposition}[мультипликативность модуля]
	Для любых $z_1, z_2 \in \Cm$ выполнено равенство $|z_1z_2| = |z_1||z_2|$.
\end{proposition}

\begin{proof}
	Заметим, что выполнена следующая цепочка равенств:
	\[|z_1z_2|^2 = z_1z_2\overline{z_1z_2} = z_1\overline{z_1}z_2\overline{z_2} = |z_1|^2|z_2|^2\]
	
	Извлекая корень из обеих частей равенства $|z_1z_2|^2 = |z_1|^2|z_2|^2$, получим требуемое.
\end{proof}

\begin{proposition}[неравенство Коши]
	Для любых $z_1, \dotsc, z_n, w_1, \dotsc, w_n \in \Cm$ выполнено следующее неравенство:
	\[\left|\sum_{k=1}^nz_kw_k\right|^2 \le \left(\sum_{k=1}^n|z_k|^2\right)\left(\sum_{k=1}^n|w_k|^2\right)\]
\end{proposition}

\begin{proof}
	Положим $A := \sum_{k=1}^n|z_k|^2$, $B := \sum_{k=1}^n|w_k|^2$, $C := \sum_{k=1}^nz_kw_k$. Случай, когда $B = 0$, тривиален, поэтому рассмотрим случай, когда $B > 0$. Заметим следующее:
	\begin{multline*}
		0 \le \sum_{k = 1}^n|Bz_k - C\overline{w_k}|^2 = \sum_{k = 1}^n(Bz_k - C\overline{w_k})(B\overline{z_k} - \overline Cw_k) =
		\\
		= B^2\sum_{k=1}^n|z_k|^2 - BC\sum_{k=1}^n\overline{z_k}\overline{w_k} - B\overline{C}\sum_{k=1}^nz_kw_k + |C|^2\sum_{k=1}^n|w_k|^2 = B^2A - B|C|^2
	\end{multline*}
	
	Разделив обе части неравенства $0 \le B^2A - B|C|^2$ на $B > 0$, получим требуемое.
\end{proof}

\begin{definition}
	\textit{Комплексной плоскостью} называется плоскость $\R^2$ с введенной на ней прямоугольной декартовой системой координат, в которой каждое комплексное число $z = x + iy \in \Cm$ имеет координаты $(x, y)$.
\end{definition}

\begin{note}
	На рисунке ниже изображена комплексная плоскость, число $z \in \Cm$ на ней, а также числа $-z$ и $\overline{z}$.
	\begin{center}
		\scalebox{1}{
			\begin{tikzpicture}
				\clip (-3.4, -3) rectangle (3.4, 3);
				\draw [->] (-3, 0) -- (3, 0) node [above, black] {$\re z$};
				\draw [->] (0, -2.7) -- (0, 2.7) node [right, black] {$\im z$};
				
				\draw [black] (1.4,3pt) -- (1.4,-3pt) node [below, black] {$1$};
				\draw [black] (3pt,1.4) -- (-3pt,1.4) node [left, black] {$i$};
				
				\draw [->, black] (0, 0) -- (2, 1.5) node [black, above right, scale = 1.2] {$z$};
				\node[draw, circle, inner sep=1pt, fill, black] at (2.06, 1.54) {};
				
				\draw [->, black] (0, 0) -- (-2, -1.5) node [black, below left, scale = 1.2] {$-z$};
				\node[draw, circle, inner sep=1pt, fill, black] at (-2.06, -1.54) {};
				
				\draw [->, black] (0, 0) -- (2, -1.5) node [black, below right, scale = 1.2] {$\overline z$};
				\node[draw, circle, inner sep=1pt, fill, black] at (2.06, -1.54) {};
				
				\coordinate (a1) at (2, 1.5);
				\coordinate (a2) at (2, -1.5);
				\coordinate (a3) at (-2, -1.5);
				\coordinate (b) at (0, 0);
				\coordinate (c) at (1, 0);
				\coordinate (c2) at (-1, 0);
				
				\pic [draw, ->] {angle = c--b--a1};
				\pic [draw, <-, angle radius = 0.7cm] {angle = a2--b--c};
				\pic [draw, ->] {angle = c2--b--a3};
				\node [] at (0.75, 0.25) {$\phi$};
				\node [] at (0.9, -0.3) {$\phi$};
				\node [] at (-0.75, -0.25) {$\phi$};
			\end{tikzpicture}
		}
	\end{center}
\end{note}

\section{Тригонометрическая форма комплексного числа}

\begin{definition}
	\textit{Аргументом} числа $z \in \Cm \backslash \{0\}$ называется следующий набор чисел:
	\[\Arg{z} := \left\{\phi \in \R: \cos\phi = \frac{\re z}{|z|}, \sin\phi = \frac{\im z}{|z|}\right\}\]
	
	Любой элемент множества $\Arg{z}$ обозначается через $\arg{z}$. \textit{Главным значением аргумента} числа $z$ называется единственное число $\arg_0{z}$ такое, что $\arg_0{z} \in \Arg{z} \cap (-\pi, \pi]$.
\end{definition}

\begin{note}
	Легко видеть, что $\forall z \in \Cm \bs \{0\}: \Arg{z} = \{\arg_0{z} + 2\pi k: k \in \Z\}$. Отметим также, что для любых $z_1, z_2 \in \Cm$ выполнены следующие равенства:
	\begin{gather*}
		|z_1z_2| = |z_1||z_2|\\
		\arg{z_1z_2} = \arg{z_1} + \arg{z_2}
	\end{gather*}
	
	Первое равенство выполнено в силу мультипликативности модуля, второе же легко проверить, используя формулы тригонометрии.
\end{note}

\begin{definition}
	\textit{Тригонометрической формой записи} комплексного числа $z \in \Cm \bs \{0\}$ с модулем $r$ и значением аргумента $\phi$ называется выражение $r(\cos\phi + i\sin\phi)$.
\end{definition}

\begin{note}
	Рассмотрим числа $z_1, z_2 \in \Cm \bs \{0\}$ с алгебраическими записями $x_1 + iy_1, x_2 + iy_2$ и тригонометрическими записями $r_1(\cos\phi_1 + i\sin\phi_1), r_2(\cos\phi_2 + i\sin\phi_2)$. Тогда выполнены следующие эквивалентности:
	\[z_1 = z_2 \lra \System{x_1 = x_2 \\ y_1 = y_2} \lra \System{&r_1 = r_2 \\ &\exists k \in \Z: \phi_1 = \phi_2 + 2\pi k}\]
\end{note}

\section{Комплексная экспонента}

\begin{definition}
	\textit{Комплексной экспонентой} называется функция $f : \Cm \to \Cm$, сопоставляющая каждому числу $z = x + iy \in \Cm$ число следующего вида:
	\[f(z) = e^z = e^{x + iy} := e^x(\cos{y} + i\sin{y})\]
\end{definition}

\begin{proposition}[свойства комплексной экспоненты]~
	\begin{enumerate}
		\item Определение комплексной экспоненты от действительного числа согласованно со стандартным определением экспоненты.
		\item $\forall z \in \Cm: e^z \ne 0$.
		\item $\forall z_1, z_2 \in \Cm: e^{z_1}e^{z_2} = e^{z_1 + z_2}$.
		\item $\forall z \in \Cm: e^{-z} = \frac{1}{e^z}$.
		\item $\forall z \in \Cm: \forall n \in \N: (e^{z})^n = e^{zn}$.
	\end{enumerate}
\end{proposition}

\begin{proof}~
	\begin{enumerate}
		\item Следует непосредственно из определения.
		\item Заметим, что для любого числа $w \in \Cm$ выполнено $w = 0 \lra |w| = 0$. Но для любого числа $z = x + iy \in \Cm$ выполнено $|e^z| = e^x$, поэтому $e^z \ne 0$.
		\item Пусть $z_1 = x_1 + iy_1$, $z_2 = x_2 + iy_2$. Тогда выполнена следующая цепочка равенств:
		\begin{multline*}
			e^{z_1}e^{z_2} = e^{x_1}(\cos{y_1} + i\sin{y_1})e^{x_2}(\cos{y_2} + i\sin{y_2}) = \\
			= e^{x_1 + x_2}(\cos(y_1 + y_2) + i\sin(y_1 + y_2)) = e^{z_1 + z_2}
		\end{multline*}
		\item Следует непосредственно из свойства $(3)$.
		\item Проведем индукцию по $n$. База, $n = 1$, тривиальна. Докажем переход, пользуясь свойством $(3)$:
		\[\left(e^z\right)^n = \left(e^z\right)^{n-1}e^z = e^{z(n-1)}e^z = e^{z(n-1) + z} = e^{zn}\qedhere\]
	\end{enumerate}
\end{proof}

\begin{definition}
	Пусть $E \subset \Cm$, $T \in \Cm \bs \{0\}$. Функция $f : E \to \Cm$ называется \textit{периодической с периодом $T$}, если выполнены следующие условия:
	\begin{enumerate}
		\item $\forall z \in E: z \pm T \in E$
		\item $\forall z \in E: f(z + T) = f(z)$
	\end{enumerate}
\end{definition}

\begin{proposition}
	Любое число вида $2\pi k i, k \in \Z$, является периодом комплексной экспоненты, причем других периодов у нее нет.
\end{proposition}

\begin{proof}
	С одной стороны, для любого $k \in \Z$ выполнено равенство $e^{z + 2\pi ki} = e^z$. С другой стороны, если $T = T_1 + iT_2 \in \Cm$ "--- период комплексной экспоненты, то, в частности, $e^{T_1 + iT_2} = e^0 = 1$, откуда $T_1 = 0$ и $T_2 = 2\pi k$ для некоторого $k \in \Z$.
\end{proof}

\begin{definition}
	\textit{Показательной формой записи} комплексного числа $z \in \Cm \bs \{0\}$ с модулем $r$ и значением аргумента $\phi$ называется выражение $re^{i\phi}$.
\end{definition}

\section{Логарифм комплексного числа}

\begin{definition}
	\textit{Логарифмом} числа $z \in \Cm \bs \{0\}$ называется следующий набор чисел:
	\[\Ln{z} := \left\{w \in \Cm : e^w = z\right\}\]
	
	Любой элемент множества $\Ln{z}$ обозначается через $\ln{z}$.
\end{definition}

\begin{proposition}
	$\forall z = re^{i\phi} \in \Cm \bs \{0\} : \Ln{z} = \{ \ln{r} + i(\phi + 2\pi k) : k \in \Z\}$.
\end{proposition}

\begin{proof}
	С одной стороны, для любого $k \in \Z$ выполнены следующие равенства: \[e^{\ln{r} + i(\phi + 2\pi k)} = re^{i\phi} = z\]
	
	С другой стороны, если $w = u + iv \in \Ln{z}$, то $e^{u + iv} = z$, откуда $e^u = r \lra u = \ln{r}$ и $v = 2\pi k$ для некоторого $k \in \Z$.
\end{proof}

\begin{definition}
	\textit{Главным значением логарифма} числа $z \in \Cm \bs \{0\}$ называется число $\ln_0{z} := \ln|z| + i\arg_0{z}$.
\end{definition}

\section{Корень из комплексного числа}

\begin{definition}
	\textit{Множеством корней степени $n \in \N$, $n > 1$}, из числа $z \in \Cm$ называется следующий набор чисел:
	\[\Root{n}{z} := \left\{w \in \Cm: w^n = z\right\}\]
\end{definition}

\begin{note}
	Легко видеть, что $\forall n \in \N$, $n > 1: \Root{n}{0} = \{0\}$.
\end{note}

\begin{proposition}
	$\forall z = re^{i\phi} \in \Cm \bs \{0\}: \forall n \in \N$, $n > 1: \Root{n}{z} = \left\{\!\sqrt[n]{r}e^{i\frac{\phi + 2\pi k}{n}} : k \in \Z\right\}$.
\end{proposition}

\begin{proof}
	С одной стороны, для любого $k \in \Z$ выполнено следующие равенства:
	\[\left(\!\sqrt[n]{r}e^{i\frac{\phi + 2\pi k}{n}}\right)^n = re^{i\phi} = z\]
	
	С другой стороны, если $w \in \Root{n}{z}$, то $w \ne 0$, тогда $w$ можно представить в показательной форме как $w = \rho e^{i\theta}$, причем $\rho^ne^{i\theta n} = re^{u\phi}$, откуда $\rho = \sqrt[n]{r}$ и $\theta n = \phi + 2\pi k$ для некоторого $k \in \Z$.
\end{proof}

\begin{note}
	В силу периодичности комлпексной экспоненты, чтобы получить все корни степени $n \in \N$, $n > 1$, по формуле из утверждения выше, достаточно взять все $k$ из множества $\{0, \dotsc, n - 1\}$, причем все полученные корни будут различными. На комплексной плоскости $n$ корней образуют правильный $n$-угольник.
\end{note}

\begin{definition}
	\textit{Главным значением корня степени} $n \in \N$, $n > 1$, из числа $z \in \Cm \bs \{0\}$ называется следующее число:
	\[\left(\!\sqrt[n]{z}\right)_0 := \sqrt[n]{|z|}e^{i\frac{\arg_0z}n}\]
\end{definition}