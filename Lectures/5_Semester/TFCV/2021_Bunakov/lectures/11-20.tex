\section{Дифференцируемость и регулярные функции}

\begin{definition}
	Пусть функция $f$ определена в некоторой окрестности $B_{\delta_0}(z_0)$ точки  $z_0 \in \Cm$. Функция $f$ называется \textit{дифференцируемой в точке $z_0$}, если в окрестности $\dot B_{\delta_0}{(0)}$ определена функция $\alpha$ такая, что $\lim_{\Delta z \to 0}\alpha(\Delta z) = 0$, и для некоторого $A \in \Cm$ выполнено условие $\forall \Delta z \in \dot B_{\delta_0}{(0)}: f(z_0 + \Delta z) - f(z_0) = A\Delta z + |\Delta z|\alpha(\Delta z)$
\end{definition}

\begin{theorem}
	Пусть $f(z) = f(x + iy) = u(x, y) + iv(x, y)$. Тогда $f$ дифференцируема в точке $z_0 = x + iy_0 \in \Cm \lra$ $u, v$ дифференцируемы в точке $(x_0, y_0)$ и выполнены следующие равенства, называемые условиями Коши-Римана:
	\[\System{u'_x(x_0, y_0) &= v'_y(x_0, y_0) \\ u'_y(x_0, y_0) &= -v'_x(x_0, y_0)}\]
\end{theorem}

\begin{proof}~
	\begin{itemize}
		\item[$\ra$] Пусть $\Delta z = \Delta x + i \Delta y$, $\alpha(\Delta z) = \beta(\Delta x, \Delta y) + i\gamma(\Delta x, \Delta y)$, $A = B + iC$. По определению дифференцируемости, $\lim_{(\Delta x, \Delta y) \to (0, 0)}\beta(\Delta x, \Delta y) = 0$ и $\lim_{(\Delta x, \Delta y) \to (0, 0)}\gamma(\Delta x, \Delta y) = 0$. Тогда:
		\begin{multline*}
			u(x + \Delta x, y + \Delta y) + iv(x + \Delta x, y + \Delta y) - u(x, y) - iv(x, y) =  A\Delta z + |\Delta z|\alpha(\Delta z) =
			\\
			= (B + iC)(\Delta x + i\Delta y) + \sqrt{(\Delta x)^2 + (\Delta y)^2}(\beta(\Delta x, \Delta y) + i\gamma(\Delta x, \Delta y))
		\end{multline*}
		
		Приравнивая действительную и мнимую части по отдельности, получим следующее:
		\[\System{
			u(x + \Delta x, y + \Delta y) - u(x, y) &=  B\Delta x - C\Delta y + \sqrt{(\Delta x)^2 + (\Delta y)^2}\beta(\Delta x, \Delta y)
			\\
			v(x + \Delta x, y + \Delta y) - v(x, y) &= C\Delta x + B\Delta y + \sqrt{(\Delta x)^2 + (\Delta y)^2}\gamma(\Delta x, \Delta y)
		}\]
	
		Значит, $u, v$ дифференцируемы в точке $(x_0, y_0)$. Кроме того, $B = u'_x(x_0, y_0) = v'_y(x_0, y_0)$ и $C = v'_x(x_0, y_0) = -u'_y(x_0, y_0)$.
		
		\item[$\la$] Положим $B := u'_x(x_0, y_0) = v'_y(x_0, y_0)$, $C := v'_x(x_0, y_0) = -u'_y(x_0, y_0)$, тогда снова выполнена система из доказательства $\ra$. Умножая второе равенство системы на $i$ и складывая с первым, получим требуемое.\qedhere
	\end{itemize}
\end{proof}

\begin{note}
	Легко видеть, что дифференцируемость функции $f$ в точке $z_0 \in \Cm$ равносильна существованию следующего предела:
	\[\lim_{\Delta z \to 0}\frac{f(z_0 + \Delta z) - f(z_0)}{\Delta z} = A \in \Cm\]
	
	Число $A$ "--- это в точности число из определения дифференцируемости.
\end{note}

\begin{definition}
	Пусть функция $f$ дифференцируема в точке $z_0 \in \Cm$. Значение предела $\lim_{\Delta z \to 0}\frac{f(z_0 + \Delta z) - f(z_0)}{\Delta z}$ называется \textit{производной} функции $f$ в $z_0$ и обозначается через $f'(z_0)$.
\end{definition}

\begin{note}
	Из определения производной и теоремы выше легко видеть, что если $f$ дифференцируема в точке $z_0 = x_0 + iy_0 \in \Cm$ и $f(z) = f(x + iy) = u(x, y) + iv(x, y)$, то выполнены следующие равенства:
	\begin{multline*}
		f'(z_0) = u'_x(x_0, y_0) + iv'_x(x_0, y_0) = v'_y(x_0, y_0) + iv'_x(x_0, y_0) =
		\\
		= v'_y(x_0, y_0) - iu'_y(x_0, y_0) = u'_x(x_0, y_0) - iu'_y(x_0, y_0)
	\end{multline*}
\end{note}

\begin{note}
	Пусть функции $f, g$ дифференцируемы в точке $z_0 \in \Cm$. Аналогично вещественному случаю, можно доказать, что тогда функции $f + g$ и $fg$ тоже дифференцируемы в точке $z_0$, причем выполнены следующие равенства:
	\begin{gather*}
		(f + g)'(z_0) = f'(z_0) + g'(z_0)\\
		(fg)'(z_0) = f'(z_0)g(z_0) + f(z_0)g'(z_0)
	\end{gather*}
	
	Если при этом $g(z_0) \ne 0$, то и функция $\frac fg$ дифференцируема в точке $z_0$, причем выполнено следующее равенство:
	\[
	\left(\frac fg\right)'(z_0) = \frac{f'(z_0)g(z_0) - f(z_0)g'(z_0)}{g(z_0)^2}\]
\end{note}

\begin{note}
	Пусть функция $f$ дифференцируема в точке $z_0 \in \Cm$, $f(z_0) := w_0 \in \Cm$, функция $g$ дифференцируема в точке $w_0$. Аналогично вещественному случаю, можно доказать, что тогда функция $h := g \circ f$ дифференцируема в точке $z_0$, причем выполнено следующее равенство:
	\[h'(z_0) = g'(w_0)f'(z_0)\]
	
	Наконец, упомянутые свойства позволяют заключить, что элементарные функции дифференцируемы в любой точке из $\Cm$.
\end{note}

\begin{example}
	Проверка условий Коши-Римана для функций ниже дает следующее:
	\begin{enumerate}
		\item $f(z) =\overline{z}$ не дифференцируема ни в одной точке из $\Cm$
		\item $f(z) = |z|^2 = z\overline{z}$ дифференцируема только в точке $0$
		\item $f(z) = e^z$ дифференцируема в любой точке из $\Cm$
	\end{enumerate}
\end{example}

\begin{definition}
	Функция $f$ называется \textit{регулярной} на открытом множестве $G \subset \Cm$, если выполнены следующие условия:
	\begin{enumerate}
		\item $f$ дифференцируема в любой точке из $G$
		\item $f'$ непрерывна в любой точке из $G$
	\end{enumerate}

	Обозначение "--- $f \in C^1(G)$.
\end{definition}

\begin{definition}
	Функция $f$ называется \textit{регулярной} на множестве $A \subset \Cm$, если $A$ включено в открытое множество $G \subset \Cm$ такое, что $f \in C^1(G)$. Обозначение "--- $f \in C^1(A)$.
\end{definition}

\begin{note}
	В частности, функция $f$ регулярна в точке $z_0 \in \Cm$, если $f$ регулярна в достаточно малой окрестности $B_{\delta_0}(z_0)$ этой точки.
\end{note}

\begin{note}
	В силу уже полученных свойств дифференцируемости и непрерывности, справедливы утверждения, аналогичные утверждениям о дифференцируемости:
	\begin{itemize}
		\item Если $f, g$ регулярны в точке $z_0 \in \Cm$, то $f + g$, $fg$ тоже регулярны в $z_0$
		\item Если $f, g$ регулярны в точке $z_0 \in \Cm$ и $g(z_0) \ne 0$, то $\frac fg$ тоже регулярна в $z_0$
		\item Если $f$ регулярна в точке $z_0 \in \Cm$, $g$ регулярна в точке $w_0 := f(z_0) \in \Cm$, то $g \circ f$ тоже регулярна в $z_0$
	\end{itemize}
\end{note}

\section{Интеграл вдоль кривой}

\begin{definition}
	Пусть $\gamma$ "--- кусочно-гладкая кривая, $z = \sigma(t)$, $t \in [a, b]$ "--- ее параметризация, функция $f$ такова, что композиция $f\circ\sigma$ непрерывна на $[a, b]$.
	\begin{itemize}
		\item \textit{Интегралом первого рода} функции $f$ по кривой $\gamma$ называется следующая величина:
		\[\int_{\gamma}f(z)|dz| := \int_a^bf(\sigma(t))|\sigma'(t)|dt\]
		\item \textit{Интегралом второго рода} функции $f$ по кривой $\gamma$ называется следующая величина:
		\[\int_{\gamma}f(z)dz := \int_a^bf(\sigma(t))\sigma'(t)dt\]
	\end{itemize}
\end{definition}

\begin{note}
	Если $f(z) = f(x + iy) = u(x, y) + iv(x, y)$, $\sigma = \xi + i\eta$, то, как нетрудно убедиться, выполнены следующие равенства:
	\begin{gather*}
		\int_{\gamma}f(z)|dz| = \int_\gamma u(x, y)ds + i\int_\gamma v(x, y)ds
		\\
		\int_{\gamma}f(z)dz = \int_\gamma udx - vdy + i\int_\gamma vdx + udy
	\end{gather*}
\end{note}

\begin{theorem}
	Пусть для кривой $\gamma$ и функций $f, g$ выполнены условия в определении интеграла первого и второго рода. Тогда выполнены следующие свойства:
	\begin{enumerate}
		\item (\textit{Линейность}) Если $\alpha, \beta \in \Cm$, то:
		\[\int_{\gamma}(\alpha f+\beta g)dz = \alpha\int_\gamma fdz + \beta\int_\gamma gdz\]
		
		\item (\textit{Аддитивность}) Если $\gamma = \gamma_1\gamma_2$, то:
		\[\int_{\gamma}fdz = \int_{\gamma_1}fdz + \int_{\gamma_2}fdz\]
		
		\item (\textit{Независимость от параметризации}) Если $\phi : [\alpha, \beta] \to [a, b]$ "--- кусочно-гладкая строго возрастающая функция, то:
		\[\int_{\gamma}f(z)dz := \int_a^bf(\sigma(t))\sigma'(t)dt = \int_\alpha^\beta f(\sigma(\phi(u)))\sigma'(\phi(u))\phi'(u)du\]
		
		\item Аналогичные пунктам $(1), (2), (3)$ свойства выполнены для интегралов первого рода
		
		\item Если $\gamma^{-1}$ "--- кривая с параметризацией $z = \sigma(-t)$, $t \in [-b, -a]$, то:
		\[\int_{\gamma^{-1}}f|dz| = \int_\gamma f|dz|,\text{ но }\int_{\gamma^{-1}}fdz = -\int_\gamma fdz\]
		
		\item Если $M := \max_{z \in M(\gamma)}|f(z)|$, $|\gamma|$ "--- длина кривой $\gamma$, то:
		\[\left|\int_\gamma fdz\right| \le \int_\gamma|f||dz| \le M|\gamma|\]
	\end{enumerate}
\end{theorem}

\begin{proof}
	Справедливость всех свойств, кроме последнего, непосредственно следует из справедливости соответствующих свойств вещественных интегралов. Докажем последнее свойство. Если $I := \int_\gamma fdz = 0$, то утверждение тривиально. Пусть теперь это не так, тогда $I = |I|e^{i\theta}$ для некоторого $\theta \in \R$, откуда $|I| = Ie^{-i\theta}$. Зафиксируем параметризацию $z = \sigma(t)$, $t \in [a, b]$, тогда:
	\begin{multline*}
		|I| = \int_{a}^be^{-i\theta} f(\sigma(t))\sigma'(t)dt = \int_{a}^b\re\left(e^{-i\theta} f(\sigma(t))\sigma'(t)\right)dt \le \\
		\le \int_{a}^b\left|e^{-i\theta} f(\sigma(t))\sigma'(t)\right|dt = \int_{a}^b|f(\sigma(t))||\sigma'(t)|dt = \int_\gamma |f||dz|
	\end{multline*}

	Таким образом, первое неравенство уже получено, а второе следует из свойств вещественных интегралов.
\end{proof}

\begin{example}
	Пусть $n \in \Z$, $z_0 \in \Cm$, $R > 0$. Вычислим следующий интеграл:
	\[I_n := \int_{\{z \in \Cm: |z - z_0| = R\}}(z - z_0)^ndz\]
	
	Будем считать, что окружность $\{z \in \Cm: |z - z_0| = R\}$ ориентирована против часовой стрелки, и параметризуем ее как $z(t) = z_0 + Re^{it}$, $t \in [0, 2\pi]$. Тогда:
	\[I_n = \int_0^{2\pi}\left(Re^{it}\right)^n(iRe^{it})dt = iR^{n+1}\int_0^{2\pi}e^{it(n+1)}dt = \System{
	&2\pi i,\text{ если }n = -1
	\\
	&0,\text{ если }n \ne -1
	}\]
\end{example}

\begin{note}
	Если $z = \sigma(t)$, $t \in [a, b]$ "--- абсолютно непрерывная функция, а композиция $f \circ \sigma$ измерима и ограниченна на $[a, b]$, то формула интеграла второго рода остается применимой.
\end{note}

\section{Интегральная теорема Коши}

\begin{theorem}[\textit{без доказательства}]
	Пусть выполнены следующие условия:
	\begin{enumerate}
		\item $G \subset \R^2$ "--- односвязная область
		
		\item $P, Q \in C^1(G)$
		
		\item $\D{P}{y} \equiv \D{Q}{x}$ на $G$
		
		\item $\gamma$ "--- кусочно-гладкая замкнутая кривая такая, что $M(\gamma) \subset G$
	\end{enumerate}

	Тогда верно следующее равенство:
	\[\int_\gamma Pdx + Qdy = 0\]
\end{theorem}

\begin{theorem}[интегральная теорема Коши для односвязной области]
	Пусть выполнены следующие условия:
	\begin{enumerate}
		\item $G \subset \Cm$ "--- односвязная область
		\item $f \in C^1(G)$
		\item $\gamma$ "--- кусочно-гладкая замкнутая кривая такая, что $M(\gamma) \subset G$
	\end{enumerate}
	
	Тогда верно следующее равенство:
	\[\int_\gamma f(z)dz = 0\]
\end{theorem}

\begin{proof}
	Пусть $f(z) = f(x + iy) = u(x, y) + iv(x, y)$, тогда:
	\[\int_\gamma fdz = \int_\gamma udx - vdy + i\int_\gamma vdx + udy\]
	
	Проверим, что первый интеграл в сумме выше равен нулю, поскольку равенство нулю второго интеграла проверяется аналогично. В силу регулярности функции $f$, $u, v \in C^1(G)$, причем $\D{u}{y} \equiv -\D{v}{x}$ на $G$ по условиям Коши-Римана, тогда $\int_\gamma udx - vdy = 0$ по предыдущей теореме.
\end{proof}

\begin{note}
	Условие односвязности области $G$ существенно. Рассмотрив неодносвязную область $G := \{z \in \Cm : \frac12 < |z| < 2\}$, кривую $\gamma := \{z \in \Cm : |z| = 1\}$, ориентированную против часовой стрелки, и функцию $f(z) := \frac 1z \in C^1(G)$, тогда:
	\[\int_\gamma fdz = 2\pi i  \ne 0\]
\end{note}

\begin{definition}
	Область $G \subset \CM$ называется \textit{областью с простой границей}, если $\partial{G} = M(\gamma_1) \sqcup \dotsb \sqcup M(\gamma_n)$ для некоторых кусочно-гладких контуров $\gamma_1, \dotsc, \gamma_n$. Граничные кривые $\gamma_1, \dotsc, \gamma_n$ называются \textit{положительно ориентированными} относительно $G$, если при обходе каждой из них область всегда остается слева.
\end{definition}

\begin{example}
	На рисунке ниже слева приведен пример области $G_1$ с простой положительно ориентированной границей, а справа --- пример области $G_2$, не являющейся областью с простой границей.
		\begin{center}
			\begin{tikzpicture}
				\clip (-6, -2.2) rectangle (6, 2.2);
				
				\fill [opacity=0.05] (-3,0) circle [radius=2];
				\draw[
					black,
					decoration={markings, mark=at position 0.1 with {\arrow{>}}},
					decoration={markings, mark=at position 0.6 with {\arrow{>}}},
					postaction={decorate}
				] (-3,0) circle[radius=2];
				\node[] at (-1, 1.2) {$\gamma_1$};
				
				\fill [white] (-3.9,0) circle [radius=0.7];
				\draw[
				black,
				decoration={markings, mark=at position 0.1 with {\arrow{<}}},
				decoration={markings, mark=at position 0.6 with {\arrow{<}}},
				postaction={decorate}
				] (-3.9,0) circle[radius=0.7];
				\node[] at (-3.8, 0.9) {$\gamma_2$};
				\node[] at (-3, -1) {$G_1$};
				
				\fill [white] (-2.1,0) circle [radius=0.7];
				\draw[
				black,
				decoration={markings, mark=at position 0.1 with {\arrow{<}}},
				decoration={markings, mark=at position 0.6 with {\arrow{<}}},
				postaction={decorate}
				] (-2.1,0) circle[radius=0.7];
				\node[] at (-2.2, 0.9) {$\gamma_3$};
				
				\fill [opacity=0.05] (3,0) circle [radius=2];
				\draw[black] (3,0) circle[radius=2];
				\node[] at (1.03, 1.2) {$\gamma$};
				\node[] at (3, -0.67) {$G_2 = \dot B_\epsilon(z_0)$};
				
				\node[] at (3.2, 0.2) {$z_0$};
				
				\fill [white] (3,0) circle [radius=0.06];
				\draw[black,=] (3,0) circle[radius=0.06];
			\end{tikzpicture}
	\end{center}
	
	Кроме того, не являются областями с простой границей, например, область $\Cm$ и область $\{z \in \Cm: \im{z} > 0\}$.
\end{example}

\begin{note}
	Пусть $G \subset \CM$ "--- область с простой границей $M(\gamma_1) \sqcup \dotsb \sqcup M(\gamma_n)$. Под интегралом функции $f \in C^1(G)$ по границе $\partial G$ этой области понимается следующая величина:
	\[\int_{\partial G}fdz := \sum_{k = 1}^n\int_{\gamma_k}fdz\]
\end{note}

\begin{definition}
	\textit{Замыканием} множества $A \subset \Cm$ называется множество $\overline{A} := A \cup \partial A$.
\end{definition}

\begin{theorem}[\textit{без доказательства}, формула Грина для области с простой границей]
	Пусть выполнены следующие условия:
	\begin{enumerate}
		\item $G \subset \R^2$ "--- ограниченная область с простой границей $M(\gamma_1) \sqcup \dotsb \sqcup M(\gamma_n)$
		
		\item Граничные кривые $\gamma_1, \dotsc, \gamma_n$ положительно ориентированы относительно $G$
		
		\item $P, Q \in C(\overline{G})$ и $\D{P}{y}, \D{Q}{x} \in C(\overline{G})$, то есть функции $\D{P}{y}, \D{Q}{x}$ можно непрерывным образом продолжить на $\partial G$
	\end{enumerate}
	
	Тогда верно следующее равенство:
	\[\iint_G\left(\D{Q}{x} - \D{P}{y}\right)dxdy = \int_{\partial G} Pdx + Qdy\]
\end{theorem}

\begin{theorem}[интегральная теорема Коши для ограниченной области с простой границей]
	Пусть выполнены следующие условия:
	\begin{enumerate}
		\item $G \subset \Cm$ "--- ограниченная область с простой границей $M(\gamma_1) \sqcup \dotsb \sqcup M(\gamma_n)$
		
		\item Граничные кривые $\gamma_1, \dotsc, \gamma_n$ положительно ориентированы относительно $G$
		
		\item $f \in C^1(\overline{G})$
	\end{enumerate}

	Тогда верно следующее равенство:
	\[\int_{\partial G} fdz = 0\]
\end{theorem}

\begin{proof}
	В силу регулярности функции $f$, можно выбрать открытое множество $A \subset \Cm$ такое, что $A \supset \overline{G}$ и $f \in C^1(A)$. Пусть $f(z) = f(x + iy) = u(x, y) + iv(x, y)$, тогда:
	\[\int_{\partial G} fdz = \int_\gamma udx - vdy + i\int_\gamma vdx + udy\]
	
	Применяя предыдущую теорему к парам функций $u, -v$ и $v, u$, а также пользуясь условиями Коши-Римана, получим:
	\begin{gather*}
		\int_{\partial G} udx - vdy  = \iint_G \left(-\D{v}{x} - \D{u}{y}\right)dxdy = 0
		\\
		\int_{\partial G} vdx + udy  = \iint_G \left(\D{u}{x} - \D{v}{y}\right)dxdy = 0
	\end{gather*}

	Таким образом, $\int_{\partial G} fdz = 0$.
\end{proof}

\section{Интегральная формула Коши}

\begin{theorem}
	Пусть выполнены следующие условия:
	\begin{enumerate}
		\item $G \subset \Cm$ "--- ограниченная область с простой границей $M(\gamma_1) \sqcup \dotsb \sqcup M(\gamma_n)$
		
		\item Граничные кривые $\gamma_1, \dotsc, \gamma_n$ положительно ориентированы относительно $G$
		
		\item $f \in C^1(\overline{G})$
		
		\item $z_0 \in G$
	\end{enumerate}
	
	Тогда верно следующее равенство:
	\[f(z_0) = \frac{1}{2\pi i}\int_{\partial G}\frac{f(z)}{z - z_0}dz\]
\end{theorem}

\begin{proof}
	Зафиксируем произвольное $\epsilon > 0$. Достаточно доказать, что выполнено следующее неравенство:
	\[\left|\int_{\partial G}\frac{f(z)}{z - z_0}dz - 2\pi if(z_0)\right| \le 2\pi\epsilon\]
	
	Так как $f$ непрерывна в точке $z_0$, а множество $G$ "--- открытое, то существует $\delta > 0$ такое, что $\overline{B_\delta(z_0)} \subset G$ и для всех $z \in \overline{B_\delta(z_0)}$ выполнено $|f(z) - f(z_0)| \le \epsilon$. Рассмотрим область с простой границей $D := G \bs \overline{B_\delta(z_0)}$, тогда $\partial D = \partial G \cup \omega$, где $\omega$ "--- это окружность $\{z \in \Cm: |z - z_0| = \delta\}$, ориентированная по часовой стрелке. Применим к $\frac{f(z)}{z - z_0}$ интегральную теорему Коши для области $D$ и получим следующее:
	\[\int_{\partial D}\frac{f(z)}{z - z_0}dz = \int_{\partial G}\frac{f(z)}{z - z_0}dz + \int_{\omega}\frac{f(z)}{z - z_0}dz = 0 \ra \int_{\partial G}\frac{f(z)}{z - z_0}dz = -\int_{\omega}\frac{f(z)}{z - z_0}dz\]
	
	Заметим теперь, что верно следующее равенство:
	\[\int_{\omega}\frac{dz}{z - z_0} = -2\pi i\]
	
	C учетом полученных равенств, имеем:
	\[\left|\int_{\partial G}\frac{f(z)}{z - z_0}dz - 2\pi if(z_0)\right| = \left|\int_{\omega}\frac{f(z) - f(z_0)}{z - z_0}dz\right| \le \max_{z \in \omega}\frac{|f(z) - f(z_0)|}{|z - z_0|}|\omega| \le 2\pi\epsilon\]
	
	Таким образом, получено требуемое.
\end{proof}

\section{Первообразная}

\begin{definition}
	Функция $F$ назвывается \textit{первообразной} функции $f$ на области $G \subset \Cm$, если $F$ дифференцируема на $G$ и $F' \equiv f$ на $G$.
\end{definition}

\begin{note}
	Если $F$ "--- первообразная функции $f$ на области $G$, то для любого $C \in \Cm$ функция $F + C$ "--- тоже первообразная.
\end{note}

\begin{proof}
	Тривиально, поскольку $C' \equiv 0$ на $\Cm$.
\end{proof}

\begin{proposition}
	Если $F_1, F_2$ "--- первообразные функции $f$ на области $G$, то для некоторого $C \in \Cm$ выполнено $F_1 - F_2 \equiv C$ на $G$.
\end{proposition}

\begin{proof}
	Положим $F := F_2 - F_1$, тогда $F$ тоже дифференцируема на $G$, причем $F' = F_2 - F_1 \equiv 0$ на $G$, откуда $F' \in C^1(G)$. Пусть $F(z) = U(x, y) + iV(x, y)$, тогда выполнено равенство $F' = U'_x +iV'_x$, и, по условиям Коши-Римана, $U'_x, U'_y, V'_x, V'_y \equiv 0$ на $G$. Значит, функции $U$ и $V$ постоянны на $G$, что и требовалось.
\end{proof}

\begin{proposition}[формула Ньютона-Лейбница]
	Пусть выполнены следующие условия:
	\begin{enumerate}
		\item Функция $f$ непрерывна на области $G \subset \Cm$
		\item $F$ "--- первообразная функции $f$ на $G$
		\item $\gamma$ "--- кусочно гладкая-кривая, соединяющая точки $z_1, z_2 \in G$, такая, что $M(\gamma) \subset G$
	\end{enumerate}
	
	Тогда верно следующее равенство:
	\[\int_{\gamma}fdz = F(z_2) - F(z_1)\]
\end{proposition}

\begin{proof}
	Пусть $z = \sigma(t)$, $t \in [a, b]$ "--- параметризация кривой $\gamma$, тогда $\sigma(a) = z_1$, $\sigma(b) = z_2$. Поскольку $F' \equiv f$ на $G$, то:
	\[F(\sigma(t))'_t = F'(\sigma(t))\sigma'(t) = f(\sigma(t))\sigma'(t),~t \in [a, b]\]
	
	Значит, выполнено следующее:
	\[\int_\gamma fdz = \int_a^b f(\sigma(t))\sigma'(t)dt = \int_a^b F(\sigma(t))'_t dt = F(\sigma(b)) - F(\sigma(a)) = F(z_2) - F(z_1)\]
	
	Получено требуемое.
\end{proof}

\begin{corollary}
	Пусть функция $f$ непрерывна и имеет первообразную на области $G \subset \Cm$. Тогда для любой замкнутой кусочно-гладкой кривой $\gamma$ такой, что $M(\gamma) \subset G$, выполнено равенство:
	\[\int_\gamma fdz = 0\]
\end{corollary}

\begin{example}
	Пусть $n \in \N$. Тогда для любых чисел $z_1, z_2 \in \Cm$ и для любой соединяющей их кусочно-гладкой кривой выполнено следующее:
	\[\int_{z_1}^{z_2}z^ndz := \int_\gamma z^ndz = \frac{z_2^{n+1} - z_1^{n+1}}{n+1}\]
\end{example}

\begin{proposition}
	Пусть выполнены следующие условия:
	\begin{enumerate}
		\item Функция $f$ непрерывна на области $G \subset \Cm$
		\item Для любой замкнутой кусочно-гладкой кривой $\gamma$ такой, что $M(\gamma) \subset G$, выполнено равенство $\int_\gamma fdz = 0$
	\end{enumerate}

	Тогда $f$ имеет первообразную на $G$.
\end{proposition}
\pagebreak

\begin{proof}
	Из условия следует, что для любых $z_1, z_2 \in G$ интеграл $\int_{z_1}^{z_2} fdz$ не зависит от выбора лежащей в $G$ кусочно-гладкой кривой, соединяющей $z_1$ и $z_2$. Действительно, если $\gamma_1, \gamma_2$ "--- две такие кривые, то:
	\[0 = \int_{\gamma_1\gamma_2^{-1}}fdz = \int_{\gamma_1}fdz + \int_{\gamma_2^{-1}}fdz = \int_{\gamma_1}fdz - \int_{\gamma_2}fdz \ra \int_{\gamma_1}fdz = \int_{\gamma_2}fdz\]
	
	Зафиксируем $z_0 \in G$ и проверим, что первообразной функции $f$ является функция следующего вида:
	\[F(z) := \int_{z_0}^zfdw,~z \in G\]
	
	Выберем произвольные $z \in G$ и $\epsilon > 0$. Достаточно доказать, что существует $\delta > 0$ такое, что для всех $0 < |\Delta z| \le \delta$ выполнено следующее неравенство:
	\[\left|\frac{F(z + \Delta z) - F(z)}{\Delta z} - f(z)\right| \le \epsilon\]
	
	Поскольку $f$ непрерывна в точке $z$, то существует $\delta > 0$ такое, что для всех $w \in \Cm$, $|w - z| \le \delta$ выполнено неравенство $|f(w) - f(z)| < \epsilon$. Тогда, выбирая в качестве кривой, соединяющей $z$ и $z + \Delta z$, отрезок, получим:
	\[\frac{F(z + \Delta z) - F(z)}{\Delta z} = \frac{1}{\Delta z}\left(\int_{z_0}^{z + \Delta z}fdw - \int_{z_0}^{z}fdw\right) = \frac{1}{\Delta z}\int_{z}^{z + \Delta z}fdw\]
	
	Таким образом, выполнено следующее:
	\[\left|\frac{F(z + \Delta z) - F(z)}{\Delta z} - f(z)\right| = \frac1{|\Delta z|}\left|\int_z^{z + \Delta z}\left(f(w) - f(z)\right)dw\right| \le \frac1{|\Delta z|}(\epsilon\Delta z) = \epsilon\]
	
	В силу произвольности выбора $z$ и $\epsilon$, получено требуемое.
\end{proof}

\begin{corollary}
	Пусть $G \subset \Cm$ "--- односвязная область, $f \in C^1(G)$. Тогда $f$ имеет первообразную на $G$.
\end{corollary}

\begin{proof}
	Воспользуемся утверждением выше и интегральной теоремой Коши для односвязной области.
\end{proof}

\begin{note}
	Условие односвязности в последнем утверждении существенно. Рассмотрим, например, область $G := \{z \in \Cm: \frac12 < |z| < 2\}$, функцию $f(z) := \frac1z$ и кривую $\gamma$, обходящую окружность $\{z \in \Cm: |z| = 1\}$ против часовой стрелки. Тогда:
	\[\int_\gamma\frac1{z} = 2\pi i \ne 0\]
	
	Значит, функция $f$ не может иметь первообразную на $G$.
\end{note}

\begin{theorem}
	Функция $f$, непрерывная на области $G \subset \Cm$, имеет первообразную $\lra$ для любой замкнутой кусочно-гладкой кривой $\gamma$ такой, что $M(\gamma) \subset G$, выполнено равенство $\int_\gamma fdz = 0$.
\end{theorem}

\begin{proof}
	Воспользуемся утверждениями выше.
\end{proof}

\section{Функциональные последовательности и ряды}

\begin{definition}
	Пусть $\{f_n\}_{n = 0}^\infty$ "--- последовательность функций, определенных на области $G \subset \Cm$, функция $f$ определена на $G$. Последовательность $\{f_n\}$ \textit{сходится локально равномерно} к $f$ на области $G$, если $\forall z_0 \in G: \exists \epsilon > 0: f_n \convu_{B_\epsilon(z_0)} f$. Обозначение "--- $f_n \convlr f$.
\end{definition}

\begin{proposition}
	Пусть выполнены следующие условия:
	\begin{enumerate}
		\item Функции $f_0, f_1, \dotsc$ непрерывны на области $G \subset \Cm$
		\item $f_n \convlr f$
	\end{enumerate}
	
	Тогда $f$ непрерывна на $G$.
\end{proposition}

\begin{proof}
	Аналогично вещественному случаю.
\end{proof}

\begin{definition}
	Пусть $\{f_n\}_{n = 0}^\infty$ "--- последовательность функций, определенных на области $G \subset \Cm$. Ряд $\sum_{k = 0}^\infty f_k$ \textit{сходится локально равномерно} на $G$, если последовательность его \textit{частных сумм} $S_n := \sum_{k = 0}^nf_k$ локально равномерно сходится к некоторой определенной на $G$ функции $S$, называемой \textit{суммой} ряда.
\end{definition}

\begin{proposition}
	Пусть выполнены следующие условия:
	\begin{enumerate}
		\item Функции $f_0, f_1, \dotsc$ непрерывны на области $G \subset \Cm$
		\item Ряд $\sum_{k = 0}^\infty f_k$ локально равномерно сходится к сумме $S$
	\end{enumerate}
	
	Тогда $S$ непрерывна на $G$.
\end{proposition}

\begin{proof}
	Воспользуемся утверждением о функциональных последовательностях для последовательности $\{S_n\}_{n=0}^\infty$.
\end{proof}

\begin{proposition}
	Пусть $\{f_n\}_{n = 0}^\infty$ "--- последовательность функций, определенных на области $G \subset \Cm$, функция $f$ определена на $G$. Тогда $f_n \convlr f$ $\lra$ для любого компакта $K \subset G$ выполнено $f_n \convu_K f$.
\end{proposition}

\begin{proof}~
	\begin{itemize}
		\item[$\ra$]Для произвольной точки $z \in K$ выберем окрестность $B(z)$ такую, что $f_n \convu_{B(z)} f$. Тогда $K \subset \bigcup_{z \in K}B(z)$, и, в силу компактности, можно выбрать конечный набор $z_1, \dotsc, z_m$ такой, что $K \subset \bigcup_{k=1}^mB(z_k)$. Поскольку для любого $k \in \{1, \dotsc, m\}$ выполнено $f_n \convu_{B_k(z)} f$, то $f_n \convu_K f$.
		
		\item[$\la$]Зафиксируем произвольную точку $z_0 \in G$ и выберем $\delta > 0$ такое, что $\overline{B_\delta(z_0)} \subset G$, тогда $f_n \convu_{\overline{B_\delta(z_0)}} f$.\qedhere
	\end{itemize}
\end{proof}

\begin{proposition}
	Пусть выполнены следующие условия:
	\begin{enumerate}
		\item Функции $f_0, f_1, \dotsc$ непрерывны на области $G \subset \Cm$
		\item $f_n \convlr f$
		\item $\gamma$ "--- кусочно-гладкая кривая такая, что $M(\gamma) \subset G$
	\end{enumerate}
	
	Тогда верно следующее равенство:
	\[\int_\gamma f_ndz \xrightarrow{n \to \infty} \int_\gamma fdz\]
\end{proposition}

\begin{proof}
	Из условия следует, что $f$ непрерывна на $G$, а $M(\gamma)$ "--- компакт, поскольку $M(\gamma)$ является непрерывным образом отрезка. Тогда $f_n \convu_{M(\gamma)} f$, и выполнено следующее:
	\[\left|\int_\gamma fndz - \int_\gamma fdz\right| = \left|\int_\gamma (f_n - f)dz\right| \le \left(\max_{z \in M(\gamma)}|f_n(z) - f(z)|\right)|\gamma| \xrightarrow{n \to \infty} 0\]
	
	Получено требуемое.
\end{proof}

\begin{note}
	Переформулируем утверждение выше для случая функциональных рядов. Пусть выполнены следующие условия:
	\begin{enumerate}
		\item Функции $f_0, f_1, \dotsc$ непрерывны на области $G \subset \Cm$
		\item $\sum_{n=0}^\infty f_n$ сходится локально равномерно на $G$
		\item $\gamma$ "--- кусочно-гладкая кривая такая, что $M(\gamma) \subset G$
	\end{enumerate}
	
	Тогда верно следующее равенство:
	\[\int_\gamma \left(\sum_{n=0}^\infty f_n\right)dz = \sum_{n=0}^\infty \left(\int_\gamma f_ndz\right)\]
\end{note}

\begin{definition}
	Пусть $\{f_n\}_{n = 0}^\infty$ "--- последовательность функций, определенных на множестве $G \subset \Cm$. Ряд $\sum_{n=0}^\infty f_n$ \textit{сходится регулярно} на $E$, если ряд $\sum_{n=0}^\infty |f_n|$ сходится равномерно на $E$.
\end{definition}

\begin{note}
	Из регулярной сходимости следует равномерная сходимость в силу критерия Коши и неравенства треугольника.
\end{note}

\begin{theorem}[признак Вейерштрасса]
	Пусть выполнены следующие условия:
	\begin{enumerate}
		\item Пусть $f_0, f_1, \dotsc$ определены на множестве $E \subset \Cm$
		\item $\forall n \in \mathbb{N}_0 : \exists a_n > 0 \forall z \in E: |f_n(z)| \le a_n$
		\item Ряд $\sum_{n=0}^\infty a_n$
	\end{enumerate}
	
	Тогда ряд $\sum_{n=0}^\infty f_n$ сходится регулярно на $E$.
\end{theorem}

\begin{proof}
	Аналогично вещественному случаю.
\end{proof}

\section{Cтепенные ряды}

\begin{definition}
	Пусть $c_0, c_1, \dotsc \in \Cm$, $z_0 \in \Cm$. \textit{Степенным рядом} называется выражение вида $\sum_{n=0}^\infty c_n(z - z_0)^n$. Точка $z_0$ называется \textit{центром разложения}.
\end{definition}

\begin{note}
	Для удобства обозначений, в дальнейшем мы будем рассматривать только степенные ряды с центром в нуле, то есть выражения вида $\sum_{n=0}^\infty c_nz^n$.
\end{note}

\pagebreak

\begin{definition}
	\textit{Радиусом сходимости} степенного ряда $\sum_{n=0}^\infty c_nz^n$ называется величина $R := \sup\{|z| : \text{ряд }\sum_{n=0}^\infty c_nz^n\text{ сходится}\}$. \textit{Кругом сходимости} ряда называется множество $K := \{z \in \Cm : |z| < R\}$.
\end{definition}

\begin{theorem}[Абеля]
	Пусть степенной ряд $\sum_{n=0}^\infty c_nz^n$ имеет радиус сходимости $R > 0$ и круг сходимости $K$. Тогда:
	\begin{enumerate}
		\item Ряд $\sum_{n=0}^\infty c_nz^n$ сходится абсолютно для любого $z \in K$
		\item Ряд $\sum_{n=0}^\infty c_nz^n$ сходится регулярно в круге $K_r := \{z \in \Cm: |z| \le r\}$ при любом $r < R$.
	\end{enumerate}
\end{theorem}

\begin{proof}
	Достаточно доказать вторую часть. Зафиксируем $r < R$, тогда, по определению, $\exists z_1 \in \Cm$ такое, что $|z_1| > r$ и ряд $\sum_{n=0}^\infty c_nz_1^n$ сходится. Следовательно, $c_nz_1^n \xrightarrow{n \to \infty} 0$, тогда существует $M > 0$ такое, что для любого $n \in \N_0$ выполнено $|c_nz_1^n| \le M$. Тогда для любых $n \in \N_0$ и $z \in K_r$ имеем:
	\[|c_nz^n| = \left|c_nz_1^n\right|\left|\frac{z}{z_1}\right|^n \le M\left(\frac{r}{|z_1|}\right)^n\]
	
	По признаку Вейерштрасса, получено требуемое.
\end{proof}

\begin{note}
	Если радиус сходимости $R$ ряда $\sum_{n=0}^\infty c_nz^n$ конечен, то при $z \in \Cm$ таких, что $|z| > R$, ряд расходится по определению радиуса сходимости.
\end{note}