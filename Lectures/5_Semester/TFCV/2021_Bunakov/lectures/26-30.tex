\section{Приращение аргумента вдоль кривой}

\begin{note}
	Мы стремимся формализовать понятие приращения аргумента вдоль кривой. Например, на рисунке ниже кривая $\gamma$ совершает полтора оборота против часовой стрелки, то есть аргумент числа при перемещении вдоль этой кривой изменяется на $3\pi$.
\begin{center}
	\begin{tikzpicture}
		\clip (-3.5, -3) rectangle (3.5, 3);
		
		\clip (-3.4, -3) rectangle (3.4, 3);
		\draw [->] (-3, 0) -- (3, 0) node [above, black] {$\re z$};
		\draw [->] (0, -2.7) -- (0, 2.7) node [right, black] {$\im z$};
		
		\draw [black] (1.4,3pt) -- (1.4,-3pt) node [below, black] {$1$};
		\draw [black] (3pt,1.4) -- (-3pt,1.4) node [left, black] {$i$};
		
		
		\path
		coordinate (p1) at (-2.3, -2.3)
		coordinate (p2) at (0, 0.7)
		coordinate (p3) at (1.8, 0)
		coordinate (p4) at (0.8, -1.4)
		coordinate (p5) at (-0.7, -0.5)
		coordinate (p6) at (-0.3, 0.8)
		coordinate (p7) at (2.2, 2.2);
		
		\draw[
		black,
		decoration={markings, mark=at position 0.12 with {\arrow{<}}},
		decoration={markings, mark=at position 0.5 with {\arrow{<}}},
		decoration={markings, mark=at position 0.87 with {\arrow{<}}},
		postaction={decorate}
		] plot [smooth, tension=0.85] coordinates {(p1) (p2) (p3) (p4) (p5) (p6) (p7)};
		
		\draw [->, black] (0, 0) -- (1.1, 1.67) node [black, below right, scale = 1.1] {$z$};
		\node[draw, circle, inner sep=1pt, fill, black] at (1.13, 1.7) {};
		
		\coordinate (a1) at (1.1, 1.6);
		\coordinate (a2) at (0,0);
		\coordinate (a3) at (1, 0);
		
		\pic [draw, ->] {angle = a3--a2--a1};
		\node [] at (0.6, 0.45) {$\phi$};
		
		\node[draw, circle, inner sep=1pt, fill, black, label={above:$\phi_0 = \frac\pi4$}] at (p7) {};
		\node[draw, circle, inner sep=1pt, fill, black, label={below:$\phi_1 = \frac{13\pi}4$}] at (p1) {};
		\node[] at (1.1, -1.7) {$\gamma$};
	\end{tikzpicture}
\end{center}
\end{note}

\begin{proposition}
	Пусть выполнены следующие условия:
	\begin{enumerate}
		\item Функция $z(t) = x(t) + iy(t)$ непрерывно дифференцируема на отрезке $[t_0, t_1]$
		\item Для любого $t \in [t_0, t_1]$ выполнено $z(t) \ne 0$
		\item Зафиксировано значение $\phi_0 \in \Arg{z(t_0)}$
	\end{enumerate}

	Тогда существует единственная функция $\phi \in C^1[t_0, t_1]$ такая, что для любого числа $t \in [t_0, t_1]$ выполнено $\phi(t) \in \Arg z(t)$. Более того, $\phi$ задается следующей формулой:
	\[\phi(t) = \phi_0 + \int_{t_0}^t\frac{x(\tau)y'(\tau) - x'(\tau)y(\tau)}{x(\tau)^2 + y(\tau)^2}d\tau\]
\end{proposition}

\begin{proof}
	Пусть $\phi$ "--- функция, заданная формулой из условия. Условие, что для любого $t \in [t_0, t_1]$ выполнено $\phi(t) \in \Arg z(t)$, равносильно следующей системе:
	\[\System{
		\cos\phi(t) \equiv \frac{x(t)}{\sqrt{x(t)^2 + y(t)^2}} \\
		\sin\phi(t) \equiv \frac{y(t)}{\sqrt{x(t)^2 + y(t)^2}}
	}\]
	
	Положим $u(t) := \cos\phi(t)$, $v(t) := \sin\phi(t)$, функции в правой части системы выше обозначим через $\widetilde u(t)$, $\widetilde v(t)$, а подынтегральную функцию из условия --- через $a(\tau)$. Тогда:
	\begin{gather*}
		u'(t) = (\cos\phi(t))'_t = (-\sin\phi(t))\phi'(t) = -v(t)a(t)\\
		v'(t) = (\sin\phi(t))'_t = (\cos\phi(t))\phi'(t) = u(t)a(t)
	\end{gather*}
	
	Значит, набор $(u(t), v(t))$ является решением следующей задачи Коши:
	\[\System{
	&u'(t) = -a(t)v(t)\\
	&v'(t) = a(t)u(t)\\
	&u(t_0) = \cos\phi(t_0)\\
	&v(t_0) = \sin\phi(t_0)}\]
	
	Остается заметить, что $(\widetilde u(t), \widetilde v(t))$ тоже является решением этой задачи Коши, откуда $u \equiv \widetilde u$ и $v \equiv \widetilde v$ на $[t_0, t_1]$. Проверим единственность функции $\phi$. Пусть $\psi \in C^1[t_0, t_1]$ "--- такая функция, что $\phi(t_0) = \psi(t_0)$, $\cos\phi(t) \equiv \cos\psi(t)$ и $\sin\phi(t) \equiv \sin\psi(t)$ на $[t_0, t_1]$. То $\phi' \equiv \psi'$ на $[t_0, t_1]$. Интегрируя это соотношение, получим, что $\phi \equiv \psi$ на $[t_0, t_1]$.
\end{proof}

\begin{definition}
	Пусть $\gamma$ "--- кривая с непрерывно дифференцируемой параметризацией $z(t) = x(t) + iy(t)$, $t \in [t_0, t_1]$, не проходящая через точку $0$. \textit{Приращением аргумента вдоль кривой} $\gamma$ называется следующая величина:
	\[\Delta_\gamma\arg{z} := \int_{t_0}^t\frac{x(\tau)y'(\tau) - x'(\tau)y(\tau)}{x(\tau)^2 + y(\tau)^2}d\tau\]
\end{definition}

\begin{note}
	Формулу для приращения аргумента можно переписать в следующем виде:
	\[\Delta_\gamma \arg{z} = \im\int_{t_0}^{t_1}\frac{z'(\tau)}{z(\tau)}d\tau = \im\int_\gamma \frac{dz}{z}\]
\end{note}

\begin{definition}
	Пусть $\gamma$ "--- кривая с непрерывно дифференцируемой параметризацией $z(t)$, $t \in [t_0, t_1]$, функция $f$ регулярна на $M(\gamma)$, и кривая $\Gamma := f(\gamma)$ с параметризацией $f(z(t))$, $t \in [t_0, t_1]$, не проходит через точку $0$. \textit{Приращением аргумента функции $f$ вдоль кривой} $\gamma$ называется величина $\Delta_\gamma\arg{f} := \Delta_\Gamma\arg{z}$.
\end{definition}

\begin{note}
	В силу уже доказанного, верна следующая формула:
	\[\Delta_\gamma\arg{f} = \im\int_\gamma \frac{f'(z)}{f(z)}dz\]
	
	Из этой формулы легко вывести следующие свойства:
	\begin{enumerate}
		\item Если кривая $\gamma$ представима в виде $\gamma = \gamma_1\dotsc \gamma_k$ и для кривых $\gamma_1, \dotsc, \gamma_k$ определены $\Delta_{\gamma_1}\arg f, \dotsc, \Delta_{\gamma_k}\arg f$, то:
		\[\Delta_{\gamma}\arg f = \sum_{i = 1}^k\Delta_{\gamma_i}\arg f\]
		
		Это позволяет обобщить определение приращения аргумента вдоль кривых, являющихся кусочно-непрерывно дифференцируемыми.
		
		\item Если для кривой $\gamma$ определено $\Delta_{\gamma}\arg f$, то:
		\[\Delta_{\gamma^{-1}}\arg f = -\Delta_{\gamma}\arg f\]
		
		\item (\textit{Логарифмическое свойство}) Пусть определены $\Delta_{\gamma}\arg f_1$ и $\Delta_{\gamma}\arg f_2$. Тогда:
		\begin{gather*}
			\Delta_{\gamma}\arg (f_1f_2) = \Delta_{\gamma}\arg f_1 + \Delta_{\gamma}\arg f_2
			\\
			\Delta_{\gamma}\arg \left(\frac{f_1}{f_2}\right) = \Delta_{\gamma}\arg f_1 - \Delta_{\gamma}\arg f_2
		\end{gather*}
	
		Убедимся в справедливости первой формулы:
		\[\Delta_{\gamma}\arg (f_1f_2) =  \im\int_\gamma \frac{(f_1f_2)'(z)}{f_1(z)f_2(z)}dz = \im\int_\gamma \left(\frac{f_1'(z)}{f_1(z)} + \frac{f_2'(z)}{f_2(z)}\right)dz = \Delta_{\gamma}\arg f_1 + \Delta_{\gamma}\arg f_2\]
	\end{enumerate}
\end{note}

\section{Принцип аргумента и теорема Руше}

\begin{theorem}[принцип аргумента]
	Пусть выполнены следующие условия:
	\begin{enumerate}
		\item $G$ "--- ограниченная односвязная область с простой границей
		
		\item Граничная кривая $\partial G$ положительно ориентирована относительно $G$
		
		\item Функция $f$ регулярна на $\overline{G} \bs \{a_1, \dotsc, a_m\}$, где $a_1, \dotsc, a_m \in G$ "--- полюса
		
		\item $f \ne 0$ на $\partial G$
	\end{enumerate}
	
	Тогда $f$ имеет лишь конечное число нулей на $G$. Более, если $N$ и $P$ "--- числа нулей и полюсов функции $f$ на $G$ с учетом их порядков, то:
	\[\Delta_{\partial G} \arg f = 2\pi (N - P)\]
\end{theorem}

\begin{proof}
	Предположим, что $f$ имеет бесконечное число нулей на $G$. Тогда множество нулей имеет предельную точку $z_0 \in \overline{G}$, причем $z_0$ не может быть полюсом, поскольку $f(z) \to 0$ при $z \to z_0$. Значит, $f$ непрерывна в точке $z_0$, откуда $f(z_0) = 0$, поэтому $z_0 \in G$. Но тогда, по теореме единственности, $f \equiv 0$ на $G \bs \{a_1, \dotsc, a_m\}$, и $f \equiv 0$ на $\partial G$ в силу непрерывности --- противоречие.
	
	Пусть $a_1, \dotsc, a_m \in G$ "--- полюса функции $f$ порядков $p_1, \dotsc, p_m$, $b_1, \dots, b_n \in G$ "--- нули функции $f$ порядков $q_1, \dotsc, q_n$, тогда $\sum_{k = 1}^mp_k = P$ и $\sum_{k = 1}^nq_k = N$. По формуле для приращения аргумента функции вдоль кривой:
	\[\Delta_{\partial G} \arg f = \im\int_{\partial G}\frac{f'(z)}{f(z)}dz = 2\pi\re \left(\sum_{k = 1}^m\res_{a_k}\frac{f'}{f} + \sum_{k = 1}^n\res_{b_k}\frac{f'}{f}\right)\]
	
	Зафиксируем нуль $b_k$. В некоторой его проколотой окрестности $\mathring B(b_k)$ выполнено равенство $f(z) = (z-b_k)^{p_k}h(z)$, где $h$ "--- регулярная на $B(b_k)$ функция, $h \ne 0$ на $B(b_k)$. Тогда $f'(z) = p_k(z-b_k)^{p_k - 1}h(z) + (z-b_k)^{p_k}h'(z)$. Значит, выполнено следующее:
	\[\res_{b_k}\frac{f'}{f} = \res_{b_k}\frac{p_k}{z-b_k} + \res_{b_k}\frac{h'}{h} = p_k + 0 = p_k\]
	
	Аналогично, зафиксируем полюс $a_k$. В некоторой его проколотой окрестности $\mathring B(a_k)$ выполнено равенство $f(z) = (z-a_k)^{-q_k}h(z)$, где $h$ "--- регулярная на $B(a_k)$ функция, $h \ne 0$ на $B(a_k)$. Тогда $f'(z) = -q_k(z-a_k)^{-q_k - 1}h(z) + (z-a_k)^{-q_k}h'(z)$. Значит, выполнено следующее:
	\[\res_{a_k}\frac{f'}{f} = \res_{a_k}\frac{-q_k}{z-a_k} + \res_{a_k}\frac{h'}{h} = -q_k + 0 = -q_k\]
	
	Подставляя полученные вычеты в формулу для приращения аргумента функции, полуачем требуемое.
\end{proof}

\begin{note}
	Число полюсов или нулей в теореме выше может быть и нулевым.
\end{note}

\begin{theorem}[Руше]
	Пусть выполнены следующие условия:
	\begin{enumerate}
		\item $G$ "--- ограниченная односвязная область с простой границей
		
		\item Граничная кривая $\partial G$ положительно ориентирована относительно $G$
		
		\item Функции $f, g$ регулярна на $\overline{G}$
		
		\item $|f| > |g|$ на $\partial G$
	\end{enumerate}
	
	Тогда функции $f$ и $f+g$ на $G$ имеют одинаковое число нулей с учетом их порядков.
\end{theorem}

\begin{proof}
	Из условия, в частности, следует, что $f \ne 0$ на $\partial G$, поскольку $|f| > 0$ на $\partial G$. Пусть $N_f$, $N_{f+g}$ "--- число нулей функций $f$ и $f+g$ на $G$ с учетом их порядков. По принципу аргумента, выполнено следующее:
	\begin{gather*}
		\Delta_{\partial G}\arg f = 2\pi N_f
		\\
		\Delta_{\partial G}\arg (f+g) = 2\pi N_{f+g}
	\end{gather*}
	
	С другой стороны:
	\[\Delta_{\partial G}\arg (f+g) = \Delta_{\partial G}\arg \left(f\left(1 + \frac gf\right)\right) = \Delta_{\partial G}\arg f + \Delta_{\partial G}\arg \left(1 + \frac gf\right)\]
	
	Рассмотрим функцию $h = 1 + \frac gf$, регулярную на $\partial G$. Пусть кривая $\Gamma$ "--- образ границы $\partial G$ под действием функции $h$. Тогда:
	\[\Delta_{\partial G}\arg \left(1 + \frac gf\right) = \Delta_\Gamma \arg z\]
	
	Но для каждого $z \in \Gamma$ выполнено $|z - 1| = \left|\frac{g(z)}{f(z)}\right| < 1$. Тогда:
	\begin{center}
		\begin{tikzpicture}
			\clip (-1.5, -2.5) rectangle (4, 2.5);
			
			\draw [->] (-1, 0) -- (3.6, 0) node [above, black] {$\re z$};
			\draw [->] (0, -2.1) -- (0, 2.1) node [right, black] {$\im z$};
			
			\draw [black] (1.5,3pt) -- (1.5,-3pt) node [below, black] {$1$};
			\draw [black] (3pt,1.5) -- (-3pt,1.5) node [left, black] {$i$};
			
			
			\path
			coordinate (p1) at (2, 1.1)
			coordinate (p2) at (2.3, 0.8)
			coordinate (p3) at (2, -0.3)
			coordinate (p4) at (1.6, -0.6)
			coordinate (p5) at (1.2, -0.7)
			coordinate (p6) at (0.3, -0.3)
			coordinate (p66) at (0.6, -0.8)
			coordinate (p666) at (0.9, -0.3)
			coordinate (p7) at (0.3, 0.3)
			coordinate (p8) at (0.6, 0.5)
			coordinate (p9) at (1.1, 1.1);
			
			\draw[
			black,
			decoration={markings, mark=at position 0.12 with {\arrow{<}}},
			decoration={markings, mark=at position 0.64 with {\arrow{<}}},
			decoration={markings, mark=at position 0.87 with {\arrow{<}}},
			postaction={decorate}
			] plot [smooth, tension=0.85] coordinates {(p1) (p2) (p3) (p4) (p5) (p6) (p66) (p666) (p7) (p8) (p9) (p1)};
			
			\draw[black, dashed] (1.5, 0) circle[radius=1.5];
			
			\draw [->, black] (0, 0) -- (1.97, 1.07) node [black, below, scale = 1.1] {$z$};
			
			\node[draw, circle, inner sep=1pt, fill, black] at (p1) {};
			
			\coordinate (a2) at (0,0);
			\coordinate (a3) at (1, 0);
			
			\pic [draw, ->, angle radius = 0.8cm] {angle = a3--a2--p1};
			\node [] at (0.98, 0.25) {$\phi$};
			\node[] at (1.93, -0.93) {$\Gamma$};
		\end{tikzpicture}
	\end{center}
	
	Геометрически ясно, что $\Delta_\Gamma\arg{z} = 0$. Значит:
	\[\Delta_{\partial G}\arg (f+g) = \Delta_{\partial G}\arg f \ra N_f = N_{f+g}\]
	
	Получено требуемое.
\end{proof}

\section{Целые функции и основная теорема алгебры}

\begin{definition}
	Функция $f$, регулярная на $\Cm$, называется \textit{целой}.
\end{definition}

\begin{note}
	Каждая целая функция представима рядом Тейлора на всей плоскости $\Cm$:
	\[f(z) = \sum_{n=0}^{+\infty}c_nz^n,~z \in \Cm\]
\end{note}

\begin{proposition}
	Пусть выполнены следующие условия:
	\begin{enumerate}
		\item Функция $f$ "--- целая
		\item Существуют числа $R_0, A > 0$ и $m \in \N \cup \{0\}$ такие, что на кольце $\{z \in \Cm: |z| > R_0\}$ выполнено неравенство $|f(z)| \le A|z|^m$
	\end{enumerate}
	
	Тогда $f$ является многочленом степени не выше $m$.
\end{proposition}

\begin{proof}
	Для каждого $R > R_0$ положим $M(R) := \max\{f(z): |z| = R\}$. По неравенству Коши для коэффициентов ряда Лорана, для каждого $n >m $ имеем:
	\[|c_n| \le \frac{M(R)}{R^n} \le \frac{AR^m}{R^n} \xrightarrow{R \to +\infty} 0\]
	
	Значит, $f = \sum_{n = 0}^mc_nz^n$.
\end{proof}

\begin{corollary}[теорема Лиувилля]
	Если $f$ "--- целая функция, и для некоторого $R_0 > 0$ она ограниченна на $\{z \in \Cm: |z| > R_0\}$, то $f$ постоянна на $\Cm$.
\end{corollary}

\begin{proof}
	Применим утверждение выше для случая, когда $m = 0$.
\end{proof}

\begin{theorem}[основная теорема алгебры]
	Любой многочлен $p \in \Cm[z]$, степень которого не меньше одного, имеет корень в $\Cm$.
\end{theorem}

\begin{proof}[Доказательство с помощью теоремы Лиувилля]
	Предположим противное, тогда функция $\frac 1p$ "--- целая, причем выполнено $\lim_{z \to \infty} \frac1p(z) = 0$, то есть эта функция ограниченна в $\Cm$. Тогда, по теореме Лиувилля, $\frac 1p$ постоянна на $\Cm$, но тогда и функция $p$ постоянна на $\Cm$, что неверно.
\end{proof}

\begin{proof}[Доказательство с помощью теоремы Руше]
	Пусть $p(z) = z^n + a_{n-1}z^{n-1} + \dotsb + a_0$. Выберем $R > \max\{1, \sum_{k = 0}^{n-1} |a_k|\}$. Проверим условия теоремы Руше для области $\{z \in \Cm: |z| < R\}$: 
	\[\left|\sum_{k = 0}^{n-1}a_kR^k\right| \le R^{n-1}\left(\sum_{k = 0}^{n-1}a_k\right) < R^n = |z^n|,~|z| = R\]
	
	Значит, многочлен $p$ имеет столько же нулей с учетом порядков на этой области, сколько и $z^n$, а именно --- $n$ нулей. Ясно также, что других корней у многочлена $p$ нет.
\end{proof}

\section{Теорема об обратной функции}

\begin{theorem}[\textit{без доказательства}]
	Пусть выполнены следующие условия:
	\begin{enumerate}
		\item $G \subset \R^2$ "--- область
		\item Функции $u, v$ непрерывно дифференцируемы на $G$
		\item $(x_0, y_0) \in G$, $u_0 := u(x_0, y_0)$, $v_0 := v(x_0, y_0)$
		\item Якобиан $\D{(u, v)}{(x, y)}$ в точке $(x_0, y_0)$ отличен от нуля
	\end{enumerate}
	
	Тогда существуют числа $\delta > 0$ и $\epsilon > 0$ такие, что $\D{(u, v)}{(x, y)} \ne 0$ на $B_\delta(x_0, y_0)$ и для любой точки $(\widetilde u, \widetilde v) \in B_\epsilon(u_0, v_0)$ на $B_\delta(x_0, y_0)$ имеет единственное решение следующая система:
	\[\System{u(x, y) = \widetilde u \\ v(x, y) = \widetilde v}\]
	
	Решения этой системы задают на $B_\delta(x_0, y_0)$ вектор-функцию $\left(x(u, v), y(u, v)\right)$, обратную к вектор-функции $(u(x, y), v(x, y))$, причем $x, y \in C^1(B_\epsilon(u_0, v_0))$. Наконец, на $B_\epsilon(u_0, v_0)$ имеют место следующие тождества:
	\vspace{-8pt}
	\begin{gather*}
		u(x(\widetilde u, \widetilde v), y(\widetilde u, \widetilde v)) \equiv \widetilde u
		\\
		v(x(\widetilde u, \widetilde v), y(\widetilde u, \widetilde v)) \equiv \widetilde v
		\\
		\begin{pmatrix}
			u'_x & u'_y \\
			v'_x & v'_y
		\end{pmatrix}
		\begin{pmatrix}
			x'_u & x'_v \\
			y'_u & y'_v
		\end{pmatrix}
		\equiv
		\begin{pmatrix}
			1 & 0 \\
			0 & 1
		\end{pmatrix}
	\end{gather*}
\end{theorem}

\begin{theorem}
	Пусть выполнены следующие условия:
	\begin{enumerate}
		\item $G \subset \Cm$ "--- область
		\item Функция $f$ регулярна на $G$
		\item $z_0 \in G$, $w_0 := f(z_0)$
		\item $f'(z_0) \ne 0$
	\end{enumerate}

	Тогда существуют числа $\delta > 0$ и $\epsilon > 0$ такие, что $f' \ne 0$ на $B_\delta(z_0)$ и для любой точки $\widetilde w \in B_\epsilon(w_0)$ на $B_\delta(z_0)$ уравнение $f(z) = \widetilde w$ имеет единственное решение. Решения этого уравнения задают на $B_\delta(z_0)$ функцию $g(w)$, обратную к $f(z)$, причем $g \in C^1(B_\epsilon(w_0))$. Наконец, на $B_\epsilon(w_0)$ имеют место следующие тождества:
	\vspace{-6pt}
	\begin{gather*}
		f(g(w)) \equiv w \\ g'(w)f'(g(w)) \equiv 1
	\end{gather*}
\end{theorem}

\begin{proof}
	Пусть $f(z) = f(x + iy) = u(x, y) + iv(x, y)$, $z_0 = x_0 + iy_0$, $w_0 = u_0 + iv_0$. Тогда, по условиям Коши-Римана:
	\[\left.\D{(u, v)}{(x, y)}\right|_{(u_0, v_0)} = \begin{vmatrix}
		u'_x & u'_y \\ v'_x & v'_y
	\end{vmatrix} = \left(u'_x(x_0, y_0)\right)^2 + \left(v'_x(x_0, y_0)\right)^2 = |f'(z_0)|^2 \ne 0\]

	Значит, выполнены все условия предыдущей теоремы, и на $B_\epsilon(u_0, v_0)$ определены функции $x(u, v), y(u, v)$ со свойствами выше. Положим $g(w) = g(u + iv) := x(u, v) + iy(u, v)$ и проверим условия Коши-Римана для функции $g$ на $B_\epsilon(w_0)$:
	\[\begin{pmatrix}
		x'_u & x'_v \\
		y'_u & y'_v
	\end{pmatrix}
	= \begin{pmatrix}
		u'_x & u'_y \\
		v'_x & v'_y
	\end{pmatrix}^{-1}
	=
	\frac1{|f'(z)|^2}\begin{pmatrix}
		v'_y & -u'_y \\
		-v'_x & u'_x
	\end{pmatrix} = \frac1{|f'(z)|^2}\begin{pmatrix}
	v'_y & v'_x \\
	-v'_x & v'_y
	\end{pmatrix}\]
	
	Следовательно, $g \in C^1(B_\epsilon(w_0))$. Остается проверить тождество для производных на круге $B_\epsilon(w_0)$:
	\vspace*{-6pt}
	\[g'(w)f'(g(w)) \equiv \frac{u'_x(g(w)) - iv'_x(g(w))}{|f'(g(w))|^2}f'(g(w)) \equiv \frac{\overline{f'(g(w))}f'(g(w))}{|f'(g(w))|^2} \equiv 1\]
	
	Получено требуемое.
\end{proof}

\section{Теорема о локальной структуре отображения}

\begin{theorem}[о локальной структуре отображения]
	Пусть выполнены следующие условия:
	\begin{enumerate}
		\item $G \subset \Cm$ "--- область
		\item Функция $f$ регулярна на $G$
		\item $z_0 \in G$, $w_0 := f(z_0)$
		\item Для некоторого натурального $n \ge 2$ выполнено, что $f'(z_0) = \dotsb = f^{(n - 1)}(z_0) = 0$, но $f^{(n)}(z_0) \ne 0$
	\end{enumerate}
	
	Тогда существуют числа $\delta > 0$ и $\epsilon > 0$ такие, что для любой точки $\widetilde w \in \mathring B_\epsilon(w_0)$ на $B_\delta(z_0)$ уравнение $f(z) = \widetilde w$ имеет ровно $n$ различных решений. Если же $w_0 \ne 0$, то доказанную часть теоремы можно применить к функции $f - w_0$.
\end{theorem}

\begin{proof}
	Пусть сначала $w_0 = 0$. Тогда точка $z_0$ "--- изолированный ноль функций $f$ и $f'$, поскольку $f^{(n)}(z_0) \ne 0$. Выберем $\delta > 0$ такое, что других нулей на $B_\delta(z_0)$ у этих функций нет. Положим $\epsilon = \min\{|f(z)| : |z - z_0| = \delta\}$. Зафиксируем $\widetilde w \in \mathring B_\epsilon(w_0)$ и проверим условия теоремы Руше для функций $f$ и $-\widetilde{w}$ на области $B_\delta(z_0)$:
	\[|f(z)| \ge \epsilon > |-\widetilde{w}|,~|z - z_0| = \delta\]
	
	Значит, уравнение $f(z) = \widetilde w$ имеет $n$ корней на $B_\delta(z_0)$ с учетом кратности, причем все они имеют порядок $1$, поскольку $(f- \widetilde w)' \ne 0$ на $\mathring B_\delta(z_0)$.
\end{proof}

\begin{note}
	Таким образом, на $B_\epsilon(w_0)$ определены функции $g_1, \dotsc, g_n$, не совпадающие ни в одной точке, такие, что для каждого $k \in \{1, \dotsc, n\}$ на $\mathring B_\epsilon(w_0)$ выполнено тождество $f(g_k(w)) \equiv w$. Кроме того, вместо пары $\delta$ и $\epsilon$ в доказательстве можно выбрать любые пары $\delta' \in (0, \delta)$ и $\epsilon' := \min\{|f(z)| : |z - z_0| = \delta'\}$, поэтому для каждого $k \in \{1, \dotsc, n\}$ выполнено $\lim_{w \to w_0}g_k(w) \equiv z_0$.
\end{note}

\begin{theorem}[принцип сохранения области]
	Пусть непостоянная функция $f$ регулярна на области $G$. Тогда $f(G)$ "--- тоже область.
\end{theorem}

\begin{proof}
	Докажем, что множество $f(G)$ "--- открытое. Пусть $w_0 \in f(G)$ тогда для некоторого $z_0 \in G$ выполнено $f(z_0) = w_0$. Если $f'(z_0) \ne 0$, то по теореме об обратной функции существуют числа $\delta > 0$ и $\epsilon > 0$ такие, что для любого $w \in B_\epsilon(w_0)$ уравнение $f(z) = w$ имеет единственное решение на $B_\delta(z_0)$, поэтому $B_\epsilon(w_0) \subset f(G)$. Если же, наоборот, $f'(z_0) = 0$, то, поскольку $f$ непостоянна, существует натуральное $n \ge 2$ такое, что $f^{(n)}(z_0) \ne 0$, и применима теорема о локальной структуре отображения, дающая аналогичный результат.
	
	Теперь докажем связность множества $f(G)$. Пусть $w_1, w_2 \in f(G)$ тогда для некоторых $z_1, z_2 \in G$ выполнено $f(z_1) = w_1$ и $f(z_2) = w_2$. Пусть $\gamma$ "--- кривая, соединяющая точки $z_1$ и $z_2$, такая что $M(\gamma) \subset G$. Тогда кривая $\Gamma = f(\gamma)$ соединяет точки $w_1$ и $w_2$, причем $M(\Gamma) \subset f(G)$. Получено требуемое.
\end{proof}

\begin{proposition}
	Пусть $z_0 \in \Cm$, $\epsilon > 0$, и функция $f$ непрерывна на $B_\epsilon(z_0)$ и регулярна на $\mathring B_\epsilon(z_0)$. Тогда $f$ регулярна на $B_\epsilon(z_0)$.
\end{proposition}

\begin{proof}
	По условию, функция $f$ представима рядом Лорана на $\mathring B_\epsilon(z_0)$:
	\[f(z) = \sum_{n = -\infty}^{+\infty}c_n(z-z_0)^n,~z \in \mathring B_\epsilon(z_0)\]
	
	В силу непрерывности функции $f$, точка $z_0$ является ее устранимой особой точкой, поэтому ряд Лорана не содержит главной части, то есть является степенным рядом и потому сходится на $B_\epsilon(z_0)$ по теореме Абеля. Тогда $f$ совпадает с суммой ряда в точке $z_0$ в силу непрерывности, поэтому $f \in C^1(B_\epsilon(z_0))$.
\end{proof}

\begin{note}
	Утверждение выше означает, что если переопределить функцию $f$ в ее устранимой особой точке $z_0$, то получится функция, регулярная в точке $z_0$.
\end{note}