\section{Приращение аргумента вдоль кривой}

\begin{note}
	Мы стремимся формализовать понятие приращения аргумента вдоль кривой. Например, на рисунке ниже кривая $\gamma$ совершает полтора оборота против часовой стрелки, то есть аргумент числа при перемещении вдоль этой кривой изменяется на $\frac{3\pi}2$:
\begin{center}
	\begin{tikzpicture}
		\clip (-3.5, -3) rectangle (3.5, 3);
		
		\clip (-3.4, -3) rectangle (3.4, 3);
		\draw [->] (-3, 0) -- (3, 0) node [above, black] {$\re z$};
		\draw [->] (0, -2.7) -- (0, 2.7) node [right, black] {$\im z$};
		
		\draw [black] (1.4,3pt) -- (1.4,-3pt) node [below, black] {$1$};
		\draw [black] (3pt,1.4) -- (-3pt,1.4) node [left, black] {$i$};
		
		
		\path
		coordinate (p1) at (-2.3, -2.3)
		coordinate (p2) at (0, 0.7)
		coordinate (p3) at (1.8, 0)
		coordinate (p4) at (0.8, -1.4)
		coordinate (p5) at (-0.7, -0.5)
		coordinate (p6) at (-0.3, 0.8)
		coordinate (p7) at (2.2, 2.2);
		
		\draw[
		black,
		decoration={markings, mark=at position 0.12 with {\arrow{<}}},
		decoration={markings, mark=at position 0.5 with {\arrow{<}}},
		decoration={markings, mark=at position 0.87 with {\arrow{<}}},
		postaction={decorate}
		] plot [smooth, tension=0.85] coordinates {(p1) (p2) (p3) (p4) (p5) (p6) (p7)};
		
		\draw [->, black] (0, 0) -- (1.1, 1.67) node [black, below right, scale = 1.1] {$z$};
		\node[draw, circle, inner sep=1pt, fill, black] at (1.13, 1.7) {};
		
		\coordinate (a1) at (1.1, 1.6);
		\coordinate (a2) at (0,0);
		\coordinate (a3) at (1, 0);
		
		\pic [draw, ->] {angle = a3--a2--a1};
		\node [] at (0.6, 0.45) {$\phi$};
		
		\node[draw, circle, inner sep=1pt, fill, black, label={above:$\phi_0 = \frac\pi4$}] at (p7) {};
		\node[draw, circle, inner sep=1pt, fill, black, label={below:$\phi_1 = \frac{13\pi}4$}] at (p1) {};
		\node[] at (1.1, -1.7) {$\gamma$};
	\end{tikzpicture}
\end{center}
\end{note}

\begin{proposition}
	Пусть выполнены следующие условия:
	\begin{enumerate}
		\item Функция $z(t) = x(t) + iy(t)$ непрерывно дифференцируема на отрезке $[t_0, t_1]$
		\item Для любого $t \in [t_0, t_1]$ выполнено $z(t) \ne 0$
		\item Зафиксировано значение $\phi_0 \in \Arg{z(t_0)}$
	\end{enumerate}

	Тогда существует единственная функция $\phi \in C^1[t_0, t_1]$ такая, что для любого числа $t \in [t_0, t_1]$ выполнено $\phi(t) \in \Arg z(t)$. Более того, $\phi$ задается следующей формулой:
	\[\phi(t) = \phi_0 + \int_{t_0}^t\frac{x(\tau)y'(\tau) - x'(\tau)y(\tau)}{x(\tau)^2 + y(\tau)^2}d\tau\]
\end{proposition}

\begin{proof}
	Пусть $\phi$ "--- функция, заданная формулой из условия. Условие, что для любого $\forall t \in [t_0, t_1]$ выполнено $\phi(t) \in \Arg z(t)$, равносильно следующей системе:
	\[\System{
		\cos\phi(t) \equiv \frac{x(t)}{\sqrt{x(t)^2 + y(t)^2}} \\
		\sin\phi(t) \equiv \frac{y(t)}{\sqrt{x(t)^2 + y(t)^2}}
	}\]
	
	Положим $u(t) := \cos\phi(t)$, $v(t) := \sin\phi(t)$, функции в правой части системы выше обозначим через $\widetilde u(t)$, $\widetilde v(t)$, а подынтегральную функцию из условия --- через $a(\tau)$. Тогда:
	\begin{gather*}
		u'(t) = (\cos\phi(t))'_t = (-\sin\phi(t))\phi'(t) = -v(t)a(t)\\
		v'(t) = (\sin\phi(t))'_t = (\cos\phi(t))\phi'(t) = u(t)a(t)
	\end{gather*}
	
	Значит, набор $(u(t), v(t))$ является решением следующей задачи Коши:
	\[\System{
	&u'(t) = -a(t)v(t)\\
	&v'(t) = a(t)u(t)\\
	&u(t_0) = \cos\phi(t_0)\\
	&v(t_0) = \sin\phi(t_0)}\]
	
	Остается заметить, что $(\widetilde u(t), \widetilde v(t))$ тоже является решением этой задачи Коши, откуда $u \equiv \widetilde u$ и $v \equiv \widetilde v$ на $[t_0, t_1]$. Проверим единственность функции $\phi$. Пусть $\psi \in C^1[t_0, t_1]$ "--- такая функция, что $\phi(t_0) = \psi(t_0)$, $\cos\phi(t) \equiv \cos\psi(t)$ и $\sin\phi(t) \equiv \sin\psi(t)$ на $[t_0, t_1]$. То $\phi' \equiv \psi'$ на $[t_0, t_1]$. Интегрируя это соотношение, получим, что $\phi \equiv \psi$ на $[t_0, t_1]$.
\end{proof}

\begin{definition}
	Пусть $\gamma$ "--- кривая с непрерывно дифференцируемой параметризацией $z(t)$, $t \in [t_0, t_1]$, такая, что для любого $t \in [t_0, t_1]$ выполнено $z(t) \ne 0$. \textit{Приращением аргумента вдоль кривой} $\gamma$ называется величина $\Delta_\gamma\arg{z} := \phi(t_1) - \phi(t_0)$, где $\phi$ "--- функция из утверждения выше.
\end{definition}

\begin{note}
	Заметим, что в утверждении выше $a(\tau) \equiv \im\frac{z'(t)}{z(t)}$ на $[t_0, t_1]$, поэтому формулу для приращения аргумента можно переписать в следующем виде:
	\[\Delta_\gamma \arg{z} = \im\int_{t_0}^{t_1}\frac{z'(\tau)}{z(\tau)}d\tau = \im\int_\gamma \frac{dz}{z}\]
\end{note}

\begin{definition}
	Пусть $\gamma$ "--- кривая с непрерывно дифференцируемой параметризацией $z(t)$, $t \in [t_0, t_1]$, функция $f$ регулярна на $M(\gamma)$, и для любого $t \in [t_0, t_1]$ выполнено $f(z(t)) \ne 0$. Пусть $\Gamma$ "--- кривая с параметризацией $f(z(t))$, $t \in [t_0, t_1]$. \textit{Приращением аргумента функции $f$ вдоль кривой} $\gamma$ называется величина $\Delta_\gamma\arg{f} := \Delta_\Gamma\arg{z}$.
\end{definition}

\begin{note}
	В силу уже доказанного, верна следующая формула:
	\[\Delta_\gamma\arg{f} = \im\int_\gamma \frac{f'(z)}{f(z)}dz\]
	
	Из этой формулы легко вывести следующие свойства:
	\begin{enumerate}
		\item Если кривая $\gamma$ представима в виде $\gamma = \gamma_1\dotsc \gamma_k$ и для кривых $\gamma_1, \dotsc, \gamma_k$ определены $\Delta_{\gamma_1}\arg f, \dotsc, \Delta_{\gamma_k}\arg f$, то:
		\[\Delta_{\gamma}\arg f = \sum_{i = 1}^k\Delta_{\gamma_i}\arg f\]
		
		Это позволяет обобщить определение приращения аргумента вдоль кривых, являющихся кусочно-непрерывно дифференцируемыми.
		
		\item Если для кривой $\gamma$ определено $\Delta_{\gamma}\arg f$, то:
		\[\Delta_{\gamma^{-1}}\arg f = -\Delta_{\gamma}\arg f\]
		
		\item (\textit{Логарифмическое свойство}) Пусть определены $\Delta_{\gamma}\arg f_1$ и $\Delta_{\gamma}\arg f_2$. Тогда:
		\begin{gather*}
			\Delta_{\gamma}\arg (f_1f_2) = \Delta_{\gamma}\arg f_1 + \Delta_{\gamma}\arg f_2
			\\
			\Delta_{\gamma}\arg \left(\frac{f_1}{f_2}\right) = \Delta_{\gamma}\arg f_1 - \Delta_{\gamma}\arg f_2
		\end{gather*}
	
		Убедимся в справедливости первой формулы:
		\[\Delta_{\gamma}\arg (f_1f_2) =  \im\int_\gamma \frac{(f_1f_2)'(z)}{f_1(z)f_2(z)}dz = \im\int_\gamma \left(\frac{f_1'(z)}{f_1(z)} + \frac{f_2'(z)}{f_2(z)}\right)dz = \Delta_{\gamma}\arg f_1 + \Delta_{\gamma}\arg f_2\]
	\end{enumerate}
\end{note}

\section{Принцип аргумента и теорема Руше}

\begin{theorem}[принцип аргумента]
	Пусть выполнены следующие условия:
	\begin{enumerate}
		\item $G$ "--- ограниченная односвязная область с простой границей
		
		\item Граничные кривые положительно ориентированы относительно $G$
		
		\item Функция $f$ регулярна на $\overline{G} \bs \{a_1, \dotsc, a_m\}$, где $a_1, \dotsc, a_m \in G$ "--- полюса
		
		\item $f \ne 0$ на $\partial G$
	\end{enumerate}
	
	Тогда $f$ имеет лишь конечное число нулей на $G$. Более, если $N$ и $P$ "--- числа нулей и полюсов функции $f$ на $G$ с учетом их порядков, то:
	\[\Delta_{\partial G} \arg f = 2\pi (N - P)\]
\end{theorem}

\begin{proof}
	Предположим, что $f$ имеет бесконечное число нулей на $G$. Тогда множество нулей имеет предельную точку $z_0 \in \overline{G}$, причем $z_0$ не может быть полюсом, поскольку иначе $f(z) \to \infty$, $z \to z_0$. Значит, $f$ непрерывна в точке $z_0$, откуда $f(z_0) = 0$, поэтому $z_0 \in G$. Но тогда, по теореме единственности, $f \equiv 0$ на $G \bs \{a_1, \dotsc, a_m\}$, и $f \equiv 0$ на $\partial G$ в силу непрерывности --- противоречие.
	
	Пусть $a_1, \dotsc, a_m \in G$ "--- полюса функции $f$ порядков $p_1, \dotsc, p_m$, $b_1, \dots, b_n \in G$ "--- нули функции $f$ порядков $q_1, \dotsc, q_n$, тогда $\sum_{k = 1}^mp_k = P$ и $\sum_{k = 1}^nq_k = N$. По формуле для приращения аргумента функции вдоль кривой:
	\[\Delta_{\partial G} \arg f = \im\int_{\partial G}\frac{f'(z)}{f(z)}dz = 2\pi\re \left(\sum_{k = 1}^m\res_{a_k}\frac{f'}{f} + \sum_{k = 1}^n\res_{b_k}\frac{f'}{f}\right)\]
	
	Зафиксируем нуль $b_k$. В некоторой его проколотой окрестности $\mathring B(b_k)$ выполнено равенство $f(z) = (z-b_k)^{p_k}h(z)$, где $h$ "--- регулярная на $B(b_k)$ функция, $h \ne 0$ на $B(b_k)$. Тогда $f'(z) = p_k(z-b_k)^{p_k - 1}h(z) + (z-b_k)^{p_k}h'(z)$. Значит, выполнено следующее:
	\[\res_{b_k}\frac{f'}{f} = \res_{b_k}\frac{p_k}{z-b_k} + \res_{b_k}\frac{h'}{h} = p_k + 0 = p_k\]
	
	Аналогично, зафиксируем полюс $a_k$. В некоторой его проколотой окрестности $\mathring B(a_k)$ выполнено равенство $f(z) = (z-a_k)^{-q_k}h(z)$, где $h$ "--- регулярная на $B(a_k)$ функция, $h \ne 0$ на $B(a_k)$. Тогда $f'(z) = -q_k(z-a_k)^{-q_k - 1}h(z) + (z-a_k)^{-q_k}h'(z)$. Значит, выполнено следующее:
	\[\res_{a_k}\frac{f'}{f} = \res_{a_k}\frac{-q_k}{z-a_k} + \res_{a_k}\frac{h'}{h} = -q_k + 0 = -q_k\]
	
	Подставляя полученные вычеты в формулу для приращения аргумента функции, полуачем требуемое.
\end{proof}

\begin{note}
	Число полюсов или нулей в теореме выше может быть и нулевым.
\end{note}

\begin{theorem}[Руше]
	Пусть выполнены следующие условия:
	\begin{enumerate}
		\item $G$ "--- ограниченная односвязная область с простой границей
		
		\item Граничные кривые положительно ориентированы относительно $G$
		
		\item Функции $f, g$ регулярна на $\overline{G}$
		
		\item $|f| > |g|$ на $\partial G$
	\end{enumerate}
	
	Тогда функции $f$ и $f+g$ на $G$ имеют одинаковое число нулей с учетом их порядков.
\end{theorem}

\begin{proof}
	Из условия, в частности, следует, что $f \ne 0$ на $\partial G$, поскольку $|f| > 0$ на $\partial G$. Пусть $N_f$, $N_{f+g}$ "--- число нулей функций $f$ и $f+g$ на $G$ с учетом их порядков. По принципу аргумента, выполнено следующее:
	\begin{gather*}
		\Delta_{\partial G}\arg f = 2\pi N_f
		\\
		\Delta_{\partial G}\arg (f+g) = 2\pi N_{f+g}
	\end{gather*}
	
	С другой стороны:
	\[\Delta_{\partial G}\arg (f+g) = \Delta_{\partial G}\arg \left(f\left(1 + \frac gf\right)\right) = \Delta_{\partial G}\arg f + \Delta_{\partial G}\arg \left(1 + \frac gf\right)\]
	
	Рассмотрим функцию $h = 1 + \frac gf$, регулярную на $\partial G$. Пусть кривая $\Gamma$ "--- образ границы $\partial G$ под действием функции $h$. Тогда:
	\[\Delta_{\partial G}\arg \left(1 + \frac gf\right) = \Delta_\Gamma \arg z\]
	
	Но для каждого $z \in \Gamma$ выполнено $|z - 1| = \left|\frac gf\right| < 1$. Тогда:
	\begin{center}
		\begin{tikzpicture}
			\clip (-1.5, -2.5) rectangle (4, 2.5);
			
			\draw [->] (-1, 0) -- (3.6, 0) node [above, black] {$\re z$};
			\draw [->] (0, -2.1) -- (0, 2.1) node [right, black] {$\im z$};
			
			\draw [black] (1.5,3pt) -- (1.5,-3pt) node [below, black] {$1$};
			\draw [black] (3pt,1.5) -- (-3pt,1.5) node [left, black] {$i$};
			
			
			\path
			coordinate (p1) at (2, 1.1)
			coordinate (p2) at (2.3, 0.8)
			coordinate (p3) at (2, -0.3)
			coordinate (p4) at (1.6, -0.6)
			coordinate (p5) at (1.2, -0.7)
			coordinate (p6) at (0.3, -0.3)
			coordinate (p66) at (0.6, -0.8)
			coordinate (p666) at (0.9, -0.3)
			coordinate (p7) at (0.3, 0.3)
			coordinate (p8) at (0.6, 0.5)
			coordinate (p9) at (1.1, 1.1);
			
			\draw[
			black,
			decoration={markings, mark=at position 0.12 with {\arrow{<}}},
			decoration={markings, mark=at position 0.64 with {\arrow{<}}},
			decoration={markings, mark=at position 0.87 with {\arrow{<}}},
			postaction={decorate}
			] plot [smooth, tension=0.85] coordinates {(p1) (p2) (p3) (p4) (p5) (p6) (p66) (p666) (p7) (p8) (p9) (p1)};
			
			\draw[black, dashed] (1.5, 0) circle[radius=1.5];
			
			\draw [->, black] (0, 0) -- (1.97, 1.07) node [black, below, scale = 1.1] {$z$};
			
			\node[draw, circle, inner sep=1pt, fill, black] at (p1) {};
			
			\coordinate (a2) at (0,0);
			\coordinate (a3) at (1, 0);
			
			\pic [draw, ->, angle radius = 0.8cm] {angle = a3--a2--p1};
			\node [] at (0.98, 0.25) {$\phi$};
			\node[] at (2, -1) {$\Gamma$};
		\end{tikzpicture}
	\end{center}
	
	Геометрически ясно, что $\Delta_\Gamma\arg{z} = 0$. Значит:
	\[\Delta_{\partial G}\arg (f+g) = \Delta_{\partial G}\arg f \ra N_f = N_{f+g}\]
	
	Получено требуемое.
\end{proof}

\section{Целые функции и основная теорема алгебры}

\begin{definition}
	Функция $f$, регулярная на $\Cm$, называется \textit{целой}.
\end{definition}

\begin{note}
	Каждая целая функция представима рядом Тейлора на всей плоскости $\Cm$:
	\[f(z) = \sum_{n=0}^{+\infty}c_nz^n,~z \in \Cm\]
\end{note}

\begin{proposition}
	Пусть выполнены следующие условия:
	\begin{enumerate}
		\item Функция $f$ "--- целая
		\item Существуют числа $R_0, A > 0$ и $m \in \N \cup \{0\}$ такие, что на кольце $\{z \in \Cm: |z| > R_0\}$ выполнено неравенство $|f(z)| \le A|z|^m$
	\end{enumerate}
	
	Тогда $f$ является многочленом степени не выше $m$.
\end{proposition}

\begin{proof}
	Для каждого $R > R_0$ положим $M(R) := \max\{f(z): |z| = R\}$. По неравенству Коши для коэффициентов ряда Лорана, для каждого $n >m $ имеем:
	\[c_n \le \frac{M(R)}{R^n} \le \frac{AR^m}{R^n} \xrightarrow{R \to +\infty} 0\]
	
	Значит, $f = \sum_{n = 0}^mc_nz^n$.
\end{proof}

\begin{corollary}[теорема Лиувилля]
	Если $f$ "--- целая функция, и для некоторого $R_0 > 0$ она ограниченна на $\{z \in \Cm: |z| > R_0\}$, то $f$ постоянна на $\Cm$.
\end{corollary}

\begin{proof}
	Применим утверждение выше для случая, когда $m = 0$.
\end{proof}

\begin{theorem}[основная теорема алгебры]
	Любой многочлен $p \in \Cm[z]$, степень которого не меньше одного, имеет корень в $\Cm$.
\end{theorem}

\begin{proof}
	Предположим противное, тогда функция $\frac 1p$ "--- целая, причем выполнено $\lim_{z \to \infty} \frac1p(z) = 0$, то есть эта функция ограниченна в $\Cm$. Тогда, по теореме Лиувилля, $\frac 1p$ постоянна на $\Cm$, но тогда и функция $p$ постоянна на $\Cm$ --- противоречие.
\end{proof}