\section{Комплексные числа}

\begin{definition}
	\textit{Множеством комплексных чисел} $\Cm$ называется множество $\R^2$ с арифметическими операциями, введенными следующим образом:
	\begin{itemize}
		\item $\forall (x_1, y_1), (x_2, y_2) \in \Cm: (x_1, y_1) + (x_2, y_2) := (x_1 + x_2, y_1 + y_2)$
		\item $\forall (x_1, y_1), (x_2, y_2) \in \Cm: (x_1, y_1)\cdot(x_2, y_2) := (x_1x_2 - y_1y_2, x_1y_2 + x_2y_1)$
	\end{itemize}
\end{definition}

\begin{note}
	В предыдущих курсах уже доказывалось, что комплексные числа с определенными выше арифметическими операциями образуют \textit{поле}. Отметим также, что имеет место естественное вложение $\R \emb \Cm$, осуществляемое сопоставлением $x \mapsto (x, 0)$, поэтому числа вида $(x, 0) \in \Cm$ мы далее будем обозначать через $x$.
\end{note}

\begin{definition}
	\textit{Модулем} числа $z = (x, y) \in \Cm$ называется величина $|z| := \sqrt{x^2 + y^2}$.
\end{definition}

\begin{note}
	Модуль комплексного числа "--- это евклидова норма на $\R^2$. Для евклидовой нормы в предыдущих курсах доказывалось \textit{неравенство треугольника}, согласно которому для любых $z_1, z_2 \in \Cm$ выполнено неравенство $|z_1 + z_2| \le |z_1| + |z_2|$.
\end{note}

\begin{definition}
	Элемент $i = (0, 1) \in \Cm$ называется \textit{мнимой единицей}.
\end{definition}

\begin{note}
	Из определения выше следует, что $i^2 = -1$. Другой способ построения поля комплексных чисел предполагает расширение поля $\R$ корнем многочлена $x^2 + 1$.
\end{note}

\begin{definition}
	\textit{Алгебраической формой записи} комплексного числа $z = (x, y) \in \Cm$ называется выражение $x + iy$. \textit{Вещественной частью} числа $z$ называется число $\re{z} := x$, \textit{мнимой частью} --- число $\im{z} := y$. Комплексные числа с нулевой вещественной частью называются \textit{чисто мнимыми}.
\end{definition}

\begin{definition}
	Числом, \textit{сопряженным} к числу $z = x + iy \in \Cm$, называется число $\overline{z} := x - iy$. Сопоставление $z \mapsto \overline{z}$ называется \textit{сопряжением}.
\end{definition}

\begin{note}
	Нетрудно проверить, что операция сопряжения является автоморфизмом поля $\Cm$. В частности, она коммутирует со всеми арифметическими операциями в $\Cm$. Отметим также, что для любого $z \in \Cm$ выполнено $|z|^2 = z\overline{z}$.
\end{note}

\begin{proposition}[мультипликативность сопряжения]
	Для любых $z_1, z_2 \in \Cm$ выполнено равенство $|z_1z_2| = |z_1||z_2|$.
\end{proposition}

\begin{proof}
	Заметим, что выполнена следующая цепочка равенств:
	\[|z_1z_2|^2 = z_1z_2\overline{z_1z_2} = z_1\overline{z_1}z_2\overline{z_2} = |z_1|^2|z_2|^2\]
	
	Извлекая корень из обеих частей равенства $|z_1z_2|^2 = |z_1|^2|z_2|^2$, получим требуемое.
\end{proof}

\begin{proposition}[неравенство Коши]
	Для любых $z_1, \dotsc, z_n, w_1, \dotsc, w_n \in \Cm$ выполнено следующее неравенство:
	\[\left|\sum_{k=1}^nz_kw_k\right|^2 \le \left(\sum_{k=1}^n|z_k|^2\right)\left(\sum_{k=1}^n|w_k|^2\right)\]
\end{proposition}

\begin{proof}
	Положим $A := \sum_{k=1}^n|z_k|^2$, $B := \sum_{k=1}^n|w_k|^2$, $C := \sum_{k=1}^nz_kw_k$. Случай, когда $B = 0$, тривиален, поэтому рассмотрим случай, когда $B > 0$. Заметим следующее:
	\begin{multline*}
		0 \le \sum_{k = 1}^n|Bz_k - C\overline{w_k}|^2 = \sum_{k = 1}^n(Bz_k - C\overline{w_k})(B\overline{z_k} - \overline Cw_k) =
		\\
		= B^2\sum_{k=1}^n|z_k|^2 - BC\sum_{k=1}^n\overline{z_k}\overline{w_k} - B\overline{C}\sum_{k=1}^nz_kw_k + |C|^2\sum_{k=1}^n|w_k|^2 = B^2A - B|C|^2
	\end{multline*}
	
	Разделив обе части неравенства $0 \le B^2A - B|C|^2$ на $B > 0$, получим требуемое.
\end{proof}

\begin{definition}
	\textit{Комплексной плоскостью} называется плоскость $\R^2$ с введенной на ней прямоугольной декартовой системой координат, в которой каждое комплексное число $z = x + iy \in \Cm$ имеет координаты $(x, y)$.
\end{definition}

\begin{note}
	На рисунке ниже изображена комплексная плоскость, число $z \in \Cm$ на ней, а также числа $-z$ и $\overline{z}$.
	\begin{center}
		\scalebox{1}{
			\begin{tikzpicture}
				\clip (-3.4, -3) rectangle (3.4, 3);
				\draw [->] (-3, 0) -- (3, 0) node [above, black] {$\re z$};
				\draw [->] (0, -2.7) -- (0, 2.7) node [right, black] {$\im z$};
				
				\draw [line width = 1pt, black!15!blue] (1.4,3pt) -- (1.4,-3pt) node [below, black] {$1$};
				\draw [line width = 1pt, black!15!blue] (3pt,1.4) -- (-3pt,1.4) node [left, black] {$i$};
				
				\draw [->, black!15!blue] (0, 0) -- (2, 1.5) node [black, above right, scale = 1.2] {$z$};
				\node[draw, circle, inner sep=1.4pt, fill, black!15!blue] at (2.06, 1.54) {};
				
				\draw [->, black!15!blue] (0, 0) -- (-2, -1.5) node [black, below left, scale = 1.2] {$-z$};
				\node[draw, circle, inner sep=1.4pt, fill, black!15!blue] at (-2.06, -1.54) {};
				
				\draw [->, black!15!blue] (0, 0) -- (2, -1.5) node [black, below right, scale = 1.2] {$\overline z$};
				\node[draw, circle, inner sep=1.4pt, fill, black!15!blue] at (2.06, -1.54) {};
				
				\coordinate (a1) at (2, 1.5);
				\coordinate (a2) at (2, -1.5);
				\coordinate (a3) at (-2, -1.5);
				\coordinate (b) at (0, 0);
				\coordinate (c) at (1, 0);
				\coordinate (c2) at (-1, 0);
				
				\pic [draw, ->] {angle = c--b--a1};
				\pic [draw, <-, angle radius = 0.7cm] {angle = a2--b--c};
				\pic [draw, ->] {angle = c2--b--a3};
				\node [] at (0.75, 0.25) {$\phi$};
				\node [] at (0.9, -0.3) {$\phi$};
				\node [] at (-0.75, -0.25) {$\phi$};
			\end{tikzpicture}
		}
	\end{center}
\end{note}

\section{Тригонометрическая форма записи комплексного числа}

\begin{definition}
	\textit{Аргументом} числа $z \in \Cm \backslash \{0\}$ называется следующий набор чисел:
	\[\Arg{z} := \left\{\phi \in \R: \cos\phi = \frac{\re z}{|z|}, \sin\phi = \frac{\im z}{|z|}\right\}\]
	
	Любой элемент множества $\Arg{z}$ обозначается через $\arg{z}$. \textit{Главным значением аргумента} числа $z$ называется единственное число $\arg_0{z}$ такое, что $\arg_0{z} \in \Arg{z} \cap (-\pi, \pi]$.
\end{definition}

\begin{note}
	Легко видеть, что $\forall z \in \Cm \bs \{0\}: \Arg{z} = \{\arg_0{z} + 2\pi k: k \in \Z\}$. Отметим также, что для любых $z_1, z_2 \in \Cm$ выполнены следующие равенства:
	\begin{align*}
		|z_1z_2| = |z_1||z_2|\\
		\arg{z_1z_2} = \arg{z_1} + \arg{z_2}
	\end{align*}
	
	Первое равенство выполнено в силу мультипликативности модуля, второе же легко проверить, используя формулы тригонометрии.
\end{note}

\begin{definition}
	\textit{Тригонометрической формой записи} комплексного числа $z \in \Cm \bs \{0\}$ с модулем $r$ и значением аргумента $\phi$ называется выражение $r(\cos\phi + i\sin\phi)$.
\end{definition}

\begin{note}
	Рассмотрим числа $z_1, z_2 \in \Cm \bs \{0\}$ с алгебраическими записями $x_1 + iy_1, x_2 + iy_2$ и тригонометрическими записями $r_1(\cos\phi_1 + i\sin\phi_1), r_2(\cos\phi_2 + i\sin\phi_2)$. Тогда выполнены следующие эквивалентности:
	\[z_1 = z_2 \lra \System{x_1 = x_2 \\ y_1 = y_2} \lra \System{&r_1 = r_2 \\ &\exists k \in \Z: \phi_1 = \phi_2 + 2\pi k}\]
\end{note}

\section{Комплексная экспонента}

\begin{definition}
	\textit{Комплексной экспонентой} называется функция $f : \Cm \to \Cm$, сопоставляющая каждому числу $z = x + iy \in \Cm$ число следующего вида:
	\[f(z) = e^z = e^{x + iy} := e^x(\cos{y} + i\sin{y})\]
\end{definition}

\begin{proposition}[свойства комплексной экспоненты]~
	\begin{enumerate}
		\item Определение комплексной экспоненты от действительного числа согласованно со стандартным определением экспоненты.
		\item $\forall z \in \Cm: e^z \ne 0$.
		\item $\forall z_1, z_2 \in \Cm: e^{z_1}e^{z_2} = e^{z_1 + z_2}$.
		\item $\forall z \in \Cm: e^{-z} = \frac{1}{e^z}$.
		\item $\forall z \in \Cm: \forall n \in \N: (e^{z})^n = e^{zn}$.
	\end{enumerate}
\end{proposition}

\begin{proof}~
	\begin{enumerate}
		\item Следует непосредственно из определения.
		\item Заметим, что для любого числа $w \in \Cm$ выполнено $w = 0 \lra |w| = 0$. Но для любого числа $z = x + iy \in \Cm$ выполнено $|e^z| = e^x$, поэтому $e^z \ne 0$.
		\item Пусть $z_1 = x_1 + iy_1$, $z_2 = x_2 + iy_2$. Тогда выполнена следующая цепочка равенств:
		\begin{multline*}
			e^{z_1}e^{z_2} = e^{x_1}(\cos{y_1} + i\sin{y_1})e^{x_2}(\cos{y_2} + i\sin{y_2}) = \\
			= e^{x_1 + x_2}(\cos(y_1 + y_2) + i\sin(y_1 + y_2)) = e^{z_1 + z_2}
		\end{multline*}
		\item Следует непосредственно из свойства $(3)$.
		\item Проведем индукцию по $n$. База, $n = 1$, тривиальна. Докажем переход, пользуясь свойством $(3)$:
		\[\left(e^z\right)^n = \left(e^z\right)^{n-1}e^z = e^{z(n-1)}e^z = e^{z(n-1) + z} = e^{zn}\qedhere\]
	\end{enumerate}
\end{proof}

\begin{definition}
	Пусть $E \subset \Cm$, $T \in \Cm \bs \{0\}$. Функция $f : E \to \Cm$ называется \textit{периодической с периодом $T$}, если выполнены следующие условия:
	\begin{enumerate}
		\item $\forall z \in E: z \pm T \in E$
		\item $\forall z \in E: f(z + T) = f(z)$
	\end{enumerate}
\end{definition}

\begin{proposition}
	Любое число вида $2\pi k i, k \in \Z$, является периодом комплексной экспоненты, причем других периодов у нее нет.
\end{proposition}

\begin{proof}
	С одной стороны, для любого $k \in \Z$ выполнено равенство $e^{z + 2\pi ki} = e^z$. С другой стороны, если $T = T_1 + iT_2 \in \Cm$ "--- период комплексной экспоненты, то, в частности, $e^{T_1 + iT_2} = e^0 = 1$, откуда $T_1 = 0$ и $T_2 = 2\pi k$ для некоторого $k \in \Z$.
\end{proof}

\begin{definition}
	\textit{Показательной формой записи} комплексного числа $z \in \Cm \bs \{0\}$ с модулем $r$ и значением аргумента $\phi$ называется выражение $re^{i\phi}$.
\end{definition}

\section{Логарифм комплексного числа}

\begin{definition}
	\textit{Логарифмом} числа $z \in \Cm \bs \{0\}$ называется следующий набор чисел:
	\[\Ln{z} := \left\{w \in \Cm : e^w = z\right\}\]
	
	Любой элемент множества $\Ln{z}$ обозначается через $\ln{z}$.
\end{definition}

\begin{proposition}
	$\forall z = re^{i\phi} \in \Cm \bs \{0\} : \Ln{z} = \{ \ln{r} + i(\phi + 2\pi k) : k \in \Z\}$.
\end{proposition}

\begin{proof}
	С одной стороны, для любого $k \in \Z$ выполнены следующие равенства: \[e^{\ln{r} + i(\phi + 2\pi k)} = re^{i\phi} = z\]
	
	С другой стороны, если $w = u + iv \in \Ln{z}$, то $e^{u + iv} = z$, откуда $e^u = r \lra u = \ln{r}$ и $v = 2\pi k$ для некоторого $k \in \Z$.
\end{proof}

\begin{definition}
	\textit{Главным значением логарифма} числа $z \in \Cm \bs \{0\}$ называется число $\ln_0{z} := \ln|z| + i\arg_0{z}$.
\end{definition}

\section{Корень из комплексного числа}

\begin{definition}
	\textit{Множеством корней степени $n \in \N$, $n > 1$}, из числа $z \in \Cm$ называется следующий набор чисел:
	\[\Root{n}{z} := \left\{w \in \Cm: w^n = z\right\}\]
\end{definition}

\begin{note}
	Легко видеть, что $\forall n \in \N$, $n > 1: \Root{n}{0} = \{0\}$.
\end{note}

\begin{proposition}
	$\forall z = re^{i\phi} \in \Cm \bs \{0\}: \forall n \in \N$, $n > 1: \Root{n}{z} = \left\{\!\sqrt[n]{r}e^{i\frac{\phi + 2\pi k}{n}} : k \in \Z\right\}$.
\end{proposition}

\begin{proof}
	С одной стороны, для любого $k \in \Z$ выполнено следующие равенства:
	\[\left(\!\sqrt[n]{r}e^{i\frac{\phi + 2\pi k}{n}}\right)^n = re^{i\phi} = z\]
	
	С другой стороны, если $w \in \Root{n}{z}$, то $w \ne 0$, тогда $w$ можно представить в показательной форме как $w = \rho e^{i\theta}$, причем $\rho^ne^{i\theta n} = re^{u\phi}$, откуда $\rho = \sqrt[n]{r}$ и $\theta n = \phi + 2\pi k$ для некоторого $k \in \Z$.
\end{proof}

\begin{note}
	В силу периодичности комлпексной экспоненты, чтобы получить все корни степени $n \in \N$, $n > 1$, по формуле из утверждения выше, достаточно взять все $k$ из множества $\{0, \dotsc, n - 1\}$, причем все полученные корни будут различными. На комплексной плоскости $n$ корней образуют правильный $n$-угольник.
\end{note}

\begin{definition}
	\textit{Главным значением корня степени} $n \in \N$, $n > 1$, из числа $z \in \Cm \bs \{0\}$ называется следующее число:
	\[\left(\!\sqrt[n]{z}\right)_0 := \sqrt[n]{|z|}e^{i\frac{\arg_0z}n}\]
\end{definition}

\section{Комплекснозначные функции вещественного переменного}

\begin{note}
	Зафиксируем функцию $f : \R \to \Cm$ и представим ее в виде $f = g + ih$, $g, h : \R \to \R$. Тогда для любого $t \in \R$ выполнено $f(t) = (g(t), h(t))$. Из этого равенства следуют такие свойства, ранее полученные для вектор-функций:
	\begin{enumerate}
		\item Функция $f$ непрерывна в точке $t_0 \in \R$ $\lra$ функции $g, h$ непрерывны в $t_0$
		\item Функция $f$ дифференцируема в точке $t_0 \in \R$ $\lra$ функции $g, h$ дифференцируемы в $t_0$, причем $f'(t_0) = g'(t_0) + ih'(t_0)$
		\item Функция $f$ интегрируема на отрезке $[a, b] \subset \R$ $\lra$ функции $g, h$ интегрируемы на $[a, b]$, причем $\int_a^bf(t)dt = \int_a^bg(t)dt + i\int_a^bh(t)dt$
		\item Функция $F = G + iH$ является первообразной функции $f$ на отрезке $[a, b] \subset \R \hm\lra$~функции $G, H$ являются первообразными функций $g, h$ на $[a, b]$
	\end{enumerate}
\end{note}

\begin{example}
	Вычислим производную функции $f(t) = e^{i\omega t} = \cos(\omega t) + i\sin(\omega t)$, $\omega \in \R$:
	\[f'(t) = \left(e^{i\omega t}\right)' = \left(\cos(\omega t)\right)' + i \left(\sin(\omega t)\right)' = i\omega(\cos(\omega t) + i\sin(\omega t)) = i\omega e^{i \omega t}\]
	
	Аналогичным образом можно убедиться, что для функции $g(t) = e^{at}$ при любом $a \in \Cm$ выполнено $g'(t) = ae^{at}$. Следовательно, если $a \ne 0$, то $\int e^{at}dt = \frac{e^{at}}{a} + C$, где $C \in \Cm$.
\end{example}

\section{Кривые и множества на комплексной плоскости}

\begin{definition}
	Пусть $\epsilon > 0$, $z_0 \in \Cm$. Множество $B_\epsilon(z_0) := \{z \in \Cm: |z - z_0| < \epsilon\}$ называется \textit{$\epsilon$-окрестностью} точки $z_0$.
\end{definition}

\begin{definition}
	\textit{Расширенной комплексной плоскостью} называется множество вида $\overline{\Cm} := \Cm \cup \{\infty\}$, где $\infty$ "--- объект, называемый \textit{бесконечно удаленной точкой}. Считается также, то $B_\epsilon(\infty) := \{z \in \Cm: |z| > \epsilon\} \cup \{\infty\}$ для любого $\epsilon > 0$
\end{definition}

\begin{definition}
	Пусть $\epsilon > 0$, $z_0 \in \overline{\Cm}$. Множество $\dot B_\epsilon(z_0) := B_\epsilon(z_0) \backslash \{z_0\}$ называется \textit{проколотой $\epsilon$-окрестностью} точки $z_0$.
\end{definition}

\begin{definition}
	\textit{Кривой} $\gamma$ на комплексной плоскости называется непрерывное отображение отрезка $[a, b] \subset \R$, $a < b$, в $\Cm$. \textit{Параметризацией} кривой $\gamma$ называется соответствующая 	функция $\sigma(t) = \xi(t) + i\eta(t)$, $t \in [a, b]$. \textit{Носителем} кривой $\gamma$ называется множество $M(\gamma) := \sigma([a, b])$. 
\end{definition}

\begin{note}
	Рассмотрим кривые $\gamma_1$ и $\gamma_2$ с параметризациями $z = \sigma_1(t_1)$, $t \in [a_1, b_1]$ и $z \hm= \sigma_2(t)$, $t \in [a_1, b_1]$. Пусть существует $f : [a_1, b_1] \to [a_2, b_2]$ "--- непрерывная строго монотонная функция такая, что $\sigma_2(f(t)) \equiv \sigma(t)$ на $[a_1, b_1]$. Тогда естественно считать, что кривые $\gamma_1, \gamma_2$ совпадают, то есть $\gamma_1 = \gamma_2$.
\end{note}

\begin{definition}
	Пусть $\gamma$ "--- кривая с параметризацией $z = \sigma(t)$, $t \in [a, b]$. Кривая называется:
	\begin{itemize}
		\item \textit{Простой}, если параметризация $\sigma$ инъективна
		\item \textit{Замкнутой}, если $\sigma(a) = \sigma(b)$
		\item \textit{Простой замкнутой}, или \textit{контуром}, если она замкнута и сужение $\sigma|_{[a, b)}$ инъективно
	\end{itemize}
\end{definition}

\begin{example}
	Зафиксируем кривую $\gamma$ и рассмотрим кривую $\gamma^{-1}$ с параметризацией $z = \sigma(-t)$, $t \in [-b, -a]$. Тогда <<конкатенация>> этих кривых $\gamma\gamma^{-1}$ "--- замкнутая, но не простая.
\end{example}

\begin{definition}
	Пусть $\gamma$ "--- кривая с параметризацией $z = \sigma(t)$, $t \in [a, b]$, и зафиксировано $\{t_k\}_{k = 0}^n$ "--- \textit{разбиение отрезка} $[a, b]$, то есть $a = t_0 < t_1 < \dotsb < t_n = b$. Кривая, заданная сужением $\sigma|_{[t_{k-1}, t_k]}$, $k \in \{1, \dotsc, n\}$, называется \textit{дугой} кривой $\gamma$ и обозначается через $\gamma_k$. Говорят, что $\gamma$ \textit{разбита} на дуги $\gamma_1, \dotsc, \gamma_n$, или \textit{составлена} из этих дуг.
\end{definition}

\begin{definition}
	Кривая $\gamma$ называется \textit{гладкой}, если она имеет параметризацию вида $z = \sigma(t) = \xi(t) + i\eta(t)$, $t \in [a, b]$, где $\xi, \eta \in C^{1}[a, b]$ и для всех $t \in [a, b]$ выполнено $\sigma'(t) \ne 0$, \textit{кусочно-гладкой}, если ее можно разбить на конечное число гладких дуг.
\end{definition}

\begin{note}
	В предыдущих курсах было доказано, что если $\gamma$ "--- кусочно-гладкая кривая, то она \textit{спрямляема}, то есть имеет конечную длину $|\gamma|$, которую можно вычислить по следующей формуле:
	\[|\gamma| = \int_a^b|\sigma'(t)|dt\]
\end{note}

\begin{note}
	Множество $E \subset \overline\Cm$ называется \textit{открытым}, если каждая точка $z \in E$ содержится в нем вместе с некоторой окрестностью $B_\epsilon(z)$.
\end{note}

\begin{definition}
	Пусть $z_0 \in \overline\Cm$, $E \subset \overline{\Cm}$. Точка $z_0$ называется \textit{предельной точкой} множества $E$, если любая ее окрестность $B_\epsilon(z)$ содержит точку из множества $E$, \textit{граничной точкой} множества $E$, если любая ее окрестность $B_\epsilon(z)$ содержит и точку из $E$, и точку не из $E$. \textit{Границей} множества $E$ называется множество $\partial E$ всех ее граничных точек.
\end{definition}

\begin{definition}
	Множество $E \subset \Cm$ называется \textit{линейно связным}, если для любых точек $z_1, z_2 \in E$ существует кривая $\gamma$ с параметризацией $z = \sigma(t)$, $t \in [a, b]$, такая, что $\sigma(a) = z_1$, $\sigma(b) = z_2$ и $M(\gamma) \subset E$.
\end{definition}

\begin{definition}
	Множество $G \subset \Cm$ называется \textit{областью}, если выполнены следующие условия:
	\begin{enumerate}
		\item Множество $G$ открыто
		\item Множество $G \bs \{\infty\}$ линейно связно
	\end{enumerate}
\end{definition}

\begin{note}
	Из предыдущих курсов известно, что, например, множества $B_\epsilon(z)$ и $\dot B_\epsilon(z)$ являются областями для любых $z \in \overline\Cm$ и $\epsilon > 0$.
\end{note}

\begin{definition}
	Множество $E \subset \Cm$ называется \textit{ограниченным}, если существует $C > 0$ такое, что для любого $z \in E$ выполнено $|z| < E$.
\end{definition}

\begin{note}
	Пусть $D \subset \Cm$ "--- ограниченная область. Если для некоторого кусочно-гладкого контура $\gamma$ выполнено равенство $M(\gamma) = \partial D$, то естественно считать этот контур границей области $D$.
\end{note}

\begin{definition}
	Область $G \subset \overline\Cm$ называется \textit{односвязной}, если для любой ограниченной области $D \subset \Cm$, границей которой является кусочно-гладкий контур $\gamma$, такой, что $M(\gamma) \subset G$, выполнено либо $D \subset G$, либо $(\overline\Cm \bs D) \subset G$.
\end{definition}

\section{Последовательности и ряды комплексных чисел}

\begin{definition}
	\textit{Последовательностью} в $\overline\Cm$ называется отображение $\N \to \overline\Cm$. Обозначение "--- $\{z_n\}$. \textit{Элементом} последовательности называется пара $(n, z_n)$ такая, что $n \mapsto z_n$, \textit{значением элемента} называется число $z_n \in \overline\Cm$.
\end{definition}

\begin{definition}
	Число $a \in \overline\Cm$ называется \textit{пределом последовательности} $\{z_n\}$, если выполнено условие $\forall \epsilon > 0 : \exists N \in \N: \forall n \ge N: z_n \in B_{\epsilon}(a)$. Обозначение "--- $\lim_{n \to \infty}z_n = a$.
\end{definition}

\begin{note}
	Легко видеть, что если $\lim_{n \to \infty}z_n = a \in \Cm$, то, начиная с некоторого номера, все значения элементов $z_n$ конечны, тогда $\lim_{n \to \infty}|z_n - a| = 0$. Кроме того, положим $|\infty| := \infty$ и заметим, что выполнены следующие эквивалентности:
	\begin{itemize}
		\item $\lim_{n \to \infty}z_n = \infty \lra \lim_{n \to \infty}|z_n| = \infty$
		\item $\lim_{n \to \infty}z_n = 0 \lra \lim_{n \to \infty}|z_n| = 0$
	\end{itemize}
\end{note}

\begin{note}
	Пусть $\{z_n\}, \{w_n\} \subset \Cm$, $\lim_{n \to \infty}z_n = a \in \Cm$, $\lim_{n \to \infty}w_n = b \in \Cm$. Аналогично вещественному случаю, можно доказать следующие факты:
	\begin{itemize}
		\item $\lim_{n \to \infty}|z_n| = |a|$
		\item $\lim_{n \to \infty}(z_n+w_n) = a+b$
		\item $\lim_{n \to \infty}z_nw_n = ab$
		\item Если $\forall n \in \N: w_n \ne 0$ и $b \ne 0$, то $\lim_{n \to \infty}\frac{z_n}{w_n} = \frac ab$
	\end{itemize}
\end{note}

\begin{proposition}
	Пусть $\{z_n\} = \{x_n + iy_n\} \subset \Cm$, $a = b+ic \in \Cm$. Тогда:
	\[\lim_{n \to \infty}z_n = a \lra \System{
		\lim_{n \to \infty}x_n = a\\
		\lim_{n \to \infty}y_n = b
	}\]
\end{proposition}

\begin{proof}
	Достаточно заметить, что выполнены следующие неравенства:
	\[\max\{|x_n - b|, |y_n - b|\} \le |z_n - a| \le |x_n - b| + |y_n - c|\qedhere\]
\end{proof}

\begin{definition}
	\textit{Числовым рядом} в $\Cm$ называется выражение вида $\sum_{n = 0}^\infty z_n$, где $z_n \in \Cm$ при всех $n \in \N$. \textit{Частной суммой} числового ряда называется число $S_n := \sum_{k = 0}^nz_k$, $n \in \N$. Если $\lim_{n \to \infty}S_n = S \in \Cm$, то ряд $\sum_{n = 0}^\infty z_n$ называется \textit{сходящимся}, а число $S$ называется его \textit{суммой}.
\end{definition}

\begin{definition}
	Числовой ряд $\sum_{n = 0}^\infty z_n$ называется \textit{сходящимся абсолютно}, если сходится ряд $\sum_{n = 0}^\infty|z_n|$.
\end{definition}

\begin{note}
	Для комплексных числовых рядов выполнен аналог \textit{критерия Коши} сходимости вещественных числовых рядов. В частности, из него легко вывести, что любой абсолютно сходящийся ряд сходится.
\end{note}

\begin{example}
	Ряд $\sum_{n=0}^\infty z^n$ сходится при любом $z \in \Cm$, $|z| < 1$, поскольку при таком $z$ сходится ряд $\sum_{n=0}^\infty |z|^n = \frac1{1-|z|}$. Вычислим его сумму:
	\[S_n = 1 + z + \dotsb + z^n = \frac{1-z^{n+1}}{1-z} \xrightarrow{n \to \infty} \frac1{1-z} \ra \sum_{n=0}^\infty z^n = \frac1{1-z}\]
\end{example}

\section{Предел и непрерывность функции в точке}

\begin{definition}
	Пусть функция $f$ определена в некоторой окрестности $\dot B_{\delta_0}(z_0)$ точки $z_0 \in \overline\Cm$. Число $A \in \CM$ называется \textit{пределом функции $f$ в точке $z_0$}, если выполнено одно из следующих условий:
	\begin{enumerate}
		\item $\forall \epsilon > 0: \exists 0 < \delta \le \delta_0: f(\dot B_\delta(z_0)) \subset B_\epsilon(A)$
		\item Для любой последовательности $\{z_n\} \subset \dot B_{\delta_0}(z_0)$ такой, что $\lim_{n \to \infty}z_n = z_0$, выполнено $\lim_{n \to \infty}f(z_n) = A$
	\end{enumerate}
	
	Обозначение "--- $\lim_{z \to z_0}f(z) = A$.
\end{definition}

\begin{note}
	Аналогично вещественному случаю, можно доказать эквивалентность условий в определении предела функции в точке. Из второго условия и свойств предела последовательности легко вывести, что если $a = b + ic, z_0 = x_0 + iy_0 \in \Cm$, $f(z) = f(x + iy) \hm= u(x, y) + iv(x, y)$, то:
	\[\lim_{z \to z_0}f(z) = a \lra \System{
		\lim_{(x, y) \to (x_0, y_0)}u(x, y) = b\\
		\lim_{(x, y) \to (x_0, y_0)}v(x, y) = c
	}\]
\end{note}

\begin{note}
	Пусть $z_0 \in \CM$, функции $f, g$ таковы, что $\lim_{z \to z_0}f(z) = a \in \Cm$, $\lim_{z \to z_0}g(z) \hm= b \in \Cm$. Аналогично вещественному случаю, можно доказать следующие факты:
	\begin{itemize}
		\item $\lim_{z \to z_0}(f(z) + g(z)) = a+b$
		\item $\lim_{z \to z_0}f(z)g(z) = ab$
		\item Если $\exists \delta > 0: \forall z \in \dot B_\delta(z_0) : g(z) \ne 0$ и $b \ne 0$, то $\lim_{z \to z_0}\frac{f(z)}{g(z)} = \frac ab$
	\end{itemize}
\end{note}

\begin{definition}
	Пусть функция $f$ определена в некоторой окрестности $B_{\delta_0}(z_0)$ точки $z_0 \in \CM$. Функция $f$ называется \textit{непрерывной в точке $z_0$}, если $\lim_{z \to z_0}f(z) = f(z_0)$.
\end{definition}

\begin{note}
	Легко заметить, что если $f(z) = f(x + iy) \hm= u(x, y) + iv(x, y)$, то $f$ непрерывна в точке $z_0 = x_0 + iy_0 \in \Cm$ $\lra$ $u, v$ непрерывны в точке $(x_0, y_0)$, поскольку аналогичная эквивалентность выполнена для предела функции в точке.
\end{note}

\begin{example}
	Рассмотрим функцию $f : \overline{\Cm} \to \CM$ следующего вида:
	\[f(z) = \System{
		&\frac1{z-i}, \text{ если } z \in \CM \bs \{i, \infty\}\\
		&0, \text{ если } z  = \infty\\
		&\infty, \text{ если } z = i
	}\]

	Согласно определению выше, $f$ непрерывна в каждой точке из $\CM$.
\end{example}

\begin{note}
	Пусть функции $f, g$ непрерывны и конечны в точке $z_0 \in \Cm$. Аналогично вещественному случаю, можно доказать, что тогда непрерывны функции $f + g$, $fg$ и $\frac fg$ (последняя --- если $g(z_0) \ne 0$).
\end{note}

\begin{proposition}
	Пусть функция $f$ непрерывна в точке $z_0 \in \CM$, $w_0 := f(z_0) \in \CM$, функция $g$ непрерывна в точке $w_0$. Тогда функция $g \circ f$ непрерывна в точке $z_0$.
\end{proposition}

\begin{proof}
	Рассмотрим произвольную последовательность $\{z_n\} \subset B_{\delta_0}(z_0)$ такую, что $\lim_{n \to \infty}z_n = z_0$. Тогда, в силу непрерывности функции $f$, $\lim_{n \to \infty}f(z_n) = f(z_0) = w_0$. Но тогда, в силу непрерывности функции $g$, $\lim_{n \to \infty}g(f(z_n)) = g(w_0) = f(g(z_0))$. Получено требуемое в силу произвольности выбора $\{z_n\}$.
\end{proof}

\section{Основные элементарные функции}

\begin{definition}
	\textit{Косинусом} и \textit{синусом} комплексного аргумента называются следующие функции:
	\begin{align*}
		\cos{z} := \frac{e^{iz} + e^{-iz}}2,~z \in \Cm
		\\
		\sin{z} := \frac{e^{iz} - e^{-iz}}{2i},~z \in \Cm
	\end{align*}
\end{definition}

\begin{definition}
	\textit{Гиперболическим косинусом} и \textit{гиперболическим синусом} комплексного аргумента называются следующие функции:
	\begin{align*}
		\ch{z} := \frac{e^{z} + e^{-z}}2,~z \in \Cm
		\\
		\sh{z} := \frac{e^{z} - e^{-z}}{2},~z \in \Cm
	\end{align*}
\end{definition}

\begin{note}
	Комплексная экспонента, многочлены, а также тригонометрические и гиперболические синус и косинус считаются основными элементарными функциями. По уже доказанным свойствам, они непрерывны в каждой точке из $\Cm$.
\end{note}