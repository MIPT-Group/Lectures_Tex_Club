\section{Теоремы Вейерштрасса}

\begin{theorem}[Первая теорема Вейерштрасса]
	Пусть выполнены следующие условия:
	\begin{enumerate}
		\item Функции $f_0, f_1, \dotsc$ регулярны на области $G \subset \Cm$
		\item $f_n \convlr f$ на $G$
	\end{enumerate}
	
	Тогда $f$ регулярна на $G$.
\end{theorem}

\begin{proof}
	Уже было доказано, что $f$ непрерывна на $G$. Достаточно проверить, что $f$ регулярна в произвольном круге $K \subset G$. Пусть $\gamma$ "--- кусочно-гладкая замкнутая кривая такая, что $M(\gamma) \subset K$. Тогда, по интегральной теореме Коши, $\forall n \in \N \cup \{0\}: \int_\gamma f_ndz = 0$, и $f_n \convu_{M(\gamma)} f$, поскольку $M(\gamma)$ "--- компакт. Следовательно:
	\[\int_\gamma fdz = \lim_{n \to \infty}\int_\gamma f_ndz = 0\]
	
	Значит, по теореме Мореры, $f$ регулярна на $K$.
\end{proof}

\begin{theorem}[Вторая теорема Вейерштрасса]
	Пусть выполнены следующие условия:
	\begin{enumerate}
		\item Функции $f_0, f_1, \dotsc$ регулярны на области $G \subset \Cm$
		\item $f_n \convlr f$ на $G$
	\end{enumerate}
	
	Тогда $\forall m \in \N: f_n^{(m)} \convlr f^{(m)}$ на $G$.
\end{theorem}
 
\begin{proof}
	Зафиксируем точку $z_0 \in G$ и выберем $R > 0$ такое, что $\overline{B_R(z_0)} \subset G$. Поскольку $\overline{B_R(z_0)}$ "--- компакт, то $f_n \convu_{\overline{B_R(z_0)}} f$ на $G$. Достаточно для фиксированного числа $0 < R_1 < R$ доказать следующее:
	\[\forall m \in \N: f_n^{(m)} \convu_{\overline{B_{R_1}(z_0)}} f^{(m)}\]
	
	Воспользуемся интегральной формулой Коши для производной порядка $m$:
	\begin{gather*}
		f_n^{(m)}(z) = \frac{m!}{2\pi i}\int_{\{w \in \Cm: |w - z| = R\}}\frac{f_n(w)}{(w - z)^{m+1}}dw,~|z - z_0| < R\\
		f^{(m)}(z) = \frac{m!}{2\pi i}\int_{\{w \in \Cm: |w - z| = R\}}\frac{f(w)}{(w - z)^{m+1}}dw,~|z - z_0| < R
	\end{gather*}
	
	При $|z - z_0| = R_1$ выражение $|w - z|$ отделено от нуля числом $R - R_1$, тогда, по свойству оценки интегралов:
	\[|f_n^{(m)}(z) - f^{(m)}(z)| \le \frac{m!R}{(R - R_1)^m}\max\{|f_n(w) - f(w)| : w \in \Cm,~|w - z_0| = R\} \xrightarrow{n \to \infty} 0\]
	
	В силу произвольности выбора точки $z_0$, получено требуемое.
\end{proof}

\begin{note}
	Для функциональных рядов верны аналогичные теоремы. Пусть выполнены следующие условия:
	\begin{enumerate}
		\item Функции $f_0, f_1, \dotsc$ регулярны на области $G \subset \Cm$
		\item Ряд $\sum_{n=0}^\infty f_n$ сходится локально равномерно к сумме $S$ на $G$
	\end{enumerate}
	
	Тогда $S$ регулярна на $G$, и при любом $m \in \N$ ряд $\sum_{n=0}^\infty f_n^{(m)}$ сходятся локально равномерно к $S^{(m)}$ на $G$.
\end{note}

\section{Ряд Лорана}

\begin{definition}
	Пусть $z_0 \in \Cm$ и $\forall n \in \Z: c_n \in \Cm$. \textit{Рядом Лорана} называется выражение вида $\sum_{n = -\infty}^{+\infty}c_n(z - z_0)^n$. Ряд Лорана \textit{сходится} в точке $z \in \Cm$, если $z \ne z_0$ и сходятся оба следующих ряда:
	\begin{itemize}
		\item $\sum_{n = -\infty}^{-1}c_n(z - z_0)^n$ "--- \textit{главная часть} ряда Лорана
		\item $\sum_{n = 0}^{+\infty}c_n(z - z_0)^n$ "--- \textit{правильная часть} ряда Лорана
	\end{itemize}
	
	\textit{Суммой} ряда Лорана называется величина $\sum_{n = -\infty}^{-1}c_n(z - z_0)^n + \sum_{n = 0}^{+\infty}c_n(z - z_0)^n$.
\end{definition}

\begin{note}
	Ряду $\sum_{n = -\infty}^{-1}c_n(z - z_0)^n$ поставим в соответствие ряд $\sum_{m = 1}^{+\infty}c_{-m}w^m$, получаемый из данного заменой $w = \frac1{z - z_0}$. Пусть этот ряд имеет радиус сходимости $\rho > 0$, тогда он сходится регулярно на любом круге $\{w \in \Cm: |w| \le \rho_1\}$ при $\rho_1 < \rho$, поэтому ряд $\sum_{n = -\infty}^{-1}c_n(z - z_0)^n$ сходится локально равномерно на $\{z \in \Cm: |z - z_0| > \frac1\rho\}$. Тогда, по теоремам Вейерштрасса:
	\begin{enumerate}
		\item Сумма ряда $\sum_{n = -\infty}^{-1}c_n(z - z_0)^n$ регулярна на множестве $\{z \in \Cm: |z - z_0| > \frac1\rho\}$
		\item Ряд $\sum_{n = -\infty}^{-1}c_n(z - z_0)^n$ можно почленно дифференцировать любое число раз на множестве $\{z \in \Cm: |z - z_0| > \frac1\rho\}$
	\end{enumerate}
\end{note}

\begin{definition}
	Пусть ряды $\sum_{n = 0}^{+\infty}c_n(z - z_0)^n$ и $\sum_{m = 1}^{+\infty}c_{-m}w^m$ имеют радиусы сходимости $R > 0$ и $\rho > 0$. Кольцом сходимости ряда Лорана $\sum_{n = -\infty}^{+\infty}c_n(z - z_0)^n$ называется следующее множество:
	\[K := \left\{z \in \Cm: \frac1\rho < |z - z_0| < R\right\}\]
\end{definition}

\begin{proposition}
	Пусть кольцо сходимости $K$ ряда Лорана $\sum_{n = -\infty}^{+\infty}c_n(z - z_0)^n$ непусто. Тогда:
	\begin{enumerate}
		\item Ряд Лорана сходится локально равномерно и абсолютно на $K$
		\item Сумма ряда Лорана регулярна на $K$
		\item Ряд Лорана можно почленно дифференцировать любое число раз на $K$
	\end{enumerate}
\end{proposition}

\begin{proof}
	Ряд $\sum_{n = -\infty}^{-1}c_n(z - z_0)^n$ сходится локально равномерно и абсолютно при $|z - z_0| > \frac{1}\rho$, а ряд $\sum_{n = 0}^{+\infty}c_n(z - z_0)^n$ "--- при $|z - z_0| < R$, из чего следует требуемое.
\end{proof}

\begin{theorem}
	Пусть $z_0 \in \Cm$, $0 \le R_1 < R_2 \le +\infty$, и функция $f$ регулярна на кольце $K := \{z \in \Cm: R_1 < |z - z_0| < R_2\}$. Тогда существует единственное представление функции $f$ в виде ряда Лорана:
	\[f(z) = \sum_{n = -\infty}^{+\infty}c_n(z - z_0)^n,~z \in K\]
	
	Кроме того, если $K_\rho := \{z \in \Cm: |z - z_0| = \rho\}$ при произвольном $\rho \in (R_1, R_2)$, то для каждого $n \in \Z$ коэффициент $c_n$ в разложении удовлетворяет равенству:
	\[c_n = \frac1{2\pi i}\int_{K_{\rho}} \frac{f(w)}{(w - z_0)^{n+1}}dw\]
\end{theorem}

\begin{proof}
	Зафиксируем $z \in K$ и числа $\rho_1, \rho_2 \in \R$ такие, что выполнены неравенства $R_1 < \rho_1 < |z - z_0| < \rho_2 < R_2$. Тогда функция $f$ регулярна на замыкании $\overline{K_1}$ кольца $K_1 := \{w \in \Cm: \rho_1 < |w - z_0| < \rho_2\}$. Тогда, по интегральной формуле Коши:
	\[f(z) = \frac1{2\pi i} \int_{\partial K_1}\frac{f(w)}{w - z}dw = \frac1{2\pi i} \int_{K_{\rho_2}}\frac{f(w)}{w - z}dw - \frac1{2\pi i} \int_{K_{\rho_1}}\frac{f(w)}{w - z}dw\]
	
	Обе окружности в равенстве выше ориентированы против часовой стрелки:
	\begin{center}
		\begin{tikzpicture}
			\clip (-2.8, -2.12) rectangle (2.8, 2.12);
			
			\fill [opacity=0.05] (0,0) circle [radius=2];
			\draw[black, dashed] (0,0) circle[radius=2];
			\draw[
				black,
				decoration={markings, mark=at position 0.1 with {\arrow{>}}},
				decoration={markings, mark=at position 0.6 with {\arrow{>}}},
				postaction={decorate}
			] (0,0) circle[radius=2];
			\node[] at (-1, 1.) {$K_1$};
			
			\fill [white] (0,0) (0, 0) circle[radius=0.9];
			\draw[
			black,
			decoration={markings, mark=at position 0.1 with {\arrow{>}}},
			decoration={markings, mark=at position 0.6 with {\arrow{>}}},
			postaction={decorate}
			] (0,0) circle[radius=0.9];
			\node[] at (1.88, 1.42) {$K_{\rho_2}$};
			\node[] at (0.92, -0.9) {$K_{\rho_1}$};
			
			\node[draw, circle, inner sep=1pt, fill, black, label={right, shift={(-3pt,-3pt)}:\scalebox{0.85}{$z_0$}}] at (0, 0) {};
			\node[draw, circle, inner sep=1pt, fill, black, label={right, shift={(-3pt,-3pt)}:\scalebox{0.85}{$z_1$}}] at (-1, -1.15) {};
			
			\draw[->] (0, 0) -- (-1.92, -0.5) node[shift={(10pt, 10pt)}] {$\rho_2$};
			\draw[->] (0, 0) -- (-0.4, -0.81) node[shift={(12pt, 5pt)}] {$\rho_1$};
		\end{tikzpicture}
	\end{center}
	
	Представим первый интеграл в правой части степенным рядом как интеграл Коши:
	\[\frac1{2\pi i} \int_{K_{\rho_2}}\frac{f(w)}{w - z}dw = \sum_{n = 0}^{+\infty} c_n(z - z_0)^n\]
	
	В равенстве выше для любого $n \in \N \cup \{0\}$ коэффициент $c_n$ вычисляется по следующей формуле:
	\[c_n = \int_{K_{\rho_2}}\frac{f(w)}{(w - z_0)^{n+1}}dw\]
	
	Чтобы преобразовать второй интеграл, заметим следующее:
	\[-\frac1{w - z} = \frac1{(z-z_0) - (w - z_0)} = \frac1{z-z_0}\cdot \frac1{1 - \frac{w-z_0}{z - z_0}} = \sum_{m=0}^{+\infty}\frac{(w-z_0)^m}{(z-z_0)^{m + 1}}\]
		
	Ряд выше сходится равномерно на $K_{\rho_1}$, тогда, в силу ограниченности функции $f$ на $K_{\rho_1}$, его можно домножить на $f(w)$ и проинтегрировать почленно $K_{\rho_1}$:
	\[- \frac1{2\pi i} \int_{K_{\rho_1}}\frac{f(w)}{w - z}dw = \frac1{2\pi i}\sum_{m=0}^\infty\frac1{(z-z_0)^{m + 1}}\int_{K_{\rho_1}}f(w)(w-z_0)^mdw\]
	
	Заменяя индекс суммирования на $n := -m - 1$, получим:
	\[-\frac1{2\pi i} \int_{K_{\rho_1}}\frac{f(w)}{w - z}dw = \sum_{n = -\infty}^{-1}c_n(z - z_0)^n\]
	
	В равенстве выше для любого $n \in \Z \bs (\N \cup \{0\})$ коэффициент $c_n$ вычисляется по следующей формуле:
	\[c_n = \int_{K_{\rho_1}}\frac{f(w)}{(w - z_0)^{n+1}}dw\]
	
	Таким образом, значение функции в каждой точке $z \in K$ действительно представляется в виде ряда Лорана, однако коэффициенты ряда задаются как интегралы по окружностям, радиусы которых зависят от $z$. Значит, нужно проверить, что значения интегралов в формулах для коэффициентов не зависят от выбора радиуса окружности. Зафиксируем числа $\rho', \rho'' \in \R$ такие, что выполнены неравенства $R_1 < \rho' < \rho'' < R_2$. Зафиксируем $n \in \Z$ и рассмотрим следующую функцию:
	\[g_n(w) := \frac1{2\pi i} \frac{f(w)}{(w - z_0)^{n+1}}\]
	
	Функция $g_n$ регулярна на замыкании кольца $\{w \in \Cm: \rho' < |w - z_0| < \rho''\}$, поэтому применима интегральная теорема Коши:
	\[\int_{K_{\rho'}}g_ndw - \int_{K_{\rho''}}g_ndw = 0 \ra \int_{K_{\rho'}}g_ndw = \int_{K_{\rho''}}g_ndw\]
	
	В силу произвольности выбора чисел $\rho'$, $\rho''$ и $n$, представление функции рядом Лорана получено. Проверим единственность представления. Пусть в каждой точке $z \in K$ выполнены следующие равенства:
	\[f(z) = \sum_{n = -\infty}^{+\infty}c_n(z - z_0)^n = \sum_{n = -\infty}^{+\infty}d_n(z - z_0)^n\]
	
	Зафиксируем $m \in \Z$ и домножим обе части равенства на $(z - z_0)^m$. Поскольку оба ряда локально равномерно на $K$, то, интегрируя по окружности $R_{\rho}$ при $\rho \in (R_1, R_2)$, получим:
	\[{(2\pi i)}{c_{-m - 1}} = \sum_{n = -\infty}^{+\infty}\left(\int_{K_{R_\rho}}c_{n + m}(z - z_0)^n\right) = \sum_{n = -\infty}^{+\infty}\left(\int_{K_{R_\rho}}d_{n + m}(z - z_0)^n\right) = {(2\pi i)}{d_{-m - 1}}\]
	
	Таким образом, для всех $m \in \Z$ выполнено $c_m = d_m$, то есть получено требуемое.
\end{proof}

\begin{proposition}[неравенство Коши для коэффициентов ряда Лорана]
	Пусть $z_0 \in \Cm$, $0 \le R_1 < R_2 \le +\infty$, и на кольце $K = \{z \in \Cm: R_1 < |z - z_0| < R_2\}$ функция $f$ представима рядом Лорана. Тогда для любого $\rho \in (R_1, R_2)$ и любого $n \in \Z$ выполнено неравенство:
	\[|c_n| \le \frac{\max_{w \in K_{\rho}}\left|{f(w)}\right|}{\rho^n}\]
\end{proposition}

\begin{proof}
	Пользуясь свойством оценки интегралов, имеем:
	\[|c_n| = \frac1{2\pi}\left|\int_{K_{\rho}} \frac{f(w)}{(w - z_0)^{n+1}}dw\right| \le \rho\max_{w \in K_\rho}\left|\frac{f(w)}{(w - z_0)^{n+1}}\right| = \frac1{\rho^n}\max_{w \in K_{\rho}}\left|{f(w)}\right|\]
	
	Получено требуемое.
\end{proof}

\section{Изолированные особые точки регулярных функций}

\begin{definition}
	Точка $z_0 \in \CM$ называется \textit{изолированной особой точкой} функции $f$, если существует $\epsilon > 0$ такое, что $f$ регулярна на $\mathring B_\epsilon(z_0)$, но не регулярна или не определена в точке $z_0$.
\end{definition}

\begin{example}
	Функция $f(z) = \frac{\sin{z}}{z}$ регулярна на $\Cm \bs \{0\}$, поэтому точки $0$ и $\infty$ являются ее изолированными особыми точками.
\end{example}

\begin{definition}
	Изолированная особая точка $z_0 \in \CM$ функции $f$ называется:
	\begin{itemize}
		\item \textit{Устранимой}, если $\exists \lim_{z \to z_0} f(z) \in \Cm$
		\item \textit{Полюсом}, если $\exists \lim_{z \to z_0} f(z) = \infty$
		\item \textit{Существенно особой}, если предел $\lim_{z \to z_0} f(z)$ не существует
	\end{itemize}
\end{definition}

\begin{proposition}
	Пусть $z_0 \in \Cm$ "--- изолированная особая точка функции $f$, и $f$ регулярна в кольце $\mathring B_\epsilon(z_0)$. Тогда $z_0$ "--- устранимая особая точка $\lra$ главная часть ряда Лорана функции $f$ в $\mathring B_\epsilon(z_0)$ тождественно равна нулю.
\end{proposition}

\begin{proof}~
	\begin{itemize}
		\item[$\la$]Поскольку главная часть ряда Лорана равна нулю, в любой точке $z \in \mathring B_\epsilon(z_0)$ функцию $f$ можно \pagebreak представить в следующем виде:
		\[f(z) = \sum_{n = 0}^{+\infty}c_n(z - z_0)^n\]
		
		По теореме Абеля, ряд в правой части сходится на $B_\epsilon(z_0)$, причем его сумма является регулярной функцией. Но она совпадает с $f$ на $\mathring B_\epsilon(z_0)$, поэтому $\exists \lim_{z \to z_0}f(z) = c_0$.
		
		\item[$\ra$]Из условия следует, что существуют числа $M > 0$ и $0 < \delta \le \epsilon$ такие, что $|f| \le M$ на $\mathring B_\delta(z_0)$. Воспользуемся неравенством Коши для коэффициентов ряда Лорана при произвольном $\rho \in (0, \delta)$ и $n \in \Z \bs (\N \cup \{0\})$:
		\[|c_n| \le \frac{M}{\rho^n} = M\rho^{|n|} \xrightarrow{\rho \to 0} 0\]
		
		Значит, главная часть ряда Лорана тождественно равна нулю.\qedhere
	\end{itemize}
\end{proof}

\begin{note}
	В ходе доказательства выше было получено еще одно утверждение: если функция $f$ регулярна и ограниченна на $\mathring B_\epsilon(z_0)$, то ее ряд Лорана на этом кольце не имеет главной части.
\end{note}

\begin{proposition}
	Пусть $z_0 \in \Cm$ "--- изолированная особая точка функции $f$, и $f$ регулярна в кольце $\mathring B_\epsilon(z_0)$. Тогда $z_0$ "--- полюс $\lra$ главная часть ряда Лорана функции $f$ в $\mathring B_\epsilon(z_0)$ содержит конечное ненулевое число слагаемых.
\end{proposition}

\begin{proof}~
	\begin{itemize}
		\item[$\ra$]Пусть $z_0$ "--- полюс, тогда сущестувет $0 < \delta \le \epsilon$ такое, что $f \ne 0$ на $\mathring B_\delta(z_0)$. Определим на $\mathring B_\delta(z_0)$ функцию $g$ как $g(z) := \frac1{f(z)}$, тогда $\lim_{z \to z_0}g(z) = 0$. Значит, $z_0$ "--- устранимая особая точка функции $g$, поэтому ряд Лорана функции $g$ имеет следующий вид:
		\[g(z) = \sum_{m = 0}^{+\infty}c_m(z - z_0)^m\]
		
		Поскольку $g \ne 0$ на $\mathring B_\delta(z_0)$, то существует $m \in \N \cup \{0\}$ такое, что $c_m \ne 0$. Ряд Лорана функции $\frac{g}{(z - z_0)^m}$ сходится на $B_\delta(z_0)$ по теореме Абеля, и его сумма $h$ регулярна на $B_\delta(z_0)$, причем $h(z_0) = c_m \ne 0$. Значит, $h \ne 0$ на $B_\delta(z_0)$. Тогда на $\mathring B_\delta(z_0)$ выполнены равенства:
		\[f(z) = \frac{1}{g(z)} = \frac1{(z - z_0)^m}\frac1{h(z)}\]
		
		Но функция $\frac1{h(z)}$ регулярна на $B_\delta(z_0)$, поэтому она представима степенным рядом на $B_\delta(z_0)$, то есть ее ряд Лорана не имеет главной части:
		\[\frac{1}{h(z)} = \sum_{n=0}^{+\infty}a_0(z - z_0)^n\]
		
		Наконец, $\frac1{h(z_0)}$, поэтому $a_0 \ne 0$. Получено требуемое.
		\item To be continued.\qedhere
	\end{itemize}
\end{proof}