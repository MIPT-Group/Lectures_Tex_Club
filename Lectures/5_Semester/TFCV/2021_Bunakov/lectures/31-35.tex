\section{Конформные отображения}

\begin{note}
	Пусть функция $f$ регулярна в точке $z_0 \in \Cm$, $f(z_0) = w_0$ и $f'(z_0) \ne 0$. Тогда, по теореме об обратной функции, $f$ взаимно однозначно отображает некоторую область $G \subset B_\delta(z_0)$ на круг $B_\epsilon(w_0)$.
	\begin{center}
		\begin{tikzpicture}
			\clip (-5, -2) rectangle (5, 2);
			
			\fill [opacity=0.05] (-2,0) circle [radius=1.6];
			\draw[black, dashed] (-2,0) circle [radius=1.6];
			
			\fill [opacity=0.05] (-1.8, 0) circle [radius=1];
			\draw[black, dashed] (-1.8, 0) circle [radius=1];
			
			\fill [opacity=0.05] (2, 0) circle [radius=1.2];
			\draw[black, dashed] (2, 0) circle [radius=1.2];
			
			\draw[->] (-1.5, 0.5) arc (120:60:3);
			\draw[<-] (-1.5, -0.5) arc (-120:-60:3);
			
			\node[] at (0, 1.2) {$f$};
			\node[] at (0.2, -1.2) {$f^{-1}$};
			\node[] at (3, 1.5) {$B_\epsilon(w_0)$};
			\node[] at (-3.7, -1.5) {$B_\delta(z_0)$};
			\node[] at (-2.4, 0.3) {$G$};
			\node[] at (-2.27, -0.6) {$\gamma$};
			\node[] at (2, 0.7) {$\Gamma$};
			
			\path
			coordinate (p1) at (-2, 0)
			coordinate (p2) at (-2.1, -0.3)
			coordinate (p3) at (-1.8, -0.7)
			coordinate (q1) at (2, 0)
			coordinate (q2) at (2.3, 0.6)
			coordinate (q3) at (2.8, 0.4);
			
			\draw[
			black,
			decoration={markings, mark=at position 0.5 with {\arrow{>}}},
			postaction={decorate}
			] plot [smooth, tension=0.85] coordinates {(p1) (p2) (p3)};
			
			\draw[
			black,
			decoration={markings, mark=at position 0.6 with {\arrow{>}}},
			postaction={decorate}
			] plot [smooth, tension=0.85] coordinates {(q1) (q2) (q3)};
			
			\node[draw, circle, inner sep=1pt, fill, black, label={right:$z_0$}] at (-2, 0) {};
			\node[draw, circle, inner sep=1pt, fill, black, label={below right:$w_0$}] at (2, 0) {};
			\node[draw, circle, inner sep=1pt, fill, black] at (p3) {};
			\node[draw, circle, inner sep=1pt, fill, black] at (q3) {};
		\end{tikzpicture}
	\end{center}
	
	В частности, любую гладкую простую кривую $\gamma$ на $G$ функция $f$ переводит в простую гладкую кривую $\Gamma$ на $B_\epsilon(z_0)$.
\end{note}

\begin{definition}
	Пусть функция $f$ регулярна в точке $z_0 \in \Cm$, $\gamma$ "--- гладкая кривая с параметризацией $z(t)$, $t \in [t_0, t_1]$, $z(t_0) = z_0$, и $\Gamma = f(\gamma)$ "--- гладкая кривая с параметризацией $w(t) = f(z(t))$, $w(t_0) = f(z_0) = w_0$. Линейным растяжением кривой $\gamma$ в точке $z_0$ называется следующий предел, если он существует:
	\[K := \lim_{t \to t_0 + 0}\frac{|w(t) - w_0|}{|z(t) - z_0|}\]
\end{definition}

\begin{proposition}
	Если $f$ регулярна в точке $z_0 \in \Cm$ и $f'(z_0) \ne 0$, то линейное растяжение любой гладкой кривой $\gamma$ в точке $z_0$ существует и равно $|f'(z_0)|$.
\end{proposition}

\begin{proof}
	В силу непрерывности модуля и регулярности функции $f$, имеем:
	\[K = \left|\lim_{t \to t_0 + 0}\frac{w(t) - w_0}{z(t) - z_0}\right| = \left|\lim_{t \to t_0 + 0}\frac{f(z(t)) - f(z_0)}{z(t) - z_0}\right| = |f'(z_0)|\]
	
	Получено требуемое.
\end{proof}

\begin{note}
	Поскольку $w - w_0 = f'(z_0)(z-z_0) + o(|z - z_0|)$, $z \to z_0$, то для любого $\rho > 0$ окружность $\{z \in \Cm : |z - z_0| = \rho\}$ под действием функции $f$ с точностью до $o(\rho)$ переходит в окружность $\{w \in \Cm : |w - w_0| = |f'(z_0)|\rho\}$.
\end{note}

\begin{definition}
	Пусть $\gamma_1, \gamma_2$ "--- кривые, пересекающиеся в точке $z_0 \in \Cm$. \textit{Углом между кривыми $\gamma_1, \gamma_2$ в точке $z_0$} называется ориентированный угол, образованный касательными к этим кривым в точке $z_0$.  Обозначение "--- $\angle(\gamma_1, \gamma_2)_{z_0}$.
\end{definition}

\begin{note}
	Пусть $\theta_0 := \arg{z'(t_0)}$, $\theta_1 := \arg{w'(t_0)}$, тогда поскольку $w'(t_0) = f'(z_0)z'(t_0)$, то $\theta_1 = \arg{f'(z_0)} + \theta_0$. Значит, отображение $f$ поворачивает все кривые в точке $z_0$ на один и тот же угол, сохраняя углы между ними.
\end{note}

\begin{definition}
	Пусть $G \subset \CM$ "--- область. Отображение $f : G \to \CM$ называется \textit{конформным в точке} $z_0 \in G$, если выполнено одно из следующих условий:
	\begin{enumerate}
		\item Функция $f$ регулярна в точке $z_0$ и $f'(z_0) \ne 0$
		\item $z_0$ является полюсом первого порядка функции $f$
		\item $z_0 = \infty$ является устранимой особой точкой функции $f$, и $\res_{z_0}f \ne 0$
	\end{enumerate}
\end{definition}

\begin{note}
	В каждом из случаев выше функция $f$ регулярна на $\mathring B_\epsilon(z_0)$.
\end{note}

\begin{example}
	Рассмотрим следующую функцию:
	\[f(z) := \System{
		& e^{\frac 1z}, & \text{если } &z \ne 0\text{ и } z \ne \infty
		\\
		& 1, & \text{если } &z = 0\text{ или } z = \infty
	}\]
	
	Исследуем отображение $f$ на конформность. На $\Cm \bs \{0\}$ функция $f$ регулярна, а ее производная отлична от нуля. Точка $z = 0$ является существенно особой, поэтому $f$ не конформна в нуле. Наконец, $z = \infty$ является устранимой особой точкой, и $\res_\infty f = -1 \ne 0$, поэтому $f$ конформна в бесконечно удаленной точке.
\end{example}

\begin{definition}
	Пусть $G \subset \CM$ "--- область. Отображение $f : G \to \CM$ называется \textit{однолистным} на $G$, если оно инъектвно на $G$.
\end{definition}

\begin{definition}
	Пусть $G \subset \CM$ "--- область. Отображение $f : G \to \CM$ называется \textit{конформным на области $G$}, если оно однолистно на $G$ и конформно в каждой точке из $G$.
\end{definition}

\begin{example}
	Функция $f(z) := e^z$ конформна в любой точке из $\Cm$, при этом она конформна на области $\{z \in \Cm : |z| < 3\}$, но не конформна на области $\{z \in \Cm : |z| < 4\}$, потому что на ней уже не выполнена инъективность.
\end{example}

\begin{example}
	Рассмотрим следующую функцию:
	\[
		f(z) := \System{
			& \frac{1}{z^2 - 2z}, & \text{если } &z \ne 0, z \ne 2 \text{ и } z \ne \infty
			\\
			& 0, & \text{если } &z = 0, z = 2\text{ или } z = \infty
		}
	\]
	
	Отображение $f$ не является конформным ни на одной области, содержащей одновременно точки $0$ и $2$, поскольку $f(0) = f(2) = 0$.
\end{example}

\begin{definition}
	Пусть $z = z(t)$, $t \in [t_0, +\infty)$ "--- непрерывная функция такая, что $\lim_{t \to +\infty} z(t) = \infty$. Говорят, что функция $z(t)$ задает \textit{бесконечную кривую}.
\end{definition}

\begin{definition}
	Пусть $\gamma_1, \gamma_2$ "--- бесконечные кривые, $\Gamma_1, \Gamma_2$ "--- образы этих кривых при отображении $\zeta(z) = \frac 1z$. \textit{Углом между кривыми $\gamma_1, \gamma_2$ в точке $\infty$} называется угол между кривыми $\Gamma_1, \Gamma_2$ в нуле, если он существует. Обозначение "--- $\angle(\gamma_1, \gamma_2)_\infty$.
\end{definition}

\begin{example}
	Пусть бесконечные кривые $\gamma_1, \gamma_2$ задаются параметризациями $z_1(t) = te^{i\frac\pi 6}$ и параметризациями $z_2(t) = te^{i\frac\pi 3}$, $t \in [1, +\infty)$. Тогда в точке $\infty$ угол между ними равен $-\frac\pi 6$.
	\begin{center}
		\begin{tikzpicture}
			\clip (-5.5, -2.2) rectangle (5.9, 2.6);
			
			\draw[->] (-5, 0) -- (-1, 0) node [right] {$\re z$};
			\draw[->] (-3, -2) -- (-3, 2) node[above] {$\im {z}$};
			
			\draw[black] (-3 + 1.2,0.1) -- (-3 + 1.2,-0.1) node [below right] {$1$};
			\draw[black] (-3 + 0.1,1.2) -- (-3 -0.1,1.2) node [above left] {$i$};
			\draw[black] (3 + 1.2,0.1) -- (3 + 1.2,-0.1) node [above right] {$1$};
			\draw[black] (3 + 0.1,1.2) -- (3 -0.1,1.2) node [above right] {$i$};
			
			\draw[->] (1, 0) -- (5, 0) node [right] {$\re \zeta$};
			\draw[->] (3, -2) -- (3, 2) node[above] {$\im {\zeta}$};
			
			\draw[black, dashed] (-3, 0) circle [radius=1.2];
			\draw[black, dashed] (3, 0) circle [radius=1.2];
			
			\draw[->] (-0.5, 0.5) arc (120:60:2);
			\node[] at (0.55, 1.15) {$\zeta$};
			\node[] at (-1, 1.5) {$\gamma_1$};
			\node[] at (-2, 2.2) {$\gamma_2$};
			\node[] at (4.35, -0.9) {$\Gamma_1$};
			\node[] at (3.6, -1.43) {$\Gamma_2$};
			
			\node[draw, circle, inner sep=1pt, fill, black] at (-3 + 0.866 * 1.2, 0 + 0.5 * 1.2) {};
			\node[draw, circle, inner sep=1pt, fill, black] at (-3 + 0.5 * 1.2, 0 + 0.866 * 1.2) {};
			
			\node[draw, circle, inner sep=1pt, fill, black] at (3 + 0.866 * 1.2, 0 - 0.5 * 1.2) {};
			\node[draw, circle, inner sep=1pt, fill, black] at (3 + 0.5 * 1.2, 0 - 0.866 * 1.2) {};
			
			\draw[black,
			decoration={markings, mark=at position 0.8 with {\arrow{>}}},
			postaction={decorate}
			] (-3 + 0.866 * 1.2, 0 + 0.5 * 1.2) -- (-3 + 0.866 * 2.5, 0 + 0.5 * 2.5);
			\draw[black,
			decoration={markings, mark=at position 0.8 with {\arrow{>}}},
			postaction={decorate}
			] (-3 + 0.5 * 1.2, 0 + 0.866 * 1.2) -- (-3 + 0.5 * 2.5, 0 + 0.866 * 2.5);
			
			\draw[black,
			decoration={markings, mark=at position 0.3 with {\arrow{>}}},
			postaction={decorate}
			] (3 + 0.866 * 1.2, 0 - 0.5 * 1.2) -- (3, 0);
			\draw[black,
			decoration={markings, mark=at position 0.3 with {\arrow{>}}},
			postaction={decorate}
			] (3 + 0.5 * 1.2, 0 - 0.866 * 1.2) -- (3, 0);
			
			\coordinate (a1) at (3 + 0.866 * 1.2, 0 - 0.5 * 1.2);
			\coordinate (a2) at (3, 0);
			\coordinate (a3) at (3 + 0.5 * 1.2, 0 - 0.866 * 1.2);
			\pic [draw, <-, angle radius = 0.6cm] {angle = a3--a2--a1};
		\end{tikzpicture}
	\end{center}
\end{example}

\begin{proposition}
	Пусть выполнены следующие условия:
	\begin{enumerate}
		\item $z_0 \in \Cm$ "--- полюс первого порядка функции $f$
		\item $\gamma_1, \gamma_2$ "--- гладкие кривые, проходящие через точку $z_0$
		\item $\Gamma_1 = f(\gamma_1)$, $\Gamma_2 = f(\gamma_2)$
	\end{enumerate}
	
	Тогда $\angle(\Gamma_1, \Gamma_2)_\infty = \angle(\gamma_1, \gamma_2)_{z_0}$.
\end{proposition}

\begin{proof}
	Пусть $\widetilde\gamma_1, \widetilde\gamma_2$ "--- кривые, в которые переходят бесконечные $\Gamma_1, \Gamma_2$ при отображении $\zeta(w) = \frac 1w$. Достаточно проверить, что $\angle(\widetilde\gamma_1, \widetilde\gamma_2)_0 = \angle(\gamma_1, \gamma_2)_{z_0}$. Поскольку точка $z_0$ "--- полюс первого порядка функции $f$, то на $B_\epsilon(z_0)$ выполнено тождество $f(z) \equiv \frac{g(z)}{z - z_0}$, где $g$ "--- регулярная на $B_\epsilon(z_0)$ функция такая, что $g(z_0) \ne 0$. Значит, функция $\zeta(f(z)) = \frac{z - z_0}{g(z)}$ регулярна на некотором круге $B_\delta(z_0)$, причем $(\zeta \circ f)'(z_0) \ne 0$. Такая функция сохраняет углы между кривыми в точке $z_0$, что и требовалось.
\end{proof}

\begin{note}
	Нетрудно проверить, что и во всех остальных случаях конформное в точке $z_0$ отображение сохраняет углы между кривыми, проходящими через эту точку.
\end{note}

\begin{proposition}
	Пусть функция $f$ регулярна и однолистна на области $G \subset \Cm$. Тогда функция $f$ конформна на $G$.
\end{proposition}

\begin{proof}
	Достаточно проверить, что $f' \ne 0$ на $G$. Пусть это не так, то есть существует точка $z_0 \in G$ такая, что $f'(z_0) \ne 0$. Если для всех $n \in \N$ выполнено $f^{(n)}(z_0) = 0$, то $f \equiv f(z_0)$ на $G$. Если же существтует $m \in \N$ такое, что $f^{(m)}(z_0) \ne 0$, то $f$ не однолистно по теореме о локальной структуре отображения.
\end{proof}

\begin{note}
	Если функция $f$ регулярна и однолистна на области $G \subset \Cm$, то:
	\begin{enumerate}
		\item Множество $D := f(G)$ является областью по принципу сохранения области
		\item Функция $f^{-1} : D \to G$ тоже является регулярной и однолистной по теореме об обратной функции
	\end{enumerate}
\end{note}

\pagebreak
\begin{note}
	Если функция $f$ регулярна и однолистна на области $G \subset \Cm$, а функция $g$ регулярна и однолистна на $D := f(G)$, то функция $g \circ f$ регулярна и однолистна на $G$.
\end{note}

\begin{note}
	Можно проверить, что замечания выше справедливы и для конформных отображений, не являющихся регулярными, то есть обратное к конформному отображение и композиция конформных отображений являются конформными.
\end{note}

\begin{example}
	Построим конформное отображение, переводящее область $D$, изображенную ниже, в область $\{z \in \Cm: \im z > 0\}$.
	\begin{center}
		\begin{tikzpicture}
			\clip (-2.8, -2.6) rectangle (2.8, 2.8);
			
			\fill [opacity=0.24, path fading=fade out] (0,0) circle [radius=2.5];
			\fill [white] (0,0) rectangle (2.8, -2.8);
			
			\draw[->] (-2.4, 0) -- (2.4, 0) node [above] {$\re z$};
			\draw[->] (0, -2.4) -- (0, 2.4) node[right] {$\im {z}$};
			
			\draw[thick, dashed] (0, 0) -- (2.4, 0);
			\draw[thick, dashed] (0, -2.4) -- (0, 0);
			
			\node[] at (-1.1, 1.1) {$D$};
		\end{tikzpicture}
	\end{center}
	
	Сначала переведем область $D$ в область $D_1$ следующего вида:
	\begin{center}
		\begin{tikzpicture}
			\clip (-2.8, -2.6) rectangle (2.8, 2.8);
			
			\begin{scope}
				\clip (0,0) rectangle (2.8, 2.8);
				\fill [opacity=0.24, path fading=fade out] (0,0) circle [radius=2.5];
			\end{scope}
			
			\draw[->] (-2.4, 0) -- (2.4, 0) node [above] {$\re z$};
			\draw[->] (0, -2.4) -- (0, 2.4) node[right] {$\im {z}$};
			
			\draw[thick, dashed] (0, 0) -- (2.4, 0);
			\draw[thick, dashed] (0, 0) -- (0, 2.4);
			
			\node[] at (1.1, 1.1) {$D_1$};
		\end{tikzpicture}
	\end{center}
	
	Для этого подойдет функция $z_1(z) = \sqrt[3]{z}$. Действительно, поскольку функция $z(z_1) = z_1^3$ конформна на $D_1$, то и обратная к ней функция конформна на прообразе $D$. Остается перевести область $D_1$ в $\{z \in \Cm: \im z > 0\}$, но для этого подойдет функция $z_2(z_1) = z_1^2$. Ответом будет композиция $z_2(z_1(z))$, также являющаяся конформным отображением.
\end{example}

\section{Теорема Римана и принцип соответствия границ}

\begin{theorem}[Римана, \textit{без доказательства}]
	Пусть выполнены следующие условия:
	\begin{enumerate}
		\item $D, G \subset \CM$ "--- односвязные области
		\item Оба множества $\partial D, \partial G$ содержат более одной точки
		\item $z_0 \in D \bs \{\infty\}$, $w_0 \in G \bs \{\infty\}$, $\alpha \in [0, 2\pi)$
	\end{enumerate}
	
	Тогда существует единственное конформное отображение $f : D \to G$ такое, что выполнены равенства $f(D) = G$, $f(z_0) = w_0$ и $\arg f'(z_0) = \alpha$.
\end{theorem}

\begin{example}
	Зададимся вопросом, существует ли конформное отображение плоскости $\Cm$ на $\{w \in \Cm: |w| < 1\}$. Теорема Римана не дает ответа на этот вопрос, поскольку $\partial \Cm = \{\infty\}$. Покажем, что такого отображения не существует. Пусть не так, и $f$ "--- искомое отображение. Тогда $f$ регулярна на $\Cm$, поскольку ни одна из областей не содержит точку $\infty$, и $f$ ограниченна на $\Cm$. Тогда, по теореме Лиувилля, $f$ постоянна на $\Cm$ --- противоречие.
\end{example}

\begin{theorem}[принцип соответствия границ]
	Пусть выполнены следующие условия:
	\begin{enumerate}
		\item $D, G \subset \Cm$ "--- ограниченные односвязные области с простыми кусочно-гладкими границами
		\item Функция $f$ регулярна на $\overline{D}$
		\item Функция $f$ взаимно однозначно отображает $\partial D$ на $\partial G$ с сохранением ориентации
	\end{enumerate}
	
	Тогда $f$ конформно отображает $D$ на $G$.
\end{theorem}

\begin{proof}
	Выберем $w_0 \in G$ и $w_1 \in \Cm \bs \overline G$.
		\begin{center}
		\begin{tikzpicture}
			\clip (-5, -2) rectangle (6, 2);
			
			\fill [opacity=0.05] (-2,0) circle [radius=1.6];
			\draw[
				decoration={markings, mark=at position 0.2 with {\arrow{>}}},
				decoration={markings, mark=at position 0.7 with {\arrow{>}}},
				postaction={decorate}
			] (-2,0) circle [radius=1.6];
			
			\fill [opacity=0.05] (2, 0) circle [radius=1.2];
			\draw[
				decoration={markings, mark=at position 0.3 with {\arrow{>}}},
				decoration={markings, mark=at position 0.8 with {\arrow{>}}},
				postaction={decorate}
			] (2, 0) circle [radius=1.2];
			
			\draw[->] (-0.3, 0.7) arc (120:60:1.1);
			
			\node[] at (0.25, 1.2) {$f$};
			\node[] at (-2.9, 0.6) {$D$};
			\node[] at (2.5, -0.5) {$G$};
			\node[] at (-0.6, -1.6) {$\partial D$};
			\node[] at (2.9, -1.4) {$\partial G$};
			
			
			\node[draw, circle, inner sep=1pt, fill, black, label={below:$w_0$}] at (1.5, 0) {};
			\node[draw, circle, inner sep=1pt, fill, black, label={below:$w_1$}] at (4, -0.6) {};
			
			\draw[->] (4, -0.6) -- (3.23, 0.12);
			\node[draw, circle, inner sep=1pt, fill, black, label={above right:$w$}] at (3.19, 0.15) {};
			
			\draw[->] (1.5, 0) -- (2.12, 1.16);
			\node[draw, circle, inner sep=1pt, fill, black, label={above:$w$}] at (2.13, 1.19) {};
		\end{tikzpicture}
	\end{center}
	
	Найдем число $N_1$ корней уравнения $f(z) - w_1$ в области $D$. Для этого воспользуемся принципом аргумента для функции $f(z) - w_1$ и тем фактом, что $f$ взаимно однозначно отображает $\partial D$ на $\partial G$ с сохранением ориентации:
	\[N_1 = \frac1{2\pi}\Delta_{\partial D}\arg(f - w_1) = \frac1{2\pi}\Delta_{\partial G}\arg(w - w_1) = 0\]
	
	Значит, уравнение не имеет корней, поэтому $f(D) \subset G$ в силу произвольности выбора точки $w_1$. Теперь найдем число $N_0$ корней уравнения $f(z) - w_0$ в области $D$. Для этого снова воспользуемся принципом аргумента, но для функции $f(z) - w_0$:
	\[N_0 = \frac1{2\pi}\Delta_{\partial D}\arg(f - w_0) = \frac1{2\pi}\Delta_{\partial G}\arg(w - w_0) = 1\]
	
	Значит, уравнение имеет ровно один корень для каждого $w_0 \in G$, то есть $f(D) = G$, причем отображение $f$ однолистно. Тогда, поскольку функция $f$ регулярна на $D$, она конформно отображает $D$ на $G$.
\end{proof}

\section{Дробно-линейные отображения}

\begin{definition}
	\textit{Линейным отображением} называется функция вида $w(z) = az + b$, где $a \in \Cm \bs \{0\}$, $b \in \Cm$, при этом считается, что $w(\infty) = \infty$.
\end{definition}

\begin{note}
	Легко видеть, что линейное отображение непрерывно на $\CM$ и обратимо, причем обратное отображение $z(w) = \frac{w - b}{a}$ тоже линейно. Кроме того, композиция линейных отображений также является линейным отображением, поэтому линейные отображения с операцией композиции образуют группу.
\end{note}

\begin{example}
	Группа линейных отображений не является абелевой. Рассмотрим отображения $w = z + 1$ и $\zeta = 2z$, тогда $w(\zeta(z)) = 2z + 1$, но $\zeta(w(z)) = 2z + 2$.
\end{example}

\begin{note}
	Каждое линейное отображение конформно на $\CM$. Действительно, в конечных точках из $\Cm$ оно регулярно, а точка $\infty$ является его полюсом первого порядка.
\end{note}

\begin{definition}
	\textit{Линейным отображением} называется функция вида $w(z) = \frac{az + b}{cz + d}$, где $a, b, c, d \in \Cm$, $ad - bc \ne 0$.
\end{definition}

\begin{note}
	В случае, когда в определении выше $c = 0$, получается уже линейное отображение, поэтому далее будем считать, что $c \ne 0$. Для такого дробно-линейного отображения можно также считать, что $w(\infty) = \frac ac$ и $w(-\frac dc) = \infty$.
\end{note}

\begin{proposition}
	Каждое дробно-линейное отображение обратимо, причем его обратное отображение тоже является дробно-линейным.
\end{proposition}

\begin{proof}
	Пусть $w(z) = \frac{az + b}{cz + d}$, тогда непосредственной подстановкой легко убедиться, что отображение $z(w) = \frac{dw - b}{-cw + a}$ является обратным к $w$.
\end{proof}

\begin{corollary}
	Дробно-линейные отображения с операцией композиции образуют группу.
\end{corollary}

\begin{proof}
	Композиция дробно-линейных отображений также является дробно-ли-нейным отображением, и каждое дробно-линейное отображение имеет обратное дробно-линейное отображение, что и означает требуемое.
\end{proof}

\begin{note}
	Назовем \textit{матрицей отображения} $w(z) = \frac{az + b}{cz + d}$ следующую матрицу:
	\[A_w := \begin{pmatrix} a&b\\c&d\end{pmatrix}\]
	
	Тогда при взятии композиции двух дробно-линейных отображения их матрицы перемножаются в соответствующем порядке.
\end{note}

\begin{proposition}
	Каждое дробно-линейное отображение конформно на $\CM$.
\end{proposition}

\begin{proof}
	Пусть $w(z) = \frac{az + b}{cz + d}$. Отображение $w$ однолистно на $\CM$ в силу обратимости, поэтому достаточно проверить, что оно конформно в каждой точке из $\CM$:
	\begin{itemize}
		\item Если $z \in \Cm \bs \{-\frac dc\}$, то $w$ регулярно в точке $z$, причем $w'(z) = \frac{ad - bc}{(cz + d)^2} \ne 0$
		
		\item Точка $z = -\frac dc$ является полюсом первого порядка отображения $w$
		
		\item Точка $z = \infty$ является устранимой особой точкой отображения $w$, причем выполнено условие $\res_\infty w = \frac{ad - bc}{c^2} \ne 0$\qedhere
	\end{itemize}
\end{proof}

\begin{proposition}
	Пусть $z_1, z_2, z_3 \in \CM$ и $w_1, w_2, w_3 \in \CM$  "--- тройки попарно различных точек. Тогда существует единственное дробно-линейное отображение $w$ такое, что $w(z_k) = w_k$ для каждого $k \in \{1, 2, 3\}$
\end{proposition}

\begin{proof}
	Сначала будем считать, что все точки из условия конечны. Рассмотрим отображения $\zeta(z) := \frac{z - z_1}{z - z_2}\cdot\frac{z_3 - z_2}{z_3 - z_1}$ и $\eta(z) := \frac{w - w_1}{w - w_2}\cdot\frac{w_3 - w_2}{w_3 - w_1}$. Тогда:
	\begin{center}
		\begin{tikzcd}[row sep = tiny, column sep = normal]
			z_1 \rar[maps to, "\zeta"] & 0 & \lar[maps to, swap, "\eta"]w_1
			\\
			z_2 \rar[maps to, "\zeta"] & \infty & \lar[maps to, swap, "\eta"]w_2
			\\
			z_3 \rar[maps to, "\zeta"] & 1 & \lar[maps to, swap, "\eta"]w_3
		\end{tikzcd}
	\end{center}
	
	Значит, отображение $w = \eta^{-1} \circ \zeta$ является искомым. Его удобнее задавать следующим равенством:
	\[\frac{z - z_1}{z - z_2}\cdot\frac{z_3 - z_2}{z_3 - z_1} = \frac{w - w_1}{w - w_2}\cdot\frac{w_3 - w_2}{w_3 - w_1}\]
	
	Если же среди точек из условия есть бесконечно удаленная точка, то формальное сокращение бесконечностей в равенстве выше дает новое равенство, задающее искомое отображение для этого случая. Опустим непосредственную проверку этого факта. \textit{Единственность лектор не доказал.}
\end{proof}

\begin{note}
	Образом прямой под действием линейного отображения всегда является прямая, а образом окружности под действием линейного отображения всегда является окружность. Далее мы получим аналогичное свойство для дробно-линейных отображений.
\end{note}

\begin{definition}
	\textit{Обобщенной окружностью} на $\CM$ называется множество, являющееся либо окружностью, либо объединением прямой и точки $\infty$.
\end{definition}

\begin{proposition}[круговое свойство]
	Образом обобщенной окружности под действием дробно-линейного отображения всегда является обобщенная окружность.
\end{proposition}

\begin{proof}
	Пусть $w(z) = \frac{az + b}{cz + d}$, тогда $w(z) = \frac ac - \frac{ad - bc}{c(cz + d)}$, то есть $w$ можно представить в виде композиции $w = z_3 \circ z_2 \circ z_1$, где $z_1(z) := cz + d$, $z_2(z_1) := \frac 1{z_1}$, $z_3(z_2) := \frac ac - \frac{ ad - bc}c z_2$. Значит, остается проверить, что отображение $z \mapsto \frac 1z$ переводит обобщенные окружности в обобщенные окружности. Известно, что любая прямая или окружность следующим уравнением с вещественными коэффициентами:
	\[A(x^2 + y^2) + Bx + Cy + D = 0,~B^2 + C^2 > 4AD\]
	
	Пусть $z = x + iy$, тогда $\frac 1z = \frac{x - iy}{x^2 + y^2} = \frac{x}{x^2 + y^2} - i\frac{x}{x^2 + y^2}$. Преобразование $(x, y) \mapsto \big(\frac{x}{x^2 + y^2}, -\frac{y}{x^2 + y^2}\big)$ дает следующее уравнение:
	\[D(x^2 + y^2) + Bx - Cy + A = 0,~B^2 + C^2 > 4AD\]
	
	Как видно, это уравнение тоже задает прямую или окружность. Наконец, образом обобщенной окружности является вся обобщенная окружность, поскольку отображение $z \mapsto \frac 1z$ является обратным к самому себе.
\end{proof}

\begin{definition}
	Пусть $\Gamma$ "--- обобщенная окружность. Точки $z_1, z_2 \in \CM$ называются симметричными относительно $\Gamma$, если выполнено одно из условий:
	\begin{itemize}
		\item $\Gamma$ "--- прямая, и точки $z_1, z_2$ симметричны относительно $\Gamma$ как прямой
		\item $\Gamma$ "--- окружность с центром $z_0 \in \Cm$ и радиусом $R > 0$, и либо $z_1 = z_0$ и $z_1 = \infty$, либо точки $z_1, z_2$ лежат на одном луче с началом в точке $z_0$ и $|z_0 - z_1||z_0 - z_2| = R^2$
	\end{itemize}
\end{definition}

\begin{proposition}
	Пусть $\Gamma$ "--- обобщенная окружность. Тогда $z_1, z_2 \in \CM$ симметричны относительно $\Gamma$ $\lra$ любая обобщенная окружность $\gamma$, проходящая через $z_1$ и $z_2$, пересекает $\Gamma$ под прямым углом.
\end{proposition}

\begin{proof}
	Случай, когда $\Gamma$ "--- прямая, тривиален, поэтому далее будем считать, что $\Gamma = \{z \in \Cm: |z - z_0| = R\}$.
	\begin{itemize}
		\item[$\ra$] Пусть $\gamma$ "--- обобщенная окружность, проходящая через $z_1$ и $z_2$. Если $z_1 = z_0$, $z_2 = \infty$, то $\gamma$ является прямой, проходящей через центр окружности $\Gamma$, и получено требуемое. Иначе --- либо снова имеем тривиальный случай, когда $\gamma$ "--- прямая, либо ситуацию, изображенную ниже.
		\begin{center}
			\scalebox{1}{
				\begin{tikzpicture}
					\clip (-4, -2) rectangle (4, 3.72);
					
					\draw[black] (-1, 0) circle [radius = 1.5]; 
					
					\node [black, below right, scale = 0.9] at (-1 - 0.9, 0.5) {$R$};
					\draw[->] (-1, 0) -- (-1 - 1.05, 1.05);
					\node[draw, circle, inner sep=1.1pt, fill, black, label={below:$z_0$}] at (-1, 0) {};
					
					\draw[] (-1, 0) -- (3, 0);
					\node[draw, circle, inner sep=1.1pt, fill, black, label={below:$z_1$}] at (-1 + 0.75, 0) {};
					\node[draw, circle, inner sep=1.1pt, fill, black, label={below:$z_2$}] at (-1 + 3, 0) {};
					\node[draw, circle, inner sep=1.1pt, fill, black, label={[xshift=-6pt, yshift=1pt]:$z^*$}] at (-1.11, 1.5) {};
					\draw[] (-1, 0) -- (-1.11, 1.5);
					
					\draw[black] (-1 + 1.875, 1.64) circle [radius = 2]; 
					
					\node[] at (3, 1) {$\gamma$};
					\node[] at (-2.4, -1.4) {$\Gamma$};
			\end{tikzpicture}}
		\end{center}
		
		Пусть $z^*$ "--- одна из точек пересечения $\gamma$ и $\Gamma$, тогда $|z^* - z_0|^2 = R^2 = |z_1 - z_0||z_2 - z_0|$. Но это и означает, что отрезок $[z_0, z^*]$ касается окружности $\gamma$.
		
		\item[$\la$] Пусть $\gamma$ "--- прямая, проходящая через $z_1$ и $z_2$. Тогда, по условию, $\gamma$ пересекает $\Gamma$ под прямым углом и потому проходит через $z_0$. Значит, точки $z_0$, $z_1$ и $z_2$ лежат на одной прямой. При этом легко видеть, что если $z_1$ и $z_2$ не лежат на одном луче с началом в $z_0$, то окружность с диаметром $[z_1, z_2]$ не может пересекать $\Gamma$ под прямым углом. Значит, эти точки лежат на одном луче. Тогда можно провести рассуждение, симметричное прошлому пункту, пользуясь той же картинкой.\qedhere
	\end{itemize}
\end{proof}