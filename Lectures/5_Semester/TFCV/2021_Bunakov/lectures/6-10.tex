\section{Комплекснозначные функции вещественного переменного}

\begin{note}
	Зафиксируем функцию $f : \R \to \Cm$ и представим ее в виде $f = g + ih$, $g, h : \R \to \R$. Тогда для любого $t \in \R$ выполнено $f(t) = (g(t), h(t))$. Из этого равенства следуют такие свойства, ранее полученные для вектор-функций:
	\begin{enumerate}
		\item Функция $f$ непрерывна в точке $t_0 \in \R$ $\lra$ функции $g, h$ непрерывны в $t_0$
		\item Функция $f$ дифференцируема в точке $t_0 \in \R$ $\lra$ функции $g, h$ дифференцируемы в $t_0$, причем $f'(t_0) = g'(t_0) + ih'(t_0)$
		\item Функция $f$ интегрируема на отрезке $[a, b] \subset \R$ $\lra$ функции $g, h$ интегрируемы на $[a, b]$, причем $\int_a^bf(t)dt = \int_a^bg(t)dt + i\int_a^bh(t)dt$
		\item Функция $F = G + iH$ является первообразной функции $f$ на отрезке $[a, b] \subset \R \hm\lra$~функции $G, H$ являются первообразными функций $g, h$ на $[a, b]$
	\end{enumerate}
\end{note}

\begin{example}
	Вычислим производную функции $f(t) = e^{i\omega t} = \cos(\omega t) + i\sin(\omega t)$, $\omega \in \R$:
	\[f'(t) = \left(e^{i\omega t}\right)' = \left(\cos(\omega t)\right)' + i \left(\sin(\omega t)\right)' = i\omega(\cos(\omega t) + i\sin(\omega t)) = i\omega e^{i \omega t}\]
	
	Аналогичным образом можно убедиться, что для функции $g(t) = e^{at}$ при любом $a \in \Cm$ выполнено $g'(t) = ae^{at}$. Следовательно, если $a \ne 0$, то $\int e^{at}dt = \frac{e^{at}}{a} + C$, где $C \in \Cm$.
\end{example}

\section{Кривые и множества на комплексной плоскости}

\begin{definition}
	Пусть $\epsilon > 0$, $z_0 \in \Cm$. Множество $B_\epsilon(z_0) := \{z \in \Cm: |z - z_0| < \epsilon\}$ называется \textit{$\epsilon$-окрестностью} точки $z_0$.
\end{definition}


\begin{definition}
	\textit{Расширенной комплексной плоскостью} называется множество вида $\overline{\Cm} := \Cm \cup \{\infty\}$, где $\infty$ "--- объект, называемый \textit{бесконечно удаленной точкой}. \pagebreak Считается также, что $B_\epsilon(\infty) := \{z \in \Cm: |z| > \epsilon\} \cup \{\infty\}$ для любого $\epsilon > 0$
\end{definition}

\begin{definition}
	Пусть $\epsilon > 0$, $z_0 \in \overline{\Cm}$. Множество $\mathring B_\epsilon(z_0) := B_\epsilon(z_0) \backslash \{z_0\}$ называется \textit{проколотой $\epsilon$-окрестностью} точки $z_0$.
\end{definition}

\begin{definition}
	\textit{Кривой} $\gamma$ на комплексной плоскости называется непрерывное отображение отрезка $[a, b] \subset \R$, $a < b$, в $\Cm$. \textit{Параметризацией} кривой $\gamma$ называется соответствующая 	функция $\sigma(t) = \xi(t) + i\eta(t)$, $t \in [a, b]$. \textit{Носителем} кривой $\gamma$ называется множество $M(\gamma) := \sigma([a, b])$. 
\end{definition}

\begin{note}
	Рассмотрим кривые $\gamma_1$ и $\gamma_2$ с параметризациями $z = \sigma_1(t)$, $t \in [a_1, b_1]$ и $z \hm= \sigma_2(t)$, $t \in [a_2, b_2]$. Пусть существует $f : [a_1, b_1] \to [a_2, b_2]$ "--- непрерывная строго монотонная функция такая, что $\sigma_2(f(t)) \equiv \sigma_1(t)$ на $[a_1, b_1]$. Тогда естественно считать, что кривые $\gamma_1, \gamma_2$ совпадают, то есть $\gamma_1 = \gamma_2$.
\end{note}

\begin{definition}
	Пусть $\gamma$ "--- кривая с параметризацией $z = \sigma(t)$, $t \in [a, b]$. Кривая называется:
	\begin{itemize}
		\item \textit{Простой}, если параметризация $\sigma$ инъективна
		\item \textit{Замкнутой}, если $\sigma(a) = \sigma(b)$
		\item \textit{Простой замкнутой}, или \textit{контуром}, если она замкнута и сужение $\sigma|_{[a, b)}$ инъективно
	\end{itemize}
\end{definition}

\begin{example}
	Зафиксируем кривую $\gamma$ и рассмотрим кривую $\gamma^{-1}$ с параметризацией $z = \sigma(-t)$, $t \in [-b, -a]$. Тогда <<конкатенация>> этих кривых $\gamma\gamma^{-1}$ "--- замкнутая, но не простая.
\end{example}

\begin{definition}
	Пусть $\gamma$ "--- кривая с параметризацией $z = \sigma(t)$, $t \in [a, b]$, и зафиксировано $\{t_k\}_{k = 0}^n$ "--- \textit{разбиение отрезка} $[a, b]$, то есть $a = t_0 < t_1 < \dotsb < t_n = b$. Кривая, заданная сужением $\sigma|_{[t_{k-1}, t_k]}$, $k \in \{1, \dotsc, n\}$, называется \textit{дугой} кривой $\gamma$ и обозначается через $\gamma_k$. Говорят, что $\gamma$ \textit{разбита} на дуги $\gamma_1, \dotsc, \gamma_n$, или \textit{составлена} из этих дуг.
\end{definition}

\begin{definition}
	Кривая $\gamma$ называется \textit{гладкой}, если она имеет параметризацию вида $z = \sigma(t) = \xi(t) + i\eta(t)$, $t \in [a, b]$, где $\xi, \eta \in C^{1}[a, b]$ и для всех $t \in [a, b]$ выполнено $\sigma'(t) \ne 0$, \textit{кусочно-гладкой}, если ее можно разбить на конечное число гладких дуг.
\end{definition}

\begin{note}
	В предыдущих курсах было доказано, что если $\gamma$ "--- кусочно-гладкая кривая, то она \textit{спрямляема}, то есть имеет конечную длину $|\gamma|$, которую можно вычислить по следующей формуле:
	\[|\gamma| = \int_a^b|\sigma'(t)|dt\]
\end{note}

\begin{definition}
	Множество $E \subset \overline\Cm$ называется \textit{открытым}, если каждая точка $z \in E$ содержится в нем вместе с некоторой окрестностью $B_\epsilon(z)$.
\end{definition}

\begin{definition}
	Пусть $z_0 \in \overline\Cm$, $E \subset \overline{\Cm}$. Точка $z_0$ называется \textit{предельной точкой} множества $E$, если любая ее проколотая окрестность $\mathring B_\epsilon(z)$ содержит точку из $E$, и \textit{граничной точкой} множества $E$, если любая ее окрестность $B_\epsilon(z)$ содержит и точку из $E$, и точку не из $E$. \textit{Границей} множества $E$ называется множество $\partial E$ всех ее граничных точек.
\end{definition}

\begin{definition}
	Множество $E \subset \Cm$ называется \textit{линейно связным}, если для любых точек $z_1, z_2 \in E$ существует кривая $\gamma$ с параметризацией $z = \sigma(t)$, $t \in [a, b]$, такая, что $\sigma(a) = z_1$, $\sigma(b) = z_2$ и $M(\gamma) \subset E$.
\end{definition}

\begin{definition}
	Множество $G \subset \CM$ называется \textit{областью}, если выполнены следующие условия:
	\begin{enumerate}
		\item Множество $G$ открыто
		\item Множество $G \bs \{\infty\}$ линейно связно
	\end{enumerate}
\end{definition}

\begin{note}
	Из предыдущих курсов известно, что, например, множества $B_\epsilon(z)$ и $\mathring B_\epsilon(z)$ являются областями для любых $z \in \overline\Cm$ и $\epsilon > 0$.
\end{note}

\begin{definition}
	Множество $E \subset \Cm$ называется \textit{ограниченным}, если существует $C > 0$ такое, что для любого $z \in E$ выполнено $|z| < C$.
\end{definition}

\begin{note}
	Пусть $D \subset \Cm$ "--- ограниченная область. Если для некоторого кусочно-гладкого контура $\gamma$ выполнено равенство $M(\gamma) = \partial D$, то естественно считать этот контур границей области $D$.
\end{note}

\begin{definition}
	Область $G \subset \overline\Cm$ называется \textit{односвязной}, если для любой ограниченной области $D \subset \Cm$, границей которой является кусочно-гладкий контур $\gamma$, такой, что $M(\gamma) \subset G$, выполнено либо $D \subset G$, либо $(\overline\Cm \bs D) \subset G$.
\end{definition}

\section{Последовательности и ряды комплексных чисел}

\begin{definition}
	\textit{Последовательностью} в $\overline\Cm$ называется отображение $\N \to \overline\Cm$. Обозначение "--- $\{z_n\}$. \textit{Элементом} последовательности называется пара $(n, z_n)$ такая, что $n \mapsto z_n$, \textit{значением элемента} называется число $z_n \in \overline\Cm$.
\end{definition}

\begin{definition}
	Число $a \in \overline\Cm$ называется \textit{пределом последовательности} $\{z_n\}$, если выполнено условие $\forall \epsilon > 0 : \exists N \in \N: \forall n \ge N: z_n \in B_{\epsilon}(a)$. Обозначение "--- $\lim_{n \to \infty}z_n = a$.
\end{definition}

\begin{note}
	Легко видеть, что если $\lim_{n \to \infty}z_n = a \in \Cm$, то, начиная с некоторого номера, все значения элементов $z_n$ конечны, тогда $\lim_{n \to \infty}|z_n - a| = 0$. Кроме того, положим $|\infty| := \infty$ и заметим, что выполнены следующие эквивалентности:
	\begin{itemize}
		\item $\lim_{n \to \infty}z_n = \infty \lra \lim_{n \to \infty}|z_n| = \infty$
		\item $\lim_{n \to \infty}z_n = 0 \lra \lim_{n \to \infty}|z_n| = 0$
	\end{itemize}
\end{note}

\begin{note}
	Пусть $\{z_n\}, \{w_n\} \subset \Cm$, $\lim_{n \to \infty}z_n = a \in \Cm$, $\lim_{n \to \infty}w_n = b \in \Cm$. Аналогично вещественному случаю, можно доказать следующие факты:
	\begin{itemize}
		\item $\lim_{n \to \infty}|z_n| = |a|$
		\item $\lim_{n \to \infty}(z_n+w_n) = a+b$
		\item $\lim_{n \to \infty}z_nw_n = ab$
		\item Если $\forall n \in \N: w_n \ne 0$ и $b \ne 0$, то $\lim_{n \to \infty}\frac{z_n}{w_n} = \frac ab$
	\end{itemize}
\end{note}

\begin{proposition}
	Пусть $\{z_n\} = \{x_n + iy_n\} \subset \Cm$, $a = b+ic \in \Cm$. Тогда:
	\[\lim_{n \to \infty}z_n = a \lra \System{
		\lim_{n \to \infty}x_n = b\\
		\lim_{n \to \infty}y_n = c
	}\]
\end{proposition}

\begin{proof}
	Достаточно заметить, что выполнены следующие неравенства:
	\[\max\{|x_n - b|, |y_n - c|\} \le |z_n - a| \le |x_n - b| + |y_n - c|\qedhere\]
\end{proof}

\begin{definition}
	\textit{Числовым рядом} в $\Cm$ называется выражение вида $\sum_{n = 0}^\infty z_n$, где $z_n \in \Cm$ при всех $n \in \N$. \textit{Частной суммой} числового ряда называется число $S_n := \sum_{k = 0}^nz_k$, $n \in \N$. Если $\lim_{n \to \infty}S_n = S \in \Cm$, то ряд $\sum_{n = 0}^\infty z_n$ называется \textit{сходящимся}, а число $S$ называется его \textit{суммой}.
\end{definition}

\begin{definition}
	Числовой ряд $\sum_{n = 0}^\infty z_n$ называется \textit{сходящимся абсолютно}, если сходится ряд $\sum_{n = 0}^\infty|z_n|$.
\end{definition}

\begin{note}
	Для комплексных числовых рядов выполнен аналог \textit{критерия Коши} сходимости вещественных числовых рядов. В частности, из него легко вывести, что любой абсолютно сходящийся ряд сходится.
\end{note}

\begin{example}
	Ряд $\sum_{n=0}^\infty z^n$ сходится при любом $z \in \Cm$, $|z| < 1$, поскольку при таком $z$ сходится ряд $\sum_{n=0}^\infty |z|^n = \frac1{1-|z|}$. Вычислим его сумму:
	\[S_n = 1 + z + \dotsb + z^n = \frac{1-z^{n+1}}{1-z} \xrightarrow{n \to \infty} \frac1{1-z} \ra \sum_{n=0}^\infty z^n = \frac1{1-z}\]
\end{example}

\section{Предел и непрерывность функции в точке}

\begin{definition}
	Пусть функция $f$ определена в некоторой окрестности $\mathring B_{\delta_0}(z_0)$ точки $z_0 \in \overline\Cm$. Число $A \in \CM$ называется \textit{пределом функции $f$ в точке $z_0$}, если выполнено одно из следующих условий:
	\begin{enumerate}
		\item $\forall \epsilon > 0: \exists 0 < \delta \le \delta_0: f(\mathring B_\delta(z_0)) \subset B_\epsilon(A)$
		\item Для любой последовательности $\{z_n\} \subset \mathring B_{\delta_0}(z_0)$ такой, что $\lim_{n \to \infty}z_n = z_0$, выполнено $\lim_{n \to \infty}f(z_n) = A$
	\end{enumerate}
	
	Обозначение "--- $\lim_{z \to z_0}f(z) = A$.
\end{definition}

\begin{note}
	Аналогично вещественному случаю, можно доказать эквивалентность условий в определении предела функции в точке. Из второго условия и свойств предела последовательности легко вывести, \pagebreak что если $a = b + ic, z_0 = x_0 + iy_0 \in \Cm$, $f(z) = f(x + iy) \hm= u(x, y) + iv(x, y)$, то:
	\[\lim_{z \to z_0}f(z) = a \lra \System{
		\lim_{(x, y) \to (x_0, y_0)}u(x, y) = b\\
		\lim_{(x, y) \to (x_0, y_0)}v(x, y) = c
	}\]
\end{note}

\begin{note}
	Пусть $z_0 \in \CM$, функции $f, g$ таковы, что $\lim_{z \to z_0}f(z) = a \in \Cm$, $\lim_{z \to z_0}g(z) \hm= b \in \Cm$. Аналогично вещественному случаю, можно доказать следующие факты:
	\begin{itemize}
		\item $\lim_{z \to z_0}(f(z) + g(z)) = a+b$
		\item $\lim_{z \to z_0}f(z)g(z) = ab$
		\item Если $\exists \delta > 0: \forall z \in \mathring B_\delta(z_0) : g(z) \ne 0$ и $b \ne 0$, то $\lim_{z \to z_0}\frac{f(z)}{g(z)} = \frac ab$
	\end{itemize}
\end{note}

\begin{definition}
	Пусть функция $f$ определена в некоторой окрестности $B_{\delta_0}(z_0)$ точки $z_0 \in \CM$. Функция $f$ называется \textit{непрерывной в точке $z_0$}, если $\lim_{z \to z_0}f(z) = f(z_0)$.
\end{definition}

\begin{note}
	Легко заметить, что если $f(z) = f(x + iy) \hm= u(x, y) + iv(x, y)$, то $f$ непрерывна в точке $z_0 = x_0 + iy_0 \in \Cm$ $\lra$ $u, v$ непрерывны в точке $(x_0, y_0)$, поскольку аналогичная эквивалентность выполнена для предела функции в точке.
\end{note}

\begin{example}
	Рассмотрим функцию $f : \overline{\Cm} \to \CM$ следующего вида:
	\[f(z) = \System{
		&\frac1{z-i}, \text{ если } z \in \CM \bs \{i, \infty\}\\
		&0, \text{ если } z  = \infty\\
		&\infty, \text{ если } z = i
	}\]
	
	Согласно определению выше, $f$ непрерывна в каждой точке из $\CM$.
\end{example}

\begin{note}
	Пусть функции $f, g$ непрерывны и конечны в точке $z_0 \in \Cm$. Аналогично вещественному случаю, можно доказать, что тогда непрерывны функции $f + g$, $fg$ и $\frac fg$ (последняя --- если $g(z_0) \ne 0$).
\end{note}

\begin{proposition}
	Пусть функция $f$ непрерывна в точке $z_0 \in \CM$, $w_0 := f(z_0) \in \CM$, функция $g$ непрерывна в точке $w_0$. Тогда функция $g \circ f$ непрерывна в точке $z_0$.
\end{proposition}

\begin{proof}
	Рассмотрим произвольную последовательность $\{z_n\} \subset B_{\delta_0}(z_0)$ такую, что $\lim_{n \to \infty}z_n = z_0$. Тогда, в силу непрерывности функции $f$, $\lim_{n \to \infty}f(z_n) = f(z_0) = w_0$. Но тогда, в силу непрерывности функции $g$, $\lim_{n \to \infty}g(f(z_n)) = g(w_0) = f(g(z_0))$. Получено требуемое в силу произвольности выбора $\{z_n\}$.
\end{proof}

\section{Основные элементарные функции}

\begin{definition}
	\textit{Косинусом} и \textit{синусом} комплексного аргумента называются следующие функции:
	\[\cos{z} := \frac{e^{iz} + e^{-iz}}2,~\sin{z} := \frac{e^{iz} - e^{-iz}}{2i},~z \in \Cm\]
\end{definition}

\pagebreak

\begin{definition}
	\textit{Гиперболическим косинусом} и \textit{гиперболическим синусом} комплексного аргумента называются следующие функции:
	\[
		\ch{z} := \frac{e^{z} + e^{-z}}2,~\sh{z} := \frac{e^{z} - e^{-z}}{2},~z \in \Cm
	\]
\end{definition}

\begin{note}
	Комплексная экспонента, многочлены, а также тригонометрические и гиперболические синус и косинус считаются \textit{основными элементарными функциями}. По уже доказанным свойствам, они непрерывны в каждой точке из $\Cm$.
\end{note}