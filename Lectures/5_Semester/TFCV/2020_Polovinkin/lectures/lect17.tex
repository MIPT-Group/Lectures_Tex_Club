\begin{flushright}
    \textit{Лекция 17 (от 02.11)}
\end{flushright}
\section{$\S 22.$ Конформные отображения в $\overline{\CC}$.}
\begin{center}
    \textbf{Геометрический смысл аргумента и модуля производной}
\end{center}
Рассмотрим $f: B_r(z_0) \mapsto \CC$, $f'(z_0) \neq 0$.
\\
Пусть $w = f(z)$, $w_0 = f(z_0)$, $w-w_0 = f'(z_0)(z-z_0) + o(z-z_0)$.
\\
Покоординатно:
\begin{align*}
  & f'(z_0) = u_x+iv_x, \ f=u+iv
\end{align*}
\begin{align*}
  & \left( \begin{matrix}
          \Delta u \\
          \Delta v
      \end{matrix} \right) = \left( \begin{matrix}
          u_x & -v_x \\
          v_x & u_x
      \end{matrix} \right)  \left( \begin{matrix}
          \Delta x \\
          \Delta y
      \end{matrix} \right) = K \left( \begin{matrix}
          \frac{u_x}{K} & \frac{-v_x}{K} \\
          \frac{v_x}{K} & \frac{u_x}{K}
      \end{matrix} \right) \left( \begin{matrix}
          \Delta x \\
          \Delta y
      \end{matrix} \right)
\end{align*}
\begin{align*}
  & K = \sqrt{u^2_x+v^2_x} = \left| f'(z_0) \right|
\end{align*}
Видим, что это ортогональное преобразование.
\\
\textbf{Свойство сохранения окружности в малом:}
\\
Рассмотрим $\gamma_r = \left\{ z: \left| z-z_0 \right| = r, \ 0 < r <
    r_0\right\}$. Пусть в области, ограниченной кривой, производная ненулевая.
\begin{align*}
  & \left| \Delta w \right| \approx \left| f'(z_0) \right|\cdot \left| \Delta z \right| \approx K r
\end{align*}
(получаем <<примерно окружность>>).
\\
\textbf{Свойство сохранения углов:}
\\
Рассмотрим теперь $\gamma_1, \gamma_2: \ z = z_k(t), \ t \in \left[ t_0 -\delta;
    t_0+\delta\right], \ z_k(t_0) = z_0$. Пусть при таких $t$ $z'_k\neq 0$, угол
между кривыми $\alpha$. Пусть $\gamma_1^* = f(\gamma_1)$, $\gamma_2^* =
f(\gamma_2)$. Тогда угол между $\gamma_1^*$ и $\gamma_2^*$ также равен $\alpha$.
\\
Действительно,
\begin{align*}
  & w'_k(t_0) = f'(z_0)z'_k(t_0)
\end{align*}
\begin{align*}
  & \Arg w'_k(t_0) = \argm f'(z_0) + \Arg z'_k(t_0)
\end{align*}
(изменение на одинаковый угол).
\begin{center}
    \textbf{Конформные отображения в $\CC$}
\end{center}
\Def
Функция $f: G \mapsto \CC$ называется \textbf{конформной в точке $z_0$}, если
$f= u+iv$, $u$ и $v$ дифференцируемы в $z_0$ и линейное отображение вида
\begin{equation}\label{(22.1)}
    \begin{cases}
        du = u_x(x_0,y_0)\Delta x + u_y(x_0,y_0) \Delta y \\
        dv = v_x(x_0,y_0)\Delta x + v_y(x_0,y_0) \Delta y
    \end{cases}
\end{equation}
является суперпозицией линейного растяжения и поворота относительно нуля.
\theorem
Функция $f$ конформна в $Z_0 \in \CC$ тогда и только тогда, когда она в этой
точке дифференцируема, а производная отлична от нуля.
\pr
~
\begin{itemize}
    \item $\Leftarrow$
    \\
    Следует из геометрического смысла и определения.
    \item $\Rightarrow$
    \\
    Пусть $f$ конформна в $z_0$. Тогда $\exists K > 0$, $\exists \theta \in
    [0;2\pi)$ такие, что из \eqref{(22.1)} получаем
    \begin{equation}\label{(22.2)}
        \left( \begin{matrix}
                \Delta u \\
                \Delta v
            \end{matrix} \right) = K \left( \begin{matrix}
                \cos \theta & \sin \theta \\
                -\sin \theta & \cos \theta
            \end{matrix} \right) \left( \begin{matrix}
                \Delta x \\
                \Delta y
            \end{matrix} \right)
    \end{equation}
    В силу дифференцируемости $u$ и $v$
    \begin{align*}
      & u_x = K \cos \theta, \ u_y = K \sin \theta, \ v_x = -K \sin \theta, \ u_y = K \cos \theta
    \end{align*}
    Выполняется УКР, значит, $\exists f'(z_0)$, $\left| f'(z_0) \right| = K >
    0$.
\end{itemize}
\Def
$f: G \mapsto \CC$ \textbf{конформна на области $G$}, если $f$ однолистна на $G$
и конформна в каждой ее точке.
\corollary
$f$ конформна в $G \subset \CC$ $\Leftrightarrow$ $f$ однолистна и регулярна в
$G$.
\begin{center}
    \textbf{Конформные отображения в $\CCC$}
\end{center}
\underline{\textbf{Свойства стереографической проекции}}
\begin{enumerate}
    \item Образы любых двух пересекающихся кривых на комплексной плоскости будут
    пересекаться на сфере Римана под тем же углом.
    \item $w = \frac{1}{z}: \CCC \mapsto \CCC$ соответствует при
    стереографической проекции отображению сферы Римана на себя путем поворота
    ее на $\pi$ относительно ее диаметра с концами в точках, являющихся образами
    $1$ и $-1$ на $\CC$.
\end{enumerate}
\Def
Пусть $f$ имеет УОТ в $\infty$. Тогда $f$ называется \textbf{конформной в
  $\infty$}, если $g(z) = f\left( \dst \frac{1}{z} \right)$, доопределенная по
непрерывности в нуле, конформна в нуле.
\Def
Пусть $a \in \CCC$~--- полюс или СОТ $f$. Тогда $f$ называется
\textbf{конформной в $a$}, если $\varphi(z) = \dst \frac{1}{f(z)}$,
доопределенная по непрерывности, конформна в этой точке.
\Exse
Доказать, что в определении $22.4$ допустим лишь полюс $1$ порядка.
\Def
Пусть $f: G \mapsto \CCC$. Тогда $f$ называется \textbf{конформной в области
  $G$}, если она однолистна на ней и конформна в каждой ее точке.
\prop
$f$ конформна в $G \subseteq \CCC$, если $f$ однолистна на ней и регулярна на
ней, за исключением, быть может, двух точек:
\begin{itemize}
    \item $\infty$, если $\infty \in G$ и является УОТ или полюсом $1$ порядка;
    \item $a \in G$, $a \neq \infty$~--- полюс $1$ порядка, если $\infty$~---
    УОТ или $\infty \not \in G$.
\end{itemize}
\section{$\S 23.$ Дробно-линейные функции.}
\Def
Функция
\begin{equation}\label{(23.1)}
    w = \frac{az+b}{cz+d}, \ a,b,c,d \in \CC, \ ad-cb \neq 0
\end{equation}
называется \textbf{дробно-линейной функцией (ДЛФ)} и задает
\textbf{дробно-линейное отображение (ДЛО).} Бывает:
\begin{equation}\label{(23.2)}
    \left[ \begin{matrix}
            c = 0 \Rightarrow w = az+b \Rightarrow w(\infty) = \infty \\
            c \neq 0 \Rightarrow w(\infty) = \dst \frac{a}{c}, \ w\left( -\dst \frac{d}{c} \right) = \infty
        \end{matrix} \right.
\end{equation}
ДЛФ \eqref{(23.1)}, \eqref{(23.2)} действует из $\CCC$ в $\CCC$.
\theorem
ДЛФ \eqref{(23.1)}, \eqref{(23.2)} конформно отображает $\CCC$ на $\CCC$.
\pr
~
\begin{enumerate}
    \item Проверим однолистность на $\CCC$.
    \\
    Из \eqref{(23.1)}
    \begin{equation}\label{(23.3)}
        z = \frac{dw-b}{cw-a}
    \end{equation}
    Видим, что $ad - cb \neq 0$, а значит, существует обратное ДЛО.
    \item Покажем конформность в каждой точке.
    \begin{itemize}
        \item $z_0 \neq - \dst \frac{d}{c}$, $z_0 \in \CC$. Тогда
        \begin{align*}
          & w'(z_0) = \frac{ad-cb}{(cz_0+d)^2} \neq 0
        \end{align*}
        что и хотим видеть.
        \item $z_0  = -\dst \frac{d}{c}$. Положим
        \begin{align*}
          & \varphi(z) = \frac{1}{f(z)} = \frac{cz+d}{az+b}
        \end{align*}
        и тогда
        \begin{align*}
          & \varphi'(z_0) = \frac{cb - ad}{\left( -a\dst \frac{d}{c} + b \right)^2} = \frac{c^2}{-ad + bc} \neq 0
        \end{align*}
        что и хотим видеть.
        \item $z_0 = \infty$. Положим
        \begin{align*}
          & g(z) = f\left( \frac{1}{z} \right) = \frac{a+bz}{c+dz}
        \end{align*}
        и тогда
        \begin{align*}
          & g'(z_0) = \frac{bc - ad}{c^2} \neq 0
        \end{align*}
        что и хотим видеть.
    \end{itemize}
\end{enumerate}
\Exse
Пусть $f: \CCC \mapsto \CCC$ конформно, тогда $f$~--- ДЛФ. Доказать это
утверждение.
\theorem
При ДЛО \eqref{(23.1)}, \eqref{(23.2)} образом окружности или прямой будет
окружность или прямая.
\pr
~
\begin{enumerate}
    \item Рассмотрим аффинное отображение $w = az+b$ ($c = 0$).
    \\
    Знаем из аналитической геометрии, что окружность переходит в окружность, а
    прямая~--- в прямую.
    \item Рассмотрим теперь $c \neq 0$.
    \\
    Представим отображение в виде
    \begin{align*}
      & w = \frac{az+b}{cz+d} = \frac{a}{c} + \frac{-ad+bc}{c}\cdot \frac{1}{cz+d}
    \end{align*}
    \begin{equation}\label{(23.4)}
        w = \alpha + \beta t, \ \alpha = \frac{a}{c}, \ \beta = \frac{-ad+bc}{c}, \ t = \frac{1}{\zeta}, \ \zeta = cz+d
    \end{equation}
    Видим, что $w(t)$ и $\zeta(z)$~--- аффинные, проверим выполнимость
    утверждения теоремы для $z = \dst \frac{1}{\zeta}$.
    \\
    Положим $\zeta = \xi + i \eta$. Уравнение
    \begin{align*}
      & A(\xi^2+\eta^2) + B\xi + C\eta + D = 0, \ 4AD < B^2+C^2
    \end{align*}
    задает невырожденную окружость при $A \neq 0$ и невырожденную прямую при $A =
    0$. Полагая $t = \dst \frac{1}{\zeta}$, учитывая $\xi^2 + \eta^2 =
    \zeta\bar{\zeta}$, $\xi = \dst \frac{\zeta + \bar{\zeta}}{2}$, $\eta = \dst
    \frac{\zeta - \bar{\zeta}}{2i}$, запишем уравнение в виде
    \begin{align*}
      & A\zeta\bar{\zeta} + \left( \frac{B}{2} + \frac{C}{2i}\right)\zeta + \left( \frac{B}{2} - \frac{C}{2i}\right)\bar{\zeta} + D = 0
    \end{align*}
    и отсюда получим
    \begin{align*}
      & A + \left( \frac{B}{2} + \frac{C}{2i}\right)\bar{t} + \left( \frac{B}{2} - \frac{C}{2i}\right)t + Dt\bar{t} = 0
    \end{align*}
    что задает окружность при $D \neq 0$ и прямую при $D = 0$. Суперпозиция
    преобразований, переводящих окружности и прямые в окружности и прямые,
    переводит окружности и прямые в окружности и прямые.
\end{enumerate}
\Note
Окружность или прямая $\gamma$ переходит при ДЛО в прямую, если нуль знаменателя
принадлежит $\gamma$, и в окружность иначе.
\\
Это называется \textbf{круговым свойством}.
\Def
Точки $M$ и $M^*$ называются \textbf{симметричными относительно окружности с
  центром в точке $A$ радиуса $R > 0$}, если они лежат на одном луче, исходящем
из точки $A$, и $\left| AM \right| \cdot \left| AM^* \right| = R^2$. На $\CCC$
$z$ и $z^*$ симметричны относительно окружности $\gamma_r$ с центром в точке
$a$, если
\begin{equation}\label{(23.5)}
    z^* - a = \frac{R^2}{\bar{z} - \bar{a}}
\end{equation}
Заметим, что при $z \to a$ $z^* \to \infty$, т.~е. $a$ и $\infty$ симметричны.