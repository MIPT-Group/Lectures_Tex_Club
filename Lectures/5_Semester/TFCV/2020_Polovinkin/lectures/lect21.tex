\begin{flushright}
    \textit{Лекция 21 (от 16.11)}
\end{flushright}
\theorem
В общей задаче Дирихле при существовании решения в ограниченной области онобудет
единственным. Решение ищем как ограниченную функцию.
\pr
От противного. Допустим, существуют два различных решения $u_1$, $u_2$. Тогда $w
= u_1-u_2 : G \mapsto \RR$, $\Delta w = 0$, $w\Big|_\Gamma = 0$. Тогда по лемме
$1$ $w \equiv 0$.
\Note
Условие ограниченности области является лишь техническим. Условие же
ограниченности функции на области существенно.
\Example
\begin{align*}
  & u(x,y) = \frac{x^2+y^2-2x}{x^2+y^2}
\end{align*}
\begin{align*}
  & G = \left\{ (x,y) \mid x^2+y^2<2x \right\} = \left\{ z: \left| z \right| <1\right\}
\end{align*}
При граничном условии~--- нуле эта функция является решением задачи Дирихле, но
и тождественный нуль также решение. При условии ограниченности эта функция не
подходит.
\begin{center}
    \textbf{Класическая задача Дирихле в круге $B_R(0), \ R > 0$}
\end{center}
\begin{equation}\label{(26.4)}
    \begin{cases}
        \Delta u = 0, \ \left| z \right|< R \\
        u \big|_{\left| z \right| = R} = u_0(z)
    \end{cases}
\end{equation}
причем $u_0(z)$ непрерывна на $\gamma_r$ и решение \eqref{(26.4)} непрерывно на
некотором $B_{R_1}(0), \ R_1 > R$.
\\
Тогда существует регулярная $f: B_{R_1}(0) \mapsto \CC$, что $\Real f(z) =
u(z)$. Тогда по интегральной формуле Коши
\begin{align*}
  & \forall z \in B_R(0) \ f(z) = \frac{1}{2 i \pi} \int_{\gamma_R} \frac{f(\zeta)}{\zeta - z}d\zeta = \frac{1}{2 \pi} \int_{0}^{2\pi} \frac{f(Re^{i\psi})\zeta(\psi)}{\zeta(\psi) - z}d\psi
\end{align*}
Введем теперь симметричную точку $z^* = \dst \frac{R^2}{z}$. Тогда
\begin{align*}
  & \frac{1}{2 i \pi} \int_{\gamma_R} \frac{f(\zeta)}{\zeta - z^*}d\zeta = \frac{1}{2 \pi} \int_{0}^{2\pi} \frac{f(Re^{i\psi})\zeta(\psi)}{\zeta(\psi) - z^*}d\psi = 0
\end{align*}
\begin{align*}
  & f(z) = \frac{1}{2\pi}\int_{0}^{2\pi}f(\zeta(\psi))\left( \frac{\zeta(\psi)}{\zeta(\psi) - z} - \frac{\zeta(\psi)}{\zeta(\psi) - z^*}\right)d\psi = \frac{1}{2\pi}\int_{0}^{2\pi}f(\zeta(\psi))\left( \frac{\zeta(\psi)}{\zeta(\psi) - z} - \right. \\
  & \left. - \frac{\zeta(\psi)\ol{z}}{\zeta(\psi)\ol{\zeta(\psi)} - \zeta(\psi)\ol{z}}\right)d\psi = \frac{1}{2\pi}\int_{0}^{2\pi}f(\zeta(\psi))\left( \frac{\zeta(\psi)\ol{\zeta(\psi)} - z\ol{z}}{\left| \zeta(\psi) - z \right|^2}\right)d\psi = \frac{1}{2\pi}\int_{0}^{2\pi}f(\zeta(\psi))\cdot \\
  & \cdot \left( \frac{\abs{\zeta(\psi)}^2 - \abs{z}^2}{\left| \zeta(\psi) - z \right|^2}\right) d\psi = \frac{1}{2\pi}\int_{0}^{2\pi}f(Re^{i\psi})\left( \frac{R^2 - \abs{z}^2}{\left| Re^{i\psi} - z \right|^2}\right) d\psi
\end{align*}
\begin{equation}\label{(26.5)}
    u(z) = \frac{1}{2\pi}\int_{0}^{2\pi}\tilde{u}_0(\psi)\left( \frac{R^2 - \abs{z}^2}{\left| Re^{i\psi} - z \right|^2}\right) d\psi = \frac{1}{2\pi}\int_{0}^{2\pi}u_0(Re^{i\psi})\left( \frac{R^2 - \abs{z}^2}{\left| Re^{i\psi} - z \right|^2}\right) d\psi
\end{equation}
\eqref{(26.5)} называется \textbf{формулой Пуассона}, интеграл~---
\textbf{интегралом Пуассона}.
\begin{equation}\label{(26.6)}
    K(\zeta,z) = \frac{1}{2\pi} \frac{\abs{\zeta}^2 - \abs{z}^2}{\left| \zeta - z \right|^2}
\end{equation}
\eqref{(26.6)} называется \textbf{ядром (интеграла) Пуассона}.
\begin{equation}\label{(26.7)}
    u(z) = \int_{0}^{2\pi}\tilde{u}_0(\psi)K\left(Re^{i\psi},z\right) d\psi
\end{equation}
\begin{equation}\label{(26.8)}
    K(\zeta,z) = \Real\left( \frac{1}{2\pi} \frac{\zeta+z}{ \zeta - z }\right)
\end{equation}
\begin{align*}
  & f(z) = \frac{1}{2\pi}\int_{0}^{2\pi}\tilde{u}_0(\psi)\frac{\zeta+z}{\zeta-z}d\psi = \frac{1}{2i\pi}\int_{\gamma_R}u_0(\zeta)\frac{\zeta+z}{\zeta-z}\frac{d\zeta}{\zeta}
\end{align*}
\begin{equation}\label{(26.9)}
    u(z) = \Real \frac{1}{2i\pi}\int_{\gamma_R}u_0(\zeta)\frac{\zeta+z}{\zeta-z}\frac{d\zeta}{\zeta}
\end{equation}
\lemma
\begin{align*}
  &\forall z \in B_R(0) \ I(z) = \int_0^{2\pi}K(Re^{i\psi}, z) d\psi = 1
\end{align*}
\pr
\begin{align*}
  & I(z) = \Real I^*(z) = \Real \frac{1}{2\pi}\int_0^{2\pi}\frac{\zeta + z}{\zeta - z} d\psi = \Real \frac{1}{2\pi}\int_{\gamma_R}\frac{\zeta + z}{\zeta - z} \frac{d\zeta}{\zeta} = \Real \left( \us{z}{\res}g(\zeta) + \us{0}{\res}g(\zeta)\right) = \\
  & = \Real(2 - 1) = 1
\end{align*}
\lemma
Пусть $\delta \in \left( 0;\dst \frac{\pi}{2} \right)$, $\zeta_0 \in \gamma_R: \
\zeta_0 = Re^{i\psi_0}, \ \psi_0 \in [0;2\pi)$. Пусть
\begin{align*}
  & \gamma(0, \delta) = \left\{ \zeta \in \gamma_R \mid \zeta = Re^{i\psi}, \ \psi \in (\psi_0+\delta, \psi_0 +2\pi - \delta) \right\}
\end{align*}
Тогда
\begin{align*}
  & \lim_{z \os{B_R(0)}{\to} \zeta_0} \max_{\zeta \in \gamma(0;\delta)}K(\zeta, z) = 0
\end{align*}
\pr
\begin{align*}
  & K(\zeta,z) = \frac{1}{2\pi} \frac{R^2-\abs{z}}{ \abs{\zeta - z }^2}
\end{align*}
$\forall \varepsilon > 0$ рассмотрим $z \in B_\varepsilon(\zeta_0) \cap B_R(0)$;
тогда
\begin{align*}
  & \forall \zeta \in \gamma_R \ \exists \varepsilon > 0: \ \left| \zeta - \zeta_0 \right| = R\left| 1-e^{i(\psi_0-\psi)} \right| > \varepsilon_0, \ \left| z - \zeta_0 \right| < \frac{\varepsilon_0}{2}
\end{align*}
\begin{align*}
  & \forall \zeta \in \gamma_R \ \exists \varepsilon > 0: \ \left| \zeta - z \right| \geq \left| \zeta - \zeta_0 \right| - \left| \zeta_0 - z \right| > \frac{\varepsilon_0}{2}
\end{align*}
\begin{align*}
  & 0 \leq \lim_{z \os{B_R(0)}{\to} \zeta_0} \max_{\zeta \in \gamma(0;\delta)}K(\zeta, z) \leq \lim_{z \os{B_R(0)}{\to} \zeta_0}\frac{2}{\pi\varepsilon_0}\left( R^2-\left| z \right|^2 \right) = 0
\end{align*}
\theorem
Решение общей задачи Дирихле существует в круге $B_r(0)$ и описывается
интегралом Пуассона.
\pr
\begin{equation}\label{(26.10)}
    u(z) = \frac{1}{2\pi}\int_{0}^{2\pi}\tilde{u}_0(\psi)\left( \frac{R^2 - \abs{z}^2}{\left| Re^{i\psi} - z \right|^2}\right) d\psi
\end{equation}
Докажем гармоничность.
\begin{align*}
  & f(z) = \frac{1}{2\pi}\int_{0}^{2\pi}\tilde{u}_0(\psi)\frac{\zeta + z}{\zeta-z}d\psi, \ \zeta = Re^{i\psi}
\end{align*}
Пусть $z \in B_R(0)$, $z+\Delta z\in B_R(0)$. Тогда
\begin{align*}
  & \frac{f(z+\Delta z) - f(z)}{\Delta z} - \frac{1}{\pi}\int_{0}^{2\pi}\tilde{u}_0(\psi)\frac{\zeta}{(\zeta-z)^2}d\psi = \frac{1}{\pi}\int_{0}^{2\pi}\tilde{u}_0(\psi)\left( \frac{z+\Delta z + \zeta}{\zeta - \Delta z - z} + \frac{\zeta + z}{\zeta - z} - \frac{\zeta}{(\zeta-z)^2}\right) d\psi = \frac{1}{\pi}\int_{0}^{2\pi}\tilde{u}_0(\psi)\left( \frac{\zeta}{(\zeta - \Delta z - z)(\zeta - z)} - \frac{\zeta}{(\zeta-z)^2}\right) d\psi = \frac{1}{\pi}\int_{0}^{2\pi}\tilde{u}_0(\psi)\left( \frac{\zeta \Delta z}{(\zeta - \Delta z - z)(\zeta - z)^2}\right) d\psi \us{\Delta z \to 0} 0
\end{align*}
Производная существует, а значит, функция регулярна, а $u$ гармоническая.
\\
Докажем ограниченность. Из определения общей задачи Дирихле на границе за
вычетом конечного числа точек ограничена $u_0(z)$, а поскольку интеграл ядра
равен $1$, то и $u(z)$ ограничена тем же числом.
\\
Докажем непрерывность на границе. Для этого докажем, что
\begin{align*}
  & \lim_{z\os{B_R(0)}{\to} \zeta_0} u(z) = u(\zeta_0)
\end{align*}
Пусть
\begin{align*}
  &\Delta I=  \int_0^{2\pi}\tilde{u}_0(z)K(\zeta(\psi), z)d\psi - u_0(\zeta_0) = \int_0^{2\pi}(\tilde{u}_0-u_0(\zeta_0))K(\zeta(\psi), z)d\psi
\end{align*}
В силу непрерывности $u_0$ в $\zeta_0$
\begin{align*}
  & \forall \varepsilon > 0 \ \exists \delta > 0: \ \left| \psi - \psi_0 \right| < \delta \ \left| u_0(\zeta) - u_0(\zeta_0) \right| <\varepsilon, \ \zeta_0 = Re^{i\varphi_0}
\end{align*}
Положим
\begin{align*}
  & \gamma(1, \delta) = \left\{ \zeta \mid \zeta = Re^{i\psi}, \ \left| \psi - \psi_0 \right| < \delta \right\}, \ \gamma(0, \delta) = \gamma_R\setminus\gamma(1,\delta)
\end{align*}
\begin{align*}
  &\Delta I=  \int_{\psi_0-\delta}^{\psi_0+\delta}(\tilde{u}_0-u_0(\zeta_0))K(\zeta(\psi), z)d\psi + \int_{\psi_0+\delta}^{\psi_0+2\pi-\delta}(\tilde{u}_0-u_0(\zeta_0))K(\zeta(\psi), z)d\psi
\end{align*}
Первый интеграл ограничен сверху $\varepsilon$; для второго же по лемме $2$
$\exists \rho: \ \forall z \in B_\rho(z_0) \max_{\zeta \in
  \gamma(1,\delta)}K(\zeta,z) <\varepsilon$, а значит, модуль интеграла
ограничен сверху $4\pi M \varepsilon$. Значит, можем заметить, что предел
суммарного интеграла будет равен нулю.
\theorem
На ограниченной односвязной области $G$ с кусочно гладкой границей $\Gamma$
существует решение обобщенной задачи Дирихле.
\pr
По теореме Римана существует регулярная $f: G \mapsto B_1(0)$, конформная на
этой области. Тогда $\exists z=g(w): B_1(0) \mapsto G$, и это также конформное
отображение.
\\
То же верно и для замыканий по принципу соответствия границ. Пусть $f(\zeta) =
\alpha$, $g(\alpha) = \zeta$, $\left| \alpha \right|= 1$, $\zeta \in \Gamma$. В
исходной задаче Дирихле $u_0(\zeta)$ непрерына на границе за исключением
конечного числа точек $\tilde{u}(\alpha) = u(g(\alpha))$ обладает тем же
свойством на $\gamma_1$, и по теореме $3$ существует решение $u(w)$ задачи
Дирихле с граничным условием $\tilde{u}(\alpha)$. По формуле Пуассона
\begin{align*}
  & u(w) = \frac{1}{2i\pi}\int_{\gamma_1}\tilde{u}(\alpha)\frac{\alpha + w}{\alpha - w} \frac{d\alpha}{\alpha}
\end{align*}
$u(z) = u(f(z)): G \mapsto \RR$, эта функция гармоническая по теореме $1$, а на
границе $u(\zeta) = u_0(\zeta) = \tilde{u}(\alpha)$. Тогда
\begin{equation}\label{(26.11)}
    u(z) = \Real \frac{1}{2i\pi}\int_{\Gamma}u_0(\zeta)\frac{f(\zeta)+f(z)}{f(\zeta) - f(z)} \frac{f'(\zeta)d\zeta}{f(\zeta)}
\end{equation}
Заметим, что существует дробно-линейное отображение $\Img z > 0 \mapsto B_1(0)$.
Тогда, используя формулу \eqref{(26.11)}, получим теорему:
\theorem
Пусть $u_0(x)$ непрерывна на $\RR$ за исключением, быть может, конечного числа
точек разрыва первого рода, и ограничена. Тогда решение общей задачи Дирихле в
$\Img z > 0$ существует и описывается формулой
\begin{equation}\label{(26.12)}
    u(z) = \frac{1}{\pi}\int_{-\infty}^\infty \frac{y u_0(t)}{(x-t)^2 +y^2} dt, \ z = x+iy
\end{equation}
