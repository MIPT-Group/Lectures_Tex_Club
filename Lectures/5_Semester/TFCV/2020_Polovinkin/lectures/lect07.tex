\begin{flushright}
    \textit{Лекция 7 (от 28.09)}
\end{flushright}
\begin{equation}\label{(9.5)}
    \sum_{n=1}^{\infty}f_n(z), \ f_n: G \mapsto \CC
\end{equation}
\Def
Функциональный ряд \eqref{(9.5)} \textbf{сходится локально равномерно на $G$},
если $\forall z \in G \ \exists B_r(z) \subseteq G$, на котором \eqref{(9.5)}
сходится равномерно.
\theorem (Вейерштрасса)
Пусть $f_n: G \mapsto \CC$ регулярны на $G$, а ряд $\dst
\sum_{n=1}^{\infty}f_n(z)$ сходится локально равномерно на $G$.
\\
Тогда
\begin{enumerate}
    \item $S(z)$~--- сумма ряда \eqref{(9.5)}~--- регулярна на $G$.
    \item ряд \eqref{(9.5)} можно почленно дифференцировать:
    \begin{equation}\label{(9.6)}
        \forall k \in \NN, \ \forall z \in G \ S^{(k)}(z) = \sum_{n=1}^{N}f^{(k)}_n(z)
    \end{equation}
    причем ряд \eqref{(9.6)} сходится локально равномерно на $G$.
\end{enumerate}
\pr
Фиксируем произвольную $z_0 \in G$ и $r>0, \ r_1>0: \ \overline{B_{r+r_1}(z_0)}
\subseteq G$. По определению $3$
\begin{equation}\label{(9.7)}
    \forall \varepsilon > 0 \ \exists N(\varepsilon): \ \forall N \geq N(\varepsilon) \sup_{\zeta \in \overline{B_{r+r_1}(z_0)}}\left| S(\zeta) - S_N(\zeta) \right| \leq \varepsilon
\end{equation}
где
\begin{equation}\label{(9.8)}
    S_N(\zeta) = \sum_{n=1}^{\infty}f_n(\zeta)
\end{equation}
\begin{enumerate}
    \item Пусть $\gamma_1 = \{\zeta \mid \left| \zeta - z_0 \right| =
    r+r_1\}$~--- положительно ориентированная.
    $S_N(z)$ регулярна, тогда по интегральной формуле Коши
    \begin{equation}\label{(9.9)}
        \forall z \in B_r(z_0) \ S_N(z) = \frac{1}{2 \pi i}\int_{\gamma_1}\frac{S_N(\zeta)}{(\zeta - z)}d \zeta
    \end{equation}
    По теореме $3$ $\S 6$ $S(z)$ непрерывна на $\overline{B_{r+r_1}}(z_0)$.
    Рассмотрим
    \begin{align*}
      & \left| S_N(z) - \frac{1}{2 \pi i}\int_{\gamma_1}\frac{S(\zeta)}{\zeta - z} d\zeta \right| = \frac{1}{2 \pi}\left| \int_{\gamma_1} \frac{S_N(\zeta) - S(\zeta)}{\zeta - z} d \zeta \right| \leq \frac{1}{2 \pi} \varepsilon \int_{\gamma_1}\frac{\left| d \zeta \right|}{\left| \zeta - z \right|} \leq \frac{\varepsilon}{2 \pi r_1}2 \pi \cdot \\
      & \cdot (r+r_1) = \frac{r+r_1}{r_1}\varepsilon
    \end{align*}
    \begin{align*}
      \forall z \in B_r(z_0) \ S(z) = \frac{1}{2\pi i}\int_{\gamma_1}\frac{S(\zeta)}{\zeta - z}d\zeta
    \end{align*}
    Отсюда следует бесконечная дифференцируемость $S$ в $B_r(z_0)$, а значит,
    регулярность в $z_0$, а в силу произвольности выбора $z_0$~--- во всей $G$.
    \item $S$, $S_N$~--- регулярные функции; по интегральным формулам Коши и
    теореме $3$ $\S 8$
    \begin{align*}
      \forall z \in B_r(z_0) \ S^{(k)}(z) = \frac{k!}{2 \pi i} \int_{\gamma_1} \frac{S(\zeta)}{(\zeta - z)^{k+1}}d\zeta
    \end{align*}
    \begin{align*}
      \forall z \in B_r(z_0) \ S_N^{(k)}(z) = \frac{k!}{2 \pi i}\int_{\gamma_1}\frac{S_N(\zeta)}{(\zeta - z)^{k+1}} d \zeta 
    \end{align*}
    Из \eqref{(9.7)} $\forall \varepsilon > 0 \ \exists N(\varepsilon): \
    \forall N\geq N(\varepsilon)$ выполняется \eqref{(9.7)}.
    \begin{align*}
      \left| S^{(k)}(z) - S_N^{(k)}(z) \right| \leq \frac{k!}{2\pi}\int_{\gamma_1}\frac{\left| S(\zeta) - S_N(\zeta) \right|}{\left| \zeta - z \right|^{k+1}} \left| d\zeta \right|\leq \frac{\varepsilon 2 \pi (r+r_1)k!}{2 \pi r^{k+1}}
    \end{align*}
    что для любого фиксированного $k$ сколь угодно мало.
    \begin{align*}
      \forall z \in B_r(z_0) \ \lim_{N \to \infty}S^{(k)}_N (z) = S^{(k)}(z)
    \end{align*}
\end{enumerate}
\corollary
Всякий степенной ряд
\begin{align*}
  \sum_{n=0}^{\infty}c_n(z-a)^n
\end{align*}
имеет суммой регулярную функцию и дифференцируем почленно.
\corollary
Любая регулярная функция представима в виде ряда Тейлора, а любая представимая
рядом Тейлора регулярна.
\section{$\S 10.$ Некоторые свойства регулярных функций}
\theorem (единственности)
Пусть $f: G \mapsto \CC$ регулярна на $G$; пусть $\exists z_n \to a, \ z_n \in
G, \ a \in G, \ z_n \neq a: \ \forall n \ f(z_n) = 0$.
\\
Тогда $f(z) \equiv 0$ на $G$.
\pr
~
\begin{enumerate}
    \item Т.~к. $a \in G$, то $\exists \rho_0 = \dst \inf_{\zeta \in \partial
      G}\left| a-\zeta \right| > 0: \ B_{\rho_0}(a) \subseteq G $. По теореме
    $2$ $\S 9$ существует разложение $f(z) = \dst \sum_{n=0}^{\infty}c_n(z-a)^n$
    в ряд Тейлора для любого $z \in B_{\rho_0}(a)$.
    \begin{align*}
      f(a) = \lim_{n \to \infty} f(z_n) = 0 = c_0
    \end{align*}
    Пусть существует наименьшее натуральное $m$ такое, что $c_m \neq 0$. тогда
    \begin{align*}
      f(z) = \sum_{n=m}^{\infty}c_n(z-a)^n = (z-a)^mh(z)
    \end{align*}
    \begin{align*}
      h(z) = \sum_{j=0}^{\infty}c_{m+j}(z-a)^j; \ h(a) = c_m \neq 0
    \end{align*}
    $h$ регулярна, значит, $\exists \varepsilon \in (0; \rho_0): \ h(z) \neq 0
    \ \forall z \in B_{\varepsilon}(a)$.
    \\
    Но тогда в $\overset{\circ}{B}_{\varepsilon}(a)$ $((z-a)^m \neq 0) \wedge
    (h(z) \neq 0)$ $\Rightarrow$ $\forall z \in
    \overset{\circ}{B}_{\varepsilon}(a) \ f(z) \neq 0$.
    \\
    Противоречие.
    \\
    Значит, $\forall z \in B_{\rho_0}(a) \ f(z) = 0$.
    \item Рассмотрим произвольное $b \in G \setminus B_{\rho_0}(a)$.
    \\
    Существует спрямляемая $\gamma_{ab} \subseteq G$. Пусть $\rho = \inf\{\left|
        \zeta - z \right| \mid \zeta \in \gamma_{ab}, \ z \in \Gamma\} > 0$;
    $\rho_0 \geq \rho$.
    \\
    Можем построить разбиение $\zeta_0 = a, \zeta_1, \dots, \zeta_{K-1}, \zeta_K
    = b$, такое, что $l(\zeta_k, \zeta_{k+1}) \leq \dst \frac{\rho}{2}$. Пусть
    $B_k = B_{\rho}(\zeta_k), \ k \in \{0, \dots, K\}$.
    \\
    По построению $\zeta_{k+1}\in B_{\rho}(\zeta_k) \cap B_{\rho}(\zeta_{k+1}) =
    B_k \cap B_{k+1}$.
    \\
    По индукции:
    \begin{enumerate}
        \item База: $\zeta_0 = a$, $B_0 \subseteq B_{\rho_0}(a): \ \forall z \in
        B_0 \ f(z) = 0$.
        \item Шаг: если $\forall z \in B_k \ f(z) = 0$, то $[\zeta_k;
        \zeta_{k+1}]\subseteq B_{k+1}\cap B_{k}$, значит, $\forall z \in B_{k+1}
        \ f(z) = 0$ (по п. $1$), а значит, $f(z) = 0$ на $B_k$ и $f(b) = 0 $.
    \end{enumerate}
\end{enumerate}
\corollary
Пусть $f, g: G \mapsto \CC$ регулярны на $G$ и $\exists E \subseteq G$,
содержащее нестационарные последовательности, сходящиеся к точке из $G$. Пусть
$\forall z \in E \ f(z) = g(z)$.
\\
Тогда $\forall z \in G \ f(z) = g(z)$.
\pr
Очевидно из теоремы $1$, полагая $h(z) = f(z) - g(z)$.
\example
$f(z) = e^z = e^{x}e^{iy}$~--- регулярна во всей $\CC$.
\\
$g(z) = \dst \sum_{n=0}^{\infty} \dst \frac{z^n}{n!}$ по теореме Вейерштрасса
регулярна в $G$ как равномерно сходящийся ряд.
\\
Известно, что $\forall z \in \RR \ f(z) = g(z)$, а значит, по теореме
единственности и $\forall z \in \CC \ f(z) = g(z)$.
\example
Докажем, что $\sin^2 z + \cos^2 z = 1$.
\\
Пусть $f(z) = \sin^2 z + \cos^2 z - 1$. Она регулярна в $\CC$.
\\
Известно, что $\forall z \in \RR \ f(z) = 0$, а значит, по теореме
единственности и $\forall z \in \CC \ f(z) = 0$.
\theorem (Морера)
Пусть $f: G \mapsto \CC$ непрервывна в $G$. Пусть для любого замкнутого
треугольника из $G$
\begin{align*}
  \int_{\triangle}f(z) dz = 0
\end{align*}
Тогда $f$ регулярна на $G$.
\pr
Фиксируем $z_0 \in G$. Пусть $\exists B_r(z_0)\subseteq G$.
\\
Внутри $B_r(z_0)$ рассмотрим
\begin{align*}
  g(z) = \int_{\gamma_{z_0 z}}f(\zeta) d \zeta
\end{align*}
причем $\gamma_{z_0z}\subseteq B_{r}(z_0)$.
\\
Аналогично теореме $4$ $\S 6$ $\forall z \in B_r(z_0) \ \exists g'(z) = f(z)$,
т.~е. $g(z)$ регулярна в $B_r(z_0)$.
\\
Из теоремы $3$ $\S8$ $g(z)$ бесконечно дифференциуема. Значит, и $f(z)$
бесконечно дифференцируема, а значит, и регулярна в $B_r(z_0)$.
\\
В силу произвольности $z_0$ $f(z)$ регулярна во всей $G$.
\theorem (о стирании разреза)
Пусть $G$~--- односвязная область, которая интервалом $(A;B)$ разрезана на $2$
односвязные подобласти, т.~е. $G = G_1 \cup G_2 \cup (A;B)$, $G_1 \cap G_2 =
\varnothing$.
\\
Пусть $f_1: G_1 \cup (A;B) \mapsto \CC$ регулярна на $G_1$ и непрерывна на $G_1
\cup (A;B)$; $f_2: G_2 \cup (A;B) \mapsto \CC$ регулярна на $G_2$ и непрерывна
на $G_2 \cup (A;B)$; $\forall z \in (A;B) \ f_1(z) = f_2(z)$.
\\
Тогда
\begin{align*}
  f(z) = \left\{ \begin{matrix}
          f_1(z), \ z \in G_1 \cup (A;B) \\
          f_2(z), \ z \in G_2
      \end{matrix} \right.
\end{align*}
регулярна на $G$.
\pr
Рассмотрим произвольный $\triangle CDE$. Если $\triangle CDE \subseteq G_1 \cup
(A;B)$, то по теореме $2$ $\S7$ следует
\begin{align*}
  \int_{\partial \triangle CDE}f dz = 0
\end{align*}
Аналогично для $\triangle CDE \subseteq G_2 \cup (A;B)$.
\\
Если же $(\triangle CDE \cap G_1 \neq \varnothing)\wedge (\triangle CDE \cap G_2
\neq \varnothing)$,  то пусть $\triangle CDE \cap (A;B)  = (P;Q)$.
\begin{align*}
  \int_{\partial \triangle CDE}f dz =   \int_{\partial \triangle PEQ}f_1 dz +   \int_{\partial PQDC}f_2 dz = 0
\end{align*}
Значит, по теореме Морера $f(z)$ регулярна.
