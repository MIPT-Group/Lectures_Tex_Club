\section{Линейные нормированные пространства}

\begin{definition}
	\textit{Линейным нормированным пространством} называется пара из линейного пространства $E$ над полем $\Kk$, где $\Kk = \R$ или $\Kk = \Cm$, и функции $\|\cdot\| : E \to \R$, обладающей следующими свойствами:
	\begin{enumerate}
		\item $\forall x \in E: \|x\| \ge 0$, причем $\|x\| = 0 \lra x = 0$
		\item $\forall x \in E: \forall \alpha \in \Kk: \|\alpha x\| = |\alpha|\|x\|$
		\item $\forall x, y \in E: \|x + y\| = \|x\| + \|y\|$ (\textit{неравенство треугольника})
	\end{enumerate}

	Функция $\|\cdot\|$ называется \textit{нормой} на пространстве $E$.
\end{definition}

\begin{example}
    Легко видеть, что любое линейное нормированное пространство $E$ является метрическим с метрикой, заданной для произвольных $x, y \in E$ как $\rho(x, y) := \|x - y\|$. Приведенные ранее в качестве примеров метрические пространства являются линейными нормированными, и метрика в них имеет вид нормы разности элементов.
\end{example}

