\section{Линейные нормированные пространства}

\begin{definition}
	\textit{Линейным нормированным пространством} называется пара из линейного пространства $E$ над полем $\Kk$, где $\Kk = \R$ или $\Kk = \Cm$, и функции $\|\cdot\| : E \to \R$, обладающей следующими свойствами:
	\begin{enumerate}
		\item $\forall x \in E: \|x\| \ge 0$, причем $\|x\| = 0 \lra x = 0$
		\item $\forall x \in E: \forall \alpha \in \Kk: \|\alpha x\| = |\alpha|\|x\|$
		\item $\forall x, y \in E: \|x + y\| = \|x\| + \|y\|$ (\textit{неравенство треугольника})
	\end{enumerate}

	Функция $\|\cdot\|$ называется \textit{нормой} на пространстве $E$.
\end{definition}

\begin{example}
    Легко видеть, что любое линейное нормированное пространство $E$ является метрическим с метрикой, заданной для произвольных $x, y \in E$ как $\rho(x, y) := \|x - y\|$. Приведенные ранее в качестве примеров метрические пространства являются линейными нормированными, и метрика в них имеет вид нормы разности элементов.
\end{example}

\begin{note}
	Норма непрерывна как функция $E \to \R$. Если для последовательности $\{x_n\} \subset E$ выполнено $x_n \to_E x$, то $\rho(x_n, x) = \|x_n - x\| \to 0$. Дважды воспользуемся неравенством треугольника:
	\begin{itemize}
		\item $\|x_n\| \le \|x_n - x\| + \|x\|$
		\item $\|x\| \le \|x_n - x\| + \|x_n\|$
	\end{itemize}

	Таким образом, $\big|\|x_n\| - \|x\|\big| \to 0$, то есть $\|x_n\| \to \|x\|$, что и требовалось.
\end{note}

\begin{definition}
	Пусть $E$ "--- линейное нормированное пространство. Множество $L \subset E$ называется:
	\begin{itemize}
		\item \textit{Линейным многообразием} в $E$, если $L$ замкнуто относительно сложения и умножения на скаляры из $\Kk$
  		\item \textit{Подпространством} в $E$, если $L$ является линейным многообразием в $E$ и при этом замкнуто 
	\end{itemize}
\end{definition}

\begin{definition}
	Пусть $E$ "--- линейное нормированное пространство, $M \subset E$. \textit{Линейной оболочкой} множества $M$ называется множсетво следующего вида:
	\[[M] := \left\{\sum_{k = 1}^n \alpha_im_i : \alpha_1, \dotsc \alpha_n \in \Kk, m_1, \dotsc, m_n \in M\right\}\]
\end{definition}

\begin{note}
	Линейная оболочка множества всегда является линейным многообразием в $E$, но не всегда является подпространством.
\end{note}

\begin{definition}
	Пусть $E$ "--- линейное пространство. Нормы $\|\cdot\|$ и $\|\cdot\|'$ на $E$ называются \textit{эквивалентными}, если существуют числа $C_1, C_2 > 0$ такие, что для любого $x \in E$ выполнены неравенства:
	\[C_1\|x\|' < \|x\| < C_2\|x\|'\]
\end{definition}

\begin{note}
	Если нормы $\|\cdot\|$ и $\|cdot\|'$ эквивалентны, то последовательность $\{x_n\} \subset E$ сходится или расходится относительно обеих норм одновременно.
\end{note}

\begin{definition}
	Пусть $E$ "--- линейное пространство. \textit{Размерностью} пространства $E$ называется наибольший размер линейно независимой системы в $E$. Обозначение "--- $\dim E$.
\end{definition}

\begin{proposition}\label{normequiv}
	Пусть $E$ "--- линейное пространство, и $\dim E < +\infty$. Тогда любые две нормы на $E$ эквивалентны.
\end{proposition}

\begin{proof}[Доказательство для случая, когда {$\Kk = \R$}]
	Поскольку $E$ конечномерно, то в нем можно выбрать максимальную по включению линейно независимую систему $e = \{e_1, \dotsc, e_n\} \subset E$, тогда $e$ будет являться базисом в $E$. Для произвольного элемента $x \in E$, имеющего в базисе $e$ координатный столбец $\alpha \in \R^n$, зададим его \textit{евклидову норму} следующим образом:
	\[\|x\|_{euc} := \sqrt{\sum_{k = 1}^n \alpha_k^2}\]

	Зафиксируем произвольную норму $\|\cdot\|$ и докажем, что она эквивалентна норме $\|\cdot\|_{euc}$
	\begin{enumerate}
		\item Покажем, что $\|\cdot\| < C \|\cdot\|_{euc}$ для некоторого $C > 0$. Для произвольного $x \in E$, имеющего в базисе $e$ координатный столбец $\alpha \in \R^n$, выполнено следующее:
		\[\|x\| = \left\|\sum_{k = 1}^n\alpha_k e_k \right\| \le \max_{1 \le k \le n}\|e_k\|\left(\sum_{k = 1}^n|\alpha_k|\right)\]
  
		Поскольку для любого $k \in \{1, \dotsc, n\}$ выполнено $|\alpha_k| < \|x\|_{euc}$, то достаточно взять число $C := n(\max_{1 \le k \le n}\|e_k\|)$.

		\item Покажем теперь, что $\|\cdot\|_{euc} < \widetilde C \|\cdot\|$ для некоторого $\widetilde C > 0$. Предположим противное, тогда для любого $n \in \N$ существует $x_n \in E$ такое, что $\|x\|_{euc} > n\|x\|$. Можно без ограничения общности считать, что $\|x_n\|_{euc} = 1$ для любого $n \in \N$, откуда $\|x\| < \frac 1n$.
		
		Поскольку последовательность $\{x_n\}$ содержится в единичной сфере $S_{euc}(0, 1)$ и относительно евклидовой нормы сфера компактна, можно выделить из $\{x_n\}$ подпоследовательность $\{x_{n_k}\}$, сходящуюся относительно евклидовой нормы. Тогда $\|x_{n_k} - x\|_{euc} \to 0$ для некоторого $x \in S_{euc}(0, 1)$, тогда, в силу пункта $(1)$, $\|x_{n_k} - x\| \to 0$. Но по построению $\|x_{n_k}\| < \frac 1{n_k}\to 0$, поэтому $x = 0$ --- противоречие с тем, что $x \in S_{euc}(0, 1)$.\qedhere
	\end{enumerate}
\end{proof}

\begin{corollary}
	Пусит $E$ "--- линейное нормированное пространство, $x_1, \dotsc, x_n$. Тогда линейная оболочка $L := [x_1, \dotsc, x_n]$ образует подпространство в $E$.
\end{corollary}

\begin{proof}
	Заметим, что $\dim L < +\infty$, и по утверждению \ref{normequiv} сужение нормы из $E$ на $L$ эквивалентно евклидовой норме. Относительно евклидовой нормы конечномерное пространство полно, поэтому и $L$ полно относительно нормы из $E$. Следовательно, $L$ замкнуто как подмножество в $E$.
\end{proof}

\begin{proposition}[лемма о <<почти перпендикуляре>>]\label{almostperp}
	Пусть $E$ "--- линейное нормированное пространство, $L \subsetneq E$ "--- подпространство. Тогда для любого $\epsilon > 0$ существует $y \in E$ такой, что $\|y\| = 1$ и $\rho(y, L) = \sup_{z \in L}\|y - z\| \ge 1 - \epsilon$.
\end{proposition}

\begin{proof}
		Зафиксируем $y_0 \in E \bs L$ положим $d := \rho(y, L) > 0$. Выберем вектор $z_0 \in L$ такой, что $d \le \|y_0 - z_0\| \le d(1 + \epsilon)$ и покажем, что подходит вектор $y := \frac{y_0 - z_0}{\|y_0 - z_0\|}$. Действительно, $\|y\| = 1$, и для любого $z \in L$ выполнены неравенства:
		\[\|y - z\| = \frac1{\|y_0 - z_0\|}\big\|y_0 - (z_0 + \|y_0 - z_0\|z)\big\| \ge \frac{d}{\|y_0 - z_0\|} \ge \frac1{1-\epsilon} \ge 1 + \epsilon\]

		Получено требуемое.
\end{proof}

\begin{theorem}[Рисса]\label{thm4.1}
	Пусть $E$ "--- линейное нормированное пространство. Тогда единичная сфера $S(0, 1)$ компактна в $E$ $\lra$ $\dim E < +\infty$.
\end{theorem}

\begin{proof}~
	\begin{itemize}
		\item[$\la$] По утверждению \ref{normequiv}, все нормы на конечномерном линейном пространстве эквивалентны, а относительно евклидовой нормы сфера $S(0, 1)$ компактна.
  
  		\item[$\ra$] Зафиксируем $\epsilon > 0$ и построим последовательность $\{x_n\} \subset S(0, 1)$ с попарными расстояниями не меньше $1 - \epsilon$, из чего будет следовать, что сфера $S(0, 1)$ не вполне ограниченна и потому не компактна:
		\begin{itemize}
			\item[$\bullet$] Выберем $x_1 \in S(0, 1)$ произвольным образом
			\item[$\bullet$] По утверждению \ref{almostperp} выберем $x_2 \in S(0, 1) \bs [x_1]$ такое, что $\rho(x_2, [x_1]) \ge 1- \epsilon$
			\item[$\bullet$] По утверждению \ref{almostperp} выберем $x_3 \in S(0, 1) \bs [x_1, x_2]$ такое, что $\rho(x_3, [x_1, x_2]) \ge 1- \epsilon$
		\end{itemize}

		Поскольку $\dim E = + \infty$, то процесс не закончится, и будет получена искомая последовательность $\{x_n\}$.\qedhere
	\end{itemize}
\end{proof}

\begin{definition}
	\textit{Евклидовым пространством} называется пара из линейного пространства $E$ над полем $\Kk$, где $\Kk = \R$ или $\Kk = \Cm$, и функции $(\cdot, \cdot) : E^2 \to \Kk$, обладающей следующими свойствами:
	\begin{enumerate}
		\item $\forall x \in E: (x, x) \ge 0$, причем $(x, x) = 0 \lra x = 0$
		\item $\forall x, y \in E: (x, y) = \overline{(y, x)}$
		\item $\forall x, y, z \in E: \forall \alpha, \beta \in \Kk (\alpha x + \beta y, z) = \alpha(x, z) + \beta(y, z)$
	\end{enumerate}

	Функция $(\cdot, \cdot)$ называется \textit{скалярным произведением} на пространстве $E$.
\end{definition}

\begin{example}
	Легко видеть, что любое евклидово пространство $E$ является линейным нормированным с нормой, заданной для произвольного $x \in E$ как $\|x\| := \sqrt{(x, x)}$. Некоторые из приведенных ранее в качестве примеров метрических пространств являются евклидовыми, и метрика в них порождается скалярным произведением:
	\begin{enumerate}
		\item Скалярное произведение на $\R^n_2$ задается для произвольных $x, y \in \R^n$ следующим образом:
  		\[(x, y) = \sum_{k = 1}^n x_ky_k\]

		\item Скалярное произведение на $l_2$ задается для произвольных $x, y \in l_2$ следующим образом:
		\[(x, y) = \sum_{n = 1}^\infty x_ny_n\]

		\item Скалярное произведение на $L_2[a, b]$ задается для произвольных $f, g \in L_2[a, b]$ следующим образом:
		\[(f, g) = \int_a^b f(x)g(x)dx\]
	\end{enumerate}
\end{example}

\begin{definition}~
	\begin{itemize}
		\item Полное линейное нормированное пространство $E$ называется \textit{банаховым}.
		\item Полное евклидово пространство $H$ называется \textit{гильбертовым}.
	\end{itemize}
\end{definition}