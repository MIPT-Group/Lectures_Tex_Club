\subsection{Евклидовы кольца и деление с остатком	}

\begin{definition}
	Область целостности $K$ называется \textit{евклидовым кольцом}, если определена функция $N: K\backslash\{0\} \to \N \cup \{0\}$, называемая \textit{нормой}, удовлетворяющая условиям:
	\begin{enumerate}
		\item $\forall a, b \in K\backslash\{0\}: N(ab) \ge N(a)$
		\item $\forall a, b \in K\backslash\{0\}: \exists q, r \in K: a = bq + r$, причем $r = 0$ или $N(r) < N(b)$ ($q$ называется \textit{частным}, а $r$ "--- \textit{остатком} при делении $a$ на $b$)
	\end{enumerate}
\end{definition}

\begin{note}
	Первое свойство в определении выше не является обязательным. Кроме того, часто его заменяют на неэквивалентное свойство $\forall a, b \in K \backslash \{0\}: N(ab) = N(a)N(b)$.
\end{note}

\begin{note}
	Любое поле является евклидовым кольцом: норму в поле можно положить тождественно равной 0 или 1, поскольку при делении в поле остаток всегда равен $0$.
\end{note}

\begin{example}
	Рассмотрим несколько примеров евклидовых колец:
	\begin{enumerate}
		\item $\Z$ "--- евклидово кольцо с нормой $\forall z \in \Z: N(z) = |z|$
		\item Если $F$ "--- поле, то $F[x]$ "--- евклидово кольцо с нормой $\forall p \in F[x]: N(p) = \deg{p}$ (причем если положить норму равной $2^{\deg{p}}$, выполняться будет даже более сильное мультипликативное условие)
		\item $\Z[i]$ "--- евклидово кольцо с нормой $\forall z \in \Z[i]: N(z) = |z|$
	\end{enumerate}
\end{example}