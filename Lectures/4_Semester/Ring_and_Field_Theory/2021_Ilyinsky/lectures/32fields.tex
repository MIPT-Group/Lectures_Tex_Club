\subsection{Поля и расширения}

\begin{definition}
	\textit{Полем} называется коммутативное кольцо $F$ такое, что $F^* = F \backslash \{0\}$.
\end{definition}

\begin{note}
	Будем считать целыми числами в поле $F$ суммы соответствующего числа единиц или минус единиц, и обозначать их через $\overline{n} \in F$.
\end{note}

\pagebreak

\begin{definition}
	\textit{Характеристикой} поля $F$ называется минимальное число $n \in \N$ такое, что $\overline{n} = 0$, если оно существует, и $0$ в противном случае. Обозначение "--- $\cha{F}$.
\end{definition}

\begin{proposition}
	Пусть $F$ "--- поле. Если $\cha{F} > 0$, то $\cha{F}$ "--- простое число.
\end{proposition}

\begin{proof}
	Предположим, что $\cha{F} = ab$, где $a, b > 1$. Тогда $\overline{a}\overline{b} = \overline{ab} = 0$, поэтому $\overline{a}, \overline{b}$ необратимы --- противоречие.
\end{proof}

\begin{theorem}[о простом подполе]
	Пусть $F$ "--- поле, тогда:
	\begin{itemize}
		\item Если $\cha{F} = p$ "--- простое число, то $F$ содержит подполе, изоморфное $\Z_p$
		\item Если $\cha{F} = 0$, то $F$ содержит подполе, изоморфное $\Q$
	\end{itemize}
\end{theorem}

\begin{proof}
	Заметим, что имеют место следующие мономорфизмы из $\Z_p$ и $\Q$ в $F$ соответственно:
	\begin{itemize}
		\item $a \mapsto \overline{a}$, если $\cha{F} = p$
		\item $\frac{a}{b} \mapsto \overline{a}\overline{b}^{-1}$, если $\cha{F} = 0$\qedhere
	\end{itemize}
\end{proof}

\begin{proposition}
	Пусть $\phi: F \to L$ "--- нетривиальный гомоморфизм полей. Тогда $\cha{F} = \cha{L}$ и гомоморфизм $\phi$ инъективен.
\end{proposition}

\begin{proof}
	Поскольку $\phi(1) = 1$, то $\forall a \in \Z: \overline{a}_F \mapsto \overline{a}_L$, поэтому характеристики полей $F$ и $L$ совпадают. Кроме того, $\ke\phi$ "--- идеал в поле $F$, поэтому $\ke\phi = \{0\}$ в силу нетривиальности гомоморфизма $\phi$. Значит, $\phi$ инъективен.
\end{proof}

\begin{definition}
	Пусть $F, L$ "--- поля, причем $F \subset L$. Тогда $L$ называется \textit{расширением поля} $F$. \textit{Степенью расширения} $L$ называется размерность расширения $L$ как линейного пространства над $F$. Обозначение "--- $[L : F]$.  Расширение $L$ называется \textit{конечным}, если его степень конечна, и \textit{бесконечным} в противном случае.
\end{definition}

\begin{example}
	Рассмотрим следующие примеры расширений полей:
	\begin{itemize}
		\item $\R \supset \Q$ "--- бесконечное расширение
		\item $\Cm \supset \R$ "--- конечное расширение с базисом $(1, i)$
		\item $F(x) = \quot F[x] \supset F$, где $F$ "--- поле, "--- бесконечное расширение, поскольку в нем есть линейно независимая система $(1, x, x^2, \dotsc)$
	\end{itemize}
\end{example}

\begin{proposition}
	Пусть $F$ "--- поле, $f \in F[x]$ "--- неприводимый многочлен, $L = F[x] / (f)$. Тогда $[L : F] = \deg{f}$.
\end{proposition}

\begin{proof}
	Положим $n := \deg{f}$ и докажем, что $([1], [x], \dotsc, [x^{n - 1}])$ "--- базис в $L$. Эта система порождает $L$, поскольку у любого класса в $L$ есть представитель со степенью меньше $n$. Проверим, что система линейно независима. Пусть $\lambda_0[1] + \dotsb + \lambda_{n - 1}[x^{n - 1}] = [0]$ для некоторых $\lambda_0, \dotsc, \lambda_{n - 1} \in F$. Тогда $[\lambda_0 + \dotsb + \lambda_{n - 1}x^{n - 1}] = [0] \lra f \mid \lambda_0 + \dotsb + \lambda_{n - 1}x^{n - 1}$, поэтому $\lambda_0 = \dotsb = \lambda_{n - 1} = 0$.
\end{proof}

\begin{definition}
	Пусть $F, L$ "--- поля, причем $F \subset L$. \textit{Расширением поля $F$ элементами} $\alpha_1, \dotsc, \alpha_n \in L \backslash F$ называется наименьшее по включению подполе в $L$, содержащее $F$ и все элементы $\alpha_1, \dotsc, \alpha_n$. Обозначение "--- $F(\alpha_1, \dotsc, \alpha_n)$.
\end{definition}

\begin{note}
	Легко проверить, что $F(\alpha_1, \dotsc, \alpha_n)$ имеет следующий вид:
	\[F(\alpha_1, \dotsc, \alpha_n) = \left\{\frac{f(\alpha_1, \dotsc, \alpha_n)}{g(\alpha_1, \dotsc, \alpha_n)}: f, g \in F[x_1, \dotsc, x_n], g(\alpha_1, \dotsc, \alpha_n) \ne 0\right\}\]
	
	С одной стороны, множество выше "--- это поле. С другой стороны, любое поле, содержащее $F$ и $\alpha_1, \dotsc, \alpha_n$, обязано содержать его. Из этого, в частности, следует, что $F(\alpha_1, \dotsc, \alpha_n) \cong \quot(F[\alpha_1, \dotsc, \alpha_n])$.
\end{note}

Теперь \textbf{до конца раздела} зафиксируем поле $F$ и его расширение $L$.

\begin{definition}
	Элемент $\alpha \in L$ называется \textit{алгебраическим над $F$}, если существует нетривиальный многочлен $f \in F[x] \backslash \{0\}$ такой, что $f(\alpha) = 0$. В противном случае элемент $\alpha$ называется \textit{трансцендентным над $F$}.
\end{definition}

\begin{proposition}
	Если $\alpha \in L$ "--- трансцендентный над $F$, то расширение $F(\alpha)$ бесконечно, и в этом случае $F(\alpha) \cong F(x)$.
\end{proposition}

\begin{proof}
	В силу трансцендентности, в $F(\alpha)$ есть линейно независимая система $(1, \alpha, \alpha^2, \dotsc)$, поэтому расширение бесконечно. Очевидно также, что отображение $x \mapsto \alpha$ осуществляет изоморфизм полей $F(x)$ и $F(\alpha)$.
\end{proof}

\begin{example}
	Рассмотрим поле $\Q(\sqrt{2})$. Избавляясь от иррациональности в знаменателе, получим, что $\Q(\sqrt{2}) = \Q[\sqrt{2}]$. Значит, $[\Q(\sqrt{2}) : \Q] = 2$, и $(1, \sqrt{2})$ "--- базис в $\Q(\sqrt{2})$. Далее мы будем обобщать это рассуждение.
\end{example}

\begin{definition}
	Пусть $\alpha \in L$ "--- алгебраический над $F$. \textit{Минимальным многочленом} элемента $\alpha$ называется многочлен $m_\alpha \in F[x]$ со старшим коэффициентом, равным 1, такой, что идеал $\{f \in F[x]: f(\alpha) = 0\}$ порождается многочленом $m_\alpha$.
\end{definition}

\begin{note}
	Минимальный многочлен определен корректно, так как $F[x]$ "--- евклидово кольцо и, в частности, кольцо главных идеалов. Идеал в определении выше прост, поскольку $\forall f, g \in F[x]: fg \in I \ra f(\alpha)g(\alpha) = 0 \ra f(\alpha) = 0$ или $g(\alpha) = 0$. Значит, минимальный многочлен неразложим и прост над $F[x]$.
\end{note}

\begin{proposition}
	Пусть $\alpha \in L$ "--- алгебраический над $F$, $m_\alpha \in F[x]$ "--- многочлен со старшим коэффициентом, равным 1. Тогда следующие условия эквивалентны:
	\begin{enumerate}
		\item $m_\alpha$ "--- минимальный многочлен элемента $\alpha$
		\item $m_\alpha$ "--- многочлен минимальной степени такой, что $m_\alpha(\alpha) = 0$
		\item $m_\alpha$ "--- неприводимый многочлен такой, что $m_\alpha(\alpha) = 0$
	\end{enumerate}
\end{proposition}

\begin{proof}~
	\begin{itemize}
		\item\imp{1}{2}Достаточно заметить, что $\forall g \in (m_\alpha): m_\alpha \mid g \ra \deg{g} \ge m_\alpha$, поэтому степень многочлена $m_\alpha$ минимальна.
		\item\imp{2}{3}Если $m_\alpha = fg$ для некоторых $f, g \in F[x]$, то $m_\alpha(\alpha) = 0 \ra f(\alpha)g(\alpha) = 0 \hm\ra f(\alpha) = 0$ или $g(\alpha) = 0$, и в обоих случаях один из многочленов ассоциирован с $m_\alpha$, а второй обратим.
		\item\imp{3}{1}Если идеал минимального многочлена порождается некоторым $f \in F[x]$, то $f \sim m_\alpha$ в силу неприводимости, поэтому $m_\alpha$ также порождает данный идеал.\qedhere
	\end{itemize}
\end{proof}

\begin{theorem}
	Элемент $\alpha \in L$ "--- алгебраический над $F$ $\lra$ расширение $F(\alpha)$ конечно, и в этом случае $F(\alpha) = F[\alpha] \cong F[x]/(m_\alpha)$.
\end{theorem}

\begin{proof}~
	\begin{itemize}
		\item[$\ra$] Зафиксируем $m_\alpha$ "--- минимальный многочлен элемента $\alpha$, и рассмотрим гомоморфизм $\phi: F[x] \to F[\alpha]$ такой, что $\forall f \in F[x]: f \mapsto f(\alpha)$. Тогда $\phi$ сюръективен и $\ke\phi = (m_\alpha)$, значит, по теореме о гомоморфизме, $F[\alpha] \cong F[x] / (m_\alpha)$. Следовательно, $F[\alpha]$ "--- поле, поэтому $F[\alpha] = \quot F[\alpha] = F(\alpha)$ и $[F(\alpha) : F] = \deg{m_\alpha}$.
		
		\item[$\la$] В силу конечности расширения $F(\alpha)$, система $(1, \alpha, \alpha^2, \dotsc)$ линейно зависима, поэтому существует такой нетривиальный многочлен $f \in F[x] \backslash \{0\}$, что $f(\alpha) = 0$.\qedhere
	\end{itemize}
\end{proof}

\begin{note}
	Легко проверить по индукции, что если элементы $\alpha_1, \dotsc, \alpha_n \in L$ "--- алгебраические, то $F(\alpha_1, \dotsc, \alpha_n) = F[\alpha_1, \dotsc, \alpha_n]$.
\end{note}

\begin{note}
	Таким образом, мы полностью описали все возможные расширения поля $F$ одним элементом $\alpha \in L$:
	\begin{enumerate}
		\item Если $\alpha$ "--- трансцендентный над $F$, то $F(\alpha)$ "--- бесконечное расширение, и в этом случае $F(\alpha) \cong F(x)$.
		\item Если $\alpha$ "--- алгебраический над $F$, то $F(\alpha)$ "--- конечное расширение, и в этом случае $F(\alpha) = F[\alpha]  \cong F[x] / (m_\alpha)$, $[F(\alpha) : F] = \deg{m_\alpha}$, где $m_\alpha$ "--- минимальный многочлен элемента $\alpha$.
	\end{enumerate}
\end{note}

\begin{proposition}
	Если $f \in F[x]$ "--- неприводимый многочлен, то в поле $F[x] / (f) \supset F$ у $f$ есть корень.
\end{proposition}

\begin{proof}
	Рассмотрим канонический эпиморфизм $\pi: F[x] \to F[x] / (f)$. Тогда $f([x]) = [f(x)] = [0]$.
\end{proof}

\begin{corollary}
	Для любого неприводимого многочлена $f \in F[x]$ расширение $F[x] / (f)$ соответствует расширению поля $F$ некоторым алгебраическим элементом.
\end{corollary}