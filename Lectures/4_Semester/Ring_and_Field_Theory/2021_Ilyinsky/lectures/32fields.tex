\subsection{Поля}

\begin{definition}
	\textit{Полем} называется коммутативное кольцо $F$ такое, что $F^* = F \backslash \{0\}$.
\end{definition}

\begin{note}
	Будем считать целыми числами в поле $F$ суммы соответствующего числа единиц и минус единиц и обозначать их через $\overline{n} \in F$.
\end{note}

\pagebreak

\begin{definition}
	\textit{Характеристикой} поля $F$ называется минимальное число $n \in \N$ такое, что $\overline{n} = 0\}$, если оно существует, и $0$ в противном случае. Обозначение "--- $\cha{F}$.
\end{definition}

\begin{proposition}
	Пусть $F$ "--- поле. Если $\cha{F} > 0$, то $\cha{F}$ "--- простое число.
\end{proposition}

\begin{proof}
	Предположим, что $\cha{F} = ab$, где $a, b > 1$. Тогда $\overline{a}\overline{b} = \overline{ab} = 0$, поэтому $\overline{a}, \overline{b}$ необратимы --- противоречие.
\end{proof}

\begin{theorem}[о простом подполе]
	Пусть $F$ "--- поле, тогда:
	\begin{itemize}
		\item Если $\cha{F} = p$ "--- простое число, то $F$ содержит подполе, изоморфное $\Z / p\Z$
		\item Если $\cha{F} = 0$, то $F$ содержит подполе, изоморфное $\Q$
	\end{itemize}
\end{theorem}

\begin{proof}
	Заметим, что имеют место следующие мономорфизмы из $\Z / p\Z$ и $\Q$ в $F$ соответственно:
	\begin{itemize}
		\item $a \mapsto \overline{a}$, если $\cha{F} = p$
		\item $\frac{a}{b} \mapsto \overline{a}\overline{b}^{-1}$, если $\cha{F} = 0$\qedhere
	\end{itemize}
\end{proof}

\begin{proposition}
	Пусть $\phi: F \to L$ "--- нетривиальный гомоморфизм полей. Тогда $\cha{F} = \cha{L}$ и гомоморфизм $\phi$ инъективен.
\end{proposition}

\begin{proof}
	Поскольку $\phi(1) = 1$, то $\forall a \in \Z: \overline{a}_F \mapsto \overline{a}_L$, поэтому характеристики полей $F$ и $L$ совпадают. Кроме того, $\ke\phi$ "--- идеал в поле $F$, поэтому $\ke\phi = \{0\}$ в силу нетривиальности гомоморфизма $\phi$. Значит, $\phi$ инъективен.
\end{proof}