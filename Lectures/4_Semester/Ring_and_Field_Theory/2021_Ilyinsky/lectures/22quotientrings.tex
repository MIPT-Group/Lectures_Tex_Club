\subsection{Факторкольца, простые и максимальные идеалы}

\begin{definition}
	Пусть $K$ "--- коммутативное кольцо, $I \subset K$ "--- его идеал. \textit{Факторкольцом} $K / I$ называется абелева факторгруппа $K / I$ с операцией умножения, введенной следующим образом: $\forall a, b \in K: [a][b] := [ab]$.
\end{definition}

\begin{proposition}
	Умножение в факторкольце $K / I$ определено корректно.
\end{proposition}

\begin{proof}
	Пусть $a' \in [a], b' \in [b]$, тогда $a' = a + x$, $b' = b + y$ для некоторых $x, y \in I$. Значит, $a'b' = ab + bx + ay + xy \in [ab]$, поскольку $I$ "--- идеал.
\end{proof}

\begin{definition}
	\textit{Гомоморфизмом} колец $K, L$ называется отображение $\phi: K \to L$ такое, что:
	\begin{enumerate}
		\item $\forall a, b \in K: \phi(a + b) = \phi(a) + \phi(b)$
		\item $\forall a, b \in K: \phi(ab) = \phi(a)\phi(b)$
	\end{enumerate}

	\textit{Образом} гомоморфизма называется $\im\phi := \phi(K)$, \textit{ядром} --- $\ke\phi := \phi^{-1}(0)$.
\end{definition}

\begin{note}
	Если $K, L$ "--- кольца с единицей, то иногда от гомоморфизма $\phi$ также требуют, чтобы выполнялось равенство $\phi(1) = 1$.
\end{note}

\begin{theorem}[Основная теорема о гомоморфизме]
	Пусть $\phi : K \to L$ "--- гомоморфизм коммутативных колец. Тогда $\ke\phi$ "--- идеал в $K$, причем $\im\phi \cong K / \ke{\phi}$.
\end{theorem}

\begin{proof}
	Если $a, b \in \ke\phi$, то $\phi(a + b) = \phi(a) + \phi(b) = 0$ и $\forall c \in K: \phi(ac) \hm= \phi(a)\phi(c) = 0$, поэтому $\ke\phi$ "--- идеал. Зададим отображение $\psi: K/\ke\phi \to \im\phi$ следующим образом: $[x] \mapsto \phi(x)$. Легко проверить, что $\psi$ определено корректно и действительно является изоморфизмом.
\end{proof}

\begin{proposition}
	Если $K$ "--- коммутативное кольцо, $J$ "--- идеал в $K$, а $I$ "--- идеал в $J$, то $K / J \cong (K / I) / (J / I)$.
\end{proposition}

\begin{proof}
	Рассмотрим гомоморфизм $\phi: K / I \to K / J$, заданный следующим образом: $\forall a + I \in K: \phi(a + I) = a + J$. Поскольку $I \subset J$, то определение корректно, тогда, по основной теореме о гомоморфизме, $K / J = \im\phi \cong (K / I) / \ke\phi = (K / I) / (J / I)$.
\end{proof}

\begin{example}
	Пусть $K$ "--- коммутативное кольцо. Рассмотрим гомоморфизм $\phi: K[x] \to K$ такой, что $\forall p \in K[x]: \phi(p) = p(0)$. Тогда $K = \im\phi \cong K[x] / \ke\phi = K[x] / (x)$. Более того, если $K = \Z$, то, например, $\Z[x] / (x, 2) \cong (\Z[x] / (x)) / ((x, 2) / (2)) = \Z / (2) = \Z_2$.
\end{example}

\begin{proposition}
	Пусть $K$ "--- коммутативное кольцо, $I$ "--- идеал в $K$. Тогда $K / I$ является областью целостности $\lra$ $\forall a, b \in K: ab \in I \ra a \in I$ или $b \in I$.
\end{proposition}

\begin{proof}
	$K / I$ "--- область целостности $\lra$ $\forall [a], [b] \in K / I: [ab] = [0] \ra [a] = [0]$ или $[b] = [0]$ $\lra$ $\forall a, b \in K: ab \in I \ra a \in I$ или $b \in I$.
\end{proof}

\begin{note}
	Если в терминах предыдущего утверждения идеал $I$ "--- главный, $I = (x)$ для некоторого $x \in K$, то $\forall a, b \in K: ab \in I \ra a \in I$ или $b \in I$ $\lra$ $\forall a, b \in K: x \mid ab \ra x \mid a$ или $x \mid b$. Значит, $K / (x)$ "--- область целостности $\lra$ $x$ прост в $K$.
\end{note}

\begin{definition}
	Пусть $K$ "--- коммутативное кольцо. Идеал $I$ в $K$ называется \textit{простым}, если $\forall a, b \in K: ab \in I \ra a \in I$ или $b \in I$.
\end{definition}

\begin{note}
	Нетривиальный главный идеал $I$ "--- простой в $K$ $\lra$ элемент, порождающий $I$, прост в $K$.
\end{note}

\begin{example}
	Пусть $K$ "--- факториальное кольцо, $p$ "--- неразложимый элемент в $K$. Докажем, что $p$ также неразложим и прост в $K[x]$. Рассмотрим $K[x] / (p) \cong (K / (p))[x]$ и заметим, что $K / (p)$ "--- область целостности. Но тогда и $K[x] / (p) \cong (K / (p))[x]$ "--- область целостности, поэтому $p$ прост в $K[x]$.
\end{example}

\begin{proposition}
	Пусть $K$ "--- коммутативное кольцо. Тогда $K$ "--- поле $\lra$ в $K$ нет нетривиальных идеалов.
\end{proposition}

\begin{proof}~
	\begin{itemize}
		\item[$\ra$] Пусть $I \ne \{0\}$ "--- идеал в $K$. Поскольку $\exists a \in I$, $a \ne 0$, то $aa^{-1} = 1 \in I$ и $I = K$.
		\item[$\la$] Пусть $a \in K \backslash \{0\}$. Поскольку $(a) \ne \{0\}$, то $(a) = K$, и, в частности, $\exists b \in K: ab = 1$. Значит, $K^* = K \backslash \{0\}$, то есть $K$ "--- поле.\qedhere
	\end{itemize}
\end{proof}

\begin{note}
	Легко проверить, что если $\pi: K \to L$ "--- эпиморфизм коммутативных колец, то образ и прообраз идеала при действии $\pi$ тоже являются идеалами. Более того, если $I$ "--- идеал в $K$ и $\pi: K \to K / I$ "--- канонический эпиморфизм, то $\pi$ осуществляет биекцию между идеалами в $K$, содержащими $I$, и идеалами в $K / I$.
\end{note}

\begin{definition}
	Идеал $I$ в коммутативном кольце $K$ называется \textit{максимальным}, если в $K$ нет идеала $J$ такого, что $I \subsetneq J \subsetneq K$.
\end{definition}

\begin{proposition}
	Пусть $K$ "--- коммутативное кольцо, $I$ "--- идеал в $K$. Тогда $K / I$ является полем $\lra$ идеал $I$ максимален.
\end{proposition}

\begin{proof}
	Поскольку канонический эпиморфизм $\pi: K \to K/I$ осуществляет биекцию между идеалами в $K$, содержащими $I$, и идеалами в $K / I$, то $K / I$ "--- поле $\lra$ в $K / I$ нет нетривиальных идеалов $\lra$ в $K$ нет нетривиальных идеалов между $I$ и $K$ $\lra$ идеал $I$ максимален.
\end{proof}

\begin{corollary}
	Если идеал $I$ максимален в коммутативном кольце $K$, то $I$ прост.
\end{corollary}

\begin{proposition}
	Если $K$ "--- кольцо главных идеалов и $I$ "--- нетривиальный простой идеал в $K$, то $I$ максимален.
\end{proposition}

\begin{proof}
	Пусть существует такой идеал $J$, что $I \subsetneq J \subsetneq K$. Поскольку $K$ "--- кольцо главных идеалов, то $I = (x), J = (y)$, где $x, y \in K \backslash (\{0\} \cup K^*)$, причем $x$ "--- простой элемент. Тогда $y \mid x$, что противоречит неразложимости $x$ в $K$.
\end{proof}

\begin{corollary}
	Если $K$ "--- кольцо главных идеалов и $K / I$ не является полем, то $K / I$ также не является областью целостности.
\end{corollary}

\begin{example}
	Рассмотрим кольцо $\Z[x]$. Поскольку $\Z[x] / (x) \cong \Z$ и $\Z$ "--- область целостности, но не поле, то $\Z[x]$ не может быть кольцом главных идеалов.
\end{example}

\begin{corollary}
	Пусть $F$ "--- поле, $f \in F[x]$ "--- неприводимый многочлен. Тогда $F[x] / (f)$ является полем.
\end{corollary}

\begin{proof}
	Кольцо $F[x]$ евклидово, поэтому оно является кольцом главных идеалов. Элемент $f$ прост в $F[x]$, поэтому идеал $(f)$ прост и максимален в $F[x]$. Значит, $F[x] / (f)$ "--- поле.
\end{proof}