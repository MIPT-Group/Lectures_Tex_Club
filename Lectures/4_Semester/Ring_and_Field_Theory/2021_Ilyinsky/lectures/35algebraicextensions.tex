\subsection{Алгебраическое замыкание}

\begin{proposition}
	Пусть $F \subset L \subset K$ "--- башня расширений такая, что расширения $L \supset F$ и $K \supset L$ "--- алгебраические. Тогда $K \supset F$ также является алгебраическим расширением.
\end{proposition}

\begin{proof}
	Пусть $\alpha \in K$. По условию, существует нетривиальный многочлен $f \hm= a_nx^n + \dotsb + a_0 \in L[x] \backslash \{0\}$ такой, что $f(\alpha) = 0$, причем элементы $a_0, \dotsc, a_n$ являются алгебраическими над $F$. Расширения $F(a_0, \dotsc, a_n, \alpha) \supset F(a_0, \dotsc, a_n) \supset F$ конечны, поэтому $\alpha$ является алгебраическим над $F$.
\end{proof}

\begin{definition}
	Поле $K$ называется \textit{алгебраически замкнутым}, если выполнено одно из следующих условий:
	\begin{enumerate}
		\item Любое конечное расширение поля $K$ тривиально
		\item Любое алгебраическое расширение поля $K$ тривиально
		\item $\forall f \in K[x], \deg{f} \ge 1: f$ имеет корень в $K$
		\item $\forall f \in K[x], \deg{f} \ge 1: f$ раскладывается на линейные сомножители над $K[x]$
		\item Любой неприводимый многочлен $f \in K[x]$ имеет степень $1$
	\end{enumerate}
\end{definition}

\begin{proposition}
	Условия в определении алгебраически замкнутого поля эквивалентны.
\end{proposition}

\begin{proof}~
	\begin{itemize}
		\item\imp{2}{1}Тривиально, поскольку любое конечное расширение является алгебраическим.
		\item\imp{1}{4}Поле разложения многочлена $f$ является конечным расширением поля $K$, поэтому оно совпадает с $K$ и $f$ раскладывается на линейные сомножители над $K[x]$.
		\item\imp{4}{3}Тривиально, если многочлен $f$ степени $\ge 1$ раскладывается на линейные сомножители, то он имеет корень.
		\item\imp{3}{5}Если многочлен $f$ имеет корень $\alpha \in K$, то $x - \alpha \mid f$. Тогда, в силу неприводимости, $f \sim x - \alpha$.
		\item\imp{5}{2}Пусть $L \supset K$ "--- алгебраическое расширение, $\alpha \in L$. Рассмотрим минимальный многочлен $m_\alpha \in K[x]$ элемента $\alpha$ над $K$. Поскольку $m_\alpha$ неприводим, то $\deg{m_\alpha} = 1$, откуда $\alpha \in K$.\qedhere
	\end{itemize}
\end{proof}

\begin{definition}
	\textit{Алгебраическим замыканием} поля $F$ называется алгебраическое расширение $\overline{F} \supset F$, являющееся алгебраически замкнутым.
\end{definition}

\begin{proposition}
	Пусть $F$ "--- поле, $K \supset F$ "--- его алгебраически замкнутое расширение. Тогда множество $\overline{F} := \{\alpha \in K: \alpha\text{ "--- алгебраический над }F\}$ является алгебраическим замыканием поля $F$.
\end{proposition}

\begin{proof}
	По построению, все элементы в $\overline{F}$ "--- алгебраические над $F$. Проверим алгебраическую замкнутость множества $\overline{F}$. Пусть $L \supset \overline{F} \supset F$ "--- башня алгебраических расширений. Тогда, как уже было доказано, $L \supset F$ "--- алгебраическое расширение, поэтому выполнено равенство $L = \overline{F}$.
	
	\pagebreak 
	Проверим теперь, что $\overline{F}$ "--- поле. Действительно, если $\alpha, \beta \in \overline{F} \backslash \{0\}$, то расширение $F(\alpha, \beta) \supset F$ "--- алгебраическое, поэтому $\alpha + \beta, \alpha \cdot \beta, -\alpha, \alpha^{-1} \in F(\alpha, \beta) \subset \overline{F}$.
\end{proof}

\begin{proposition}
	Пусть $F$ "--- не более чем счетное поле, $f_1, f_2, \dotsc \in F[x]$ "--- все неприводимые над $F[x]$ многочлены. Рассмотрим следующую последовательность расширений:
	\begin{align*}
		&F_1 \supset F\text{ "--- поле разложения многочлена } f_1\\
		&F_2 \supset F_1\text{ "--- поле разложения многочлена } f_2 \text{ над } F_1\\
		&\dots
	\end{align*}
	
	Тогда множество $\overline{F} := \bigcup_{i = 1}^\infty F_i$ является алгебраическим замыканием поля $F$.
\end{proposition}

\begin{proof}
	Аналогично случаю идеалов, множество $\overline{F}$ является полем. Кроме того, все расширения $F_n \supset F$ являются алгебраическими, поэтому каждый элемент $\alpha \in L$ является алгебраическим над $F$.
	
	Проверим алгебраическую замкнутость поля $\overline{F}$. Пусть $L \supset \overline{F}$ "--- алгебраическое расширение. Тогда расширение $L \supset F$ "--- тоже алгебраическое. Рассмотрим элемент $\alpha \in L$ и его минимальный многочлен $m_\alpha \in F[x]$. Поскольку $m_\alpha$ неприводим над $F[x]$, то $\exists n \in \N: m_\alpha = f_n$, откуда $\alpha \in F_n$. Значит, $L = \overline{F}$.
\end{proof}

\begin{note}
	В случае, когда $F$ более чем счетно, аналогичное утверждение тоже верно, но его доказательство требует использования теоремы Цермело. Значит, алгебраическое замыкание поля $F$ всегда существует. Можно также показать, оно единственно с точностью изоморфизма.
\end{note}