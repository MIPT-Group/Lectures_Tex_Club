\documentclass[../main.tex]{subfiles}
\begin{document}

    \section{Термодинамические системы}
    \begin{definition}
        \emph{Термодинамическая система (ТС)}~--- система из большого количества частиц, окруженных термостатом (вообще всем остальным). 
    \end{definition}
    \begin{note}
        Вселенная не является ТС, так как нет ограничений (иначе, у неё нет термостата).
    \end{note}
    
    \begin{definition}
        \emph{Изолированная система}~--- система, которая вообще никак не взаимодействует с окружающей средой.
    \end{definition}
    
    \subsection{Минус-первое начало термодинамики}
        Это утверждение иногда называют нулевым началом термодинамики, но мы делать так не будем, ведь есть другое нулевое начало термодинамики. Так же, часто его называют \emph{основным} или \emph{общим}. В общем, знатный рецидивист.

    \begin{proposition}[Минус-первое начало термодинамики]
        Любая изолированная система придёт в равновесие, её можно охарактеризовать макроскопическими параметрами и она не может сама покинуть состояние равновесия.
    \end{proposition}
        
    \begin{note}
        Время релаксации зависит от размеров системы.
    \end{note}
    \begin{note}
        У некоторых объектов существует метастабильное состояние.
    \end{note}

    \subsection{Классификация систем}
        \begin{definition}
            Над \emph{адиабатической (или теплоизолированной) ТС} мы можем только совершать работу (не можем передавать тепло, изменять количество частиц).
        \end{definition}    
        \begin{definition}
            Система, к которой мы можем только подводить тепло, называется \emph{калорической} (по Колдунову).
        \end{definition}
        \begin{definition}
            Над \emph{закрытой системой} можно совершать работу и подводить к ней тепло, но нельзя менять состав частиц.
        \end{definition}
        \begin{definition}
            У \emph{открытой системы} можно изменять только количество частиц. 
        \end{definition}
        \begin{note}
            Комбинация открытой системы с другими даёт нам окончательную классификацию.
        \end{note}

        \begin{definition}
            Термодинамическая система с двумя каналами называется \emph{простой}.
        \end{definition}
       

        \section{Температура}
        \begin{definition}
            \emph{Экстенсивная величина} пропорциональна количеству частиц.
        \end{definition}
        \begin{note}
            Экстенсивные величины аддитивны. Их можно измерять по своим эталонам.
        \end{note}
        \begin{examples}
            Объём, количество частиц.
        \end{examples}

        \begin{definition}
            \emph{Интенсивная величина} пропорциональна количеству частиц.
        \end{definition}
        \begin{note}
            Для измерения интенсивных величин необходимо прибегать к сторонним параметрам и величинам.
        \end{note}
        \begin{examples}
            Температура, давление.
        \end{examples}

    \segment{Методы измерения температуры}
    Вообще, температуру можно измерять по расширению, сопротивлению и др. Первые попытки унификации измерения температуры состояли в определении \emph{реперных точек}, однако все такие точки не были стабильны (зависели от множества параметров, в том числе иногда и от температуры). В 1968 году договорились до эталонов. Температуру начали измерять с помощью газового термометра. Основная идея состоит в измерении температуры через давление, так и получили кельвины (а именно, абсолютный ноль). 

    Как только мы договорились как измерять, можно считать, что мы ввели новую физическую величину.

    \section{Нулевое начало термодинамики}

    \begin{proposition}[Нулевое начало термодинамики]
        Если два тела находятся в термодинамическом равновесии с третьим, то тогда они в термодинамическом равновесии между собой. Иначе, равенство температуры транзитивно.
    \end{proposition}
    
    \section{Уравнение состояния (термическое уравнение)}

    \begin{note}
        В термодинамике оно существует, независимо от того, как его получать (и как оно выглядит).
    \end{note}

    \subsection{Вывод уравнения состояния}
    Рассмотрим три ТС, находящихся в термодинамическом равновесии. Пусть у каждой из них параметры $p_i, V_i, T$. Можно записать три уравнения, задающих термодинамическое равновесие: 
    \begin{equation}
        \Phi (p_i, V_i, p_j, V_j) = 0,\ \mbox{где ($i \neq j$)}.
    \end{equation}
    Можно выразить, например, $p_1$ из двух уравнений: 
    \begin{eqnarray}
        &p_1 = g_2 (V_1, p_2, V_2), \\
        &p_1 = g_3 (V_1, p_3, V_3).
    \end{eqnarray}
    Вычитая, получаем $g_2 - g_3 = 0$. Поскольку это верно для любого $V_1$, получаем уравнение, которое зависит только от $p_i, V_i$:
    \begin{equation}
        \phi (p_i, V_i) = const.
    \end{equation}
    \begin{addition}
        Вообще говоря, из молекулярной физики можно вывести различные уравнения состояния: 
        \begin{align}
            &pV = \nu RT, \\
            &(p + p_{\text{in}})(V - V_{\text{мол}}) = RT, \\
            &p = \frac{\alpha}{3} T^4, \\
            &F = ES \left(\frac{l}{l_\alpha}\big(1 - \alpha(T - T_0)\big) - 1\right),
        \end{align}
    \end{addition}

    \subsection{Математическое отступление}
    Для дальнейшего прочтения курса важно понимание некоторых математических конструкций.
    \begin{note}
        Для $f(x, y)$ запись \extpartder{f}{x}{y} означает частную производную по $x$ при постоянном $y$.
    \end{note}
    \begin{itemize}
        \item $df(x,y) = \extpartder{f}{x}{y} dx + \extpartder{f}{y}{x} dy$,
        \item $\partder{}{x} \partder{}{y} f= \partder{}{y} \partder{}{x} f$,
        \item если $f(x, y, z) = 0$, то $\extpartder{x}{y}{z} \extpartder{z}{x}{y} \extpartder{y}{z}{x} = -1$.
    \end{itemize}
    Последнее называется \emph{свойством циклической перестановки}.
    \begin{proof}
        Выразим $x = x(y, z)$, тогда 
        \begin{equation}
            dx = \extpartder{x}{y}{z} dy + \extpartder{x}{z}{y} dz,
        \end{equation}
    откуда при постоянном $x$ получаем:
    \begin{equation}
        \extpartder{x}{y}{z} dy = - \extpartder{x}{z}{y} dz. 
    \end{equation}
    Учитывая что дифференциалы берутся при постоянном $x$ можем переписать в таком виде:
    \begin{equation}
        \extpartder{x}{y}{z} \extpartder{z}{x}{y} \extpartder{y}{z}{x} = -1.
    \end{equation}
    \end{proof}
    
    \begin{note}
        В частности, для термодинамической системы получаем:
        \begin{equation}
            \label{eq:cyclic_permutation_pvt}
            \extpartder{p}{V}{T} \extpartder{T}{p}{V} \extpartder{V}{T}{p} = -1.
        \end{equation}
    \end{note}

    \subsection{Термические коэффициенты}

    \begin{definition}
        \emph{Коэффициент термического расширения}
        \begin{equation}
            \alpha  =   \frac{1}{V} \cdot \extpartder{V}{T}{P}.
        \end{equation}
    \end{definition}

    \begin{definition}
        \emph{Коэффициент термического давления}
        \begin{equation}
            \beta   =   \frac{1}{P} \cdot \extpartder{P}{T}{V}.
        \end{equation}
    \end{definition}
    
    \begin{definition}
        \emph{Термический коэффициент всестороннего сжатия}
        \begin{equation}
            \chi    = - \frac{1}{V} \cdot \extpartder{V}{P}{T}.
        \end{equation}
    \end{definition}
    \begin{corollary}
        Нетрудно получить из определений и \eqref{eq:cyclic_permutation_pvt} соотношение:
        \begin{equation}
            \frac{\chi \beta}{\alpha} = \frac{1}{P}.
        \end{equation}
    \end{corollary}

    \section{Первое начало термодинамики} 

    \begin{definition}
        \emph{Квазистатический процесс}~--- процесс, при котором мы мало изменяем параметры, дожидаемся пока установится равновесие и только потом изменяем дальше.
    \end{definition}
    \begin{addition}
        \emph{Неквазистатическим процессом}~можно назвать, например, удар гирей. Такой процесс необратим.
    \end{addition}

    \begin{proposition}[Первое начало термодинамики]
        Изменение внутренней энергии термодинамической системы может быть осуществлено двумя путями~--- путём совершения механической работы и путём теплопередачи:
        \begin{equation}
            \delta Q + \delta A = dU,
        \end{equation}
        где $\delta Q$~--- количество теплоты переданной системе, $\delta A$~--- работа совершённая \emph{над} телом и $dU$~--- изменение внутренней энергии системы.
    \end{proposition}

    \begin{note}
        Фигурные дельты подчёркивают, что $\delta Q, \delta A$~--- функции процесса, а $dU$~--- функция состояния. Это следует из минус-первого начала термодинамики: вся энергия <<движения>> при переходе в состояние равновесия уходит именно туда. 
    \end{note}

    \begin{note}
        Часто пишут, что $\delta A = p dV$. Это внешнее давление, которое <<навязал>> термостат. Частая путаница возникает из-за квазистатического процесса, когда ТС постоянно находится в состоянии равновесия, и давления равны. Когда процесс неквазистатический, о давлении газа \emph{в принципе} не может быть и речи.
    \end{note}



\end{document}