\documentclass[../main.tex]{subfiles}
\begin{document}
\section*{Предисловие}
\segment{Рекомендованная литература}
\begin{itemize}
    \item Термодинамика, статистическая и молекулярная физика. Кириченко Н.А. (рекомендуется использовать как настольную книгу, справочник по решению задач).
    \item Общий курс физики (том 2). Термодинамика и молекулярная физика. Сивухин Д.В. (из него можно сделать выводы).
    \item Введение в молекулярную физику и термодинамику. Де Бур Я. (кратенькая, полезная).
    \item Книги Пригожина И. по термодинамике. (более глубокая, современная теория).
\end{itemize}
\segment{Вступительное слово}
Не хотел я это начинать. Я попробовал разобраться, то как я это понял я вам расскажу, не судите строго. Заботал, как вы за одну ночь перед экзаменом.

Ко всей термодинамике, которую вы читаете, относитесь критично: сколько человек, столько и мнений.
Семестр делится на две половины: термодинамика и статистическая физика. В термодинамике есть аксиоматика, в этом её прелесть. Ещё большое количество определений, очень большое...
\end{document}