\subsection{Собственные векторы}

\begin{definition}
	Пусть $\phi \in \mathcal{L}(V)$, $\overline{v} \in V$, $\overline{v} \ne \overline{0}$, $\lambda \in F$. Вектор $\overline{v}$ называется \textit{собственным вектором} оператора $\phi$ с \textit{собственным значением} $\lambda$, если $\phi(\overline{v}) = \lambda\overline{v}$. Аналогично, $\lambda \in F$ называется \textit{собственным значением} оператора $\phi$, если $\exists \overline{u} \in V$ "--- собственный для $\phi$ с собственным значением $\lambda$.
\end{definition}

\begin{note}
	Пусть $\lambda \in F$ "--- собственное значение $\phi$. Тогда $\overline{v} \in V, \overline{v} \ne \overline{0}$ "--- собственный вектор оператора $\phi$ со значением $\lambda \hm{\Leftrightarrow} \phi(\overline{v}) = \lambda\overline{v} \hm{\Leftrightarrow} \overline{v} \hm{\in} \ke{(\phi - \lambda)}$.
\end{note}

\begin{definition}
	Пусть $\phi \in \mathcal{L}(V)$, $\lambda \in F$ "--- собственное значение оператора $\phi$. Подпространство $V_\lambda := \ke{(\phi - \lambda)} \le V$ называется \textit{собственным подпространством} оператора $\phi$, соответствующим $\lambda$.
\end{definition}

\begin{note}
	$v \in V$, $\overline{v} \ne \overline{0}$ "--- собственный вектор оператора $\phi \in \mathcal{L}(V)$ $\Leftrightarrow$ $\langle \overline{v} \rangle$ инвариантно относительно $\phi$. Если $U \le V_\lambda$, то $U$ тоже инвариантно относительно $\phi$.
\end{note}

\begin{theorem}
	Пусть $\phi \in \mathcal{L}(V)$, $\lambda_1, \dots, \lambda_k \in F$ "--- различные собственные значения оператора $\phi$. Тогда сумма $V_{\lambda_1} + \dots + V_{\lambda_k}$ "--- прямая.
\end{theorem}

\begin{proof}
	Докажем данное утверждение индукцией по $k$. База, $k = 1$, тривиальна, докажем переход. Пусть для $1, \dots, k - 1$ теорема верна, а для $k$ "--- нет, тогда, по критерию прямой суммы, $\exists i \in \{1, \dots, k\}: V_{\lambda_i} \cap (V_{\lambda_1} + \dots \hm{+} V_{\lambda_{i - 1}} + V_{\lambda_{i + 1}} + \dots + V_{\lambda_k}) \ne \{\overline{0}\}$. Пусть без ограничения общности $i = k$, тогда для некоторого набора векторов $\overline{v_1} \in V_{\lambda_1}, \cdots, \overline{v_k} \in V_{\lambda_k}$ верно следующее:
	\[\overline{0} \ne \overline{v_k} \hm{=} \overline{v_1} + \dots + \overline{v_{k - 1}}\]
	
	Если к данному равенству применить $\phi$, а из полученного равенства вычесть исходное, умноженное на $\lambda_k$, получим:
	\[\overline{0} = (\lambda_k - \lambda_k)\overline{v_k} = (\lambda_1 - \lambda_k)\overline{v_1} + \dots + (\lambda_{k - 1} - \lambda_k)\overline{v_{k - 1}}\]
	
	Поскольку все коэффициенты в правой части по условию ненулевые, а также хотя бы один из векторов в правой части ненулевой (так как иначе $\overline{v_k} = \overline{0}$), то получено нетривиальное разложение нуля, что невозможно по предположению индукции, --- противоречие.
\end{proof}

\begin{corollary}
	Количество собственных значений оператора $\phi$ не превосходит $\dim{V}$.
\end{corollary}

\begin{proof}
	Если собственных значений у $\phi$ больше чем $\dim{V}$, то соответствующие им собственные подпространства образуют прямую сумму размерности большей, чем $\dim{V}$, что невозможно.
\end{proof}

\begin{note}
	Поскольку $V_{\lambda_1} \oplus \dots \oplus V_{\lambda_k}$ "--- прямая сумма, то объединение базисов в $V_{\lambda_1}, \dots, V_{\lambda_k}$ можно дополнить до базиса $e$ в $V$. В полученном базисе матрица преобразования $\phi$ принимает следующий вид:
	\[
	\phi \xleftrightarrow[e]{}
	\left(\begin{array}{@{}ccc|cccc@{}}
		\lambda_1 & \dots & 0 & * & \dots & *\\
		\vdots & \ddots & \vdots & \vdots & \ddots & \vdots\\
		0 & \dots & \lambda_k & \vdots &\ddots & \vdots\\
		\cline{1-3}
		0 &\dots & 0 & \vdots &\ddots & \vdots\\
		\vdots & \ddots & \vdots & \vdots & \ddots & \vdots\\
		0 & \dots & 0 & * &\dots & * 
	\end{array}\right)
	\]
	
	Возможно, некоторые $\lambda_i$ повторяются (в точности те, для которых $\dim{V_{\lambda_i}} > 1$).
\end{note}

\begin{definition}
	Пусть $A \in M_n(F)$. \textit{Характеристическим многочленом} матрицы $A$ называется $\chi_A(\lambda) := |A - \lambda E|$.
\end{definition}

\begin{note}
	Степень характеристического многочлена $\chi_A$ равна $n$, поскольку единственное слагаемое с $\lambda^n$ в формуле определителя можно получить, взяв $\sigma = \id$ и $n$ раз перемножив $(-\lambda)$. Значит, коэффициент при $\lambda^n$ "--- это $(-1)^n$.
\end{note}

\begin{proposition}
	Пусть $\phi \in \mathcal{L}(V)$, $\dim{V} = n$, $\phi \leftrightarrow_e A \in M_n(F)$. Тогда $\lambda_0$ "--- собственное значение $\phi$ $\Leftrightarrow$ $\chi_A(\lambda_0) = 0 \Leftrightarrow (\lambda - \lambda_0)\mid \chi_A(\lambda)$.
\end{proposition}

\begin{proof}
	$\lambda_0$ "--- собственное значение оператора $\phi$ $\Leftrightarrow$ $\ke{(\phi - \lambda_0)} \ne \{\overline{0}\}$. Последнее равносильно тому, что $\rk{(A - \lambda_0 E)} < n \hm{\Leftrightarrow} |A - \lambda_0 E| = 0 \hm{\Leftrightarrow} \lambda_0$ "--- корень $\chi_A(\lambda)$.
\end{proof}

\begin{definition}
	Матрицы $A, B \in M_n(F)$ называются \textit{подобными}, если $B = S^{-1}AS$ для некоторой $\exists S \in \GL_n(F)$, то есть эти они могут быть матрицами одного и того же оператора в разных базисах.
\end{definition}

\begin{proposition}
	Пусть $A, B \in M_n(F)$ "--- подобные матрицы. Тогда $\chi_A(\lambda) = \chi_B(\lambda)$.
\end{proposition}

\begin{proof}
	$\forall \lambda \in F: |A - \lambda E| = |S^{-1}(B - \lambda E)S| = |S^{-1}||B - \lambda E||S| = |B - \lambda E|$, значит, $\forall \lambda \in F: \chi_A(\lambda) \hm{=} \chi_B(\lambda)$.
\end{proof}

\begin{definition}
	Пусть $\phi \in \mathcal{L}(V)$. \textit{Характеристическим многочленом} оператора $\phi$ называется характеристический многочлен его матрицы в произвольном базисе. Обозначение "--- $\chi_\phi(\lambda)$.
\end{definition}

\begin{definition}
	Пусть $A \in M_n(F)$. \textit{Следом} матрицы $A$ называется $\sum_{i = 1}^na_{ii}$. Обозначение "--- $\tr{A}$.
\end{definition}

\begin{proposition}
	Пусть $A \in M_n(F)$. В характеристическом многочлене $\chi_A(\lambda)$ коэффициент при $\lambda^{n - 1}$ равен $(-1)^{n - 1}\tr{A}$, а свободный член равен $\det{A}$.
\end{proposition}

\begin{proof}
	Поскольку во всех нетождественных перестановках степень получаемых в $\chi_A(\lambda)$ мономов не превосходит $n - 2$, то слагаемое с $\lambda^{n - 1}$ может возникнуть только при $\sigma = \id$, когда число $(-\lambda)$ перемножается $n - 1$ раз и умножается на один из оставшихся коэффициентов, поэтому коэффициент при $\lambda^{n - 1}$ равен $(-1)^{n - 1}\tr{A}$. Свободный член у $\chi_A(\lambda)$ равен $\chi_A(0) = |A|$.
\end{proof}

\begin{corollary}
	Если $A$ и $B$ подобны, то $\tr{A} = \tr{B}$ и $\det{A} = \det{B}$.
\end{corollary}

\begin{definition}
	Пусть $\phi \in \mathcal{L}(V)$. \textit{Следом} оператора $\phi$ называется след матрицы $\phi$ в произвольном базисе, \textit{определителем} оператора "--- определитель матрицы $\phi$ в произвольном базисе. Обозначения "--- $\tr{\phi}$ и $\det{\phi}$ соответственно.
\end{definition}

\begin{theorem}
	Пусть $\phi \in \mathcal{L}(V)$, $\dim{V} = n$, $\chi_\phi(\lambda)$ имеет $n$ различных корней $\lambda_1, \dots, \lambda_n$. Тогда в $V$ существует базис $e$ такой, что $\phi \leftrightarrow_e A$, где $A \in M_n(F)$ имеет следующий вид:
	\[A = \begin{pmatrix}
		\lambda_1 & 0 & \dots & 0\\
		0 & \lambda_2 & \dots & 0\\
		\vdots & \vdots & \ddots & \vdots\\
		0 & 0 & \dots & \lambda_n
	\end{pmatrix}\]
\end{theorem}

\begin{proof}
	Уже было доказано, что корни $\chi_\phi(\lambda)$ "--- это собственные значения $V$. Тогда $V_{\lambda_1} \oplus \dots \oplus V_{\lambda_n} = V$, поэтому объединение базисов $V_{\lambda_1}, \dots, V_{\lambda_n}$ "--- это и есть искомый базис $e$ в $V$.
\end{proof}

\begin{corollary}
	В условиях предыдущей теоремы выполнены равенства $\tr{\phi} = \sum_{i = 1}^n\lambda_i$ и $\det{\phi} \hm{=} \prod_{i = 1}^n\lambda_i$.
\end{corollary}

\begin{definition}
	Оператор $\phi \in \mathcal{L}(V)$ называется \textit{диагонализуемым}, если существует базис в $V$, в котором матрица $\phi$ имеет диагональный вид (при этом элементы на диагонали не обязаны быть различными). Матрица $A \in M_n(F)$ называется \textit{диагонализуемой}, если она подобна некоторой диагональной.
\end{definition}

\begin{definition}
	Пусть $\phi \in \mathcal{L}(V)$, $\lambda_0 \in F$ "--- собственное значение $\phi$. \textit{Алгебраической кратностью} собственного значения $\lambda_0$ называется кратность корня $\lambda_0$ в $\chi_\phi(\lambda)$, \textit{геометрической кратностью} "--- $\dim{V_{\lambda_0}} = \dim{\ke{(\phi - \lambda_0)}}$.
\end{definition}

\begin{theorem}
	Алгебраическая кратность собственного значения $\lambda_0$ преобразования $\phi \in \mathcal{L}(V)$ не меньше его геометрической кратности.
\end{theorem}

\begin{proof}
	Обозначим геометрическую кратность $\lambda_0$ через $k$. Выберем базис $(\overline{e_1}, \dots, \overline{e_k})$ в $V_{\lambda_0}$ и дополним этот базис до базиса $e \hm{=} (\overline{e_1}, \dots, \overline{e_n})$ в $V$. Тогда матрица $A \in M_n(F)$ преобразования $\phi$ в этом базисе имеет следующий вид:
	\[
	\phi \leftrightarrow_e A =
	\left(\begin{array}{@{}c|c@{}}
		\lambda_0 E & *\\
		\hline
		0 & D\end{array}\right),~E \in M_k(F),~D \in M_{n - k}(F)
	\]
	
	По теореме об определителе с углом нулей, $\chi_\phi(\lambda) = |A - \lambda E| = |(\lambda_0 - \lambda)E_k||D - \lambda E_{n - k}| \hm= (\lambda_0 - \lambda)^k|D - \lambda E_{n - k}|$. Значит, $\lambda_0$ "--- корень кратности не меньше $k$ в $\chi_\phi(\lambda)$.
\end{proof}

\begin{note}
	Неравенство в предыдущей теореме может быть строгим. Рассмотрим, например, следующую матрицу:
	\[A = \begin{pmatrix}0 & 1\\0 & 0\end{pmatrix} \in M_2(\mathbb{R})\]
	
	Тогда $\chi_\phi(\lambda) = \lambda^2$, и $0$ "--- корень кратности $2$, но $\dim{V_0} = \dim{\ke{\phi}} = 2 - \rk{A} = 1$.
\end{note}

\begin{note}
	Рассмотрим оператор $\phi - \alpha \in \mathcal{L}(V)$, $\alpha \in F$. Тогда $\chi_{\phi - \alpha}(\lambda) = |A - (\lambda + \alpha)E| \hm= \chi_{\phi}(\lambda + \alpha)$. Значит, $\lambda_0$ "--- собственное значение у $\phi - \alpha$ $\Leftrightarrow$ $\lambda_0 + \alpha$ "--- собственное значение у $\phi$. Кроме того, собственные векторы у $\phi - \alpha$ и у $\phi$ совпадают.
\end{note}

\begin{theorem}
	Пусть $\phi \in \mathcal{L}(v)$, $U \le V$ "--- инвариантное относительно $\phi$ подпространство. Тогда $\psi := \phi|_U \in \mathcal{L}(U)$ и $\chi_\psi\mid \chi_\phi$.
\end{theorem}

\begin{proof}
	Дополним базис $e' \hm{=} (\overline{e_1}, \dots, \overline{e_k})$ в $U$ до базиса $e = (\overline{e_1}, \dots, \overline{e_n})$ в $V$. Тогда $\phi \leftrightarrow_e A \in M_n(F)$, где $A$ имеет следующий вид:
	\[A = \left(\begin{array}{@{}c|c@{}}
		B & C\\
		\hline
		0 & D
	\end{array}\right),~B \in M_k(F),~D \in M_{n - k}(F)\]
	
	По теореме об определителе с углом нулей, $\chi_A(\lambda) \hm{=} |B - \lambda E_k||D \hm{-} \lambda E_{n - k}|$, поэтому $\psi \hm{=} \phi|_U \leftrightarrow_{e'} B$, и $\chi_\psi\mid \chi_\phi$.
\end{proof}

\begin{note}
	Предыдущую теорему можно считать следствием из только что доказанной.
\end{note}

\begin{theorem}
	Пусть $\phi \in \mathcal{L}(V)$. Тогда равносильны следующие условия:
	\begin{enumerate}
		\item $\phi$ диагонализуем.
		\item $\chi_\phi$ раскладывается на линейные сомножители:
		\[\chi_\phi(\lambda) = \prod_{i = 1}^k(\lambda_i - \lambda)^{\alpha_i},~\sum_{i = 1}^k\alpha_i = n\]
		
		При этом у каждого собственного значения $\lambda_i$ геометрическая кратность равна алгебраической.
		\item $V = V_{\lambda_1} \oplus \dots \oplus V_{\lambda_k}$, где $V_{\lambda_1}, \dots, V_{\lambda_k}$ "--- собственные подпространства оператора $\phi$.
		\item В $V$ есть базис, состоящий из собственных векторов оператора $\phi$.
	\end{enumerate}
\end{theorem}

\begin{proof}~
	\begin{itemize}
		\item\imp{1}{2}Пусть в некотором базисе $e$ в $V$ выполнено $\phi \leftrightarrow_e A \in M_n(F)$ для матрицы вида $A = \diag(\lambda_1, \dotsc, \lambda_n)$. Тогда, так как $\chi_\phi = \prod_{j = 1}^n(\lambda_j - \lambda) = \prod_{i = 1}^k(\lambda_i - \lambda)^{\alpha_i}$, если у собственного значения $\lambda_i$ алгебраическая кратность равна $\alpha_i$, то в $V$ есть $\alpha_i$ собственных векторов с соответствующим собственным значением, образующих линейно независимую систему. Отсюда $\dim{V_{\lambda_i}} \ge \alpha_i$, и, по уже доказанной теореме, геометрическая кратность каждого корня равна алгебраической.
		
		\item\imp{2}{3}По условию, $\dim{V_{\lambda_i}} = \alpha_i$ и $\sum_{i = 1}^n\alpha_i = n$. Сумма $V_{\lambda_1} \oplus \dots \oplus V_{\lambda_k}$ "--- прямая, поэтому $\sum_{i = 1}^k\dim{V_{\lambda_i}} = \sum_{i = 1}^k\alpha_i = n$ и $V_{\lambda_1} \oplus \dots \oplus V_{\lambda_k} = V$.
		
		\item\imp{3}{4}Выберем базис в каждом $V_{\lambda_i}$. Тогда, так как сумма $V_{\lambda_1} \oplus \dots \oplus V_{\lambda_k}$ "--- прямая, то объединение этих базисов дает базис в $V$, который и является искомым.
		
		\item\imp{4}{1}Если $e$ "--- базис из собственных векторов, то именно в этом базисе матрица оператора $\phi$ имеет требуемый диагональный вид.\qedhere
	\end{itemize}
\end{proof}

\begin{note}
	Рассмотрим $V_2$ (над $\mathbb{R}$). Пусть $\phi$ "--- поворот на угол $\alpha \in (0, \pi)$. Тогда ни один ненулевой вектор в $V_2$ не переходит в коллинеарный себе, значит, у $\phi$ нет собственных значений.
\end{note}