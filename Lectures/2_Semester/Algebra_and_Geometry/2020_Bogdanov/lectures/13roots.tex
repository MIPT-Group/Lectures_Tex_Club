\subsection{Корни многочленов}

\textbf{До конца раздела} зафиксируем поле $F$.

\begin{definition}
	Пусть $P \in F[x]$. Скаляр $a \in F$ называется \textit{корнем} многочлена $P$, если выполнено равенство $P(a) = 0$.
\end{definition}

\begin{theorem}[Безу]
	Скаляр $a \in F$ является корнем многочлена $P \in F[x]$ $\Leftrightarrow$ $(x - a)\mid P$.
\end{theorem}

\begin{proof}
	Разделим $P$ с остатком на $(x - a)$, то есть выберем $Q, R \in F[x]$ такие, что $P = Q(x - a) \hm{+} R$ и $\deg{R} \le 0$. Заметим, что $P(a) = R$, тогда выполнены равносильности $P(a) = 0 \Leftrightarrow R = 0 \hm{\Leftrightarrow} (x - a)\mid P$.
\end{proof}

\begin{definition}
	Пусть $a \in F$ "--- корень многочлена $P \in F[x]$. \textit{Кратностью} корня $a$ называется наибольшее $\gamma \in \mathbb{N}$ такое, что $(x - a)^\gamma\mid P$. Если $\gamma > 1$, то корень $a$ называется \textit{кратным}, иначе --- \textit{простым}.
\end{definition}

\begin{theorem}
	Пусть $P \in F[x] \bs \{0\}$, и $a_1, \dots, a_k$ "--- корни многочлена $P$, имеющие кратности $\gamma_1, \dots, \gamma_k \in \N$. Тогда $\gamma_1 + \dots + \gamma_k \le \deg{P}$.
\end{theorem}

\begin{proof}
	По условию, для любого $i \in \{1, \dots, k\}$ выполнено $(x - a_i)^{\gamma_i}\mid P$, причем для любых индексов $i, j \hm{\in} \{1, \dots, k\}$, $i \ne j$, выполнены следующие равенства:
	\[\nd(x - a_i, x - a_j) = \nd(x - a_i, a_i - a_j) = 1\]
	
	Значит, многочлены $(x - a_1)^{\gamma_1}, \dotsc, (x - a_k)^{\gamma_k}$ попарно неассоциированы, тогда, в силу единственности разложения многочлена $P$ на неприводимые сомножители, выполнено неравенство $\gamma_1 + \dots + \gamma_k \le \deg{P}$.
\end{proof}

\begin{note}
	В коммутативном кольце, не являющемся целостным данная теорема неверна, поскольку неверна единственность разложения на неприводимые сомножители. Например, в кольце $\mathbb{Z}_4$ у многочлена $P = x^2 = (x - 2)^2$ степени $2$ есть корень $0$ кратности $2$ и корень $2$ кратности $2$.
\end{note}

\begin{note}
	Над полем $\mathbb{C}$ число корней любого ненулевого многочлена с учетом кратности равно его степени. Это утверждение называется \textit{основной теоремой алгебры}, но в рамках данного курса мы не будем его доказывать.
\end{note}

\begin{definition}
	Пусть $P = p_0 + p_1x + \dots + p_nx^n \in F[x]$. \textit{Формальной производной} многочлена $P(x)$ называется многочлен $P' \hm{:=} p_1 + 2p_2x + \dots + np_nx^{n - 1}$, где целочисленные скаляры понимаются как суммы соответствующего числа единиц.
\end{definition}

\begin{proposition}
	Формальная производная обладает следующими свойствами:
	\begin{enumerate}
		\item $\forall \alpha, \beta \in F: \forall P, Q \in F[x]: (\alpha P+ \beta Q)' \hm= \alpha P' + \beta Q'$ (линейность)
		\item $\forall P, Q \in F[x]: (PQ)' = P'Q + PQ'$ (правило Лейбница)
	\end{enumerate}
\end{proposition}

\begin{proof}~
	\begin{enumerate}
		\item Пусть $n := \max\{\deg{P}, \deg{Q}\}$, тогда многочлены $P$ и $Q$ можно представить в виде $P = \sum_{i = 0}^np_ix^i$ и $Q = \sum_{i = 0}^nq_ix^i$, откуда $\alpha P + \beta Q \hm= \sum_{i = 0}^n(\alpha p_i + \beta q_i)x^i$. Проверим требуемое равенство непосредственной проверкой:
		\[(\alpha P + \beta Q)' = \sum_{i = 1}^ni(\alpha p_i + \beta q_i)x^{i - 1} = \alpha\sum_{i = 1}^nip_ix^{i - 1} + \beta\sum_{i = 1}^niq_ix^{i - 1} = \alpha P' + \beta Q'\]
		
		\item Левая и правая части требуемого равенства линейны по $P$ и по $Q$, поэтому равенство достаточно проверить на некотором базисе пространства многочленов, например, для произвольных многочленов вида $P(x) = x^i$, $Q(x) = x^j$, $i, j \in \N \cup \{0\}$:
		\[(PQ)' = (i + j)x^{i + j - 1} = ix^{i - 1}x^j + jx^ix^{j - 1} = P'Q + PQ'\qedhere\]
	\end{enumerate}
\end{proof}

\begin{note}
	Формальная производная не обладает аналитическими свойствами. В $\mathbb{Z}_p[x]$, например, выполнены равенства $(x^p)' = px^{p - 1} = 0$.
\end{note}

\begin{corollary} Формальная производная обладает следующими свойствами:
	\begin{enumerate}
		\item $\forall P_1, \dotsc, P_n \in F[x]: (P_1P_2\dots P_n)' = P_1'P_2\dots P_n + P_1P_2'\dots P_n + \dots + P_1P_2\dots P_n'$
		\item $\forall P \in F[x]: \forall n \in \N: (P^n)' = nP^{n - 1}P'$
		\item $\forall P, Q \in F[x]: \big(P(Q)\big)' = P'(Q)Q'$
	\end{enumerate}
\end{corollary}

\begin{proof}~
	\begin{enumerate}
		\item Достаточно провести индукцию по $n$.
		\item Достаточно применить первое равенство к многочлену $P^n$.
		\item Считая, что $P(x) = p_0 + p_1x + \dots + p_nx^n$, воспользуемся вторым равенством:
		\[\big(P(Q)\big)' \hm= \left(\sum_{i = 0}^mp_iQ^i\right)' = \sum_{i = 0}^mip_iQ^{i - 1}Q' = P'(Q)Q'\qedhere\]
	\end{enumerate}
\end{proof}

\begin{theorem}
	Пусть $P \in F[x]$, $c \in F$. Тогда следующие условия эквивалентны:
	\begin{enumerate}
		\item $c$ "--- кратный корень $P$
		\item $P(c) = P'(c) = 0$
		\item $(x - c)\mid \nd(P, P')$
	\end{enumerate}
\end{theorem}

\begin{proof}~
	\begin{itemize}
		\item\implr{1}{2}Пусть $c$ "--- корень многочлена $P$, тогда $P = (x - c)Q$ и $P' = Q + (x - c)Q'$, поэтому $c$ "--- кратный корень многочлена $P$ $\Leftrightarrow$ $Q(c) = 0$ $\Leftrightarrow$ $P'(c) = 0$.
		\item\implr{2}{3}$P(c) = P'(c) = 0 \lra (x - c)\mid P, P'\lra (x - c)\mid \nd(P, P')$.\qedhere
	\end{itemize}
\end{proof}

\begin{theorem}
	Пусть $c \in F$ "--- корень многочлена $P \in F[x]$ кратности $k \in \N$, $k > 1$. Тогда выполнены следующие свойства:
	\begin{enumerate}
		\item $c$ "--- корень многочлена $P'$ кратности хотя бы $k - 1$
		\item Если $\cha{F} > k$ или $\cha{F} = 0$, то $c$ "--- корень многочлена $P'$ кратности ровно $k - 1$
	\end{enumerate}
\end{theorem}

\begin{proof}
	Многочлен $P$ имеет вид $(x - c)^kQ$ для некоторого $Q \in F[x]$ такого, что $(x - c)\nmid Q$. Тогда:
	\[P' = k(x - c)^{k - 1}Q + (x - c)^kQ' = (x - c)^{k - 1}\big(kQ + (x-c)Q'\big)\]
	
	Из равенства выше уже следует, что $c$ "--- корень многочлена $P'$ кратности хотя бы $k - 1$. Рассмотрим теперь многочлен $kQ \hm{+} (x - c)Q'$. Если $\cha{F} > k$ или $\cha{F} = 0$, то $kQ(c) \ne 0$, поэтому кратность корня $c$ у многочлена $P'$ равна $k - 1$.
\end{proof}

\begin{corollary}
	Пусть $c \in F$ "--- корень многочлена $P \in F[x]$ кратности $k \in \N$, $k > 1$. Тогда выполнены равенства $P(c) = P'(c) \hm{=} \dots = P^{(k - 1)}(c) = 0$.
\end{corollary}

\begin{proof}
	Заметим, что $(x - c)^k\mid P \Rightarrow (x - c)^{k - 1}\mid P' \hm\Rightarrow \dots \Rightarrow (x - c)\mid P^{(k - 1)}$.
\end{proof}

\begin{corollary}
	Пусть $k \in \N$, $k > 1$, $\cha{F} \ge k$ или $\cha{F} = 0$, и пусть для многочлена $P \in F[x]$ выполнены равенства $P(c) = \dots = P^{(k - 1)}(c) = 0$. Тогда $c$ "--- корень многочлена $P$ кратности хотя бы $k$.
\end{corollary}

\begin{proof}
	Предположим, что $c$ "--- корень кратности $l < k$ многочлена $P$. Тогда $c$ является простым корнем многочлена $P^{(l - 1)}$, откуда $P^{(l)}(c) \ne 0$ --- противоречие.
\end{proof}

\begin{corollary}[теорема Вильсона]
	Пусть $p$ "--- простое число. Тогда выполнено следующее:
	\[(p - 1)! \equiv_p -1\]
\end{corollary}

\begin{proof}
	Рассмотрим многочлен $P = x^{p - 1} - 1 \in \mathbb{Z}_p[x]$. Его производная $P'$ равна $-x^{p - 2}$. Заметим, что $\nd(P, P') = 1$, так как все делители многочлена $P'$ имеют вид $x^k$, а $0$ не является корнем $P$. Значит, все корни многочлена $P$ "--- простые, причем, по малой теореме Ферма, его корнями являются все элементы $1, \dots, p - 1 \in \Z_p$. Тогда, поскольку степень многочлена $P$ равна $p - 1$, выполнено следующее равенство:
	\[x^{p - 1} - 1 = (x - 1)(x - 2)\dots(x - (p-1))\]
	
	Поскольку левая и правая части "--- это один и тот же многочлен в $\mathbb{Z}_p$, то в $\Z_p$ выполнено равенство $(-1)^{p - 1}(p - 1)!= -1$, то есть $(-1)^{p - 1}(p - 1)! \equiv_p -1$. Наконец, $1 \equiv_2 -1$, и все простые числа, отличные от $2$, нечетны, поэтому $(p - 1)! \equiv_p -1$.
\end{proof}

\begin{corollary}
	Пусть $p$ "--- простое число. Тогда для любого $x \in \Z$ выполнено следующее:
	\[x^p - 1 \equiv_p (x - 1)^p\]
\end{corollary}

\begin{proof}
	Рассмотрим многочлен $Q = x^p - 1 \in \mathbb{Z}_p[x]$. Все его производные тождественно равны нулю, поэтому $\overline{1}$ "--- корень кратности хотя бы $p$ этого многочлена. Тогда, поскольку степень многочлена $Q$ равна $p$, для любого $x \in \Z_p$ выполнено равенство $x^p - 1 = (x - 1)^p$, то есть для любого $x \in \Z$ выполнено сравнение $x^p - 1 \equiv_p (x - 1)^p$.
\end{proof}