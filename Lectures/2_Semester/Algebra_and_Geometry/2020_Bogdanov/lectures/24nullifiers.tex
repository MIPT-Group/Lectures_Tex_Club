\subsection{Аннулирующие многочлены}

\textbf{До конца раздела} зафиксируем линейное пространство $V$ над полем $F$ и положим $n := \dim{V}$.

\begin{definition}
	Пусть $\phi \in \mathcal{L}(V)$. Многочлен $P \in F[x] \bs \{0\}$ называется \textit{аннулирующим многочленом} оператора $\phi$, если $P(\phi) = 0$. \textit{Минимальным многочленом} оператора $\phi$ называется аннулирующий многочлен $\mu_\phi$ наименьшей степени.
\end{definition}

\begin{note}
	Из теоремы Гамильтона-Кэли следует, что многочлен $\Chi_\phi$ "--- аннулирующий для $\phi$. Но и без этой теоремы можно установить существование аннулирующего многочлена у произвольного оператора $\phi$: система $(1, \phi, \dots, \phi^{n^2})$ линейно зависима в $\mathcal{L}(V)$, поскольку $\dim{\mathcal{L}(V)} = n^2$, значит, у нее есть нетривиальная линейная комбинация, равная нулю, которая и является искомым многочленом.
\end{note}

\begin{proposition}
	Пусть $\phi \in \mathcal{L}(V)$, $\mu_\phi$ "--- минимальный многочлен для $\phi$. Тогда многочлен $P \in F[x]$ "--- аннулирующий для $\phi$ $\Leftrightarrow$ $\mu_\phi\mid P$.
\end{proposition}

\begin{proof}
	Разделим $P$ на $\mu_\phi$ с остатком, то есть выберем $Q, R \in F[x]$ такие, что $P = Q\mu_\phi + R$ и $\deg{R} < \deg{\mu_\phi}$, тогда $P(\phi) = R(\phi)$. Тогда, поскольку $\mu_\phi$ "--- минимальный, выполнены следующие равносильности:
	\[P(\phi) = 0 \Leftrightarrow R(\phi) = 0 \Leftrightarrow R = 0 \Leftrightarrow \mu_\phi\mid P\qedhere\]
\end{proof}

\begin{corollary}
	Минимальный многочлен оператора $\phi \in \mc L(V)$ единственен с точностью до ассоциированности.
\end{corollary}

\begin{proof}
	Пусть $\mu_1, \mu_2 \in F[x]$ "--- различные минимальные многочлены для $\phi$. Тогда, поскольку оба они ненулевые, выполнено следующее:
	\[\left\{\begin{aligned}
		\mu_1\mid \mu_2\\
		\mu_2\mid \mu_1\\
	\end{aligned}\right. \Rightarrow
	\left\{\begin{aligned}
		\deg{\mu_1} \le \deg{\mu_2}\\
		\deg{\mu_2} \le \deg{\mu_1}\\
	\end{aligned}\right. \Rightarrow \deg{\mu_1} = \deg{\mu_2}\]
	
	Таким образом, $\mu_1\mid \mu_2$ и $\deg{\mu_1} = \deg{\mu_2}$, откуда $\mu_2 = \alpha\mu_1$ для некоторого $\alpha \in F^*$.
\end{proof}

\begin{note}
	Из теоремы Гамильтона-Кэли и утверждения выше следует, что $\mu_\phi\mid \Chi_\phi$. В частности, если многочлен $\Chi_\phi$ имеет вид $\eqref*{charpol}$, то многочлен $\mu_\phi$ имеет следующий вид при некоторых $\beta_1, \dotsc, \beta_k \in \N \cup \{0\}$ таких, что $\beta_i \le \alpha_i$ для каждого $i \in \{1, \dotsc, n\}$:
	\[\mu_\phi(\lambda) = \prod_{i = 1}^k(\lambda_i - \lambda)^{\beta_i}\]
\end{note}

\begin{proposition}
	Пусть $\phi \in \mathcal{L}(V)$, $\lambda_0 \in F$ "--- собственное значение оператора $\phi$. Тогда $(\lambda - \lambda_0)\mid \mu_\phi$.
\end{proposition}

\begin{proof}
	Достаточно показать, что $\lambda_0$ "--- корень многочлена $\mu_\phi$. Рассмотрим собственный вектор $\overline{v} \in V \bs \{\overline 0\}$ оператора $\phi$ со значением $\lambda_0$, тогда для любого $k \in \mathbb{N}$ выполнено равенство $\phi^k(\overline{v}) = \lambda_0^k\overline{v}$. В частности, для многочлена $\mu_\phi$ выполнены следующие равенства:
	\[\mu_\phi(\phi)(\overline{v}) = \mu_\phi(\lambda_0)\overline{v} = \overline{0}\]
	
	Но вектор $\overline{v}$ "--- ненулевой, поэтому $\mu_\phi(\lambda_0) = 0$.
\end{proof}

\begin{note}
	Можно также показать, что любой неприводимый делитель многочлена $\Chi_\phi$ делит многочлен $\mu_\phi$.
\end{note}

\begin{theorem}
	Пусть $\phi \in \mathcal{L}(V)$, $P \in F[x]$ "--- аннулирующий многочлен оператора $\phi$, и $P = P_1P_2$ для многочленов $P_1, P_2 \in F[x]$ таких, что $\nd(P_1, P_2) = 1$. Тогда $V = V_1 \oplus V_2$, где $V_1 := \ke{P_1(\phi)}, V_2 := \ke{P_2(\phi)}$ "---  инвариантные относительно $\phi$ подпространства.
\end{theorem}

\begin{proof}
	Подпространства из условия инвариантны относительно $\phi$, поскольку операторы $P_1(\phi), P_2(\phi)$ коммутируют с $\phi$. Покажем, что $\im{P_1(\phi)} \le V_2$. Действительно, $P_2(\phi)(P_1(\phi)(V)) \hm{=} P(\phi)(V) = \{\overline{0}\}$, то есть $\im{P_1(\phi)} \le \ke{P_2(\phi)} = V_2$. Аналогично, выполнено включение $\im{P_2(\phi)} \le V_1$. Поскольку $\nd(P_1, P_2) = 1$, то существуют многочлены $Q_1, Q_2 \in F[x]$ такие, что выполнено равенство $P_1Q_1 + P_2Q_2 = 1$. Подставим $\phi$ в это равенство и получим следующее:
	\[P_1(\phi)Q_1(\phi) + P_2(\phi)Q_2(\phi) = \id\]
	
	Значит, для произвольного $\overline{v} \in V$ выполнены следующие равенства:
	\[\overline{v} = \id(\overline{v}) \hm{=} Q_1(\phi)(P_1(\phi)(\overline{v})) + Q_2(\phi)(P_2(\phi)(\overline{v}))\]
	
	Заметим теперь, что $P_1(\phi)(\overline{v}) \in \im P_1(\phi) \le V_2$ и $P_2(\phi)(\overline{v}) \in \im P_2(\phi) \le V_1$, откуда $Q_1(\phi)(P_1(\phi)(\overline{v})) \in V_2$ и $Q_2(\phi)(P_2(\phi)(\overline{v})) \in V_1$ в силу инвариантности подпространств $V_1, V_2$ относительно $\phi$. Значит, $V_1 + V_2 = V$, причем эта сумма "--- прямая, поскольку для любого вектора $\overline{w} \in V_1 \cap V_2$ выполнены следующие равенства:
	\[\overline{w} = \id(\overline{w}) \hm{=} Q_1(\phi)(P_1(\phi)(\overline{w})) + Q_2(\phi)(P_2(\phi)(\overline{w})) = Q_1(\phi)(\overline{0}) + Q_2(\phi)(\overline{0}) = \overline{0}\]
	
	Таким образом, $V = V_1 \oplus V_2$.
\end{proof}

\begin{note}
	На самом деле, в теореме выше выполнены равенства $\im{P_1(\phi)} = V_2$ и $\im{P_2(\phi)} = V_1$. Согласно теореме, $V_1 \oplus V_2 = V$, поэтому выполнено следующее:
	\[\dim{V_1} \hm{+} \dim{V_2} = \dim{V} = \dim{\im{P_1(\phi)}} + \dim{\ke{P_1(\phi)}}\]
	
	Следовательно, $\dim{V_2} \hm{=} \dim{\im{P_1(\phi)}}$, и, в силу включения $\im{P_1(\phi)} \le V_2$, имеем $\im{P_1(\phi)} = V_2$. Аналогичное рассуждение показывает, что $\im{P_2(\phi)} = V_1$.
\end{note}

\begin{corollary}
	Пусть $\phi \in \mathcal{L}(V)$, $P \in F[x]$ "--- аннулирующий многочлен оператора $\phi$, и ${P = P_1 \dotsm P_n}$ для попарно взаимно простых многочленов $P_1, \dotsc, P_n \in F[x]$. Тогда выполнено равенство $V = V_1 \oplus \dots \oplus V_n$, где $V_i := \ke{P_i(\phi)}$ "--- инвариантное относительно $\phi$ подпространство для каждого $i \in \{1, \dotsc, n\}$.
\end{corollary}

\begin{proof}
	Проведем индукцию по $n$. База, $n = 2$, уже доказан, докажем переход. Пусть $n > 2$, тогда $P = P_1\dots P_n = (P_1\dots P_{n - 1})P_n$, причем многочлены $(P_1\dots P_{n-1})$ и $P_n$ взаимно просты, тогда выполнено следующее:
	\[V = \ke{(P_1\dots P_{n-1})(\phi)} \oplus \ke{P_n(\phi)}\]
	
	Положим $\widetilde{V} := \ke{(P_1\dots P_{n-1})(\phi)}$, $V_n := \ke{P_n(\phi)}$, и применим предположение индукции к оператору $\psi := \phi|_{\widetilde{V}} \in \mathcal L(\widetilde V)$ и $P_1\dots P_{n-1}$, тогда:
	\[\widetilde{V} = \ke{P_1(\psi)} \oplus \dots \oplus \ke{P_{n-1}(\psi)}\]
	
	Остается заметить, что для каждого индекса $i \in \{1, \dotsc, n - 1\}$ выполнено включение $V_i \hm{\le} \ke{(P_1\dots P_{n-1})(\phi)} \hm{=} \widetilde{V}$, откуда $\ke{P_i(\psi)} = V_i \cap \widetilde{V} = V_i$, поэтому $V_i' = V_i$, Таким образом, $V = V_1 \oplus \dots \oplus V_n$.
\end{proof}