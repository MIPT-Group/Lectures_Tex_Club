\subsection{Сопряженные операторы}

\textbf{В данном разделе} зафиксируем евклидово (эрмитово) пространство $V$. Положим $\theta := 2$, если $V$ "--- евклидово, и $\theta := \frac{3}{2}$, если $V$ "--- эрмитово. Пространство $\theta$-линейных форм на $V$ обозначим через $\mathcal{B}_\theta(V)$. В частности, скалярное произведение является $\theta$-линейной формой на $V$. 

\begin{definition}
	Пусть $\phi \in \mathcal{L}(V)$. Для всех $\overline{u}, \overline{v} \in V$ положим $f_\phi(\overline{u}, \overline{v}) := (\phi(\overline{u}), \overline{v})$.
\end{definition}

\begin{proposition}
	Пусть $e$ "--- ортонормированный базис в $V$, $\phi \hm{\in} \mathcal{L}(V)$ "--- такой оператор, что $\phi \leftrightarrow_e A$, $f_\phi \leftrightarrow_e B$. Тогда $B = A^T$.
\end{proposition}

\begin{proof}
	Если $\overline{u} \leftrightarrow_e x$, $\overline{v} \leftrightarrow_e y$, то $f_\phi(\overline{u}, \overline{v}) = (Ax)^T\overline{y} = x^TA^T\overline{y}$, что и означает требуемое в силу биективности сопоставления матриц $\theta$-линейным формам.
\end{proof}

\begin{corollary}
	Сопоставление $\phi \mapsto f_\phi$ осуществляет изоморфизм линейных пространств $\mathcal{L}(V)$ и $\mathcal{B}_\theta(V)$.
\end{corollary}

\begin{proof}
	Сопоставление $\phi \mapsto f_\phi$ является композицией изоморфизмов соответствующих линейных пространств вида $\phi \mapsto_e A \mapsto A^T \mapsto_e f_\phi$.
\end{proof}

\begin{note}
	Пусть $\phi \in \mathcal{L}(V)$. Аналогичным образом для всех $\overline{u}, \overline{v} \in V$ положим $g_\phi(\overline{u}, \overline{v}) \hm{:=} (\overline{u}, \phi(\overline{v}))$, тогда $g_\phi$ "--- $\theta$-линейная форма, как и $f_\theta$. Если $\phi \leftrightarrow_e A$, то $g_\phi \leftrightarrow_e \overline{A}$, причем сопоставление $\phi \mapsto g_\phi$ является сопряженно-линейной биекцией, то есть в евклидовом пространстве оно осуществляет изоморфизм, а в эрмитовом "--- антиизоморфизм.
\end{note}

\begin{definition}
	Пусть $\phi \in \mathcal{L}(V)$. Оператором, \textit{сопряженным к $\phi$}, называется оператор $\phi^* \in \mathcal{L}(V)$ такой, что $f_\phi = g_{\phi^*}$, то есть $\forall \overline{u}, \overline{v} \in V: (\phi(\overline{u}), \overline{v}) \hm{=} (\overline{u}, \phi^*(\overline{v}))$.
\end{definition}

\begin{note}
	Поскольку сопоставления $\phi \mapsto f_\phi = g_{\phi^*} \hm{\mapsto} \phi^*$ биективны, то сопряженный оператор $\phi^*$ существует и единственен. Более того, сопоставление $\phi \mapsto \phi^*$ осуществляет автоморфизм в евклидовом случае и антиавтоморфизм в эрмитовом случае.
\end{note}

\begin{proposition}
	Пусть $e$ "--- ортонормированный базис в $V$, $\phi \hm{=} \mathcal{L}(V)$ "--- такой линейный оператор, что $\phi \leftrightarrow_e A$. Тогда $\phi^* \leftrightarrow_e A^*$.
\end{proposition}

\begin{proof}
	Поскольку $\phi \leftrightarrow_e A$, то $f_\phi = g_{\phi^*} \leftrightarrow_e A^T$. Значит, $\phi^* \leftrightarrow_e A^*$.
\end{proof}

\begin{note}
	В неортонормированном базисе $e$ формула получается из аналогичных рассуждений, но вычисления несколько усложняются: если $\phi \leftrightarrow_e A$, то $f_\phi = g_{\phi^*} \leftrightarrow_e A^T\Gamma$, тогда $\phi^* \leftrightarrow_e \overline{\Gamma^{-1}A^T\Gamma}$.
\end{note}

\begin{proposition} Сопряженные операторы обладают следующими свойствами:
	\begin{enumerate}
		\item Сопоставление $\phi \mapsto \phi^*$ сопряженно-линейно
		\item $\forall \phi, \psi \in \mathcal{L}(V): (\phi\psi)^* = \psi^*\phi^*$
		\item $\forall \phi \in \mathcal{L}(V): \phi^{**} = \phi$
	\end{enumerate}
\end{proposition}

\begin{proof}
	Первые свойство уже было отмечено, докажем два последних. Зафиксируем произвольные $\overline{u}, \overline{v} \in V$, тогда:
	\begin{gather*}
		((\phi\psi)(\overline{u}), \overline{v}) = (\psi(\overline{u}), \phi^*(\overline{v})) = (\overline{u}, (\psi^*\phi^*)(\overline{v}))\\
		(\phi^*(\overline{u}), \overline{v}) = \overline{(\overline{v}, \phi^*(\overline{u}))} = \overline{(\phi(\overline{v}), \overline{u})} = (\overline{u}, \phi(\overline{v}))
	\end{gather*}
	
	В силу единственности сопряженного оператора, получено требуемое.
\end{proof}

\begin{proposition}
	Пусть $\phi \in \mathcal{L}(V)$. Тогда $\overline{\chi_\phi(\lambda)} = \chi_{\phi^*}(\overline{\lambda})$.
\end{proposition}

\begin{proof}
	Пусть $A$ "--- матрица $\phi$ в ортонормированном базисе $e$. Тогда:
	\[\overline{\chi_\phi(\lambda)} = \overline{|A - \lambda E|} = |\overline{A} - \overline{\lambda}E| = \chi_{\overline{A}}(\overline{\lambda}) = \chi_{A^*}(\overline{\lambda}) = \chi_{\phi^*}(\overline{\lambda})\qedhere\]
\end{proof}

\begin{proposition}
	Пусть $\phi \in \mathcal{L}(V)$, и подпространство $U \le V$ инвариантно относительно $\phi$. Тогда $U^\perp$ тоже инвариантно относительно $\phi^*$.
\end{proposition}

\begin{proof}
	Пусть $\overline{v} \in U^\perp$. Тогда $\forall \overline{u} \in U: (\overline{u}, \phi^*(\overline{v})) = (\phi(\overline{u}), \overline{v}) = (\phi(\overline{u}), \overline{v}) = 0$ в силу инвариантности $U$. Значит, $\phi^*(\overline{v}) \in U^\perp$.
\end{proof}

\begin{note}
	В силу канонического изоморфизма между $V$ и $V^*$ и справедливости соответствующего свойства аннуляторных подпространств, $U_1 \le U_2 \Rightarrow U_1^\perp \ge U_2^\perp$.
\end{note}

\begin{theorem}[Фредгольма]
	Пусть $\phi \in \mathcal{L}(V)$. Тогда $\ke{\phi^*} \hm{=} (\im{\phi})^\perp$.
\end{theorem}

\begin{proof}~
	\begin{itemize}
		\item[$\subset$] Пусть $\overline{v} \in \ke{\phi^*}$, тогда $\phi^*(\overline{v}) = \overline{0}$, и $\forall \overline{u} \in V: (\phi(\overline{u}), \overline{v}) \hm{=} (\overline{u}, \phi^*(\overline{v})) = 0 \ra \overline{v} \in (\im{\phi})^\perp$.
		
		\item[$\supset$] Заметим, что $\rk{\phi} = \rk{\phi^*} = \dim{\im{\phi}} = \dim{\im{\phi^*}}$, тогда $\dim{\ke{\phi^*}} = \dim{(\im{\phi})^\perp}$, из чего следует требуемое в силу обратного включения.\qedhere
	\end{itemize}
\end{proof}

\begin{corollary}
	Пусть $\phi \in \mathcal{L}(V)$. Тогда $\im{\phi^*} = (\ke{\phi})^\perp$.
\end{corollary}

\begin{note}
	В общем случае, когда $\phi \in \mathcal{L}(U, V)$, \textit{сопряженным к $\phi$ отображением} называется такое $\phi^* \in \mathcal{L}(V^*, U^*)$, что $\forall f \in V^*, \forall \overline{u} \in U: \phi^*(f)(\overline{u}) = f(\phi(\overline{u}))$. Свойства такого отображения будут похожи на доказанные выше, например, аналог теоремы Фредгольма имеет вид $\ke{\phi^*} = (\im{\phi})^0$. Отметим, что в основном рассмотренном случае нам удалось избежать перехода в $V^*$ в силу существования канонического изоморфизма (антиизоморфизма) между $V$ и $V^*$.
\end{note}

\begin{definition}
	Оператор $\phi \in \mathcal{L}(V)$ называется \textit{самосопряженным}, если $\phi^* = \phi$, то есть $\forall \overline{u}, \overline{v} \in V: (\phi(\overline{u}), \overline{v}) = (\overline{u}, \phi(\overline{v}))$.
\end{definition}

\begin{note}
	Если самосопряженный оператор $\phi \in \mathcal{L}(V)$ в ортонормированном базисе имеет матрицу $A$, то $A \leftrightarrow_e \phi = \phi^* \leftrightarrow_e A^*$, то есть $A = A^*$ --- симметрична в евклидовом случае и эрмитова в эрмитовом случае.
\end{note}

\begin{proposition}
	Пусть $\phi \in \mathcal{L}(V)$ "--- самосопряженный, $U \le V$. Тогда $U$ инвариантно относительно $\phi$ $\Leftrightarrow$ $U^\perp$ инвариантно относительно $\phi$.
\end{proposition}

\begin{proof}~
	\begin{itemize}
		\item[$\ra$] Это свойство уже было доказано.
		\item[$\la$] $(U^\perp)^\perp = U$, поэтому $U$ инвариантно относительно $\phi$.\qedhere
	\end{itemize}
\end{proof}

\begin{proposition}
	Пусть $\phi \in \mathcal{L}(V)$ "--- самосопряженный. Тогда его характеристический многочлен $\chi_\phi$ раскладывается на линейные сомножители над $\mathbb{R}$.
\end{proposition}

\begin{proof}
	Пусть сначала $V$ "--- эрмитово пространство, $\lambda \in \Cm$ "--- корень $\chi_\phi$. Тогда $\lambda$ является собственным значением оператора $\phi$ с собственным вектором $\overline{v} \in V$, $\overline{v} \ne \overline{0}$, откуда $\lambda||\overline{v}||^2 = (\phi(\overline{v}), \overline{v}) = (\overline{v}, \phi(\overline{v})) = \overline{\lambda}||\overline{v}||^2$. Значит, $\lambda = \overline\lambda \ra \lambda \in \R$.
	
	Пусть теперь $V$ "--- евклидово пространство с ортонормированным базисом $e$, тогда $\phi \leftrightarrow_e A \in M_n(\mathbb{R})$, $A = A^T$. Рассмотрим $U$ "--- эрмитово пространство той же размерности с ортонормированным базиом $\mathcal{F}$ и оператор $\psi \in \mathcal{L}(U)$, $\psi \leftrightarrow_{\mathcal{F}} A$. Тогда $\psi$ "--- тоже самосопряженный, поэтому для $\chi_\psi$ утверждение выполнено. Остается заметить, что $\chi_\psi \hm{=} \chi_A = \chi_\phi$.
\end{proof}

\begin{proposition}
	Пусть $\phi \in \mathcal{L}(V)$ "--- самосопряженный, $\lambda_1, \lambda_2 \hm{\in} \mathbb{R}$ "--- два различных собственных значения $\phi$. Тогда $V_{\lambda_1} \perp V_{\lambda_2}$.
\end{proposition}

\begin{proof}
	Пусть $\overline{v_1} \in V_{\lambda_1}, \overline{v_2} \in V_{\lambda_2}$. Тогда:
	\[\lambda_1(\overline{v_1}, \overline{v_2}) = (\phi(\overline{v_1}), \overline{v_2}) = (\overline{v_1}, \phi(\overline{v_2})) = \lambda_2(\overline{v_1}, \overline{v_2}) \Rightarrow (\overline{v_1}, \overline{v_2}) = 0\qedhere\]
\end{proof}

\begin{theorem}
	Пусть $\phi \in \mathcal{L}(V)$ "--- самосопряженный. Тогда в $V$ существует ортонормированный базис $e$, в котором матрица оператора $\phi$ диагональна.
\end{theorem}

\begin{proof}
	Проведем индукцию по $n := \dim{V}$. База, $n = 1$, тривиальна. Пусть теперь $n > 1$. Поскольку корни $\chi_\phi$ вещественны, то у $\phi$ есть собственное значение $\lambda_0 \in \mathbb{R}$. Пусть $\overline{e_0} \in V$ "--- соответствующий ему собственный вектор длины $1$. Тогда подпространство $U \hm{:=} \langle\overline{e_0}\rangle^\perp$ инвариантно относительно $\phi$, поэтому можно рассмотреть оператор $\phi|_{U} \in \mathcal{L}(U)$, который также является самосопряженным. По предположению индукции, в $U$ есть ортонормированный базис из собственных векторов, тогда его объединение с $\overline{e_0}$ дает искомый базис в $V$.
\end{proof}

\begin{note}~
	\begin{enumerate}
		\item Пусть $\phi$ имеет диагональный вид в некотором ортонормированном базисе. Тогда в евклидовом случае $\phi$ "--- самосопряженный, поскольку имеет симметричную матрицу, а в эрмитовом случае $\phi$ "--- самосопряженный $\Leftrightarrow$ все элементы на диагоналы вещественны.
		\item Геометрический смысл самосопряженного оператора $\phi$ "--- это композиция растяжений вдоль взаимно ортогональных осей.
		\item На практике при диагонализации самосопряженного оператора $\phi$ удобнее искать ортонормированные базисы отдельно в каждом его собственном подпространстве.
	\end{enumerate}
\end{note}