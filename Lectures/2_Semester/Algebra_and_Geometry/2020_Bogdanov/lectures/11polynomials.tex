\section{Многочлены}

\subsection{Кольцо многочленов}

\begin{note}
	В общем случае, определять многочлены как функции не вполне правильно. Например, полезно различать многочлены $P(x) = x$ и $Q(x) = x^p$ над полем $\mathbb{Z}_p$, однако для любого $x \in \mathbb{Z}_p$ выполнено равенство $P(x) = Q(x)$. Поэтому нам потребуется другое определение.
\end{note}

\begin{definition}
	Пусть $K$ "--- коммутативное кольцо. Последовательность $(a_0, a_1, \dots)$ элементов из $K$ называется \textit{финитной}, если она содержит конечное число ненулевых элементов. Обозначение финитной последовательности "--- $(a_i)$.
\end{definition}

\begin{definition}
	Для финитных последовательностей $(a_i)$ и $(b_i)$ можно определить операции сложения и умножения:
	\begin{itemize}
		\item $(a_i) + (b_i) := (a_i + b_i)$
		\item $(a_i)(b_i) := (c_k)$, $c_k = \sum_{i + j = k}a_ib_j$
	\end{itemize}
\end{definition}

\begin{note}
	Последовательность $(c_k)$ действительно финитна: поскольку $(a_i)$ и $(b_i)$ финитны, то существует число $N \in \N$ такое, что для любого $i \in \N$, $i > N$, выполнены равенства $a_i = 0$ и $b_i = 0$, поэтому для любого $k \in \N$, $k > 2N$, выполнено равенство $c_k = 0$.
\end{note}

\begin{theorem}
	Пусть $R$ "--- множество всех финитных последовательностей над коммутативным кольцом $K$. Тогда $(R, +, \cdot)$ также является коммутативным кольцом.
\end{theorem}

\begin{proof}~
	\begin{enumerate}
		\item Покажем сначала, что $(R, +)$ "--- абелева группа, пользуясь тем, что $(K, +)$ "--- абелева группа:
		\begin{itemize}
			\item $\forall (a_i), (b_i) \in R: (a_i) + (b_i) = (a_i + b_i) = (b_i + a_i) = (b_i) + (a_i)$
			\item $\forall (a_i), (b_i), (c_i) \in R:  ((a_i) + (b_i)) + (c_i) = (a_i + b_i + c_i) \hm{=} (a_i) + ((b_i) + (c_i))$
			\item $\exists 0 := (0, 0, 0, \dots) \in R: \forall (a_i) \in R: (a_i) + 0 = (a_i)$
			\item $\forall (a_i) \in R: \exists\! -(a_i) = (-a_i) \in R: (a_i) + (-(a_i)) = 0$
		\end{itemize}
	
		Последние два свойства достаточно проверять <<с одной стороны>> в силу коммутативности сложения в $R$.
		
		\item Покажем теперь, что $(R, +, \cdot)$ "--- коммутативное кольцо. Это, в свою очередь, следует из того, что $(K, +, \cdot)$ "--- коммутативное кольцо:
		\begin{itemize}
			\item $\forall (a_i), (b_i) \in R: (a_i)(b_i) = \big(\sum_{j + k = i}a_jb_k\big) = (b_i)(a_i)$
			
			\item Заметим, что для любых $(a_i), (b_i), (c_i) \in R$ выполнены следующие равенства:
			\[((a_i)(b_i))(c_i) = \left(\sum_{j + k = i}a_jb_k\right)(c_i) = \left(\sum_{l + m = i}\left(\sum_{j + k = l}a_jb_k\right)c_m\right) \hm{=} \left(\sum_{j+k+m = i}a_jb_kc_m\right)\]
			
			Поскольку последовательность $(a_i)((b_i)(c_i))$ можно привести к такому же виду, то $((a_i)(b_i))(c_i) = (a_i)((b_i)(c_i))$.
			\item $(a_i), (b_i), (c_i) \in R: (a_i)((b_i) + (c_i)) = \big(\sum_{j + k = i}(a_jb_k + a_jc_k)\big) \hm{=} (a_i)(b_i) + (a_i)(c_i)$
			\item $\exists 1 := (1, 0, 0, \dots) \in R: \forall (a_i) \in R: (a_i)1 = (a_i)$
		\end{itemize}
	
		Последние два свойства также достаточно проверять <<с одной стороны>> в силу коммутативности умножения в $R$.\qedhere
	\end{enumerate}
\end{proof}

\begin{note}
	Положим $x := (0, 1, 0, 0, \dots)$, тогда $x^k = (0, \dotsc, 0, \overset{(k)}{1}, 0, \dotsc)$ по правилам умножения в $R$. В таких обозначениях $(a_i) = (a_1, \dotsc, n, 0, \dotsc)$ можно представить в виде $(a_i) = a_0 + a_1x + \dots + a_nx^n$, где $a_i \equiv (a_i, 0, 0,\dots)$ для каждого $i \in \{1, \dotsc, n\}$. Более того, такое представление единственно: если $(a_i) = b_0 + b_1x + \dots + b_nx^n$, то $b_i = a_i$ для каждого $i \in \{1, \dotsc, n\}$.
\end{note}

\begin{definition}
	Пусть $K$ "--- коммутативное кольцо. Кольцо финитных последовательностей элементов из $K$ называется \textit{кольцом многочленов над $K$}. Обозначение "--- $K[x]$.
\end{definition}

\begin{definition}
	Пусть $K$ "--- коммутативное кольцо. \textit{Степенью многочлена $P \in K[x]$} называется позиция последнего ненулевого элемента в $P$. Обозначение "--- $\deg{P}$. Считается также, что $\deg{0} = -\infty$.
\end{definition}

\begin{note}
	Если не требовать от последовательностей финитности, то построенное аналогичным образом кольцо будет называться \textit{кольцом формальных степенных рядов над $K$}. Обозначение "--- $K[[x]]$.
\end{note}

\begin{definition}
	Коммутативное кольцо $K$ называется \textit{целостным}, если для любых элементов $a, b \in K \backslash \{0\}$ выполнено $ab \ne 0$.
\end{definition}

\begin{example}
	Рассмотрим несколько примеров целостных колец:
	\begin{itemize}
		\item Поле $F$ является целостным кольцом: если для некоторых $a, b \in F^*$ выполнено равенство $ab = 0$, то, умножая обе его части на $a^{-1}$, получим, что $b = 0$ --- противоречие
		\item Кольцо $\mathbb{Z}$ является целостным
	\end{itemize}
\end{example}

\begin{note}
	В отличие от поля $\mathbb{Z}_p$ при простом $p$, кольцо $\Z_n$ при составном $n$ не является целостным: если $n = ab$ для некоторых $a, b \in \N$ таких, что $a, b > 1$, то $\overline{a}, \overline{b} \ne \overline{0}$, но $\overline{a}\overline{b} = \overline{0}$
\end{note}

\begin{proposition}
	В целостном кольце $K$ можно <<сокращать>>, то есть для любых $a, b, c \in K$ таких, что $ab = ac$ и $a \ne 0$, выполнено $b = c$.
\end{proposition}

\begin{proof}
	Поскольку $a(b - c) = 0$ и кольцо $K$ "--- целостное, то один из множителей $a$, $(b - c)$ равен $0$. По условию, $a \ne 0$, поэтому $b - c = 0$, откуда $b = c$.
\end{proof}

\begin{proposition}
	Пусть $K$ "--- коммутативное кольцо, $P, Q \hm{\in} K[x]$. Тогда:
	\begin{enumerate}
		\item $\deg{(P + Q)} \le \max\{\deg{P}, \deg{Q}\}$
		\item $\deg{PQ} \le \deg{P} + \deg{Q}$, причем если $K$ "--- целостное, то $\deg{PQ} = \deg{P} + \deg{Q}$
	\end{enumerate}
\end{proposition}

\begin{proof}~
	\begin{enumerate}
		\item Пусть $n := \max\{\deg{P}, \deg{Q}\}$, тогда для любого $i \in \N$, $i > n$, выполнено равенство $p_i + q_i = 0$.
		\item Положим $n := \deg{P}$, $m := \deg{Q}$ и представим многочлены $P$ и $Q$ в виде $\sum_{i = 0}^{n}p_ix^i$ и $\sum_{j = 0}^{m}q_jx^j$ соответственно. Тогда выполнено следующее равенство:
		\[PQ = \sum_{i = 0}^{n}\sum_{j = 0}^{m}p_iq_jx^{i + j}\]
		
		Значит, $\deg{PQ} \le m + n$. Более того, коэффициент при $x^{n + m}$ равен $p_nq_m$, поэтому если $K$ "--- целостное, то $p_nq_m \ne 0$ и $\deg{(PQ)} = m + n$.\qedhere
	\end{enumerate}
\end{proof}

\begin{corollary}
	Если кольцо $K$ "--- целостное, то кольцо $K[x]$ "--- тоже целостное.
\end{corollary}

\begin{proof}
	Пусть $P, Q \in K[x] \bs \{0\}$, тогда $\deg{P}, \deg{Q} \ge 0$. Но тогда выполнено равенство $\deg{PQ} = \deg{P} + \deg{Q} \ge 0$, откуда $PQ \ne 0$.
\end{proof}

\begin{note}
	Если $F$ "--- поле, то $F[x]$ "--- алгебра над $F$.
\end{note}

\begin{definition}
	\textit{Гомоморфизмом колец} $R$ и $S$ называется отображение $\phi: R \rightarrow S$ такое, что для любых элементов $a, b \in R$ выполнены равенства $\phi(a + b) = \phi(a) + \phi(b)$и $\phi(ab) = \phi(a)\phi(b)$, а также $\phi(1) = 1$.
\end{definition}

\begin{definition}
	\textit{Гомоморфизмом алгебр} $R$ и $S$ над полем $F$ называется отображение $\phi: R \rightarrow S$, являющееся одновременно линейным отображением и гомоморфизмом колец.
\end{definition}

\begin{proposition}
	Пусть $A$ "--- алгебра над полем $F$, $a \in A$. Тогда существует единственный гомоморфизм алгебр $\phi: F[x] \rightarrow A$ такой, что $\phi(x) = a$.
\end{proposition}

\begin{proof}
	Покажем, что искомый гомоморфизм $\phi$ не более чем единственен. Если он существует, то, в силу свойств гомоморфизма колец, для любого $n \in \mathbb{N} \cup \{0\}$ выполнено равенство $\phi(x^n) \hm{=} a^n$, тогда для любого многочлена $P = p_0 + \dotsb + p_nx^n \in F[x]$ значение $\phi(P)$ определяется однозначно:
	\[\phi(P) \hm{=} \sum_{i = 0}^{n}p_ia^i\]
	
	Покажем теперь, что определенное таким образом отображение действительно является гомоморфизмом алгебр. Оно, очевидно, является линейным отображением, и, кроме того, $\phi(1) \hm{=} 1$. Остается проверить лишь свойство мультипликативности. Действительно, для любых $P = p_0 + \dotsb + p_nx^n, Q = q_0 + \dotsb + q_mx^m \in F[x]$ выполнено следующее:
	\[\phi(P)\phi(Q) = \sum_{i = 0}^np_ia^i\sum_{j = 0}^mq_ja^j = \sum_{k = 0}^{n + m}\left(\sum_{i + j = k}p_iq_j\right)a^k = \phi(PQ)\]
	
	Таким образом, $\phi$ "--- гомоморфизм алгебр.
\end{proof}

\begin{definition}
	Пусть $A$ "--- алгебра над полем $F$. \textit{Значением многочлена $P \in F[x]$ в точке $a \in A$} называется $P(a) := \phi(P)$, где $\phi$ "--- \textit{гомоморфизм подстановки} из утверждения выше.
\end{definition}

\begin{note}
	Для любых многочленов $P, Q \in F[x]$ и любого $a \in A$ выполнены следующие равенства:
	\begin{itemize}
		\item $(PQ)(a) = P(a)Q(a)$
		\item $(P + Q)(a) = P(a) + Q(a)$
	\end{itemize}
\end{note}

\begin{example}
	Пусть $A = \mathcal{L}(V)$, где $V$ "--- некоторое линейное пространство над полем $F$, $\Theta \in A$ и $P(x) = x^2 + 3x + 2$. Тогда $\phi(\Theta) = \Theta^2 + 3\Theta + 2 = (\Theta + 1)(\Theta + 2)$, причем под единицей в $A$ понимается тождественное отображение $\id$.
\end{example}