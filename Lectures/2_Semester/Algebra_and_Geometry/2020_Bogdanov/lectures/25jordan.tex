\section{Жорданова нормальная форма и ее приложения}

\subsection{Жорданова нормальная форма}

\begin{definition}
	Пусть $F$ "--- поле, $\lambda_0 \in F$. \textit{Жордановой клеткой} размера $k \in \N$ с собственным значением $\lambda_0$ называется матрица $J_k(\lambda_0) \in M_k(F)$, имеющая следующий вид:
	\[J_k(\lambda_0) := \begin{pmatrix}
		\lambda_0 & 1 & \dots & 0 & 0 \\ 
		0 & \lambda_0 & \ddots & \ddots & 0 \\ 
		\vdots & \ddots & \ddots & \ddots & \vdots \\ 
		0 & \ddots & \ddots & \lambda_0 & 1 \\ 
		0 & 0 & \dots & 0 & \lambda_0
	\end{pmatrix}\]
	
	Cчитается также, что $J_k := J_k(0)$.
\end{definition}

\begin{definition}
	Пусть $F$ "--- поле, $A \in M_n(F)$. Матрица $A$ имеет \textit{жорданов вид}, если она имеет блочно-диагональный вид, в котором каждый блок является жордановой клеткой, то есть имеет следующий вид для некоторых $k_1, \dotsc, k_m \in \N$ и $\lambda_1, \dotsc, \lambda_m \in F$:
	\[A = \left(\begin{array}{@{}cccc@{}}
		\cline{1-1}
		\multicolumn{1}{|c|}{J_{k_1}(\lambda_1)} & 0 & \dots & 0\\
		\cline{1-2}
		0 & \multicolumn{1}{|c|}{J_{k_2}(\lambda_2)} & \dots & 0\\
		\cline{2-2}
		\vdots & \vdots & \ddots & \vdots\\
		\cline{4-4}
		0 & 0 & \dots & \multicolumn{1}{|c|}{J_{k_m}(\lambda_m)}\\
		\cline{4-4}
	\end{array}\right)\]
\end{definition}

\begin{definition}
	Пусть $V$ "--- линейное пространство, $\phi \in \mc L(V)$. Если в некотором базисе матрица оператора $\phi$ имеет жорданов вид, то она называется \textit{жордановой нормальной формой} оператора $\phi$, а соответствующий базис "--- его \textit{жордановым базисом}.
\end{definition}

\textbf{До конца раздела} зафиксируем линейное пространство $V$ над полем $F$ и положим $n := \dim{V}$, а также зафиксируем оператор $\phi \in \mathcal L(V)$ такой, что  $\Chi_\phi(\lambda)$ имеет вид $\eqref*{charpol}$.

\begin{note}
	В силу предположения выше, многочлен $\Chi_\phi(\lambda)$ можно представить в виде произведения многочленов $P_1 := (\lambda_1 - \lambda)^{\alpha_1}, \dotsc, P_k := (\lambda_k - \lambda)^{\alpha_k}$. Эти многочлены попарно взаимно просты.
\end{note}

\begin{definition}
	Пусть $\lambda_i$ "--- собственное значение оператора $\phi$. \textit{Корневым подпространством}, соответствующим $\lambda_i$, называется $V^{\lambda_i} := \ke{P_i(\phi)} \hm{=} \ke(\phi_{\lambda_i})^{\alpha_i}$.
\end{definition}

\begin{note}
	Поскольку $V_{\lambda_i} = \ke{\phi_{\lambda_i}}$, то $V_{\lambda_i} \le V^{\lambda_i}$, но обратное включение верно не всегда. Кроме того, $V = V^{\lambda_1} \oplus \dotsb \oplus V^{\lambda_k}$ по уже доказанному утверждению.
\end{note}

\begin{proposition}
	Пусть $\lambda_i$, $\lambda_j$ "--- различные собственные значения оператора $\phi$. Тогда оператор $\phi_{\lambda_i}|_{V^{\lambda_j}} \in \mathcal{L}(V^{\lambda_j})$ "--- невырожденный.
\end{proposition}

\begin{proof}
	Отметим сначала, что сужение $\phi_{\lambda_i}|_{V^{\lambda_j}}$ корректно, поскольку $V^{\lambda_j}$ инвариантно относительно $\phi$. Тогда, поскольку выполнены равенства $V^{\lambda_i} \cap V^{\lambda_j} = \{\overline{0}\}$ и $V_{\lambda_i} = \ke{\phi_{\lambda_i}} \le V^{\lambda_i}$, имеем $\ke{\phi_{\lambda_i}|_{V^{\lambda_j}}} = V^{\lambda_j} \cap \ke{\phi_{\lambda_i}} \hm{=} \{\overline{0}\}$, что и означает невырожденность оператора.
\end{proof}

\begin{proposition}
	Пусть $\lambda_i$ "--- собственное значение оператора $\phi$. Тогда выполнено следующее равенство:
	\[V^{\lambda_i} = \{\overline{v} \in V: \exists m \in \mathbb{N}: (\phi_{\lambda_i})^m(\overline{v}) = \overline{0}\}\]
\end{proposition}

\begin{proof}
	Нетривиально только включение $(\ge)$. Пусть $\overline{v} \in V$ "--- вектор такой, что $(\phi_{\lambda_i})^m(\overline{v}) = \overline{0}$ для некоторого $m \in \N$. Представим его в виде $\overline{v} = \overline{v_1} + \dots + \overline{v_k}$, где $\overline{v_j} \in V^{\lambda_j}$ для каждого $j \in \{1, \dotsc, k\}$, тогда выполнено следующее равенство:
	\[(\phi_{\lambda_i})^m(\overline{v_1}) + \dots +  (\phi_{\lambda_i})^m(\overline{v_k}) = \overline{0}\]
	
	В силу инвариантности, для каждого $j \in \{1, \dotsc, k\}$ выполнено $(\phi_{\lambda_i})^m(\overline{v_j}) \in V^{\lambda_j}$, поэтому каждый такой вектор равен $\overline 0$ по свойству прямой суммы. Но для каждого индекса $j \in \{1, \dotsc n\} \bs \{i\}$ оператор $\phi_{\lambda_i}|_{V^{\lambda_j}}$ "--- невырожденный, откуда $\overline{v_j} = \overline{0}$. Значит, $\overline{v} = \overline{v_i} \in V^{\lambda_i}$, и получено требуемое.
\end{proof}

\begin{note}
	Утверждение выше означает, что если вектор $\overline{v} \in V$ обнуляется под действием какой-либо степени оператора $\phi_{\lambda_i}$, то он также обнуляется под действием этого оператора в некоторой степени, не превосходящей $\alpha_i$.
\end{note}

\begin{definition}
	Пусть $\psi \in \mathcal{L}(V)$. Оператор $\psi$ называется \textit{нильпотентным}, если существует $m \in \mathbb{N}$ такое, что $\psi^m = 0$.
\end{definition}

\begin{note}
	Все собственные значения оператора являются корнями любого аннулирующего многочлена этого оператора, а для нильпотентного оператора $\psi$ многочлен $x^m$ является аннулирующим, поэтому его единственное собственное значение "--- это $0$.
\end{note}

\begin{definition}
	Пусть $\psi \in \mathcal{L}(V)$ "--- нильпотентный оператор. Подпространство $U \le V$ называется \textit{циклическим} относительно $\psi$, если оно инвариантно относительно $\psi$ и существует вектор $\overline{v} \in V$ такой, что $U$ "--- минимальное по включению инвариантное подпространство, содержащее $\overline{v}$.
\end{definition}

\begin{note}
	Если подпространство $U \le V$ "--- циклическое относительно нильпотентного оператора $\psi \in \mc L(V)$ и вектора $\overline{v} \in V$, то $U = \langle\overline{v}, \psi(\overline{v}), \psi^2(\overline{v}), \dots\rangle$, причем порождающий набор конечен в силу нильпотентности оператора $\psi$.
\end{note}

\begin{definition}
	Пусть $\psi \in \mathcal{L}(V)$ "--- нильпотентный оператор, $\overline{v} \in V \bs \{\overline 0\}$. \textit{Высотой} вектора $\overline{v}$ относительно $\psi$ называется наименьшее $n \in \mathbb{N}$ такое, что $\psi^n(\overline{v}) = \overline{0}$.
\end{definition}

\begin{proposition}
	Пусть $\psi \in \mathcal{L}(V)$ "--- нильпотентный оператор, и пусть векторы ${\overline{v_1}, \dots, \overline{v_k} \in V \bs \{\overline 0\}}$ имеют попарно различные высоты относительно $\psi$. Тогда эти векторы образуют линейно независимую систему.
\end{proposition}

\begin{proof}
	Предположим, что это не так, тогда существует нетривиальная линейная комбинация с коэффициентами $\alpha_1, \dotsc, \alpha_k \in F$, равная нулю:
	\[\alpha_1\overline{v_1} + \dots + \alpha_k\overline{v_k} = \overline{0}\]
	
	Пусть $\overline{v_i}$ "--- вектор с наибольшей высотой $n_i$, коэффициент при котором не равен нулю. Применяя к данному равенству $\psi^{n_i - 1}$, получим, что $\alpha_i\psi^{n_i-1}(\overline{v_i}) = \overline{0}$. Значит, $\alpha_i = 0$, что противоречит нашему предположению.
\end{proof}

\begin{corollary}
	Пусть $\psi \in \mathcal{L}(V)$ "--- нильпотентный оператор, и пусть $U \le V$ "--- циклическое подпространство, порожденное вектором $\overline{v} \in V \bs \{\overline 0\}$ высоты $n$. Тогда система $(\overline{v}, \psi(\overline{v}), \dots, \psi^{n-1}(\overline{v}))$ образует базис в $U$.
\end{corollary}

\begin{proof}
	Как уже было отмечено, $U = \langle\overline{v}, \psi(\overline{v}), \psi^2(\overline{v}), \dots\rangle$, тогда, поскольку высота веткора $\overline{v}$ равна $n$, имеем $U = \langle\overline{v}, \psi(\overline{v}), \dots, \psi^{n - 1}(\overline{v})\rangle$. Кроме того, система $(\overline{v}, \psi(\overline{v}), \dots, \psi^{n-1}(\overline{v}))$ линейно независима по утверждению выше.
\end{proof}

\begin{note}
	Матрица оператора $\psi$ в базисе $(\psi^{n-1}(\overline{v}), \dots, \psi(\overline{v}), \overline{v})$ имеет вид жордановой клетки $J_n \in M_n(F)$.
\end{note}

\begin{proposition}
	Пусть $\psi \in \mathcal{L}(V)$ "--- нильпотентный оператор, $\overline{v} \in V \bs \{\overline 0\}$ "--- вектор наибольшей высоты $n$ в пространстве $V$, и $U \le V$ "--- циклическое подпространство, порожденное $\overline{v}$. Тогда существует $W \le V$ такое, что $W$ инвариантно относительно $\psi$ и $V = U \oplus V$.
\end{proposition}

\begin{proof}
	Отметим сначала, что вектор $\overline{v}$ из условия определен корректно, поскольку все векторы из $V \bs \{\overline 0\}$ имеют конечную высоту, ограниченную сверху величиной $\dim{V}$. Выберем инвариантное подпространство $W \le V$ наибольшей размерности такое, что $W \cap U \hm{=} \{\overline{0}\}$. Такое подпространство точно существует, потому что по меньшей мере $\{\overline{0}\} \le V$ удовлетворяет условию. Если $U \oplus W = V$, то утверждение доказано. Если же $U \oplus W \hm{\ne} V$, то выберем $\overline{x} \not\in U \oplus W$. Поскольку в наборе $\overline{x}, \psi(\overline{x}), \psi^2(\overline{x}), \dots, \overline{0}$ первый вектор не лежит в $U \oplus W$, а последний --- лежит, то в некоторый момент происходит <<скачок>> из-за пределов подпространства $U \oplus W$ в $U \oplus W$. Пусть без ограничения общности это происходит на первом шаге, то есть $\psi(\overline{x}) \in U \oplus W$. По свойству прямой суммы, для некоторых скаляров $\alpha_0, \dotsc, \alpha_{n-1} \in F$ и вектора $w \in W$ выполнено следующее равенство:
	\[\psi(\overline{x}) = \alpha_0\overline{v} + \dots + \alpha_{n - 1}\psi^{n-1}(\overline{v}) + \overline{w}\]
	
	Поскольку $n$ "--- наибольшая высота в $V$, то $\overline{0} = \psi^n(\overline{x}) = \psi^{n-1}(\psi(\overline{x}))$. Тогда, применив оператор $\psi^{n - 1}$ к обеим частям равенства, получим следующее:
	\[\overline{0} = \alpha_0\psi^{n - 1}(\overline{v}) + \psi^{n - 1}(\overline{w})\]
	
	Поскольку сумма $U \oplus W$ "--- прямая, то оба слагаемых в правой части равенства равны нулю, то есть $\alpha_0 = 0$ и $\psi^{n - 1}(\overline{w}) = \overline{0}$. Положим теперь $\overline{x'} := \overline{x} - \alpha_1\overline{v} - \dots - \alpha_{n-1}\psi^{n-1}(\overline{v})$ и заметим, что $\overline{x'} \not\in U \oplus W$, поскольку $\overline x \not\in U \oplus W$. Кроме того, выполнено следующее:
	\[\psi(\overline{x'}) = \psi(\overline{x}) - \alpha_1\overline{v_1} - \dots - \alpha_{n-1}\overline{v_{n-1}} = \overline{w}\]
	
	Значит, пространство $W \oplus \langle\overline{x'}\rangle$ "--- тоже инвариантное, причем $(W \oplus \langle\overline{x'}\rangle) \cap U = \{\overline{0}\}$. Получено противоречие с максимальностью размерности подпространства $W$.
\end{proof}

\begin{corollary}
	Пусть $\psi \in \mathcal{L}(V)$ "--- нильпотентный оператор. Тогда $V$ раскладывается в прямую сумму циклических относительно $\psi$ подпространств.
\end{corollary}

\begin{proof}
	Проведем индукцию по размерности пространства $V$. База, $\dim{V} = 0$, тривиальна, докажем переход. Выберем в $V$ вектор $\overline{v}$ наибольшей высоты и порожденное им циклическое подпространство $U$, а также $W \le V$ такое, что $W$ инвариантно относительно $\psi$ и  $V = U \oplus W$. Для подпространства $W$ и оператора $\psi|_W \in \mc L(W)$ применимо предположение индукции.
\end{proof}

\begin{note}
	Размерности циклических подпространств, построенных в доказательстве выше, образуют невозрастающую последовательность.
\end{note}

\begin{theorem}[о существовании жордановой нормальной формы]
	Пусть $\phi \in \mathcal{L}(V)$, и $\Chi_\phi$ имеет вид $\eqref*{charpol}$. Тогда у оператора $\phi$ есть жорданова нормальная форма.
\end{theorem}

\begin{proof}
	Представим $V$ в виде прямой суммы корневых подпространств:
	\[V = V^{\lambda_1} \oplus \dots \oplus V^{\lambda_k}\]

	Для любого $i \in \{1, \dotsc, k\}$ оператор $\phi_{\lambda_i}|_{V^{\lambda_i}} \in \mc L({V^{\lambda_i}})$ "--- нильпотентный, поэтому он раскладывается в прямую сумму циклических подпространств и, как следствие, имеет жорданов базис $e_i$. В этом базисе оператор $\phi_{\lambda_i}|_{V^{\lambda_i}}$ имеет жорданову нормальную форму с нулями на главной диагонали, то есть для некоторых $k_1, \dotsc, k_m \in \N$ выполнено следующее:
	\[\phi_{\lambda_i}|_{V^{\lambda_i}} \leftrightarrow_{e_i} \left(\begin{array}{@{}ccc@{}}
		\cline{1-1}
		\multicolumn{1}{|c|}{J_{k_1}} & \dots & 0\\
		\cline{1-1}
		\vdots & \ddots & \vdots\\
		\cline{3-3}
		0 & \dots & \multicolumn{1}{|c|}{J_{k_m}}\\
		\cline{3-3}
	\end{array}\right)\]
	
	В этом же базисе $e_i$ оператор $\phi|_{V^{\lambda_i}} \in \mc L({V^{\lambda_i}})$ имеет жорданову нормальную форму следующего вида:
	\[\phi|_{V^{\lambda_i}} \leftrightarrow_{e_i} \left(\begin{array}{@{}ccc@{}}
		\cline{1-1}
		\multicolumn{1}{|c|}{J_{k_1}(\lambda_i)} & \dots & 0\\
		\cline{1-1}
		\vdots & \ddots & \vdots\\
		\cline{3-3}
		0 & \dots & \multicolumn{1}{|c|}{J_{k_m}(\lambda_i)}\\
		\cline{3-3}
	\end{array}\right)\]
	
	Объединение жордановых базисов в подпространствах $V^{\lambda_1},\dotsc, V^{\lambda_k}$ дает искомый жорданов базис в $V$.
\end{proof}

\begin{note}
	Существует и более конструктивный подход к получению жорданова базиса для нильпотентного оператора. Опишем его ниже.
	
	Пусть $\psi \in \mathcal{L}(V)$ "--- нильпотентный, $k$ "--- наибольшая высота вектора в $V$ относительно $\psi$. Для каждого $i \in \{1, \dotsc, k\}$ положим $V_i := \ke\psi^i$ "--- пространство векторов высоты, не превосходящей $i$. Выберем $U_k$ "--- прямое дополнение  подпространства $V_{k - 1}$ в $V_k$, тогда все ненулевые векторы в $U_k$ имеют высоту $k$, поэтому $\psi(U_k) \le V_{k - 1}$, причем все ненулевые векторы в $\psi(U_k)$ имеют высоту $k - 1$, откуда $\psi(U_k) \cap V_{k - 2} = \{\overline{0}\}$. Значит, можно также выбрать $U_{k-1}$ "--- такое прямое дополнение подпространства $V_{k - 2}$ в $V_{k - 1}$, что $\psi(U_k) \le U_{k - 1}$. Продолжая процесс, получим $U_k, \dots, U_1 \le V$ такие, что для каждого $i \hm{\in} \{1, \dots, k - 1\}$ выполнено $\psi(U_{i+1}) \le U_i$. Для каждого $i \in \{1, \dots, k - 1\}$ также выберем $W_i$ "--- прямое дополнение подпространства $\psi(U_{i + 1})$ в $U_i$. Тогда пространство $V$ примет следующий вид:
	\[\begin{array}{@{}rccccc@{}}
		\cline{2-2}
		U_k \left\{\right.&\multicolumn{1}{|c|}{U_{k}} &&&&\\
		\cline{2-3}
		U_{k - 1} \left\{\right.&\multicolumn{1}{|c|}{\psi(U_k)} & \multicolumn{1}{c|}{W_{k - 1}} &&&\\
		\cline{2-4}
		U_{k - 2} \left\{\right.&\multicolumn{2}{|c|}{\psi(U_{k - 1})} & \multicolumn{1}{c|}{W_{k - 2}} &&\\
		\cline{2-4}
		&\vdots & \vdots & \vdots & \ddots &\\
		\cline{2-6}
		U_1 \left\{\right.&\multicolumn{4}{|c|}{\psi(U_2)} & \multicolumn{1}{c|}{W_1}\\
		\cline{2-6}
	\end{array}\]
	
	Заметим теперь, что для любого $i \in \{1, \dots, k - 1\}$ линейно независимая система $(\overline{v_1}, \dots, \overline{v_t})$ векторов из $U_{i+1}$ под действием $\psi$ переходит в линейно независимую систему $(\psi(\overline{v_1}), \dots, \psi(\overline{v_t}))$ векторов из $U_{i}$. Действительно, любая нетривиальная линейная комбинация системы имеет высоту $i$ и потому не обращается в ноль под действием $\psi$. Значит, если на каждой <<ступеньке>> $U_i$ выбрать базис $e_i$, то образ этого базиса $\psi(e_i)$ будет базисом в $\psi(U_i)$, который можно будет дополнить до базиса $e_{i - 1}$ в $U_{i - 1}$. Тогда система $e_k \cup \dotsb \cup e_1$ и будет искомым жордановым базисом в $V$, а каждая вертикальная <<цепочка>> вида $\overline{v}, \psi(\overline{v}), \dots$ будет порождать очередное циклическое подпространство $C$, и сумма таких циклических подпространств будет прямой и равной $V$.
\end{note}

\begin{note}
	Диагональный вид матрицы также является жордановым видом: каждый элемент главной диагонали "--- это жорданова клетка размера 1.
\end{note}

\begin{theorem}[о единственности жордановой нормальной формы]
	Пусть $\phi \in \mathcal{L}(V)$, и $\Chi_\phi$ имеет вид $\eqref*{charpol}$. Тогда жорданова нормальная форма оператора $\phi$ единственна с точностью до перестановки клеток.
\end{theorem}

\begin{proof}
	Пусть $\lambda_0 \in F$ "--- собственное значение оператора $\phi$. Выберем жорданов базис $e = (\overline{e_1}, \dotsc, \overline{e_n})$ такой, что $\phi \leftrightarrow_e A \in M_n(F)$, где $A$ "--- жорданова нормальная форма, в которой все клетки со значением $\lambda_0$ стоят в начале и имеют суммарный размер $d \le n$, тогда этим клеткам соответствует начальный фрагмент базиса $(\overline{e_1}, \dotsc, \overline{e_d})$. Обозначим размеры этих клеток через $k_1, \dotsc, k_s$, тогда $\sum_{j = 1}^sk_j = d$. Достаточно показать, что набор $\{k_1, \dots, k_s\}$ определен однозначно, поскольку для клеток с другими собственными значениями рассуждение будет аналогичным.
	
	Пусть $\alpha_0$ "--- алгебраическая кратность значения $\lambda_0$, тогда выполнено равенство $d \hm{=} \alpha_0$. Рассмотрим оператор $\phi_{\lambda_0}$ и заметим, что $\phi_{\lambda_0} \leftrightarrow_e A_{\lambda_0} = A -\lambda_0E$, то есть первые $d$ элементов на главной диагонали $A_{\lambda_0}$ равны нулю, а остальные --- отличны от нуля. При возведении матрицы $A_{\lambda_0}$ в некоторую степень каждая клетка возводится в степень независимо, причем ранг вырожденной клетки в каждой следующей степени уменьшается на один, пока клетка не станет нулевой, а невырожденные клетки остаются невырожденными. Значит, $\rk{(A_{\lambda_0})^d} = n - d$, и выполнены следующие равенства:
	\[\dim{V^{\lambda_0}} = \dim{\ke{(\phi_{\lambda_0})^d}} = n - \rk{(A_{\lambda_0})^d} = d\]
	
	Кроме того, поскольку $V^{\lambda_0}$ "--- это пространство всех векторов, обнуляемых оператором $(\phi_{\lambda_0})^d$, то $\langle\overline{e_1}, \dots, \overline{e_d}\rangle = V^{\lambda_0}$. Исследуем нильпотентный оператор $\psi := \phi_{\lambda_0}|_{V^{\lambda_0}} \in \mc L(V^{\lambda_0})$. Его матрица в базисе $e' := (\overline{e_1}, \dots, \overline{e_d})$ имеет следующий вид:
	\[\psi \leftrightarrow_{e'} B := \left(\begin{array}{@{}ccc@{}}
		\cline{1-1}
		\multicolumn{1}{|c|}{J_{k_1}} & \dots & 0\\
		\cline{1-1}
		\vdots & \ddots & \vdots\\
		\cline{3-3}
		0 & \dots & \multicolumn{1}{|c|}{J_{k_s}}\\
		\cline{3-3}
	\end{array}\right)\]
	
	Пусть $n_1$ "--- число клеток размера $\ge 1$, $n_2$ "--- число клеток размера $\ge 2$, и так далее. Число клеток размера $j \in \{1, \dotsc, d\}$ равно $n_j - n_{j + 1}$, поэтому для определения числа клеток каждого размера достаточно найти все числа $n_1, \dotsc, n_d$. Для каждого $i \in \{1, \dotsc, d\}$ положим $V_i := \ke{\psi^i}$, тогда $V_{\lambda_0} \hm{=} V_1 \le V_2 \hm{\le} \dots \hm{\le} V_d = V^{\lambda_0}$. Чтобы определить величины $\dim{V_1}, \dots, \dim{V_d}$, снова воспользуемся замечанием о том, что возведение клетки $J_k$ в каждую следующую степень уменьшает ее ранг на один, пока клетка не станет нулевой:
	\begin{align*}
		\dim{V_1} &= \dim{\ke{\psi}} = d - \rk{B} = n_1\\
		\dim{V_2} &= \dim{\ke{\psi^2}} = (d - \rk{B}) + (\rk{B} - \rk{B^2}) = n_1 + n_2\\
		\dim{V_3} &= \dim{\ke{\psi^3}} = (d - \rk{B^2}) + (\rk{B^2} - \rk{B^3}) = n_1 + n_2 + n_3\\
		&\dots\\
		\dim{V_d} &= \dim{\ke{\psi^d}} = (d - \rk{B^{d - 1}}) + (\rk{B^{d - 1}} - \rk{B^d})= \sum_{j = 1}^dn_j
	\end{align*}
	
	Таким образом, числа $n_1, \dotsc, n_d$ выражаются через величины $\dim{V_1}, \dotsc, \dim{V_d}$ вне зависимости от выбора базиса, и по ним однозначно определяется набор $\{k_1, \dots, k_s\}$. Таким образом, жорданова нормальная форма оператора $\phi$ определена однозначно с точностью до перестановки клеток задается свойствами оператора, не зависящими от выбора базиса, что и означает ее единственность.
\end{proof}