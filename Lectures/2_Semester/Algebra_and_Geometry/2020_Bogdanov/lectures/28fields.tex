\subsection{Поле разложения}

\begin{theorem}
	Пусть $F$ "--- поле, $p \in F[x]$ "--- неприводимый над $F$ многочлен, $n := \deg{p} > 1$. Тогда существует такое поле $K \supset F$, в котором $p$ имеет корень.
\end{theorem}

\begin{proof}
	Рассмотрим $A_p$ "--- сопутствующую матрицу многочлена $p$ и докажем, что условию удовлетворяет множество $K \hm{:=} F[A_p] = \{f(A_p)\mid f \in F[x]\} \subset M_n(F)$.
	\begin{enumerate}
		\item Очевидно, что $K$ "--- подкольцо в $M_n(F)$. Более того, если $f \hm{\in} F[x]$ "--- константа, то $f(A_p) = fE$. Матрицы вида $fE$ образуют поле, изоморфное полю $F$, и лежат в $K$.
		\item Умножение в $K$ коммутативно, поскольку кольцо многочленов коммутативно, а подстановка матрицы $A_p$ в $f \in F[x]$ "--- это гомоморфизм.
		\item Пусть $f(A_p) \in K \bs \{0\}$, тогда $p \hm{=} \mu_{A_p}\nmid f$. Но многочлен $p$ неприводим, поэтому $\nd(f, p) = 1$, и существуют многочлены $u, v \in F[x]$ такие, что $uf + vp = 1$. Подставляя в данное равенство $A_p$, получаем, что $u(A_p)f(A_p) = E$, то есть элемент $f(A_p)$ обратим. Значит, $K$ является полем.
		\item В поле $K$ многочлен $p$ имеет корень $A_p \in K$, поскольку $p(A_p) \hm{=} 0$.\qedhere
	\end{enumerate}
\end{proof}

\begin{corollary}
	Пусть $F$ "--- поле, $p \in F[x]$, $n := \deg{p} > 1$. Тогда существует такое поле $K \supset F$, над которым $p$ раскладывается на линейные сомножители.
\end{corollary}

\begin{proof}
	Пусть $p$ раскладывается на $k$ неприводимых сомножителей над $F$. Проведем <<дедукцию>> по $k$. База, $k = n$, тривиальна, докажем переход. Если $k < n$, то в разложении многочлена $p$ есть неприводимый сомножитель $q$, $\deg{q} > 1$. Тогда существует поле $F_1 \subset F$, в котором $q$ имеет корень, и, следовательно, над $F_1$ многочлен $p$ раскладывается на хотя $k + 1$ неприводимых сомножителей, и применимо предположение <<дедукции>>.
\end{proof}

\begin{note}
	Поле $\mathbb{C}$ было получено из $\mathbb{R}$ такой же процедурой: мы расширяли поле $\mathbb{R}$ корнями многочлена $p(x) := x^2 + 1 \in \R[x]$ с сопутствующей матрицей следующего вида:
	\[A_p = \begin{pmatrix}0&1\\-1&0\end{pmatrix} \in M_2(\R)\]
	
	Так как уже во второй степени эта матрица дает $-E$, можно считать, что все многочлены имеют степень не выше первой:
	\[\mathbb{R}\left[\begin{pmatrix}0&1\\-1&0\end{pmatrix}\right] = \left\{\begin{pmatrix}a&b\\-b&a\end{pmatrix} : a, b \in \R\right\}\]
\end{note}

\begin{note}
	Теперь мы можем доказать теорему Гамильтона-Кэли в общем случае. Пусть $\phi \in \mathcal{L}(V)$, $V$ "--- линейное пространство над $F$, $\phi \leftrightarrow_e A$. Рассмотрим $K$ "--- надполе $F$, над которым $\chi_A$ раскладывается на линейные сомножители. Тогда, считая $A$ матрицей над $K$, можно утверждать, что $\chi_A(A) = 0$, причем, поскольку все элементы $A$ и все коэффициенты $\chi_A$ лежат в $F$, то в поле $F$ вычисление $\chi_A(A)$ происходит аналогично, поэтому и над $F$ данное утверждение верно.
\end{note}

\begin{note}
	В прошлом семестре доказывалось, что если $F$ "--- конечное поле, $\cha{F} = p$, то $|F| = p^n$. Применяя теорему Лагранжа к группе $F^*$, получим, что $\forall a \in F^*: a^{p^n - 1} = 1$, тогда $\forall a \in F: a^{p^n} = a$, то есть все элементы поля являются корнями многочлена $x^{p^n} - x$.
\end{note}

\begin{theorem}
	Пусть $p$ "--- простое число, $n \in \mathbb{N}$. Тогда существует поле $F$ такое, что $|F| = p^n$.
\end{theorem}

\begin{proof}
	Рассмотрим поле $\mathbb{Z}_p$ и найдем его надполе $K$, над которым многочлен $P := x^{p^n} - x$ раскладывается на линейные сомножители. Пусть $F \subset K$ "--- множество корней многочлена $P(x)$. Его производная $P'(x) \hm{=} p^nx^{p^n - 1} - 1 = -1$ не имеет корней, поэтому все корни $P$ "--- простые, то есть $|F| = p^n$. Докажем, что $F = \{a \in K: a^{p^n} = a\}$ "--- поле.		
	\begin{enumerate}
		\item Если $a, b \in F$, то $(ab)^{p^n} = a^{p^n}b^{p^n} = ab$, то есть $ab \in F$.
		\item Если $a \in F$, то $(a^{-1})^{p^n} = (a^{p^n})^{-1} = a^{-1}$, то есть $a^{-1} \in F$.
		\item Если $a, b \in F$, то $(a + b)^p = \sum_{i = 0}^{p}C_p^ia^ib^{p - i} = a^p + b^p + pC \hm{=} a^p + b^p$, следовательно, $(a + b)^{p^n} = (a^p + b^p)^{p^{n - 1}} = \dots = a^{p^n} + b^{p^n} = a + b$, то есть $a + b \in F$.
		\item Если $a \in F$, то $(-a)^{p^n} = (-1)^{p^n}a = -a$, то есть $-a \in F$.
	\end{enumerate}
	
	Таким образом, $F$ "--- подмножество в $K$, замкнутое относительно всех операций, поэтому $F$ "--- подполе в $K$.
\end{proof}

\begin{note}
	Пусть $F$ "--- поле, $P \in F[x]$, $K$ "--- надполе $F$, над которым $P$ раскладывается на линейные сомножители. $K$ называется \textit{полем разложения $P$}, если $K$ "--- единственное подполе $K$, содержащее поле $F$ и все корни многочлена $P$. Можно доказать, что поле разложения $P$ единственно с точностью до изоморфизма. Значит, и поле из $p^n$ элементов единственно с точностью до изоморфизма. Его обозначают через $\mathbb{F}_{p^n}$.
\end{note}