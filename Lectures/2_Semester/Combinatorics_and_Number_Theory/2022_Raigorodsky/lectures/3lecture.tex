\subsection{Матрицы Адамара}

\begin{definition}
	\textit{Матрицей Адамара} порядка $n$ называется квадратная матрица $A$ такая, что $A_{ij} \in \{\pm 1\}$ и любые 2 не одинаковые строки ортогональны (сумма произведений элементов строк постолбцово даёт 0).
\end{definition}

\begin{proposition}
	Если матрица Адамара $A \in M_n$ существует, то у неё и столбцы попарно ортогональны.
\end{proposition}

\begin{proof}
	Заметим следующее равенство:
	\[
		A \cdot A^T = nE_n
	\]
	Коль скоро $\det (A \cdot A^T) = \det A \cdot \det A^T$, то $A$ и $A^T$ не вырождены. Домножим слева на $A^T$ и справа на $A$:
	\[
		A^T \cdot (A \cdot A^T) \cdot A = (A^T \cdot A)^2 = A^T \cdot nE_n \cdot A = nE_n \cdot (A^T \cdot A)
	\]
	Осталось домножить на $(A^T \cdot A)^{-1} = A^{-1} \cdot (A^T)^{-1}$ и получить равенство:
	\[
		A^T \cdot A = nE_n
	\]
\end{proof}

\begin{corollary}
	Если домножить строку или столбец матрицы Адамара на $-1$, то она останется матрицей Адамара. Отсюда получаем, что если для некоторого $n$ нашлась матрица Адамара, то найдётся и другая, которая имеет вид:
	\[
		H_n = \Matrix{
			&1 &1 & &\cdots& &1 \\
			&1 &&&&\\
			&\vdots &&&\pm 1 & \\
			&1 &&&&
		}
	\]
	Матрицу Адамара в таком виде будем называть \textit{нормальной}.
\end{corollary}

\begin{example}~
	\begin{itemize}
		\item $n = 1$
		\[
			H_1 = (1)
		\]
		
		\item $n = 2$
		\[
			H_2 = \Matrix{&1 &1 \\ &1 &-1}
		\]
		
		\item $n = 3 \Ra \emptyset$
	\end{itemize}
\end{example}

\begin{proposition}
	Если $n \ge 2$, то матрица Адамара \underline{может} существовать только для чётного $n$.
\end{proposition}

\begin{proof}
	Рассмотрим произвольную не верхнуюю строку в $H_n$. Тогда, в произведении с верхней она должна давать 0. Такое возможно тогда и только тогда, когда количество $-1$ и 1 в строке совпадают, отсюда следует чётность $n$.
\end{proof}

\begin{proposition}
	Если $n \ge 4$, то матрица Адамара \underline{может} существовать только для $n$, кратного 4.
\end{proposition}

\begin{proof}
	В любой матрице $H_n$ можно переставить столбцы так, чтобы где-то собралась строка вида
	\[
		\underbrace{1\ \ldots\ 1}_{n/2} \underbrace{-1\ \ldots\ -1}_{n/2}
	\]
	Такая строка при <<умножении>> с любой другой должна давать 0. Пусть $x$ - это количество единиц другой строки, которые попали под позиции, где у полученной строки стоят 1. Тогда, $x > 0$ и на остальных $n/2 - x$ позициях стоят, естественно, $-1$. Так как количеств 1 и $-1$ поровну, то позициям -1 полученной строки соответствует $n/2 - x$ единиц и $x$ минус единиц. Отсюда имеем равенство:
	\[
		1 \cdot x - \left(\frac{n}{2} - x\right) - \left(\frac{n}{2} - x\right) + x = 0;\ \ n = 4x
	\]
\end{proof}

\begin{hypothesis} (Адамара, не доказана/опровергнута)
	Матрица Адамара \underline{существует} для $n \ge 4$ тогда и только тогда, когда $n = 4k,\ k \ge 1$.
\end{hypothesis}

\begin{note}
	 Для чисел, меньших 1000, гипотеза не доказана только для 668, 716 и 892.
\end{note}

\begin{example}
	Матрицу $H_4$ можно построить по подобию $H_2$:
	\[
		H_4 = \Matrix{
			&1 &1& &1& &1 \\
			&1 &-1& &1& &-1 \\
			&1 &1& &-1& &-1 \\
			&1 &-1& &-1& &1
		} = \Matrix{
			&H_2 & H_2 \\
			&H_2 & -H_2
		}^{\square}
	\]
	Этот же метод работает и для любого $n = 2^k,\ k \ge 1$.
\end{example}

\begin{definition}
	\textit{Кронекеровским произведением} $A * B$ матриц $A \in M_{n \times m},\ B \in M_{p \times q}$ называется матрица вида
	\[
		A * B = \Matrix{
			&a_{11}B & &a_{12}B & &\cdots & &a_{1m}B \\
			&\vdots & &\vdots & &\vdots & &\vdots \\
			&a_{n1}B & &a_{n2}B & & \cdots & &a_{nm}B
		}^{\square} \in M_{np \times mq}
	\]
\end{definition}

\begin{proposition}
	Если $A$ и $B$ - матрицы Адамара, то и $A * B$ - тоже матрица Адамара.
\end{proposition}

\begin{proof}
	Достаточно доказать, что строки полученной матрицы ортогональны. Посмотрим, как бы выглядели произвольные строки, которые мы выбрали для умножения:
	\begin{align*}
		&{a_{i1}B_{l*}\ a_{i2}B_{l*}\ \cdots \ a_{im}B_{l*}}
		\\
		&{a_{t1}B_{r*}\ a_{t2}B_{r*}\ \cdots \ a_{tm}B_{r*}}
	\end{align*}
	При умножении можно собрать некоторые слагаемые в скобки при $b_{lx} \cdot b_{ry}$. Выражение, которое находится внутри, получается аналогичным тому, что получается при умножении строк матрицы $A$.
\end{proof}

\begin{definition}
	\textit{Матрицей Якобсталя} порядка $p$, где $p$ - простое число, называется матрица $Q$ вида
	\[
		Q_{ij} := \legSym{i - j}{p}
	\]
\end{definition}

\begin{proposition}
	Произведение любых двух строк матрицы Якобсталя равно $-1$. То есть
	\[
		\suml_{j = 1}^p \legSym{i_1 - j}{p} \legSym{i_2 - j}{p} = -1
	\]
\end{proposition}

\begin{proof}
	Так как $j$ пробегает всю систему вычетов, то и $i_1 - j$ делает так же. Сделаем замену $b = i_1 - j$. В таком случае, $i_2 - j = i_1 - j + (i_2 - i_1) = b + (i_2 - i_1) = b + c$, а сумма запишется в следующем виде:
	\[
		\suml_{j = 1}^p \legSym{i_1 - j}{p} \legSym{i_2 - j}{p} = \suml_{b = 1}^p \legSym{b}{p} \legSym{b + c}{p}
	\]
	Если $i_1 \neq i_2$, то $c \not\equiv 0 \pmod p$ и наоборот. Сделаем некоторые преобразования над суммой:
	\begin{multline*}
		\suml_{b = 1}^p \legSym{b}{p} \legSym{b + c}{p} = \suml_{b = 1}^{p - 1} \legSym{b}{p} \legSym{1 \cdot (b + c)}{p} = \suml_{b = 1}^{p - 1} \legSym{b}{p} \legSym{b \cdot b^{-1} (b + c)}{p} =
		\\
		\suml_{b = 1}^{p - 1} \legSym{b}{p}^2 \legSym{1 + b^{-1}c}{p} = \suml_{b = 1}^{p - 1} \legSym{1 + b^{-1}c}{p} = -1
	\end{multline*}
\end{proof}

\begin{proposition} (Конструкция Пэли)
	Рассмотрим $p = 4k + 3$ (считаем известным, что данных простых чисел бесконечно много и они распределены $\pm$ равномерно). Если $n = p + 1$, то существует матрица Адамара этого порядка, имеющая вид
	\[
		H_n = \Matrix{
			&1 & &1 & &\cdots\cdots & &1 \\
			&1 &&&&&& \\
			&\vdots & & & &Q_p - E_p&& \\
			&1 &&&&&&
		}
	\]
	где $Q_p$ - матрица Якобсталя порядка $p$.
\end{proposition}

\begin{proof}
	Проверим ортогональность строк
	\begin{itemize}
		\item Умножение какой-то строки с первой, очевидно, даст 0: у нас было поровну единиц и минус единиц в $Q_p$, но ещё мы добавили единицу слева и минус единицу вместо нуля на диагонали.
		
		\item Рассмотрим произведение строк $i_1$ и $i_2$, где ни одна не является первой. Распишем слагаемые слева-направо:
		\[
			1 + \suml_{j = 1}^p \legSym{i_1 - j}{p} \legSym{i_2 - j}{p} + (-1) \cdot \legSym{i_2 - i_1}{p} + (-1) \cdot \legSym{i_1 - i_2}{p} = 0
		\]
		Последние 2 слагаемых - это то, что получается, когда мы попадаем на диагональ матрицы.
	\end{itemize}
\end{proof}

\begin{corollary}
	Зная распределение простых чисел и конструкцию Пэли, теперь можно заявить следующее:
	\[
		\forall \eps > 0\ \exists N \in \N \such \forall n \ge N \text{ на отрезке } [n; (1 + \eps)n] \text{ найдётся порядок матрицы Адамара}
	\]
\end{corollary}

\subsubsection*{Применение матриц Адамара}

\begin{enumerate}
	\item Задача об уклонении
	
	Определим множество $V_n = \{1, \ldots, n\}$. В нём выбрано $s$ подмножеств $M_1, \ldots, M_s \subseteq V_n$. Обозначим за $\mathcal{M} = \{M_1, \ldots, M_s\}$. Введём функцию \textit{раскраски}:
	\[
		\chi \colon V_n \to \{\pm 1\}
	\]
	Дополнительно с тем же обозначением введём значение функции для подмножества:
	\[
		\chi(M_i) = \suml_{j \in M_i} \chi(j)
	\]
	\textit{Разбросом} $\mathcal{M}$ по $\chi$ назовём величину
	\[
		\disc(\mathcal{M}, \chi) = \max\limits_{i \in \{1, \ldots, s\}} |\chi(M_i)|
	\]
	А разбросом $\mathcal{M}$ обозначим величину
	\[
		\disc(\mathcal{M}) = \min\limits_{\chi} \disc (\mathcal{M}, \chi)
	\]
	Фактически, мы хотим раскрасить множество так, что по выбранным подмножествам баланс цветов стремится к идеальному, то есть синих и красных почти (или вообще) поровну.
	
	\begin{theorem}
		Пусть $s = n$, где $n$ - порядок матрицы Адамара. Тогда существует $\mathcal{M} = \{M_1, \ldots, M_s\}$ такая, что
		\[
			\disc(\mathcal{M}) \ge \frac{\sqrt{n}}{2}
		\]
	\end{theorem}

	% Сюда доказательство
	
	\begin{corollary}
		Если $s = n$, то $\disc(\mathcal{M}) \ge (1 - \eps_n) \frac{\sqrt{n}}{2}$, где $\eps_n \to 0$ при $n \to \infty$
	\end{corollary}

	\begin{theorem} (без доказательства)
		Если $s = n$, то
		\[
			\forall \mathcal{M}\ \disc(\mathcal{M}) \le 6\sqrt{n}
		\]
	\end{theorem}
	
	\item Кодирование
\end{enumerate}