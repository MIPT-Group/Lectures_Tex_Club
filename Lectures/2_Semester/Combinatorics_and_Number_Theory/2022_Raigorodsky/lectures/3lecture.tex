\subsection{Матрицы Адамара}

\begin{definition}
	\textit{Матрицей Адамара} порядка $n$ называется квадратная матрица $A$ такая, что $A_{ij} \in \{\pm 1\}$ и любые 2 не одинаковые строки ортогональны (сумма произведений элементов строк постолбцово даёт 0).
\end{definition}

\begin{proposition}
	Если матрица Адамара $A \in M_n$ существует, то у неё и столбцы попарно ортогональны.
\end{proposition}

\begin{proof}
	Заметим следующее равенство:
	\[
		A \cdot A^T = nE_n
	\]
	Коль скоро $\det (A \cdot A^T) = \det A \cdot \det A^T$, то $A$ и $A^T$ не вырождены. Домножим слева на $A^T$ и справа на $A$:
	\[
		A^T \cdot (A \cdot A^T) \cdot A = (A^T \cdot A)^2 = A^T \cdot nE_n \cdot A = nE_n \cdot (A^T \cdot A)
	\]
	Осталось домножить на $(A^T \cdot A)^{-1} = A^{-1} \cdot (A^T)^{-1}$ и получить равенство:
	\[
		A^T \cdot A = nE_n
	\]
\end{proof}

\begin{corollary}
	Если домножить строку или столбец матрицы Адамара на $-1$, то она останется матрицей Адамара. Отсюда получаем, что если для некоторого $n$ нашлась матрица Адамара, то найдётся и другая, которая имеет вид:
	\[
		H_n = \Matrix{
			&1 &1 & &\cdots& &1 \\
			&1 &&&&\\
			&\vdots &&&\pm 1 & \\
			&1 &&&&
		}
	\]
	Матрицу Адамара в таком виде будем называть \textit{нормальной}.
\end{corollary}

\begin{example}~
	\begin{itemize}
		\item $n = 1$
		\[
			H_1 = (1)
		\]
		
		\item $n = 2$
		\[
			H_2 = \Matrix{&1 &1 \\ &1 &-1}
		\]
		
		\item $n = 3 \Ra \emptyset$
	\end{itemize}
\end{example}

\begin{proposition}
	Если $n \ge 2$, то матрица Адамара \underline{может} существовать только для чётного $n$.
\end{proposition}

\begin{proof}
	Рассмотрим произвольную не верхнуюю строку в $H_n$. Тогда, в произведении с верхней она должна давать 0. Такое возможно тогда и только тогда, когда количество $-1$ и 1 в строке совпадают, отсюда следует чётность $n$.
\end{proof}

\begin{proposition}
	Если $n \ge 4$, то матрица Адамара \underline{может} существовать только для $n$, кратного 4.
\end{proposition}

\begin{proof}
	В любой матрице $H_n$ можно переставить столбцы так, чтобы где-то собралась строка вида
	\[
		\underbrace{1\ \ldots\ 1}_{n/2} \underbrace{-1\ \ldots\ -1}_{n/2}
	\]
	Такая строка при <<умножении>> с любой другой должна давать 0. Пусть $x$ - это количество единиц другой строки, которые попали под позиции, где у полученной строки стоят 1. Тогда, $x > 0$ и на остальных $n/2 - x$ позициях стоят, естественно, $-1$. Так как количеств 1 и $-1$ поровну, то позициям -1 полученной строки соответствует $n/2 - x$ единиц и $x$ минус единиц. Отсюда имеем равенство:
	\[
		1 \cdot x - \left(\frac{n}{2} - x\right) - \left(\frac{n}{2} - x\right) + x = 0;\ \ n = 4x
	\]
\end{proof}

\begin{hypothesis} (Адамара, не доказана/опровергнута)
	Матрица Адамара \underline{существует} для $n \ge 4$ тогда и только тогда, когда $n = 4k,\ k \ge 1$.
\end{hypothesis}

\begin{note}
	 Для чисел, меньших 1000, гипотеза не доказана только для 668, 716 и 892.
\end{note}

\begin{example}
	Матрицу $H_4$ можно построить по подобию $H_2$:
	\[
		H_4 = \Matrix{
			&1 &1& &1& &1 \\
			&1 &-1& &1& &-1 \\
			&1 &1& &-1& &-1 \\
			&1 &-1& &-1& &1
		} = \Matrix{
			&H_2 & H_2 \\
			&H_2 & -H_2
		}^{\square}
	\]
	Этот же метод работает и для любого $n = 2^k,\ k \ge 1$.
\end{example}

\begin{definition}
	\textit{Кронекеровским произведением} $A * B$ матриц $A \in M_{n \times m},\ B \in M_{p \times q}$ называется матрица вида
	\[
		A * B = \Matrix{
			&a_{11}B & &a_{12}B & &\cdots & &a_{1m}B \\
			&\vdots & &\vdots & &\vdots & &\vdots \\
			&a_{n1}B & &a_{n2}B & & \cdots & &a_{nm}B
		}^{\square} \in M_{np \times mq}
	\]
\end{definition}

\begin{proposition}
	Если $A$ и $B$ - матрицы Адамара, то и $A * B$ - тоже матрица Адамара.
\end{proposition}

\begin{proof}
	Достаточно доказать, что строки полученной матрицы ортогональны. Посмотрим, как бы выглядели произвольные строки, которые мы выбрали для умножения:
	\begin{align*}
		&{a_{i1}B_{l*}\ a_{i2}B_{l*}\ \cdots \ a_{im}B_{l*}}
		\\
		&{a_{t1}B_{r*}\ a_{t2}B_{r*}\ \cdots \ a_{tm}B_{r*}}
	\end{align*}
	При умножении можно собрать некоторые слагаемые в скобки при $b_{lx} \cdot b_{ry}$. Выражение, которое находится внутри, получается аналогичным тому, что получается при умножении строк матрицы $A$.
\end{proof}

\begin{definition}
	\textit{Матрицей Якобсталя} порядка $p$, где $p$ - простое число, называется матрица $Q$ вида
	\[
		Q_{ij} := \legSym{i - j}{p}
	\]
\end{definition}

\begin{proposition}
	Произведение любых двух строк матрицы Якобсталя равно $-1$. То есть
	\[
		\suml_{j = 1}^p \legSym{i_1 - j}{p} \legSym{i_2 - j}{p} = -1
	\]
\end{proposition}

\begin{proof}
	Так как $j$ пробегает всю систему вычетов, то и $i_1 - j$ делает так же. Сделаем замену $b = i_1 - j$. В таком случае, $i_2 - j = i_1 - j + (i_2 - i_1) = b + (i_2 - i_1) = b + c$, а сумма запишется в следующем виде:
	\[
		\suml_{j = 1}^p \legSym{i_1 - j}{p} \legSym{i_2 - j}{p} = \suml_{b = 1}^p \legSym{b}{p} \legSym{b + c}{p}
	\]
	Если $i_1 \neq i_2$, то $c \not\equiv 0 \pmod p$ и наоборот. Сделаем некоторые преобразования над суммой:
	\begin{multline*}
		\suml_{b = 1}^p \legSym{b}{p} \legSym{b + c}{p} = \suml_{b = 1}^{p - 1} \legSym{b}{p} \legSym{1 \cdot (b + c)}{p} = \suml_{b = 1}^{p - 1} \legSym{b}{p} \legSym{b \cdot b^{-1} (b + c)}{p} =
		\\
		\suml_{b = 1}^{p - 1} \legSym{b}{p}^2 \legSym{1 + b^{-1}c}{p} = \suml_{b = 1}^{p - 1} \legSym{1 + b^{-1}c}{p} = -1
	\end{multline*}
\end{proof}

\begin{proposition} (Конструкция Пэли)
	Рассмотрим $p = 4k + 3$ (считаем известным, что данных простых чисел бесконечно много и они распределены $\pm$ равномерно). Если $n = p + 1$, то существует матрица Адамара этого порядка, имеющая вид
	\[
		H_n = \Matrix{
			&1 & &1 & &\cdots\cdots & &1 \\
			&1 &&&&&& \\
			&\vdots & & & &Q_p - E_p&& \\
			&1 &&&&&&
		}
	\]
	где $Q_p$ - матрица Якобсталя порядка $p$.
\end{proposition}

\begin{proof}
	Проверим ортогональность строк
	\begin{itemize}
		\item Умножение какой-то строки с первой, очевидно, даст 0: у нас было поровну единиц и минус единиц в $Q_p$, но ещё мы добавили единицу слева и минус единицу вместо нуля на диагонали.
		
		\item Рассмотрим произведение строк $i_1$ и $i_2$, где ни одна не является первой. Распишем слагаемые слева-направо:
		\[
			1 + \suml_{j = 1}^p \legSym{i_1 - j}{p} \legSym{i_2 - j}{p} + (-1) \cdot \legSym{i_2 - i_1}{p} + (-1) \cdot \legSym{i_1 - i_2}{p} = 0
		\]
		Последние 2 слагаемых - это то, что получается, когда мы попадаем на диагональ матрицы.
	\end{itemize}
\end{proof}

\begin{corollary}
	Зная распределение простых чисел и конструкцию Пэли, теперь можно заявить следующее:
	\[
		\forall \eps > 0\ \exists N \in \N \such \forall n \ge N \text{ на отрезке } [n; (1 + \eps)n] \text{ найдётся порядок матрицы Адамара}
	\]
\end{corollary}

\subsubsection*{Применение матриц Адамара}

\begin{enumerate}
	\item Задача об уклонении
	
	Определим множество $V_n = \{1, \ldots, n\}$. В нём выбрано $s$ подмножеств $M_1, \ldots, M_s \subseteq V_n$. Обозначим за $\mathcal{M} = \{M_1, \ldots, M_s\}$. Введём функцию \textit{раскраски}:
	\[
		\chi \colon V_n \to \{\pm 1\}
	\]
	Дополнительно с тем же обозначением введём значение функции для подмножества:
	\[
		\chi(M_i) = \suml_{j \in M_i} \chi(j)
	\]
	\textit{Разбросом} $\mathcal{M}$ по $\chi$ назовём величину
	\[
		\disc(\mathcal{M}, \chi) = \max\limits_{i \in \{1, \ldots, s\}} |\chi(M_i)|
	\]
	А разбросом $\mathcal{M}$ обозначим величину
	\[
		\disc(\mathcal{M}) = \min\limits_{\chi} \disc (\mathcal{M}, \chi)
	\]
	Фактически, мы хотим раскрасить множество так, что по выбранным подмножествам баланс цветов стремится к идеальному, то есть синих и красных почти (или вообще) поровну.
	
	\begin{theorem}
		Пусть $s = n$, где $n$ - порядок матрицы Адамара. Тогда существует набор подмножеств $\mathcal{M} = \{M_1, \ldots, M_s\}$ такой, что
		\[
			\disc(\mathcal{M}) \ge \frac{\sqrt{n}}{2}
		\]
		Более того, маски $M_i$ являются строками в матрице $\frac{H_n + J}{2}$, где $J$ - матрица, состоящая полностью из единиц.
	\end{theorem}

	\begin{proof}
		Фактически надо доказать, что $\forall \vec{v} = (v_1, \ldots, v_n)^T,\ v_i \in \{\pm 1\}$ у вектора вида
		\[
			\vec{u} = \left(\frac{H_n + J}{2}\right) \cdot \vec{v}
		\]
		есть координата, модуль которой $\ge \frac{\sqrt{n}}{2}$. Иначе говоря, $\vec{v}$ - это раскраска $V_n$, и так совпало, что произведение будет давать в координатах разброс каждого подмножества.
		
		Распишем произведение из определения вектора $\vec{u}$:
		\[
			\vec{u} = \left(\frac{H_n + J}{2}\right) \cdot \vec{v} = \frac{1}{2} \cdot (H_n \vec{v} + J \vec{v})
		\]
		Отдельно разберёмся с первым слагаемым в скобках. Обозначим $H\vec{v} = (L_1, \ldots, L_n)^T$ и рассмотрим скалярный квадрат:
		\[
			\trbr{H_n \vec{v}, H_n \vec{v}} = L_1^2 + \ldots + L_n^2
		\]
		Дополнительно скажем, что $H_n = (\vec{h}_1 \cdots \vec{h}_n)$ и $h_{ij}$ - это элемент матрицы Адамара. Тогда
		\begin{multline*}
			\trbr{H_n \vec{v}, H_n \vec{v}} = \trbr{v_1\vec{h}_1 + \ldots + v_n\vec{h}_n, v_1\vec{h}_1 + \ldots + v_n\vec{h}_n} =
			\\
			v_1^2 \underbrace{\trbr{\vec{h}_1, \vec{h}_1}}_{n} + \ldots + v_n^2 \underbrace{\trbr{\vec{h}_n, \vec{h}_n}}_{n} + \suml_{i \neq j} v_i v_j \underbrace{\trbr{\vec{h}_i, \vec{h}_j}}_{0} = n^2
		\end{multline*}
		Отсюда следует, что $\exists i \colon |L_i| \ge \sqrt{n}$. Теперь распишем всё произведение, за исключением домножения на скаляр:
		\[
			(H + J)\vec{v} = \left(L_1 + \suml_{i = 1}^n v_i \cdots L_n + \suml_{i = 1}^n v_i\right)^T
		\]
		где $\suml_{i = 1}^n v_i = \lambda$, причём $\lambda$ должна быть чётным числом. Снова возьмём скалярный квадрат от всего выражения, получим следующее:
		\[
			\trbr{(H + J)\vec{v}, (H + J)\vec{v}} = L_1^2 + \ldots + L_n^2 + 2\lambda \suml_{i = 1}^n L_i + \lambda^2 n = n^2 + 2\lambda \suml_{i = 1}^n L_i + \lambda^2 n
		\]
		Отдельно посчитаем оставшуюся сумму:
		\[
			\suml_{i = 1}^n L_i = \suml_{i = 1}^n \left(\suml_{j = 1}^n h_{ij} v_j\right) = \suml_{j = 1}^n v_j \left(\suml_{i = 1}^n h_{ij}\right) = v_1 \cdot n
		\]
		Подставим полученное в выражение выше:
		\[
			n^2 + 2\lambda \suml_{i = 1}^n L_i + \lambda^2 n = n^2 + 2nv_1 \lambda + n\lambda^2
		\]
		Для оценки модулей координат нам надо оценить минимум скалярного квадрата. Это сделать мы можем, так как имеем дело с параболой ветвями вверх относительно $\lambda$:
		\[
			\lambda_{min} = \frac{-2nv_1}{2n} = -v_1 \in \{\pm 1\}
		\]
		Из-за того, что $\lambda$ - чётное число, то необходимо произвести разбор случаев:
		\[
			\lambda_{min} \in \System{
				&{\{-2, 0\},\ v_1 = 1}
				\\
				&{\{0, 2\},\ v_1 = -1}
			}
		\]
		В обоих случаях есть вариант с $\lambda = 0$, поэтому
		\[
			\trbr{(H + J)\vec{v}, (H + J)\vec{v}} = n^2 + 2nv_1\lambda + n\lambda^2 \ge n^2
		\]
		Отсюда уже следует существование координаты с модулем $\ge \sqrt{n}$, ну а стало быть в исходном виде $\ge \frac{\sqrt{n}}{2}$.
	\end{proof}
	
	\begin{corollary}
		Если $s = n$, то $\disc(\mathcal{M}) \ge (1 - \eps_n) \frac{\sqrt{n}}{2}$, где $\eps_n \to 0$ при $n \to \infty$
	\end{corollary}

	\begin{theorem} (без доказательства)
		Если $s = n$, то
		\[
			\forall \mathcal{M}\ \disc(\mathcal{M}) \le 6\sqrt{n}
		\]
	\end{theorem}
	
	\item Задача о кодах, исправляющих ошибки.
	
	Есть источник и приёмник. Между ними установлен канал связи, по которому можно передавать слова в виде двоичного кода, однозначно сопоставленного каждому слову. К сожалению, при передаче возникают помехи, из-за которых в произвольном месте кода 1 может замениться на 0 и наоборот. Какое максимальное количество слов можно передать?
	
	\begin{definition}
		\textit{Расстоянием Хэмминга} между двумя двоичными кодами длины $n$ назовём число позиций, в которых они различаются (или же квадрат евклидова расстояния между точками $n$-мерного пространства).
	\end{definition}
	
	\begin{example}
		Для кодов $01110$ и $11001$ расстояние Хэмминга будет 4.
	\end{example}

	\begin{proposition}
		Рассмотрим произвольный код длины $n$. Число кодов такой же длины, для которых расстояние Хэмминга с исходным не превышает $d$, будет $C_n^0 + \ldots + C_n^d$
	\end{proposition}

	%% Вот сюда можно картинку с лекции с чуть большим числом деталей (ещё каких-нибудь последовательностей нарисовать в шарике) ОКТЧ 20. 42:00

	\begin{proof}
		В сумме $C_n^i$ символизирует количество способов выбрать $i$ позиций, на местах которых мы изменим число в исходном коде.
	\end{proof}

	\begin{definition}
		\textit{Кодом, исправляющим ошибки} $(n, M, d)$ будем называть код, который может передать $M$ слов, каждое закодировано при помощи $n$ нулей и единиц (то есть длина кода) и $d$ - минимальное расстояние Хэмминга между словами.
	\end{definition}

	\begin{note}
		Если такой код существует, то он исправляет не более чем $\floor{\frac{d - 1}{2}}$ ошибок, чтобы иметь возможность однозначно раскодировать слова.
	\end{note}

	\begin{theorem} (Граница Плоткина)
		Пусть $2d > n$. Тогда $M \le \floor{\frac{2d}{2d - n}}$.
	\end{theorem}

	\begin{proof}
		Пусть $a_1, \ldots, a_M$ - кодовые слова, то есть последовательности из 0 и 1 длины $n$. Запишем их в виде матрицы $A$:
		\[
			A = (a_{ij}) = \Matrix{&a_1 \\ &a_2 \\ &\vdots \\ &a_M}^\square \in M_{M \times n}
		\]
		Посчитаем суммарное количество ошибок в кодовых словах, если рассмотреть пары с точностью до перестановки слов:
		\[
			\suml_{1 \le i < j \le M} \left(\suml_{k = 1}^n \mathbb{I}_{\{a_{ik} \neq a_{jk}\}}\right) \ge \frac{M(M - 1)}{2} \cdot d
		\]
		где $\mathbb{I}$ - это функция-индикатор, то есть
		\[
			\mathbb{I}_{\{\text{условие}\}} = \System{
				&{1, \text{ условие истинно}}
				\\
				&{0, \text{ условие ложно}}
			}
		\]
		А теперь попробуем посчитать сумму с другой стороны, поменяв знаки суммирования местами. Пусть в столбце было $x_k$ единиц. Тогда
		\[
			\suml_{1 \le i < j \le M} \mathbb{I}_{\{a_{ik} \neq a_{jk}\}} = x_k \cdot (M - x_k)
		\]
		Снова столкнулись с параболой, но уже ветвями вниз. Её можно оценить сверху как $\frac{M^2}{4}$. Отсюда
		\[
			\suml_{k = 1}^n \left(\suml_{1 \le i < j \le M} \mathbb{I}_{\{a_{ik} \neq a_{jk}\}}\right) \le n \cdot \frac{M^2}{4}
		\]
		В итоге получили следующую оценку:
		\[
			\frac{M(M - 1)}{2} \cdot d \le n \frac{M^2}{4};\ \ 2(M - 1)d \le nM;\ \ M(2d - n) \le 2d
		\]
	\end{proof}

	\begin{proposition}
		Граница Плоткина достигается, если рассмотреть код $(n - 1, n, n/2)$. Кодовые слова записаны в матрице Адамара $H_n$, если удалить левый столбец и заменить все $-1$ на 0.
	\end{proposition}

	\begin{proof}
		При удалении левого столбца число $-1$ не изменяется, а потому расстояние между первой и любой другой строкой будет в аккурат $n/2$. Подставим все величины кода в оценку Плоткина:
		\[
			n \le \floor{\frac{n}{n - (n - 1)}} = n
		\]
	\end{proof}
\end{enumerate}