\subsection{Начала теории графов}

\begin{definition}
	\textit{Графом} называется пара \textit{множества вершин} и \textit{множества рёбер}.
\end{definition}

\begin{definition}
	Граф $G = (V, E)$ называется \textit{обыкновенным (простым)}, если выполнены следующие условия:
	\begin{enumerate}
		\item Нет <<петель>>, то есть
		\[
			\forall x \in V\ \ \not\exists (x, x) \in E
		\]
		
		\item Нет ориентации, то есть
		\[
			\forall x, y \in V\ \ (x, y) = (y, x)
		\]
		
		\item Нет кратных рёбер, то есть
		\[
			E \subseteq C_V^2
		\]
	\end{enumerate}
\end{definition}

\begin{note}
	$C_V^2$ обозначает множество пар вершин из $V$ без повторений наборов в них.
\end{note}

\begin{anote}
	Во втором пункте я бы лучше сформулировал так:
	\[
		\forall x, y \in V\ \ (x, y) \in E \lra (y, x) \in E
	\]
\end{anote}

\textcolor{red}{Сюда бы картиночку с каким-то графом}

\begin{definition}
	Если мы отказываемся в определении обыкновенного графа от \textbf{первого} свойства, то он называется \textit{псевдографом}.
\end{definition}

\begin{definition}
	Если мы отказываемся в определении обыкновенного графа от \textbf{второго} свойства, то он называется \textit{орграфом (ориентированным графом)}.
\end{definition}

\begin{definition}
	Если мы отказываемся в определении обыкновенного графа от \textbf{третьего} свойства, то он называется \textit{мультиграфом} (не путать с гиперграфом!!!).
\end{definition}

\begin{note}
	Естественно, определения можно комбинировать.
\end{note}

\begin{definition}
	Граф $K_n = (V, E),\ |V| = n$ называется \textit{полным}, если у него есть все возможные рёбра. То есть $|E| = C_n^2$
\end{definition}

\begin{example}
	Сколько существует графов на $V = \{1, \ldots, n\}$ вершинах?
\end{example}

\begin{definition}
	Графы $G_1 = (V_1, E_1)$ и $G_2 = (V_2, E_2)$ называются \textit{изоморфными}, если существует биекция, удовлетворяющая следюущему условию:
	\[
		\phi \colon V_1 \to V_2 \such \forall e = (x, y)\ \ e \in E_1 \lra (\phi(x), \phi(y)) \in E_2
	\]
\end{definition}

\begin{definition}
	\textit{Степенью вершины} $v \in V$ называется количество рёбер, инциндентных ей. Обозначается как $\deg v$
\end{definition}

\begin{definition}
	Величиной $\indeg v$ называется число рёбер, инциндентных данной вершине, в которых $v$ стоит на \textbf{втором} месте:
	\[
		\indeg v = |\{y \such (y, v) \in E\}|
	\]
	Величиной $\outdeg v$ называется число рёбер, инциндентных данной вершине, в которых $v$ стоит на \textbf{первом} месте:
	\[
		\outdeg v = |\{y \such (v, y) \in E\}|
	\]
\end{definition}

\begin{lemma} (О рукопожатиях)
	\[
		\suml_{v \in V} \deg v = 2|E|
	\]
\end{lemma}

\begin{proof}
	Давайте мысленно зафиксируемся на каком-то из рёбер, и начнём суммировать степени вершин. Наше ребро может быть учтено лишь тогда, когда мы будем считать вершины ему инциндентные, коих всего 2. Отсюда и получается равенство.
\end{proof}

\begin{definition}
	\textit{Маршрутом в графе} $G = (V, E)$ будем называть чередующуюся последовательность вершин и рёбер, которая начиная и заканчивается на вершинах.
	\[
		v_1 e_1 v_2 e_2 \ldots e_n v_{n + 1}
	\]
\end{definition}

\begin{note}
	Определение маршрута, естественно, допускает возможность появления одинаковых рёбер и вершин в последовательности.
\end{note}

\begin{definition}
	Маршрут называется \textit{замкнутым}, если $v_1 = v_{n + 1}$
\end{definition}

\begin{definition}
	Если в замкнутом маршруте все рёбра разные, то он называется \textit{циклом}
\end{definition}

\begin{definition}
	Цикл называется \textit{простым}, если помимо разных рёбер, у него все промежуточные вершины тоже разные (то есть кроме $v_1$ и $v_{n + 1}$).
\end{definition}

\begin{definition}
	Если маршрут не замкнут и все его рёбра разные, то он называется \textit{путём (или же цепью)}.
\end{definition}

\begin{definition}
	Путь называется \textit{простым}, если все его вершины разные.
\end{definition}

\begin{definition}
	Граф $G = (V, E)$ называется \textit{связным}, если для любой пары вершин $x, y \in V$ существует маршрут, начинающийся в $x$ и заканчивающийся в $y$.
\end{definition}

\begin{definition}
	Граф называется \textit{ациклическим}, если в нём не содержится циклов.
\end{definition}

\begin{definition}
	Граф $G = (V, E)$ называется \textit{деревом}, если он является связным ациклическим графом.
\end{definition}

\begin{theorem}
	Для графа $G = (V, E)$ следующие 4 утверждения эквивалентны:
	\begin{enumerate}
		\item $G$ - дерево
		
		\item В $G$ любые 2 вершины соединены единственной простой цепью
		
		\item $G$ связен и если $|V| = n$, то $|E| = n - 1$
		
		\item $G$ ацикличен и если $|V| = n$, то $|E| = n - 1$
	\end{enumerate}
\end{theorem}

\begin{proof}
	\textcolor{red}{Доказать}
\end{proof}

\begin{example}
	Пусть $t_n$ - число деревьев на $n$ вершинах.
	\begin{align*}
		&{t_1 = 1}
		\\
		&{t_2 = 1}
		\\
		&{t_3 = 3}
		\\
		&{t_4 = 16}
		\\
		&{t_5 = 125}
		\\
		&{\vdots}
	\end{align*}
\end{example}

\begin{theorem} (Формула Кэли, 1857г.)
	\[
		t_n = n^{n - 2}
	\]
\end{theorem}

\begin{proof}
	Приведём идею с \textit{кодами Прюфера}: из формулы логично предположить, что мы можем каждому дереву на $n$ вершинах сопоставить размещение $n - 2$ чисел из множества $\{1, \ldots, n\}$ с повторениями.
	
	\textcolor{red}{Дописать алгоритм построения кода и восстановления}
\end{proof}

\begin{definition}
	\textit{Унициклическим графом (одноцикловым)} называется связный граф с ровно одним циклом.
\end{definition}

\begin{note}
	Из того, что в унициклическом графе всего 1 цикл следует, что этот цикл простой.
\end{note}

\begin{example}
	Сколько существует унициклических графов на $n$ вершинах?
\end{example}