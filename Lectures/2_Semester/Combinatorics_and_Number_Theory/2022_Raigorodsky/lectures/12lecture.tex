\begin{proof}
	Для доказательства критерия Вейля нам потребуется \textit{теорема Вейерштрасса} из математического анализа:
	\begin{theorem}
		Если $f$ - непрерывная комплекснозначная функция с периодом 1, то $\forall \eps > 0$ существует функция $\psi(x)$ такая, что
		\[
			\psi(x) = \suml_{m \neq 0} c_m e^{2\pi i m x}
		\]
		причём
		\[
			\sup\limits_{x \in \R} |f(x) - \psi(x)| < \eps
		\]
	\end{theorem}

	\begin{note}
		Сумма по $m \neq 0$ подразумевает, что мы смотрим на конечный набор целых $m \neq 0$.
	\end{note}

	Из сказанного выше, нам достаточно доказать теорему лишь в одну сторону: от предела к равномерной распределённости. Зафиксируем какую-то непрерывную периодическую комплекснозначную функцию $f$ с периодом 1 и $\eps > 0$. Теперь, воспользуемся теоремой выше и выберем $\psi(x)$:
	\[
		\psi(x) = \suml_{m \in M} c_m e^{2\pi i mx}
	\]
	\textcolor{red}{Дописать доказательство}
\end{proof}

\begin{definition}
	\textcolor{red}{Сумма Гаусса}
\end{definition}

\begin{theorem}
	\[
		|S(q)| = \System{
			&{\sqrt{q}, \text{ если } q \text{ нечётно}}
			\\
			&{0, \text{ если } q \text{ чётно, но не делится на } 4}
			\\
			&{\sqrt{2q}, \text{ иначе}}
		}
	\]
\end{theorem}

\begin{proof}
	\begin{multline*}
		|S(q)|^2 = S(q) \cdot \overline{S(q)} = \left(\suml_{x = 1}^q e^{2\pi i \frac{ax^2}{q}}\right) \cdot \left(\suml_{y = 1}^q e^{-2\pi i \frac{ay^2}{q}}\right) = \left(\suml_{x = 1}^q e^{2\pi i \frac{ax^2}{q}} \cdot \suml_{y = 1}^q e^{-2\pi i \frac{a}{q}(y + x)^2}\right) =
		\\
		\suml_{y = 1}^q e^{-2\pi \frac{ay^2}{q}} \cdot \left(\suml_{x = 1}^q e^{2\pi i \frac{2axy}{q}}\right)
	\end{multline*}
	Разберём случаи:
	\begin{enumerate}
		\item $q$ - нечётное. Тогда, так как $(a, q) = 1$:
		\[
			\suml_{x = 1}^q e^{2\pi i \frac{2ax}{q}}
		\]
	\end{enumerate}
\end{proof}