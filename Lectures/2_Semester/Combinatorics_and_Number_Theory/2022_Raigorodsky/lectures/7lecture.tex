\subsubsection*{Бесконечные цепные дроби}

\begin{definition}
	Назовём число $\alpha = [a_0; a_1, \ldots, a_n, \ldots]$ \textit{бесконечной цепной дробью}, если
	\[
		\exists \liml_{k \to \infty} [a_0; a_1, \ldots, a_k] = \alpha
	\]
\end{definition}

\begin{theorem}
	Если $\forall i \in \N_0\ a_i \in \N$, то предел всегда существует.
\end{theorem}

\begin{proof}
	Так как мы имеем дело с последовательностью подходящих дробей, то верны все утверждения и теоремы, которые мы доказали выше. В частности, последовательность подходящих дробей с чётными номерами возрастает, но ограничена сверху через $p1/q1$. Значит, по теореме Вейерштрасса о монотонной последовательности существует предел. Аналогично с нечётными, и из обоих утверждений уже следует требуемое.
\end{proof}

\begin{proposition}
	Любое число $\alpha \in \R$ можно представить в виде цепной дроби.
\end{proposition}

\begin{proof}
	Представим число $\alpha$ как сумму целой и дробной частей:
	\[
		\alpha = \floor{\alpha} + \{\alpha\}
	\]
	Возможно 2 варианта:
	\begin{enumerate}
		\item $\{\alpha\} = 0$. В таком случае алгоритм разложения закончен.
		
		\item $\{\alpha\} \neq 0$. Тогда сделаем шаг рекурсии:
		\[
			\alpha = \floor{\alpha} + \frac{1}{1 / \{\alpha\}}
		\]
		Теперь нужно рекурсивно разложить $1 / \{\alpha\}$.
	\end{enumerate}
\end{proof}

\begin{note}
	Если какая-то последовательность подходящих дробей имеет предел, равный $\alpha$, то эти дроби совпадают с построенным в утверждении разложением. Доказывается индукцией по $k$.
\end{note}

\begin{definition}
	\textit{Периодической цепной дробью} назовём дробь $\alpha$ вида
	\[
		\alpha = [a_0; a_1, \ldots, a_n, (b_1, \ldots, b_k)]
	\]
	где $(b_1, \ldots, b_k)$ - период, который бесконечно повторяется.
\end{definition}

\begin{example}
	Рассмотрим периодическую цепную дробь $\alpha = [1; (1)] > 1$. Что это за число?
	\[
		[1; (1)] = 1 + \frac{1}{[1; (1)]} \Ra \alpha = 1 + \frac{1}{\alpha}
	\]
	Решив это уравнение, получим следующее значение:
	\[
		\alpha = \frac{1 + \sqrt{5}}{2} = \phi
	\]
\end{example}

\begin{definition}
	\textit{Квадратичной иррациональностью} называется \underline{иррациональное} число, которое является корнем квадратного уравнения.
\end{definition}

\begin{example}
	$\sqrt{2}$ - квадратичная иррациональность. Является одним из корней уравнения $x^2 = 2$.
\end{example}

\begin{proposition}
	Если $\alpha$ - квадратичная иррациональность, $v \in \N$, то $\alpha^{-1}$, $\alpha + v$ - тоже квадратичные иррациональности.
\end{proposition}

\begin{proof}
	Для доказательства достаточно предъявить квадратные уравнения, где эти числа являются корнями. По условию есть $a, b, c \in \R$ такие, что
	\[
		a\alpha^2 + b\alpha + c = 0
	\]
	Для $1/\alpha$ нужно найти аналогичные $d, e, f \in \R$ такие, что
	\[
		d\frac{1}{\alpha^2} + e\frac{1}{\alpha} + f = \frac{d + e\alpha + f\alpha^2}{\alpha^2} = 0
	\]
	То есть видно, что нужно положить $f = a,\ e = b,\ c = d$ соответственно. Аналогичным образом поступаем и с $\alpha + v$:
	\[
		d'(\alpha + v)^2 + e'(\alpha + v) + f' = d'\alpha^2 + \alpha(2d'v + e') + d'v^2 + e'v + f' = 0 
	\]
	Отсюда получаем систему уравнений, которая очевидным образом разрешима: 
	\[
		\System{
			&{d' = a}
			\\
			&{2d'v + e' = b}
			\\
			&{d'v^2 + e'v + f' = c}
		}
	\]
\end{proof}

\begin{proposition}
	Периодическая цепная дробь $\alpha$ является квадратичной иррациональностью.
\end{proposition}

\begin{proof}
	Пусть $\alpha = [a_0; a_1, \ldots, a_m, (b_1, \ldots, b_k)]$. Обозначим за $\beta$ следующее число:
	\[
		\beta = [0; (b_1, \ldots, b_k)] = \frac{1}{b_1 + \frac{1}{\cdots + \frac{1}{b_k + \beta}}}
	\]
	Несложно показать по индукции, что эту дробь можно развернуть и получить уравнение с некоторыми известными $a, b, c, d \in \Z$:
	\[
		\frac{a\beta + b}{c\beta + d} = \beta
	\]
	Его существования уже достаточно, чтобы заключить, что $\beta$ - квадратичная иррациональность, откуда следует по тем же индуктивным соображениям квадратичная иррациональность $\alpha$.
\end{proof}

\begin{theorem} (без доказательства)
	$\alpha$ - квадратичная иррациональность тогда и только тогда, когда $\alpha$ раскладывается в периодическую цепную дробь.
\end{theorem}

Открытой проблемой остаётся вопрос о кубической иррациональности - об устройстве таких цепных дробей известно крайне мало.

\begin{hypothesis}
	$\forall p$ - простого числа,  существует $a \le p - 1$ такое, что все неполные частные цепной дроби $\frac{a}{p}$ ограничены константой 5.
\end{hypothesis}

\begin{note}
	Данная гипотеза имеет большое значение в вычислении определённых интегралов <<по сеточкам>>. Компьютерные данные говорят, что она работает, но доказать пока удалось лишь ограниченность сверху через $\ln p$.
\end{note}

\begin{definition}
	Число называется \textit{алгебраическим}, если оно является корнем уравнения, задаваемого многочленом с целыми коэффициентами.
	
	Множество этих чисел обозначается как $\A$
	\[
		a \in \A \lra \exists P \in \Z[x] \such P(a) = 0
	\]
\end{definition}

\begin{note}
	Без доказательства отметим, что множество алгебраических чисел $\A$ образует поле.
\end{note}

\begin{definition}
	Любое число, не являющееся алгебраическим, называется \textit{трансцендентным}.
\end{definition}

\begin{note}
	Алгебраических чисел в рамках действительных крайне мало - счётное число.
\end{note}

\begin{definition}
	\textit{Степень алгебраического числа} - это минимальная степень уравнения, корнем которого это число является.
\end{definition}

\begin{theorem} (Лиувилля)
	Для любого $\alpha \in (\A \bs \Q) \cap \R$ степени $d(\alpha)$ существует $c = c(\alpha) > 0$ такое, что
	\[
		\forall p, q \in \Z\ \left|\alpha - \frac{p}{q}\right| \ge \frac{c}{q^d}
	\]
\end{theorem}

\begin{note}
	Суть теоремы состоит в том, что алгебраические числа нельзя слишком хорошо аппроксимировать.
\end{note}

\begin{proof}
	Разберём несколько случаев:
	\begin{enumerate}
		\item Пусть $p$ и $q$ таковы, что $|\alpha - p/q| \ge 1$. Тогда очевидным образом
		\[
			\left|\alpha - \frac{p}{q}\right| \ge \frac{1}{q^d}
		\]
		
		\item Теперь $|\alpha - p/q| < 1$. Пусть многочлен $f(x)$, имеющий вид
		\[
			f(x) = a_d x^d + \ldots + a_0
		\]
		является минимальным многочленом, корнем которого служит $\alpha$. Согласно основной теореме алгебры, у $f(x) = 0$ будет ровно $d$ корней $\alpha_1 := \alpha, \alpha_2, \ldots, \alpha_d$ (при этом $\alpha_i$ может быть комплексным, без проблем).
		\begin{proposition}
			У $f(x)$ нет рациональных корней
		\end{proposition}
		
		\begin{proof}
			Действительно, если он есть, то, разложив многочлен на линейные сомножители, скобка с таким корнем при подстановке $\alpha$ не обнулится - следовательно, на неё можно <<сократить>> и останется многочлен степени $d - 1$, что противоречит с условием.
		\end{proof}
	
		Ради интереса, подставим $p/q$ в $f$:
		\[
			0 \neq f(p/q) = a_d (p/q)^d + \ldots + a_1 (p/q) + a_0 = \frac{h}{q^d}
		\]
		где $h$ - целое число. То есть $|f(p/q)| \ge 1/q^d$. С другой стороны, распишем это же значение, но в разложенном виде:
		\[
			f(p/q) = a_d (p/q - \alpha) \cdot \prodl_{\tau = 2}^d (p/q - \alpha_\tau)
		\]
		Снова оценим модуль:
		\begin{multline*}
			\frac{1}{q^d} \le |f(p/q)| = |a_d| \cdot \left|\alpha - \frac{p}{q}\right| \cdot \prodl_{\tau = 2}^d \left(\left|\alpha_\tau - \frac{p}{q}\right|\right) \le
			\\
			|a_d| \cdot |\alpha - p/q| \cdot \prodl_{\tau = 2}^d (|\alpha_\tau - \alpha| + \underbrace{|\alpha - p/q|}_{< 1}) < |\alpha - p/q| \cdot \underbrace{\left(|a_d| \cdot \prodl_{\tau = 2}^d (|\alpha_\tau - \alpha| + 1)\right)}_{1/c(\alpha)}
		\end{multline*}
	\end{enumerate}
\end{proof}

\begin{theorem}
	Для любой функции $\psi(q)$ такой, что $\psi$ монотонно стремится к бесконечности, существует число $\alpha$ такое, что найдётся последовательность $\{p_n/q_n\}_{n = 1}^\infty$ со свойством:
	\[
		\left|\alpha -  \frac{p_n}{q_n}\right| \le \frac{1}{q_n \cdot \psi(q_n)}
	\]
\end{theorem}

\begin{note}
	Теорема описывает собою конструкцию трансцендентного числа для любой скорости приближения.
\end{note}

\begin{proof}
	Будем строить цепную дробь $\alpha$ индуктивно при помощи подходящих дробей.
	\begin{enumerate}
		\item База - произвольное рациональное число, разложенное в цепную дробь $[a_0; a_1, \ldots, a_k]$, для подходящих дробей которого выполнено условие (такое точно есть, ибо можно просто взять целое число).
		\item Переход $n > k$ - пусть мы узнали $n + 1$ число в разложении $\alpha$, то есть
		\[
			\alpha = [a_0; a_1, \ldots, a_n, \ldots]
		\]
		Понятно, что $p_n/q_n = [a_0; a_1, \ldots, a_n]$. При этом мы можем гарантировать индуктивное неравенство Дирихле, следующее из свойств подходящих дробей:
		\[
			\left|\alpha - \frac{p_n}{q_n}\right| \le \frac{1}{q_n q_{n + 1}} = \frac{1}{q_n (a_{n + 1}q_n + q_{n - 1})}
		\]
		где $a_{n + 1}$ - число, которое мы хотим найти. При этом мы хотим соблюсти другое неравенство:
		\[
			\frac{1}{q_n (a_{n + 1} q_n + q_{n - 1})} < \frac{1}{q_n \psi(q_n)}
		\]
		Отсюда можно получить явное неравенство на $a_{n + 1}$ и, соответственно, можно выбрать $a_{n + 1}$ из подходящих чисел.
	\end{enumerate}
\end{proof}

\begin{theorem} (без доказательства, Рота, 50-е годы XX в.)
	$\forall \alpha \in (\A \bs \Q) \cap \R$ выполнено утверждение:
	\[
		\forall \eps > 0\ \exists c > 0 \colon \left|\alpha - \frac{p}{q}\right| \ge \frac{c}{q^{2 + \eps}}
	\]
\end{theorem}

\begin{theorem} (без доказательства)
	$\forall \alpha \in \R$ существует $\{p_n/q_n\}_{n = 1}^\infty$ такая, что
	\[
		\left|\alpha - \frac{p_n}{q_n}\right| \le \frac{1}{q^2 \sqrt{5}}
	\]
\end{theorem}

\begin{note}
	То есть последняя теорема представляет собой улучшение результата Дирихле на коэффициент $\sqrt{5}$. Если при этом убрать из рассмотрения числа, которые линейно выражаются через $\phi = [1; (1)]$, то оценку можно улучшить до $\sqrt{8}$. Улучшать можно и дальше, если выкидывать числа, для которых оценка достигается. Доказано, что тогда выражение справа будет стремиться к $1/3$.
\end{note}