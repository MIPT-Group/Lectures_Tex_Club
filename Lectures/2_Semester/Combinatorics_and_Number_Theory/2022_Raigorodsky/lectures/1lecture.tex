\section{Основы комбинаторики и теории чисел}

\subsection{Распределение простых чисел}

\begin{definition}
	\textit{Пи-функцией от натурального числа } $x$ будем называть количество простых чисел, меньших либо равных $x$:
	\[
		\pi(x) = \suml_{p \le x} 1
	\]
\end{definition}

\begin{definition}
	\textit{Тета-функцией от натурального числа} $x$ будем называть сумму натуральных логарифмов простых чисел, меньших либо равных $x$:
	\[
		\Theta(x) = \suml_{p \le x} \ln p
	\]
\end{definition}

\begin{definition}
	\textit{Пси-функцией от натурального числа} $x$ будем называть сумму натуральных логарифмов от простых чисел $p$ по парам $(p, \alpha)$ так, что верно соотношение $p^\alpha \le x$:
	\[
		\psi(x) = \suml_{(p, \alpha) \colon p^\alpha \le x} \ln p
	\]
\end{definition}

\begin{theorem} (Чебышёва, 1848-1850гг.)
	Для всех достаточно больших $x$ при фиксированном $\eps$ верно, что
	\[
		\pi(x) \in \left[(1 - \eps) \cdot \ln 2 \frac{x}{\ln x}; (1 + \eps) \cdot 4\ln 2 \frac{x}{\ln x}\right]
	\]
\end{theorem}

\begin{theorem} (без доказательства, Адамара и Валле-Пуссена, 1896г.)
	На бесконечности для $\pi(x)$ справедливо следующее утверждение:
	\[
		\pi(x) \sim \frac{x}{\ln x},\ x \to \infty
	\]
\end{theorem}

\begin{proof} (теоремы Чебышёва)
	Обозначим следующие пределы через $\lambda$ и $\mu$:
	\begin{align*}
		&{\lambda_1 := \varlimsup\limits_{x \to \infty} \frac{\Theta(x)}{x};} & &{\mu_1 := \varliminf\limits_{x \to \infty} \frac{\Theta(x)}{x}} \\
		&{\lambda_2 := \varlimsup\limits_{x \to \infty} \frac{\phi(x)}{x};} & &{\mu_2 := \varliminf\limits_{x \to \infty} \frac{\phi(x)}{x}} \\
		&{\lambda_3 := \varlimsup\limits_{x \to \infty} \frac{\pi(x)}{x / \ln x};} & &{\mu_3 := \varliminf\limits_{x \to \infty} \frac{\pi(x)}{x / \ln x}}
	\end{align*}
	
	\begin{lemma}
		Утверждается, что
		\[
			\lambda_1 = \lambda_2 = \lambda_3;\ \ \mu_1 = \mu_2 = \mu_3
		\]
	\end{lemma}

	\begin{proof}
		
	\end{proof}
\end{proof}