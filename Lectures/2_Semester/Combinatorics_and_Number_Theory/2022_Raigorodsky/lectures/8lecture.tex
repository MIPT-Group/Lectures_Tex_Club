\begin{theorem}
	Число Эйлера $e$ - иррациональное.
\end{theorem}

\begin{proof}
	\textcolor{red}{Пока поверим, что}
	\[
		e = \suml_{k = 0}^\infty \frac{1}{k!}
	\]
	Предположим, что $e = a/n$ для некоторых $a \in \Z,\ n \in \N$. С одной стороны, очевидно, $en! \in \Z$. С другой стороны,
	\begin{multline*}
		en! = A + \frac{n!}{(n + 1)!} + \frac{n!}{(n + 2)!} + \ldots = A + \frac{1}{n + 1} + \frac{1}{(n + 1)(n + 2)} + \ldots =
		\\
		A + \frac{1}{n + 1} \left(1 + \frac{1}{n + 2} + \frac{1}{(n + 2)(n + 3)} + \ldots\right)
	\end{multline*}
	где $A \in \Z$, а сумма в скобках
	\[
		1 < 1 + \frac{1}{n + 2} + \frac{1}{(n + 2)(n + 3)} + \ldots < 1 + \frac{1}{2} + \frac{1}{4} + \frac{1}{8} + \ldots = 2
	\]
	То есть
	\[
		0 < \frac{1}{n + 1} \left(1 + \frac{1}{n + 2} + \frac{1}{(n + 2)(n + 3)} + \ldots\right) < \frac{2}{n + 1} \le \frac{2}{2} = 1
	\]
	Получили, что $en!$ не целое число. Противоречие.
\end{proof}

\begin{theorem}
	Числа Эйлера $e$ - трансцендентное.
\end{theorem}

\begin{lemma} (тождество Эрмита)
	Пусть $f(t) = b_{\nu}t^{\nu} + \ldots + b_1 t + b_0$, где $b_i \in \R$. Утверждается, что
	\[
		\int_0^x f(t) e^{-t} dt = F(0) - F(x)e^{-x}
	\]
	где $F(x) = f(x) + f'(x) + \ldots + f^{(\nu)}(x)$
\end{lemma}

\begin{proof}
	Рассмотрим следующий интеграл от многочлена $f(t)$:
	\[
		\int_0^x f(t) e^{-t} dt = -\int_0^x f(t) d(e^{-t}) = -f(t) e^{-t} |_0^x + \int_0^x f'(t) e^{-t} dt = f(0) - f(x)e^{-x} + \int_0^x f'(t)e^{-t} dt
	\]
	Рекурсивно вычислим оставшийся интеграл. В итоге получится следующее выражение:
	\begin{multline*}
		\int_0^x f(t) e^{-t} dt = \underbrace{f(0) - f(x)e^{-x} + f'(0) - f'(x)e^{-x} + \ldots + f^{(\nu)}(0) - f^{(\nu)}(x)e^{-x}}_{F(0) - F(x)e^{-x}} +
		\\
		\underbrace{\int_0^x f^{(\nu + 1)}(t) e^{-t} dt}_{0}
	\end{multline*}
\end{proof}

\begin{proof} (трансцендентности)
	Предположим, что $e$ - не трансцендентное. Это значит существование многочлена $g(x)$, для которого $e$ - это корень:
	\[
		g(x) = a_m x^m + \ldots + a_1 x + a_0
	\]
	Перепишем тождество Эрмита для $x = k \in \range{0, m}$ в следующем виде:
	\[
		e^k F(0) - F(k) = e^k \int_0^k f(t) e^{-t} dt
	\]
	Определим для числа $n \in \N$ многочлен $f(x)$:
	\[
		f(x) = \frac{1}{(n - 1)!} x^{n - 1} \big((x - 1) \cdot \ldots \cdot (x - m)\big)^n
	\]
	Теперь, рассмотрим следующую сумму для нашего $f(x)$ ($n$ выберем позже):
	\[
		\suml_{k = 0}^m a_k (e^k F(0) - F(k)) = F(0) \underbrace{\suml_{k = 0}^m a_k e^k}_{0} - \suml_{k = 0}^m a_k F(k) = \suml_{k = 0}^m a_k e^k \int_0^k f(t) e^{-t} dt 
	\]
	Чтобы посчитать $F(k)$, нам нужно посчитать все $\nu := n - 1 + nm$ производные в целых точках:
	\begin{itemize}
		\item Производные в нуле
		\begin{enumerate}
			\item Для $\mu \in \range{0}{n - 2}$ верно, что
			\[
				f^{(\mu)}(0) = 0
			\]
			так как за одно дифференцирование степень каждого слагаемого в производной уменьшается лишь на 1.
			
			\item Производная в нуле при $\mu = n - 1$ получается лишь из того слагаемого, где всегда брали производную лишь от $x^{n - 1}$. То есть
			\[
				f^{(n - 1)}(0) = (-1)^{mn} (m!)^n
			\]
			
			\item $\mu \ge n$. Тут уже конкретное значение производной сложно узнать. Тем не менее, можно заключить следующее:
			\[
				f^{(\mu)}(0) = nB,\ B \in \Z
			\]
		\end{enumerate}
		
		\item Производные для $k \in \range{1}{m}$
		\begin{enumerate}
			\item Для $\mu \in \range{0}{n - 1}$ очевидно
			\[
				f^{(\mu)}(k) = 0
			\]
			
			\item Для $\mu \ge n$ скажем то же, что и в последнем пункте про производные в нуле:
			\[
				f^{(\mu)}(k) = nC,\ C \in \Z
			\]
		\end{enumerate}
	\end{itemize}
	Посчитав производные, мы можем вычислить сумму:
	\[
		\suml_{k = 0}^m a_k F(k) = a_0 F(0) + \suml_{k = 1}^m a_k F(k) = a_0 (-1)^{mn}(m!)^n + a_0 nD + nE \equiv a_0 (-1)^{mn} (m!)^n \pmod n
	\]
	Выберем такое $n$, что $n > |a_0|$ и $(n, m!) = 1$. Тогда гарантированно можем заявить следующее:
	\[
		\left|-\suml_{k = 0}^m a_k F(k)\right| \ge 1
	\]
	Если мы докажем, что правая часть исходного равенства стремится к нулю при $n \to \infty$, то мы получим необходимое противоречие. Рассмотрим эту часть:
	\[
		\left|\sum_{k = 0}^m a_k e^k \int_0^k f(t) e^{-t} dt\right| \le \suml_{k = 0}^m |a_k| e^k \int_0^k |f(t)| e^{-t} dt
	\]
	Оценим модуль $|f(t)|$. Заметим, что
	\[
		\forall t \in [0; m]\ \forall i \in \range{0}{m}\ \ |t - i| \le m
	\]
	Отсюда следует
	\begin{multline*}
		\suml_{k = 0}^m |a_k|e^k \int_0^k |f(t)|e^{-t} dt \le \suml_{k = 0}^m |a_k|e^k \int_0^k \frac{1}{(n - 1)!} m^{mn + n - 1} e^{-t} dt =
		\\
		\frac{1}{(n - 1)!} m^{mn + n - 1} \suml_{k = 0}^m |a_k| e^k \left(\frac{1}{e^0} - \frac{1}{e^k}\right) \le \frac{1}{(n - 1)!} \underbrace{m^{mn + n - 1} \suml_{k = 0}^m |a_k|e^k}_{C^n \cdot C'} \xrightarrow[n \to \infty]{} 0
	\end{multline*}
\end{proof}

\begin{theorem} (без доказательства)
	Число Пи $\pi$ - трансцендентное.
\end{theorem}

\begin{theorem} (Гельфонда, 1929г., одна из проблем Гильберта. Без доказательства)
	Если $\alpha, \beta \in \A$, $\alpha \notin \{0, 1\}$, $\beta \notin \Q$, то $\alpha^\beta \notin \A$.
\end{theorem}

\begin{corollary}
	$e^\pi \notin \A$ - тоже трансцендентное число.
\end{corollary}

\begin{proof}
	Предположим, что $e^\pi \in \A$. Но заметим, что $i \in \A \bs \Q$:
	\[
		(e^\pi)^i = e^{i\pi} = -1 \in \A
	\]
	Получили противоречие с теоремой Гельфонда.
\end{proof}

\begin{definition}
	Числа $x, y$ называются \textit{алгебраически независимыми}, если для любого многочлена $P$ выполнено неравенство:
	\[
		P(x, y) \neq 0
	\]
\end{definition}

\begin{theorem} (Нестеренко. Без доказателсьтва)
	Числа $\pi, e^\pi, \Gamma(1/4)$ алгебраически независимы. $\Gamma(x)$ - обобщение факториала на всю действительную числовую прямую.
\end{theorem}

\subsection{Решётки в $\R^n$}

\begin{note}
	Зафиксируем линейно независимые вектора $\vec{a}_1, \ldots, \vec{a}_k \in \R^n$.
\end{note}

\begin{definition}
	\textit{Решёткой} $\Lambda$ в пространстве $\R^n$ будем называть следующее множество векторов:
	\[
		\Lambda = \{b_1\vec{a}_1 + \ldots + b_n\vec{a}_k \such \forall i\ b_i \in \Z\}
	\]
\end{definition}

\begin{definition}
	Вектора $\vec{a}_1, \ldots, \vec{a}_k$ также можно назвать \textit{базисом решётки} $\Lambda$.
\end{definition}

\begin{note}
	Базис в $\Lambda$, понятное дело, определён неоднозначно. Тем не менее, матрица перехода от базиса к базису должна быть целочисленной, а следовательно, её детерминант - это $\pm 1$. Отсюда вытекает следующее определение:
\end{note}

\begin{definition}
	\textit{Определителем решётки} называется модуль детерминанта матрицы
	\[
		\det \Lambda = \left|\det \Matrix {\alpha_1 & &\cdots & &\alpha_k}\right|,\ \ \vec{a}_i \leftrightarrow_e \alpha_i
	\]
	где $e$ - ортонормированный базис
\end{definition}

\begin{note}
	В $\R^2$ $\det \Lambda$ символизирует объём параллелограмма, натянутого на базисные вектора решётки. В $\R^3$ - объём соответственно.
\end{note}

\begin{definition}
	Множество $\Omega \subset \R^2$ называется \textit{выпуклым}, если для любых двух точек из этого множества отрезок, соединяющих их, тоже лежит в этом множестве.
\end{definition}

\begin{definition}
	Зафиксируем ПДСК на плоскости. Будем называть множество $F \subset \R^2$ \textit{простым}, если оно представляет собой совокупность прямоугольников, чьи стороны параллельны осям.
\end{definition}

\begin{definition}
	\textit{Площадь простой фигуры} $F$ - это просто сумма площадей прямоугольников, её составляющих. 
\end{definition}

\begin{definition}
	Будем говорить, что множество $\Omega \subset \R^2$ обладает \textit{площадью} $S$, если выполнено условие:
	\[
		\mu_*(\Omega) = \mu^*(\Omega)
	\]
	где $\mu_*(\Omega) = \sup\limits_{F \subset \Omega} S(F)$, $\mu^*(\Omega) = \inf\limits_{\Omega \subset F} S(F)$
\end{definition}

\begin{note}
	Величина $S$ является частным случаем \textit{меры Жордана}.
\end{note}

\begin{theorem} (Минковского)
	Пусть $\Omega \subset \R^2$, $S(\Omega) > 4$, $S$ - выпукло и центрально симметрично относительно центра ПДСК. Тогда
	\[
		(\Omega \cap \Z^2) \bs \{0\} \neq \emptyset
 	\]
\end{theorem}