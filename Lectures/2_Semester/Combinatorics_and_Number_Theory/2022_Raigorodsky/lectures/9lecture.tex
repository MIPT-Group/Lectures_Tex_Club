\begin{proof}
	\textcolor{red}{Здесь стоит вообразить картинку некоторого овала, покрытого сеточкой}
	Рассмотрим решётку $(1/p)\Z^2,\ p \in \N$. Обозначим за $N_p$ - количество точек в пересечении этой решётки с $\Omega$:
	\[
		N_p := \left|\frac{1}{p}\Z^2 \cap \Omega\right|
	\]
	Без доказательства поверим, что если мы возьмём площади квадратиков, у которых левой верхней вершиной выступает точка из $N_p$, то эта площадь будет стремиться в $S(\Omega)$:
	\[
		N_p \cdot \frac{1}{p^2} \xrightarrow[p \to \infty]{} S(\Omega) > 4
	\]
	Следовательно
	\[
		\exists P \in \N \such \forall p \ge P\ \ N_p \cdot \frac{1}{p^2} > 4
	\]
	Это можно записать в несколько другом виде:
	\[
		N_p > (2p)^2
	\]
	Как задаётся любая точка в решётке $(1/p)\Z^2$? Её координаты будут иметь вид
	\[
		\vec{v} \leftrightarrow_e \left(v_1 / p,\ v_2 / p\right)^T,\ v_1, v_2 \in \Z
	\]
	Из неравенства на $N_p$ по принципу Дирихле следует, что
	\[
		\exists \vec{a}, \vec{b} \in \Omega \such a_1 \equiv b_1\!\!\!\! \pmod {2p},\quad a_2 \equiv b_2\!\!\!\! \pmod {2p}
	\]
	Теперь мы можем рассмотреть точку $\vec{c} = (\vec{a} + \vec{b}) / 2 \neq \vec{0}$. В силу центральной симметрии, $-\vec{b} \in \Omega$, а из-за выпуклости $\vec{c} \in \Omega$. Более того, эта точка - целая, так как
	\[
		a_i - b_i \equiv 0 \pmod {2p}
	\]
\end{proof}

\begin{theorem} (без доказательства)
	Пусть $\Omega \subset \R^2$ - выпуклое, центрально симметричное, замкнутое множество. При этом $S(\Omega) \ge 4$. Тогда
	\[
		(\Omega \cap \Z^2) \bs \{0\} \neq \emptyset
	\]
\end{theorem}

\begin{corollary}
	Следствием уже этой теоремы является геометрическое доказательство теоремы Дирихле.
	
	Пусть $\alpha \in \R \bs \Q$. Тогда, рассмотрим множество точек $\{(x, y) \colon\ |y - \alpha x| \le 1 / Q,\ |x| \le Q\}$, где $Q \in \N$.
	
	\textcolor{red}{Здесь должен быть рисунок, который можно найти в 26й лекции ОКТЧ за 2022й год на моменте 1:03:20}
	
	Это множество точек задаёт параллелограмм около прямой $y = \alpha x$, и его площадь $S = (2 / Q) \cdot 2Q = 4$. Согласно теореме, внутри него найдётся целая нетривиальная точка $(q, p)$. Повторим операцию, <<растянув>> параллелограмм до такого состояния, что эта точка уже в него не входит. Очевидно $q \le Q$, а поэтому
	\[
		|p - \alpha q| \le \frac{1}{Q} \le \frac{1}{q}
	\]
	откуда уже возникает знакомое неравенство теоремы Дирихле.
\end{corollary}

\begin{theorem} (Многомерная теорема Минковского)
	Пусть $\Omega \subset \R^n$ - выпуклое, центрально симметричное множество, $V(\Omega) > 2^n$. Тогда
	\[
		(\Omega \cap \Z^2) \bs \{0\} \neq \emptyset
	\]
\end{theorem}

\begin{proof}
	Абсолютно аналогично случаю на плоскости, просто мера маленького $n$-мерного гиперкуба теперь $(1/p^n)$.
\end{proof}

\begin{theorem}
	Пусть $\Omega \subset \R^n$ - выпуклое, центрально симметриченое множество. $\Lambda \subset \R^n$ - произвольная решётка, причём $V(\Omega) > 2^n \cdot \det \Lambda$. Тогда
	\[
		(\Omega \cap \Lambda) \bs \{0\} \neq \emptyset
	\]
\end{theorem}

\begin{proof}
	\textcolor{red}{Тут можно произвести геометрическое доказательство, исходя из элементарной центрально симметричной фигуры, образованной решёткой, чей объём будет в аккурат $2^n \cdot \det \Lambda$.}
\end{proof}

\begin{definition}
	\textit{Критическим определителем} $\Omega \subset \R^n$ называется следующая величина
	\[
		\Delta(\Omega) := \inf \{\det \Lambda \such (\Lambda \cap \Omega) \bs \{0\} = \emptyset\}
	\]
\end{definition}

\begin{theorem} (Минковского через критический определитель, без доказательства)
	Если $\Omega \subset \R^n$ - выпуклое и центрально симметричное множество, то
	\[
		\frac{V(\Omega)}{\Delta(\Omega)} \le 2^n
	\]
\end{theorem}

\begin{theorem} (Минковского-Главки, 1945 г., без доказательства)
	Для любого $\Omega \subset \R^n$ верна оценка
	\[
		\frac{V(\Omega)}{\Delta(\Omega)} \ge 1 - \eps(n)
	\]
	где $\eps(n) \xrightarrow[n \to \infty]{} 0$
\end{theorem}

\begin{proposition}
	Эквивалентным определением решётки будет следующее утверждение:
	
	$\Lambda \subset \R^n$ - решётка, если она:
	\begin{enumerate}
		\item Образует дискретное множество в $\R^n$ (то есть любая точка $\Lambda$ изолирована). То есть $\exists r \such \forall \vec{x} \in \R^n$ \textit{в шаре с центром $\vec{x}$ и радиуса $r$ не больше одной точки этого множества}.
		
		\item $\exists R \such \forall \vec{x} \in \R^n$ в шаре с центром $\vec{x}$ и радиуса $R$ есть хотя бы одна точка этого множества.
		
		\item $\Lambda$ - подгруппа $\R^n$ по сложению.
	\end{enumerate}
\end{proposition}

\begin{proof}~
	\begin{itemize}
		\item $\Ra$ Из определения решётки 3 свойства очевидны.
		
		\item $\La$ \textcolor{red}{А вот это красиво написать пока не удалось.}
	\end{itemize}
\end{proof}

\begin{definition}
	Рассмотрим следующий вектор в $\Z^n$:
	\[
		\vec{a} \leftrightarrow_e \left(\frac{a_1}{q}, \ldots, \frac{a_n}{q}\right)^T,\quad (a_1, \ldots, a_n, q) = 1
	\]
	Тогда, обозначим за $\Lambda_{\vec{a}}$ решётку следующего вида:
	\[
		\Lambda_{\vec{a}} = \{\vec{a}l + \vec{b},\ l \in \Z, \vec{b} \in \Z^n\}
	\]
	$\Lambda_{\vec{a}}$ называется \textit{циклической центрировкой} $\Z^n$. Это связано со следующим фактом:
	\[
		\Lambda_{\vec{a}} / \Z^n = \trbr{\vec{a}}
	\]
\end{definition}

\begin{note}
	То есть фактически мы взяли все целые точки $\Z^n$ и объединили это множество с другими $\Z^n$, где каждая точка сдвинулась на $\vec{a}l$.
\end{note}

\begin{proposition}
	Для вектора $\vec{a}$ из определения следует, что
	\[
		\Lambda_{\vec{a}} \subset \frac{1}{q} \Z^n
	\]
\end{proposition}

\begin{theorem} (без доказательства)
	Утверждается, что
	\[
		\det \Lambda_{\vec{a}} = \frac{1}{q}
	\]
\end{theorem}

\begin{reminder}
	$n$-мерный октаэдр обозначается как $O^n$, в $\R^n$ задаётся уравнением
	\[
		|x_1| + \ldots + |x_n| \le 1
	\]
	и имеет меру, равную
	\[
		V(O^n) = \frac{2^n}{n!}
	\]
\end{reminder}

\begin{corollary}
	Если в $O^n$ нет нетривиальных точек (то есть нуля и его вершин) $\Lambda_{\vec{a}}$, то в каноническом разложении $q$ число простых множителей $\le n$.
\end{corollary}

\begin{proof}
	По теореме Минковского
	\[
		V(O^n) \le 2^n \cdot \det \Lambda_{\vec{a}}
	\]
	Подставим известные величины и получим следующее неравенство:
	\[
		\frac{2^n}{n!} \le 2^n \cdot \frac{1}{q} \lra q \le n!
	\]
	Разложим $q$ в произведение простых и сделаем самую базовую оценку:
	\[
		1 \cdot \ldots \cdot s \le p_1^{\alpha_1} \cdot \ldots \cdot p_s^{\alpha_s} \le n! \Ra s < n
	\]
\end{proof}

\begin{theorem}
	В случае $n$-мерного октаэдра существуте более сильный аналог теоремы Минковского-Главки:
	\[
		\forall \eps > 0\ \exists n_0 \in \N \such \forall n \ge n_0\ \exists \vec{a} \leftrightarrow_e \left((a_1/q), \ldots, (a_n/q)\right)^T
	\]
	такой, что в $O^n$ нет нетривиальных точек $\Lambda_{\vec{a}}$ и при этом
	\[
		\frac{V(O^n)}{\det \Lambda_{\vec{a}}} \ge 1 - \eps
	\]
\end{theorem}