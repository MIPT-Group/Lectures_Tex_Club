\subsection{Основные свойства интеграла Римана-Стильтьеса}

\begin{note}
	До конца параграфа зафиксируем возрастающую функцию $\alpha$, а также то, что мы рассматриваем интеграл на отрезке $[a; b]$.
\end{note}

% Он что-то непонятное писал в начале лекции. Посмотреть, дописать

\begin{theorem}
	Пусть $f, \alpha' \in R[a; b]$. Тогда $f \in R(\alpha, [a; b])$ и при этом
	\[
		\int_a^b f(x) d(\alpha(x)) = \int_a^b f(x) \alpha'(x) dx
	\]
\end{theorem}

\begin{proof}
	Коль скоро $\alpha'$ интегрируема по Риману, то $\alpha'$ - ограниченная, а стало быть $\alpha$ - непрерывная функция ограниченной вариации. \textcolor{red}{Ведь так?} Распишем, что означает интегральная сумма слева от равенства:
	\[
	`	\forall \eps > 0\ \exists \delta_1 > 0 \such \forall P, \Delta(P) < \delta_1\ \ \forall \{t_i\}\ \left|\suml_{i = 1}^n f(t_i) \alpha'(t_i) \Delta x_i - \int_a^b f(x) \alpha'(x) dx\right| < \frac{\eps}{4}
	\]
	По уже заявленному, $\exists M \such \forall x \in [a; b]\ |f(x)| \le M$. Запишем интеграл Римана от $\alpha'(x)$ по определению:
	\[
		\forall \eps > 0\ \exists \delta_2 > 0 \such \forall P, \Delta(P) < \delta_2\ \ \forall \{t_i\}\ \left|\suml_{i = 1}^n \alpha'(t_i)\Delta x_i - \int_a^b \alpha'(x)dx\right| < \frac{\eps}{4M}
	\]
	
	% Дописать, я ничего не понимаю(
\end{proof}

\begin{theorem}
	Пусть $f$ ограничена на $[a; b]$ и непрерывна в точках $\{x_1, \ldots, x_N\} \subset (a; b)$. Функция $\alpha$ постоянна между этими точками, а значит непрерывна на $[a; b] \ \{x_1, \ldots, x_N\}$. Тогда $f \in R(\alpha, [a; b])$ и
	\[
		\int_a^b f(x)d(\alpha(x)) = \suml_{k = 1}^N f(x_k) (\alpha(x_k + 0) - \alpha(x_k - 0))
	\]
\end{theorem}

\begin{proof}
	%  Сюда стоит запихнуть картинку с лекции
	Положим $P_\delta := \bigcup\limits_{i = 1}^N \{x_i \pm \delta\} \cup \{a, b\}$. Посмотрим на верхнюю сумму интеграла Римана-Стильтьеса по данному разбиению:
	\[
		U(P_\delta, f, \alpha) = \suml_{k = 1}^N \sup\limits_{x \in [x_k - \delta; x_k + \delta]} f(x) (\alpha(x_k + 0) - \alpha(x_k - 0))
	\]
	Аналогично поступим с нижней суммой:
	\[
		L(P_\delta, f, \alpha) = \suml_{k = 1}^N \inf\limits_{x \in [x_k - \delta; x_k + \delta]} f(x) (\alpha(x_k + 0) - \alpha(x_k - 0))
	\]
	Посмотрим на разность этих сумм:
	\[
		U(P_\delta, f, \alpha) - L(P_\delta, f, \alpha) = \suml_{k = 1}^N (M_k(\delta) - m_k(\delta)) (\alpha(x_k + 0) - \alpha(x_k - 0))
	\]
	Несложно понять, что при $\delta \to 0+$ каждое слагаемое в сумме будет тоже стремиться к нулю. Отсюда следует, что и разность стремится к нулю.
	\textcolor{red}{А дальше я не понял}
\end{proof}

\begin{theorem}
	Пусть $f$ - непрерывна, $\alpha$ - ограниченной вариации или же $f$ - ограниченной вариации и $\alpha$ - непрерывная функция ограниченной вариации. Тогда, если положить $v(x) := V(\alpha, [a; x])$, то
	\[
		\left|\int_a^b f(x) d(\alpha(x))\right| \le \int_a^b |f(x)| d(v(x))
	\]
\end{theorem}

\begin{proof}
	Оценим модуль интеграла:
	\begin{multline*}
		|S(P, f, \{t_i\}, \alpha)| = |\suml_{i = 1}^n f(t_i) \Delta \alpha_i| \le \suml_{i = 1}^n |f(t_i)| |\alpha(x_i) - \alpha(x_{i - 1})| \le
		\\
		\suml_{i = 1}^n |f(t_i)| V(\alpha, [x_{i - 1}; x_i]) = S(P, |f|, \{t_i\}, v)
	\end{multline*}
	Так как интеграл - это предел сумм, то из данного неравенства уже следует требуемое.
\end{proof}

\begin{corollary}
	Если дополнительно известно, что $|f(x)| \le M$, то
	\[
		\left|\int_a^b f(x) d(\alpha(x))\right| \le M \cdot V(\alpha, [a; b])
	\]
\end{corollary}

\begin{theorem} (Формула интегрирования по частям для интеграла Римана-Стильтьеса)
	Если $f, \alpha$ - функции ограниченной вариации, $f$ - непрерывна на $[a; b]$, то
	\[
		\int_a^b f d\alpha = f(b)\alpha(b) - f(a)\alpha(a) - \int_a^b \alpha df
	\]
\end{theorem}

\begin{proof}
	\textcolor{red}{Скоро будет}
\end{proof}

\begin{theorem} (Первая теорема о среднем для интеграла Римана-Стильтьеса)
	Если $f$ непрерывна на $[a; b]$, $\alpha$ - неубывающая функция на $[a; b]$, то
	\[
		\exists \xi \in [a; b] \such \int_a^b f(x) d(\alpha(x)) = f(\xi)(\alpha(b) - \alpha(a))
	\]
\end{theorem}

\begin{proof}
	Снова распишем сумму по разбиению:
	\[
		S(P, f, \{t_i\}, \alpha) = \suml_{i = 1}^n f(t_i) \Delta \alpha_i \le \max\limits_{x \in [a; b]} (\alpha(b) - \alpha(a))
	\]
\end{proof}

\begin{theorem} (Формула Бонн\'{е} для интеграла Римана-Стилтьеса)
	Пусть $f$ - монотонная функция ограниченной вариации на $[a; b]$, $\alpha$ - непрерывная на $[a; b]$. Тогда
	\[
		\exists \xi \in [a; b] \such \int_a^b f(x) d(\alpha(x)) = f(a)(\alpha(\xi) - \alpha(a)) + f(b)(\alpha(b) - \alpha(\xi))
	\]
\end{theorem}

\begin{proof}
	Применим формулу интегрирования по частям:
	\[
		\int_a^b f(x) d(\alpha(x)) = f(b)\alpha(b) - f(a)\alpha(a) - \int_a^b \alpha(x) d(f(x))
	\]
	Сделаем ещё один переход, используя первую теорему о среднем для оставшегося интеграла:
	\[
		f(b)\alpha(b) - f(a)\alpha(a) - \int_a^b \alpha(x) d(f(x)) = f(b)\alpha(b) - f(a)\alpha(a) - \alpha(\xi)(f(b) - f(a))
	\]
\end{proof}

\begin{theorem} (Формула замены переменной)
	Пусть $f, \phi$ - непрерывны на $[a; b]$, $\phi$ - строго монотонна, а $\phi := \phi^{-1}$. Тогда
	\[
		\int_a^b f(x)dx = \int_\phi(a)^\phi(b) f((\phi(t))) d\psi(t)
	\]
\end{theorem}

\begin{proof}
	Возьмём произвольное разбиение $P$ отрезка $[a; b]$. Будем считать, что $\phi$ возрастает, не умаляя общности. Тогда
	\[
		\phi(P) \colon \phi(a) < \phi(x_1) < \ldots < \phi(b)
	\]
	Обозначим $t_0 := \phi(a), t_n := \phi(b), t_i := \phi(x_i)$ и рассмотрим интегральную сумму:
	\[
		S(P, f, \{t_i\}) = \suml_{i = 1}^n f(t_i) \Delta x_i = \suml_{i = 1}^n f(\psi(\xi_i))(\psi(t_i) - \psi(t_{i - 1})) = S(\phi(P), f \circ \psi, \psi)
	\]
\end{proof}

\subsection{Несобственный интеграл Римана}

\begin{definition}
	Пусть $f$ интегрируема по Риману на $\forall \tilde{b} \in [a; b)\ [a; \tilde{b}]$. Тогда $\int_a^b f(x)dx$ - \textit{несобственным интегралом Римана}
	\begin{itemize}
		\item \textit{первого рода}, если $b = +\infty$
		
		\item \textit{второго рода}, если $b < +\infty$ и $f$ не является ограниченной на $[a; b)$
	\end{itemize}
	Его понимают как $\liml_{\tilde{b} \to b-0} \int_a^{\tilde{b}} f(x)dx$ или $\liml_{\tilde{b} \to +\infty} \int_a^{\tilde{b}} f(x)dx$. Если этот предел существует и конечен, то несобственный интеграл \textit{сходится}. В противном случае, интеграл \textit{расходится}.
\end{definition}