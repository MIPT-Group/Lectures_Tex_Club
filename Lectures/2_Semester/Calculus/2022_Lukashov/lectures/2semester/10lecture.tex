\begin{theorem} (Критерий Коши сходимости несобственного интеграла Римана)
	Пусть $f \in R[a;b]\ \forall \tilde{b},\ a < \tilde{b} < b$. Тогда 
	\[
		\int_a^bf(x)dx \Longleftrightarrow \forall \eps > 0\ \exists B \in (a;b)\ \forall B_1, B_2,\ B < B_1 < B_2 < b\ \left | \int_{B_1}^{B_2}f(x)dx\right | < \eps
	\]
\end{theorem}

\begin{proof}
	$F(x) = \int_a^xf(t)dt\ x \in [a;b)$.
	\[
		\int_a^bf(x)dx \text{ сходится } \Longleftrightarrow \exists \liml_{x \to b - 0} F(x) \in \R
	\]
	По критерию Коши для существования конечного предела:
	\[
		\forall \eps > 0\ \exists B \in (a;b)\ \forall B_1, B_2,\ B < B_1 < B_2 < b\ |F(B_2) - F(B_1)| < \eps
	\]
	Причем, $F(B_2) - F(B_1) = \int_{B_1}^{B_2}f(x)dx$
\end{proof}

\begin{lemma}
	Пусть $f(x) \geq 0\ \forall x \in (a;b),\ f \in R[a;b]\ \forall \tilde{b},\ a < \tilde{b} < b$. Тогда
	\[
		\int_a^bf(x)dx \text{ сходится } \Longleftrightarrow F(x) = \int_a^xf(t)dt \text{ ограничена на } [a;b)		
	\]
\end{lemma}

\begin{proof}
	$f(x) \geq 0 \Ra F(x)$ монотонно неубывающая, а значит существует предел. Он будет конечен лишь, когда функция ограничена.
\end{proof}

\begin{theorem} (Признак сравнения)
	Пусть $f, g \in R[a;b],\ \forall \tilde{b},\ a < \tilde{b} < b,\ \forall x \in [a;b)\ 0 \leq f(x) \leq g(x)$. Тогда из сходимости $\int_a^bg(x)dx$ следует сходимость $\int_a^bf(x)dx$, а из расходимости $\int_a^bf(x)dx$ следует расходимость $\int_a^bg(x)dx$.
\end{theorem}

\begin{proof}
	$F(x) = \int_a^xf(t)dt,\ G(x) = \int_a^xg(t)dt$.  Причем
	\[
		0 \leq F(x) \leq G(x)
	\]
	Тогда по лемме 1 верны утверждения теоремы.
\end{proof}

\begin{example}
	\[
		\int_0^b \frac{dx}{x^\alpha} = \liml_{\eps \to +0} \int_\eps^1\frac{dx}{x^\alpha} 
	\]
	Интересует лишь $\alpha > 0$.
	\begin{equation}
		\liml_{\eps \to +0}
		\begin{cases}
			\frac{x^{1 - \alpha}}{1 - \alpha}\ |_\eps^1 , & \alpha \neq 1
			 \\
			\ln x\ |_\eps^1 & \alpha = 1
		\end{cases}
	\end{equation}
	Получается, что сходится при $\alpha < 1$.
\end{example}

\begin{example}
	\[
	\int_1^{+\infty} \frac{dx}{x^\alpha} = \liml_{b  \to +\infty} \int_1^b\frac{dx}{x^\alpha} 
	\]
	\begin{equation}
		\liml_{b \to +\infty}
		\begin{cases}
			\frac{x^{1 - \alpha}}{1 - \alpha}\ |_1^{+\infty} , & \alpha \neq 1
			\\
			\ln x\ |_1^{+\infty} & \alpha = 1
		\end{cases}
	\end{equation}
	Получается, что сходится при $\alpha > 1$.
\end{example}

\begin{example}
	\[
		\int_0^{+\infty}\frac{dx}{x^\alpha} = \int_0^1\frac{dx}{x_\alpha} + \int_1^{+\infty}\frac{dx}{x^\alpha}
	\]
	Расходится по прошлым 2 примерам.
\end{example}

\begin{example}
	\[
	\int_{-1}^1\frac{dx}{x} = \int_{-1}^0\frac{dx}{x} + \int_0^1\frac{dx}{x}
	\]
	Расходится, так как оба интеграла расходятся.
	если ввести понятие главноого значения (в смысле Коши) v.p(value principale)/p.v(principle value), то
	\[
		v.p\ \int_{-1}^1\frac{dx}{x} = \liml_{\eps \to 0} \left ( \int_{-1}^{-\eps}\frac{dx}{x}  + \int_\eps^1\frac{dx}{x}\right ) = 0
	\]
	Еще одним примером может служить такой интеграл:
	\[
			v.p.\ \int_{-\infty}^\infty\frac{dx}{x} = \liml_{\eps \to 0,\ b \to \infty} \left ( \int_{-b}^{-\eps}\frac{dx}{x}  + \int_\eps^b\frac{dx}{x}\right ) = 0
	\]
\end{example}

\begin{corollary}
	Если $f, g \in R[a;\tilde{b}],\ \forall \tilde{b},\ a < \tilde{b} < b$ и $f(x) \geq 0,\ g(x) \geq 0$.
	\[
		\exists \liml_{x \to b - 0(+\infty)}\frac{f(x)}{g(x)} = C \in (0;+\infty) \Ra \int_a^bf(x)dx \text{ и } \int_a^bg(x)dx 
	\]
	Сходятся и расходятся одновременно. Так как в некоторой окрестности $b$ верно, что $0 \leq \frac{f(x)}{g(x)} \leq C$
\end{corollary}

\begin{definition}
	Пусть $f \in R[a;\tilde{b}]\ \forall \tilde{b}, a < \tilde{b} < b$, Если сходится $\int_a^b|f(x)|dx$, то говорят, что $\int_a^bf(x)dx$ сходится абсолютно.
\end{definition}

\begin{theorem}
	Если $\int_a^bf(x)dx$ сходится абсолютно, то он сходится.
\end{theorem}

\begin{proof}
	Так как функция сходится абсолютно, то по критерию Коши можно записать:
	\[
		\forall \eps > 0\ \exists B, a < B < b\ \forall B_1, B_2, B < B_1 < B_2 < b\ \left | \int_{B_1}^{B_2}|f(x)|dx\right | < \eps
	\]
	очевидно, что 
	\[
		\left | \int_{B_1}^{B_2}f(x)dx \right | \leq \int_{B_1}^{B_2}|f(x)|dx < \eps \Ra \int_a^bf(x)dx \text{ сходится}
	\]
\end{proof}

\begin{definition}
	Если $\int_a^bf(x)dx$ сходится, но не сходится абсолютно, то говорят, что он сходится относительно.
\end{definition}

\begin{theorem}(Признак Дирихле)
	Если
	\begin{enumerate}
		\item $f \in R[a;b]\ \forall \tilde{b}, a < \tilde{b} < b$ и $F(x) = \int_a^xf(t)dt$ ограничена на $[a;b)$.
		\item $g$ монотонна на $[a;b)$ и бесконечно мало при $x \to b - 0(+\infty)$
	\end{enumerate}
	то $\int_a^bf(x)g(x)dx$ сходится.
\end{theorem}

\begin{proof}
	$\forall B_1 < B_2$, причем $a < B_1 < B_2 < b$. Тогда по формуле Бонне:
	\[
		\int_{B_1}^{B_2}f(x)g(x)dx = g(B_1)\int_{B_1}^\xi f(x)dx + f(B_2)\int_\xi^{B_2}f(x)dx,\ B_1 < \xi < B_2
	\]
	Из ограниченности $\exists M\ \forall x \in [a;b)\ |F(x)| \leq M$. 
	Так как $g$ бесонечно мала, то:
	\[
		\forall \eps > 0\ \exists B,\ a < B < b \forall x \in(B; b) |g(x)| < \frac{\eps}{4M}
	\]
	Теперь выберем случайные $B_1,\ B_2$, для которых верно, что $B < B_1 < B_2 < b$. А для них спарведливо, что:
	\[
		\left | \int_{B_1}^{B_2}f(x)g(x)dx\right | \leq |g(B_1)||F(\xi) - F(B_1)| + |g(B_2)||F(B_2) - F(\xi)| < \eps
	\]
\end{proof}

\begin{corollary} (Признак Абеля)
	Если 
	\begin{enumerate}
		\item $\int_a^bf(x)dx$ сходится.
		\item $g$ монотонна и ограничена на $[a;b)$.
	\end{enumerate}
	то $\int_a^bf(x)g(x)dx$ сходится.
\end{corollary}

\begin{proof}
	Сходимость $\int_a^bf(x)dx$ влечет ограниченность$F(x) = \int_a^xf(t)dt$. Также определим функцию $g_1(x) = g(x) - L$, где $L = \liml_{x\to b - 0(+\infty)}g(x)$.
	Из этих двух утверждений по признаку Дирихле получаем:
	\[
		\int_a^bf(x)g_1(x)dx = \int_a^bf(x)g(x)dx - L\int_a^bf(x)dx
	\]
	Левая часть сходится по признаку, второй член в правой части по положению теоремы.
\end{proof}

\begin{example}
	Рассмотрим сразу 2 интеграла.
	\[
		\int_1^{+\infty}\frac{\sin x}{x^\lambda} \left (\int_1^{+\infty}\frac{\cos x}{x^\lambda}\right )
	\]
	Заметим, что $|\frac{\sin x(\cos x)}{x^\lambda}| \leq \frac{1}{x^\lambda}$. Ранее было показано, что $\int_1^{+\infty}\frac{dx}{x^\lambda} \Longleftrightarrow \lambda > 1$ сходится абсолютно.
	$\\$
	Возьмем $f(x) = \sin x(\cos x) \Ra F(x) = -\cos x(\sin x) + C$ (ограничена). $g(x) = \frac{1}{x^\lambda}$.
	Тогда при $\lambda > 0$ интегралы сходятся по признаку Дирихле.
	$\\$
	Теперь оценим $|\frac{\sin x(\cos x)}{x^\lambda}| \geq \frac{\sin^2 x(\cos^2 x)}{x^\lambda} = \frac{1 - \cos 2x(1 + \cos 2x)}{2x^\lambda}$. Нетрудно показать, что $\int_1^{+\infty}\frac{\sin^2 x}{x^\lambda}$  расходится при $\lambda \in (0;1]$. Откуда по признаку сравнения видим, что рассматриваемые интагралы сходятся \textbf{условно} при $\lambda > 0$.
	$\\$
	Покажем по критерию Коши расходимость интегралов при $\lambda \leq 0$. Положим $B_1 = \frac{\pi}{2} + 2\pi n\ B_2 = \frac{3\pi}{4} + 2\pi n,\ x \in [B_1; B_2] \Ra \sin x \leq \frac{\sqrt {2}}{2}$. 
	\[
		\left | \int_{B_1}^{B_2}\frac{\sin x}{x^\lambda}dx\right | \geq \frac{\sqrt{2}}{2} \int_{B_1}^{B_2}\frac{dx}{x^\lambda} \geq \frac{\sqrt{2}}{2}B_1^{-\lambda}(B_2 - B_1) = \frac{\pi \sqrt{2}}{4}(\frac{\pi}{4})^{-\lambda}
	\]
\end{example}

\begin{theorem} (Признак Харди)
	Пусть $f$ - периодическая(с периодом $\omega$) и интегрируемая по Риману на любом отрезке $[a;b]$ функция. $g$ - монотонная, бесконечно малая при $x\to +\infty$. Тогда
	\begin{enumerate}
		\item если $\int_a^{a + \omega}f(x)dx = 0$, то $\int_a^{+\infty}f(x)g(x)dx$ сходится.
		\item если $\int_a^{a + \omega}f(x)dx \neq 0$, то $\int_a^{+\infty}f(x)g(x)dx$ и $\int_a^{+\infty}g(x)dx$ сходятся и расходятся одновременно.
	\end{enumerate}
\end{theorem}

\begin{proof}
	\begin{enumerate}
		\item Пусть $F(x) = \int_a^xf(t)dt = \left | (\int_a^{a + \omega} + \int_{a +m \omega}^{a + 2\omega} + \ldots \int_{a + k\omega}^x)f(t)dt \right |$, где  $x \in [a + k\omega;a + (k + 1)\omega)$. 
		Так как все интегралы между крайними равны 0, то получаем $\left | \int_{a + k\omega}^xf(t)dt\right | \leq M$, где из непрерывности $\exists M \such |F(x)| \leq M\ x \in [a;a + \omega]$.
		\item Пусть $K = \int_a^{a + \omega}$, а $f_1(x) := f(x) - \frac{K}{\omega}$. По этим положениям $\int_a^{a + \omega}f_1(x)dx = 0$. то есть по первому пункту $\int_a^{+\infty}f_1(x)g(x)dx$ сходится. Однако, если подставить $f_1$, то:
		\[
			\int_a^{+\infty}f_1(x)g(x)dx = \int_a^{+\infty}f(x)g(x)dx - \frac{K}{\omega}\int_a^{+\infty}g(x)dx
		\]
		То есть сходимость зависит от второго слагаемого из правой части.
	\end{enumerate}
\end{proof}

%\begin{example}
%	\[
%		\int_0^{+\infty}\exp^{\cos x}\sin(\sin x)\frac{dx}{x}
%	\]
%\end{example}