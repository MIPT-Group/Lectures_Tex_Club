\begin{theorem} (Обобщённая теорема Больцано-Коши)
	Если $f \colon E \to \R$ непрерывна на связном множестве $E \subset \R^n$, то для любых своих значений $A < B$ она принимает и любое промежуточное, то есть
	\[
		\left(\exists \vv{x}_1, \vv{x}_2 \in E,\ f(\vv{x}_1) = A,\ f(\vv{x}_2) = B\right) \Ra \forall A < C < B\ \exists \vv{\xi} \in E,\ f(\vv{\xi}) = C
	\]
\end{theorem}

\begin{proof}
	Рассмотрим метрическое пространство $E$ с метрикой, индуцированной евклидовой метрикой из $\R^n$. Так как $f$ - это функция из одного метрического пространства в другое, то условие непрерывности равносильно утверждению:
	\[
		\forall G \subset \R \text{ - открытое } f^{-1}(G) \text{ - относительно открытое в } E
	\]
	Предположим противное, то есть
	\[
		\exists C,\ A < C < B \such \forall \vv{\xi} \in E\ \ f(\vv{\xi}) \neq C
	\]
	Теперь рассмотрим множества $(-\infty; C),\ (C; +\infty)$ - открытые в $\R$. Обозначим за $G_1 := f^{-1}(-\infty; C),\ G_2 := f^{-1}(C, +\infty)$ - относительно открытые в $E$.
	Тогда $E = G_1 \cup G_2$, что противоречит определению связности.
\end{proof}

\begin{theorem} (Линейная связность и связность)
	\begin{itemize}
		\item Каждое линейно связное подмножество $\R^n$ связно. 
		\item Если $G$ - открытое связное множество, то оно линейно связно.
	\end{itemize}
\end{theorem}

\begin{proof}~
	\begin{itemize}
		\item От противного. Пусть $E$ - линейно связное, но не связное. То есть
		\[
		E = E_1 \sqcup E_2
		\]
		где $E_1, E_2$ - непустые относительно открытые подмножества $E$. Более точно это означает, что
		\[
		E_1 = G_1 \cap E,\ E_2 = G_2 \cap E,\ \ G_1, G_2 \text{ - открытые в } \R^n
		\]
		Теперь рассмотрим произвольные $\vv{x}_1 \in E_1, \vv{x}_2 \in E_2$. Так как $E$ - линейно связное, то найдётся кривая $\gamma \subset E$, которая соединяет $\vv{x}_1$ и $\vv{x}_2$, то есть
		\[
		\gamma \colon [a; b] \to E,\ \gamma(a) = \vv{x}_1,\ \gamma(b) = \vv{x}_2
		\]
		Положим $T := \sup \{t \in [a; b] \such \gamma(t) \in E_1\},\ T \in [a; b]$. Рассмотрим все возможные случаи, чем может быть $T$:
		\begin{enumerate}
			\item Покажем, что $T \neq b$. Предположим противное:
			\[
			\gamma(b) = \gamma(T) = \vv{x}_2 \in G_2 \Ra \exists U(\vv{x}_2) \subset G_2 \Ra U(\vv{x}_2) \cap E \subset E_2
			\]
			Так как $\gamma$ - непрерывная, то 
			\[
			\exists \delta > 0 \such \forall t \in [T - \delta, T]\ \gamma(t) \in U(\vv{x}_2) \cap E \Ra \gamma(t) \in E_2
			\]
			Получили противоречие с определением $T$.
			
			\item Пусть $\gamma(T) \in E_1$. Это означает, что 
			\[
			\gamma(T) \in G_1 \Ra \exists U(\gamma(T)) \subset G_1
			\]
			В силу непрерывности опять имеем
			\[
			\exists \delta > 0 \such \forall t \in (T - \delta, T + \delta)\ \ \gamma(t) \in U(\gamma(T)) \Ra \gamma(t) \in E_1
			\]
			И снова противоречие с определением $T$. (есть точки правее, $T$ не супремум)
			
			\item $\gamma(T) \in E_2$. Это означает, что 
			\[
			\gamma(T) \in G_2 \Ra \exists U(\gamma(T)) \subset G_2
			\]
			В силу непрерывности
			\[
			\exists \delta > 0 \such \forall t \in (T - \delta; T + \delta)\ \ \gamma(t) \in U(\gamma(T)) \Ra \gamma(t) \in E_2
			\]
			Снова получили противоречие с принадлежностью $\gamma(T)$ к $E_1$
			
			\item Последний возможный случай - это $T = a$. Но и тут у нас возникнут проблемы:
			\[
				\gamma(T) = \gamma(a) = \vv{x}_1 \in G_1 \Ra \exists U(\vv{x}_1) \subset G_1 \Ra U(\vv{x}_1) \cap E \subset E_1
			\]
			Снова по непрерывности получаем, что
			\[
				\exists \delta > 0 \such \forall t \in [T; T + \delta]\ \gamma(t) \in U(\gamma(T)) \cap E \Ra \gamma(t) \in E_1
			\]
		\end{enumerate}
		Получили противоречие, потому $E$ связно.
		\item Пусть $G$ - открытое связное, но не линейно связное. Тогда $\exists \vv{x}_1, \vv{x}_2 \in G$ такие, что их нельзя соединить кривой в $G$. Положим за 
		\[
			G_1 := \{\vv{x} \in G, \text{ которые можно соединить с } \vv{x}_1\} 
		\]
		\[
			G_2 := G \bs G_1
		\]
		 Пусть $\vv{x} \in G_1 \subset G \Ra \exists U(\vv{x}) \subset G$. Ясно, что $\vv{x}_1 \in G_1,\ \vv{x}_2 \in G_2$. Заметим, мы можем соединить любую точку окрестности $U(\vv{x})$ с $\vv{x_1}$, просто взяв непрерывную кривую от $\vv{x}_1$ до $\vv{x}$ и добавив отрезок до точки от $\vv{x}$. Стало быть, $U(\vv{x}) \subset G_1$, то есть $G_1$ - открытое множество.
		 
		 Аналогично рассмотрим $G_2$, $\vv{x} \in G_2 \subset G$. Тогда $\exists U(\vv{x}) \subset G$. Если в этой окрестности найдётся хотя бы одна точка из $G_1$, то мы можем соединить её с $\vv{x}_1$ и затем с исходной точкой $\vv{x}$, получив непрерывную кривую. То есть $G_2$ - действительно открытое множество.
		 
		 \[
		  G = G_1 \sqcup G_2 \ \ G \ \text{ - не связное}
		 \]
	\end{itemize}
	Противоречие.
\end{proof}

\begin{example}
	Рассмотрим $E = \{(x, y) \such y = \sin \frac{1}{x},\ x \neq 0\} \cup \{(0, y) \such y \in [-1; 1]\}$ - данное множество нельзя разбить на два открытых. \textcolor{red}{Почему?}
\end{example}

\subsection{Дифференцируемость функций многих переменных}

\begin{definition}
	\textit{(Полным) приращением функции $f(\vv{x})$ в точке} $\vv{x}_0 \in \R^n$, отвечающим приращению $\vv{\Delta x} := \vv{x} - \vv{x_0}$, называется
	\[
		\Delta f(\vv{x}_0) = f(\vv{x}) - f(\vv{x}_0)
	\]
\end{definition}

\begin{definition}
	Пусть $f$ определена в некоторой окрестности точки $\vv{x}_0$. Тогда $f$ называется \textit{дифференцируемой} в точке $\vv{x}_0$, если её приращение $\Delta f(\vv{x}_0)$ может быть записано в виде
	\[
		\Delta f(\vv{x}_0) = \trbr{\vv{A}, \vv{\Delta x}} + o(|\vv{\Delta x}|),\ \vv{\Delta x} \to \vv{0}
	\]
	где $\vv{A} \in \R^n$ - \textit{градиент} $f$ в точке $\vv{x}_0$. Обозначается как
	\[
		\grad f(\vv{x}_0) := \vv{A}
	\]
	Выражение $\trbr{\vv{A}, \vv{\Delta x}}$ называется \textit{дифференциалом} функции $f$ в точке $\vv{x}_0$:
	\[
		df(\vv{x}_0) := \trbr{\vv{A}, \vv{\Delta x}}
	\]
\end{definition}

\begin{anote}
	Стоит напомнить, что $\Delta x = (dx_1, \ldots, dx_n)$, а дифференциал функции $f$ является функцией от $\vv{x}_0$ и $\vv{\Delta x}$. В частности, когда на письменном экзамене просят посчитать дифференциал в точке $\vv{x}_0$, то его нужно записывать через $dx_i$, а не $\vv{x}_i - \vv{x}_{0, i}$
\end{anote}

\begin{definition}
	\textit{Частным (частичным) приращением функции $f(\vv{x})$ в точке} $\vv{x}_0 \in \R^n$ называется приращение функции $f(\vv{x})$, отвечающее 	приращению $\vv{\Delta x}_j$, имеющуему вид
	\[
		\vv{\Delta x}_j = (0, \ldots, \Delta x_j, \ldots, 0)
	\]
	Частичное приращение обозначается как
	\[
		\Delta_j f(\vv{x}_0) = f(\vv{x}_0 + \vv{\Delta x_j}) - f(\vv{x}_0) = f(x_{1, 0}, \ldots, x_{j - 1, 0}, x_{j, 0} + \Delta x_j, x_{j + 1, 0}, \ldots, x_{n, 0}) - f(x_{1, 0}, \ldots, x_{n, 0})
	\]
\end{definition}

\begin{note}
	Заметим, что частичное приращение является приращением функции одной переменной:
	\[
		\Delta_j f(\vv{x}_0) = \Delta \phi_j (x_{j, 0})
	\]
	где $\phi_j(x_j) = f(x_{1, 0}, \ldots, x_{j - 1, 0}, x_j, x_{j + 1, 0}, \ldots, x_{n, 0})$
\end{note}

\begin{definition}
	\textit{Частной производной} функции $f(\vv{x})$ в точке $\vv{x}_0$ по $j$-й переменной называется производная функции $\phi_j$ в точке $x_{j, 0}$, если она существует. Обозначается как 
	\[
		f'_{x_j}(\vv{x}_0) = \pd{f}{x_j} (\vv{x}_0) := \phi'_j (x_{j, 0})
	\]
\end{definition}

\begin{definition}
	Также мы будем говорить о просто частной производной функции $f(\vv{x})$, которая является ничем иным как функцией многих переменных:
	\[
		\pd{f}{x} (\vv{x}) := \phi'_j(\vv{x}) = \phi'_j (x_{1, 0}, \ldots, x_{n, 0})
	\]
\end{definition}

\begin{note}
	То есть взяли производную от $\phi_j$, при этом не подставляли ни один аргумент как числовое значение. Полученная функция как функция от $\vv{x}$ является просто частной производной.
\end{note}

\begin{theorem}
	Если $f$ дифференцируема в точке $\vv{x}_0$, то существуют частные производные $\forall j \in \range{n}$, причём
	\[
		\grad f(\vv{x}_0) = \left(\frac{\vdelta f}{\vdelta x_1}(\vv{x}_0), \ldots, \frac{\vdelta f}{\vdelta x_n}(\vv{x}_0)\right)
	\]
\end{theorem}

\begin{proof}
	Пусть $\vv{\Delta x} := \vv{\Delta x_j}$, то есть $\Delta f(\vv{x}_0) := \Delta_j f(\vv{x}_0)$. Отсюда получаем, что
	\[
		\Delta_j f(\vv{x}_0) = A_j \Delta x_j + o(|\Delta x_j|),\ \Delta x_j \to 0
	\]
	При этом $\Delta_j f(\vv{x}_0) = \Delta \phi_j (x_{j, 0})$. То есть $\phi_j$ дифференцируема в точке $x_{j, 0}$. Значит
	\[
		\forall j \in \range{n}\ \ \phi'_j (x_{j, 0}) = \frac{\vdelta f}{\vdelta x_j} (\vv{x}_0) = A_j
	\]
\end{proof}

\begin{example}
	Рассмотрим функцию $f(x, y)$ следующего вида:
	\[
		f(x, y) = \System{
			&{\frac{xy}{x^2 + y^2},\ (x, y) \neq (0, 0)}
			\\
			&{0,\ (x, y) = (0, 0)}
		}
	\]
	У этой функции существуют частные производные в (0, 0):
	\[
		y = 0,\ \forall x \  \phi_1(x) = 0 
	\]
	\[
		x = 0,\ \forall y \ \phi_2(y) = 0
	\]
	Но она не дифференцируема в нуле, так как
	\[
		f(x,\ y) \neq o\left(\left|\sqrt{x^2 + y^2}\right|\right) \ \ \text{при} \  (x,\ y) \to \vv{0}
	\]
\end{example}

\begin{theorem}
	Если $f$ дифференцируема в $\vv{x}_0$, то она непрерывна в $\vv{x}_0$.
\end{theorem}

\begin{proof}
	Непрерывность в точке $\vv{x}_0$ равносильна наличию предела $\liml_{\vv{\Delta x} \to \vv{0}} \Delta f(\vv{x}_0) = 0$. При этом
	\[
		|\Delta f(\vv{x}_0)| \le |\grad f(\vv{x}_0)| \cdot |\vv{\Delta x}| + |o(|\vv{\Delta x}|)| \xrightarrow[\vv{\Delta x} \to \vv{0}]{} 0
	\]
\end{proof}

\begin{example}
	Обратное неверно. Примером служит $f(\vv{x}) = |\vv{x}|$ в точке 0.
\end{example}

\begin{theorem} (Достаточное условие дифференцируемости)
	Если $f$ определена в окрестности точки $\vv{x}_0$ вместе со своими частными производными, причём они непрерывны в $\vv{x}_0$, то $f$ дифференцируема в $\vv{x}_0$
\end{theorem}

\begin{proof}
	Распишем полное приращение $f$:
	\begin{multline*}
		\Delta f(\vv{x}_0) = f(x_{1, 0} + \Delta x_1, \ldots, x_{n, 0} + \Delta x_n) - f(x_{1, 0}, \ldots, x_{n, 0}) =
		\\
		\left(f(x_{1, 0} + \Delta x_1, \ldots, x_{n, 0} + \Delta x_n) - f(x_{1, 0}, x_{2, 0} + \Delta x_2, \ldots, x_{n, 0} + \Delta x_n)\right) +
		\\
		\left(f(x_{1, 0}, x_{2, 0} + \Delta x_2, \ldots, x_{n, 0} + \Delta x_n) - f(x_{1, 0}, x_{2, 0}, x_{3, 0} + \Delta x_3, \ldots, x_{n, 0} + \Delta x_n)\right) + \ldots +
		\\
		\left(f(x_{1, 0}, \ldots, x_{n - 1, 0}, x_{n, 0} + \Delta x_n) - f(x_{1, 0}, \ldots, x_{n, 0})\right)
	\end{multline*}
	Положим $\psi_j(x_j) := f(x_{1, 0}, \ldots, x_j, \ldots, x_{n, 0} + \Delta x_n)$. Так как это функция одной переменной, то, например, для $\psi_1$ по теореме Лагранжа можно записать следующее:
	\[
		\psi_1(x_{1, 0} + \Delta x_1) - \psi_1(x_{1, 0}) = \psi'_1(\xi_1)\Delta x_1 = \frac{\vdelta f}{\vdelta x_1}(\xi_1, x_{2, 0} + \Delta x_2, \ldots, x_{n, 0} + \Delta x_n)\Delta x_1
	\]
	Тогда приращение можно переписать как:
	\begin{multline*}
		\Delta f(\vv{x}_0) = \pd{f}{x_1}(\xi_1, x_{2, 0} + \Delta x_2, \ldots, x_{n, 0} + \Delta x_n)\Delta x_1 + \ldots + \pd{f}{x_n}(x_{1, 0}, x_{2, 0}, \ldots, \xi_n)\Delta x_n =
		\\
		\suml_{j = 1}^n \pd{f}{x_j} (\vv{x}_0) \Delta x_j + \left(\pd{f}{x_1} (\xi_1, x_{2, 0} + \Delta x_2, \ldots, x_{n, 0} + \Delta x_n) - \pd{f}{x_1}(\vv{x}_0)\right)\Delta x_1 + \ldots +
		\\
		\left(\pd{f}{x_n} (x_{1, 0}, x_{2, 0}, \ldots, \xi_n) - \pd{f}{x_n} (\vv{x}_0)\right)\Delta x_n
	\end{multline*}
	Так как $\forall j \in \range{n}\ \left|\frac{\Delta x_j}{|\Delta \vv{x}|}\right| \leq 1$, то справедливо утверждение
	\[
		\forall j \in \range{n}\ \ \liml_{\Delta\vv{x} \to \vv{0}}\frac{\left(\frac{\vdelta f}{\vdelta x_j}(x_{1, 0}, \ldots, \xi_j, \ldots, x_{n, 0} + \Delta x_n) - \frac{\vdelta f}{\vdelta x_j}(\vv{x}_0)\right) \Delta x_j}{|\Delta \vv{x}|} = 0
	\]
	Значит, все скобки после суммы являются $o\left(|\vv{\Delta x}|\right), |\vv{\Delta x}| \to 0$
\end{proof}