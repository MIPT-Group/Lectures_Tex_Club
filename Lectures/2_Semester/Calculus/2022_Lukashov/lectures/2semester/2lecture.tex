\begin{theorem} (Обобщённая теорема Больцано-Коши)
	Если $f \colon E \to \R$ непрерывна на связном множестве $E \subset \R^n$, то для любых своих значений $A < B$ она принимает и любое промежуточное, то есть
	\[
		\left(\exists \vec{x}_1, \vec{x}_2 \in E,\ f(\vec{x}_1) = A,\ f(\vec{x}_2) = B\right) \Ra \forall A < C < B\ \exists \vec{\xi} \in E,\ f(\vec{\xi}) = C
	\]
\end{theorem}

\begin{proof}
	Рассмотрим метрическое пространство $E$ с метрикой, индуцированной евклидовой метрикой из $\R^n$. Так как $f$ - это функция из одного метрического пространства в другое, то условие непрерывности равносильно утверждению:
	\[
		\forall G \subset \R\ \ f^{-1}(G) \ \ G\text{ - открытое в } E, относительно открытое в \R^n.
	\]
	Предположим противное, то есть
	\[
		\exists C,\ A < C < B \such \forall \vec{\xi} \in E\ \ f(\vec{\xi}) \neq C
	\]
	Теперь рассмотрим множества $(-\infty; C),\ (C; +\infty)$ - открытые в $R$. Обозначим за $G_1 := f^{-1}(-\infty; C),\ G_2 := f^{-1}(C, +\infty)$ - открытые в $E$, а значит относительно открытые в $\R^n$.
	Тогда $E = G_1 \cup G_2$, что противоречит определению связности.
\end{proof}

\begin{theorem} (Линейная связность и связность)
	\begin{itemize}
		\item Каждое линейно связное подмножество $\R^n$ связно. 
		\item Если $G$ - открытое связное множество, то оно линейно связно.
	\end{itemize}
\end{theorem}

\begin{proof}
	\begin{itemize}
		\item От противного. Пусть $E$ - линейно связное, но не связное. То есть
		\[
		E = E_1 \sqcup E_2
		\]
		где $E_1, E_2$ - непустые относительно открытые подмножества $E$. Более точно это означает, что
		\[
		E_1 = G_1 \cap E,\ E_2 = G_2 \cap E,\ \ G_1, G_2 \text{ - открытые в } \R^n
		\]
		Теперь рассмотрим произвольные $\vec{x}_1 \in E_1, \vec{x}_2 \in E_2$. Так как $E$ - линейно связное, то найдётся кривая $\gamma \subset E$, которая соединяет $\vec{x}_1$ и $\vec{x}_2$, то есть
		\[
		\gamma \colon [a; b] \to E,\ \gamma(a) = \vec{x}_1,\ \gamma(b) = \vec{x}_2
		\]
		Положим $T := \sup \{t \in [a; b] \such \gamma(t) \in E_1\},\ T \in [a; b]$. Рассмотрим все возможные случаи, чем может быть $T$:
		\begin{enumerate}
			\item Покажем, что $T \neq b$. Предположим противное:
			\[
			\gamma(b) = \gamma(T) = \vec{x}_2 \in G_2 \Ra \exists U(\vec{x}_2) \subset G_2 \Ra U(\vec{x}_2) \cap E \subset E_2
			\]
			Так как $\gamma$ - непрерывная, то 
			\[
			\exists \delta > 0 \such \forall t,\ [T - \delta, T]\ \gamma(t) \in U(\vec{x}_2) \Ra \gamma(t) \in E_2
			\]
			Получили противоречие с предположением о $T$.
			
			\item Пусть $\gamma(T) \in E_1$. Это означает, что 
			\[
			\gamma(T) \in G_1 \Ra \exists U(\gamma(T)) \subset G_1
			\]
			В силу непрерывности опять имеем
			\[
			\exists \delta > 0 \such \forall t \in (T - \delta, T + \delta)\ \ \gamma(t) \in U(\gamma(T)) \Ra \gamma(t) \in E_1
			\]
			И снова противоречие с предположением о $T$.
			\item $\gamma(T) \in E_2$ Это означает, что 
			\[
			\gamma(T) \in G_2 \Ra \exists U(\gamma(T)) \subset G_2
			\]
			В силу непрерывности
			\[
			\exists \delta > 0 \such \forall t \in (T - \delta, T + \delta)\ \ \gamma(t) \in U(\gamma(T)) \Ra \gamma(t) \in E_2
			\]
			Опять получаем, что $T$ не супремум.
			% Дописать + переписать первые 2 пункта без разбора \gamma(a), \gamma(b). Можно где-то скобочки подогнать
		\end{enumerate}
		Получили противоречие, потому $E$ связно.
		\item Пусть $G$ - открытое связное, но не линейно связное. Тогда $\exists \vec{x}_1, \vec{x}_2 \in G$ такие, что их нельзя соединить кривой в $G$. Положим за 
		\[
			G_1 := \{\vec{x} \in G, \text{ которые можно соединить с } \vec{x}_1\} 
		\]
		\[
			G_2 := G \bs G_1
		\]
		 Пусть $\vec{x} \in G_1 \subset G \Ra \exists U(\vec{x}) \subset G$.
		 
		 Ясно, что $\vec{x}_1 \in G_1,\ \vec{x}_2 \in G_2$. Заметим, что мы можем соединить любую точку окрестности $U(\vec{x})$ с $\vec{x_1}$, просто взяв непрерывную кривую от $\vec{x}_1$ до $\vec{x}$ и добавив отрезок до точки от $\vec{x}$. Стало быть, $U(\vec{x}) \subset G_1$, то есть $G_1$ - открытое множество.
		 
		 Аналогично рассмотрим $G_2$. Посмотрим на  $\vec{x} \in G_2 \subset G$. Тогда $\exists U(\vec{x}) \subset G$. Если в этой окрестности найдётся хотя бы одна точка из $G_1$, то мы можем соединить её с $\vec{x}_1$ и затем с исходной точкой $\vec{x}$, получив непрерывную кривую. То есть $G_2$ - действительно открытое множество.
		 
		 \[
		  G = G_1 \sqcup G_2 \ \ G \ \text{ - не связное}
		 \]
	\end{itemize}
	Противоречие.
\end{proof}

\begin{example}
	Рассмотрим $E = \{(x, y) \such y = \sin \frac{1}{x},\ x \neq 0\} \cup \{(0, y) \such y \in [-1; 1]\}$ - данное множество нельзя разбить на два открытых.
\end{example}

\subsection{Дифференцируемость функций многих переменных}

\begin{definition}
	\textit{(Полным) приращением функции $f(\vec{x})$ в точке} $\vec{x}_0 \in \R^n$, отвечающим приращению $\overrightarrow{\Delta x} := \vec{x} - \vec{x_0}$, называется
	\[
		\Delta f(\vec{x}_0) = f(\vec{x}) - f(\vec{x}_0)
	\]
\end{definition}

\begin{definition}
	Пусть $f$ определена в некоторой окрестности точки $\vec{x}_0$. Тогда $f$ называется \textit{дифференцируемой} в точке $\vec{x}_0$, если её приращение $\Delta f(\vec{x}_0)$ может быть записано в виде
	\[
		\Delta f(\vec{x}_0) = (\vec{A}, \overrightarrow{\Delta x}) + o(|\overrightarrow{\Delta x}|),\ \overrightarrow{\Delta x} \to \vec{0}
	\]
	где $\vec{A} \in \R^n$ - \textit{градиент} $f$ в точке $\vec{x}_0$. Обозначается как
	\[
		\grad f(\vec{x}_0) := \vec{A}
	\]
	Выражение $(\vec{A}, \overrightarrow{\Delta x})$ называется \textit{дифференциалом} функции $f$ в точке $\vec{x}_0$:
	\[
		df(\vec{x}_0) := (\vec{A}, \overrightarrow{\Delta x})
	\]
\end{definition}

\begin{definition}
	\textit{Частным (частитчным) приращением функции $f(\vec{x})$ в точке} $\vec{x}_0 \in \R^n$ называется приращение функции $f(\vec{x})$, отвечающее 	приращению $\overrightarrow{\Delta x}_j$, имеющуему вид
	\[
		\overrightarrow{\Delta x}_j = (0, \ldots, \Delta x_j, \ldots, 0)
	\]
	Частичное приращение обозначается как
	\[
		\Delta_j f(\vec{x}_0) = f(\vec{x}_0 + \overrightarrow{\Delta x_j}) - f(\vec{x}_0) = f(x_{1, 0}, \ldots, x_{j - 1, 0}, x_{j, 0} + \Delta x_j, x_{j + 1, 0}, \ldots, x_{n, 0}) - f(x_{1, 0}, \ldots, x_{n, 0})
	\]
\end{definition}

\begin{note}
	Заметим, что частичное приращение является приращением функции одной переменной:
	\[
		\Delta_j f(\vec{x}_0) = \Delta \phi_j (x_{j, 0})
	\]
	где $\phi_j(x_j) = f(x_{1, 0}, \ldots, x_{j - 1, 0}, x_j, x_{j + 1, 0}, \ldots, x_{n, 0}) - f(x_{1, 0}, \ldots, x_{n, 0})$
\end{note}

\begin{definition}
	\textit{Частной производной} функции $f(\vec{x})$ в точке $\vec{x}_0$ по $j$-й переменной называется производная функции $\phi_j$ в точке $x_{j, 0}$, если она существует. Обозначается как $\frac{\vdelta f}{\vdelta x_j}(\vec{x}_0),\ f'_{x_j}(\vec{x_0})$
\end{definition}

\begin{theorem}
	Если $f$ дифференцируема в точке $\vec{x}_0$, то существуют частные производные $\forall j \in \range{n}$, причём
	\[
		\grad f(\vec{x}_0) = \left(\frac{\vdelta f}{\vdelta x_1}(\vec{x}_0), \ldots, \frac{\vdelta f}{\vdelta x_n}(\vec{x}_0)\right)
	\]
\end{theorem}

\begin{proof}
	Пусть $\Delta f(\vec{x}_0) := \Delta_j f(\vec{x}_0)$, то есть $\overrightarrow{\Delta x} := \overrightarrow{\Delta x_j}$. Отсюда получаем, что
	\[
		\Delta_j f(\vec{x}_0) = A_j \Delta x_j + o(|\Delta x_j|),\ \Delta x_j \to 0
	\]
	При этом $\Delta_j f(\vec{x}_0) = \Delta \phi_j (x_{j, 0})$. То есть $\phi_j$ дифференцируема в точке $x_{j, 0}$. Значит
	\[
		\forall j \in \range{n}\ \ \phi'_j (x_{j, 0}) = \frac{\vdelta f}{\vdelta x_j} (x_{j, 0}) = A_j
	\]
\end{proof}

\begin{example}
	Рассмотрим функцию $f(x, y)$ следующего вида:
	\[
		f(x, y) = \System{
			&{\frac{xy}{x^2 + y^2},\ (x, y) \neq (0, 0)}
			\\
			&{0,\ (x, y) = (0, 0)}
		}
	\]
	У этой функции существуют частные производные в (0, 0).
	\[
		y = 0,\ \forall x \  \phi_1(x) = 0 
	\]
	\[
		x = 0,\ \forall y \ \phi_2(y) = 0
	\]
	Но так как она не дифференцируема в (0, 0) то
	\[
		f(x,\ y) \neq o(|(x,\ y)|) \ \ \text{при} \  (x,\ y) \to \vec{0}
	\]
\end{example}

\begin{theorem}
	Если $f$ дифференцируема в $\vec{x}_0$, то она непрерывна в $\vec{x}_0$.
\end{theorem}

\begin{proof}
	Непрерывность в точке $\vec{x}_0$ равносильно наличию предела $\liml_{\overrightarrow{\Delta x} \to \vec{0}} \Delta f(\vec{x}_0) = 0$. При этом
	\[
		|\Delta f(\vec{x}_0)| \le |\grad f(\vec{x}_0)| \cdot |\overrightarrow{\Delta x}| + |o(|\overrightarrow{\Delta x}|)| \xrightarrow[\overrightarrow{\Delta x} \to \vec{0}]{} 0
	\]
\end{proof}

\begin{example}
	Обратное неверно. Примером служит $f(\vec{x}) = |\vec{x}|$
\end{example}

\begin{theorem} (Достаточное условие дифференцируемости)
	Если $f$ определена в окрестности точки $\vec{x}_0$ вместе со своими частными производными, причём они непрерывны в $\vec{x}_0$, то $f$ дифференцируема в $\vec{x}_0$
\end{theorem}

\begin{proof}
	Распишем полное приращение $f$:
	\begin{multline*}
		\Delta f(\vec{x}_0) = f(x_{1, 0} + \Delta x_1, \ldots, x_{n, 0} + \Delta x_n) - f(x_{1, 0}, \ldots, x_{n, 0}) =
		\\
		\left(f(x_{1, 0} + \Delta x_1, \ldots, x_{n, 0} + \Delta x_n) - f(x_{1, 0}, x_{2, 0} + \Delta x_2, \ldots, x_{n, 0} + \Delta x_n)\right) +
		\\
		\left(f(x_{1, 0}, x_{2, 0} + \Delta x_2, \ldots, x_{n, 0} + \Delta x_n) - f(x_{1, 0}, x_{2, 0}, x_{3, 0} + \Delta x_3, \ldots, x_{n, 0} + \Delta x_n)\right) + 
		\\
		\ldots\\
		\left(f(x_{1, 0}, \ldots, x_{n - 1, 0}, x_{n, 0} + \Delta x_n) - f(x_{1, 0}, \ldots, x_{n, 0})\right)
	\end{multline*}
	Положим $\psi_j(x_j) := f(x_{1, 0}, \ldots, x_j, \ldots, x_{n, 0} + \Delta x_n),\  j = 1\ldots n$. Так как это функция одной переменной, то, например для $\psi_1$ по теореме Лагранжа можно записать следующее:
	\[
		\psi_1(x_{1, 0} + \Delta x_1) - \psi_1(x_{1, 0}) = \psi'_1(\xi_1)\Delta x_1 = \frac{\vdelta f}{\vdelta x_1}(\xi_1, x_{2, 0} + \Delta x_2, \ldots, x_{n, 0} + \Delta x_n)
	\]
	Тогда приращение можно переписать как:
	\begin{multline*}
		\Delta f(\vec{x}_0) = \frac{\vdelta f}{\vdelta x_1}(\xi_1, x_{2, 0} + \Delta x_2, \ldots, x_{n, 0} + \Delta x_n) + \ldots + \frac{\vdelta f}{\vdelta x_n}(x_{1, 0}, x_{2, 0}, \ldots, \xi_n) = 
		\grad f(\vec{x}_0)\cdot \Delta\vec{x}\\
		+ \frac{\vdelta f}{\vdelta x_1}(\xi_1, x_{2, 0} + \Delta x_2, \ldots, x_{n, 0} + \Delta x_n) \Delta x_1 + \ldots + \frac{\vdelta f}{\vdelta x_n}(x_{1, 0}, \ldots, \xi_n)\Delta x_n - \grad f(\vec{x}_0)\cdot \Delta\vec{x}
	\end{multline*}
	Так как $|\frac{\Delta x_j}{|\Delta \vec{x}|}| \leq 1$, то справедливо
	\[
		\forall j \ \liml_{\Delta\vec{x} \to \vec{0}}\frac{\frac{\vdelta f}{\vdelta x_j}(x_1, \ldots, \xi_j, \ldots, x_{n, 0} + \Delta x_n) - \frac{\vdelta f}{\vdelta x_j}(\vec{x}_0)}{|\Delta \vec{x}|} = 0
	\]
\end{proof}