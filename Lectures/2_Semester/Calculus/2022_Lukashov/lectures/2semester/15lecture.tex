\begin{example}
	Рассмотрим функциональную последовательность $f_n(x) = \frac{\sin (nx)}{\sqrt{n}}$. По уже ранее рассмотренному примеру, $f_n \rra 0$ на $\R$. При этом
	\[
		f'_n(x) = \sqrt{n} \cos (nx) \centernot{\xrightarrow[n \to \infty]{}} 0
	\]
\end{example}

\begin{theorem} (Пример Ван дер Вардена)
	Существует непрерывная нигде не дифференцируемая функция.
\end{theorem}

\begin{note}
	Изначально данную теорему доказал Вейерштрасс, но пример Ван дер Вардена просто красивее.
\end{note}

\begin{proof}
	Определим $\phi(x)$ следующим образом:
	\[
		\phi(x) := 1 - |x - 1| = \System{
			&{x, 0 \le x \le 1}
			\\
			&{2 - x, 1 \le x \le 2}
		}
	\]
	Остальные значения определяются через свойство:
	\[
		\forall x \in \R \quad \phi(x + 2) = \phi(x)
	\]
	Теперь определим $f(x)$:
	\[
		f(x) := \row{n = 0}{\left(\frac{3}{4}\right)^n \phi(4^n x)}
	\]
	Данная функция очевидным образом определена и непрерывна, так как
	\[
		\left|\left(\frac{3}{4}\right)^n \phi(4^n x)\right| \le \left(\frac{3}{4}\right)^n
	\]
	Попробуем представить себе эту функцию:
	\textcolor{red}{Сюда бы картинку, которую можно найти на 9й минуте 15й лекции за весну 2022.}
	То есть $\phi(x)$ была <<пилой>>, а потом мы её в каждой точке ещё раз исказили и получили $f(x)$.
	
	Верно следующее утверждение:
	\[
		\forall x \in \R, m \in \N\ \exists k \in \Z \such K \le 4^m x \le K + 1
	\]
	Положим за $\alpha_m := K \cdot 4^{-m}$ и $\beta_m := (K + 1) \cdot 4^{-m}$. Исследуем выражение $|\phi(4^n\beta_m) - \phi(4^n\alpha_m)|$:
	\begin{itemize}
		\item $n > m$. Тогда $\phi(4^n \alpha_m) = \phi(4^n \beta_m) = \phi(0) = 0$
		
		\item $n = m$. Тогда $\phi(4^n \alpha_m) = \phi(K)$ и $\phi(4^n \beta_m) = \phi(K + 1)$. По свойству $\phi$ это означает, что в таком случае $\phi(4^n \alpha_m)$ и $\phi(4^n \beta_m)$ может принимать только значения из $\{0, 1\}$, причём всегда разные. Значит
		\[
			|\phi(4^n \beta_m) - \phi(4^n \alpha_m)| = 1
		\]
		
		\item $n < m$. В таком случае, между $4^n \beta_m$ и $4^n \alpha_m$ нету целых точек. Значит, они находятся на одной наклонной функции $\phi$, и мы можем явно выписать, чему равен модуль разности значений:
		\[
			|\phi(4^n\beta_m) - \phi(4^n\alpha_m)| = 4^n(\beta_m - \alpha_m) = 4^{n - m}
		\]
	\end{itemize}

	Теперь, мы уже можем сказать кое-что про модуль $f(\beta_m) - f(\alpha_m)$:
	\begin{multline*}
		|f(\beta_m) - f(\alpha_m)| = \left|\row{n = 0}{\left(\frac{3}{4}\right)^n (\phi(4^n\beta_m) - \phi(4^n\alpha_m))}\right| = \left|\row{n = 0}{\left(\frac{3}{4}\right)^n(\phi(4^n\beta_m) - \phi(4^n\alpha_m))}\right| \ge
		\\
		\left(\frac{3}{4}\right)^m - \frac{1}{4^m} \cdot \frac{3^m - 1}{2} = \frac{1}{2}\left(\frac{3}{4}\right)^m + \frac{1}{2 \cdot 4^m} > \frac{1}{2}\left(\frac{3}{4}\right)^m
	\end{multline*}
	Тогда, для выражения производной верно следующее:
	\[
		\left|\frac{f(\beta_m) - f(\alpha_m)}{\beta_m - \alpha_m}\right| > \frac{\frac{1}{2}\left(\frac{3}{4}\right)^m}{4^{-m}} = \frac{1}{2} \cdot 3^m
	\]
	Это ключевое выражение данной функции. Напомним, что это было проделано $\forall x \in \R, m \in \N$. При этом, верны следующие условия:
	\begin{align*}
		&{0 \le \beta_m - \alpha_m = 4^{-m} \xrightarrow[m \to \infty]{} 0}
		\\
		&{\alpha_m \le x \le \beta_m}
	\end{align*}
	Значит, $\liml_{m \to \infty} \alpha_m = \liml_{m \to \infty} \beta_m = x$.
	
	Если бы существовала производная $f'(x)$, то её можно было бы записать в следующем виде:
	\[
		f'(x) = \liml_{m \to \infty} \frac{f(\beta_m) - f(x)}{\beta_m - x} = \liml_{m \to \infty} \frac{f(x) - f(\alpha_m)}{x - \alpha_m}
	\]
	При этом возможно и такое, что $\alpha_m = x$ или $\beta_m = x$, но мы такие случаи не рассматриваем. Далее, увидим следующее равенство:
	\[
		\frac{f(\beta_m) - f(\alpha_m)}{\beta_m - \alpha_m} = \frac{\beta_m - x}{\beta_m - \alpha_m} \cdot \left(\frac{f(\beta_m) - f(x)}{\beta_m - x} - f'(x)\right) + \frac{x - \alpha_m}{\beta_m - \alpha_m} \cdot \left(\frac{f(x) - f(\alpha_m)}{\alpha_m - x} - f'(x)\right) + f'(x)
	\]
	где имеют место следующие оценки:
	\begin{align*}
		&{\frac{\beta_m - x}{\beta_m - \alpha_m} \text{ и } \frac{x - \alpha_m}{\beta_m - \alpha_m} \text{ ограничены единицей}}
		\\
		&{\frac{f(\beta_m) - f(x)}{\beta_m - x} \xrightarrow[m \to \infty]{} f'(x)}
		\\
		&{\frac{f(x) - f(\alpha_m)}{\alpha_m - x} \xrightarrow[m \to \infty]{} f'(x)}
	\end{align*}
	Отсюда
	\[
		\liml_{m \to \infty} \frac{f(\beta_m) - f(\alpha_m)}{\beta_m - \alpha_m} = f'(x)
	\]
	А это противоречит оценке модуля выражения под пределом. Теорема доказана.
\end{proof}

\subsection{Степенные ряды}

\begin{note}
	До сих пор мы рассматривали ряды с обычными действительными числами. В степенных рядах нам будет удобнее работать с полем комплексных чисел, для которых всё определяется аналогично.
	
	Однако, теперь будет иметь место понятие \textit{безусловной сходимости}, и оно не одинаково с абсолютной сходимостью. Что мы потеряли из доказанного? Признак Лейбница, Дирихле и Абеля, ибо у нас нет понятия монотонности в комплекснозначных функциях.
\end{note}

\begin{definition}
	Ряд $\row{n = 0}{c_n(z - z_0)^n}$, где $c_n, z_0 \in \Cm$ называется \textit{степенным рядом с центром в точке $z_0$ и коэффициентами $\{c_n\}_{n = 0}^\infty$}.
\end{definition}

\begin{definition}
	\textit{Радиусом сходимости} степенного ряда $\row{n = 0}{c_n(z - z_0)^n}$ называется $R$:
	\[
		R = \frac{1}{\varlimsup\limits_{n \to \infty} \sqrt[n]{|c_n|}},\ 0 \le R \le +\infty
	\]
	При этом принято соглашение, что $R = +\infty$, если предел равен нулю и наоборот.
\end{definition}

\begin{definition}
	\textit{Кругом сходимости} ряда $\row{n = 0}{c_n(z - z_0)^n}$ называется множество $\{z \colon |z - z_0| < R\}$, где $R$ - радиус сходимости этого ряда.
\end{definition}

\begin{theorem} (Коши-Адамара)
	Если $R \in [0; +\infty]$ - радиус сходимости ряда $\row{n = 0}{c_n(z - z_0)^n}$, то
	\begin{enumerate}
		\item $\forall z, |z - z_0| < R$ ряд абсолютно сходится
		
		\item $\forall z, |z - z_0| > R$ ряд расходится
	\end{enumerate}
\end{theorem}

\begin{proof}
	\begin{enumerate}
		\item Будем считать, что $R > 0$, ибо иначе подходящих $z$ просто нет. По признаку Коши:
		\[
			\varlimsup\limits_{n \to \infty} \sqrt[n]{|c_n(z - z_0)^n|} < 1 \Longrightarrow \row{n = 0}{c_n(z - z_0)^n} \text{ - сходится}
		\]
		При этом $\varlimsup\limits_{n \to \infty} \sqrt[n]{|c_n(z - z_0)^n|} = |z - z_0| \cdot \varlimsup\limits_{n \to \infty} \sqrt[n]{|c_n|}$, что, при учёте соглашения, равносильно утверждению
		\[
			|z - z_0| < \frac{1}{\varlimsup\limits_{n \to \infty} \sqrt[n]{|c_n|}} = R
		\]
		
		\item Аналогично $R < +\infty$, иначе интересующих нас $z$ просто нет. В доказательстве признака Коши для расходимости мы бы показывали следующее утверждение:
		\[
			\varlimsup\limits_{n \to \infty} \sqrt[n]{|c_n(z - z_0)^n|} > 1 \Longrightarrow \liml_{n \to \infty} |c_n(z - z_0)^n| \neq 0
		\]
		Значит, ряд расходится. При этом, если $R = 0$, то верхний предел равен по соглашению $+\infty$ и утверждение остаётся верным.
	\end{enumerate}
\end{proof}

\begin{corollary} (Первая теорема Абеля)
	Если ряд $\row{n = 0}{c_n(z - z_0)^n}$ сходится в точке $z_1$ то он абсолютно сходится для $\forall z, |z - z_0| < |z_1 - z_0|$.
\end{corollary}

\begin{proof}
	Из теоремы Коши-Адамара следует, что $|z_1 - z_0| \le R$. Тогда $|z - z_0| < |z_1 - z_0| < R$ и по той же теореме получаем абсолютную сходимость.
\end{proof}

\begin{theorem} (Другая формула радиуса сходимости)
	Если $\exists \liml_{n \to \infty} |c_n / c_{n + 1}| > R$, то ряд $\row{n = 0}{c_n(z - z_0)^n}$ имеет радиус сходимости $R$.
\end{theorem}

\begin{proof}
	Воспользуемся признаком Даламбера, причём в предельной форме:
	\[
		\liml_{n \to \infty} \frac{|c_{n + 1}(z - z_0)^{n + 1}|}{|c_n(z - z_0)^n|} < 1 \Longrightarrow \row{n = 0}{c_n(z - z_0)^n} \text{ сходится}
	\]
	При этом, напишем эквивалентные условия для левой части, с учётом нашего соглашения:
	\[
		\liml_{n \to \infty} \frac{|c_{n + 1}(z - z_0)^{n + 1}|}{|c_n(z - z_0)^n|} = |z - z_0| \liml_{n \to \infty} \left|\frac{c_{n + 1}}{c_n}\right| < 1 \Longleftrightarrow |z - z_0| < \liml_{n \to \infty} \left|\frac{c_n}{c_{n + 1}}\right| = R
	\]
	Но этого мало. Нужно ещё показать, что для $|z - z_0| > R$ у нас будут проблемы:
	\[
		|z - z_0| > \liml_{n \to \infty} \left|\frac{c_n}{c_{n + 1}}\right| \Longrightarrow \liml_{n \to \infty} \left|\frac{c_{n + 1}(z - z_0)^{n + 1}}{c_n(z - z_0)^n}\right| > 1
	\]
	А отсюда по доказательству признака Даламбера следует, что
	\[
		\liml_{n \to \infty |c_n(z - z_0)^n|} \neq 0
	\]
	То есть ряд $\row{n = 0}{c_n(z - z_0)^n}$ действительно расходится.
\end{proof}

\begin{theorem} (Равномерная сходимость степенного ряда)
	Если ряд $\row{n = 0}{c_n(z - z_0)^n}$ имеет радиус сходимости $R > 0$, то он равномерно сходится в любом круге $|z - z_0| \le r$, где $0 < r < R$.
\end{theorem}

\begin{proof}
	Заметим, что если $|z_1 - z_0| = r$, то по теореме Коши-Адамара $\row{n = 0}{c_n(z_1 - z_0)^n}$ сходится абсолютно. Иначе говоря, ряд $\row{n = 0}{|c_n|r^n}$ сходится. Тогда
	\[
		\forall z, |z - z_0| \le r \quad |c_n(z - z_0)^n| \le |c_n|r^n \Longrightarrow \text{сходится абсолютно по признаку Вейерштрасса}
	\]
\end{proof}

\begin{corollary}
	Сумма степенного ряда - непрерывная функция в круге сходимости $|z - z_0| < R$.
\end{corollary}

\begin{theorem} (Вторая теорема Абеля)
	Если ряд $\row{n = 0}{c_n(z - z_0)^n}$ сходится в точке $z_1$, то его сумма - непрерывная функция на $[z_0; z_1]$.
\end{theorem}

\begin{proof}
	Данное утверждение нетривиально лишь для случая, когда $|z_1 - z_0| = R$. Рассмотрим какое-то $z \in [z_0; z_1]$. Тогда
	\[
		z \in [z_0; z_1] \lra z = z_0 + t(z_1 - z_0),\ t \in [0; 1]
	\]
	Для доказательства непрерывности, нам нужно установить, что ряд $\row{n = 0}{c_n t^n(z_1 - z_0)^n}$ равномерно сходится на $[0; 1]$. Имеет место 2 утверждения:
	\begin{enumerate}
		\item $\row{n = 0}{c_n(z_1 - z_0)^n}$ - равномерно сходится на $[0; 1]$ как функциональный ряд по $t$.
		
		\item $\{t^n\}_{n = 0}^\infty$ - монотонная, равномерно ограниченная единицей последовательность
	\end{enumerate}
	Уже было сказано, что мы не можем напрямую применить признак Абеля. Однако, давайте рассмотрим отдельно действительную и мнимую части нашего функционального ряда, которые являются действительными числами. Значит, можно по отдельности применить к ним признак Абеля, а потом <<склеить>> результат. Отсюда следует равномерная сходимость исходного ряда, что и требовалось показать.
\end{proof}

\begin{corollary}
	Если произведение по Коши сходящихся рядов $\row{n = 0}{a_n}$ и $\row{n = 0}{b_n}$ сходится, то оно сходится к произведению сумм этих рядов.
\end{corollary}

\begin{proof}
	Рассмотрим степенные ряды $\row{n = 0}{a_n z^n}$ и $\row{n = 0}{b_n z^n}$. По второй теореме Абеля, суммы этих рядов непрерывны на $[0; 1]$. Радиусы сходимости этих рядов $\ge 1$. Значит, они сходятся абсолютно при $|z| < 1$. Произведение по Коши этих степенных рядов имеет вид:
	\[
		\row{n = 0}{\left(\suml_{k = 0}^n a_n z^k b_{n - k} z^{n - k}\right)} = \suml_{n = 0}^\infty c_n z^n
	\]
	где $\row{n = 0}{c_n}$ - произведение по Коши рядов $\row{n = 0}{a_n}$ и $\row{n = 0}{b_n}$.
	
	Коль скоро ряды сходятся абсолютно, то произведение будет сходится к произведению при $|z| < 1$. При $|z| = 1$ произведение по Коши сходится, а значит, по второй теореме Абеля, оно непрерывно на $[0; 1]$. В итоге имеем, что
	\[
		\liml_{z \to 1-} \row{n = 0}{c_n z^n} = \liml_{z \to 1-} \left(\left(\row{n = 0}{a_n z^n}\right) \cdot \left(\row{n = 0}{b_n z^n}\right)\right) = \left(\row{n = 0}{a_n}\right) \cdot \left(\row{n = 0}{b_n}\right) = \row{n = 0}{c_n}
	\]
\end{proof}