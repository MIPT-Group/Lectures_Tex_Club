\begin{theorem} (Формула Тейлора с остаточным членом в форме Лагранжа для функции многих переменных)
	Пусть $f$ --- $m + 1$ раз дифференцируемая функция в окрестности точки $\vv{x}_0 \in \R^n$. Тогда, для любого $\vv{x}$ из этой окрестности существует $\vv{\xi} = \vv{x}_0 + c(\vv{x} - \vv{x}_0),\ 0 \le c \le 1$ такое, что
	\[
		f(\vv{x}) = f(\vv{x}_0) + \suml_{k = 1}^m \frac{d^k f(\vv{x}_0)}{k!} + \frac{d^{m + 1} f(\vv{\xi})}{(m + 1)!}
	\]
	где соблюдаются следующие равенства:
	\begin{itemize}
		\item \(m \ge 0\)
		\item \(\vv{x} = (x_1, \ldots, x_n)\)
		\item \(\vv{x}_0 = (x_{1, 0}, \ldots, x_{n, 0})\)
		\item \(\forall j \in \range{n}\ dx_j = x_j - x_{j, 0}\)
	\end{itemize}
\end{theorem}

\begin{proof}
	Положим $F(t) := f(\vv{x}_0 + t(\vv{x} - \vv{x}_0)),\  \forall t \in [0, 1]$. Докажем, что
	\[
		\forall k \in \range{m + 1} \quad F^{(k)} (t) = d^k f(\vv{x}_0 + t(\vv{x} - \vv{x}_0))
	\]
	В дифференциале при этом нужно положить $\forall j \in \range{n} \ dx_j = x_j - x_{j, 0}$.
	
	Сделаем это при помощи индукции:
	\begin{itemize}
		\item База $k = 1$:
		\[
			F'(t) = \suml_{j = 1}^n \pd{f}{x_j} (\vv{x}_0 + t(\vv{x} - \vv{x}_0)) (x_j - x_{j, 0}) = df (\vv{x}_0 + t(\vv{x} - \vv{x}_0))
		\]
		% Дописать последние 2 равенства
		\item Переход $k > 1$:
		\begin{multline*}
			F^{(k + 1)}(t) = \left(d^k f(\vv{x}_0 + t(\vv{x} - \vv{x}_0))\right)' =
			\\
			\left(\suml_{i_1, \ldots, i_k  = 1}^n \frac{\vdelta^k f}{\vdelta x_{i_1} \cdot \ldots \cdot \vdelta x_{i_k}} (\vv{x}_0 + t(\vv{x} - \vv{x}_0)) (x_{i_1} - x_{i_1, 0}) \cdot \ldots \cdot (x_{i_k} - x_{i_k, 0})\right)' =
			\\
			\suml_{i_1, \ldots, i_k = 1}^n \left(\suml_{j = 1}^n \pd{}{x_j} \left(\frac{\vdelta^k f}{\vdelta x_{i_1} \cdot \ldots \cdot \vdelta{x_{i_k}}}\right) (\vv{x}_0 + t(\vv{x} - \vv{x}_0))(x_j - x_{j, 0})\right) (x_{i_1} - x_{i_1, 0}) \cdot \ldots \cdot (x_{i_k} - x_{i_k, 0}) = 
			\\
			\suml_{i_1, \ldots, i_n}^n\frac{\vdelta^{k + 1} f}{\vdelta x_{i_1}\cdot\ldots\cdot \vdelta x_{i_n}}(\vv{x}_0 + t(\vv{x} - \vv{x}_0))(x_{i_1} - x_{i_1, 0})\cdot\ldots\cdot(x_{i_n} - x_{i_n, 0}) = d^{k + 1}f(\vv{x}_0 + t(\vv{x} - \vv{x}_0))
		\end{multline*}
	\end{itemize}
	Разложим $F(t)$ в ряд Тейлора с остаточным членом в форме Лагранжа в точке $t = 1$:
	\[
		F(1) = F(0) + \suml_{k = 1}^m\frac{F^{(k)}(0)}{k!} + \frac{F^{(m + 1)}(c)}{(m + 1)!}
	\]
	Это в свою очередь равняется:
	\[
		f(\vv{x}) = f(\vv{x}_0) + \suml_{k = 1}^m\frac{d^{k}f(\vv{x}_0)}{k!} + \frac{d^{m + 1}f(\vv{\xi)}}{(m + 1)!}
	\]
\end{proof}

\begin{theorem} (Формула Тейлора с остаточным членом в форме Пеано для функции многих переменных)
	Если $f$ дифференцируема $m$ раз в точке $\vv{x}_0 \in \R^n$ и $m - 1$ раз в окрестности точки $\vv{x}_0$, то
	\[
		f(\vv{x}) = f(\vv{x}_0) + \suml_{k = 1}^m \frac{d^k f(\vv{x}_0)}{k!} + o\left(|\vv{x} - \vv{x}_0|^m\right),\ \vv{x} \to \vv{x}_0
	\]
	где соглашения про $dx_j$, $\vv{x}$, $\vv{x}_0$ те же, что и в предыдущий раз, лишь теперь $m \geq 1$
\end{theorem}

\begin{proof}
	Определим функцию ошибки формулы от истинного значения:
	\[
		g_m(\vv{x}) := f(\vv{x}) - f(\vv{x}_0) - \suml_{k = 1}^m \frac{d^k f(\vv{x}_0)}{k!}
	\]
	Доказательство по сути сводится к применению двух лемм:
	\begin{lemma}
		В условиях теоремы $g_m$ и все её частные производные до порядка $m$ включительно обращаются в нуль в точке $\vv{x}_0$.
	\end{lemma}

	\begin{proof}
		По индукции
		\begin{itemize}
			\item База $m = 1$:
			\[
				g_1(\vv{x}) = f(\vv{x}) - f(\vv{x}_0) - d f(\vv{x}_0) = f(\vv{x}) - f(\vv{x}_0) - \suml_{k = 1}^n \pd{f}{x_k} (\vv{x}_0) (x_k - x_{k, 0})
			\]
			Из записи явно видна справедливость леммы.
			
			\item Переход $m = t \Ra m = t + 1$:
			
			Запишем вид $g_{t + 1}$:
			\[
				g_{t + 1}(\vv{x}) = f(\vv{x}) - f(\vv{x}_0) - \suml_{k = 1}^{t + 1} \frac{d^k f(\vv{x}_0)}{k!}
			\]
			Сразу видно, что $g_{t + 1} (\vv{x}_0) = 0$. Теперь посмотрим на частную производную, например, по $x_1$:
			\begin{multline*}
				\pd{g_{t + 1}}{x_1} (\vv{x}) = \pd{f}{x_1} (\vv{x}) - \suml_{k = 1}^{t + 1} \frac{1}{k!} \suml_{i_1 + \ldots + i_n = k \over i_1 \ge 1} \frac{\vdelta^k f}{\vdelta x_1^{i_1} \cdot \ldots \cdot \vdelta x_n^{i_n}} (\vv{x}_0) \cdot
				\\
				\frac{k!}{i_1! \cdot \ldots \cdot i_n!} i_1 (x_1 - x_{1, 0})^{i_1 - 1} \cdot \ldots \cdot (x_n - x_{n, 0})^{i_n}
			\end{multline*}
			Переобозначим $\hat{i}_1 = i_1 - 1,\ \hat{i}_2 = i_2,\ldots, \hat{i}_n = i_n$ и разложим сумму на случаи, когда $k = 1$ и $k \ge 2$:
			\begin{multline*}
				\pd{f}{x_1} (\vv{x}) - \pd{f}{x_1} (\vv{x}_0) - \suml_{k = 2}^{t + 1} \frac{1}{(k - 1)!} \suml_{\hat{i}_1 + \ldots + \hat{i}_n = k - 1} \frac{(k - 1)!}{\hat{i}_1! \cdot \ldots \cdot \hat{i}_n!} \cdot
				\\
				\frac{\vdelta^{k - 1}}{\vdelta x_1^{\hat{i}_1} \cdot \ldots \cdot \vdelta x_n^{\hat{i}_n}} \left(\pd{f}{x_1}\right) (\vv{x}_0) (x_1 - x_{1, 0})^{\hat{i}_1} \cdot \ldots \cdot (x_n - x_{n, 0})^{\hat{i}_n} =
				\\
				\pd{f}{x_1} (\vv{x}) - \pd{f}{x_1} (\vv{x}_0) - \suml_{k = 1}^t \frac{1}{k!} d^k \left(\pd{f}{x_1}\right) (\vv{x}_0)
			\end{multline*}
		\end{itemize}
	\end{proof}
	\begin{lemma}
		Пусть функция $g$ удовлетворяет следующим условиям:
		\begin{enumerate}
			\item $g$ $m$ раз дифференцируема в точке $\vv{x}_0 \in \R^n$
			
			\item $g$ $m - 1$ раз дифференцируема в окрестности точки $\vv{x}_0$
			
			\item $g$ обращается в нуль вместе со всеми своими частными производными до $m - 1$-го порядка включительно в точке $\vv{x}_0$
		\end{enumerate}
		Тогда верно, что
		\[
			g(\vv{x}) = o\left(|\vv{x} - \vv{x}_0|^m\right),\ \vv{x} \to \vv{x}_0
		\]
	\end{lemma}
	
	\begin{proof}
		По индукции
		\begin{itemize}
			\item База $m = 1$: очевидно, ибо условие теоремы обращается в простую дифференцируемость в точке $\vv{x}_0$.
			
			\item Переход $m = t \Ra m = t + 1$: из условия следует, что $g$ хотя бы $t$ раз дифференцируема в окрестности точки $\vv{x}_0$. Запишем $g$ по формуле Тейлора с остаточным членом в форме Лагранжа с одним дифференциалом:
			\[
				g(\vv{x}) = g(\vv{x}_0) + \frac{dg(\vv{\xi})}{1!},\ \vv{\xi} = \vv{x}_0 + c(\vv{x} - \vv{x}_0),\ 0 \le c \le 1
			\]
			Распишем $dg(\vv{\xi})$:
			\[
				dg(\vv{\xi}) = \suml_{j = 1}^n \pd{g}{x_j} (\vv{\xi}) (x_j - x_{j, 0})
			\]
			$\pd{g}{x_j}$ удовлетворяет лемме для $m = t$, тогда:
			\[
				\pd{g}{x_j}(\vv{\xi}) = o(|\vv{\xi} - \vv{x}_0|^t) = o(|\vv{x} - \vv{x}_0|^t),\ \vv{x} \to \vv{x}_0
			\]
			Теперь мы можем переписать $dg(\vv{\xi})$ как:
			\[
				dg(\vv{\xi}) = \suml_{j = 1}^n \pd{g}{x_j}(\vv{\xi})(x_j - x_{j, 0}) = \suml_{j = 1}^n o(|\vv{x} - \vv{x}_0|^t)(x_j - x_{j, 0}) = \suml_{j = 1}^n o(|\vv{x} - \vv{x}_0|^{t + 1})\frac{(x_j - x_{j, 0})}{|\vv{x} - \vv{x}_0|}
			\]
			Очевидно $\forall j\ \frac{(x_j - x_{j, 0})}{|\vv{x} - \vv{x}_0|} \leq 1$. Поэтому окончательно $dg(\vv{x}) = o(|\vv{x} - \vv{x}_0|^{t + 1}),\ \vv{x} \to \vv{x}_0$
		\end{itemize}
	\end{proof}
\end{proof}

\begin{note}
	Если $f$ дифференцируема $m$ раз в точке $\vv{x}_0$ и $f$ имеет вид:
	\[
		f(x) = \suml_{k = 0}^m \suml_{i_1 + \ldots + i_n = k} a_{i_1, \ldots, i_n} (x_1 - x_{1, 0})^{i_1} \cdot \ldots \cdot (x_n - x_{n, 0})^{i_n} + o\left(|\vv{x} - \vv{x}_0|^k\right),\ \vv{x} \to \vv{x}_0
	\]
	то это представление совпадает с формулой Тейлора с остаточным членом в форме Пеано.
\end{note}

\section{Интегральное исчисление функции одной переменной}

\subsection{Определение интеграла Римана}

%% Здесь нужен график функции. Есть на записи лекции, либо пнуть меня

\begin{definition}
	\textit{Разбиением отрезка} $[a; b]$ называется конечное множество точек отрезка $[a; b]$. Обозначается как
	\[
		P: a = x_0 < x_1 < \ldots < x_n = b
	\]
	За длину $k$-го подотрезка положим величину $\Delta x_k$:
	\[
		\forall k \in \range{n}\ \Delta x_k := x_k - x_{k - 1}
	\]
\end{definition}

\begin{definition}
	\textit{Диаметром разбиения} называется величина $\Delta P = \max\limits_{1 \le k \le n} \Delta x_k$
\end{definition}

\begin{note}
	До конца параграфа зафиксируем, что $f$ - это ограниченная функция на $[a; b]$ и определим супремум и инфинум функции на подотрезках:
	\[
		M_k := \sup\limits_{x \in [x_{k - 1}; x_k]} f(x);\ \ m_k := \inf\limits_{x \in [x_{k - 1}; x_k]} f(x)
	\]
\end{note}

% Дописать обозначение минимума и максимума конкретной функции как M_k(f), m_k(f)

\begin{definition}
	\textit{Верхней суммой Дарбу} называется величина
	\[
		U(P, f) = \suml_{k = 1}^n M_k \Delta x_k
	\]
	а \textit{нижней суммой Дарбу} будет соответственно
	\[
		L(P, f) = \suml_{k = 1}^n m_k \Delta x_k
	\]
\end{definition}

\begin{definition}
	\textit{Нижним интегралом} $\underline{I}(f)$ называется величина
	\[
		\underline{I}(f) := \sup\limits_{P} L(P, f)
	\]
	а \textit{верхним интегралом} соответственно будет
	\[
		\overline{I}(f) := \inf\limits_{P} U(P, f)
	\]
	Если $\underline{I}(f) = \overline{I}(f)$, то $f$ называется интегрируемой по Риману на $[a; b]$, а общее значение $I(f)$ называется \textit{интегралом Римана} от $f$ по отрезку $[a; b]$: $\int_a^b f(x)dx$.
\end{definition}

\begin{definition}
	Разбиение $P^*$ называется \textit{измельчением} разбиения $P$, если $[a; b] \supset P^* \supset P$.
\end{definition}

\begin{lemma}
	Для любых разбиений $P_1, P_2$ отрезка $[a; b]$ верно, что
	\[
		L(P_1, f) \le U(P_2, f)
	\]
\end{lemma}

\begin{proof}
	Если $P_1 = P_2$, то это очевидно, в противном случае докажем, что для любого измельчения $P^*$ разбиения $P$ верно 2 неравенства:
	\begin{align*}
		&{L(P, f) \le L(P^*, f)}
		\\
		&{U(P, f) \ge U(P^*, f)}
	\end{align*}
	Рассмотрим измельчение $P^* = P \cup \{x^*\},\ x^* \in (x_{k - 1}, x_k)$.
	Все слагаемые в $L(P^*, f)$ совпадают с $L(P, f)$, кроме $m_k \Delta x_k$, которое стало $\inf\limits_{x\in [x_{k - 1}, x^*]} f(x)(x^* - x_{k - 1}) + \inf\limits_{x\in[x^*, x_k]}f(x)(x_k - x^*)$. Так как инфимум на отрезке не может быть больше, чем на подотрезках, то это слагаемое возрасло, а значит $L(P^*, f) \geq L(P, f)$. Аналогичное рассуждение справедливо и для $U(P^*, f) \leq U(P, f)$. Для доказательства теоремы достаточно рассмотреть общее измельчение:
	\[
		P^* = P_1 \cup P_2 \Ra L(P_1, f) \leq L(P^*, f) \leq U(P^*, f) \leq U(P_2, f)
	\]
\end{proof}