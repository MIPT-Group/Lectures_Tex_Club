\begin{theorem} (Структура открытых множеств в $\R^n$)
	Каждое открытое множество в $\R^n$ представимо не более чем счётным объединением непересекающихся брусьев
\end{theorem}

\textcolor{red}{Сюда надо картиночку с 20й лекции весны 2022, 8:57}

\begin{proof}
	Для начала заметим, что мы можем записать $\R^n$ следующим образом:
	\[
		\R^n = \bigcup_{\vv{l} \in \Z^n} \prod_{k = 1}^n [l_k; l_k + 1)
	\]
	где понятно, что $\vv{l} = (l_1, \ldots, l_n)$. Обозначим каждый брусок в этом представлении через $P_{\vv{l}, 0}$. Теперь, посмотрим на другое разложение:
	\[
		\R^n = \bigcup_{\vv{l} \in (\Z / 2^m)^n} \prod_{k = 1}^n \left[\frac{l_k}{2^m}; \frac{l_k + 1}{2^m}\right),\ \ m \in \N
	\]
	Здесь уже каждый брус обозначается как $P_{\vv{l}, m}$. Теперь, рассмотрим произвольное открытое множество $G$. Введём следующее множество:
	\[
		M_0 = \bigcup_{\vv{l} \colon P_{\vv{l}, 0} \subset G} P_{\vv{l}, 0}
	\]
	Аналогично будем строить $M_m$, но с небольшим нюансом:
	\[
		M_m = \left(\bigcup_{\vv{l} \colon P_{\vv{l}, m} \subset G} P_{\vv{l}, m}\right) \bs \bigcup_{j = 0}^{m - 1} M_j
	\]
	По построению получили, что все $M_m$ не пересекаются. Остаётся показать следующий факт:
	\[
		G = \bigcup_{m = 0}^\infty M_m
	\]
	По построению сразу верно включение $\supset$. Чтобы доказать в другую сторону, нам поможет свойство открытости $G$. 
	
	Рассмотрим $\forall x \in G$. Тогда \(\exists r > 0 \such U_r(x) \subset G\) и более того, довольно очевидно, что выполнится следующее условие (как минимум из тех соображений, что мы можем делать сетку сколь угодно малой и найти точку рядом с $x$):
	\[
		\exists m_0\ \exists \vv{l} \such x \in \prod_{k = 1}^n \left[\frac{l_k}{2^{m_0}}; \frac{l_k + 1}{2^{m_0}}\right) \subset G
	\]
	Этого достаточно, чтобы $x$ оказался внутри объединения $M_m$.
\end{proof}

\begin{anote}
	Лично для меня открыт вопрос: а почему бесконечное объединение чисто брусьев окажется счётным? Разве тут не $\N^\N \cong \R$? Если мы выкинем из определения $M_m$ разность, то полученное множество будет счётным, ибо брусья в $\Z^n$ точно счётны (а это всего лишь какое-то подмножество из этих брусьев). Более того, на $m$-й итерации выясняется, что $\bigcup_{j = 1}^{m - 1} M_j$ является подмножеством возникшего подмножества, то есть счётным. Отсюда вроде всё и следует, но это не точно.
\end{anote}

\begin{corollary}
	Все открытые и замкнутые множества в $\R^n$ измеримы по Лебегу.
\end{corollary}

\begin{proof}
	Замкнутое множество - это дополнение открытого до $\R^n$, которое в свою очередь измеримо из-за теоремы об измеримости объединения счётного числа измеримых множеств (брусья ведь)
\end{proof}

\begin{corollary}~
	\begin{itemize}
		\item Для любого ограниченного замкнутого множества $F$ верно, что
		\[
			\upjm(F) = \mu(F)
		\]
		
		\item Для любого ограниченного открытого множества $G$ верно, что
		\[
			\downjm(G) = \mu(G)
		\]
	\end{itemize}
\end{corollary}

\begin{proof}~
	\begin{itemize}
		\item Уже знаем, что $\mu(F) \le \upjm(F)$. Так как $F$ ограничено и замкнуто, то оно компактно. Отсюда следует следующий факт для любого счётного покрытия:
		\[
			F \subset \bigcup_{i = 1}^\infty P_i \Lora \exists N \in \N \such F \subset \bigcup_{i = 1}^N P_i
		\]
		Значит, инфинум для верхней меры Лебега подойдёт и для верхней меры Жордана, то есть $\upjm(F) \le \mu(F)$.
		
		\item \textcolor{red}{Что такое нижняя мера для произвольного множества?}
	\end{itemize}
\end{proof}

\begin{theorem} (Структура множеств в $\R^n$)
	 Пусть $A$ - измеримое по Лебегу множество, причём $\mu(A) < +\infty$. Тогда
	 \[
	 	A = \bigcap_{i = 1}^\infty \bigcup_{j = 1}^\infty A_{i, j} \bs A_0
	 \]
	 где $A_{i, j}$ - элементарное множество, при этом
	 \[
	 	\forall i \in \N, j \in \N \quad A_{i, j} \subset A_{i, j + 1}
	 \]
	 Обозначив $B_i := \bigcup_{j = 1}^\infty A_{i, j}$, должно выполняться следующее:
	 \[
	 	\forall i \in \N \quad B_i \supset B_{i + 1}
	 \]
	 Ну и напоследок $\mu(B_1) < +\infty$, $\mu(A_0) = 0$. $A_0$ - какое-то измеримое по Лебегу множество.
\end{theorem}

\begin{proof}
	Раз $A$ измеримо по Лебегу, то мы можем покрытием элементарными множествами сколь угодно близко подойти к этой величине:
	\[
		\forall i \in \N\ \exists C_i = \bigcup_{j = 1}^\infty D_{i, j} \such C_i \supset A,\ \mu(C_i \bs A) < \frac{1}{i}
	\]
	Положим $B_i := \bigcap_{r = 1}^i C_r$. Так как мы пересекаем конечное число $C_i$, то $B_i$ можно записать через объединение каких-то элементарных множеств:
	\[
		B_i := \bigcup_{j = 1}^\infty E_{i, j}
	\]
	Теперь по определению делаем $A_{i, j} := \bigcup_{s = 1}^j E_{i, s}$. Тогда требования на включения точно выполнены. Более того
	\[
		B_1 = C_1 \Lora \mu(B_1) = \mu(C_1) < \mu(A) + 1
	\]
	Теперь посмотрим на меру пересечения всех $B_i$:
	\[
		\mu\left(\bigcap_{i = 1}^\infty B_i\right) = \liml_{i \to \infty} \mu(B_i) \le \varlimsup_{i \to \infty} \mu(C_i)
	\]
	Неравенство следует очевидно из того, что $B_i \subset C_i \Ra \mu(B_i) \le \mu(C_i)$. При этом заметим ещё одну вещь:
	\[
		\mu(A) \le \mu(C_i) \le \mu(A) + \frac{1}{i} \Lora \varlimsup_{i \to \infty} \mu(C_i) = \mu(A)
	\]
	Отсюда следует последний нужный факт:
	\[
		\mu\left(\left(\bigcap_{i = 1}^\infty B_i\right) \bs A \right) = 0
	\]
	То есть множества $A \subset \bigcap_{i = 1}^\infty B_i$ отличаются лишь на какое-то $A_0$ с нулевой мерой.
\end{proof}

\begin{corollary}
	Пусть $A$ - измеримое по Лебегу множество. Тогда
	\[
		A = \left(\bigcap_{i = 1}^\infty G_i\right) \bs P
	\]
	где $G_1 \supset G_2 \supset \ldots$ - открытые множества, $\mu(P) = 0$.
	
	Или же есть такое разложение:
	\[
		A = \left(\bigcup_{i = 1}^\infty F_i\right) \cup Q
	\]
	где $F_1 \subset F_2 \subset \ldots$ - замкнутые множества, $\mu(Q) = 0$
\end{corollary}

\begin{proof}
	Идея состоит в том, чтобы немного <<пошаманить>> над $B_i$ и получить из них открытые множества. В частности, мы знаем, что $B_i$ являются объединением элементарных множеств, то есть представимы как объединение брусьев:
	\[
		B_i = \bscup_{j = 1}^\infty P_{i, j}
	\]
	Рассмотрим объединение $\eps'$-растяжений этих брусьев, выкидывая при этом границы (то есть делая брусья открытыми множествами). Полученное таким образом $B'_i$ множество окажется открытым. При этом, если положить $\eps' = \eps / 2^{j + i}$, то возникнет следующее:
	\[
		\mu(B'_i) = \bscup_{j = 1}^\infty |P^{\eps'}_{i, j}| = \mu(B_i) + \frac{\eps}{2^i}
	\]
	Тогда, посмотрим на меру пересечения всех $B'_i$ и убедимся, что мы можем положить $G_i := B'_i$:
	\[
		\mu\left(\bigcap_{i = 1}^\infty B'_i\right) = \liml_{i \to \infty} \mu(B'_i) = \liml_{i \to \infty} \mu(B_i) = \mu(A)
	\]
	
	Чтобы доказать второе разложение, вспомним, что $A$ - измеримо тогда и только тогда, когда $A' = \R^n \bs A$ - измеримо. Если расписать $A$ как $\R^n \bs A'$ и вместо $A'$ подставить уже доказанное разложение, то получим требуемое.
\end{proof}

\begin{note}
	Если $A \subset \R^n$ - ограниченное множество, то есть эквивалентные определения для мер, как следствие последних теорем:
	\begin{align*}
		&{\upjm(A) = \inf_{A \subset M} |M|,\ M \text{ - элементарное}}
		\\
		&{\downjm(A) = \sup_{M \subset A} |M|,\ M \text{ - элементарное}}
		\\
		&{\mu^*(A) = \inf_{A \subset G} \mu(G),\ G \text{ - открытое}}
		\\
		&{\mu_*(A) = \sup_{F \subset A} \mu(F),\ F \text{ - замкнутое}}
	\end{align*}
\end{note}

\begin{definition}
	\textit{Борелевской $\sigma$-алгеброй} называется наименьшая $\sigma$-алгебра в $\R^n$, содержащая все открытые и замкнутые множества.
\end{definition}

\begin{theorem} (Критерий измеримости по Жордану)
	Ограниченное множество $A \subset \R^n$ измеримо по Жордану тогда и только тогда, когда выполнено равенство:
	\[
		\upjm(\vdelta A) = 0
	\]
\end{theorem}

\begin{proof}
	\begin{itemize}
		\item Необходимость. Пусть $A$ - измеримое по Жордану множество. В силу этого факта, можно заявить следующее (из свойств супремума и инфинума, пользуемся замечанием):
		\[
			\forall \eps > 0\ \exists E_1, E_2 \such E_2 \subset A \subset E_1,\ |E_1 \bs E_2| < \eps
		\]
		Так как $E_i$ - элементарные множества, то
		\[
			|E_1| = |\cl E_1|,\ |E_2| = |\Int E_2|
		\]
		В силу вложения внутренности/замыкания, имеют места следующие утверждения:
		\begin{align*}
			&{\cl A \subset \cl E_1 \Lora \upjm(\cl A) \le |E_1|}
			\\
			&{\Int E_2 \subset \Int A \Lora \downjm(\Int E_2) = |E_2| \le \downjm(\Int A)}
		\end{align*}
		Отсюда $\upjm(\cl A) - \downjm(\Int A) < \eps$, то есть
		\[
			\forall \eps > 0\ \upjm(\vdelta A) < \eps
		\]
		
		\item Достаточность. Заметим следующий факт:
		\[
			\upjm(A) \le \upjm(\cl A) = \mu(\cl A) = \mu(\Int A) + \mu(\vdelta A) = \mu(\Int A) = \downjm(A) \le \upjm(A)
		\]
		где равенство $\mu(\Int A) = \upjm(A)$ следует из свойства для ограниченного открытого множества.
	\end{itemize}
\end{proof}

\begin{example} (Канторово множество)
	Рассмотрим отрезок $I_0 = [0; 1]$. Поделим его на 3 равные части и удалим среднюю. Рекурсивно повторим операцию для каждого из оставшихся отрезков. Утверждается, что получившееся множество, называемое \textit{канторовым} $C$, измеримо по Лебегу, неизмеримо по Жордану, и при этом замкнуто и несчётно.
	
	Если обозначить за $\mathfrak{J}_1 := (1/3; 2/3)$, то $I_1 := I_0 \bs \mathfrak{J}_1$. Аналогичным образом $I_2$ будет уже 4 отрезка, $I_3$ - 8 отрезков и так далее. Канторово множество можно представить в следующем виде:
	\[
		C := \bigcap_{i = 1}^\infty I_i
	\]
	а его меру можно записать в виде предела:
	\[
		\mu(C) = \liml_{i \to \infty} |I_i|
	\]
	где $|I_i| = 1 - 1/3 - 2/9 - 4/27 - \ldots - 2^{i - 1} / 3^i$. Можно переписать в следующем виде:
	\[
		|I_i| = 1 - \frac{1}{3} \cdot \frac{1 - \left(\frac{2}{3}\right)^i}{1 - \frac{2}{3}}
	\]
	Несложно понять, что $\mu(C)  = 0$ в таком случае.
	
	Теперь докажем континуальность. Для этого заметим, что мы можем описать канторово множество при помощи правил в троичной системе счисления. Для начала, покажем нумерацию для получения $I_2$:
	\begin{align*}
		&{I_2 := I_1 \bs \mathfrak{J}_2}
		\\
		&{\mathfrak{J}_2 := \mathfrak{J}_{2, 1} \sqcup \mathfrak{J}_{2, 2}}
		\\
		&{\mathfrak{J}_{2, 1} = (1/9; 2/9)}
		\\
		&{\mathfrak{J}_{2, 2} = (7/9; 8/9)}
	\end{align*}
	Отлично. Если записывать точки $\mathfrak{J}_1$ в троичной системе, то как его можно описать?
	\[
		\mathfrak{J}_1 = \{0.1\ldots_{(3)}\}
	\]
	Аналогично получаем описание для $\mathfrak{J}_2$:
	\[
		\mathfrak{J}_2 = \{0.01\ldots_{(3)}\} \sqcup \{0.21\ldots_{(3)}\}
	\]
	То есть канторово множество - это все числа в отрезке $[0; 1]$, которые в троичной записи записываются после запятой только при помощи 0 или 2. Таких чисел $2^\N \cong \R$
\end{example}