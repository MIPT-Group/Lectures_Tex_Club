\begin{theorem} (Конечная аддитивность мер Лебега и Жордана)
	Если $A_1, \ldots, A_N$ - измеримые по Лебегу (Жордану) непересекающиеся подмножества $K_I$, то 
	\[
		\jlm \left(\bscup_{i = 1}^N A_i\right) = \sum_{i = 1}^N \jlm(A_i)
	\]
\end{theorem}

\begin{proof}
	Чтобы показать верность равенства, мы воспользуемся критерием измеримости и, следовательно, покрытиями $A_i$ через элементарные множества. Основная трудность состоит в том, что пересечения этих покрытий не внесут какого-то значимого вклада.
	
	Проведём индукцию по $N$:
	\begin{itemize}
		\item База $N = 2$: рассматриваем $A_1 \sqcup A_2$. По критерию измеримости
		\[
			\forall \eps > 0\ \exists M_{1, \eps}, M_{2, \eps} \such \jlm(A_1 \tr M_{1, \eps}) < \eps,\ \ \jlm(A_2 \tr M_{2, \eps}) < \eps
		\]
		Мы уже знаем, что $\jlm(A_1 \sqcup A_2) \le \jlm(A_1) + \jlm(A_2)$ по свойствам мер. Нужно доказать неравенство в другую сторону, этим мы и займёмся. Для начала, отметим следующие включения:
		\begin{align*}
			&{A_i \subset (A_i \tr M_{i, \eps}) \cup M_{i, \eps} \Lora \jlm(A_i) \le \jlm(A_i \tr M_{i, \eps}) + |M_{i, \eps}|}
			\\
			&{M_{i, \eps} \subset (M_{i, \eps} \tr A_i) \cup A_i \Lora |M_{i, \eps}| \le \jlm(A_i \tr M_{i, \eps}) + \jlm(A_i)}
		\end{align*}
		Отсюда приятное неравенство:
		\[
			\left|\jlm(A_i) - |M_{i, \eps}|\right| \le \jlm(A_i \tr M_{i, \eps}) < \eps
		\]
		Теперь, напишем цепочку включений для дизъюнктного объединения и его приближения:
		\begin{align*}
			&{A_1 \sqcup A_2 \subset (A_1 \tr M_{1, \eps}) \cup (A_2 \tr M_{2, \eps}) \cup (M_{1, \eps} \cup M_{2, \eps})}
			\\
			&{M_{1, \eps} \cup M_{2, \eps} \subset (A_1 \tr M_{1, \eps}) \cup (A_2 \tr M_{2, \eps}) \cup (A_1 \sqcup A_2)}
		\end{align*}
		Аналогичное следствие, как и из предыдущих вложений:
		\[
			\big|\jlm(A_1 \sqcup A_2) - |M_{1, \eps} \cup M_{2, \eps}|\big| < 2\eps
		\]
		Вернёмся к уже упомянутому неравенству выше и продолжим его (если этот шаг непонятен, посмотрите предыдущие теоремы):
		\[
			\jlm(A_1 \sqcup A_2) \le \jlm(A_1) + \jlm(A_2) < |M_{1, \eps}| + |M_{2, \eps}| + 2\eps
		\]
		По свойствам мер мы знаем ещё это равенство:
		\[
			|M_{1, \eps} \cup M_{2, \eps}| + |M_{1, \eps} \cap M_{2, \eps}| = |M_{1, \eps}| + |M_{2, \eps}|
		\]
		Нужно оценить пересечение, потому что на объединение неравенство написано уже выше:
		\begin{multline*}
			M_{1, \eps} \cap M_{2, \eps} \subset \big((M_{1, \eps} \tr A_1) \cup A_1\big) \cap \big((M_{2, \eps} \tr A_2) \cup A_2\big) =
			\\
			\Big(\big((M_{1, \eps} \tr A_1) \cup A_1\big) \cap (M_{2, \eps} \tr A_2)\Big) \cup \big(A_2 \cap (M_{1, \eps} \tr A_1)\big) \subset (M_{2, \eps} \tr A_2) \cup (M_{1, \eps} \tr A_1)
		\end{multline*}
		Отсюда рождается ещё один переход в неравенстве:
		\[
			\jlm(A_1) + \jlm(A_2) < |M_{1, \eps} + M_{2, \eps}| + 4\eps
		\]
		Используя неравенство с $|M_{1, \eps} \cup M_{2, \eps}|$ выше, получаем требуемое:
		\[
			\forall \eps > 0 \quad \jlm(A_1) + \jlm(A_2) < \jlm(A_1 \sqcup A_2) + 6\eps
		\]
		
		\item Переход $N > 2$: просто полагаем первые $N - 1$ множество за одно, применяем базу и раскрываем предположение индукции.
	\end{itemize}
\end{proof}

\begin{theorem}
	Если $\{A_i\}_{i = 1}^\infty$ - измеримые по Лебегу подмножества $K_I$, то $\bigcup_{i = 1}^\infty A_i$ также измеримо по Лебегу
\end{theorem}

\begin{proof}
	Сначала рассмотрим $\bscup_{i = 1}^\infty A_i$. Поскольку
	\[
		\forall m \in \N \quad \bscup_{i = 1}^m A_i \subset K_i \Ra \sum_{i = 1}^m \mu(A_i) \le 1
	\]
	то имеет место неравенство $0 \le \sum_{i = 1}^\infty \mu(A_i)\le 1$ и ряд сходится. Это даёт нам возможность оценить его конец:
	\[
		\forall \eps > 0\ \exists r \in \N \such \sum_{i = r + 1}^\infty \mu(A_i) < \frac{\eps}{2}
	\]
	Плюс отметим следующий факт, следующий из измеримости множеств по Лебегу:
	\[
		\forall i \in \N\ \forall \eps > 0\ \exists M_{i, \eps} \such \mu(A_i \tr M_{i, \eps}) < \frac{\eps}{2}
	\]
	Положим за $M = \bigcup_{i = 1}^r M_{i, \eps}$, которое по понятным причинам будет элементарным множеством. Это наш кандидат для критерия измеримости:
	\begin{multline*}
		\left(\bscup_{i = 1}^\infty A_i\right) \tr M \subset \left(\bscup_{i = 1}^r A_i \bs M\right) \cup \left(\bscup_{i = r + 1}^\infty A_i\right) \cup \left(M \bs \bscup_{i = 1}^r A_i\right) \subset
		\\
		\bscup_{i = 1}^r (A_i \bs M_{i, \eps}) \cup \left(\bscup_{i = r + 1}^\infty A_i\right) \subset \bigcup_{i = 1}^r (A_i \tr M_{i, \eps}) \cup \left(\bscup_{i = r + 1}^\infty A_i\right)
	\end{multline*}
	Отсюда вытаскиваем неравенство на меры:
	\[
		\mu^* \left(\left(\bscup_{i = 1}^\infty A_i\right) \tr M\right) \le \sum_{i = 1}^r \mu^* (A_i \tr M_{i, \eps}) + \sum_{i = r + 1}^\infty \mu^* (A_i) < \eps
	\]
	
	Теперь скажем про общий случай. Он очень просто сводится к уже рассмотренному:
	\[
		\bigcup_{i = 1}^\infty A_i = \bscup_{i = 1}^\infty \tilde{A}_i,\ \ \tilde{A}_i = A_i \bs \bigcup_{j = 1}^{i - 1} \tilde{A}_j
	\]
\end{proof}

\begin{note}
	Последняя теорема означает, что совокупность измеримых по Лебегу подмножеств $K_I$ образует \textit{$\sigma$-алгебру} множеств (то есть счётная алгебра).
\end{note}

\begin{theorem} ($\sigma$-аддитивность меры Лебега)
	Если $\{A_i\}_{i = 1}^\infty$ - измеримые по Лебегу подмножества $K_I$, то
	\[
		\mu\left(\bscup_{i = 1}^\infty A_i\right) = \sum_{i = 1}^\infty \mu(A_i)
	\]
\end{theorem}

\begin{proof}
	Положим $A := \bscup_{i = 1}^\infty A_i$. Доказывать равенство будем через неравенства в две стороны:
	\begin{itemize}
		\item $\ge$
		\[
			\forall m \in \N \quad \bscup_{i = 1}^m A_i \subset A \Ra \mu\left(\bscup_{i = 1}^m A_i\right) = \sum_{i = 1}^m \mu(A_i) \le \mu(A) \Ra \sum_{i = 1}^\infty \mu(A_i) \le \mu(A)
		\]
		
		\item $\le$
		
		\[
			A \subset \bscup_{i = 1}^\infty A_i \Ra \mu^*(A) \le \sum_{i = 1}^\infty \mu^*(A_i) \Ra \mu(A) \le \sum_{i = 1}^\infty \mu(A_i)
		\]
	\end{itemize}
\end{proof}

\begin{definition}
	$\{A_i\}_{i = 1}^\infty$ - последовательность множетсв. Определим супремум и инфинум следующим образом:
	\begin{align*}
		&{\sup_i \{A_i\} := \bigcup_{i = 1}^\infty A_i}
		\\
		&{\inf_i \{A_i\} := \bigcap_{i = 1}^\infty A_i}
	\end{align*}
\end{definition}

\begin{definition}
	Пределы для множств введём следующим образом:
	\begin{itemize}
		\item Если $\forall i \in \N\ \ A_i \subset A_{i + 1}$, то
		\[
			\liml_{i \to \infty} A_i := \sup \{A_i\} = \bigcup_{i = 1}^\infty A_i
		\]
		
		\item Если $\forall i \in \N\ \ A_i \supset A_{i + 1}$, то
		\[
			\liml_{i \to \infty} A_i := \inf \{A_i\} = \bigcap_{i = 1}^\infty A_i
		\]
		
		\item Верхний предел определим через эквивалентное свойство:
		\[
			\varlimsup_{i \to \infty} A_i := \liml_{i \to \infty} \sup_{k \ge i} \{A_k\}
		\]
		
		\item Аналогично с нижним пределом:
		\[
			\varliminf_{i \to \infty} A_i := \liml_{i \to \infty} \inf_{k \ge i} \{A_k\}
		\]
		
		\item Отсюда можно дать определение предела в общем случае как общее значение нижнего и верхнего предела:
		\[
			\liml_{i \to \infty} A_i = \varliminf_{i \to \infty} A_i = \varlimsup_{i \to \infty} A_i
		\]
	\end{itemize}
\end{definition}

\begin{theorem} (Непрерывность меры Лебега) \label{4lemth}
	Если последовательность $\{A_i\}_{i = 1}^\infty$ измеримых по Лебегу подмножеств $K_I$ имеет предел $A := \liml_{i \to \infty} A_i$, то $A$ тоже измеримо по Лебегу, причём
	\[
		\mu(A) = \liml_{i \to \infty} \mu(A_i)
	\]
\end{theorem}

\begin{lemma}
	Пусть последовательность $\{A_i\}_{i = 1}^\infty$ такова, что $\forall i \in \N\ \ A_i \subset A_{i + 1}$. При этом $A_i$ - измеримое по Лебегу подмножество $K_I$. Тогда
	\[
		\mu\left(\bigcup_{i = 1}^\infty A_i\right) = \liml_{i \to \infty} \mu(A_i)
	\]
\end{lemma}

\begin{proof}
	Отметим уже известный факт:
	\[
		\bigcup_{i = 1}^\infty A_i = \bscup_{i = 1}^\infty \tilde{A}_i, \quad \tilde{A}_i = A_i \bs \bigcup_{j = 1}^{i - 1} A_j
	\]
	Благодаря этому мы можем применить $\sigma$-аддитивность меры Лебега и записать следующую цепочку:
	\[
		\mu\left(\bigcup_{i = 1}^\infty A_i\right) = \sum_{i = 1}^\infty \mu(\tilde{A}_i) = \liml_{N \to \infty} \sum_{i = 1}^N \mu(\tilde{A}_i) = \liml_{N \to \infty} \mu\left(\bscup_{i = 1}^N \tilde{A}_i\right) = \liml_{N \to \infty} \mu(A_N)
	\]
\end{proof}

\begin{lemma}
	Пусть последовательность $\{A_i\}_{i = 1}^\infty$ такова, что $\forall i \in \N\ \ A_i \supset A_{i + 1}$. При этом $A_i$ - измеримое по Лебегу подмножество $K_I$. Тогда
	\[
		\mu\left(\bigcap_{i = 1}^\infty A_i\right) = \liml_{i \to \infty} \mu(A_i)
	\]
\end{lemma}

\begin{proof}
	Введём последовательность $\{B_i\}_{i = 1}^\infty$ следующим образом:
	\[
		\forall i \in \N \quad B_i = A_1 \bs A_i
	\]
	Тогда $\forall i \in \N\ \ B_i \subset B_{i + 1}$, что позволяет применить к ней предыдущую лемму:
	\[
		\mu\left(\bigcup_{i = 1}^\infty B_i\right) = \liml_{i \to \infty} \mu(B_i) = \mu(A_1) - \liml_{i \to \infty} \mu(A_i)
	\]
	где $\mu\left(\bigcup_{i = 1}^\infty B_i\right) = \mu(A_1) - \mu\left(\bigcap_{i = 1}^\infty B_i\right)$
\end{proof}

\begin{lemma}
	Если последовательность $\{A_i\}_{i = 1}^\infty$ - это измеримые по Лебегу подмножества $K_I$, то
	\[
		\varlimsup_{i \to \infty} \mu(A_i) \le \mu(\varlimsup_{i \to \infty} A_i)
	\]
\end{lemma}

\begin{proof}
	Распишем меру справа в неравенстве:
	\[
		\mu(\varlimsup_{i \to \infty} A_i) = \mu(\liml_{i \to \infty} \sup_{k \ge i} \{A_k\})
	\]
	Супремумы образуют невозрастающую последовательность. Отсюда по предыдущей лемме
	\[
		\mu(\liml_{i \to \infty} \sup_{k \ge i} \{A_k\}) = \liml_{i \to \infty} \mu(\sup_{k \ge i} \{A_k\}) = \liml_{i \to \infty} \mu\left(\bigcup_{k = i}^\infty A_k\right) \ge \liml_{i \to \infty} \sup_{k \ge i} \mu(A_k)
	\]
\end{proof}

\begin{lemma}
	Если последовательность $\{A_i\}_{i = 1}^\infty$ - это измеримые по Лебегу подмножества $K_I$, то
	\[
		\varliminf_{i \to \infty} \mu(A_i) \ge \mu(\varliminf_{i \to \infty} A_i)
	\]
\end{lemma}

\begin{proof}
	Аналогично третьей лемме, только ссылаться будем ещё и на первую:
	\[
		\mu(\varliminf_{i \to \infty} A_i) = \mu(\liml_{i \to \infty} \inf_{k \ge i} \{A_k\}) = \liml_{i \to \infty} \mu(\inf_{k \ge i} \{A_k\}) = \liml_{i \to \infty} \mu\left(\bigcap_{k = 1}^\infty A_k\right) \le \liml_{i \to \infty} \inf_{k \ge i} \mu(A_k)
	\]
\end{proof}

\begin{proof} (теоремы \ref{4lemth})
	Запишем неубывающую цепочку:
	\[
		\varlimsup_{i \to \infty} \mu(A_i) \le \mu(\varlimsup_{i \to \infty} A_i) = \mu(\varliminf_{i \to \infty} A_i) \le \varliminf_{i \to \infty} \mu(A_i) \le \varlimsup_{i \to \infty} \mu(A_i)
	\]
\end{proof}

\begin{note}
	Всё это время мы жили внутри куба $K_I$. Однако, пришло время выйти за его рамки: замостим всё пространство при помощи параллельных сдвигов (очевидно, мера полученного множества от этого не меняется) $\vv{m} + K_I$, где $\vv{m} \in \Z^n$.
\end{note}