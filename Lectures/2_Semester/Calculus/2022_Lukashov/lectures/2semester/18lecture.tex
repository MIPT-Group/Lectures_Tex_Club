\subsection{Внешняя и внутренняя (верхняя и нижняя) меры Жордана и Лебега}

\begin{note}
	Всюду в этом параграфе рассматриваются лишь подмножества $K_I$, определённого в предыдущем параграфе.
\end{note}

\begin{definition}
	\textit{Внешней мерой Жордана} множества $A$ называется величина
	\[
		\mu^*_\gj(A) = \inf_{A \subset \bigcup_{i = 1}^r M_i} \ \left\{\suml_{i = 1}^r |M_i|\right\}
	\]
	где $\inf$ берётся по всем покрытиям множества $A$ конечным числом элементарных множеств.
\end{definition}

\begin{definition}
	\textit{Внешней мерой Лебега} множества $A$ называется величина
	\[
		\mu^*(A) = \inf_{A \subset \bigcup_{i = 1}^\infty M_i} \ \left\{\suml_{i = 1}^\infty |M_i|\right\}
	\]
	где $\inf$ берётся по всем покрытиям $A$ счётным числом элементарных множеств.
\end{definition}

\begin{note}
	Формально говоря, во внешней мере Жордана можно отказаться от символов объединения и суммы, ибо элементарные множества образуют кольцо. Однако, определение было дано для иллюстрации разницы с внешней мерой Лебега.
\end{note}

\begin{example}
	Рассмотрим $A = \Q \cap [-1/2; 1/2]$. Тогда
	\[
		\upjm(A) = 1,\ \mu^*(A) = 0
	\]
	С внешней мерой Жордана понятно, а вот для меры Лебега не очень. В силу счётности рациональных чисел, $A$ можно представить как $A = \{x_i\}_{i = 1}^\infty$. Тогда, положим $M_i$ следующим образом:
	\[
		M_i = \left(x_i - \frac{\eps}{2^i}; x_i + \frac{\eps}{2^i}\right) \cap \left[-\frac{1}{2}; \frac{1}{2}\right]
	\]
	Объединение $M_i$ очевидным образом покроет $A$. При этом
	\[
		\suml_{i = 1}^\infty |M_i| \le \suml_{i = 1}^\infty \frac{\eps}{2^{i - 1}} = 2\eps
	\]
\end{example}

\begin{note}
	Далее принято соглашение, что если утверждение верно для обеих мер сразу, то она обозначается в нём как $\upjlm$.
\end{note}

\begin{theorem} (Основные свойства внешних мер)
	\begin{enumerate}
		\item $\forall$ элементарного множества $A$ верно, что
		\[
			|A| = \upjm(A) = \mu^*(A)
		\]
		
		\item Если $A \subset \bigcup_{i = 1}^r A_i$, то
		\[
			\upjlm(A) \le \suml_{i = 1}^r \upjlm(A_i)
		\]
		
		\item Для любого множества $A$ верно неравенство:
		\[
			\mu^*(A) \le \upjm(A)
		\]
		
		\item Если $A \subset \bigcup_{i = 1}^\infty A_i$, то
		\[
			\mu^*(A) \le \suml_{i = 1}^\infty \mu^*(A_i)
		\]
	\end{enumerate}
\end{theorem}

\begin{proof}~
	\begin{enumerate}
		\item \begin{enumerate}
			\item \(A \subset A \Lora \upjlm(A) \le |A|\)
			
			\item \(A \subset \bigcup_{i = 1}^r M_i \Lora |A| \le \sum_{i = 1}^r |M_i| \Lora |A| \le \upjm(A)\)
			
			\item \(A \subset \bigcup_{i = 1}^\infty M_i \Lora |A| \le \sum_{i = 1}^\infty |M_i| \Lora |A| \le \mu^*(A)\)
		\end{enumerate}
	
		\item Распишем внешнюю меру Жордана для $A_i$:
		\[
			\upjm(A_i) = \inf_{A_i \subset \bigcup_{j = 1}^{r_i} M_{i, j}} \left(\suml_{j = 1}^{r_i} |M_{i, j}|\right)
		\]
		Это означает, что
		\[
			\forall \eps > 0\ \exists \{M_{i, j}\}_{j = 1}^{r_i} \quad \suml_{j = 1}^{r_i} |M_{i, j}| < \upjm(A_i) + \frac{\eps}{2^i}
		\]
		Так как $A_i$ вложены в объединения $M_{i, j}$, то верен следующий факт:
		\[
			A \subset \bigcup_{i = 1}^r \bigcup_{j = 1}^{r_i} M_{i, j} \Longrightarrow \forall \eps > 0 \quad  \upjm(A) \le \suml_{i = 1}^r \suml_{j = 1}^{r_i} |M_{i, j}| < \suml_{i = 1}^r \upjm(A_i) + \eps
		\]
		Для $\mu^*$ всё аналогично, просто нужно в $r_i$ поменять на бесконечности.
		
		\item Просто по определению внешних мер.
		
		\item Если сумма ряда бесконечна, то доказывать нечего. Иначе будем делать действия, похожие на доказательство второго свойства, но ещё и вспомним идею из примера:
		\[
			\forall \eps > 0\ \exists \{M_{i, j}\}_{j = 1}^\infty \quad \suml_{j = 1}^\infty |M_{i, j}| < \mu^*(A_i) + \frac{\eps}{2^i}
		\]
		Теперь счётно покрываем $A$:
		\[
			A \subset \bigcup_{i = 1}^\infty \bigcup_{j = 1}^\infty M_{i, j} \Longrightarrow \forall \eps > 0 \quad \mu^*(A) \le \suml_{i = 1}^\infty \suml_{j = 1}^\infty |M_{i, j}| < \suml_{i = 1}^\infty \mu^*(A_i) + \eps
		\]
	\end{enumerate}
\end{proof}

\begin{note}
	Контрпримером четвёртого свойства для меры Жордана служит уже вышеупомянутый пример с $A = \Q \cap [-1/2; 1/2]$. Действительно, мера любой рациональной точки будет ноль, но вместе они дают единицу.
\end{note}

\begin{definition}
	Для $A \subset K_I$ \textit{внутренней мерой Жордана (Лебега)} называется величина
	\[
		\downjlm(A) := 1 - \upjlm(A')
	\]
	где $A' = K_I \bs A$
\end{definition}

\begin{theorem} (Основное свойство внутренней меры)
	Для любого $A \subset K_I$ верно, что
	\[
		\downjlm(A) \le \upjlm(A)
	\]
\end{theorem}

\begin{proof}
	По второму свойству внешней меры:
	\[
		K_I \subset A \cup A' \Longrightarrow \upjlm(K_I) \le \upjlm(A) + \upjlm(A')
	\]
	где $|K_I| = \upjlm(K_I) = 1$
\end{proof}

\subsection{Измеримые множества. Основные свойства мер Жордана и Лебега}

\begin{note}
	Сначала будем рассматривать лишь подмножества $K_I$
\end{note}

\begin{definition}
	Множество $A \subset K_I$ называется \textit{измеримым по Лебегу (Жордану)}, если
	\[
		\upjlm(A) = \downjlm(A)
	\]
\end{definition}

\begin{definition}
	Для измеримого по Лебегу (Жордану) множества $A \subset K_I$ \textit{мерой Лебега (Жордана)} называется общее значение соответствующих внешней и внутренней мер:
	\[
		\jlm(A) = \upjlm(A) = \downjlm(A)
	\]
\end{definition}

\begin{example}
	Рассмотрим $A = \Q \cap [-1/2; 1/2]$. Тогда
	\[
		\downjm(A) = 1 - \upjm(A') = 0 < \upjm(A) = 1
	\]
	При этом $0 \le \mu_*(A) \le \mu^*(A) = 0$. То есть наше множество измеримо по Лебегу, но неизмеримо по Жордану.
\end{example}

\begin{theorem} (Критерий измеримости)
	Множество $A \subset K_I$ измеримо по Лебегу (Жордану) тогда и только тогда, когда верно условие:
	\[
		\forall \eps > 0\ \exists M_\eps \such \upjlm(A \tr M_\eps) < \eps
	\]
	где $M_\eps$ - элементарное множество.
\end{theorem}

\begin{proof}~
	\begin{itemize}
		\item Достаточность $(\La)$. Снова воспользуемся фактом, что $K_I \subset A \cup A'$, где $A' = K_I \bs A$. Тогда
		\[
			1 = \upjlm(K_I) \le \upjlm(A) + \upjlm(A')
		\]
		И аналогичный трюк проделаем с $A$ и $A'$:
		\begin{itemize}
			\item \(A \subset (A \bs M_\eps) \cup M_\eps \Longrightarrow \upjlm(A) \le \upjlm(A \tr M_\eps) + \underbrace{\upjlm(M_\eps)}_{|M_\eps|}\)
			
			\item \(A' \subset \underbrace{(A' \bs M'_\eps)}_{M_\eps \bs A} \cup M'_\eps \Longrightarrow \upjlm(A') \le \upjlm(M_\eps \tr A) + \underbrace{\upjlm(M'_\eps)}_{|M'_\eps|}\)
		\end{itemize}
		Отметим, что в неравенстве на меры стоят симметрические разности потому, что это только увеличивает сумму. Уже отсюда получаем неравенство:
		\[
			1 \le \upjlm(A) + \upjlm(A') \le 2\upjlm(A \tr M_\eps) + |M_\eps| + |M'_\eps| = 2\eps + 1 \xrightarrow[\eps \to 0]{} 1
		\]
		
		\item Необходимость $(\Ra)$. Условие означает, что $\upjlm(A) = \downjlm(A)$. Это также значит следующее:
		\begin{itemize}
			\item \(\forall \eps > 0\ \exists \{P_i\} \such \suml_i |P_i| < \upjlm(A) + \eps\)
			
			\item \(\forall \eps > 0\ \exists \{Q_j\} \such \suml_j |Q_j| < \upjlm(A') + \eps\)
		\end{itemize}
		Зафиксируем $\eps > 0$. Сходимость последовательности частичных сумм позволяет нам ограничить остаток (это нужно для Лебега):
		\[
			\exists r \such \suml_{i = r + 1}^\infty |P_i| < \eps; \quad \exists s \such \suml_{j = s + 1}^\infty |Q_j| < \eps
		\]
		Положим $P_\eps := \bigcup_{i = 1}^r P_i$ (для Жордана просто по всем $i$, ибо там и без того конечное объединение). Мы хотим теперь показать, что $P_\eps$ подходит под условие. Для этого, рассмотрим $A \tr P_\eps$:
		\[
			A \tr P_\eps = (A \bs P_\eps) \cup (P_\eps \bs A)
		\]
		При Жордане у нас $A \bs P_\eps = \emptyset$. Для Лебега же
		\[
			A \bs P_\eps \subset \bigcup_{i = r + 1}^\infty P_i \Longrightarrow \mu^*(A \bs P_\eps) \le \suml_{i = r + 1}^\infty \mu^*(P_i) < \eps
		\]
		Далее нам потребуется $Q_\eps := \bigcup_{j = 1}^s Q_j$ (для Жордана снова берём все $j$ вообще). Заметим следующее вложение:
		\[
			P_\eps \bs A \subset (P_\eps \cap Q_\eps) \cup (A' \bs Q_\eps)
		\]
		При этом $\mu(A' \bs Q_\eps) < \eps$. Нужно оценить сверху $\mu(P_\eps \cap Q_\eps) = |P_\eps \cap Q_\eps|$ (коль скоро множество элементарное). Отсюда
		\[
			|P_\eps \cap Q_\eps| = |P_\eps| + |Q_\eps| - |P_\eps \cup Q_\eps|
		\]
		Чтобы получить оценку сверху, нужно оценить меру объединения снизу. Увидим, что
		\[
			K_I \subset \left(\bigcup_i P_i\right) \cup \left(\bigcup_j Q_j\right) = P_\eps \cup Q_\eps \cup \left(\bigcup_{i = r + 1}^\infty P_i\right) \cup \left(\bigcup_{j = s + 1}^\infty Q_j\right)
		\]
		Теперь, можно воспользоваться свойством внешней меры Лебега:
		\[
			1 = \mu^*(K_I) \le \mu^*(P_\eps \cup Q_\eps) + \mu^*\left(\bigcup_{i = r + 1}^\infty\ P_i\right) + \mu^*\left(\bigcup_{j = s + 1}^\infty Q_j\right) < |P_\eps \cup Q_\eps| + 2\eps
		\]
		Отсюда $|P_\eps \cup Q_\eps| > 1 - 2\eps$ и из $|P_\eps| \le \sum_i |P_i| < \upjlm(A) + \eps$, $|Q_\eps| \le \sum_j |Q_j| < \upjlm(A') + \eps$ получаем, что
		\[
			|P_\eps \cap Q_\eps| < 1 + 2\eps - (1 - 2\eps) = 4\eps \Longrightarrow \mu^*(P_\eps \bs A) < 5\eps \Longrightarrow \mu^*(A \tr P_\eps) < 6\eps
		\]
	\end{itemize}
\end{proof}

\begin{theorem}
	Семейство измеримых по Лебегу (Жордану) подмножеств $K_i$ образует алгебру множеств (иначе говоря, есть замкнутость относительно стандартных операций над множествами).
\end{theorem}

\begin{note}
	Для упрощения нотаций, будем обозначать любую из верхних/нижних мер как просто $\mu^*$/$\mu_*$. Аналогично и в случае просто меры.
\end{note}

\begin{proof}
	Пусть $A, B$ - измеримые множества
	\begin{itemize}
		\item Объединение $A \cup B$. По критерию измеримости
		\[
			\forall \eps > 0\ \exists M_\eps, N_\eps \such \mu^*(A \tr M_\eps) < \eps, \mu^*(B \tr N_\eps) < \eps
		\]
		Тогда покажем, что мера множества $(A \cup B) \tr (M_\eps \cup N_\eps)$ подходит под критерий измеримости:
		\begin{multline*}
			(A \cup B) \tr (M_\eps \cup N_\eps) = \big((A \cup B) \bs (M_\eps \cup N_\eps)\big) \cup \big((M_\eps \cup N_\eps) \bs (A \cup B)\big) \subset
			\\
			(A \bs M_\eps) \cup (B \bs N_\eps) \cup (M_\eps \bs A) \cup (N_\eps \bs B) = (A \tr M_\eps) \cup (B \tr N_\eps)
		\end{multline*}
		Следовательно
		\[
			\mu^*((A \cup B) \tr (M_\eps \cup N_\eps)) \le \mu^*(A \tr M_\eps) + \mu^*(B \tr N_\eps) < 2\eps
		\]
		
		\item Все остальные свойства доказываются из соображения, что $A$ - измеримо $\lra$ $A' = K_I \bs A$ тоже измеримо:
		\begin{itemize}
			\item \((A \cap B)' = A' \cup B'\)
			
			\item \(A \bs B = A \cap B'\)
			
			\item \(A \tr B = (A \bs B) \cup (B \bs A)\)
		\end{itemize}
	\end{itemize}
\end{proof}

\textcolor{red}{На этом мои полномочия всё. Я устал, я - мухожук... (Продолжение будет после конца сессии.)}