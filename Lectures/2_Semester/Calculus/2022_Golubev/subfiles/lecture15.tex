\documentclass[../main.tex]{subfiles}

\begin{document}
\section{Числовые ряды}

\begin{definition}
  Пусть задана последовательность $\{ a_k  \}^{ \infty}_{k=1} $. Будем называть \emph{частичной суммой} 
  \begin{gather} 
    S_n = \sum_{k=1}^{n} a_k.
  \end{gather}  
\end{definition}


\begin{definition}
  Будем называть \emph{рядом} 
  \begin{gather} 
    \sum_{k=1}^{\infty} = \lim_{n \to \infty} S_n  .
  \end{gather}
\end{definition}


\begin{definition}
  Если $ \sum_{k=1}^{\infty} \in \R$, будем говорить, что ряд \emph{сходится}, иначе~--- \emph{расходится}. 
\end{definition}


\begin{note}
  Будем отсылка к интегралам, которая будет прояснятся в процессе построения теории.
\end{note}


\begin{proposition}
  Пусть $ \sum_{k=1}^{\infty} a_k $~--- сходится, тогда $\lim_{k \to \infty} = 0$  
\end{proposition}


\begin{proof}
  По определению $\lim_{n \to \infty} S_n = S \in \R$, и $a_n = S_n - S_{n-1}$. Рассмотрим предел 
  \begin{gather} 
    \lim_{n \to \infty} a_n = \lim_{n \to \infty} S_n - \lim_{n \to \infty} S_{n-1} = S- S = 0 .
  \end{gather}
\end{proof}


\begin{note}
  В обратную сторону неверно.
\end{note}


\begin{proposition}[Критерий Коши]
  Пусть $ \sum_{k=1}^{\infty} a_k$~--- сходится \nas 
  \begin{gather} 
    \exists N \in \N : \forall k_1, k_2 \geq N \hence \left| \sum_{k=k_1 }^{k_2 } a_k  \right| < \eps .
  \end{gather} 
\end{proposition}


\begin{note}
  Очень идеологически похоже на интегралы.
\end{note}


\begin{proof}
  Сходимость ряда, по определению, эквивалентно сходимости $ \{S_n\}_{k=1}^{ \infty }$, что (по критерию Коши сходимости последовательностей) эквивалентно 
  \begin{gather} 
    \forall \eps > 0 \exists N: \forall k_1, k_2 \geq N \hence \left| S_{k_2 } - S_{k_{1} - 1} \right| = \left| \sum_{k=1}^{k2} a_k - \sum_{k=1}^{k_1 - 1} a_k  \right| = \left| \sum_{k=k_1 }^{k_1 -1} a_k  \right| < \eps .
  \end{gather}
\end{proof}

\begin{example}
  Рассмотрим $ \sum_{k=1}^{\infty} \frac{1}{k} $~--- гармонический ряд. Необходимое условие, очевидно, выполняется. Рассмотрим отрицание критерия Коши: 
  \begin{gather} 
    \exists \eps > 0 : \forall N \in \N \exists k_1,k_2 \geq N: \left| \sum_{k=k_1 }^{k_2 } a_k  \right| > \eps .
  \end{gather}
  Выберем $k_1 = N + 1, k_2 = 2N$. Тогда $ \sum_{k=N+1}^{2N} \frac{1}{k} > N \cdot \frac{1}{2N} = \frac{1}{2} $. Тогда 
  \begin{gather} 
    \exists \eps = \frac{1}{2} : \forall N \in \N \exists k_1 = N + 1, k_2 = 2N : \left| \sum_{k=k_1}^{k_2 } \frac{1}{k}  \right| > \frac{1}{2} = \eps  .
  \end{gather}
  Из критерия Коши получаем, что ряд расходится.
\end{example}


\begin{proposition}[Принцип локализации] \label{prop:ser:local}
  $\forall k_0 \in \N$, сходимость $ \sum_{k=1}^{\infty} a_k $ эквивалентна сходимости $ \sum_{k=k_0 }^{\infty} a_k $. 
\end{proposition}


\begin{note}
  Как и с интегралами, нас интересует сходимость или расходимость, и мало интересует число, к которому ряд сходится.
\end{note}


\begin{proof}
  Выберем $n > k_0 $, тогда $ \sum_{k=1}^{n} a_k  = \sum_{k=1}^{k_0 - 1} a_k + \sum_{k=k0}^{n} a_k  $. Переходя к пределу в обеих частях и вынося за знак предела константу $ \sum_{k=1}^{k_0 -1} a_k $ получаем одновременное существование или не существование предела. 
\end{proof}


\begin{proposition}[Линейность]
  Пусть $ \sum_{k=1}^{\infty} a_k $~--- сходится, $ \sum_{k=1}^{\infty} b_k  $~--- сходится, тогда $\forall \alpha, \beta \in \R$ ряд $ \sum_{k=1}^{\infty} (\alpha a_k + \beta b_k ) $~--- сходится.
\end{proposition}


\begin{note}
  Сходящиеся ряды образуют линейное пространство.
\end{note}


\begin{note}
  Так как у нас пока мало утверждений, практически все доказательства строятся на определении.
\end{note}


\begin{proof}
  Рассмотрим $ \sum_{k=1}^{n} (\alpha a_k +\beta b_k ) = \alpha \sum_{k=1}^{n} a_k + \beta \sum_{k=1}^{n} b_k $. Переходя к пределу в обеих частях равенства получаем искомое утверждение. 
\end{proof}


\begin{corollary}
  Если $ \sum_{k=1}^{\infty} a_k  = A$, $ \sum_{k=1}^{\infty} b_k = B  $, тогда $\forall \alpha, \beta \in \R$ верно $ \sum_{k=1}^{\infty} (\alpha a_k + \beta b_k ) = \alpha A + \beta B$ 
\end{corollary}

\begin{example}
    Исследуем $ \sum_{k=1}^{\infty} q^{k} $. Рассмотрим несколько случаев \begin{enumerate}
      \item $ \left| q \right| \geq 1$, тогда $ \lim_{n \to \infty} \left| q \right| ^{k} \neq 0$. Необходимое условие не выполняется, значит ряд расходится.
      \item $ \left| q \right| < 1$, тогда необходимое условие выполняется. Из школы $$ \sum_{k=1}^{n} q^{k} = q \frac{1-q}{(1-q^{n})}  \xrightarrow[n \to  \infty ]{} \frac{q}{1-q} .$$ Значит ряд сходится, притом мы можем предъявить конкретное число. 
    \end{enumerate}
\end{example}


\begin{note}
  Сейчас, на время, мы ограничимся лишь знакопостоянными рядами. 
\end{note}

\subsection{Знакопостоянные ряды}

\begin{proposition}[Критерий сходимости знакопостоянных рядов]
  Пусть $a_k \geq 0$, $\forall k \in \N$. Тогда $ \sum_{k=1}^{\infty} a_k $~--- сходится \nas $\sup_{k \in N} S_{n} < + \infty$. 
\end{proposition}


\begin{note}
  Утверждение полностью аналогично \ref{prop:int:sup} (аналог первообразной~--- частичная сумма). 
\end{note}


\begin{proof}
  Отметим, что $S_n$~--- нестрого возрастает. Тогда точно существует предел (возможно бесконечный), и $\lim_{n \to \infty} = \sup_{n \in \N} S_n$. Заканчивает доказательство то, что $\sup X < + \infty $ \nas $\sup X \in \R$.  
\end{proof}


\begin{proposition}[Признак сравнения]
  Пусть $\forall k \in \N \hence 0 \leq a_k \leq b_k$, тогда 
  \begin{enumerate}
    \item $ \sum_{k=1}^{\infty} b_k  $~--- сходится, тогда $ \sum_{k=1}^{\infty} a_k $~---  сходится.
    \item $ \sum_{k=1}^{\infty} a_k $~---  расходится, тогда $ \sum_{k=1}^{\infty} b_k $~--- расходится.
  \end{enumerate} 
\end{proposition}


\begin{note}
    В силу \ref{prop:ser:local} можно ослабить условие: заменить $\forall k \in \N$ на $\exists k_0 \in \N: \forall k > k_0 $.
\end{note}


\begin{proof}
  Докажем первый, второй следует из первого (предполагая противное сразу приходим к противоречию). Если $ \sum_{k=1}^{\infty} b_k $~--- сходится \nas $\sup_{n\in\N} S_n(b) < +\infty$. Заметим, что $\forall n \in N \hence S_n(a) \leq S_n(b)$. Тогда 
  \begin{gather} 
    \sup_{n\in\N}S_n(a) \leq \sup_{n\in\N} S_n(b) < +\infty .
  \end{gather}  
  Тогда ряд $ \sum_{k=1}^{\infty} a_k $~---  сходится.
\end{proof}


\begin{note}
  И утверждение, и доказательство полностью аналогичны интегралам.
\end{note}


\begin{definition}
  Пусть $ k \in \N \hence a_k, b_k \geq 0$. Будем говорить, что $a_k $ \emph{эквивалентно по сходимости} $b_k $ ( $a_k \eqincon b_k $ ), если 
  \begin{gather} 
    \exists k_0 \in N, m, M > 0: \forall k \geq k_0 \hence mb_k \leq a_k \leq Mb_k  .
  \end{gather} 
\end{definition}



\begin{proposition}[Признак сравнения]
  Пусть $\forall k \in \N \hence a_k, b_k \geq 0$, и $a_k \eqincon b_k $, тогда $ \sum_{k=1}^{\infty} a_k $ и $ \sum_{k=1}^{\infty} b_k  $ сходятся или расходятся одновременно.  
  
\end{proposition}


\begin{proof}
  Из эквивалентности 
  \begin{gather} 
    \exists k_0 \in \N, m, M > 0: mb_k \leq a_k \leq Mb_k .
  \end{gather}
  Рассмотрим неравенства $a_k \leq M b_k$, $a_k \leq \frac{1}{m} a_k $. Из этих неравенств и первого признака сравнения, получаем эквивалентность их сходимости.
\end{proof}


\begin{proposition}[Интегральный признак сравнения]
  Пусть $f:[1, + \infty) \to \R$~--- монотонна. Тогда $ \sum_{k=1}^{\infty} f(k) $ и $ \int_{1}^{ +\infty} f(x) dx$ сходятся или расходятся одновременно. 
\end{proposition}


\begin{proof}
  Рассмотрим $A = \lim_{x \to \infty } f(x)$. Он существует, так как $f$ монотонна. 
  \begin{enumerate}
    \item $A \neq 0$~--- не выполняется необходимое условие, тогда ряд расходится. Если при этом этот предел равен бесконечности, то интеграла не существует в принципе. Тогда $f(x) \eqincon 1$, значит $ \int_{1}^{+\infty} f(x) dx $ расходится.
    \item $A = 0$.  Пусть $\forall k\in\N, x \in [k, k+1]$, и без ограничения общности будем считать, что $f$ убывает. Тогда $f(k+1) \leq f(x)  \leq f(k)$. Рассмотрим 
    \begin{gather} 
      f(k+1) = \int_{k}^{k+1} f(k+1) dx \leq \int_{k}^{k+1} f(x) dx \leq \int_{k}^{k+1} f(k) dk = f(k).
    \end{gather} 
    Пусть $S_n = \sum_{k=1}^{n} f(k) $, $F(t) = \int_{1}^{t} f(x) dx$~--- нестрого возрастает. Тогда просуммируем неравенства ($\forall x \in [k, k+1]$): 
    \begin{gather} 
      S_{k+1} - f(1) \leq F(k+1) \leq S_k , \\
      S_{k} - f(1) \leq F(k) \leq F(x) \leq F(k+1) \leq S_k, \\
      S_k - f(1) \leq F(x) \leq S_k. 
    \end{gather} 
    Пусть ряд сходится, тогда $S_k  \xrightarrow[k \to  \infty ]{} S_0 \in \R $. Переходя к пределу $k \to \infty$ мы также переходим к пределу $x \to \infty$. Получаем эквивалентность существования пределов (конечных пределов), то есть эквивалентность сходимостей.   
    
  \end{enumerate}
\end{proof}

\begin{example}
  Исследуем $ \sum_{k=1}^{\infty} \frac{1}{k^{\alpha}} $, пусть $f(x) = \frac{1}{k^{\alpha}}$. Тогда мы сводим к сходимости $\int_{1}^{+\infty} \frac{1}{x^{\alpha}} dx$. Тогда при $\alpha > 1$ сходится, и $a \leq 1$ расходится.  
\end{example}


\begin{proposition}[Признак Даламбера]
  Пусть $ a_k > 0 \forall k \in \N$.
  \begin{enumerate}
    \item Если $\exists k_0 \in \N, q \in (0, 1) : \forall k \geq k_0 \hence \frac{a_{k+1}}{a_k } \leq q$, тогда $ \sum_{k=1}^{\infty} a_k $~--- сходится. 
    \item Если $\exists k_0 \in \N : \forall k \geq k_0 \hence \frac{a_{k+1}}{a_k } \geq 1$, тогда $ \sum_{k=1}^{\infty} a_k $~--- расходится.
  \end{enumerate}
\end{proposition}


\begin{proof}
  \begin{enumerate}
    \item $a_{k+1} \leq qa_k \leq q (qa_k ) \leq ... q^{k+1-k_0}a_{k_0 }$. Тогда 
    \begin{gather} 
      \sum_{k=k_0 }^{n} a_k \leq \sum_{k=k_0 }^{n} q^{k+1-k_0 } a_{k_0 } = \frac{a_{k_0 }}{q^{k_0 -1}} \sum_{k=k_0 }^{n} q^{k}  .
    \end{gather} 
  \end{enumerate}
  Переходя к пределу при $n \to \infty$, получаем что искомый ряд сходится.
  \item $\forall k \geq k_0 \hence a_{k+1} \geq a_{k_0 }$, тогда $\lim_{k \to \infty} a_k \neq 0$, значит ряд расходится.  
\end{proof}


\begin{corollary}[Предельный признак Даламбера]
  Пусть $a_k > 0, \forall k \in \N$ и $\lim_{k \to \infty} = q \in \R$. Тогда:
  \begin{enumerate}
    \item если $q < 1$, тогда $ \sum_{k=1}^{\infty} a_k$ сходится;
    \item если $q > 1$, тогда $ \sum_{k=1}^{\infty} a_k$ расходится;
    \item если $q = 1$, ряд может как сходиться, так и расходится. 
  \end{enumerate}  
\end{corollary}


\begin{proof}
  \begin{enumerate}
    \item Из определения предела при $\eps = \frac{1-q}{2} \exists k_0 : \forall k \geq k_0 \hence \frac{a_{k+1}}{a_k } < q + \frac{1-q}{2} = \frac{1+q}{2} > 1$. По первому пункту признака Даламбера ряд сходится.
    \item Из определения предела при $\eps = \frac{q-1}{2} \exists k_0 : \forall k \geq k_0 \hence \frac{a_{k+1}}{a_k } > q - \frac{q-1}{2} = \frac{1+q}{2} > 1$. По второму пункту признака Даламбера ряд расходится.
    \item Необходимо предъявить два примера. Рассмотрим $a_k = \frac{1}{k}$. Ряд $ \sum_{k=1}^{\infty} a_k $ расходится и $ \lim_{n \to \infty} \frac{a_{k+1}}{a_k } = 1$. Теперь рассмотрим $b_k = \frac{1}{k^{2}}$. Ряд $ \sum_{k=1}^{\infty} b_k  $ сходится и $ \lim_{n \to \infty} \frac{b_{k+1}}{b_k } = 1$.  
  \end{enumerate}
\end{proof}

\end{document}