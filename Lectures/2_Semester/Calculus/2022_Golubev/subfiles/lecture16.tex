\documentclass[../main.tex]{subfiles}

\begin{document}

\begin{proposition}[Признак Коши]
  Пусть $\forall k \in \N a_k \geq 0$. Тогда 
  \begin{enumerate}
    \item Если $\exists k_0 \in \N, \exists q \in (0, 1): \forall k \geq k_0 \hence \sqrt[k]{a_k } \leq q$, то $ \sum_{k=1}^{\infty} a_k $~---  сходится.
    \item Если $\exists k_0 : \forall k \geq k_0 \hence \sqrt[k]{a_k }\geq 1$, тогда $ \sum_{k=1}^{\infty} a_k $~--- расходится. 
  \end{enumerate}
\end{proposition}


\begin{note}
  Можно ослабить требование на $k$  пользуясь принципом локализации.
\end{note}


\begin{proof}
  \begin{enumerate}
    \item $a_k \leq q^{k}$. По принципу локализации нас интересует ряд $ \sum_{k=k_0 }^{\infty} a_k $. Рассмотрим $b_k = q^{k}, k\geq k_0$. Тогда $ \sum_{k=k_0 }^{\infty} b_k  $~---  сходится, при этом $\forall k \geq k_0 \hence a_k \leq b_k $, тогда по первому признаку сравнения $ \sum_{k=1}^{\infty} a_k $~--- сходится.
    \item $a_k \geq 1 $, тогда если $\exists \lim_{n \to \infty} a_k \geq 1$, в любом случае не выполняется необходимое условие сходимости. 
  \end{enumerate}
\end{proof}


\begin{corollary}[Предельный признак Коши]
  Пусть $\forall k \in \N a_k \geq 0$ и $\lim_{k \to \infty} \sqrt[k]{a_k }= a \in \R$. Тогда
  \begin{enumerate}
    \item если $a<1$, то $ \sum_{k=1}^{\infty} a_k $ сходится;
    \item если $a > 1$, то $ \sum_{k=1}^{\infty} a_k $ расходится;
    \item если $a = 1$, то ничего нельзя сказать.  
  \end{enumerate} 
\end{corollary}

\begin{proof}
  \begin{enumerate}
    \item Из определения предела выберем $\eps = \frac{1-a}{2}$. Тогда начиная с некоторого $k = k_0$ верно $\sqrt[k]{a_k } < a + \frac{1-a}{2} = \frac{a+1}{2} < 1$. Тогда по первому пункту признака Коши ряд сходится.
    \item Из определения предела выберем $\eps = \frac{a - 1}{2}$. Тогда начиная с некоторого $k = k_0$ верно $\sqrt[k]{a_k} > a + \frac{1-a}{2} = \frac{1+a}{2} > 1$. Тогда по второму пункту признака Коши ряд расходится. 
    \item Нужно привести два примера. Рассмотрим $\frac{1}{\sqrt[k]{k}}  \xrightarrow[k \to  \infty ]{} 1 $. При этом гармонический ряд расходится. \\ Рассмотрим $\frac{1}{\sqrt[k]{k^{2}}}  \xrightarrow[k \to  \infty ]{} 1 $, при этом ряд из обратных квадратов сходится.
  \end{enumerate}
\end{proof}

\subsection{Знакопеременные ряды}


\begin{definition}
  Будем говорить, что ряд $ \sum_{k=1}^{\infty} a_k $  \emph{сходится абсолютно}, если $ \sum_{k=1}^{\infty} \left| a_k \right|  $ сходится.
\end{definition}

\begin{definition}
  Будем говорить, что ряд \emph{сходится условно}, если он сходится, но не сходится абсолютно.
\end{definition}


\begin{proposition}[Критерий Коши]
  Ряд $ \sum_{k=1}^{\infty} a_k $ сходится \nas 
  \begin{gather} 
    \forall \eps > 0 \exists N \in \N : \forall k_1 > k_2 \geq N \hence \left| \sum_{k=k_1 }^{k_2 } a_k  \right| < \eps  .
  \end{gather}
\end{proposition}


\begin{proof}
  Ряд $ \sum_{k=1}^{\infty} a_k $ сходится \nas $S_n = \sum_{k=1}^{\infty} a_k $~--- сходится \nas (по критерию Коши для последовательностей) 
  \begin{gather} 
    \forall \eps > 0 \exists N \in \N : \forall k_1, k_2 \geq N \hence \left| S_{k_2 }  - S_{k_1-1 }\right| = \left| \sum_{k=k_1 }^{k_2 } \right|  < \eps.
  \end{gather} 
\end{proof}


\begin{proposition}
  Если ряд $ \sum_{k=1}^{\infty} a_k $ сходится абсолютно, тогда он сходится.
\end{proposition}

\begin{proof}
  Запишем условие критерия Коши и воспользуемся критерием Коши для $ \sum_{k=1}^{\infty} \left| a_k \right|  $: 
  \begin{gather} 
    \forall \eps > 0 \exists N \in \N: \forall k_1 > k_2 \geq N \hence \left| \sum_{k=k_1 }^{k_2 }  a_k  \right| \leq \left| \sum_{k=k_1 }^{k_2 } \left| a_k \right|  \right| < \eps.
  \end{gather} 
  По критерию Коши ряд сходится.
\end{proof}


\begin{proposition}
  Пусть $ \sum_{k=1}^{\infty} a_k $, $ \sum_{k=1}^{\infty} b_k $~--- сходятся абсолютно, тогда $\forall \alpha, \beta \in \R$ ряд $ \sum_{k=1}^{\infty} (\alpha a_k + \beta b_k ) $  сходится абсолютно. 
\end{proposition}


\begin{proof}
  Заметим, что 
  \begin{gather} 
    \left| \sum_{k=k_1 }^{k_2 } (\alpha a_k + \beta b_k )  \right| \leq \sum_{k=k_1 }^{k_2 } \left( \left| \alpha a_k  \right| + \left| \beta b_k  \right| \right) = \left| \alpha \right| \sum_{k=1}^{\infty} \left| a_k \right| + \left| \beta \right| \sum_{k=1}^{\infty} \left| b_k  \right| .  
  \end{gather}
  Подставляя в условие критерия Коши $N = \max \{ N_{a}, N_{b} \}$ получаем, что $ \left| \sum_{k=k_1 }^{k_2 } (\alpha a_k + \beta b_k )  \right| $  сходится, значит искомый ряд сходится абсолютно.
\end{proof}


\begin{proposition}[Признак Дирихле] \label{prop:ser:dirichlet}
  Если
  \begin{enumerate} 
    \item последовательность частичных сумм $ \sum_{k=1}^{\infty} a_k $ ограничена;
    \item последовательность $b_k$~--- монотонна;
    \item $\lim_{k \to \infty} b_k = 0$,
  \end{enumerate}
  тогда $ \sum_{k=1}^{\infty} a_k b_k  $ сходится. 
\end{proposition}


\begin{proof}
  Пусть $A_n = \sum_{k=1}^{n} a_k $ и $A_0 = 0$ , $a_n = A_n - A_{n-1}$, без ограничения общности $b_k $ убывает. Рассмотрим 
  \begin{multline} \label{proof:ser:dirichlet}
    \sum_{k=1}^{n} a_k b_k = \sum_{k=1}^{n} \left(A_k b_k - A_{k-1} b_k \right) =  \sum_{k=1}^{n} A_k b_k - \sum_{k=1}^{n} A_{k-1} b_k = \sum_{k=1}^{n} A_k b_k - \sum_{j=0}^{n-1} A_j b_{j+1} = \eqcom{A_0 = 0}  \\ = \sum_{k=1}^{n} A_k b_k - \sum_{j=1}^{n-1} A_j b_{j+1}  = A_n b_n + \sum_{k=1}^{n-1} A_k (b_k - b_{b+1})  .
  \end{multline} 
  Исследуем $ \sum_{k=1}^{\infty} (b_k - b_{k+1}) $. Рассмотрим 
  \begin{gather} 
    \sum_{k=1}^{n} (b_k - b_{k+1}) = b_1 - b_{n+1} \eqcom{\text{по п. 3}}  \xrightarrow[n \to  \infty ]{} b_1 \in \R  .
  \end{gather}  
  Значит $ \sum_{k=1}^{\infty} (b_k - b_{k+1}) $ сходится (в силу монотонности $b_k$ сходится абсолютно). Тогда $ \sum_{k=1}^{\infty} \left| C (b_k - b_{k+1}) \right|  $ сходится. Тогда по первому признаку сравнения $ \sum_{k=1}^{\infty} \left| A_k (b_k - b_{k+1}) \right|  $ сходится, тогда $ \sum_{k=1}^{\infty} A_k (b_k - b_{k+1}) $ сходится. Причём из пп. 1 и 2 $A_k b_k  \xrightarrow[k \to  \infty ]{} 0 $. Значит, переходя к пределу в \eqref{proof:ser:dirichlet} получаем, что искомый ряд сходится.
\end{proof}


\begin{corollary}[Признак Лейбница]
  Пусть $b_k $ монотонно стремится к $0$ при $k \to 0$. Тогда $ \sum_{k=1}^{\infty} (-1)^{k}b_k $ сходится. 
\end{corollary}


\begin{proof}
  Заметим, что $ \left| \sum_{k=1}^{n} (-1)^{k}  \right| \leq 1$. Тогда по признаку Дирихле искомый ряд сходится. 
\end{proof}


\begin{proposition}[Признак Абеля]
  Если 
  \begin{enumerate}
    \item $ \sum_{k=1}^{\infty} a_k $ сходится;
    \item $b_k$ монотонна;
    \item $b_k $ ограниченна,
  \end{enumerate}
  тогда $ \sum_{k=1}^{\infty} a_k b_k  $ сходится.
\end{proposition}


\begin{proof}
  Заметим, что $b_k$ монотонна и ограничена, тогда $\exists \lim_{k \to \infty} b_k = b_0 \in \R$. Заметим из первого пункта, что $a_k$ ограничена. Тогда по признаку Дирихле $ \sum_{k=1}^{\infty} a_k (b_k - b_0)  $ сходится. При этом из п. 1 $ \sum_{k=1}^{\infty} a_k b_0  $ сходится, тогда и $ \sum_{k=1}^{\infty} a_k (b_k - b_0 ) + \sum_{k=1}^{\infty} a_k b_0 = \sum_{k=1}^{\infty} a_k b_k  $ сходится.
\end{proof}


\begin{note}
  И снова, доказательство полностью аналогично интегралам.
\end{note}

\section{Функциональные последовательности}

\begin{definition}
  Будем называть \emph{функциональной последовательностью} $ \{ f_{k}(x) \}_{k = 1}^{\infty} $.
\end{definition}


\begin{definition}
  Будем говорить, что $ f_k(x) $ \emph{поточечно сходится} к $ f(x) $ ( $ f(x)  \xrightarrow[k \to  \infty ]{X} f(x)$) на множестве $ X$, если 
  \begin{gather} 
     \forall x_0 \in X \hence \lim_{k \to \infty } f_k(x_0 ) = f(x_0 ).
  \end{gather}
\end{definition}


\begin{definition}
  Будем говорить, что $ f_k(x) $ \emph{равномерно сходится} к $ f(x) $ ($ f_k(x)  \convergesuniformly{k \to \infty}{X} f(x) $), если 
  \begin{gather} 
    \forall \eps > 0 \exists N \in \N : \forall x \in X \forall k \geq N \Rightarrow \left| f_k(x) - f(x)  \right| < \eps .
  \end{gather} 
\end{definition}


\begin{note}
  Во втором случае номер зависит только от $ \eps$, в первом ещё и от $ x_0 $. Аналогично паре непрерывность~--- равномерная непрерывность.
\end{note}

\begin{proposition}
  Из равномерной непрерывности следует поточечная.
\end{proposition}


\begin{proof}
  Очевидно.
\end{proof}


\begin{proposition}[Критерий равномерной сходимости]
  $ \{ f_{k}(x) \}_{k = 1}^{\infty} $ сходится равномерно на множестве $ X$ к $ f(x) $  \nas $ \lim_{k \to \infty} \sup_{x\in X} \left| f_k(x) - f(x)  \right| = 0$.   
\end{proposition}


\begin{proof}[\circled{\hence}]
    Из равномерной сходимости 
    \begin{gather} 
      \forall \eps > 0 \exists N: \forall x \in X \forall k \geq N \hence \left| f_k(x) - f(x)  \right| < \eps .
    \end{gather}
    Тогда и 
    \begin{gather} 
      \forall \eps > 0 \exists N: \forall k \geq N \hence \sup_{x\in X} \left| f_k(x) - f(x)  \right| \leq \eps .
    \end{gather}
\end{proof}


\begin{proof}[\circled{\lhence}]
  Повторяем в обратном порядке.
\end{proof}


\begin{corollary}
  $ f_k(x)  \convergesuniformly{k \to \infty}{X} f(x)$  \nas $ \exists \{ a_k  \} \subset \R: \lim_{k \to \infty} a_k  = 0$ и $\exists N \in \N: \forall k \geq N \hence \sup_{x\in X} \left| f_k(x) - f(x)  \right| \leq a_k$. 
\end{corollary}

\begin{proof}[\circled{\lhence}]
  Переходя к пределу получаем, что $ \lim_{k \to \infty} \sup_{x\in X} \left| f_k(x) - f(x)  \right| = 0$ \nas  $ f_k(x)  \convergesuniformly{k \to \infty}{X} f(x)$.
\end{proof}

\begin{proof}[\circled{\hence}]
  Из определения равномерной непрерывности 
  \begin{gather} 
    \forall \eps > 0 \exists N: \forall k \geq N \hence \sup_{x\in X} \left| f_k(x) - f(x)  \right| \leq \eps .
  \end{gather}
  Пусть $a_k = 2  \sup_{x\in X} \left| f_k(x) - f(x)  \right|$. Тогда оба условия выполняются.
\end{proof}

\end{document}