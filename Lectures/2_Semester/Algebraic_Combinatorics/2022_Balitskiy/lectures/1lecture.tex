\section{Электрическая аналогия}

\begin{theorem} (Hilbert's student Max Dehn, 1903)
	Если прямоугольник разрезан на квадраты, то отношение его сторон рационально.
\end{theorem}

\begin{proof} (Brooks, Smith, Stone, Tutte, 1940)
	Рассмотрим прямоугольник, который также разрезан на прямоугольники:
	
	\textcolor{red}{Тут должна быть картинка, но пока без неё}
	
	Заметим, что это разбиение можно представить в виде графа, где вертикальные стороны соответствующих прямоугольников связаны друг с другом рёбрами. Более того, можно увидеть следующее утверждение: если взять отрезок прямой, по которой режется исходный прямоугольник, то сумма сторон прилегающих прямоугольников с одной стороны будет равна сумме сторон прилегающих прямоугольников с другой стороны. (На примере с рисунка этот будут 1 = 3 + 4 или 4 + 6 = 3 + 5, где числа - просто номер соответствующего прямоугольника).
	
	Вспомним про электрические цепи. Если положить вертикальные стороны прямоугольников за ток, горизонтальные за напряжение и добавить батарейку (или клеммы), то получим ничто иное как правила Кирхгофа для цепи:
	\begin{itemize}
		\item I правило Кирхгофа: сумма токов, втекающих в узел, равна сумме токов, из него вытекающих.
		
		\textcolor{red}{Сюда надо картиночку с узлом и формулой}
		
		\item II правило Кирхгофа: в любом замкнутом контуре алгебраическая сумма напряжений равна алгебраической сумме ЭДС, действующих в том же контуре.
		
		\textcolor{red}{Сюда надо картиночку с контуром}
	\end{itemize}
	Также известен и \textit{закон Ома для однородного участка цепи}: сила тока в проводнике прямо пропорциональна приложенному напряжению и обратно пропорциональна сопротивлению проводника:
	\[
		I = \frac{U}{R}
	\]
	Согласно нему, мы можем определить <<сопротивления>> для каждого прямоугольника, который участвует в разбиении, как отношения их горизонтальных сторон к вертикальным соответственно.
	
	Пусть мы задали в нашей цепи все $R_i$ и произвольное ЭДС $\mathcal{E}$ и хотим сделать расчёт цепи, то есть найти все токи во всех ветвях (следовательно найдём уже и любую другую характеристику при желании). Почему решение будет определено однозначно? Сделаем наблюдения:
	\begin{enumerate}
		\item Цепь образует планарный граф. Это означает справедливость теоремы Эйлера:
		\[
			V - E + F = 2
		\]
		В нашей цепи ровно $E - 1$ резисторов, $V$ узлов и $F - 1$ элементарных контуров (напомним, что в теореме Эйлера также учитывается всеобъемлющая грань). Если мы напишем относительно токов для всех узлов и элементарных контуров правила Кирхгофа, то получим $(V - 1) + (F - 1) = V + F - 2$ уравнения, причём возникнет ровно $E$ токов --- токи на каждом резисторе + общий ток в цепи, что в точности совпадает с числом уравнений. Таким образом, мы получаем СЛУ с квадратной матрицей:
		\[
			Ax = b,\ \ A \in M_{E \times E};\ x, b \in M_{E \times 1}
		\]
		
		\item Если решение и существует, то оно единственно.
		
		Предположим обратное. Тогда, вычтем одно из другого и получим нетривиальное решение для случая, когда $\mathcal{E} = 0$. Опровергнем его через построение контура, в котором сумма положительных напряжений окажется равной нулю:
		\begin{enumerate}
			\item Выберем любое ребро, где ток не равен нулю и пойдём в вершину, куда он втекает.
			
			\item По первому правилу Кирхгофа, у нас должен быть другой ток, который будет выходить из текущей вершины. Если такого нет, то уже пришли к противоречию. Иначе идём к следующей вершине, к которой ведёт этот ток и так далее.
		\end{enumerate}
		В силу конечности графа, рано или поздно мы либо придём в вершину без выхода, либо найдём цикл. Так как мы всё время шли по направлению тока, то сумма напряжений имеет однозначно определяемый знак. Пришли к заявленному противоречию.
		
		\item Так как мы решали СЛУ над $\R$, то не может случиться такого, что решение существует при $\det A = 0$, ибо иначе найдётся бесконечность других решений. Значит, что $\det A \neq 0$ и мы можем выписать решение явно при помощи метода Крамера:
		\[
			\forall j \in \range{E}\ \ I_j = \frac{\Delta_j}{|A|}
		\]
	\end{enumerate}

	В условиях исходной задачи, все сопротивления равны 1. Если положить $\mathcal{E}$, например, тоже за 1, то получим СЛУ над полем $\Q$. Следовательно, полученные токи будут тоже рациональными, а отсюда уже получим рациональное общее сопротивление, являющееся отношением сторон исходного прямоугольника.
\end{proof}