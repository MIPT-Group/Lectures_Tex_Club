\section*{55*. Формальные степенные ряды. Строгое формальное определение через последовательности. Пример вычисления суммы и произведения рядов}
\par Рассмотрим множество последовательностей $(a_0, a_1, ..., a_n, ...)$ (здесь $a_i$ берутся из множества
чисел $\mathbb{Q}$, $\mathbb{R}$ или $\mathbb{C}$) и введём на них операции сложения и умножения следующим образом:
$$(a_0, a_1, . . .) + (b_0, b_1, . . .) = (a_0 + b_0, a_1 + b_1, . . .);$$
$$(a_0, a_1, . . .) · (b_0, b_1, . . .) = (c_0, c_1, . . .),\mbox{ где } c_n =\sum\limits_{j=0}^n a_j b_{n-j}$$
\par Выражения $a_0 \cdot 1+a_1 \cdot t+a_2 \cdot t^2+. . .+a_n \cdot t^n+. . .$ (с введёнными операциями сложения и умножения) называют \textbf{формальными степенными рядами}. $a_0$ называют свободным членом ряда.
\subsection*{Пример вычисления суммы и произведения рядов (задача 14.3):}
\begin{enumerate}
    \item $F = 1 + t + t^2 + . . ., G = 1 - t + t^2 - t^3 + . . .. $
    $$F + G=(1, 1, ...) + (1, -1, 1, ...) = (2, 0, 2, 0, ...)$$
    $$F \cdot G=(1, 1, ...)(1, -1, 1, ...)=(1, 1-1, 1-1+1, ..., \sum\limits_{i=1}^n g_i, ...)=(1, 0, 1, 0,...) $$
    \item $F =\sum\limits_{k=0}^{\infty} \frac{1}{k!}t^k, \; G =\sum\limits_{k=0}^{\infty} \frac{(-1)^k}{k!}t^k$
    $$F \cdot G: \; c_n=\sum\limits_{k=0}^n \frac{(-1)^k}{k!(n-k)!}=\sum\limits_{k=0}^n \frac{(-1)^k}{n!} C_n^k=\left\{
\begin{array}{ccc}
\frac{1}{n!}=1, \mbox{ если } n=0\\
0, \mbox{ если } n \geq 1\\
\end{array}
\right. \mbox{(из знакопеременной суммы см. билет 32)}$$
$$F \cdot G = (1, 0, 0, ...)=1$$
$$F^2: \; c_n=\sum\limits_{k=0}^n \frac{1}{k!(n-k)!}=\frac{1}{n!} \sum\limits_{k=0}^n C_n^k=\frac{2^n}{n!} \mbox{ (из комбинаторного тождества)}$$
$$F^2=\sum\limits_{k=0}^{\infty} \frac{1}{k!}(2t)^k$$
\end{enumerate}