\setcounter{section}{39}

\section{Операция циклического сдвига на линейных последовательностях. Период линейной последовательности. Свойства периода. Общий вид последовательности периода $d$ и длины $n$.}
\par Пусть дана последовательность символов $a_1 a_2 ... a_n$. Назовём \textbf{циклическим сдвигом} операцию, при которой она переходит в последовательность $a_2 ... a_n a_1$.
\par \textbf{Периодом линейной последовательности} называется минимальное положительное число циклических сдвигов, после применения которых последовательность перейдёт в себя.
\subsection*{Свойства периода:}
\par \textbf{Лемма 1:} Период $d$ слова длины $n$ является делителем $n$
\par $\blacktriangle$ Пусть $n=d \cdot k+r, \; r>0$. Очевидно, что после применения $n$ циклических сдвигов последовательность перейдет в себя. $$\underbrace{a_1...a_n \xrightarrow{d \text{ сдвигов}} a_1...a_n \xrightarrow{d \text{ сдвигов}} ... \xrightarrow{d \text{ сдвигов}} a_1...a_n}_{d \cdot k \text{ сдвигов}}\xrightarrow{r \text{ сдвигов}} a_1...a_n$$
\par $r < d$, так как остаток всегда меньше делителя, и после $r$ сдвигов последовательность перешла в себя $\Rightarrow$ $r$ является периодом - противоречие $\Rightarrow \; n=d \cdot k \; \blacksquare$
\par \textbf{Лемма 2:} если $a_1...a_n$ - слово периода $d$, то оно имеет вид $\underbrace{a_1...a_da_1...a_d...a_1...a_d}_{\frac{n}{d} \text{ блоков}}$
\par $\blacktriangle$ Применим $d$ циклических сдвигов. Тогда $a_{d+1}$ окажется на месте $a_1$ и так далее. Так как $d$ - период, то последовательность перешла в себя, а значит $a_{d+1}=a_1$ и так далее. Дальше по индукции доказываем для остальных блоков. $\blacksquare$

\section{Операция циклического сдвига на линейных последовательностях. Период линейной последовательности. Свойства периода (б/д). Явная формула для числа циклических последовательностей для частных случаев ($n = 3, 4, 5, 6$)}
\begin{enumerate}
    \item Пусть в алфавите $k$ букв. Рассмотрим пример для $n=4$ более подробно так как остальные рассматриваются аналогично. Так как период является делителем $n$, то $d=1,2,4$
\begin{enumerate}
    \item $d=1$: это все последовательности вида $AAAA$. Очевидно, что их $k$ штук
    \item $d=2$: это все последовательности вида $ABAB$, где $A \neq B$. Таким образом количество таких последовательностей равно количеству способов выбрать множество $\{A, B\}$, то есть $\frac{k(k-1)}{2}$ способов
    \item $d=4$: это все остальные последовательности. Посчитаем количество линейных последовательностей с таким периодом. Количество линейных последовательностей длины 4: $k^4$; длины 4 периода 2: $k(k-1)$; длины 4 периода 1: $k$. Таким образом, количество линейных последовательностей длины 4 периода 4: $k^4-k^2+k-k=k^4-k^2$. Так как 4 линейным последовательностям длины 4 периода 4 соответствует только одна циклическая последовательность длины 4 периода 4, то количество таких циклических последовательностей: $\frac{r^4-r^2}{4}$
\end{enumerate}
\par Всего циклических последовательностей длины 4: $k+\frac{k(k-1)}{2}+\frac{r^4-r^2}{4}=\frac{k^4+k^2+2k}{4}$
\item $n=3$
\begin{enumerate}
    \item $d=1$: $k$ штук
    \item $d=3$: Всего линейных: $k^3$; линейных периода 1: $k \Rightarrow$ циклических периода 3: $\frac{k^3-k}{3}$.
\end{enumerate}
\par Всего циклических последовательностей длины 3: $k+\frac{k^3-k}{3}=\frac{k^3+2k}{3}$

\item $n=5$
\begin{enumerate}
    \item $d=1$: $k$ штук
    \item $d=5$: Всего линейных: $k^5$; линейных периода 1: $k \Rightarrow$ циклических периода 5: $\frac{k^5-k}{5}$.
\end{enumerate}
\par Всего циклических последовательностей длины 5: $k+\frac{k^5-k}{5}=\frac{k^5+4k}{5}$
\item $n=6$
\begin{enumerate}
    \item $d=1$: $k$ штук
    \item $d=2$: $C_k^2=\frac{k(k-1)}{2}$
    \item $d=3$: количество способов составить блок из 3-х букв: $k^3$, но нам нужно исключить слова периода 1, поэтому получаем $k^3-k$. Так как нас интересуют циклические слова, то их количество равно $\frac{k^3-k}{3}$
    \item $d=6$: это все остальные последовательности. Всего линейных: $k^6$; линейных периода 1: $k$; линейных периода 2: $k(k-1)$; линейных периода 3: $k^3-k$ $\Rightarrow$ циклических периода 6: $\frac{k^6-k^3-k^2+k}{6}$.
\end{enumerate}
\par Всего циклических последовательностей длины 6: $k+\frac{k(k-1)}{2}+\frac{k^3-k}{3}+\frac{k^6-k^3-k^2+k}{6}=\frac{k^6+k^3+2k^2+2k}{6}$
\end{enumerate}