\setcounter{section}{58}

\section{Числа Каталана (определение через пары скобочных последовательностей). Рекуррентное соотношение для чисел Каталана.}

$T_n$ - число правильных скобочных последовательностей с 2n скобками (n открывающих и n закрывающих).
$T_0$ = 1; $T_n = T_{n-1}T_0 + T_{n-2}T_1 + \dots + T_0T_{n-1}$.
Докажем, что это так. Очевидно, что любая правильная скобочная последовательность начинается с открывающей скобки. Между ней и соответствующей ей закрывающей скобкой можно расположить правильную последовательность из 2k скобок (где $0\leqslant k<n$). Отсюда и получается это рекуррентное соотношение.

\section{Числа Каталана. Производящая функция для чисел Каталана.}
Итак, числа Каталана имеют вид:
\[ C_n = \left\{
  \begin{array}{ccc}
    1, n = 0 \\
    \sum_{k=0}^{n-1} C_k C_{n-k-1}\\
  \end{array}
\right. \]
Ищем производящую функцию в виде: $G(z) = \sum_{n=0}^{\infty}C_n z^n$. В реккурентном соотношении домножаем $C_n$ на $z^n$: $z^0 C_0 = z^0 = 1$, $z_n C_n = z^n \sum_{k=0}^{n-1} C_k C_{n-k-1}$. Выполним суммирование по всем n: $G(z) = 1 + \sum_{n=1}^{\infty}z^n \sum_{k=0}^{n-1} C_k C_{n-k-1}$. Заметим, что $G^2(z) = T_0 + (T_0T_1 + T_1T_0)z + \dots + (T_0T_n + \dots + T_nT_0)z^n = T + T_2z + \dots + T_{n+1}z^n + \dots$. Отсюда верно: $G(z) = 1 + z*G^2(z)$. Решая квадратное уравнение и проверяя, подходит ли при z = 0 коэффициент (надо не подставлять z=0, а переходить к пределу), получаем, что нам подходит лишь один корень: $G(z) = \frac{1 - \sqrt{1-4z}}{2z}$

\section{Числа Каталана. Формула для коэффициентов ряда $\sqrt{1+x}$}
$\sqrt{1+x} = (1+x)^{1/2} = 1 + C_{1/2}^1x + C_{1/2}^2x^2 + \dots + C_{1/2}^kx^k + \dots$; $C_a^k = \frac{a!}{k!(a-k)!} = \frac{(a)(a-1)\dots(a-k+1)}{k!}$ \\
$C_{\alpha + \beta}^n = \sum_{j=0}^n C_{\alpha}^j C_{\beta}^{n-j}$. 
$\blacktriangle$
Введём понятие убывающего n-ого факториала: $a^{\underline{n}} = (a)(a-1)\dots(a-n+1)$. Тогда наше равенство перепишется в виде: $\frac{(\alpha + \beta)^{\underline{n}}}{n!} = \sum_{j=0}^n \frac{\alpha^{\underline{j}}}{j!} \frac{\beta^{\underline{n-j}}}{(n-j)!} $. Домножая обе части равенства на $n!$, получаем: $(\alpha + \beta)^{\underline{n}} = \sum_{j=0}^n \alpha^{\underline{j}} \beta^{\underline{n-j}} C_n^j$. Последняя формула доказывается по индукции.
Пусть это уже известно, тогда надо доказать, что $(\alpha + \beta)^{\underline{n+1}} = \sum_{j=0}^{n+1} \alpha^{\underline{j}} \beta^{\underline{n-j+1}} C_{n+1}^j$. $(\alpha + \beta)^{\underline{n+1}} = (\alpha + \beta)^{\underline{n}}*(\alpha + \beta - n)$. Тогда изначальное равенство нам следует лишь домножить на $(\alpha + \beta - n)$: $\sum_{j=0}^n \alpha^{\underline{j}} \beta^{\underline{n-j}} C_n^j (\alpha + \beta - n) = \sum_{j=0}^n \alpha^{\underline{j}} \beta^{\underline{n-j}} C_n^j (\alpha - j + \beta - n + j) = \sum_{j=0}^n \alpha^{\underline{j+1}}\beta^{\underline{n-j}} C_n^j + \sum_{j=0}^n \alpha^{\underline{j}} \beta^{\underline{n-j+1}} C_n^j$. Поменяем счётчики и применим тождество: $\sum_{j=-1}^{n-1} \alpha^{\underline{j}}\beta^{\underline{n-j+1}} C_n^{j-1} + \sum_{j=0}^n \alpha^{\underline{j}} \beta^{\underline{n-j+1}} C_n^j = \sum_{j=1}^n \alpha^{\underline{j}}\beta^{\underline{n-j+1}} C_{n+1}^j + \alpha^{\underline{0}}\beta^{\underline{n+1}} C_{n}^0 + \alpha^{\underline{n+1}}\beta^{\underline{0}} C_{n}^{n} = \sum_{j=1}^n \alpha^{\underline{j}}\beta^{\underline{n-j+1}} C_{n+1}^j + \alpha^{\underline{0}}\beta^{\underline{n+1}} C_{n+1}^0 + \alpha^{\underline{n+1}}\beta^{\underline{0}} C_{n+1}^{n+1} = \sum_{j=0}^{n+1} \alpha^{\underline{j}}\beta^{\underline{n-j+1}} C_{n+1}^j$. 
$\blacksquare$ \par
Коэффициент при $x^n$ по определению равен $C_{\frac12}^n = \frac12\cdot(-\frac12)\cdot(-\frac32)\cdot\ldots\cdot\frac{3-2n}2\cdot\frac1{n!} = \\ = \frac{(-1)^{n-1}\cdot1\cdot3\cdot\ldots\cdot(2n-3)}{2^n\cdot n!} = \frac{(-1)^{n-1}\cdot(2n-2)!}{2^{2n-1}\cdot n!\cdot (n-1)!} = \frac{(-1)^{n-1}\cdot(2n-2)!\cdot(2n-1)\cdot(2n)}{2^{2n}\cdot (n!)^2 \cdot(2n-1)} = \frac{(-1)^n\cdot(2n)!}{4^n\cdot (n!)^2 \cdot(1-2n)} $

\section{Числа Каталана. Формула для коэффициентов ряда $\sqrt{1+x}$ (б/д). Вывод из неё формулы для чисел Каталана.}
$\sqrt{1-4x} = \sqrt{1+(-4x)} = 1 + C_{1/2}^1(-4x) + \dots + C_{1/2}^n(-4x)^n + \dots = 1 + (-4x) + \dots + (1/2)*(-1/2)*(-3/2)*\dots*((3-2n)/2)(-4x)^n/n! = 1 -4x + \dots + (-1)^{n-1}*2^{-n}*1*3*\dots*(2n-3)*2*4*\dots*(2n-2)(-4x)^n/(n!*2*4*\dots*(2n-2)) = 1 -4x + \dots + (-1)^{n-1}*2^{-n}*2^{1-n}(2n-2)!(-4x)^n/(n!(n-1)!)$. Рассмотрим $1-\sqrt{1-4x} = 4x + \dots - (-1)^n(-1)^{n-1}*2^{1-2n}*2^{2n}(2n-2)!x^n/(n!(n-1)!) = 4x + \dots + 2C_{2n-2}^{n-1}x^n/n$; тогда $(1-\sqrt{1-4x})/2x = 2 + \dots + C_{2n-2}^{n-1}x^{n-1}/n $; но коэффициент при $x^{n-1}$ - $T_{n-1}$ - n-1-ое число Каталана. Отсюда $T_n = C_{2n}^n/(n+1)$