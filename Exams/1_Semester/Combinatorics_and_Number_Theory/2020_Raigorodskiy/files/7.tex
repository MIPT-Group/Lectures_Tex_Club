\setcounter{section}{6}

\section{Понятия образа и прообраза множества при соответствии. Критерий равенства образа пересечения и пересечения образов. Аналогичные критерии с объединением и разностью.} 
\textbf{Соответствие} между множествами A и B - произвольное подмножество декартова произведения $F \subset A \times B$. Обозначение: $F: A \to B$; иногда, чтобы подчеркнуть, что одному элементу из A может соответствовать несколько элементов из B, пишут: $F:A\rightrightarrows B$. 
\\
\par
Пусть $F: A \to B$ - соответствие, $S \subset A$, $T \subset B$. Тогда \textbf{образ} множества S - множество всех элементов B, соответствующих какому-то элементу S. Формально: $F(S) = \bigcup_{s \in S} F(s) \subset B $. \textbf{Прообраз} множества Т - множество элементов А, которым соответсвует хотя бы один элемент Т. Формально: $F^{-1}(T) = \left\{ a: F(a) \cap T \neq \varnothing \right\}$ 
\\
\par
\emph{ Образ пересечения любых двух множеств равняется пересечению образов тех же множеств $\iff$ соответствие инъективно. } 
\\
$\blacktriangle$
Пусть соответствие не инъективно. Тогда найдутся такие $a_1$ и $a_2$, что $F(a_1)$ и $F(a_2)$ пересекаются. Тогда $F(\left\{a_1\right\} \cap \left\{a_1\right\}) = F(\varnothing) = \varnothing$, но $F(\left\{a_1\right\}) \cap F(\left\{a_2\right\}) = F(a_1) \cap F(a_2)$ не пуст по предположению. Значит, образ пересечения множеств $\left\{ a_1\right\}$ и $\left\{ a_2\right\}$ не равен пересечению образов. \par
Теперь пусть соответствие инъективно. Рассмотрим произвольные подмножества S и Q множества А. Докажем, что $F(S \cap Q) = F(S) \cap F(Q)$. Для этого докажем включение в обе стороны. Вначале пусть $y \in F(S \cap Q)$. Это значит, что $y \in F(x)$ для некоторого $x \in S \cap Q$. Тогда $x \in S$ и $x \in Q$. А раз $y \in F(x)$, то $y \in F(S)$ и $y \in F(Q)$. Значит, $y \in F(S) \cap F(Q)$. (Это включение верно для всех соответствий). \par
Теперь пусть $y \in F(S) \cap F(Q)$. Значит, $y \in F(x_1)$ для некоторого $x_1 \in S$ и $y \in F(x_2)$ для некоторого $x_2 \in Q$. Но при $x_1 \neq x_2$ в силу инъективности множества $F(x_1)$ и $F(x_2)$ не пересекаются. А их пересечение содержит хотя бы y. Значит, $x_1 = x_2 = x$, и $x \in S \cap Q$. А так как $y \in F(x)$, то получаем $y \in F(S \cap Q)$.
$\blacksquare$ \\
\par
\emph{ Образ объединения любых двух множеств равняется объединению образов тех же множеств - выполняется для любых соответствий. } \\
$\blacktriangle$
1) $\left.
  \begin{array}{ccc}
    A \subseteq A \cup B \Rightarrow F(A) \subseteq F(A \cup B) \\
    B \subseteq A \cup B \Rightarrow F(B) \subseteq F(A \cup B) \\
  \end{array}
\right\} \Rightarrow F(A) \cup F(B) \subseteq F(A \cup B) $ \\
2) $y \in F(A \cup B) \Rightarrow \exists x \in A \cup B: y = F(x) \Rightarrow \exists x \in A \cup x \in B: y = F(x) \Rightarrow$ \\
$\Rightarrow y \in F(A) \cup y \in F(B) \Rightarrow y \in F(A) \cup F(B) \Rightarrow  F(A \cup B) \subseteq F(A) \cup F(B) $; \\
Из пунктов 1 и 2 следует, что $F(A \cup B) = F(A) \cup F(B) $
$\blacksquare$ \\
\par
\emph{ Образ разности любых двух множеств равняется разности образов тех же множеств $\iff$ соответствие инъективно. } \\
Доказательство аналогично доказательству для пересечения.