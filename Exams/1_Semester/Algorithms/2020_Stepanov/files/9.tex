\setcounter{section}{8}
\section{Проверка правильности скобочной последовательности с несколькими типами скобок.}
\par \textbf{Определение}: \textit{Правильные скобочные последовательности с несколькими типами скобок} (рассмотрим с двумя)
\begin{enumerate}
    \item $\varepsilon$ (пустое слово) - ПСП
    \item $S$ - ПСП $\Rightarrow (S), [S]$ - ПСП
    \item $S_1, S_2$ - ПСП $\Rightarrow S_1 S_2$ - ПСП 
\end{enumerate}
\par \textbf{Задача:} Проверить, является ли последовательность из нескольких типов скобок правильной скобочной последовательностью
\par \textbf{Решение:} Храним стек незакрытых открывающих скобок
\lstinputlisting[language=C++,
emph={int,char,double,float,unsigned},
emphstyle={\color{blue}}
]{code/9_psp.cpp}
\par \textbf{Утверждение:} Данный алгоритм корректен, то есть ПСП $\Leftrightarrow$ алгоритм вывел true
\begin{itemize}
    \item[$\blacktriangle \Rightarrow$] Индукция по построению \begin{enumerate}
    \item База: $\varepsilon$ - обработается корректно
    \item $T=(u)$, $u$ - ПСП. По предположению индукции, всё $u$ удалится из стека к моменту прихода закрывающей скобки (можно считать, что стек начинается после первой скобки, она никак не влияет на применение алгоритма к $u$), а внешние скобки обработаются корректно (аналогично для других типов скобок)
    \item $T=T_1 T_2; T_1, T_2$ - ПСП. По предположению индукции, после того как считается $T_1$ стек опустошится $\Rightarrow$ после $T_2$ - тоже $\Rightarrow$ алгоритм сработает корректно $\blacksquare$
    \end{enumerate}
    \item[$\Leftarrow$] Доказываем индукцией по количеству действий, "обращая" предыдущий пункт.
    \par Рассмотрим скобочную последовательность, на которую алгоритм выдаёт true. Алгоритм сопоставил каждой открывающейся скобке одного типа закрывающуюся скобку того же типа. Причём они обязательно идут в правильном порядке. 
\parДля доказательства факта используем индукцию.
Рассмотрим пары соответсвующих скобок в порядке закрытия пары:
\begin{enumerate}
    \item База: Если между парой скобок нет других скобок, то последовательность от одной скобки до другой - правильная.
    \item Шаг: Если между парой скобок (назовём их $a$ и $b$ соответственно) есть непустая подстрока, то все скобки из подстроки уже были рассмотрены индукцией, так как открывающиеся скобки в подстроке были позже, чем $a$ добавлены в стек и по правилу стека должны были раньше из него выйти, а значит они уже были рассмотрены индукцией. Аналогично с закрывающимеся скобками из подстроки - они идут раньше, чем $b$, следовательно, по правилу стека им ставили в соответствие открывающиеся скобки, которые были добавленны позже $a$. (окрывающаяся скобка не могла быть добавленны раньше $a$, потому что $a$ перегородила бы ей выход.) По предположению индукции получаем, что подстрока состоит из одной или нескольких ПСП $\Rightarrow$ сама подстрока ПСП. $\Rightarrow$ Подпоследовательность от $a$ до $b$ - правильная. $\blacksquare$
\end{enumerate}
\end{itemize} 