\section*{11. Возведение множества в степень. Возведение множества в степень другого множества. Булеан. Свойства возведения множества в степень: $A^B \times A^C \cong A^{B \cup C} \mbox{ (для непересекающихся $B$ и $C$)}; A^C \times B^C \cong (A \times B)^C; (A^B)^C \cong A^{B \times C} \mbox{(для произвольных $A$, $B$, $C$)}$.}
\par  \textbf{Определение: }\textit{Декартовой степенью} $A^n$ множества $A$ называется множество кортежей длины $n$ из элементов $A$.
\par \textbf{Определение: }Пусть $A$ и $B$ — два множества. Тогда множеством $B^A$ называется множество всех отображений из $A$ в $B$
\par \textbf{Смысл определения:} Если множество $A$ состоит из $n$ элементов, а множество $B$ — из $k$ элементов, то существует всего $k^n$ различных отображений из $A$ в $B$: действительно, есть по $k$ вариантов значения для каждого из $n$ элементов A.
\par \textbf{Определение: }\textit{Булеаном} множества $A$ называется множество всех подмножеств
множества $A$. Обозначение: $\mathcal{P}(A)$ или $2^A$.
\subsection*{Свойства возведения множества в степень:}
\begin{enumerate}
    \item $A^B \times A^C \cong A^{B \cup C} \mbox{ (для непересекающихся $B$ и $C$)}$
    \par $\blacktriangle$ Неформально это утверждение означает, что определить функцию на несвязном объединении двух множеств это то же самое, что определить её на каждом из этих множеств по отдельности. Метафорически элемент $A^{B \cup C}$ есть набор контейнеров, помеченных элементами $B$ или $C$. А $A^B \times A^C$ — это два набора контейнеров, в первом они помечены элементами $B$, а во втором — элементами $C$. Таким образом, достаточно один набор разбить на два. \par Формально множество с левой стороны имеет вид $\{ (f, g)\; | \; f: B \rightarrow A;\; g: C \rightarrow A \}$, а с правой - $\{h \;|\;h: B \cup C \rightarrow A \}$. Рассмотрим такое соответствие между этими множествами: $h \longmapsto (h|_B, h|_C)$. Его инъективность очевидна. Оно также является сюръективным так как $\forall (f, g) \in A^B \times A^C$ \;
    $\exists h \in A^{B \cup C}:$ 
    \\
    $h(x) = \left\{
\begin{array}{ccc}
f(x), \mbox{ если } x \in B\\
g(x), \mbox{ если } x \in C\\
\end{array}
\right. $. Таким образом равномощность доказана.
    $\blacksquare$
    \item $A^C \times B^C \cong (A \times B)^C$
    \par $\blacktriangle$ Неформально утверждение означает, что пара функций это то же самое, что одна функция, принимающая значение среди пар. Это легко объяснить при помощи метафоры с контейнерами. Пара функций — это два набора контейнеров, индексированных элементами $C$. В каждом контейнере из первого набора лежит элемент $A$, а в каждом контейнере из второго набора — элемент $B$. Если переложить все элементы из контейнеров второго набора в соответствующие контейнеры из первого набора, то получится набор контейнеров с парами элементов $A$ и $B$. 
    \par Формально пусть $F \in (A \times B)^C$. Это значит, что $F : C \rightarrow A \times B$. То есть каждому элементу $c \in C$ сопоставлена некоторая пара $(a, b) \in A \times B$. Вместо этого ему можно сопоставить отдельно элементы $a \in A$ и $b \in B$. Получится два отображения, первое отображает $c$ в $a$, а второе — $c$ в $b$, то есть пара отображений $(F_1, F_2) \in A^C \times B^C$. Можно сказать, что $F_i = pr_i \circ F$, где $pr_i$ — проектор на $i$-ю координату: $pr_i(a_1, a_2) = a_i$. Легко понять, что разные $F$ переводятся в разные пары $(F_1, F_2)$, и каждая пара получается из некоторой функции. Таким образом, эквивалентность установлена. $\blacksquare$
    \item $(A^B)^C \cong A^{B \times C} \mbox{ (для произвольных $A$, $B$, $C$)}$
    \par $\blacktriangle$ Третья эквивалентность означает, что функция двух аргументов есть то же самое, что отображение первого аргумента в функцию, зависящую от второго аргумента. Метафорически есть набор контейнеров, помеченных элементами $C$, в каждом из которых лежат контейнеры, помеченные элементами $B$, а уже в каждом из маленьких контейнеров лежит элемент $A$. Если убрать внешние контейнеры, но при этом перенести с них метки на внутренние, то получится набор контейнеров, помеченных элементами $B \times C$, что и требовалось. 
    \par Формально пусть $F \in (A^B)^C$, $F: C \rightarrow A^B$, то есть каждому элементу $c \in C$ сопосотавляется некоторая функция $G: B \rightarrow A$. Пусть $H \in A^{B \times C}$, $H: B \times C \rightarrow A$. Рассмотрим такое соответствие между этими множествами: $(F(x))(y) \longmapsto H(x, y)$. Нетрудно понять, что данное соответствие - это биекция $\blacksquare$
\end{enumerate}