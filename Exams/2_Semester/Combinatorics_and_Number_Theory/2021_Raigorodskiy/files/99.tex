\setcounter{section}{98}
\section{(15 на отл). Решетки в пространствах. Базис и определитель. Многомерная теорема Минковского (для произвольной решетки).}

\textbf{Опр} Пусть $( e_1, . . . , e_k)$ — набор линейно независимых векторов в $\mathbb{R}^n$. Тогда дискретная абелева группа в $\mathbb{R}^n$, порождённая $\{e_i\}$, называется решёткой, а набор $( e_1, . . . , e_k)$ называется базисом решётки. Иными словами, решётка есть множество $\Lambda = \{a_1 e_1 + ... + a_k e_k\}, a_i \in \mathbb{Z}$ Число $\det \Lambda = |\det(e_1^T, \dots, e_n^T)|$ (объём одной "ячейки", на которую разбивает пространство решётка (эта ячейка называется \textbf{фундаментальной областью}) - \textbf{определитель решётки}. \par
\textbf{Утверждение}. Определитель решётки не зависит от базиса в $\mathbb{R}^n$. Это результат свойств определителя.

\textbf{Многомерная теорема Минковского (для произвольной решетки)}. Пусть $\Omega \subset \mathbb{R}^n$ - выпуклое ($\forall x, y \in \Omega$ отрезок xy целиком лежит в $\Omega$), центрально-симметричное (симметрична относительно начала координат, т.к. можно выбрать систему координат, в которой центр - начало координат), измеримо, Vol$\Omega > 2^n \det \Lambda$. Тогда $\Omega \cap \Lambda \backslash \{0\} \neq \varnothing$. \par
$\blacktriangle$ \par
\includegraphics[width=8cm]{grid} \par
Возьмём некоторый p: рассмотрим пересечение $\Omega$ и решётки $\frac{1}{p}\Lambda$ и обозначим $N_p$ мощность пересечения. Поскольку $\Omega$ - измеримое, то его объём можно сколь угодно близко приблитзить значением $N_p\det(\frac{1}{p}\Lambda)$, т.е.: \[
\frac{N_p}{p^n}\det\Lambda \to_{p \to \infty} Vol \Omega > 2^n \det \Lambda
\]
Более того, для достаточно большого p выполнено $N_p > (2p)^n$. рассмотрим две произвольные точки из этого пересечения: $a = \frac{a_1}{p}e_1 + \dots + \frac{a_n}{p}e_n$ и $b = \frac{b_1}{p}e_1 + \dots + \frac{b_n}{p}e_n$. Поскольку $N_p > (2p)^n$, то по принципу Дирихле можно выбрать такие различные точки, что $a_i = b_i (mod 2p)$. Тогда в силу выпуклости и симметричности $\Omega$ точка $\frac{a - b}{2}$ лежит в $\Omega \cap \Lambda$ и не совпадает с началом координат в силу различности a и b.
$\blacksquare$