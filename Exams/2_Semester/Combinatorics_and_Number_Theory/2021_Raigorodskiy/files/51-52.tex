\setcounter{section}{50}
\section{Сравнения второй степени. Квадратичные вычеты и невычеты. Тождество для $\left(\frac{a}{p}\right)$}

\begin{definition}
    $x^2 \equiv a \md{m}$ называется \textit{сравнением второй степени}.
\end{definition}
Будем считать, что $m = p$ -- нечётное простое число, $(a, p) = 1$.
\begin{note}
    У сравнения второй степени либо нет решений, либо их два.
    \begin{proof}[$\blacktriangle$]
        По теореме Лагранжа у сравнения второй степени не более 2.\\
        Пусть $x_0$ -- решение сравнения $x^2 \equiv a \md{p}$.\\
        Тогда $-x_0$ -- также решение, но $-x_0 \not{\equiv} x_0 \md{p}$
    \end{proof}
\end{note}

\begin{definition}
    Число $a$ называется \textit{квадратичным вычетом}, если у сравнения $x^2 \equiv a \md{p}$ два решения.
    Число $a$ называется \textit{квадратичным невычетом}, если у сравнения $x^2 \equiv a \md{p}$ нет решений.
\end{definition}

\begin{proposition}
    По модулю $p$ есть ровно $\frac{p - 1}{2}$ квадратичных вычетов и $\frac{p - 1}{2}$ квадратичных невычетов.
\end{proposition}

    Утверждение было доказано в билете 12 (на "уд.").

\begin{theorem}
     $a^{ \frac{p - 1}{2} } \equiv 1 \md{p}$, если $a$ -- квадратичный вычет, и $a^{ \frac{p - 1}{2} } \equiv -1 \md{p}$, если $a$ -- квадратичный невычет.
    \begin{proof}[$\blacktriangle$]
        По малой теореме Ферма $a^{p-1} \equiv 1 \md{p}$ для всех $a$. Тогда
        $$
            (a^{ \frac{p - 1}{2} } - 1)(a^{ \frac{p - 1}{2} } + 1) \equiv 0 \md{p}
        $$
        Если $a$ -- квадратичный вычет, то 
        $$
        \exists x : x^2 \equiv a \md{p} \Rightarrow a^{ \frac{p - 1}{2} } \equiv (x^2)^{ \frac{p - 1}{2} } \equiv x^{p - 1} \equiv 1 \md{p}
        $$
        Доказательство для квадратичных невычетов аналогичное.
    \end{proof}
\end{theorem}

\begin{definition}
    \textit{Символ Лежандра}  
    $$
    \left(\frac{a}{p}\right) = 
    \begin{cases}
        1, \text{если } a \text{ -- квадратичный вычет}\\
        -1, \text{если } a \text{ -- квадратичный невычет}\\
        0, \text{если } (a, p) \neq 1\\
    \end{cases}
    $$
\end{definition}

\begin{note}
    $$
        \left(\frac{a}{p}\right) \equiv a^{ \frac{p - 1}{2} }
    $$
\end{note}

\begin{corollary}
    Символ Лежандра мультипликативен.
\end{corollary}

\begin{theorem}
    $\left(\frac{a}{p}\right) = (-1)^{ \sum \limits_{x=1}^{ \frac{p-1}{2} } \left[ \frac{2ax}{p} \right] }$
    \begin{proof}[$\blacktriangle$]
        Пусть $x \in \{1, 2, ..., \frac{p - 1}{2}\}$\\
        $a \cdot x$ загоним в систему вычетов от $-\frac{p-1}{2}$ до $\frac{p-1}{2}$.\\
        
        Переход в новую систему вычетов происходит следующим образом: левая часть системы вычетов ${1, 2, ..., p - 1}$ остаётся такой же (то есть равна $\{1, 2, ..., \frac{p - 1}{2}\}$), а правая будет равна $\{-\frac{p - 1}{2}, ..., -2, -1\}$.\\
        
        $$
            a \cdot x \equiv \varepsilon_x \cdot r_x \md{p},
        $$ где $\varepsilon_x \in \{-1, 1\}, r_x \in \{1, 2, ..., \frac{p-1}{2}\}$\\
        
        Если $a \cdot x$ попадет в левую часть системы вычетов \{1, 2, ... p - 1\}, тогда $\varepsilon_x = 1$, если в правую, то $\varepsilon_x = -1$.\\
        
        Утверждается, что математически это записывается так:
        $$
            \varepsilon_x = (-1)^{\left[ \frac{2ax}{p} \right]}
        $$
        
        Доказательство настолько скучное, что Райгородский не стал его рассказывать :) Доказать можно примерно так:\\
        
        Пусть $a \cdot x \in [kp + 1; (k + 1)p - 1]$ для некоторого $k$. Тогда $\frac{ax}{p} \in (k, k + 1)$. Соответственно, $\frac{2ax}{p} \in (2k, 2k + 2)$.\\
        
        Тогда если $ax$ лежало в левой части, то $\left[ \frac{2ax}{p} \right] = 2k$ (то есть чётному числу), иначе $\left[ \frac{2ax}{p} \right] = 2k + 1$ (то есть нечётному).
        
        Покажем, что когда $x$ пробегает значения от $1$ до $\frac{p-1}{2}$, то $r_x$ пробегает всю систему вычитов $1, 2, ..., \frac{p-1}{2}$:
        
        Когда $x$ пробегает значения от $1$ до $\frac{p-1}{2}$, то значения $ax$ по модулю $m$ не могут повториться, поскольку иначе $(a, p) \neq 1$. Также значения $r_x$ не могут повториться, так как иначе $\exists x_1, x_2 \leq \frac{p-1}{2}$ т.ч. $x_1 \neq x_2, ax_1 \equiv_m r_x, ax_2 \equiv_m -r_x \Rightarrow ax_1 + ax_2 \equiv_m 0 \Rightarrow x_1 + x_2 \equiv_m 0$, чего быть не может, поскольку $x$ пробегает значения от $1$ до $\frac{p-1}{2}$. Следовательно, когда $x$ пробегает значения от $1$ до $\frac{p-1}{2}$, то $r_x$ пробегает всю систему вычетов $1, 2, ..., \frac{p-1}{2}$
        
        С учетом этого,
        $$
        \prod \limits_{x = 1}^{\frac{p-1}{2}} (ax) \equiv \varepsilon_1 \cdot \varepsilon_2 \cdot ... \cdot \varepsilon_{\frac{p-1}{2}} \cdot \prod \limits_{x = 1}^{\frac{p - 1}{2}} r_x \equiv \varepsilon_1 \cdot \varepsilon_2 \cdot ... \cdot \varepsilon_{\frac{p-1}{2}} \cdot \prod \limits_{x = 1}^{\frac{p - 1}{2}} x
        $$
        Разделив обе части на $\prod \limits_{x = 1}^{\frac{p - 1}{2}} x$ и используя выражение для $\varepsilon_x$, получаем:
        $$
            \left(\frac{a}{p}\right) \equiv a^{\frac{p-1}{2}} \equiv \varepsilon_1 \cdot \varepsilon_2 \cdot ... \cdot \varepsilon_{\frac{p-1}{2}} \equiv (-1)^{\sum \limits_{x = 1}^{\frac{p-1}{2}} \left[ \frac{2ax}{p} \right] }
        $$
    \end{proof}
\end{theorem}


\section{Сравнения второй степени. Квадратичные вычеты и невычеты. Формула для \texorpdfstring{$\left(\frac{2}{p}\right)$}{(p/2)} (тождеством с суммой \texorpdfstring{по $\left[ \frac{2ax}{p} \right]$}{} можно пользоваться без доказательства).}

Вся теория расписана в прошлом билете.\\

\begin{theorem}
    $$
        \left(\frac{2}{p}\right) = (-1)^{\frac{p^2 - 1}{8}}
    $$
    \begin{proof}
        Для удобства введём обозначение $p_1 = \frac{p - 1}{2}$.\\
        
        Без доказательства можно пользоваться утверждением:\\
        $$
            \left(\frac{a}{p}\right) = (-1)^{ \sum \limits_{x=1}^{ p_1 } \left[ \frac{2ax}{p} \right] }
        $$
        
        Рассмотрим нечётное $a$.\\
        $$
            \left(\frac{2a}{p}\right) = \left(\frac{4 \cdot \frac{a + p}{2}}{p}\right) = \left(\frac{2^2}{p}\right) \cdot \left(\frac{ \frac{a + p}{2} }{p}\right) = 1 \cdot (-1) ^ {\sum \limits_{x=1}^{ p_1 } \left[ \frac{2 \cdot \frac{a + p}{2} \cdot x}{p} \right] }
        $$
        Для удобства распишу отдельно показатель $-1$:\\
        $$
            \sum \limits_{x=1}^{ p_1 } \left[ \frac{2 \cdot \frac{a + p}{2} \cdot x}{p} \right] = \sum \limits_{x=1}^{p_1} \left[ \frac{ax}{p} + x \right] = \sum \limits_{x=1}^{p_1} \left[ \frac{ax}{p} \right] + \sum \limits_{x=1}^{p_1} x = \sum \limits_{x=1}^{p_1} \left[ \frac{ax}{p} \right] + \frac{p_1(p_1 + 1)}{2} = \sum \limits_{x=1}^{p_1} \left[ \frac{ax}{p} \right] + \frac{p^2 - 1}{8}
        $$
        Вернёмся к $\left(\frac{2a}{p}\right)$:
        $$
            \frac{2a}{p} = (-1)^{ \sum \limits_{x=1}^{p_1} \left[ \frac{ax}{p} \right] } \cdot (-1)^{ \frac{p^2 - 1}{8} }
        $$
        Тождество верно для любого нечётного $a$, поэтому можно подставить $a = 1$.
        $$
            \left(\frac{2}{p}\right) = (-1)^{ \frac{p^2 - 1}{8} } \cdot (-1)^{ \sum \limits_{x=1}^{p_1} \left[ \frac{x}{p} \right] } = (-1)^{ \frac{p^2 - 1}{8} },
        $$
        так как $\left[ \frac{x}{p} \right] = 0$ ($x \leq p_1 < p$).
    \end{proof}
\end{theorem}