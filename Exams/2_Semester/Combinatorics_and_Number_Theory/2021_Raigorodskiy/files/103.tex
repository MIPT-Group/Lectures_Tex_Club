\setcounter{section}{102}
\section{Алгоритм AKS. Верхняя оценка на $r$: вывод из утверждения о нижней оценке \texorpdfstring{$[1, 2, \ldots , n]$}{[1, 2, ..., n]}.}
\par \textbf{Лемма:} $r \leq \max\{3, \lceil \log_2^5 n \rceil\}$
\par $\blacktriangle$ Пусть $n \geq 3 \Rightarrow B=\lceil\log_2^5 n\rceil\geq 10>7 \Rightarrow$ можем применять оценку на $[1, \ldots, B]$ из билета 80, то есть $[1, \ldots, B] \geq 2^B$
\par Рассмотрим $$S=n^{\left[\log_2 B\right]}\prod_{i=1}^{\left[ \log_2^2 n \right]}(n^i-1)$$
\par Возьмем минимальное $r$, такое что $r$ не делит $S \Rightarrow n^i \not\equiv 1 \: (\md \: r) \: i =1, \ldots, [\log_2^2 n] \Rightarrow$ если $(r,n)=1$, то $\ord_r n > \log_2^2 n$.
\par Осталось доказать, что $(r,n)=1$ и $r \leq B$. Воспользуемся тем, что $n^i-1<n^i$ и просуммируем степени по арифметической прогрессии.
$$S < n^{\left[\log_2 B\right]}\cdot n^\frac{\left[\log_2^2 n\right]\left(\left[\log_2^2 n\right]+1\right)}{2}\leq n^{\log_2^4n}=2^{\log_2^5 n} \leq 2^B$$
\par Во втором неравенстве мы прибавили $\frac{\log_2^4 n}{2}$ и отняли $[\log_2 B]=[\log_2 \log_2^5 n]$. Очевидно, что второе является двойным логарифмом и оно меньше первого.
\par Предположим, что  $r>B$. Тогда по определению $r$ $S$ делится на все числа меньшие $r$, то есть $S \geq [1, \ldots, B] \geq 2^B$ - противоречие $\Rightarrow r \leq B$
\par Пусть $r=p_1^{k_1}\cdot \ldots \cdot p_s^{k_s} \Rightarrow k_i \leq \log_2 B$, так как $r\leq B$. Предположим, что $\forall i \: n \: \vdots \: p_i$. Тогда $\forall i \: n^{[\log_2 B]} \: \vdots \: p_i^{[\log_2 B]} \: \vdots p_i^{k_i}$ (так как $k_i \leq \log_2 B$) $\Rightarrow n^{[\log_2 B]} \: \vdots \: r$ - противоречие, так как тогда $S \: \vdots \: r$. Следовательно, $\exists p_i \nmid n$. Перенумеруем $p$ так что $p_1, \ldots, p_t$ не делят $n$, $p_{t+1}, \ldots, p_s$ делят $n$. Тогда $p_1^{k_1} \cdot \ldots \cdot p_t^{k_t} \nmid \prod_{i=1}^{\left[ \log_2^2 n \right]}(n^i-1)$, так как иначе $r$ делит $S$.
\par Рассмотрим $$\frac{r}{(r,n)}=\underbrace{p_1^{k_1} \cdot \ldots \cdot p_t^{k_t}}_{\text{не делит } S} \cdot \underbrace{p_{t+1}^{k'_{t+1}} \cdot \ldots \cdot p_s^{k'_s}}_{\text{делит } S} \Rightarrow \frac{r}{(r,n)} \nmid S$$
\par Из того, что $r$ выбиралось минимальным следует, что $(r,n)=1$. Следовательно $\ord_r n > \log_2^2 n$ и все доказано $\blacksquare$