\setcounter{section}{71}
\section{Определение решётки и дискретного подмножества. Любая дискретная подгруппа $\mathbb{R}^n$ является решёткой.}

\textbf{Опр} Пусть $( e_1, . . . , e_k)$ — набор линейно независимых векторов в $\mathbb{R}^n$. Тогда дискретная абелева группа в $\mathbb{R}^n$, порождённая $\{e_i\}$, называется решёткой, а набор $( e_1, . . . , e_k)$ называется базисом решётки. Иными словами, решётка есть множество $\Lambda = \{a_1 e_1 + ... + a_k e_k\}, a_i \in \mathbb{Z}$

\textbf{Опр} Подмножество X пространства $\mathbb{R}^n$ называется дискретным, если для любой точки x $\in$ X существует окрестность этой точки, не содержащая других точек множества X.

\textbf{Теорема}. Любая дискретная подгруппа $\mathbb{R}^n$ является решёткой. \par
$\blacktriangle$
1. Дискретное множество в $\mathbb{R}^n$ является множеством изолированных точек: действительно, рассматриваем произвольную точку, принадлежащую множеству; по определению дискретности, она является изолированной точкой этого множества. \par
2. По определению группы, выполняется ассоциативность, наличие нейтрального элемента и наличие обратного элемента. $\exists$ нейтральный элемент: это начало координат. \par
3. Мы выбрали начало координат. Возьмём расстояния всех точек до начала координат. Существует inf расстояний, отличный от нуля (так как иначе в любой эпсилон-окрестности начала координат существует точка из нашего множества), inf > 0 \par
4. Докажем, что inf достигается. Предположим противное - тогда в любой эпсилон окрестности infinum'a существует бесконечное количество точек с радиусом, большим чем он. Но тогда обязательно найдутся две точки, расстояние между которыми меньше inf $\Rightarrow$ в силу бытия подгруппой мы можем отложить это расстояние от нуля и получить противоречие тому, что мы выбрали inf. Значит, inf достигается. \par
5. Выбираем этот inf и так последовательно формируем базис: получаем базис размера k, получаем линейные подпространства размерности k, находим расстояние между ними, это и есть искомый вектор базиса, получаем базис размерности k + 1. \par
6. Этот процесс должен оставиться не позднее, чем n. Почему так? Предположим противное. Тогда $\exists$ точка в фундаментальной области, которая не была получена. Но она же находится в фундаментальной области, граничащей с нулём, в силу того, что это группа $\Rightarrow$ противоречие тому, что мы всегда выбирали минимальные расстояния (мы нашли точку с меньшим расстоянием). P.S. По сути мы просто пытаемся показать, что мы не можем найти n + 1-й линейно независимый вектор в пространстве $\mathbb{R}^n$.
$\blacksquare$ \par
Доказательство 2: \par
1) В любом компакте содержится лишь конечное число точек из G

2) Будем рассматривать линейное пространство, порождённое нашей подгруппой G. 

3) Выберем базис (в подпространстве) $e_1,...,e_k$ среди элементов нашей группы, и рассмотрим подгруппу $G_0 = Ze_1+..+Ze_k \subset G$.

4) Так как G дискретная, в $G/G_0$ содержится конечное количество элементов (по пункту 1), пусть $[G:G_0]=q$. 

5) G содержится в $1/q G_0$, и поэтому является решёткой. 