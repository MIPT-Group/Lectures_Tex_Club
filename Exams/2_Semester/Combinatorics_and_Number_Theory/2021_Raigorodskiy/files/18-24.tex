\setcounter{section}{17}
\section{Разброс (уклонение, дискрепанс) системы подмножеств относительно раскраски. Теорема о верхней оценке (б/д).}
\begin{definition} Пусть $\mathcal{M} = \{M_1, M_2, ..., M_s\}, \text{где } \forall M_i \subset \mathcal{R}$ ($\mathcal{R}$ - конечное множество) -- система подмножеств, а $\chi$ -- раскраска множества $\mathcal{R}$ в 2 цвета.
\begin{equation*}
    \chi (j) = 
    \begin{cases}
        1, &\text{если $j$-ый элемент $\mathcal{R}$ окрашен в первый цвет}\\
        -1, &\text{если $j$-ый элемент $\mathcal{R}$ окрашен во второй цвет}
    \end{cases}
\end{equation*}
Тогда \textit{разброс (уклонение) системы подмножеств} $\mathcal{M}$ относительно раскраски $\chi$ обозначается $disc(\mathcal{M}, \chi)$, и по определению
$$disc(\mathcal{M}, \chi) = \max_{i=1,...,s}|\sum_{j \in M_i} \chi(j)|$$
Равно максимальной разности между количеством элементов покрашеных в разные цвета на определённых множествах. 
\end{definition}
\begin{definition} Пусть $\mathcal{M} = \{M_1, M_2, ..., M_s\} \subset \mathcal{R}$ -- система подмножеств. Тогда разброс (уклонение) системы подмножеств $\mathcal{M}$ обозначается $disc(\mathcal{M})$, и по определению
$$disc(\mathcal{M}) = \min_{\chi} disc(\mathcal{M}, \chi)$$
\end{definition}
\begin{theorem}[о верхней оценке]
Если $|\mathcal{R}| = n$, то $\forall \mathcal{M} : |\mathcal{M}| \leq n$ верно, что $disc(\mathcal{M}) \leq 6\sqrt{n}$
\end{theorem}


\section{Коды, исправляющие ошибки. Расстояние Хэмминга. Понятие (n,M,d)-кода. Число ошибок, исправляемых кодом. Граница Хэмминга.}
В этом билете $n$ -- число символов (0 и 1) в каждом кодовом слове.\\
Для канала связи известно, что на каждое кодовое слово приходится не более $k$ ошибок. (под ошибкой подразумевается замена 0 на 1, и наоборот)\\
$M$ -- число кодовых слов. Очевидно, что $M \leq 2^n$.
\begin{definition}[Расстояние Хэмминга]
    Пусть $\Vec{a} = a_1 a_2 ... a_n$, $\Vec{b} = b_1 b_2 .. b_n$ -- кодовые слова. Расстояние Хэмминга между $\Vec{a}$ и $\Vec{b}$ обозначается $d(\Vec{a}, \Vec{b})$. По определению
    $$d(\Vec{a}, \Vec{b}) = \sum_{i = 1}^{n} I_{\{a_i \neq b_i\}}$$
    - количество позиций, на которых символы отличаются.
\end{definition}
Пусть $d$ -- минимальное расстояние между словами, то есть
$$d = \min_{a, b} d(\Vec{a}, \Vec{b})$$ - самое маленькое расстояние, которое можно построить в рамках определённого кода.
\begin{note}
    $d(\Vec{a}, \Vec{b})$ можно рассматривать как метрику, соответственно можно ввести понятие шара: 
    $$B_r(\Vec{a}) = \{\Vec{b}: d(\Vec{a}, \Vec{b}) \leq r\}$$
    Объемом шара назовем количество кодовых слов в нём. Так как в допускается не более чем $r$ ошибок, а количество способов выбрать $i$ позиций для $i$ ошибок соответсвенно равно $C_{n}^{i}$, то $$V(B_r(\Vec{a})) = \sum_{i=0}^{r} C_{n}^{i}$$
\end{note}
\begin{definition} $(n, M, d)$-код это код, в котором каждое слово длины $n$, всего слов $M$, минимально расстояние между кодовыми словами $d$
\end{definition}
\begin{proposition}$(n, M, d)$-код исправляет вплоть до $[\frac{d-1}{2}]$ ошибок.
\end{proposition}
\begin{proof}[$\blacktriangle$]
    Если у каждого шара $2r < d$, то если канал допускает не более $r$ ошибок, слово однозначно восстанавливается, поскольку шары не пересекаются\\
    Пусть $r = [\frac{d-1}{2}]$. Тогда утверждение выполнено.
\end{proof}
\begin{note}[Граница Хэмминга для $(n, M, d)$-кода]
    $$|M| \leq \frac{2^n}{\sum \limits_{i=0}^{r} C_n^i}, r = \left[ \frac{d-1}{2} \right] $$
    \begin{proof}[$\blacktriangle$]
        $|M| \cdot \sum \limits_{i=0}^{r} C_n^i \leq 2^n$, так как сумма объемов непересекающихся шаров не больше объема всего пространства.
    \end{proof}
\end{note}


\section{Распределение простых чисел в натуральном ряде. Функции $\pi(x), \theta(x), \psi(x)$. Теорема о равенстве нижних и верхних пределов (формулировка). Неравенство $\lambda_1 \leq \lambda_2$. Постулат Бертрана (б/д). Теорема Адамара, Валле-Пуссена (б/д). «Дырки» между соседними простыми числами (б/д).}

Распределение простых чисел в натуральном ряде

\textbf{Определение} 

$$
\begin{array}{l} 

\pi(x) = \underset{p \leq x}{\sum} 1 \text{— количество простых чисел, не превосходящих x.}\\
\theta(x) = \underset{p \leq x}{\sum} ln(p)\\
\psi(x) = \underset{(\alpha,p) :p^{\alpha} \leq x}{\sum} ln(p)
\end{array}
$$

\textbf{Теорема} (о равенстве верхних и нижних пределов (формулировка))

$$\lambda_1 = \overline{\lim_{x \leftarrow \infty}} \frac{\theta(x)}{x},\\
\lambda_2 = \overline{\lim_{x \leftarrow \infty}} \frac{\psi(x)}{x},\\
\lambda_3 = \overline{\lim_{x \leftarrow \infty}} \frac{\pi(x)}{x / ln(x)}$$

За $\mu_i$ обозначим соответствующие нижние пределы.

Тогда $\lambda_1 = \lambda_2 = \lambda_3, \mu_1 = \mu_2 = \mu_3$.

\textbf{Утверждение.} $\lambda_1 \leq \lambda_2$

$$\blacktriangle \ \lambda_1 = \overline{\lim_{x \leftarrow \infty}} \frac{\theta(x)}{x} =
\overline{\lim_{x \leftarrow \infty}} \frac{\underset{p \leq x}{\sum} ln(p)}{x} \leq \overline{\lim_{x \leftarrow \infty}} \frac{\underset{(\alpha,p) :p^{\alpha} \leq x}{\sum} ln(p)}{x} = \overline{\lim_{x \leftarrow \infty}} \frac{\psi(x)}{x} = \lambda_2\ \blacksquare$$

\textbf{Теорема} (Постулат Бертрана (формулировка)) 

$$\forall x \ \exists p : p \in [x, 2x]$$

\textbf{Теорема} (Адамара, Валле-Пуссена)

$$\pi(x) \sim \frac{x}{ln x}$$

\textbf{«Дырки» между соседними простыми}

\textbf{Теорема} (Чебышёв) $\exists a, b : 0 < a < b < \infty$ такие, что $\frac{ax}{ln(x)} \leq \pi(x) \leq \frac{bx}{ln(x)}$

На лекции Райгородский указал конкретные границы: $a = ln(2), b = 4 ln(2)$


\section{Степень вхождения простого числа в факториал и центральный биномиальный коэффициент. Неравенство для $C_{2n}^{n}$}
\begin{lemma}
    $$[2x] - 2[x] \leq 1$$
    где $[x]$ - целая часть $x$.
    \begin{proof}[$\blacktriangle$]
        $$2x = 2([x] + \{x\}) = 2[x] + 2\{x\}, $$
        $$[2x] - 2[x] = 2[x] + [2\{x\}] - 2[x] = [2\{x\}]  \leq 1$$
    \end{proof}
\end{lemma}
\begin{theorem}
    $$C_{2n}^n \leq \prod\limits_{p \leq 2n} p ^ {[\log_{p} (2n)]}$$
    \begin{proof}[$\blacktriangle$]
        Центральный биномиальный коэффициент: $C_{2n}^{n} = \frac{(2n)!}{n! \cdot n!}$
        $$C_{2n}^{n} = \frac{(2n)!}{n! \cdot n!} = \prod\limits_{p \leq 2n} p^{\left[ \frac{2n}{p} \right] + \left[ \frac{2n}{p^2} \right] + ... - 2 \left( \left[ \frac{n}{p} \right] + \left[ \frac{n}{p^2} \right] + ... \right) },$$
        где $\left[ \frac{2n}{p} \right] + \left[ \frac{2n}{p^2} \right] + ...$ - \textbf{степень вхождения простого числа $p$ в разложение факториала $(2n)!$} на простые множители, а $ \left[ \frac{n}{p} \right] + \left[ \frac{n}{p^2} \right] + ... $ - степень вхождения простого числа $p$ в разложение $(n)!$ на простые множители.
        
        $$C_{2n}^{n} = \prod\limits_{p \leq 2n} p^{\left[ \frac{2n}{p} \right] + \left[ \frac{2n}{p^2} \right] + ... - 2 \left( \left[ \frac{n}{p} \right] + \left[ \frac{n}{p^2} \right] + ... \right) } = \prod\limits_{p \leq 2n} p^{\left( \left[ \frac{2n}{p} \right] - 2 \left[ \frac{n}{p} \right] \right) + \left( \left[ \frac{2n}{p^2} \right] - 2 \left[ \frac{n}{p^2} \right] \right) + ...}$$
        Заметим, что таких слагаемых не больше $[\log_{p} (2n)]$ и воспользуемся леммой.
        $$C_{2n}^{n} = \prod\limits_{p \leq 2n} p^{\left( \left[ \frac{2n}{p} \right] - 2 \left[ \frac{n}{p} \right] \right) + \left( \left[ \frac{2n}{p^2} \right] - 2 \left[ \frac{n}{p^2} \right] \right) + ...} \leq \prod \limits_{p \leq 2n} p ^ { \left[ \log_{p} (2n) \right] }$$
    \end{proof}
\end{theorem}


\section{Показатель. Показатель элемента из множества $\Z_m$ делит $\varphi(m)$. Первообразный корень (определение и значения при $m \leq 7$). Пример модуля, по которому не существует первообразного корня. Теорема о существовании первообразного корня (б/д).}

\textbf{Определение} \textit{Показатель (порядок)} числа $a$ по модулю $m$ обозначается $ord_m(a)$ и по определению $ord_m(a) = \min \{\delta \in \N: a^{\delta} \equiv 1 \pmod m\}$.

\begin{proposition}
    $\varphi(m) \equiv 0 \pmod \delta$ (то есть показатель элемента делит $\varphi(m)$)
    \begin{proof}[$\blacktriangle$]
        Предположим, что это не так. Тогда $\varphi(m) = k\delta + r, r \in (0, \delta)$. Тогда $1 \equiv a^{\varphi(m)} = a ^ {k\delta + r} \equiv a^r \pmod  m \Rightarrow $ противоречие с определением показателя.
    \end{proof}
\end{proposition}

\textbf{Определение} $g$ называется \textit{первообразным корнем по модулю $m$}, если его показатель равен $\varphi(m)$.\\\\
\textbf{Значения первообразного корня для $m \leq 7$}\\\\
\begin{tabular}{c|c}
     m & Первообразный корень\\
     2 & 1\\
     3 & 2\\
     4 & 3\\
     5 & 2\\
     6 & 5\\
     7 & 3\\
\end{tabular} \\
Однако для m = 8 первообразного корня не существует.\\
\begin{theorem}[Теорема о существовании первообразного корня (б/д)]
    Первообразный корень существует только для $m \in \{2, 4, p^{\alpha}, 2p^{\alpha}\}$, где $p$ -- простое нечетное, $\alpha \in \N$.
\end{theorem}


\section{Индексы. Корректность определения в случае первообразного корня. Таблицы индексов. Решение степенных сравнений (умение).}

\textbf{Определение} Зафиксируем первообразный корень $g$ по модулю $m$. Пусть $(a, m) = 1$. \textit{Индексом} $\gamma$ = $ind_g(a)$ числа $a$ по модулю $m$ при основании $g$ называется такое минимальное число $\gamma$, что $a \equiv g^{\gamma} \pmod m$. Индекс можно интерпретировать как дискретный логарифм.
\begin{theorem}[Корректность определения в случае первообразного корня]
    Пусть $g$ -- первообразный корень по модулю $m$. Степени $g: g^l, 0 \leq l < \varphi(m)$ несравнимы между собой и образуют приведённую систему вычетов. Из этого следует, что индекс для первообразного корня определён корректно.
    \begin{proof}[$\blacktriangle$]
        Докажем, что все степени $g$ не сравнимы по модулю $m$.\\
        Предположим противное: пусть $\exists k, m : g^k \equiv_{m} g^m$. (Без ограничения общности $0 \leq m < k < \varphi(m)$)
        Тогда $g^k - g^m \equiv_{m} 0$\\
        $$g^k - g^m = g^k(g^{k - m} - 1) \equiv_{m} 0$$
        Получается, что $g^{k - m} \equiv_{m} 1$, но $k - m < \varphi(m)$, а значит $g$ -- не первообразный корень. Противоречие.
    \end{proof}
\end{theorem}


\begin{proposition}[б/д]
    Пусть $\varphi(m) = p_1^{\alpha_1} \cdot ... \cdot p_s^{\alpha_s}$ -- каноническое разложение числа $\varphi(m)$ на простые множители, $(g, m) = 1$. В этом случае $g$ -- первообразный корень в $\Z_m$ тогда и только тогда, когда $g$ НЕ является решением ни одного из сравнений $g^{\frac{\varphi(m)}{p_k}} \equiv 1 \pmod{m}$ при $k = 1, 2, ..., s$.
\end{proposition}

\begin{proposition}[б/д]
    Сравнение вида $x^n \equiv a \pmod m$, где $m$ имеет вид $p^{\alpha}$ или $2p^{\alpha}$, $(a, m) = 1, d := (n, \varphi (m))$ разрешимо тогда и только тогда, когда $d | ind_{g}(a)$, где $g$ -- первообразный корень. Более того, если сравнение разрешимо, то оно имеет $d$ решений.
\end{proposition}

\begin{proposition}[б/д]
    Пусть $k = ord(g)$. $g^{k_1} \equiv g^{k_2} \pmod{m}$ тогда и только тогда, когда $k_1 \equiv k_2 \pmod{k}$
\end{proposition}

\begin{example}
    Алгоритм решения сравнения $x^n \equiv a \pmod{m}$.\\
    (только если по модулю $m$ существует первообразный корень, то есть $m \in \{2, 4, p^{\alpha}, 2p^{\alpha}\}$)\\
    1) Находим первообразный корень $g$ (любой, на практике обычно минимальный) с помощью утверждения 1.\\
    2) Находим $ind_g(a)$ перебором (достаточно перебрать значения от $1$ до $\varphi(m)$)\\
    3) С помощью утверждения 2 проверяем, есть ли решения у сравнения.\\
    4) Сравнение можно переписать в виде $g^{n \cdot ind_g(x)} \equiv x^n \equiv a \equiv g^{ind_g(a)} \pmod{m}$\\
    5) Используя утверждение 3, переписываем сравнение в виде (напоминание: $ord(g) = \varphi(m)$):
    $$
        n \cdot ind_g(x) \equiv ind_g(a) \pmod{\varphi(m)}
    $$
    6) Решая линейное сравнение относительно $ind_g(x)$ в пункте 5, получаем решения $ind_g(x) \in \{k_1, ..., k_l\}$. Тогда решением исходного сравнения будут $x \in \{g^{k_1}, ..., g^{k_l}\}$.
\end{example}

Примеры решения степенных сравнений:

\textbf{Пример 1} $x^8 \equiv 5 \pmod {17}$\\
Это сравнение вида $x^n \equiv a \pmod {m}$\\
$n = 8$, $\varphi(m) = \varphi(17) = 16$\\
$d = (n, \varphi(m)) = (8, 16) = 8$\\
Найдем первообразный корень (перебором):\\
$\varphi(m) = 16 = 2^4$.\\
$g = 3$ не является решением $g^{\frac{\varphi(17)}{2}} = g^8 \equiv 1 \pmod{17}$, следовательно, $g = 3$ -- первообразный корень.\\
Подбором находим, что $ind_g(a) = ind_3(5) = 5, ind_3(5)$ не делится на $d = 8$. Значит, решений у сравнения $n \cdot ind_g(x) \equiv ind_g(a) \pmod{\varphi(m)}$ нет.\\

\textbf{Пример 2} $x ^ 4 \equiv 4 \pmod{17}$\\
$\varphi(17) = 16, d = (4, 16) = 4, g = 3$ -- первообразный корень\\
$ind_3(4) = 12, ind_3(4)$ делится на $d = 4$. Значит, есть $d = 4$ различных решения.\\
Решаем сравнение $4 \cdot ind_3(x) \equiv 12 \pmod{16}$, получаем, что $ind_3(x) \in \{3,7,11,15\}$.\\
Тогда искомые решения: $x \in \{g^{3}, g^{7}, g^{11}, g^{15}\} = \{10, 11, 7, 6\}$ (по модулю 17).\\


\section{Теорема Дирихле о диофантовых приближениях (формулировка и доказательство любым способом).}

\begin{theorem}[Дирихле]
    Если $\alpha$ -- иррациональное, то существует бесконечно много различных $\frac{p}{q} \in \Q : \left| \alpha - \frac{p}{q} \right| \leq \frac{1}{q ^ 2}$.\\
    \textit{Замечание.} $\frac{p}{q}$ может быть как сократимой, так и несократимой дробью
    \begin{proof}[$\blacktriangle$]
        Рассмотрим $Q \in \N$. Разобьём отрезок $[0; 1]$ на $Q$ частей.\\
        Пусть $A = \{ \{ \alpha x\} : x \in \{ 0, 1, ..., Q \} \}$, где $\{ \cdot \}$ -- дробная часть числа ($\{ x \} = x - [x]$). $|A| = Q + 1$.\\
        По принципу Дирихле $\exists x_1, x_2 \in {0, 1, ..., Q}: \left| \{ \alpha x_1 \} - \{ \alpha x_2 \} \right| \leq \frac{1}{Q}$, то есть $x_1, x_2$ попадут в один отрезок. Без ограничения общности $x_1 > x_2$\\
        $$\left| \{ \alpha x_1 \} - \{ \alpha x_2 \} \right| = \left| \alpha x_1 - [\alpha x_1] - \alpha x_2 + [\alpha x_2] \right| = \left| \alpha (x_1 - x_2) - ([\alpha x_1] - [\alpha x_2]) \right| \leq \frac{1}{Q}$$\\
        Положим $q = x_1 - x_2, p = [\alpha x_1] - [\alpha x_2]$, при этом $q \leq Q$.
        $$\left| \alpha q - p \right| \leq \frac{1}{Q}$$
        Разделим неравенство на $q$.
        $$\left| \alpha - \frac{p}{q} \right| \leq \frac{1}{qQ} \leq \frac{1}{q ^ 2}$$\\
        Таким образом, мы доказали существование приближения. Докажем, что их бесконечно много.\\
        Пусть $a = \left| \alpha - \frac{p}{q} \right|, a > 0$. Выберем $Q'$ так, чтобы $\frac{1}{Q'} < a$.\\
        По доказанному $\exists p', q': \left| \alpha - \frac{p'}{q'} \right| \leq \frac{1}{q'Q'}$.\\
        Получается, что $\frac{p'}{q'}$ аппроксимирует $\alpha: \left| \alpha - \frac{p'}{q'} \right| \leq \frac{1}{q'^2}$.\\
        С другой стороны, \\
        $$\left| \alpha - \frac{p'}{q'} \right| \leq \frac{1}{Q'} < a = \left| \alpha - \frac{p}{q} \right|$$.\\
        Получается, что $\frac{p'}{q'}$ и $\frac{p}{q}$ -- различные и аппроксимируют $\alpha$. Повторяем этот процесс, и получаем, что существует бесконечно много различных аппроксимаций.
    \end{proof}
\end{theorem}