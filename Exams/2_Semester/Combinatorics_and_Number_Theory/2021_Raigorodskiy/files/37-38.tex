\setcounter{section}{36}
\section{Определение всюду плотности. Последовательность $\ln n$ всюду плотна на \texorpdfstring{$[0, 1]$}{[0, 1]}.}

\section{Определение всюду плотности. Если последовательность равномерно распределена по модулю 1, то она и всюду плотна.}
Последовательность $x_n$ \textbf{всюду плотна на отрезке [a, b]}, если $\forall$ $[c, d] \subset [a, b]$ $\exists$ бесконечного много номеров N таких, что $\{x_N\} \in [c, d]$. \\
\\
$\{ln(n)\}$ всюду плотна на [0; 1]. \par
$\blacktriangle$
Зафиксируем N. Тогда $[x_n] = k \in \{1, \dots, [ln(N)]\}$. $\{ln(n)\} \in [c; d] \Leftrightarrow ln(n) \in [k + c; k + d] \Rightarrow n \in [e^{k+c}; e^{k+d}]$. Для N число подходящих n будет $\sum_{k=1}^{[ln(N)]} (e^{k+d} - e^{k+c}) = \frac{e(e^{ln(N)} - 1)}{e-1}(e^d - e^c) ~ N \frac{e(e^d - e^c)}{e-1} \to \infty$
$\blacksquare$ \\
\\
\textbf{Теорема.} Если последовательность $x_n$ р.р. mod 1, то она всюду плотна на отрезке [0, 1]. \par
$\blacktriangle$
Из определения равномерной распределённости по модулю 1, $\forall c < d \in [0; 1]$ \\ $\lim_{N \to \infty} \frac{|k|k\leqslant N, \{x_k\} \in [c; d]|}{N} = d - c $, т.е. для любого подотрезка $\exists$ бесконечное количество точек в нём.
$\blacksquare$ 