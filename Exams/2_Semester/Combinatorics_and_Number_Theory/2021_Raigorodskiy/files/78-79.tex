\setcounter{section}{77}
\section{Определение р.р. (mod 1) последовательности. Вывод интегрального признака из того, что последовательность р.р. (mod 1). Формулировка интегрального признака через комплекснозначную функцию (б/д).}


\section{Определение р.р. (mod 1) последовательности. Вывод р.р. (mod 1) последовательности из интегрального признака. Формулировка интегрального признака через комплекснозначную функцию (б/д).}
Послед-ность $x_1, x_2, \dots, x_n, \dots$ - \textbf{равномерно распределена по модулю 1}, если:
\[ \forall a, b \in [0, 1] \lim_{N \to \infty} \frac{|\{i = 1, \dots, N: \{x_i\} \in [a, b)\}|}{N} = b - a \]
\textbf{Теорема об интегральном признаке}. $x_n$ - р.р. mod 1 $\Leftrightarrow$ $\forall$ функции f, определённой и непрерывной на [0; 1] $\lim_{N \to \infty} \frac{1}{N} \sum_{n=1}^N f(\{x_n\}) = \int_{0}^{1} f(x)dx$. \par
$\blacktriangle$
Введём индикатор "попадания" в отрезок: $I_{[a; b)}(x)$ = 1, если $x \in [a; b)$, 0 иначе. Но при этом $\lim_{N \to \infty} \frac{1}{N} \sum_{n=1}^N I_{[a; b)}(\{x_n\}) = b - a = \int_{0}^{1} I_{[a; b)}(\{x\}) dx $, если $x_n$ равномерно распределена. \\
\\
$\Rightarrow$. 1) Пусть $x_n$ - равномерно распределена. \par
2) Разобьём отрезок [0; 1] точками $a_1, a_2, \dots, a_m; a_0 = 0, a_{m+1} = 1$ на конечное число подотрезков:
$[0; a_1) \cup [a_1; a_2) \cup \dots \cup [a_m; 1)$. Рассмотрим индикатор каждой из частей и эти индикаторы сложим. $\sum_{i=1}^{m+1} c_i I_{[a_{i-1}; a_i]}$ - ступенчатая функция. \par
3) Пусть g(x) - ступенчатая функция. Тогда с учётом пункта 1, $\lim_{N \to \infty} \frac{1}{N} \sum_{n=1}^N g(\{x_n\}) = \int_0^1 g(x)dx$. Таким образом, мы доказали для индикатора и для линейной комбинации индикаторов (ступенчатой функции). Тогда мы фиксируем f определённую и непрерывную на [0; 1), фиксируем $\varepsilon > 0$. Тогда $\exists f_1, f_2$ - ступенчатые, такие, что $f_1(x) \leqslant f(x) \leqslant f_2(x) \forall x$, $\int_0^1 (f_2(x) - f_1(x))dx < \varepsilon$.  \par
4) Отсюда: $\int_0^1 f(x)dx - \varepsilon \leqslant \int_0^1f_2(x)dx - \varepsilon \leqslant \int_0^1 f_1(x)dx = \lim_{N \to \infty} \frac{1}{N} \sum_{n=1}^N f_1(\{x_n\}) \leqslant \underline\lim_{N \to \infty} \frac{1}{N} \sum_{n=1}^N f(\{x_n\}) \leqslant \overline\lim_{N \to \infty} \frac{1}{N} \sum_{n=1}^N f(\{x_n\}) \leqslant$ \\ 
$\leqslant \lim_{N \to \infty} \frac{1}{N} \sum_{n=1}^N f_2(\{x_n\}) = \int_0^1 f_2(x)dx \leqslant \int_0^1 f_1(x)dx + \varepsilon \leqslant \int_0^1 f(x)dx + \varepsilon$ \par
5) Из рассуждения выше нижний и верхний предел лежат между $\int_0^1 f(x)dx - \varepsilon$ и $\int_0^1 f(x)dx + \varepsilon$, так как $\varepsilon$ - любой, то нижний и верхний пределы равны, значит, существует предел, равный $\int_0^1 f(x)dx$ \\
\\
$\Leftarrow$ Пусть $\forall f$ выполняется условие. Тогда верны рассуждения из пункта 4, только в пункте 4 мы аппроксимировали непрерывную функцию индикаторами, а теперь мы хотим аппроксимировать индикатор непрерывными функциями: $\forall \varepsilon$ $\exists g_1, g_2$ - непрерывные: $g_1(x) \leqslant I_{[a; b)} \leqslant g_2(x)$ и $\int_0^1 (g_2(x) - g_1(x))dx \leqslant \varepsilon$. Далее аналогично.
$\blacksquare$ \\
\\
\textbf{Формулировка интегрального признака через комплекснозначную функцию.} $x_n$ - р.р. mod 1 $\Leftrightarrow$ $\forall$ \emph{комплекснозначной} функции f, имеющей период 1, $\lim_{N \to \infty} \frac{1}{N} \sum_{n=1}^N f(\{x_n\}) = \lim_{N \to \infty} \frac{1}{N} \sum_{n=1}^N f(x_n) = \int_{0}^{1} f(x)dx$