\setcounter{section}{34}
\section{Определение равномерной распределённой последовательности по модулю 1. Является ли $\sqrt{n}$ р.р. (mod 1) последовательностью?}

Послед-ность $x_1, x_2, \dots, x_n, \dots$ - \textbf{равномерно распределена по модулю 1}, если:
\[ \forall a, b \in [0, 1] \lim_{N \to \infty} \frac{|\{i = 1, \dots, N: \{x_i\} \in [a, b)\}|}{N} = b - a \]
или, что равносильно (по сути речь про вероятность, что дробная часть числа из первых N окажется на отрезке b - a):
\[ \forall \gamma \in [0, 1] \lim_{N \to \infty} \frac{|\{i = 1, \dots, N: \{x_i\} \leqslant \gamma \}|}{N} = \gamma \]
Пример: $\sqrt{n}$.
Второе определение: Фиксируем $\gamma$ и $N$. Последовательность: $\sqrt{1}, \sqrt{2}, \dots, \sqrt{N}$. Пусть переменная k принимает значения целых частей, которые возникают в такой последовательности; k $\in$ $\{ 1, [\sqrt{2}], [\sqrt{3}], \dots \}$ = $\{ 1, 2, 3, \dots, [\sqrt{N}] \}$. \par
$\{ x_n \} \leqslant \gamma$. Это может возникнуть, если число $x_n$ имеет вид $k^2, k^2+1, \dots, (k + \gamma)^2 = k^2 + 2k\gamma + \gamma^2$. Таких чисел с точностью до O(1) $2k\gamma$. Тогда общее количество таких n: $|\{n: \{ \sqrt{n} \} \leqslant \gamma\}| = \sum_{k = 1}^{[\sqrt{N}]} (2k\gamma + O(1)) = 2\gamma \frac{[\sqrt{N}][\sqrt{N} + 1]}{2} + O(\sqrt{N}) = N\gamma + O(\sqrt{N})$. Тогда \[ \forall \gamma \in [0, 1] \lim_{N \to \infty} \frac{|\{i = 1, \dots, N: \{x_i\} \leqslant \gamma \}|}{N} = \lim_{N \to \infty} \frac{N\gamma + O(\sqrt{N})}{N} = \gamma \]
Отсюда эта последовательность р.р. (mod 1) по определению.