\setcounter{section}{48}
\section{Китайская теорема об остатках}
\begin{lemma}
    Пусть $(a, b) = 1$, тогда $\exists c: ac \equiv 1 \pmod b$
    \begin{proof}[$\blacktriangle$]
        Рассмотрим числа $a, 2a, ..., (b-1)a$. Они образуют приведённую систему вычетов, а значит есть остаток $1$.
    \end{proof}
\end{lemma}

\begin{theorem}[Китайская теорема об остатках]
    Пусть $n_1, n_2, ..., n_k \in \N$ попарно взаимно простые, а $r_1, r_2, ..., r_k \in \Z$. Тогда $\exists ! M$ по модулю $\prod \limits_{i=1}^k n_i$ решение системы сравнений:\\
    $$\begin{cases}
        M \equiv r_1 \pmod {n_1}\\
        M \equiv r_2 \pmod {n_2}\\
        ...\\
        M \equiv r_k \pmod {n_k}\\
    \end{cases}\\$$
    \begin{proof}[$\blacktriangle$]
        Пусть $N = \prod \limits_{i=1}^k n_i; N_i = \frac{N}{n_i}; N_i^{-1}$ -- обратный к $N_i$ по модулю $n_i$.\\
        Существование $N_i^{-1}$ можно обосновать по лемме, так как $(N_i, n_i) = 1$.\\
        Покажем, что $M = \sum \limits_{i=1}^{k} r_i N_i N_i^{-1}$ будет решением.\\
        Рассмотрим $M$ по модулю $n_1$. Все слагаемые, кроме первого, содержат множитель $N_i$, который делится на $n_1$. Получается, что $M \equiv r_1 N_1 N_1^{-1} \pmod{n_1} \equiv r_1 \pmod {n_1}$, то есть $M$ является решением первого сравнения.\\
        Аналогично проверяем все $k$ сравнений.\\
        Теперь докажем, что решение единственно по модулю $N$.\\
        Пусть $A$ и $B$ -- различные решения по модулю $N$. Тогда $A - B \equiv 0 \pmod{n_i}$. Так как $n_i$ взаимно простые, то $A - B \equiv 0 \pmod{N}$. Получается, что $A$ и $B$ -- одинаковые решения по модулю $N$.\\
        Противоречие.
    \end{proof}
\end{theorem}