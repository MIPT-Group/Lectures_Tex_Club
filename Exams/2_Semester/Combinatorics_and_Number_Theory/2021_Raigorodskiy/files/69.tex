\setcounter{section}{68}
\section{Алгебраические и трансцендентные числа. Существование трансцендентных чисел (из соображения мощности). Теорема Лиувилля (б/д). Конструкция трансцендентного числа с помощью цепной дроби и теоремы Лиувилля. Сводка результатов о трансцендентности: $e, \pi, e+\pi, \pi+e^{\pi}, \alpha^{\beta}$ (теорема Гельфонда), вывод про $e^{\pi}$ из теоремы Гельфонда.}

\textbf{Определение} $\alpha$ - \textit{алгебраическое число}, если существует многочлен с целыми коэффициентами, корнем которого служит $\alpha$.
\textbf{Определение} \textit{Степень алгебраического числа} - это минимальная степень уравнения, корнем которого является это число.

$\mathbb{A}$ - множество алгебраических чисел.

Заметим, что $\mathbb{A}$ - счётное множество (доказывалось на матлогике), но $\mathbb{C}$ континуально. Отсюда следует, что есть не алгебраические числа.
\textbf{Определение} $\alpha \in \mathbb{C}$ - \textit{трансцендентное}, если оно не является алгебраическим.\\

\textbf{Теорема} (Лиувилль) Пусть $\alpha$ - алгебраическое степени $d$. Тогда $\exists c = c(\alpha)$, что неравенство $|\alpha - \frac{p}{q}| \leq \frac{c}{q^d}$ имеет лишь конечное число решений в $\frac{p}{q}$. (Если уменьшить $c$, то вообще не будет решений)

\textit{Конструкция трансцендентного числа с помощью цепной дроби и теоремы Лиувилля:}

\textbf{Теорема} $\forall \psi(q) \rightarrow +\infty \ \exists \alpha:$ неравенство $|\alpha - \frac{p}{q}| \leq \frac{1}{q \psi(q)}$ имеет б.м. решений в $\frac{p}{q}$.

Как пример можно взять $\psi(q) = e^q$. Из предыдущей теоремы возьмём число $\alpha$. Предположим $\alpha$ - алгебраическое число, тогда $\exists d \in \mathbb{N}: |\alpha - \frac{p}{q}| > \frac{c}{q^d}$ выполняется для $\forall p, q$. Что противоречит предыдущей теореме (неравенство $|\alpha - \frac{p}{q}| \leq \frac{1}{q \psi(q)}$ имеет б.м. решений в $\frac{p}{q}$). 

То есть мы совершенно явно, с помощью аппарата цепных дробей, построили трансцендентное число $\alpha$. 

\textbf{Cведения о некоторых числах}:
$e, \pi, \pi + e^{\pi}$ являются трансцендентными. Про $e + \pi$ на данный момент ничего неизвестно.

\textbf{Теорема} (Гельфонд) Пусть $\alpha, \beta$ алгебраические, при этом $\beta$ иррациональное, а $\alpha \notin \{0, 1\}$. Тогда $\alpha^{\beta}$ трансцендентно.

\textbf{Утверждение.} $e^\pi$ трансцендентно.

$\blacktriangle$ Предположим противное: $e^{\pi}$ - алгебраическое. Заметим, что $i$ - алгебраическое. Пусть $\alpha = e^{\pi}, \beta = i = \sqrt{-1} \Rightarrow \alpha^{\beta} = e^{i\pi}$, но $\alpha^{\beta} =e^{i\pi} = -1 \Rightarrow$ $\alpha^{\beta}$ - алгебраическое. Противоречие с теоремой Гельфонда.
$\blacksquare$