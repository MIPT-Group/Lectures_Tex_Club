\setcounter{section}{83}
\section{Алгоритм AKS. Верхняя оценка на $r$ (б/д). Обоснование неравенства $p > r > \log_2^2 n$
для подходящего делителя $p$ числа $n$. Вывод тождества $(X + a)^{n/p} = X^{n/p} + a \: (\md \: X^r - 1, p)$. Определение перестановочности многочлена и числа. Утверждения о свойствах перестановочности.}

\par \textbf{Обоснование неравенства:} Рассматриваем корректность последнего шага. Мы знаем, что $(r,n)=1 \Rightarrow (r, p)=1$ ($p$ - делитель $n$, который мы выбрали в билете 83). Также $p > r$, в противном случае мы бы останавились на 3 или 4 шаге алгоритма. Тогда $p>r>\varphi(r)\geq \ord_r n > \log_2^2 n$ (последнее неравенство следует из построения $r$).
\par \textbf{Тождество:} $(x+a)^{n/p}=x^{n/p}+a \: (\md \: x^r-1, p)$
\par $\blacktriangle$ $$(x+a)^p=x^p+a \: (\md \: x^r-1, p) \text{ при } a=0 \ldots l \text{ (см. билет 42)}$$
$$(x+a)^n=x^n+a \: (\md \: x^r-1, n) \text{ при } a =0 \ldots l \text{ (следствие того, что мы прошли шаг 5)}$$
\par Второе выражение так же выполняется, если мы заменим $\md \: n$ на $\md \: p$, так как $p$ - делитель $n$ (в дальнейшем будем часто переходить к делителям таким образом). Дальше все тождества рассматриваем для $a=0 \ldots l$.
\par Предположим, что $(x+a)^{n/p} \neq x^{n/p}+a \: (\md \: x^r-1, p)$. Возведем обе части в степень $p$. Получаем $(x+a)^n \neq (x^{n/p}+a)^p\: (\md \: x^r-1, p)$. По первому тождеству правая часть распишется как $x^n+a$. Получили, что $(x+a)^n \neq x^n+a\: (\md \: x^r-1, p)$ - противоречие со вторым тождеством $\Rightarrow$ тождество верно $\blacksquare$
\par \textbf{Определение:} Пусть $f(x)$ - многочлен, $m$ - число. Считаем, что $f(x)$ и $m$ перестановочны, если  $(f(x))^m=f(x^m) \: (\md x^r-1,p)$
\par \textbf{Утверждение 1:} Если $f$ перестановочно с $m$ и $g$ перестановочно с $m$, то $f\cdot g$ перестановочно с $m$
$$\blacktriangle \: (fg(x))^m=(f(x)g(x))^m=(f(x))^m (g(x))^m=f(x^m)g(x^m)=fg(x^m) \:
\blacksquare$$
\par \textbf{Замечание:} $x^{mr}-1 \: \vdots \: x^r-1$
\par $\blacktriangle$ $$x^r-1=(x-1)(1+x+\ldots+x^{r-1})$$
$$x^{mr}-1=(x-1)(1+x+\ldots+x^{r-1}+x^{r+1}+\ldots+x^{mr-1})=$$$$=(x-1)(1 \cdot (1+\ldots+x^{r-1})+x^r (1+\ldots+x^{r-1})+\ldots x^{(m-1)r}(1+\ldots+x^{r-1}))=$$$$=(x-1)(1+\ldots+x^{r-1})(1+\ldots+x^{(m-1)r}) \: \vdots \: x^r-1 \: \blacksquare$$
\par \textbf{Утверждение 2:} Если $f$ перестановочно с $m$ и $m'$, то $f$ перестановочно с $mm'$
\par $\blacktriangle$ $$(f(x))^{mm'}=(f(x^m))^{m'} \: (\md x^r-1, p)$$
\par Пусть $y=x^m$. Тогда
$$f(y)^{m'}=f(y^{m'}) \: (\md \: y^r-1, p)$$
\par $y^r-1=x^{mr}-1 \Rightarrow y^r-1 \: \vdots \: x^r-1$ (по замечанию) $\Rightarrow$ тождество верно и по модулю $x^r-1$ (перешли к делителю). Получаем
$$(f(x))^{mm'}=f(y^{m'})=f(x^{mm'}) \: (\md \: x^r-1, p) \: \blacksquare$$