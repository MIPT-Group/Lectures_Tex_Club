\setcounter{section}{69}
\section{Умение: решать уравнения Пелля.}

\begin{definition}
    Уравнение вида $x^2 -my^2 = 1$, где $m$ -- натуральное число, не являющееся точным квадратом, называется \textit{уравнением Пелля}. \\
    Решение $(1, 0)$ называется \textit{тривиальным}. \\
    Решение $(x, y)$ называется \textit{положительным}, если $x \geq 0$ и $y \geq 0$.
\end{definition}

\begin{note}
    Уравнение вида $x^2 - my^2 = 1$ не является уравнением Пелля по этому определению. Однако теория по решению данного уравнения есть во второй теореме и во втором примере. 
\end{note}

\begin{note}
    Ввиду симметрии для решения уравнения достаточно найти все положительные решения.
\end{note}

\begin{note}
    Если $m$ является полным квадратом, то, очевидно, у уравнения нет решений, кроме тривиальных.
\end{note}

\begin{note}
    Пара $(x, y)$ в $\Z[\sqrt{m}]$ имеет вид $x + y\sqrt{m}$. Норма числа $a = x + y\sqrt{m}$ в $\Z[\sqrt{m}]$ это $N(a) = a \cdot \overline{a} = (x + y\sqrt{m})(x - y\sqrt{m}) = x^2 - my^2$. Норма обладает свойством: $N(a) \cdot N(b) = N(a\cdot b)$
\end{note}

\begin{proposition}
    Пара $(x, y)$ является решением уравнения Пелля ($x^2 - my^2 = 1$) тогда и только тогда, когда норма числа $x + y\sqrt{m}$ в $\Z[\sqrt{m}]$ равна единице.
    \begin{proof}[$\blacktriangle$]
        $$
            N(x + y\sqrt{m}) = (x + y\sqrt{m})(x - y\sqrt{m}) = x^2 - my^2
        $$
    \end{proof}
\end{proposition}

\begin{proposition}
    Пара $(x, y)$ является решением уравнения $x^2 - my^2 = -1$ тогда и только тогда, когда норма числа $x + y\sqrt{m}$ в $\Z[\sqrt{m}]$ равна минус единице.
\end{proposition}


\begin{definition}
    $\frac{P_k}{Q_k} = [a_0; a_1, a_2, ..., a_k], (k = 0, 1, ..., n)$ называется \textit{$k$-ой подходящей дробью} к числу $[a_0; a1, a2, ..., a_n]$.
\end{definition}

\begin{theorem}
    Если $n$ -- длина периода цепной дроби, соответствующей $\sqrt{m}$, то решениями уравнения Пелля $x^2 - my^2 = 1$ являются в точности подходящие дроби числа $\sqrt{m}$ вида $\frac{P_{kn-1}}{Q_{kn-1}}$, где $kn$ -- чётно.
\end{theorem}


\begin{note}
    Способы находения корней уравнения $x^2 - my^2 = -1$. (У автора конспекта нет уверенности, что данные способы находят все корни уравнения, однако других способов он не знает)\\
    Способ 1) Если $n$ -- длина периода цепной дроби, соответствующей $\sqrt{m}$, то решениями уравнения $x^2 - my^2 = -1$ являются подходящие дроби числа $\sqrt{m}$ вида $\frac{P_{kn-1}}{Q_{kn-1}}$, где $kn$ -- нечётно.\\
    Способ 2) Находим a - минимальное положительное решение $x^2-my^2 = 1$, находим b - тривиальное (самое простое) решение $x^2-my^2 = -1$, тогда $\pm(a^{p} \cdot b^{2k + 1})$ будут решениями $x^2-my^2 = -1$ для $\forall p, k \in \mathbb{Z}$. 
\end{note}

\textbf{Пример 1.}\\
Найдите наименьшее положительное решение уравнения Пелля $x^2 - 6y^2 = 1$.\\
1) Найдём цепную дробь для $\sqrt{6}$:
$$
    \sqrt{6} = [2; \overline{2, 4}]
$$
2) Длина периода цепной дроби $n = 2$, значит минимальное $k$, такое, что $kn$ будет чётным, равно $1$. Значит, минимальное решение -- подходящие дроби вида $\frac{P_{kn-1}}{Q_{kn-1}} = \frac{P_1}{Q_1}$.\\
3) $\frac{P_1}{Q_1} = [2; 2] = 2 + \frac{1}{2} = \frac{5}{2}$.\\
Получается, пара $(x, y) = (5, 2)$ является минимальным положительным решением уравнения Пелля.\\

\textbf{Пример 2.}\\
Найдите наименьшее положительное решение уравнения $x^2 - 2y^2 = -1$.\\
1) Найдём цепную дробь для $\sqrt{2}$:
$$
    \sqrt{2} = [1; \overline{2}]
$$
2) Длина периода цепной дроби $n = 1$, значит минимальное $k$, такое, что $kn$ будет нечётным, равно $1$. Значит, минимальное решение -- подходящие дроби вида $\frac{P_{kn-1}}{Q_{kn-1}} = \frac{P_0}{Q_0}$.\\
3) $\frac{P_0}{Q_0} = [1] = \frac{1}{1}$.\\
Получается, пара $(x, y) = (1, 1)$ является решением уравнения (в данном случае оно является тривиальным).\\
4) Следующее решение $\frac{P_3}{Q_3} = [1; 2, 2] = \frac{7}{5}$.\\
Получается, пара $(x, y) = (7, 5)$ является решением уравнения.\\

\begin{theorem}
    Пусть $\alpha = a_1 + b_{1}\sqrt{m}$ -- наименьшее нетривиальное положительное решение уравнения $x^2 - my^2 = 1$, то все решения этого уравнения имеют вид $\pm(\alpha)^k, k \in \Z$.
\end{theorem}

\begin{corollary}
    В условиях предыдущей теоремы решениями уравнения Пелля будут пары:
    $$
        \pm \left( \frac{(a_1 + b_1\sqrt{m})^k + (a_1 - b_1\sqrt{m})^k}{2}, \frac{(a_1 + b_1\sqrt{m})^k - (a_1 - b_1\sqrt{m})^k}{2\sqrt{m}} \right), k \in \Z
    $$
\end{corollary}

\textbf{Пример 3.}\\
Найдите все решения уравнения Пелля $x^2 - 6y^2 = 1$.\\
Из примера 1 мы знаем, что пара $(x, y) = (5, 2)$ является минимальным положительным решением уравнения Пелля.\\
Тогда общее решение имеет вид:
    $$
        \pm \left( \frac{(5 + 2\sqrt{m})^k + (5 - 2\sqrt{m})^k}{2}, \frac{(5 + 2\sqrt{m})^k - (5 - 2\sqrt{m})^k}{2\sqrt{m}} \right), k \in \Z
    $$

\textbf{Пример 4.}\\
Решите уравнение $x^2 - 6xy + y^2 = 1$ в целых числах\\
$$
    x^2 - 6xy + y^2 = (x - 3y)^2 - 8y^2 = 1
$$
Делаем замену $z = x - 3y$, и уравнение сводится к уравнению Пелля $z^2 -8y^2 = 1$