\subsection{Теорема о вычитании вполне упорядоченных множеств.}

\textbf{Теорема}. $\alpha \leqslant \beta \Rightarrow \exists! \gamma: \alpha + \gamma = \beta$ (с точностью до изоморфизма).

$\blacktriangle$
Наше $\alpha \simeq [0, b)$ (см. предыдущий билет), тогда $\exists \gamma = \beta \backslash ([0, b))$

Докажем единственность. Пусть есть $\gamma_1 < \gamma_2 \Rightarrow \alpha + \gamma_1 < \alpha + \gamma_2 \Rightarrow$ они не могут оба равняться $\beta$. Противоречие
$\blacksquare$