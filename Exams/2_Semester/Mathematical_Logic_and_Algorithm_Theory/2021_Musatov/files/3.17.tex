\subsection{Построение неглавной универсальной вычислимой функции.}

\textbf{Теорема Успенского-Райса}. Пусть класс всех вычислимых функций (одного аргумента) - F. Пусть $A \subset F$ — произвольное нетривиальное свойство вычислимых функций (нетривиальность означает, что есть как функции, ему удовлетворяющие, так и функции, ему не удовлетворяющие, то есть что множество A непусто и не совпадает со всем F).
Пусть U — главная универсальная функция. Тогда не существует алгоритма, который по U-номеру вычислимой функции проверял бы, обладает ли она свойством A. Другими словами, множество $\{n | U_n \in A\}$ неразрешимо.

$\blacktriangle$
Верно следующее усиление этой теоремы: для любых различных вычислимых функций $\varphi$ и $\psi$ и любой главной универсальной функции U множества всех U-номеров функции $\varphi$ и функции $\psi$ не отделимы разрешимым множеством. (Эти множества к тому же не перечислимы)
$\blacksquare$

Теперь легко указать пример вычислимой универсальной функции, не являющейся главной. Достаточно сделать так, чтобы нигде не определённая функция имела единственный номер. Пусть U(n, x) — произвольная вычислимая универсальная функция. Рассмотрим множество D всех U-номеров всех функций с непустой областью определения. Это множество перечислимо: полухарактеристическая функция - начинаем запускаться параллельно от всех $x$, если область определения не пуста, то когда-нибудь мы получим значение и выведем 1, иначе зациклимся. Рассмотрим всюду определённую вычислимую функцию d, его перечисляющую: $D = \{d(0), d(1), \cdots \}$. Теперь рассмотрим функцию $V (i, x)$, для которой $V (0, x)$ не определено ни при каком x, а $V (i+1, x) = U(d(i), x)$. Другими словами, функция $V_0$ нигде не определена, а функция $V_{i+1}$ совпадает с $U_{d(i)}$. Легко понять, что функция V вычислима; она универсальна по построению, и единственным V-номером нигде не определённой функции является число 0.