\subsection{Теорема Поста: критерий разрешимости в терминах перечислимости множества и его дополнения.}
\par \textbf{Теорема:} $A$ разрешимо $\Leftrightarrow$ $A$ и $\overline{A}$ перечислимы
\par \begin{itemize}
    \item[$\blacktriangle \: \Rightarrow$:] $A$ можно перечислить даже по возрастанию. Запустим цикл по $n=0,1, \ldots$ Если $n \in A$ (вычислимо по определению разрешимого множества), то выводим $n$. Дополнение разрешимого множества также разрешимо (возьмем характеристическую функцию $A$ и поменяем местами значения 0 и 1), поэтому оно тоже перечислимо
    \item[$\Leftarrow$:] Покажем как построить характеристическую функцию для $A$. Запускаем цикл по $n=1, 2, \ldots$ \begin{enumerate}
        \item Возвращаем 1, если $x$ было перечислено в $A$ на $n$-ом шаге
        \item Возвращаем 0, если $x$ было перечислено в $\overline{A}$ на $n$-ом шаге
    \end{enumerate}
    \par Для любого $x$ что-то будет выведено, так как оно лежит либо в $A$, либо в $\overline{A}$ и в силу их перечислимости будет перечислено на каком-то шаге $\Rightarrow A$ - разрешимо $\blacksquare$
\end{itemize}

\subsection{Неразрешимость проблем самоприменимости и остановки.}
\par Пусть $U$ - универсальная вычислимая функция
\par \textbf{Проблема самоприменимости:} по входу $p$ нужно понять, определено ли $U(p,p)$.
\par \textbf{Утверждение:} это неразрешимая проблема, т.е. множество $\{p|U(p,p) \text{ определено}\}$ неразрешимо.
\par $\blacktriangle$ Предположим, что это множество разрешимо. Тогда вычислима функция $$d'(x)=\begin{cases}
   U(x,x)+1 &\text{$U(x,x)$ определено}\\
   1 &\text{$U(x,x)$ не определено}
 \end{cases}$$
 \par Тогда так как $d'$ вычислима, то по определению $U$ $\exists p \forall x \: d'(x)=U(p, x)$. Рассмотрим $U(p,p)$. Предположим, что она определена, тогда $U(p,p)=d'(p)=U(p,p)+1$ - противоречие. Если предположим, что она не определена, получим $U(p,p)=d'(p)=1$ - тоже противоречие $\Rightarrow$ это множество неразрешимо $\blacksquare$
\par \textbf{Лемма:} Область определения вычислимой функции перечислима
\par $\blacktriangle$ Построим полухарактеристическую функцию. Запустим $f(x)$ и если оно остановится, вернем 1. Это и будет полухарактеристической функцией области определения (1 - если $f(x)$ определена, $\bot$ - если не определена) $\Rightarrow$ область определения перечислима $\blacksquare$
\par \textbf{Замечание:} Множество из проблемы самоприменимости перечислимо, как область определения вычислимой функции $d(x)=U(x,x)$
\par \textbf{Проблема остановки (останова):} по входу $(p,k)$ нужно понять, определено ли $U(p,k)$.
\par \textbf{Утверждение:} эта проблема тоже неразрешима
\par $\blacktriangle$ Пусть это не так и проблема разрешима. Тогда бы разрешима проблема самоприменимости, так как она является частным случаем проблемы остановки (при $k=p$). Получили противоречие $\Rightarrow$ эта проблема неразрешима $\blacksquare$