\subsection{Построение комбинатора, возвращающего $n$-е простое число (для нумералов Чёрча).}
\par \textbf{Замечание:} Выражение $Sub$ см. в билете 3.15. Построение $Mod$ вынесено в отдельный билет, который мы не взяли, но понимать как мы его построили наверное стоит.
\par \textbf{Замечание:} $fn$ в конце названий некоторых термов это не аргументы - это обозначение того, что это вспомогательная функция ($Modfn, \: NoDivsfn, \: NthPrimefn)$
\begin{enumerate}
    \item $Inc=\lambda nfx.f(nfx)$
    \par $Inc \: \overline{n}=(\lambda nfx.f(nfx))\overline{n}=\lambda fx.f(\overline{n}fx)=$ \par $=\lambda fx.f((\lambda gy.\underbrace{g(g(\ldots(g}_{n \text{ раз}}y))\ldots)fx)=\lambda fx.f(\underbrace{f(f(\ldots(f}_{n \text{ раз}}x))\ldots)=\overline{n+1}$
    \item $False = \lambda xy.y$; $True = \lambda xy.x$
    \par $Not = \lambda p.p \: False \: True$ (если $p$, то выводим $False$, иначе $True$)
    \par $And=\lambda pq.pqp$ (если $p$, то выводим $q$, иначе - $p$)
    \item $IsZero=\lambda n.n(\lambda x.False)True$
    \par $IsZero \: \overline{0}=\overline{0}(\lambda x.False)True=(\lambda fx.x)(\lambda x.False)True=True$
    \par $IsZero \: \overline{n+1}=(\lambda fx.f(\ldots))(\lambda x.False)True=(\lambda x.False)(\ldots)=False$

    \item $GE=\lambda mn.IsZero(Sub \: n \: m)$ ($\geq$)
    \par $LT=\lambda mn.Not(GE \: m \: n)$ (less then, то есть <)
    \par $IsEqual=\lambda mn.And(GE \: m \: n)(GE \: n \: m)$
    \item $Modfn = \lambda fmn.(LT \: m \: n)m(f (Sub \: m \: n) n)$
    \par \textbf{Смысл:} если $m<n$, то выводим $m$, иначе считаем остаток от деления $m-n$ на $n$
    \par $Mod = YModfn$ ($Y$ - комбинатор неподвижной точки) - остаток от деления $m$ на $n$
    \par $IsDivisible=\lambda nm.IsZero(Mod \: n \: m)$ ($n$ кратно $m$)
    \item Выразим терм $IsPrime$. Это индикатор того, что $n$ не делится на числа $2,\ldots,n-1$. Выразим терм "число $n$ не делится на числа $m, \ldots, n-1$"
    \par $NoDivsfn=\lambda fmn.(IsEqual \: n \: m)\overline{1}((IsDivisible \: n \: m)False(f(Inc \: m)))$
    \par $IsPrime = \lambda n.And(GE \: n \: \overline{2})((YNodivsfn)\overline{2})$ (добавили условие $\geq 2$, так как наша вспомогательная функция не учитывала $0$ и $1$)
    \item Перейдем к выражению искомого терма. Вспомогательный терм: идем по числам, рассмотрели $k$, нашли $m$ простых
    \par $NthPrimefn=\lambda fkm.(IsPrime \: k)((IsEqual \: n(Inc \: m))k(f(Inc \: k)(Inc \: m))(f(Inc \: k)m)$ ($n$ свободная переменная, которая появится в итоговом терме, делаем $n$ как бы глобальным, чтобы не прокидывать во все уровни рекурсии)
    \par \textbf{Смысл:} если $k$ - простое, то если нашли все числа, возвращаем $k$, иначе переходим к следующим $m$, $k$ и вычисляем ту же функцию. Если $k$ не простое, то просто переходим к следующему $k$.
    \par $NthPrime=\lambda n.(YNthPrimefn)\overline{0}\:\overline{0}$
\end{enumerate}