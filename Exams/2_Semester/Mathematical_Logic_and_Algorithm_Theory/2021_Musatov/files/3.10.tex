\subsection{Теорема об арифметической иерархии: $\sum_n \neq \sum_{n+1}, \sum_n \neq \prod_n$}

\textbf{Опр} \textit{Классы арифметической иерархии}\\
Говорят, что множество $A $ принадлежит классу $\Sigma_n$, если существует такое разрешимое множество $R \in \mathbb{N}^{k+1}$, что \begin{center}
$x \in A \Longleftrightarrow \exists y_1 \ \forall y_2 \ \exists y_3 .... \mathcal{Q} y_n [(x, y_1,...,y_k) \in R]$
\\
\end{center}
Аналогично, говорят, что $A$ принадлежит классу $\Pi_n$, если существует такое разрешимое множество $R \in \mathbb{N}^{k+1}$, что \begin{center}
$x \in A \Longleftrightarrow \forall y_1 \ \exists y_2 \ \forall y_3 .... \mathcal{Q} y_n [(x, y_1,...,y_k) \in R]$
\end{center}

Согласно этому определению, $\Sigma_0 = \Pi_0$(классы $\Sigma_0$ и $\Pi_0$ совпадают с классом всех разрешимых множеств)
\\
$\Sigma_1$ - перечислимые, $\Pi_1$ - коперечислимые
\\

\textbf{Теорема 1}. Для любого n в классе $\sum_n$ существует множество,
универсальное для всех множеств класса $\sum_n$. (Его дополнение будет
универсальным в классе $\prod_n$.)

Говоря о дополнении к $\prod_n$, $\sum_n$  множеству, мы имеем в виду дополнение из множества всех множеств, выражаемых через n предикатов.

Говоря об универсальном множестве из класса $\sum_n$, мы имеем в виду множество пар натуральных чисел, которое принадлежит классу $\sum_n$ и среди сечений которого встречаются все множества натуральных чисел, принадлежащие классу $\sum_n$.

$\blacktriangle$
Для класса $\sum_1$ (перечислимых множеств) существование универсального множества мы уже обсуждали (билет 3.extra6.1) С его помощью можно построить универсальные множества и для более высоких классов иерархии. (Начинать надо с первого уровня, так как на «нулевом» уровне не существует универсального разрешимого множества.)

По определению свойства класса $\prod_2$ имеют вид $\forall y\exists z R(x, y, z)$,
где R — некоторое разрешимое свойство. Но их можно эквивалентно определить и как свойства вида $\forall y P(x, y)$, где P — некоторое перечислимое свойство. Теперь уже видно, как построить универсальное множество класса $\prod_2$. Возьмём универсальное перечислимое свойство U(n, x, y), из которого фиксацией различных n получаются все перечислимые свойства пар натуральных чисел. Тогда из свойства $T(n, x) = \forall y U(n, x, y)$ при различных натуральных n получаются все $\prod_2$-свойства натуральных чисел. С другой стороны, само свойство T по построению принадлежит классу $\prod_2$.

Дополнение к универсальному $\prod_2$-множеству будет, очевидно, универсальным $\sum_2$-множеством - так как отрицание чего-либо меняет все кванторы на противоположные, благодаря чему и само множество, и все его сечения по такому свойству также принадлежат $\sum_2$ - значит, это универсальное $\sum_2$-множество.

Аналогично можно действовать и для $\sum_n$- и $\prod_n$-множеств.
$\blacksquare$ \\
\\
\textbf{Теорема 2.} Универсальное $\sum_n$-множество не принадлежит классу $\prod_n$. Аналогичным образом, универсальное $\prod_n$-множество не принадлежит классу $\sum_n$.

$\blacktriangle$
Рассмотрим универсальное $\sum_n$-свойство $T(m, x)$. По определению это означает, что среди его сечений (получающихся, если зафиксировать m) есть все $\sum_n$-свойства. Пусть T принадлежит классу $\prod_n$. Тогда его диагональ, свойство $D(x) = T(x, x)$, также лежит в $\prod_n$ (например, потому, что $D \leqslant_m T$), а её отрицание, свойство $\neg D(x)$, принадлежит классу $\sum_n$. Но этого не может быть, так как $\neg D$ отлично от всех сечений свойства T (оно отличается от m-го сечения в точке m), а T универсально.
$\blacksquare$

Если $\sum_n = \sum_{n+1}$, то $\prod_{n+1} = \prod_n$ (как отрицание $\sum_n$ и $\sum_{n+1}$). Т.к. $\sum_n \subset \prod_{n+1} = \prod_n$, а $\prod_n \subset \sum_{n+1} = \sum_n$, то $\sum_{n+1} = \prod_{n+1}$, что противоречит теореме выше. Основная теорема доказана. 