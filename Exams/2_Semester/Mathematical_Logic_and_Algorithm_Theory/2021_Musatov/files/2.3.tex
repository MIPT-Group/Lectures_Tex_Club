\subsection{Теорема о структуре вполне упорядоченного множество: оно представляется как $\omega \cdot L + F$, где $L$ — множество предельных элементов (кроме, возможно, наибольшего), $F$ — конечное множество.}

\par $\blacktriangle$ Пусть $P$ - множество предельных элементов нашего ВУМа. Заметим, что $P$ - ВУМ (как подмножество ВУМа). Рассмотрим элемент $x \in P$. Пусть $Sx=y$ (следующий элемент). Построим биекцию между $\omega$ и $[x; y)$. Числу $n$ из $\omega$ поставим в соответствие число $\underbrace{SS\ldots S}_\text{$n$ раз}x$. Очевидно, что это инъекция ($x+n=x+m \Leftrightarrow n=m$).
\par Докажем, что это сюръекция. Рассмотрим элемент $t$ лежащий в $[x;y)$. Бесконечно уменьшать его на 1 (то есть брать предыдущий) нельзя по одному из эквивалентных определений фундированности $\Rightarrow$ существует предельный элемент $k$ (у которого нет предыдущего), такой что  $S\ldots Sk=t$. $k$ лежит на в $[x,y)$, но единственный предельный элемент, лежащий в этом множестве - это $x \Rightarrow k=x \Rightarrow t=S\ldots Sk$ будет получен. 
\par Повторим такие действия для всех $x$ (кроме наибольшего). Затем возможны 2 случая
\begin{enumerate}
    \item В исходном ВУМе нет наибольшего элемента. Тогда аналогично прошлым шагам строим изоморфизм между $\omega$ и оставшимися элементами. Получаем, что наш ВУМ равен $\omega \cdot P$
    \item В исходном ВУМе есть наибольший элемент. Тогда осталось лишь конечное число нерассмотренных элементов. Докажем это
    \par Обозначим наибольший элемент всего ВУМа как $a$. По определению фундированности, мы не сможем бесконечно брать предыдущий элемент $\Rightarrow$ существует $k$ - предельный, такой что $a=\underbrace{S\ldots S}_\text{$m$ раз}k$. $k \geq x,$ но $x$ - наибольший из предельных элементов $\Rightarrow$ $k=x \Rightarrow |[x;a]|=m+1$. Построим биекцию между этим отрезком и множеством $F=[0;m]$.
\end{enumerate}

\par Таким образом, получаем, что наше ВУМ равномощно $\omega \cdot L + F$, где $L$ - множество предельных элементов кроме, возможно, наибольшего, а $F$ - конечное множество $\blacksquare$