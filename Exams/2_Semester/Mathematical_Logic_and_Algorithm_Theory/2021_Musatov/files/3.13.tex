\section{Вычислимость}

\subsection{Существование универсального перечислимого множества.}

Множество натуральных чисел называется \textbf{перечислимым}, если оно перечисляется некоторым алгоритмом, то есть если существует алгоритм, который печатает (в произвольном порядке и с произвольными промежутками времени) все элементы этого множества и только их.

Множество $W \subset \mathbb{N} \times \mathbb{N}$ называют \textbf{универсальным} для некоторого класса множеств натуральных чисел, если все сечения $W_n = \{x | \langle n, x \rangle \in W\}$ множества W принадлежат этому классу и других множеств в классе нет.

\textbf{Теорема}. Существует перечислимое множество пар натуральных чисел, универсальное для класса всех перечислимых множеств натуральных чисел.

$\blacktriangle$
Рассмотрим область определения универсальной функции U. Она будет универсальным перечислимым множеством, поскольку всякое перечислимое множество является областью определения некоторой вычислимой функции $U_n$.
$\blacksquare$