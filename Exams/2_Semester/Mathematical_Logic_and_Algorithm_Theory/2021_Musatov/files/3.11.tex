\subsection{Теорема Чёрча-Россера (б/д). Единственность нормальной формы.}

\begin{theorem}[Чёрча-Россера (б/д)]
    Если для некоторого $\lambda$-терма $A$ имеется два варианта редукции $A \rightarrow B$ и $A \rightarrow C$, то существует такой $\lambda$-терм $D$, что $B \rightarrow D$ и $C \rightarrow D$.
\end{theorem}

\begin{definition}
    Термы $M$ и $N$ называются равными, если существует такой терм $T$, что $M$ сводится (некоторым количеством $\alpha$ и $\beta$ редукций) к $T$ и $N$ сводится к $T$.
\end{definition}

\begin{definition}
    Говорят, что терм $M$ находится в \textit{нормальной форме}, если к нему нельзя применить $\beta$-редукцию даже после нескольких $\alpha$-конверсий.\\
    Говорят, что $N$ -- \textit{нормальная форма} терма $M$, если $M=N$ и $N$ в нормальной форме.
\end{definition}

\begin{corollary}[из теоремы Чёрча-Россера]
    У каждого $\lambda$-терма есть не более одной нормальной формы.
    
    \begin{proof}
        Предположим, что у терма $A$ две нормальные формы: $B$ и $C$ (то есть $A \rightarrow B$ и $A \rightarrow C$). По теореме Чёрча-Россера существует такой $D$, что $B \rightarrow D$ и $C \rightarrow D$. Но по определению $B$ и $C$ -- $\lambda$-термы, к которым нельзя применить $\beta$-редукцию. Противоречие.
    \end{proof}
    
\end{corollary}

\begin{note}
    Не у всех $\lambda$-термов есть нормальная форма. Например, $\Omega = (\lambda x.xx)(\lambda x.xx)$ редуцируется сам в себя.
\end{note}