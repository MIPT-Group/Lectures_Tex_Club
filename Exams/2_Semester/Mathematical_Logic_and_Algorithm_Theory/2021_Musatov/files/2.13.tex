\section{Теория множеств}

\subsection{Существование нелинейной аддитивной функции}

Линейно независимое множество векторов называется \emph{базисом Гамеля (или просто базисом)} данного пространства, если любой вектор представим в виде конечной линейной комбинации элементов этого множества.
\par \textbf{Утверждение:} базис Гамеля в $\mathbb{R}$ над $\mathbb{Q}$ существует
\par $\blacktriangle$ Идея: рассмотрим множество всех линейно независимых над $\mathbb{Q}$ подмножеств $\mathbb{R}$. Упорядочим их по вложению. Получится множество, удовлетворяющее условиям леммы Цорна (верхняя грань цепи — объединение всех множеств из цепи)
\par Почему вехняя грань принадлежит множеству? Действительно, пусть в объединении есть нетривиальная линейная комбинация. Тогда каждый из элементов этой комбинации входит в какое-то из объединяемых множеств. Но элементов конечное число, поэтому есть наибольшее множество, которое их содержит. Уже оно будет содержать нетривиальную комбинацию, т.е. не будет линейно независимым. Это противоречит тому, что оно из семейства линейно независимых.
\par По лемме Цорна в нашем семействе множеств есть максимальный элемент, иначе, если какое-то число не выразимо, то можно его добавить и система останется линейно независимой. Иначе либо коэффициент при новом элементе нулевой и уже старая система зависима, либо коэффициент ненулевой, и тогда само число выразимо. $\Rightarrow$ Это обязательно базис $\blacksquare$
\par \textbf{Теорема}. Существует (всюду определённая) функция $f : \mathbb{R} \rightarrow \mathbb{R}$, для которой f(x + y) = f(x) + f(y) при всех x и y, но которая не есть умножение на константу.

$\blacktriangle$
Рассмотрим $\mathbb{R}$ как векторное пространство над полем $\mathbb{Q}$. В нём
есть базис Гамеля. Пусть $\alpha$ — один из векторов базиса. Рассмотрим функцию f, которая с каждым числом x (рассматриваемым как вектор в пространстве $\mathbb{R}$ над полем $\mathbb{Q}$) сопоставляет его $\alpha$-координату (коэффициент при $\alpha$ в единственном выражении x через векторы базиса). Эта функция линейна над $\mathbb{Q}$, поэтому f(x+y) = f(x)+f(y)
$\forall x, y \in \mathbb{R}$. Она отлична от нуля ($f(\alpha) = 1$) и принимает лишь рациональные значения, поэтому не может быть умножением на константу.
$\blacksquare$