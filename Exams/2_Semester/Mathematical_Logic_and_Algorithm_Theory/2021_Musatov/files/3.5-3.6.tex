\subsection{Несуществование универсальной тотально вычислимой функции.}
\par \textbf{Определение:} $U: \{0,1\}^* \times \{0,1\}^* \Rightarrow \{0,1\}^*$ называется \textit{универсальной тотально вычислимой функцией}, если \begin{enumerate}
    \item $U$ вычислима и всюду определена
    \item  Если $f$ — всюду определённая вычислимая функция одного аргумента, то $\exists p \forall x U(p,x) = f (x)$ 
\end{enumerate}
\par \textbf{Теорема:} Универсальной тотально вычислимой функции не существует
\par $\blacktriangle$ Предположим, что такая функция существует. Тогда рассмотрим функцию $d(x)=U(x, x)$ - всюду определена и вычислима. Тогда функция $d'(x)=U(x,x)+1$ также всюду определена и вычислима. Значит, по определению универсальной тотально вычислимой функции $\exists p \forall x U(p,x)=d'(x)$. Рассмотрим $U(p,p)=d'(p)=U(p,p)+1$ - противоречие $\Rightarrow$ такой функции не существует $\blacksquare$
\par \textbf{Замечание:} Для обычных универсальных вычислимых функций такого противоречие не возникает, так как равенство $U(p,p)=U(p,p)+1$ верно, если $U(p,p)$ не определена

\subsection{Неперечислимость и некоперечислимость множества всюду определённых программ.}
\par$\blacktriangle$ Пусть это множество перечислимо (обозначим его как $A$). Решим с его помощью проблему самоприменимости (см. билет 3.4). Пусть $F$ - исследуемая функция, имеющая номер $n$ в какой то главной универсальной вычислимой функции. 
Тогда $$F'(x) = \begin{cases} 
x & \text{если $F(n)$ не завершилось за $x$ шагов}\\
\bot & \text{иначе не определена}
\end{cases}$$
\par Значит $F'$ всюду определена $\Leftrightarrow$ $F(n)$ не останавливается.
Пусть $F'$ имеет номер $m$. Тогда: \begin{enumerate}
    \item Запустить и сразу остановить $F(n)$
    \item Проделать ещё $1$ шаг в работе $F(n)$. Если $F(n)$ остановилось, вывести $1$
    \item Вывести перечисляющем алгоритмом ещё один элемент множества $A$. Если он равен $m$ (то есть $F'(x) \in A$, а значит всюду определена) вывести $0$
    \item Вернуться ко второму шагу
\end{enumerate}
Так как $F$ или самоприменима, или несамоприменима (ее номер либо лежит в множестве из проблемы самоприменимости, либо нет), то или $1$ или $2$ шаг когда нибудь выведет результат, значит проблема самоприменимости решена, противоречие.
\par Коперечислимость решается аналогично, только $F'(x) = F(n)$ (получается $F'$ не всюду определена (то есть лежит в дополнении к $A$) $\Leftrightarrow$ $F(n)$ не останавливается). Нам остается только заменить в нашем алгоритме третий шаг: будем перечислять $\overline{A}$. $\blacksquare$