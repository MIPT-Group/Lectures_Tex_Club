\subsection{Теорема о трансфинитной рекурсии.}

\textbf{Теорема.} Пусть A — вполне упорядоченное множество, B — произвольное множество. Пусть имеется некоторое рекурсивное правило (отображение F, которое ставит в соответствие элементу $x \in A$ и функции $g : [0, x) \rightarrow B$ некоторый элемент B). Тогда $\exists !$ функция $f : A \rightarrow B$: $f(x) = F(x, f|_{[0,x)})$ $\forall x \in A$. (Здесь $f|_{[0,x)}$ обозначает ограничение функции f на начальный отрезок [0, x) — мы отбрасываем все значения функции на элементах, больших или равных x.)

$\blacktriangle$
Идея доказательства: значение f на минимальном элементе определено однозначно, так как предыдущих значений нет (сужение $f|_{[0,0)}$ пусто). Тогда и на следующем элементе значение функции f определено однозначно, поскольку на предыдущих (точнее, единственном предыдущем) функция f уже задана, и т. д.

Строгое док-во:

1. Утверждение о произвольном $a \in A$: существует и единственно отображение f отрезка [0, a] в множество B, для которого рекурсивное определение (равенство, приведённое в условии) выполнено при всех $x \in [0, a]$.

Пусть отображение $f : [0, a] \rightarrow B$, обладающее указанным свойством - "корректное". Таким образом, мы хотим доказать, что $\forall a \in A$ $\exists!$ корректное отображение отрезка [0, a] в B. Поскольку мы рассуждаем по индукции, можно предполагать, что для всех $c < a$ это утверждение выполнено, то есть существует и единственно корректное отображение $f_c : [0, c] \rightarrow B$. (Корректность $f_c$ означает, что при всех $d \leqslant c$ значение $f_c(d)$ совпадает с предписанным по рекурсивному правилу.)

Рассмотрим отображения $f_{c_1}$ и $f_{c_2}$ для двух различных $c_1$ < $c_2$. Отображение $f_{c_2}$ определено на большем отрезке $[0, c_2]$. Если ограничить $f_{c_2}$ на меньший отрезок $[0, c_1]$, то оно совпадёт с $f_{c_1}$, поскольку ограничение корректного отображения на меньший отрезок корректно (это очевидно), а мы предполагали единственность на отрезке $[0, c_1]$.

Таким образом, все отображения $f_c$ согласованы друг с другом (принимают одинаковое значение, если определены одновременно). Объединив их, мы получаем некоторое единое отображение h, определённое на $[0, a)$. Применив к a и h рекурсивное правило, получим некоторое значение $b \in B$. Доопределим h в точке a, положив $h(a) = b$. Получится отображение $h: [0, a] \rightarrow B$; легко понять, что оно корректно.

Чтобы завершить индуктивный переход, надо проверить, что на отрезке $[0, a]$ корректное отображение единственно. В самом деле, его ограничения на отрезки $[0, c]$ при $c < a$ должны совпадать с $f_c$, поэтому осталось проверить однозначность в точке a — что гарантируется рекурсивным определением (выражающим значение в точке a через предыдущие). На этом индуктивное доказательство заканчивается.

2. Осталось лишь заметить, что для разных a корректные отображения отрезков $[0, a]$ согласованы друг с другом (сужение корректного отображения на меньший отрезок корректно, применяем единственность) и потому вместе задают некоторую функцию $f : A \rightarrow B$, удовлетворяющую рекурсивному определению. Существование доказано; единственность тоже понятна, так как ограничение этой функции на любой отрезок [0, a] корректно и потому однозначно определено, как мы видели.
$\blacksquare$