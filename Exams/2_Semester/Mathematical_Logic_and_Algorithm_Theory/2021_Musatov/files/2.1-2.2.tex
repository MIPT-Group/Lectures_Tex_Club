\section{Теория множеств}

\subsection{Эквивалентность фундированности, отсутствия бесконечно убывающей последовательности элементов и принципа трансфинитной индукции.}
\noindent Мы будем работать с частично упорядоченным множеством $(A, \leqslant)$ и для краткости будем его просто называть множеством $A$.
\newline \par \textbf{Теорема:} Три определения фундированного множества эквивалентны друг другу:
\begin{enumerate}
    \item Множество $A$ называется фундированным, если в любом непустом подмножестве $A$ есть минимальный элемент.

    \item Множество $A$ называется фундированным, если для него выполняется принцип невозможности бесконечного спуска: не существует строго убывающей последовательности \newline $x_1>x_2>x_3>\ldots$
    
    \item Множество $A$ называется фундированным, если для него выполняется принцип трансфинитной индукции: для любого свойства $\phi(x)$ верно условие: \newline $\forall x \; (\forall y<x \;\; \phi(y)\to \phi(x) ) \;\Rightarrow\; \forall x \phi(x)$
\end{enumerate}

$\blacktriangle$ (1 $\Rightarrow$ 2) Предположим, что 2 определение неверно, и в множестве есть бесконечная убывающая цепь $x_1 > x_2>\ldots$. Но тогда в множестве $B = \{x_1, x_2, \ldots\}$ нет минимального элемента, что противоречит определению 1.

\par 
(2 $\Rightarrow$ 1) Теперь предположим, что определение 1 не выполнено. Это значит, что в $A$ есть непустое подмножество $B$, в котором нет минимального элемента. Поскольку $B\neq \O$, то $\exists x_1\in B$. Мы предположили, что в $B$ нет минимальных элементов. В частности, $x_1\neq min$, то $\exists x_2 < x_1$. Поскольку $x_2\neq min$, то $\exists x_3 < x_2$ и так далее, получим бесконечно убывающую последовательность. Это противоречит определению 2.

\par (1 $\Rightarrow$ 3) Снова предположим, что для
некоторого $A$ выполнено определение 1. Нам нужно доказать, что для данного множества выполнен также и принцип индукции. Пусть для какого-то свойства $\phi(x)$ верен “шаг индукции”: $$\forall x \; (\forall y<x \;\; \phi(y)\to \phi(x) )$$
Мы хотим показать, что в таком случае свойство $\phi(x)$ верно для всех элементов $x \in A$. Предположим противное – пусть для некоторых $x$ свойство $\phi(x)$ ложно. Выберем среди всех таких $x$ минимальный (определение фундированности гарантирует, что среди всех элементов $x$ для которого $\phi(x)$ ложно, есть хотя бы один минимальный). Тогда для данного $x_{min}$ свойство $\phi(x_{min})$ ложно, а для всех элементов $y$ меньших $x_{min}$ свойство $\phi(y)$ истинно. Получаем противоречие с предположением индукции (т.е. $1\to 0$).

\par (3 $\Rightarrow$ 1) Теперь предполагаем, что для
$A$ выполнен принцип индукции. Нам нужно проверить, что $\forall B\subset A \; | \; B\neq \O$ есть хотя бы один минимальный элемент. Пусть в некотором $B\subset A$ минимального элемента нет. Мы должны доказать, что данное $B$ пусто. Для этого мы рассмотрим свойство $\phi(x)$\;: $\phi(x)$ истинно $\Leftrightarrow \; x\notin B$. Для данного свойства верно: $$\forall x \; (\forall y<x \;\; \phi(y)\to \phi(x) )$$ 
(если все элементы $y<x$ не лежат в $B$, то и $x$ не лежит в $B$, иначе $x$ был бы минимальным элементом $B$) По принципу индукции заключаем, что свойство $\phi(x)$ истинно для всех $x\in A$. Это значит, что в $B$ нет ни одного
элемента — это подмножество пусто. $\quad \blacksquare$

\subsection{Лемма о монотонной функции из вполне упорядоченного множества в себя.}

\textbf{Лемма:} Пусть $W$ -- вполне упорядоченное множество (def: одновременно фундировано и линейно упорядочено), а $f:W\to W$ -- строго монотонная функция ($x>y \Rightarrow f(x)>f(y)$). \newline Тогда $\forall x \; f(x)\geqslant x$

$\blacktriangle$ Докажем через ринцип невозможности бесконечного спуска:
\newline Пусть для какого-то $x$ верно $f(x)<x$. Тогда по строгой монотонности выполнено: $$f(f(x))<f(x),\;\; f(f(f(x)))<f(f(x)),\;\; \ldots$$ Следовательно, образуется бесконечно убывающая последовательность $$x>f(x)>f(f(x))>f(f(f(x)))>\ldots$$ Это противоречит фундированности $W$, значит, $\forall x \; f(x)\geqslant x$. $\quad \blacksquare$