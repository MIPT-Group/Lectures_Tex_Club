\subsection{Эквивалентность следующих утверждений: множество перечислимо, полухарактеристическая функция множества вычислима, множество является областью
определения вычислимой функции, множество является проекцией разрешимого
множества пар.}

\textbf{Теорема.} Следующие утверждения для непустого $S \subseteq \mathbb{N}$ эквивалентны:

1) S перечислимо (существует печатающая машина, такая, что $\forall x \in S$ x встречается в потоке вывода, $\forall x \notin S$ x не встречается в потоке вывода);

2) Полухарактеристическая функция множества (равная 0 на элементах S и не определённая вне S) вычислима;

3) S - область определения вычислимой функции (если существует алгоритм, её вычисляющий, то
есть такой алгоритм A, что $\forall f(n)$ определённых для некоторого n алгоритм А остановится на входе n и напечатает f(n), иначе - не остановится на входе n);

4) S - проекция разрешимого (существует алгоритм, который по любому натуральному n определяет, принадлежит ли оно множеству) множества пар.

$\blacktriangle$
(1) $\Rightarrow$ (2). Запускаем эту печатающую машину. Если она выдаёт x, то значение полухарактеристической функции 1, иначе - $\perp$.

(2) $\Rightarrow$ (3). S - область определения характеристической функции, описанной ранее.

(3) $\Rightarrow$ (1). Пусть S - область определения вычислимой функции f, вычисляемой алгоритмом B. Тогда есть алгоритм, перечисляющий A: параллельно запускать B на входах 0, 1, 2, ..., делая всё больше шагов (1 шаг на входах 0 и 1, 2 шага - на входах 0, 1, 2, и.т.д.); напечатать все номера, на которых B остановился. 

(1) $\Rightarrow$ (4). S = $\{ x | \exists n (x, n) \in B\}$ - проекция множества $B = \{ (x, n):$ x в первых n шагах алгоритма, перечисляющего S$\}$

(4) $\Rightarrow$ (1). for (x=0;; ++x) \\
for (y=0;; ++y) 

    $\{ if ((x, y) \in B)$ cout $<<$ x;

    $if ((y, x) \in B)$ cout $<<$ y; $\}$
$\blacksquare$