\subsection{Сравнимость любых двух вполне упорядоченных множеств.}
\textbf{Теорема}. Если А и В - в.у.м., то верно ровно одно из трёх:

1)  $A \simeq B$
	
2) $A \simeq [0, b)$, $b \in B$
	
3) $B \simeq [0, a)$, $a \in A$

$\blacktriangle$

1. Покажем, что 2 и 3 не могут быть выполнены одновременно. $A \simeq [0, b)$, $B \simeq [0, a)$  $\Rightarrow$ начальный отрезок B изоморфен начальному отрезку начального отрезка A, а начальный отрезок начального отрезка так же является начальным отрезком. Получили что \emph{A изоморфно своему начальному отрезку}, что невозможно по следствию, противоречие. Аналогичными рассуждениями можно понять, что 1 и 2, 1 и 3 тоже не могут быть выполнены одновременно. 
Таким образом, понимаем, что не больше одного из этих пунктов может быть выполнено. 

2. Покажем, что хотя бы один из этих пунктов будет выполнен (будем использовать трансфинитную рекурсию): постепенно построим функцию с аргументами в A и значениями в B. Строим функцию $g: A  \rightarrow B \cup\{\perp\}$, где $\perp$ - специальный символ неопределённости (любую частично определённую функцию можно переделать во всюду определённую, если добавить специальный символ неопределённости)

Строим функцию рекурсивно:
$g(a) = \{min\{y \in B: y \neq g(x)$ для $x < a \}\}$ (1), если это множество не пусто, иначе - $\perp$.

Корректность определения: \emph{функция g  существует и единственна}.
Скажем, что $g|_{[0, a)}: [0,a) \rightarrow B\cup\{\perp\}$ корректна, если она удовлетворяет соотношению (1). Докажем по трансфинитной индукции, что $g|_{[0, a)}$ существует и единственна. Пусть $\forall x < a$ $g|_{[0, a)}$ существует и единственна. Тогда при $x < a$  $g|_{[0, a)}(x)$ определено однозначно. 

Пусть a < c . Тогда $g|_{[0, a)}$ и $g|_{[0, c)}$ совпадают на $[0, a)$ (ввиду однозначности). Можно рассмотреть $g: A  \rightarrow B \cup\{\perp\}$, которая продолжает все $g|_{[0, a)}$. Если в множестве А есть максимальный элемент, то он не попадёт ни в один из полуинтервалов, но он ровно один, и для него всё доопределится по (1). Если же максимального элемента нет, то нужно всё объединить.

I. $\exists a: g(a) = \perp$ $\Rightarrow$ при всех $c > a$ $g(c) = \perp$

Если $g(c) = \perp$, то пусть $a = min\{x| g(x) = \perp\}$. Тогда $B \simeq [0, a)$. Доказывается, что при $x < a$ начальный отрезок $[0, x) \simeq [0, g(x))$, g - изоморфизм. Пусть при  $y < x [0, y) \simeq [0, g(y))$.

	инъекция: $y_1 < y_2 < x \Rightarrow g(y_2) = min\{z \in B: z \neq g(x)$ для $x<y_2\}$ $\Rightarrow$ $g(y_1) \neq g(y_2)$
	
	
	сюръекция: $z < g(x) \Rightarrow z = g(v)$ при $v < x$
	

Сохранение порядка: $y_1 < y_2 < x \Rightarrow g(y_1) < g(y_2)$ . По написанному выше $g(y1) \neq g(y2)$. Но $g(y_2)$ не может быть меньше, чем $g(y_1)$, иначе бы получилось, что до $g(y_1)$ есть какие-то пустые места, и $g(y_1)$ бы определилось не так, как оно определилось, а занято было бы то пустое место.

II. $\nexists a: g(a) = \perp$: 

- все значения в B принимаются. Тогда $A \simeq B$

- не все значения в B принимаются. Тогда $b = min\{y | y \neq g(x), x \in A\}$, и $A \simeq [0, b)$
$\blacksquare$